\documentclass{standalone}
\usepackage{tikz}
\begin{document}

\begin{tikzpicture}
  % Define the grid cells
  % Let's say 3 rows, 2 columns, so coordinates (0,0) to (2,1) for 3 rows (y from 0 to 2, but maybe adjust)
  % Each cell is a rectangle, width 2cm, height 1cm, spaced by 0.1cm
  \foreach \row in {0,1,2} {
    \foreach \col in {0,1} {
      \node[draw, minimum size=2cm, rectangle] at (\col*2.2, \row*1.2) {};
    }
  }
  % Label the tiles
  % Let's say tile 1 is at (0,0), 2 at (1,0), 3 at (2,0), 4 at (0,1), 5 at (1,1), 6 at (2,1), empty at... wait, need to decide which is empty. Let's say the empty is at (1,1) for example.
  % Wait, 3 rows, 2 columns: cells are (0,0), (1,0), (2,0), (0,1), (1,1), (2,1), (0,2), (1,2), (2,2) – no, 3 rows would be y=0,1,2. So 3 rows, 2 columns: 6 cells. Let's say the empty is at (1,1) (middle cell). So:
  \node at (1*2.2, 1*1.2) {\textbf{}}; % empty
  \node at (0*2.2, 0*1.2) {1};
  \node at (1*2.2, 0*1.2) {2};
  \node at (2*2.2, 0*1.2) {3};
  \node at (0*2.2, 1*1.2) {4};
  \node at (2*2.2, 1*1.2) {5}; % wait, 6 tiles: 1,2,3,4,5,6. So maybe 6 tiles, empty at (1,1). So:
  \node at (0*2.2, 0*1.2) {1};
  \node at (1*2.2, 0*1.2) {2};
  \node at (2*2.2, 0*1.2) {3};
  \node at (0*2.2, 1*1.2) {4};
  \node at (1*2.2, 1*1.2) {5}; % wait, need 6 tiles. Oh, maybe 3 rows, 2 columns, 6 tiles, so 3 rows, 2 columns, 6 cells, one empty. So tiles 1-6, empty one. Let's say:
  \node at (0*2.2, 0*1.2) {1};
  \node at (1*2.2, 0*1.2) {2};
  \node at (2*2.2, 0*1.2) {3};
  \node at (0*2.2, 1*1.2) {4};
  \node at (1*2.2, 1*1.2) {5};
  \node at (2*2.2, 1*1.2) {6};
  % Wait, that's 6 tiles, no empty. So maybe the empty is at (1,1), so remove one tile. So:
  \node at (0*2.2, 0*1.2) {1};
  \node at (1*2.2, 0*1.2) {2};
  \node at (2*2.2, 0*1.2) {3};
  \node at (0*2.2, 1*1.2) {4};
  \node at (2*2.2, 1*1.2) {5}; % 5 tiles, need 6. Hmm, maybe 3 rows, 2 columns, with one empty in the second row, first column? No, need to have 6 tiles. Wait, maybe the grid is 3x2, so 6 cells, one empty. So 5 tiles, one empty. So let's say tiles 1,2,3,4,5,6 with one empty. Let's say the empty is at (1,1), so the tiles are at (0,0), (1,0), (2,0), (0,1), (2,1), (1,2)? No, 3 rows. Wait, maybe the grid is 2 rows, 3 columns, 6 cells, one empty. That makes more sense. 2 rows, 3 columns: 6 cells. So rows 0 and 1, columns 0,1,2. So cells (0,0), (0,1), (0,2), (1,0), (1,1), (1,2), 6 cells. One empty. Let's say empty is at (1,1). So tiles:

(0,0): 1, (0,1): 2, (0,2): 3, (1,0): 4, (1,2): 5, (1,1): empty. Wait, that's 5 tiles. Need 6. So maybe (1,1) is 6? No, need to have 6 tiles. Wait, maybe the grid is 3 rows, 2 columns, with 6 tiles, so 3 rows, 2 columns, 6 cells, one empty. So 5 tiles, one empty. Let's say the tiles are 1,2,3,4,5,6 with empty at (1,1). So:

(0,0): 1, (0,1): 2, (1,0): 3, (1,1): empty, (2,0): 4, (2,1): 5, 6? Wait, no. Maybe I'm overcomplicating. Let's just define a 3x2 grid (3 rows, 2 columns) with 6 cells, label them 1-6, and one cell is empty (not labeled). So in TikZ, each cell is a node with a rectangle, labeled with the tile number, except the empty one.

So the TikZ code would be:

\begin{tikzpicture}
  % Define the grid cells
  % 3 rows, 2 columns, each cell is a rectangle, spaced by 0.5cm
  \foreach \row in {0,1,2} {
    \foreach \col in {0,1} {
      \node[draw, minimum size=2cm, rectangle] (cell-\row-\col) at (\col*2.5, \row*1.5) {};
    }
  }
  % Label the tiles
  % Let's say the empty is at (1,1) (row 1, column 1)
  \node at (0*2.5, 0*1.5) {1};
  \node at (1*2.5, 0*1.5) {2};
  \node at (2*2.5, 0*1.5) {3};
  \node at (0*2.5, 1*1.5) {4};
  \node at (2*2.5, 1*1.5) {5}; % 5 tiles, need 6. So maybe (1,1) is 6? No, need to have 6 tiles. Wait, maybe the grid is 2 rows, 3 columns, 6 cells. Let's try that.

  \begin{tikzpicture}
    % 2 rows, 3 columns
    \foreach \row in {0,1} {
      \foreach \col in {0,1,2} {
        \node[draw, minimum size=2cm, rectangle] (cell-\row-\col) at (\col*2.5, \row*1.5) {};
      }
    }
    % Label the tiles, empty at (1,1)
    \node at (0*2.5, 0*1.5) {1};
    \node at (1*2.5, 0*1.5) {2};
    \node at (2*2.5, 0*1.5) {3};
    \node at (0*2.5, 1*1.5) {4};
    \node at (2*2.5, 1*1.5) {5};
    \node at (1*2.5, 1*1.5) {6}; % Wait, that's 6 tiles, no empty. So maybe empty at (1,0) or (1,2). Let's say empty at (1,1) in 2x3 grid. So:
    \node at (0*2.5, 0*1.5) {1};
    \node at (1*2.5, 0*1.5) {2};
    \node at (2*2.5, 0*1.5) {3};
    \node at (0*2.5, 1*1.5) {4};
    \node at (2*2.5, 1*1.5) {5};
    % Empty at (1,1), so no label
  \end{tikzpicture}

Wait, but the user said "the border square is not included", so maybe the grid is surrounded by a border, and the border square is not part of the puzzle. So the puzzle is a grid with a border, and the border square is excluded. So maybe the grid is a 3x3 grid (9 cells), with the border square (maybe the center?) excluded, leaving 8 cells, but the user says tiles 1 through 6, so maybe 6 tiles, 2 empty? No, the description says "the border square is not included", so maybe the grid is a 3x3 with one border square (like the center) excluded, so 8 cells, but the puzzle has 6 tiles, 2 empty? Hmm, this is getting confusing. Maybe the original image is a 3x2 grid (6 cells) with one empty, so the TikZ code should draw a 3x2 grid with 6 rectangles, labeled 1-6, one rectangle is empty (no label).

Alternatively, maybe the puzzle is a 3x3 grid (9 cells) with the border square (the one in the center) excluded, so 8 cells, but the user says tiles 1 through 6, so maybe 6 tiles, 2 empty. But the description says "the border square is not included", so maybe the border is the outer edge, so the grid is a 3x3 with the outer edge cells excluded, leaving a 3x3 grid minus the border, which would be a 3x3 grid with the outer cells (like the first and last rows and columns) excluded, leaving a 3x3 grid minus 8 cells (4 corners, 4 edges), leaving 1 cell? No, that doesn't make sense. Maybe the border square is the one in the center, so excluding it leaves 8 cells, but the puzzle has 6 tiles, so 2 empty. But the user says "tiles 1 through 6", so 6 tiles, 2 empty. But the original image is a black and white photo of a page with tiles 1-6, so maybe 6 tiles, one empty. So back to 3x2 grid, 6 cells, one empty.

Given that, let's proceed with a 3x2 grid (3 rows, 2 columns) with 6 cells, one empty. The TikZ code would be:

\documentclass{standalone}
\usepackage{tikz}
\begin{document}

\begin{tikzpicture}
  % Define the grid cells (3 rows, 2 columns)
  \foreach \row in {0,1,2} {
    \foreach \col in {0,1} {
      \node[draw, minimum size=2cm, rectangle] (cell-\row-\col) at (\col*2.5, \row*1.5) {};
    }
  }
  % Label the tiles (1-6), empty at (1,1) (row 1, column 1)
  \node at (0*2.5, 0*1.5) {1};
  \node at (1*2.5, 0*1.5) {2};
  \node at (2*2.5, 0*1.5) {3};
  \node at (0*2.5, 1*1.5) {4};
  \node at (2*2.5, 1*1.5) {5};
  % Empty cell (1,1) has no label
\end{tikzpicture}

\end{document}