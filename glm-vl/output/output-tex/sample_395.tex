<think>Got it, let's try to understand the problem step by step. First, the question is about local complementation of a graph G with respect to a vertex u, but wait, the description mentions u, v1, v3 connected in G u (maybe G_u?) since the edge v1v3 is in G_u, while v1v2 and v2v3 are also edges. Wait, maybe I need to recall what local complementation means. Local complementation around a vertex usually means complementing the edges incident to that vertex. But the problem mentions "local complementation of a graph G according to u", so maybe the vertex is u, and we're looking at the neighborhood around u? Wait, the problem says "u, v1 and v3 are connected in G u since the edge v1v3 E, while v1v2 E and v2v3 E." Hmm, maybe there's a typo, like "G_u" is the graph after local complementation around u, and we need to check the connections between v1, v2, v3. Wait, let's parse the sentence again: "Local complementation of a graph G according to u. v1 and v3 are connected in G u since the edge v1v3 E, while v1v2 E and v2v3 E." Wait, maybe "G u" is the graph after local complementation around vertex u, and in that graph, v1 and v3 are connected (so edge v1v3 exists), while v1v2 and v2v3 are also edges? Wait, but the original graph G might have different edges. Wait, the problem states that in G u (maybe G_u), the edge v1v3 is present, and v1v2 and v2v3 are also present. Wait, but local complementation around a vertex u would affect the edges incident to u. If u is one of the vertices, say u is connected to v1, v2, v3, then local complementation around u would complement the edges among v1, v2, v3 if they were connected, but maybe the problem is describing the result of local complementation. Alternatively, maybe the graph G has vertices u, v1, v2, v3, and local complementation around u changes the edges among v1, v2, v3. If originally, in G, the edges were, say, v1v2 and v2v3, but after local complementation around u, the edge v1v3 is added, and v1v2 and v2v3 remain? Wait, the problem says "v1 and v3 are connected in G u since the edge v1v3 E, while v1v2 E and v2v3 E." So in the graph G_u (after local complementation), the edges v1v3, v1v2, v2v3 exist. Wait, but local complementation around a vertex usually means that for the vertex u, we complement the edges incident to u. So if u is connected to v1, v2, v3, then originally, if there was an edge uv1, after complementation, it's not there, and vice versa. But the problem is talking about edges between v1, v2, v3, not involving u. Maybe the local complementation is around the edge uv1 or something? Wait, I need to clarify the definition. Local complementation around a vertex v is the graph obtained by complementing the subgraph induced by the neighbors of v. So if v has neighbors, say, w1, w2, w3, then the edges among w1, w2, w3 are complemented. So if in the original graph G, the neighbors of u are v1, v2, v3, then the subgraph induced by v1, v2, v3 would have some edges, and after local complementation around u, those edges are complemented. Suppose originally, in the subgraph of v1, v2, v3, the edges were v1v2 and v2v3, then after complementation, the edges not present before would be v1v3, so the edges would be v1v2, v2v3, v1v3. Which matches the problem statement: "v1 and v3 are connected in G u since the edge v1v3 E, while v1v2 E and v2v3 E." So that makes sense. Therefore, the local complementation around vertex u (whose neighbors are v1, v2, v3) complemented the edges among v1, v2, v3, resulting in v1v3 being added (if it wasn't there before), and v1v2 and v2v3 remaining (if they were there before). So the key idea is that local complementation around a vertex changes the edges among its neighbors, and in this case, the result is that v1 and v3 are connected (edge v1v3 exists), and v1v2 and v2v3 also exist. Therefore, the conclusion is that after local complementation around u, the edges v1v3, v1v2, v2v3 are present in the graph G_u.</think><answer>To understand the local complementation of graph \( G \) around vertex \( u \), we first recall that local complementation around a vertex involves complementing the edges within the subgraph induced by its neighbors. Suppose \( u \) is connected to vertices \( v_1, v_2, v_3 \). The subgraph induced by \( v_1, v_2, v_3 \) in \( G \) has edges \( v_1v_2 \) and \( v_2v_3 \). After local complementation around \( u \), the edges among \( v_1, v_2, v_3 \) are complemented, resulting in the addition of edge \( v_1v_3 \) (since it was not present before) while retaining \( v_1v_2 \) and \( v_2v_3 \). Thus, in the graph \( G_u \) (after local complementation), \( v_1 \) and \( v_3 \) are connected by edge \( v_1v_3 \), along with \( v_1v_2 \) and \( v_2v_3 \).

The answer is \(<|begin_of_box|>v_1v_3<|end_of_box|>\).</answer>