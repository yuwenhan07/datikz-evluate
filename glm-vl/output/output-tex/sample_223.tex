<think>Got it, let's try to figure out this problem. The question is about adding a "very tame" sequence to existing "tame" ends, and there's a graph with labels. First, I need to recall what "tame" and "very tame" mean in the context of sequences or graphs, maybe in topology or dynamical systems? But since it's a graph with labels, maybe it's about the structure of ends of graphs or spaces.

First, let's parse the image description: a black and white graph with three labels: "tame", "very tame", and an unlabeled one (which is a continuation of the x-axis). So the graph probably shows different types of ends, with "tame" and "very tame" as categories, and the unlabeled one is maybe a different type, or the x-axis is a parameter.

The task is to add a "very tame" sequence to existing "tame" ends. So maybe the existing ends are "tame", and we're adding a "very tame" sequence, which is a more restrictive type. In the context of graph ends, a "tame" end is one that can be approximated by finite graphs in a certain way, and "very tame" is a stronger condition. So adding a "very tame" sequence would mean extending the existing tame ends with a sequence that satisfies the very tame condition.

But let's think step by step. First, understand the labels: "tame" and "very tame" are likely different classes of ends. The graph might show the relationship between them, with "very tame" being a subset of "tame" or vice versa. If the unlabeled label is a continuation of the x-axis, maybe the x-axis represents some parameter, and the graph shows how adding a very tame sequence affects the existing tame ends.

Alternatively, in the context of adding sequences, maybe the existing ends are "tame", and we're adding a "very tame" sequence, which would result in a space with ends that include both the original tame ends and the new very tame sequence. So the process would involve ensuring that the new sequence is compatible with the existing structure, maintaining the tame or very tame properties.

Wait, the problem says "adding a very tame sequence to existing tame ends". So the existing structure has tame ends, and we're adding a sequence (maybe a sequence of graphs or spaces) that is very tame, and we need to show that the resulting structure still has tame ends, or that the addition is possible while preserving the properties.

In graph theory, an end of a graph is a way of looking at the graph as tending to infinity. A tame end is one that can be "tamed" by a sequence of finite graphs, and a very tame end is a stronger condition. So if you have existing tame ends, adding a very tame sequence (which is a sequence of very tame ends) would mean that the new structure has ends that include both the original tame ends and the new very tame ones, but since very tame is a subset of tame, maybe the new ends are still tame, or the addition is possible without violating the tame condition.

Alternatively, maybe the graph shows the hierarchy of end types, with "very tame" being above "tame" (i.e., every very tame end is tame, but not vice versa), and the unlabeled label is the x-axis, which could represent the number of ends or some parameter. Adding a very tame sequence would mean increasing the number of very tame ends, while the existing ones are tame, so the overall structure still has tame ends (since very tame is a type of tame).

But since the image is a black and white graph with three labels, maybe it's a diagram showing the relationship between the types. If "tame" and "very tame" are on the graph, and the unlabeled is the x-axis, then the process of adding a very tame sequence would involve placing the new sequence in the "very tame" category, connected appropriately to the existing tame ends.

In summary, the key points are recognizing the hierarchy of end types (very tame ⊆ tame), adding a sequence in the more restrictive category (very tame) to the existing ones (tame), which would result in a structure with ends that include both, but still maintaining the tame property (since very tame is a subset). Therefore, the addition is possible, and the resulting structure has tame ends with the added very tame sequence.</think><answer>To address the problem of adding a "very tame" sequence to existing "tame" ends, we first recognize the hierarchical relationship between these end types in graph theory or dynamical systems: **very tame** ends are a subset of **tame** ends (i.e., every very tame end is tame, but not all tame ends are very tame).  

The image depicts a graph with labels indicating these end types, where the unlabeled continuation of the x-axis likely represents a parameter (e.g., the number of ends or a structural dimension). Adding a "very tame" sequence to existing "tame" ends involves incorporating a sequence of very tame ends into the existing structure. Since very tame ends are a subset of tame ends, the resulting structure will still have tame ends (as the new sequence does not introduce non-tame ends).  

Thus, the process of adding a very tame sequence to existing tame ends is valid and preserves the tame property of the overall structure.  

The answer is <|begin_of_box|>tame<|end_of_box|></answer>