<think>Got it, let's try to figure out what's going on here. The problem mentions two graphs, left and right, each with points (0,0) and (0,1). Wait, both lines have the same coordinates? So maybe the left graph is a vertical line from (0,0) to (0,1), and the right graph is also a vertical line? But the description says "two white lines on a white background, both having coordinates of (0,0) and (0,1)". Hmm, maybe the left is a vertical line along the y-axis, and the right is another vertical line, or maybe a horizontal line? Wait, (0,0) to (0,1) is a vertical line segment along the y-axis from the origin up to (0,1). If both left and right graphs have that, maybe the left is the y-axis itself, and the right is another vertical line, but the problem says "the image consists of two white lines on a white background, both having coordinates of (0,0) and (0,1)". Wait, maybe each graph has a line from (0,0) to (0,1), so both are vertical lines along the y-axis, but maybe the left is the y-axis, and the right is a different line? Wait, no, the coordinates are the same. Wait, maybe the left graph is a vertical line (x=0) from (0,0) to (0,1), and the right graph is also a vertical line, but maybe shifted? But the problem says both have those coordinates. Alternatively, maybe the left is a horizontal line? Wait, (0,0) to (0,1) is vertical, not horizontal. Wait, horizontal would be (0,0) to (1,0), but here it's (0,0) to (0,1), so vertical. So both graphs have a vertical line segment along the y-axis. Maybe the left graph is the y-axis, and the right graph is another vertical line, but the problem says "the image consists of two white lines on a white background, both having coordinates of (0,0) and (0,1)". Wait, maybe the left is a line from (0,0) to (0,1), and the right is a line from (0,0) to (0,1) as well, but maybe the right one is a different orientation? No, the coordinates are the same. Wait, perhaps the problem is referring to a graph where adding a new vertex, and the left and right graphs show the process. Maybe the left graph is before adding the vertex, and the right is after, with both having the coordinate lines. But without seeing the actual image, I need to infer based on the description. The key points are two lines with endpoints (0,0) and (0,1), which is a vertical line segment. So maybe the left graph is a simple graph with a single edge from (0,0) to (0,1), and the right graph is the same, but with a new vertex attached? Wait, the question is about "attach_new_vertex", so maybe the left graph is a graph with some structure, and the right graph is the same graph with a new vertex connected appropriately. But given the coordinates, both lines have (0,0) and (0,1), so maybe the left is a graph with a single edge, and the right is a graph with that edge plus a new vertex connected, but the lines shown are the same? Hmm, this is a bit confusing without the image, but the description says both lines have those coordinates, so maybe the left is a vertical line (the y-axis), and the right is a graph with a vertical line and another line, but the key is that both have the coordinate points (0,0) and (0,1). Alternatively, maybe the left graph is a graph with vertices at (0,0) and (0,1) connected by an edge, and the right graph is the same edge plus a new vertex connected to one of them, but the lines shown are the edges. Given that both lines have those coordinates, perhaps the left is a single edge, and the right is two edges connected at a new vertex. But I need to make a logical conclusion based on the given info. The problem mentions "attach_new_vertex", so attaching a new vertex would involve connecting it to existing vertices. If the existing vertices are (0,0) and (0,1), maybe the new vertex is connected to both, forming new edges. But the graphs shown are two lines, so maybe the left graph has one edge, and the right graph has two edges connected at a new vertex, but the lines shown are the edges. Alternatively, since both lines have the same endpoints, maybe the left is a graph with a single edge, and the right is a graph with two edges forming a path, but the lines shown are the edges. Given the limited info, but the key is that both lines have coordinates (0,0) and (0,1), so they are the same edge, but maybe the left is the original graph, and the right is the graph after adding a vertex connected to that edge, making it a more complex graph with additional edges. But without seeing the image, I'll have to assume that the left graph is a simple graph with a single edge from (0,0) to (0,1), and the right graph is the same edge plus a new vertex connected to (0,0) or (0,1), forming a new edge, so the right graph has two edges. But the problem says "the image consists of two white lines on a white background, both having coordinates of (0,0) and (0,1)". Wait, maybe both graphs have two lines, each with those coordinates, so maybe the left has one line, the right has two lines, both with those endpoints. So the left graph has one edge, the right has two edges, forming a structure where a new vertex is attached, creating a new edge. Therefore, the conclusion is that the left graph is a simple edge between (0,0) and (0,1), and the right graph is that edge plus a new vertex connected to one of the endpoints, resulting in two edges. But since the question is about the graphs in Example ex:attach_new_vertex, left and right, with the described lines, the key takeaway is that attaching a new vertex would involve adding a vertex connected to existing vertices, and the graphs show the before and after with the coordinate lines representing the edges.</think><answer>The problem describes two graphs, left and right, each containing a line segment with endpoints \((0,0)\) and \((0,1)\), representing a vertical edge along the y-axis. The left graph likely shows the original graph with a single edge between these points, while the right graph demonstrates the result of attaching a new vertex to this edge, creating additional connections (e.g., a new vertex connected to one of the endpoints, forming a new edge). Both graphs use the same coordinate points for the original edge, illustrating the process of vertex attachment in graph theory.  

The answer is \(<|begin_of_box|>(0,0) \text{ and } (0,1)<|end_of_box|>\).</answer>