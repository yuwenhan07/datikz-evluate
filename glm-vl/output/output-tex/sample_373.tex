<think>Got it, let's try to figure out this problem. The user is talking about applying Lemma lemma-fiberstructure1, which probably requires a specific parallelogram, specifically the unit square. The image is a black and white square, maybe a hole or an object. So, first, I need to recall what Lemma lemma-fiberstructure1 might state. Since it's about fiber structure, maybe it's related to fiber bundles or some geometric structure where the parallelogram (like a fundamental domain) needs to be a unit square for the lemma to apply. 

The key point is that the parallelogram must be the unit square. The image is a square, so maybe the square in the image is supposed to be the unit square, but depending on context, it could be a hole or an object. So, the condition is that the parallelogram (which is the square in the image) must be the unit square. Therefore, if the image's square is the unit square, then the lemma can be applied directly. If not, maybe we need to adjust or check the context. 

So, summarizing: To apply the lemma, the parallelogram (the square in the image) must be the unit square. The image is a square, which could be a hole or object, but for the lemma, it needs to be the unit square. Therefore, the condition is that the square is the unit square, and the image's square must match that.</think><answer>To directly apply Lemma lemma-fiberstructure1, the parallelogram (depicted as a square in the image) must be the unit square. The image shows a black-and-white square, which could represent a square hole or object; however, for the lemma’s application, this square must specifically be the unit square (a square with side length 1, serving as a fundamental domain in the context of the lemma). If the square in the image is not the unit square, adjustments or additional context would be needed to satisfy the lemma’s requirements.  

The answer is <|begin_of_box|>1<|end_of_box|>.</answer>