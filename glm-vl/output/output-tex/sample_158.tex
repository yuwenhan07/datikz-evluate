<think>Got it, I need to figure out the basis functions for the space S³₀ with the given knot vector {0, 0, h, 2h, 3h, 4h, 5h, 6h, 7h, 1, 1}. Wait, first, let's recall what S³₀ means. S stands for spline, and the subscript 0 usually indicates a piecewise polynomial space of degree 3 (since the degree is the number minus 1, so 3) with a certain continuity condition. The knot vector provided has 11 knots, right? Let's count: 0, 0, h, 2h, 3h, 4h, 5h, 6h, 7h, 1, 1. That's 11 knots. For a cubic spline (degree 3), the number of basis functions should be related to the number of intervals and the continuity. 

First, let's recall that for a spline space S^k_m with knot vector of length n, the number of basis functions is usually n - k - 1 if it's a periodic spline, but here it's probably a non-periodic one. Wait, the knot vector here has two 0s, then 9 intervals (from 0 to h, h to 2h, ..., 7h to 1), then two 1s. So the number of intervals is 8 (from 0 to 1, divided into 8 intervals of h each), and the knot vector length is 11. For a cubic spline (degree 3), the number of basis functions should be the number of intervals plus the degree plus 1? Wait, maybe another approach: the basis functions for a B-spline of degree d with knot vector of length m are m - d - 1. But here, S³₀ is probably a B-spline space of degree 3 with multiplicity 2 at the endpoints (the two 0s and two 1s), which are the knot multiplicity. 

Wait, the knot vector is {0, 0, h, 2h, 3h, 4h, 5h, 6h, 7h, 1, 1}, which has multiplicity 2 at the start and end, and multiplicity 1 for the interior knots. For a cubic B-spline (degree 3), the number of basis functions is (number of knots - degree - 1) = 11 - 3 - 1 = 7? Wait, 11 knots minus 3 (degree) minus 1 is 7. Wait, let's check the formula again. For a non-periodic B-spline of degree d with knot vector of length n, the number of basis functions is n - d - 1. Here n = 11, d = 3, so 11 - 3 - 1 = 7. So there should be 7 basis functions. 

Now, the basis functions are usually denoted as N_i^d, where i ranges from 0 to 6 (since 7 basis functions), and d is the degree. The knot vector is given, so the support of each basis function N_i^3 would span from the i-th knot to the (i + 4)-th knot, because for degree 3, the support is 4 intervals (since degree + 1 = 4). Let's check the first basis function N_0^3: its support is from knot 0 (with multiplicity 2) to knot 3h (since 0, 0, h, 2h, 3h are the first four knots, but wait, the knots are 0,0,h,2h,3h,4h,5h,6h,7h,1,1. So the first basis function N_0^3 would have support from the first knot (0) up to the 3rd knot after the start, which is 3h? Wait, the support of a B-spline of degree d is d + 1 knots. So for N_0^3, the support is [0, 0, h, 2h, 3h], but wait, the knots are 0,0,h,2h,3h,4h,... So the support is from the first knot (0) to the (3 + 1) = 4th knot? Wait, maybe I need to recall the B-spline basis definition. The i-th B-spline of degree d is non-zero on the interval [u_i, u_{i + d + 1}], where u_0, ..., u_n are the knots. So for N_0^3, the interval is [u_0, u_4] = [0, 3h], since u_0=0, u_1=0, u_2=h, u_3=2h, u_4=3h. Wait, but the multiplicity at u_0 is 2, so the first basis function starts at u_0, which is a double knot, so its support starts at u_0 and goes to u_0 + 3 = u_3? Wait, maybe I should list the basis functions by their indices. 

The knot vector has 11 knots, so the number of basis functions is 11 - 3 - 1 = 7, as before. The basis functions are N_0^3, N_1^3, N_2^3, N_3^3, N_4^3, N_5^3, N_6^3. Each of these has support from their starting knot to their ending knot. For N_0^3, the support is from u_0 to u_3 (since degree 3, so 3 + 1 = 4 knots), which are u_0=0, u_1=0, u_2=h, u_3=2h? Wait, wait, let's count the knots again. The knot vector is: 0 (knot 0), 0 (knot 1), h (knot 2), 2h (knot 3), 3h (knot 4), 4h (knot 5), 5h (knot 6), 6h (knot 7), 7h (knot 8), 1 (knot 9), 1 (knot 10). So for a B-spline of degree 3, the i-th basis function N_i^3 is non-zero on the interval [u_i, u_{i + 4}], where u_0 to u_10 are the knots. So N_0^3 is non-zero on [u_0, u_4] = [0, 3h], N_1^3 on [u_1, u_5] = [0, 4h], N_2^3 on [u_2, u_6] = [h, 5h], N_3^3 on [u_3, u_7] = [2h, 6h], N_4^3 on [u_4, u_8] = [3h, 7h], N_5^3 on [u_5, u_9] = [4h, 1], N_6^3 on [u_6, u_10] = [5h, 1]. Wait, let's check the last one: N_6^3 would be non-zero on [u_6, u_10] = [5h, 1], since u_6 is 5h, u_7 is 6h, u_8 is 7h, u_9 is 1, u_10 is 1. So the support is from 5h to 1, which is 5h to 1, and since the last knot is 1 with multiplicity 2, the support ends at 1. 

Now, the basis functions for S³₀, which is likely the same as the cubic B-spline space with the given knot vector, are these 7 B-splines. Each basis function is defined as a polynomial piece on each interval where it's non-zero, with continuity conditions at the knots. For example, N_0^3 is non-zero on [0, 3h], which includes the intervals [0, h), [h, 2h), [2h, 3h), and is zero elsewhere. Similarly, N_1^3 is non-zero on [0, 4h], covering [0, h), [h, 2h), [2h, 3h), [3h, 4h), and so on. 

To write them out explicitly, each B-spline of degree 3 can be expressed using the Cox-de Boor recursion formula. For example, N_0^3(t) = (t - u_0)/(u_1 - u_0) N_0^2(t) + (u_2 - t)/(u_2 - u_0) N_1^2(t), but since u_0 = u_1 = 0, the first term becomes (t - 0)/(0 - 0) which is undefined, wait, maybe I need to start from the basis functions with simple knots. Wait, when there's a double knot at u_0, the first basis function N_0^3(t) is defined as (t/(h)) N_0^2(t) when t ∈ [0, h), and ((3h - t)/(2h)) N_1^2(t) when t ∈ [h, 3h), but this might get complicated. Alternatively, since the knot vector has two endpoints with multiplicity 2, the first and last basis functions will have support starting and ending at the first and last intervals, respectively. 

In summary, the basis functions for S³₀ with the given knot vector are the 7 cubic B-splines defined by the knot vector, each with support over 4 intervals (since degree 3), starting from the first knot (with multiplicity) and ending at the last knot (with multiplicity), with the number of basis functions being 11 - 3 - 1 = 7.</think><answer>The basis functions for the space \( S^3_0 \) with the given knot vector \( \{0, 0, h, 2h, 3h, 4h, 5h, 6h, 7h, 1, 1\} \) are cubic B-splines. For a cubic spline (degree 3) with a knot vector of length 11, the number of basis functions is calculated as \( 11 - 3 - 1 = 7 \). Each basis function \( N_i^3 \) (for \( i = 0, 1, \ldots, 6 \)) has support over 4 intervals (due to the cubic degree) and is defined by the Cox-de Boor recursion formula. The first basis function \( N_0^3 \) spans from the first knot (with multiplicity) to \( 3h \), the last basis function \( N_6^3 \) spans from \( 5h \) to the final knot (with multiplicity), and the intermediate basis functions span successively longer intervals.  

The answer is \(<|begin_of_box|>7<|end_of_box|>\).</answer>