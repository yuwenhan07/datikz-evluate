<think>Got it, let's try to figure out how to approach this problem. The question is about an ear decomposition of a graph, which is a way of building a graph by starting with a cycle (the first ear, P1) and then adding paths (ears) that connect existing vertices, with each subsequent ear not intersecting the previous ones except at their endpoints. 

First, let's recall what an ear decomposition is. An ear decomposition of a graph G is a sequence of subgraphs E1, E2, ..., En where each Ei is either a cycle (for i=1) or a path (called an ear) such that:
1. E1 is a cycle (the initial ear, P1 here).
2. For i ≥ 2, Ei is a path with one endpoint on Ei-1 and the other endpoint on some vertex in the union of previous subgraphs (E1 ∪ E2 ∪ ... ∪ Ei-1), and Ei does not intersect Ei-1 except at its endpoints.

Given the problem, we have ED = {P1, P2, P3, P4}, where:
- P1 is a cycle with 10 edges. A cycle with 10 edges has 10 vertices (since each edge connects two vertices, a cycle with n edges has n vertices).
- P2 is a path with 4 edges. A path with 4 edges has 5 vertices (since a path with k edges has k+1 vertices).
- P3 is a path with 3 edges, so 4 vertices.
- P4 is a path with 2 edges, so 3 vertices.

We need to describe how these ears are added one after another. Let's start with P1, which is a cycle. Let's assume the cycle has vertices v1, v2, ..., v10 (since 10 edges mean 10 vertices connected in a cycle: v1-v2, v2-v3, ..., v10-v1).

Next, P2 is a path with 4 edges, so it has 5 vertices. Let's say one endpoint of P2 is attached to a vertex on P1. Suppose one endpoint is attached to vertex v1 (or any vertex on P1), and the other endpoint is attached to another vertex on P1, say v6 (this is just an example; the exact attachment points depend on the graph, but the key is that the path doesn't intersect the cycle except at the endpoints). So P2 would connect, say, v1 to v6, passing through vertices v1, v2, v3, v4, v5, v6? Wait, no, a path with 4 edges has 5 vertices, so the path would have two endpoints on P1, say vertices u and w on P1, and the path goes from u to w through 3 intermediate vertices, making 5 vertices total (u, a, b, c, w), where u and w are on P1, and the path doesn't cross P1 except at u and w.

Then P3 is a path with 3 edges, so 4 vertices. It should connect two vertices on the union of P1 and P2 (which is P1 ∪ P2), so the endpoints of P3 are on either P1 or P2. Suppose one endpoint is on P1 (say, vertex v7) and the other is on P2 (say, the last vertex of P2, which is connected to v6 on P1). So P3 connects v7 to the end of P2, passing through two intermediate vertices, making a path of 3 edges.

Then P4 is a path with 2 edges, so 3 vertices. Its endpoints are on the union of P1, P2, P3, so they could be on P3 or P1 or P2. Suppose one endpoint is on P3 (say, the second vertex of P3) and the other is on P1, connecting two vertices on P1 that are not connected by a direct edge in P1, thus adding a path between them.

The key points are that each ear (path) is added such that it connects two vertices on the existing graph (which is the union of all previous ears), and doesn't intersect any previous ears except at their endpoints. Since P1 is a cycle, P2 is a path connecting two vertices on the cycle, P3 connects a vertex on the cycle to a vertex on P2, and P4 connects two vertices on the existing graph (maybe on P3 or the cycle), forming a path that doesn't cross any previous ears.

To describe the ear decomposition, we need to outline the sequence in which the ears are added, specifying how each ear connects to the previous structure. Since the problem mentions "thickest lines depict P1 and the thinnest lines depict P4", we can assume that each subsequent ear (P2, P3, P4) has lines that are thinner than the previous, indicating their order in the decomposition.

So, step by step:
1. Start with P1, a cycle with 10 edges (10 vertices, each connected to the next, forming a closed loop).
2. Add P2, a path with 4 edges. This path has one endpoint on P1 (say, vertex v1) and the other endpoint on another vertex on P1 (say, vertex v6), so the path goes from v1 to v6 through vertices v1, v2, v3, v4, v5, v6? Wait, no, a path with 4 edges has 5 vertices, so the path should have two endpoints on P1, say u and w, with u connected to w via 3 intermediate vertices, so the vertices are u, a, b, c, w, where u and w are on P1, and the edges are u-a, a-b, b-c, c-w (4 edges, 5 vertices). So P2 connects u and w on P1 with a path that doesn't cross P1 except at u and w.
3. Add P3, a path with 3 edges. This path has one endpoint on P1 (say, vertex v7) and the other endpoint on P2 (say, vertex w, which is the endpoint of P2 on P1, or maybe the last vertex of P2, which is connected to w on P1). Wait, the last vertex of P2 is w, which is on P1. So P3 connects v7 (on P1) to, say, the second last vertex of P2 (which is connected to w), so the path goes from v7 to that vertex through two intermediate vertices, making 3 edges (4 vertices: v7, x, y, z, where z is the vertex on P2).
4. Add P4, a path with 2 edges. This path connects two vertices on the existing graph (which now includes P1, P2, P3), say a vertex on P3 (say, x) and a vertex on P1 (say, v8), with one edge connecting them, and another edge connecting to another vertex on P1, making a path of 2 edges (3 vertices: x, a, v8), where x is on P3, a is an intermediate vertex, and v8 is on P1.

Each ear is added such that it's a path connecting two vertices on the current graph (which is the union of all previous ears), and doesn't intersect any previous ears except at their endpoints. This ensures that the entire graph is built up from the initial cycle, adding paths (ears) one by one without overlapping the existing structure except at the endpoints of each ear.

In summary, the ear decomposition starts with a 10-edge cycle (P1), then adds a 4-edge path (P2) connecting two vertices on the cycle, then a 3-edge path (P3) connecting a vertex on the cycle to a vertex on the previous path, then a 2-edge path (P4) connecting two vertices on the existing graph (possibly on the previous path or cycle), each time ensuring the new path doesn't intersect the previous ones except at their endpoints.</think><answer>An ear decomposition of a graph is a sequence of subgraphs (ears) where each ear is either a cycle (initial ear) or a path (subsequent ears) added such that each path connects two vertices on the existing graph (union of previous ears) without intersecting any previous ears except at their endpoints. For the given decomposition \( ED = \{P_1, P_2, P_3, P_4\} \):  

1. **\( P_1 \) (Cycle, 10 edges)**: Start with a 10-vertex cycle (10 edges), forming the initial graph. Each edge connects two consecutive vertices in a closed loop.  
2. **\( P_2 \) (Path, 4 edges)**: Add a path with 4 edges. One endpoint connects to a vertex on \( P_1 \), and the other endpoint connects to another vertex on \( P_1 \), ensuring the path does not intersect \( P_1 \) except at these two endpoints. This path has 5 vertices (4 edges + 2 endpoints).  
3. **\( P_3 \) (Path, 3 edges)**: Add a path with 3 edges. One endpoint connects to a vertex on \( P_1 \), and the other endpoint connects to a vertex on \( P_2 \), ensuring no intersection with \( P_1 \) or \( P_2 \) except at endpoints. This path has 4 vertices (3 edges + 2 endpoints).  
4. **\( P_4 \) (Path, 2 edges)**: Add a path with 2 edges. One endpoint connects to a vertex on \( P_3 \), and the other connects to a vertex on \( P_1 \) or \( P_2 \), ensuring no intersection with previous ears except at endpoints. This path has 3 vertices (2 edges + 2 endpoints).  

Each subsequent ear is added to the union of all previous ears, maintaining the ear decomposition rules. The thickest lines represent \( P_1 \) (the initial cycle), and the thinnest represent \( P_4 \) (the final path), with intermediate paths (\( P_2, P_3 \)) having lines of decreasing thickness.  

The answer is \(<|begin_of_box|>ED = \{P_1, P_2, P_3, P_4\} \text{ as described}<|end_of_box|>\).</answer>