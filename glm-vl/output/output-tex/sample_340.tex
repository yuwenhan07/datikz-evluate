<think>Got it, let's try to figure out this problem. The question is about a graphical illustration of nested opportunity sets in a discrete choice context. The opportunity set B₁ has three dark red spikes, and B₂ has three light red spikes. 

First, I need to recall what opportunity sets represent. In discrete choice, an opportunity set is the set of possible bundles (or choices) a decision-maker can make, given their constraints. Nested opportunity sets mean that one set is contained within another, so B₂ is a subset of B₁, or maybe the other way around? Wait, the problem says B₁ has dark red spikes, B₂ has light red, and they are nested. So probably B₂ is a subset of B₁, meaning the light red spikes are within the dark red ones. 

Discrete choice context implies that the choices are discrete, so the opportunity sets are discrete points. The spikes might represent different bundles, each with a certain utility or payoff. If B₁ has three dark red spikes, maybe those are the outer options, and B₂'s light red spikes are a subset, perhaps a subset of the options in B₁. So the graph would show, say, on a utility or payoff axis, the dark red spikes are at higher or lower points, and the light red are within them. 

Wait, the question is asking for a graphical illustration, but since I need to describe it, let's think: the opportunity sets are sets of points (spikes) on a graph, maybe with two axes, but since it's discrete choice, maybe one axis is the number of options. Wait, no, opportunity sets in choice theory often have two goods, so axes for two goods, and each spike is a bundle (x, y). If B₁ has three dark red spikes, maybe three bundles, and B₂ has three light red spikes, which are a subset of those three, so each light red spike is one of the dark red ones, or maybe a subset. So the graph would show, for example, three dark red points (spikes) as B₁, and three light red points inside or among them as B₂. 

Alternatively, maybe the opportunity sets are represented as regions, but the problem says "spikes", so discrete points. So the illustration would have, say, on a plane, three dark red dots (spikes) forming B₁, and three light red dots (spikes) that are part of B₁, so B₂ is a subset of B₁. So the graph shows B₁ as a set with three points, B₂ as a set with three points, all within B₁. 

I need to make sure the description matches the problem: B₁ has three dark red spikes, B₂ has three light red spikes. So the illustration would show B₁ as a larger set (three points) and B₂ as a smaller set (three points) contained within B₁. Maybe the spikes are arranged in a way that B₂'s points are inside the area of B₁'s points. 

Another angle: in discrete choice, the opportunity set is the set of available alternatives. Nested opportunity sets mean that the set of alternatives in B₂ is a subset of those in B₁. So if B₁ has three options (spikes), B₂ has three options, but those three are among the options in B₁. So the graph would show, for example, three dark red spikes (options in B₁), and three light red spikes (options in B₂), with the light red ones being some of the dark red ones, or maybe a subset. 

I think the key is that B₂ is a subset of B₁, so the light red spikes are within the dark red ones. So the graphical illustration would have, say, on a coordinate system, three dark red points (spikes) representing B₁'s options, and three light red points (spikes) that are located within or among those dark red points, representing B₂'s options. Therefore, the opportunity set B₂ is nested within B₁, meaning all options in B₂ are also in B₁, and possibly some additional options in B₁ not in B₂.</think><answer>To illustrate a family of nested opportunity sets in a discrete choice context, consider a graphical representation where opportunity sets are discrete points (spikes) on a coordinate system.  

- **Opportunity Set \( B_1 \)** (dark red spikes): Contains three distinct bundles (spikes), representing the set of all possible choices available to the decision-maker.  
- **Opportunity Set \( B_2 \)** (light red spikes): Contains three bundles, which are a subset of the bundles in \( B_1 \). This means every choice in \( B_2 \) is also a choice in \( B_1 \), illustrating the "nested" relationship where \( B_2 \subseteq B_1 \).  

In the graph, the dark red spikes ( \( B_1 \) ) form a larger set, while the light red spikes ( \( B_2 \) ) are positioned within or among the dark red spikes, visually demonstrating that \( B_2 \) is a subset of \( B_1 \).  

The answer is \(<|begin_of_box|>B_2 \subseteq B_1<|end_of_box|>\).</answer>