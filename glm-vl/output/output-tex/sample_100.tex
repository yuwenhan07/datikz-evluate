<think>Got it, let's try to parse this description step by step. The user is talking about estimating differences between two forests, probably some kind of tree structures or spatial arrangements. Let's break down the key points:

1. **Radius Changes**: The radius doubles with decremented layer, except for a step where it quadruples. So, maybe each layer's radius is a power of 2, except one transition where it's 4 times instead of doubling. For example, if a layer has radius r, the next layer (decremented, maybe lower layer) has radius 2r, except when moving from k = hS_N - n + 1 to k = hS_N - n, where radius quadruples (so 4r instead of 2r).

2. **Towers tw S_N and tw S_M**: Left circles are new tower tw S_N, right are old tower tw S_M. So these are probably two different tree structures or spatial towers, maybe representing different forest configurations.

3. **Set w_N w_N-1**: Added in the N-th round to the estimated forest. Estimated by the grey set, which is lower layers of tw S_M. The large balls centered at p_S_M are subtracted, while upper small non-concentric circles are ignored. So, when estimating the new forest, they use the lower parts of the old tower (tw S_M) but subtract the large balls (maybe some components) and ignore the upper small circles.

Let me try to structure this:

- **Forest Estimation**: Comparing two forest structures (tw S_N and tw S_M). The estimation involves adjusting radii based on layer changes, with a specific quadrupling step.
- **Layer and Radius Relationship**: Each layer's radius is typically double the previous, except one transition where it's quadruple. So, if layer k has radius r_k, then r_{k+1} = 2r_k except when moving from k = hS_N - n + 1 to k = hS_N - n, then r_{k} = 4r_{k-1} (assuming hS_N - n + 1 is the previous layer).
- **Towers as Structures**: tw S_N (new) and tw S_M (old) are towers, maybe like binary trees or spatial trees with layers. Left circles (tw S_N) vs right (tw S_M).
- **Set w_N w_N-1**: This set is added in the N-th round. Its estimation uses the grey set, which is the lower layers of tw S_M. When estimating, subtract the large balls (maybe the larger components in tw S_M's lower layers) and ignore the upper small non-concentric circles (maybe some upper parts of tw S_M that aren't considered).

I need to make sure I capture the differences: radius changes (double except one quadruple), tower structures (left vs right), set estimation (grey set from old tower's lower layers, subtract large balls, ignore upper small circles).

Maybe the key is to describe the estimation process: when comparing the new forest (tw S_N) with the old (tw S_M), the radius of each layer in the new tower follows a specific pattern (doubles except one quadruple step), while the old tower has a different radius structure. The set added in the N-th round is estimated by taking the lower layers of the old tower, removing the large balls (maybe the larger components that are not part of the new structure), and ignoring the upper small circles (maybe some artifacts or non-essential parts of the old tower's upper layers).

I should also note the specific step where the radius quadruples: from k = hS_N - n + 1 to k = hS_N - n, so that's a transition between two layers where instead of doubling, it quadruples. That might be a key difference between the new and old towers' radius progression.

Putting it all together, the estimation involves adjusting the radius of each layer according to the specified pattern (double except one quadruple step), comparing the two towers (new vs old), and modifying the set w_N w_N-1 by using the lower layers of the old tower, subtracting large balls, and ignoring upper small circles.</think><answer>To understand the estimation of differences between the forests, let’s break down the key components:  

### 1. Radius and Layer Structure  
- **Radius Progression**: For the new tower \( \text{tw } S_N \) (left circles) and old tower \( \text{tw } S_M \) (right circles), the radius typically doubles with each decremented layer (i.e., moving to a lower layer). However, there is a specific transition where the radius **quadruples** instead of doubling: when moving from layer \( k = hS_N - n + 1 \) to \( k = hS_N - n \), the radius becomes 4 times the previous layer’s radius (instead of doubling).  

### 2. Tower Comparison (\( \text{tw } S_N \) vs \( \text{tw } S_M \))  
- The left circles represent the new tower \( \text{tw } S_N \), and the right circles represent the old tower \( \text{tw } S_M \). These towers likely represent different forest structures, with \( S_N \) and \( S_M \) being parameters defining their layering or spatial properties.  

### 3. Estimation of Set \( w_N w_N-1 \)  
- This set is added in the \( N \)-th round to the estimated forest. Its estimation uses a “grey set” derived from the **lower layers** of the old tower \( \text{tw } S_M \).  
  - **Subtraction**: Large balls centered at \( p_{S_M} \) (in the lower layers of \( \text{tw } S_M \)) are subtracted from the estimation.  
  - **Ignored Elements**: Upper small, non-concentric circles (in the upper layers of \( \text{tw } S_M \)) are ignored, as they are not part of the estimated set.  

### Summary of Differences  
The estimation accounts for:  
- Radius changes in the towers (double except one quadruple step),  
- Structural differences between the new (\( \text{tw } S_N \)) and old (\( \text{tw } S_M \)) towers,  
- A targeted estimation of the set \( w_N w_N-1 \) using only the lower layers of the old tower, with specific subtractions and exclusions of certain elements.  

This process adjusts for the radius progression differences and refines the estimated forest by focusing on relevant components of the old tower’s structure.  

The answer is \(<|begin_of_box|>\text{Described the estimation process as outlined}<|end_of_box|>\).</answer>