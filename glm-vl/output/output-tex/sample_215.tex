<think>Got it, let's try to understand the question first. The user is asking about the performance comparison between ris-enabled and beacon-aided cooperative localization, specifically in terms of peb (probably position error bound or similar metric), considering different ris sizes and a fixed random phase profile setup. Also, they mention that with enough ris elements, an active anchor can be replaced by a passive ris without performance loss.

First, let's break down the components:

1. **RIS-Enabled vs Beacon-Aided Cooperative Localization**: RIS (Reconfigurable Intelligent Surface) is a passive element that can reflect signals, while beacon-aided might use active beacons (transmitters) for localization. Cooperative localization involves multiple anchors (beacons or RIS) working together.

2. **PEB (Position Error Bound)**: This is a metric for localization accuracy, so comparing PEB between the two setups.

3. **Different RIS Sizes**: RIS size refers to the number of elements (elements per RIS panel). Larger RIS means more elements, which can affect signal reflection and thus localization accuracy.

4. **Fixed Random Phase Profile Setup**: The phase profile of RIS elements is fixed (not adaptive), so the reflection coefficients are random but fixed during the experiment.

5. **Active Anchor Replacement**: With sufficient RIS elements, a passive RIS can replace an active anchor (beacon) without degrading performance. This implies that RIS, when scaled up, can provide similar localization accuracy as active anchors.

Now, let's structure the explanation. First, explain the two localization methods: beacon-aided uses active transmitters (beacons) to send signals, while ris-enabled uses passive RIS elements to reflect signals from anchors. Cooperative localization means multiple anchors (beacons or RIS) work together. Then, discuss how RIS size affects performance—more elements mean better signal coverage and accuracy, so PEB decreases. With a fixed random phase profile, the reflection is consistent, so the system's performance is stable under that condition. Then, the key point: when RIS has enough elements, it can replace an active anchor (since active anchors might have power or complexity issues), and the performance (PEB) doesn't degrade.

Need to make sure the terms are clear. PEB is likely Position Error Bound, which is a common metric in localization. RIS-enabled uses RIS as part of the anchor network, beacon-aided uses active beacons. Cooperative means multiple anchors (beacons or RIS) collaborate. Different RIS sizes mean varying numbers of elements, so testing with small, medium, large RIS. Fixed random phase means the phase shift of each RIS element is random but not changing during the experiment, so the reflection pattern is fixed.

So, putting it all together: The study compares ris-enabled and beacon-aided cooperative localization in terms of peb, considering different ris sizes (number of elements) with a fixed random phase profile. The result is that as ris size increases (more elements), the peb decreases, and when ris has a sufficient number of elements, a passive ris can replace an active anchor without performance degradation (i.e., peb remains similar or better compared to using active anchors).

Now, check if the question is asking for an explanation or a summary. The user's statement is a conclusion, so maybe they want an explanation of the setup and the conclusion. But the question is presented as a statement, so maybe the user wants a detailed explanation of what the statement means, or perhaps a summary of the findings.

Wait, the user wrote: "peb of ris-enabled vs. beacon-aided cooperative localization for different ris sizes and a fixed random phase profile setup. It is shown that, with a sufficient number of ris elements, an active anchor can be replaced with a passive one (ris) without performance degradation." So the user is presenting a research finding, and maybe wants an explanation or elaboration on that.

So, to answer, we need to explain the comparison (PEB between ris-enabled and beacon-aided), the factors (different ris sizes, fixed random phase), and the conclusion (sufficient ris elements allow replacing active anchor with passive ris without performance degradation).

Let me structure the response:

1. Define the two localization methods: ris-enabled uses RIS (passive elements) to assist in localization, beacon-aided uses active beacons (transmitters) as anchors. Cooperative localization means multiple anchors (beacons or RIS) work together to improve accuracy.

2. Explain PEB: Position Error Bound, a metric for localization accuracy (lower PEB means better accuracy).

3. Discuss the setup: different RIS sizes (number of elements), fixed random phase profile (each RIS element has a random phase shift, but this phase is fixed during the experiment, so the reflection pattern is consistent).

4. Explain the result: As RIS size increases (more elements), the PEB decreases (better accuracy) for ris-enabled localization. When the RIS has a sufficient number of elements, its performance (PEB) is comparable to beacon-aided localization. Therefore, a passive RIS can replace an active anchor (since active anchors might have higher power consumption, cost, or complexity) without degrading performance (i.e., PEB remains similar or better).

So, putting it all together in a coherent explanation.</think><answer>To address the question about the comparison of **PEB (Position Error Bound, a metric for localization accuracy)** between ris - enabled and beacon - aided cooperative localization under specific setups, here’s a breakdown:  


### 1. Key Components of the Comparison  
- **RIS - enabled localization**: Uses *Reconfigurable Intelligent Surfaces (RIS)*—passive elements that reflect signals from anchors (either active beacons or other RIS) to assist in positioning.  
- **Beacon - aided cooperative localization**: Relies on *active beacons* (transmitters) as anchors, with multiple anchors working together to improve localization accuracy.  
- **PEB (Position Error Bound)**: A standard metric for evaluating localization accuracy (lower PEB = better accuracy).    

### 2. Experimental Setup Details  
- **Different RIS sizes**: The number of elements on the RIS panel is varied (small, medium, large). More RIS elements typically provide better signal coverage and reflection quality, which can improve localization accuracy.  
- **Fixed random phase profile**: Each RIS element has a random phase shift, but this phase is *fixed* during the experiment. This means the reflection pattern from the RIS is consistent, allowing for stable performance evaluation.    

### 3. Key Finding: Active Anchor Replacement  
When the RIS has a *sufficient number of elements*, its localization performance (measured by PEB) is comparable to that of beacon - aided systems. This means a passive RIS can replace an active anchor (beacon) without degrading performance.  

### Explanation of the Result  
- For ris - enabled localization, increasing RIS size (more elements) reduces PEB (improves accuracy) because more elements allow for better signal manipulation and coverage.  
- With a fixed random phase profile, the RIS’s reflection behavior is stable, so scaling up the RIS size consistently improves performance.  
- When the RIS has enough elements, its ability to reflect signals effectively becomes equivalent to the signal strength/coverage provided by an active beacon. Thus, a passive RIS can act as a viable replacement for an active anchor, maintaining the same level of localization accuracy (PEB).    

In summary, ris - enabled localization with a large enough RIS can match or exceed the performance of beacon - aided cooperative localization, enabling the replacement of active anchors with passive RIS without performance degradation.</answer>