<think>Got it, let's try to figure out how to rotate a line used for dimensioning the radius of a circle when the `rotate around` command doesn't work with polar coordinates. First, I need to recall what the problem is. The line is probably a dimension line or a radius dimension, and when trying to rotate it around a point (maybe the center of the circle), the polar coordinates might be causing issues. 

First, let's think about the alternatives. If `rotate around` isn't working with polar points, maybe we can use a different rotation method. For example, using the `rotate` command with a base point that's not in polar coordinates, or converting the polar coordinates to Cartesian, or using other commands like `move` combined with `rotate` in a different way. Wait, maybe the issue is that the points defining the line are in polar coordinates, so when you try to rotate around a polar point, the command isn't handling it. 

Another approach: instead of rotating the line directly, maybe we can rotate the entire circle or the dimension object, but that might not be what's needed. Alternatively, use the `rotate` command with a different base point. Let's break it down step by step. 

First, identify the line you want to rotate. It's the radius dimension line, probably two points: one at the center of the circle, and the other at the circumference. If the center is the rotation point, but the command doesn't work with polar coordinates, maybe convert the center's polar coordinates to Cartesian (x, y) and use those as the base point. Then, use the `rotate` command with the base point as the Cartesian coordinates. 

Alternatively, use the `rotate` command with the "base point" option, specifying the center as the base point, even if it's in polar. Wait, maybe the problem is that the line is defined by points in polar, so when you try to rotate around a polar point, the command isn't interpreting it correctly. So, maybe the workaround is to use the `rotate` command with a different base point, like the center of the circle (which might be a Cartesian point if the circle is defined with Cartesian coordinates), or use the `move` command to move the line to a position where rotation is easier, then rotate, then move back. 

Another idea: use the `rotate` command with the "reference" method. The reference method allows you to specify a reference angle and a new angle, which might work even with polar points. Let's think about the steps:

1. Select the line to rotate.
2. Use the `rotate` command.
3. Instead of picking a point, type the base point's coordinates (if known, like the center of the circle, which might be (0,0) if it's centered at the origin, or some other Cartesian point).
4. Then specify the rotation angle.

Wait, but if the center is in polar coordinates, say (r, θ), converting that to Cartesian is (r*cosθ, r*sinθ), then use those as the base point. So, if the circle is centered at (r, θ) in polar, convert to (x, y) = (r*cosθ, r*sinθ), then use that as the base point for rotation. 

Alternatively, use the `rotate` command with the "around" option, but specify the point as a Cartesian coordinate. For example, if the center is at (h, k), type `h, k` when prompted for the base point. 

Another approach: if the line is a dimension line for the radius, maybe the dimension object is a block or a line, and rotating the block might be easier. But the question is about the line used to dimension the radius, so maybe the line is a straight line from center to circumference. 

Wait, maybe the issue is that the `rotate around` command in some CAD software (like AutoCAD) might not handle polar points directly, so the workaround is to use the `rotate` command with a different base point, or use the `move` command to move the line to a position where rotation is possible, then rotate, then move back. Or use the `rotate` command with the "base point" as the center, even if it's in polar, by specifying the coordinates. 

Let me think of a specific example. Suppose the circle is centered at point C (polar coordinates (R, 0)), and the radius dimension line is from C to point P (polar coordinates (R, θ)). To rotate the line CP around point C, you would need to rotate the line CP around C. If the CAD software's `rotate` command doesn't accept polar coordinates for the base point, you can convert C's polar coordinates to Cartesian (x = R*cos(0), y = R*sin(0)) = (R, 0) if θ=0, but if the center is at (r, θ_c), then x = r*cosθ_c, y = r*sinθ_c. Then use those x, y as the base point for rotation. 

So, the steps would be:

1. Identify the center of the circle (base point for rotation) in polar coordinates: (r, θ_c).
2. Convert this to Cartesian coordinates: (x_c, y_c) = (r*cosθ_c, r*sinθ_c).
3. Use the `rotate` command, select the radius dimension line (the line from center to circumference).
4. When prompted for the base point, enter the Cartesian coordinates (x_c, y_c).
5. Specify the rotation angle.

If the CAD software allows entering coordinates directly when prompted for the base point, this should work. Alternatively, if you can't enter coordinates, you might need to use the `move` command to move the line to a temporary position, rotate it there, then move it back. But that's more steps. 

Another workaround could be using the `rotate` command with the "reference" option. The reference option allows you to specify a reference angle and a new angle. For example, if the original line is at angle α, and you want to rotate it to angle β, you can use the reference method. The steps would be:

1. Select the line.
2. Type `rotate` (or the command name for your CAD software).
3. When prompted for base point, pick a point (maybe the center, but if polar, convert to Cartesian).
4. When prompted for rotation angle, type `r` for reference.
5. Specify the reference angle (the original angle of the line relative to the base point).
6. Specify the new angle (the desired angle after rotation).

This might work even if the base point is in polar, as long as the angles are correctly specified. 

Alternatively, if the line is a dimension object, maybe you can explode the dimension, rotate the individual lines, then recombine them, but that's more complex. 

In summary, the main workarounds are converting polar coordinates to Cartesian for the base point, using the reference rotation method, or using a combination of move and rotate commands. The key is to ensure the base point for rotation is in a coordinate system that the `rotate around` command can handle, which is usually Cartesian.</think><answer>To work around the issue of rotating a radius dimension line (in polar coordinates) when the `rotate around` command fails, follow these steps:  

### 1. Convert Polar Coordinates to Cartesian for the Base Point  
If the rotation base point (e.g., the circle’s center) is in polar coordinates `(r, θ)`, convert it to Cartesian coordinates `(x, y) = (r·cosθ, r·sinθ)`. This allows the `rotate` command to use a valid Cartesian base point.  

### 2. Use the `rotate` Command with Cartesian Base Point  
- Select the radius dimension line (the line from the circle’s center to its circumference).  
- Enter the `rotate` command (e.g., `rotate` in AutoCAD).  
- When prompted for the **base point**, type the Cartesian coordinates of the circle’s center (from step 1).  
- Specify the desired rotation angle.  

### 3. Alternative: Use the “Reference” Rotation Method  
If direct coordinate entry isn’t possible, use the reference rotation option:  
- Select the line, then enter `rotate`.  
- When prompted for the base point, pick a point (e.g., the circle’s center, converted to Cartesian if needed).  
- When prompted for the rotation angle, type `r` (for “reference”).  
- Specify the original angle of the line relative to the base point, then the new desired angle.  

### 4. Workaround with Move + Rotate  
If rotation still fails, temporarily move the line to a position where rotation is easier, rotate it, then move it back:  
- Use `move` to shift the line to a temporary location.  
- Rotate the line using the `rotate` command (with a valid base point).  
- Use `move` again to return the line to its original position (adjusted for rotation).  

By converting polar coordinates to Cartesian for the base point or using the reference method, you can successfully rotate the radius dimension line around the circle’s center.  

The answer is <|begin_of_box|>Use Cartesian coordinates for the rotation base point or the reference rotation method.<|end_of_box|></answer>