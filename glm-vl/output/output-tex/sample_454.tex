<think>Got it, let's try to figure out this problem step by step. First, let's recall what the setup is. We have the Euclidean Poincaré AdS₃, which is a hyperbolic space with a metric that looks like ds² = -dx² - dz² + dy² / (1 + x² + z²), maybe? Wait, the Poincaré AdS metric in 3D is usually written as ds² = (1 + r²)dt² + (1 + r²)^{-1}dr² + r²dΩ², but maybe in the setup here, the coordinates are x, y, z? Wait, the problem mentions a brane centered at x=0, sitting at x² + z² = R². So the brane is a circle in the x-z plane, radius R, centered at the origin. The minimal surface is anchored at x = L, and it's a portion of a semi-circle meeting the brane orthogonally. 

First, let's visualize the geometry. The Euclidean AdS space has a metric that's conformally flat, so maybe we can use conformal coordinates. The brane is a codimension-1 surface (a 2-sphere? Wait, x² + z² = R² is a 2-sphere in 3D, so the brane is a 2-sphere of radius R in the x-z plane. The minimal surface is anchored at x = L, which is a point on the x-axis, and it's a semi-circle meeting the brane orthogonally. 

A minimal surface in AdS often has a conformal diagram that's a semi-circle, especially if it's anchored to a boundary. But here, the brane is a submanifold, and the minimal surface meets it orthogonally. So the minimal surface should be tangent to the normal vector of the brane. The brane's normal vector in the x-z plane would be radial, so the minimal surface's tangent should be perpendicular to that. 

Let's consider the geometry near the brane. The brane is a circle of radius R in the x-z plane. The minimal surface is a semi-circle in the x-y plane? Wait, the problem says it's a portion of a semi-circle and meets the brane orthogonally. If the brane is in the x-z plane, then the minimal surface meeting it orthogonally would have a tangent plane that's perpendicular to the brane's tangent plane. The brane is a 2-sphere, so its tangent plane at any point is perpendicular to the position vector (x, 0, z) (since the brane is x² + z² = R², the normal vector is (x, 0, z)/R). So the minimal surface's tangent plane should be perpendicular to the normal vector of the brane, meaning the minimal surface's tangent is in the direction of the brane's tangent plane. 

If the minimal surface is a semi-circle in the x-y plane, anchored at x = L, then its equation might be y² + (x - a)² = b², but since it's a semi-circle meeting the brane orthogonally, the point of intersection should have the normal vector of the minimal surface equal to the normal vector of the brane. The normal vector of the minimal surface (which is a graph over the x-axis, say y = f(x)) would be (f'(x), -1, 0) normalized, but maybe it's easier to use the fact that in AdS, minimal surfaces are often geodesics or have conformal properties. 

Alternatively, since the minimal surface is a portion of a semi-circle and meets the brane orthogonally, the minimal surface should be a circle in the x-y plane (or x-z plane) that is orthogonal to the brane. The brane is x² + z² = R², so a surface orthogonal to it would have a normal vector that's a linear combination of ∂/∂x and ∂/∂z, and the minimal surface being a semi-circle in the x-y plane (with z = 0?) might not fit. Wait, maybe the minimal surface is in the x-y plane, with z = 0, but then the brane is in x-z plane, so the intersection would be along x² = R², z=0, which is a circle of radius R in the x-y plane? Hmm, I need to think about the geometry more carefully. 

In the Euclidean Poincaré AdS₃, the metric can be written as ds² = (1 + r²)(dx² + dy² + dz²)/r², where r is the radial coordinate. The brane is defined by x² + z² = R², which is a 2-sphere of radius R in the x-z plane. The minimal surface is anchored at x = L, which is a point on the x-axis (z=0, y arbitrary? Wait, the problem says anchored to x = L, maybe in the x direction, so the point is (L, y, 0) for some y, but the problem says "anchored to x = L", probably meaning the minimal surface connects to the boundary at x = L, which is a point on the boundary of AdS. 

A minimal surface in AdS anchored to a boundary is typically a semi-circle in the conformal diagram, but here the surface is meeting a brane orthogonally. The condition for orthogonality is that the normal vector to the minimal surface is parallel to the normal vector of the brane. The normal vector to the brane is (x, 0, z)/R, and the normal vector to the minimal surface, if it's a graph y = f(x), would be (f'(x), -1, 0) normalized. Setting them parallel would mean (f'(x), -1, 0) is proportional to (x, 0, z), but since the minimal surface is in the x-y plane (z=0), then z=0, so the normal vector to the minimal surface would be (f'(x), -1, 0), and the brane's normal vector at the intersection point (x, 0, z) = (x, 0, sqrt(R² - x²)) would be (x, 0, sqrt(R² - x²))/R. For orthogonality, the two normal vectors must be parallel, so (f'(x), -1, 0) = λ(x, 0, sqrt(R² - x²))/R. This implies that -1 = λ sqrt(R² - x²)/R, and f'(x) = λ x/R. From the first equation, λ = -R / sqrt(R² - x²), then f'(x) = -R x / [R sqrt(R² - x²)] = -x / sqrt(R² - x²). Integrating f'(x) gives f(x) = -sqrt(R² - x²) + C. If the minimal surface is anchored at x = L, then when x = L, y = f(L) should be the anchor point. Suppose the anchor point is (L, 0, 0) (on the x-axis), then f(L) = 0 = -sqrt(R² - L²) + C => C = sqrt(R² - L²). Therefore, f(x) = sqrt(R² - x²) - sqrt(R² - L²). Wait, but the problem says the minimal surface is a portion of a semi-circle. A semi-circle in the x-y plane would have the equation (x - a)² + y² = b², which is a circle, and if it's a semi-circle, maybe the upper or lower half. If f(x) is a semi-circle, then its derivative should be a linear function or something, but the derivative we found is -x / sqrt(R² - x²), which is not linear unless R is infinite, which it's not. Hmm, maybe I made a mistake in the coordinate system. 

Alternatively, since the minimal surface is a semi-circle meeting the brane orthogonally, and the brane is a circle of radius R, the minimal surface should be a circle of radius R as well, but centered somewhere. If it's anchored at x = L, then the center of the semi-circle might be at (0, 0, 0) in the x-z plane, but meeting the brane orthogonally. Wait, the brane is x² + z² = R², so a circle in the x-z plane. A semi-circle in the x-y plane meeting it orthogonally would have to have its tangent at the intersection point perpendicular to the brane's tangent. The brane's tangent plane at a point (x, z) is perpendicular to (x, 0, z), so the minimal surface's tangent plane should be parallel to the x-y plane? Wait, if the minimal surface is in the x-y plane, then its normal vector is in the z direction, but the brane's normal vector is also in the z direction (for the x-z plane), so they would be parallel, meaning the surface is orthogonal. Wait, if the minimal surface is a plane, but a plane in AdS is not a minimal surface unless it's a boundary, but the problem says it's a portion of a semi-circle. 

Wait, let's think about the conformal diagram. In the Euclidean AdS, the minimal surface anchored to the boundary is a semi-circle, and if there's a brane inside, the minimal surface would adjust to meet the brane orthogonally. The key property is that the minimal surface meets the brane orthogonally, so the angle between the minimal surface and the brane is 90 degrees. In the Poincaré AdS metric, the minimal surface equation can be derived using the variational principle, but maybe there's a simpler way here. 

Given that the minimal surface is a semi-circle and meets the brane orthogonally, and the brane is a circle of radius R, the minimal surface should also have a radius related to R. If the minimal surface is anchored at x = L, then the distance from the center of the minimal surface to the anchor point should be related to R. If the minimal surface is a semi-circle with radius R, then the center would be at x = 0, y = 0, and the semi-circle would extend from x = -R to x = R, but anchored at x = L, which would mean L = R or L = -R, but the problem says "a portion of a semi-circle", so maybe the minimal surface is a quarter-circle or something. 

Wait, another approach: in the Euclidean AdS, the minimal surface connecting two points on the boundary is a semi-circle, and if there's a submanifold (the brane) that the surface meets orthogonally, then the surface should be a geodesic in the AdS space that is orthogonal to the brane. The brane is a 2-sphere, so the minimal surface should be a geodesic that is orthogonal to the 2-sphere, which in AdS would be a circle. The radius of this circle would depend on the position of the anchor point. If the anchor point is at x = L, then the minimal surface is a circle of radius R, centered at the origin, but only the part that's a semi-circle meeting the brane. Wait, if the minimal surface is a semi-circle in the x-y plane with center at (0, 0) and radius R, then it would be the set of points (x, y) with x² + y² = R², y ≥ 0 (a semi-circle), and this meets the brane x² + z² = R² orthogonally. At the intersection points, the normal vector to the minimal surface is (x, y, 0)/R (since it's a circle in x-y plane), and the normal vector to the brane is (x, 0, z)/R. For them to be parallel, (x, y, 0) must be parallel to (x, 0, z), which implies y = 0 and z = 0, but that's only at x = ±R, y = 0, z = 0, which is a single point, not a surface. Hmm, maybe I need to consider the minimal surface as a graph over the brane. 

Alternatively, since the problem states the minimal surface is a portion of a semi-circle and meets the brane orthogonally, the minimal surface must be a circle that is orthogonal to the brane. The brane is a 2-sphere, so the minimal surface, being a circle, must be a geodesic circle in AdS that is orthogonal to the 2-sphere. In AdS, geodesic circles orthogonal to a 2-sphere of radius R would have a radius related to R by the AdS curvature. But maybe in this case, since the minimal surface is a semi-circle, its radius should be equal to R, and the anchor point is at x = L, so the distance from the center of the minimal surface to the anchor point is L, and since it's a semi-circle of radius R, we have L = R. Wait, if the minimal surface is a semi-circle of radius R centered at the origin, then it extends from x = -R to x = R, so if it's anchored at x = L, then L must be R or -R. But the problem says "a portion of a semi-circle", so maybe the minimal surface is a quarter-circle, but I need to think more carefully. 

Wait, let's recall that in the Euclidean AdS, the minimal surface equation for a surface anchored to a boundary and meeting a submanifold orthogonally can be solved using the method of images or by conformal mapping. Given that the brane is a circle of radius R, and the minimal surface is a semi-circle meeting it orthogonally, the minimal surface should have a radius equal to R, and the anchor point is at a distance R from the center of the brane. If the brane is centered at the origin, then the anchor point at x = L is at a distance L from the origin, so if the minimal surface is a semi-circle of radius R, then L should be equal to R, making the minimal surface a semi-circle from x = -R to x = R, anchored at x = R (or -R), which matches the condition. 

So, putting it all together, the minimal surface is a semi-circle of radius R centered at the origin, meeting the brane (which is also a circle of radius R centered at the origin) orthogonally. The anchor point at x = L is at x = R, so L = R. Therefore, the minimal surface is the upper (or lower) semi-circle in the x-y plane with radius R, centered at the origin, and the brane is the circle x² + z² = R² in the x-z plane. The minimal surface meets the brane orthogonally at the point where x = R, z = 0, y = 0, but wait, the intersection should be along the circle where the semi-circle meets the brane. If the minimal surface is in the x-y plane (z=0), then its intersection with the brane (x² + z² = R²) is when z=0, so x² = R², so the intersection points are (R, 0, 0) and (-R, 0, 0). The minimal surface is a semi-circle, say the upper half, so it meets the brane at (R, 0, 0) and (-R, 0, 0), but the problem says it's anchored at x = L, so if L is R, then it's anchored at (R, 0, 0), and the minimal surface is the semi-circle from x = -R to x = R, y ≥ 0, meeting the brane orthogonally at those points. 

I think the key here is recognizing that the minimal surface, being a semi-circle and meeting the brane orthogonally, must have the same radius as the brane, and the anchor point is at a distance equal to the brane's radius from the center. Therefore, the minimal surface is a semi-circle of radius R, centered at the origin, and the anchor point is at x = R (or -R), so L = R.</think><answer>To analyze the minimal surface in the Euclidean Poincaré AdS₃ with the given brane, we start by noting the brane is a 2-sphere of radius \( R \) in the \( x-z \) plane, defined by \( x^2 + z^2 = R^2 \). The minimal surface, a portion of a semi-circle, must meet the brane orthogonally, meaning its normal vector aligns with the brane's normal vector.  

In AdS geometry, a minimal surface anchored to a boundary and meeting a submanifold orthogonally often forms a geodesic circle. Given the semi-circle geometry and the brane's radius \( R \), the minimal surface must also have radius \( R \), centered at the origin. Anchoring this semi-circle at \( x = L \) implies the anchor point is at a distance \( R \) from the brane's center, so \( L = R \).  

Thus, the minimal surface is a semi-circle in the \( x-y \) plane (with \( z = 0 \)) defined by \( x^2 + y^2 = R^2 \) (for \( y \geq 0 \)), meeting the brane orthogonally at \( x = \pm R \), \( z = 0 \).  

The answer is \(<|begin_of_box|>R<|end_of_box|>\).</answer>