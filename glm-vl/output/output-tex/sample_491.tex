<think>Got it, let's try to figure out the fundamental domain of a group G. First, I need to recall what a fundamental domain is. From what I remember, a fundamental domain (or fundamental region) for a group of transformations, like a lattice in the plane or a group of isometries, is a subset of the space such that every point in the space is equivalent to exactly one point in the domain under the group's action, and the domain is chosen to be "minimal" in some sense, usually with the least area or covering the space without overlapping too much.

But the image description mentions a white and black circle with a gray arrow. Maybe the group G is related to rotations or translations, like a dihedral group or a lattice group. If it's a rotation group, say the group of rotations by 120 degrees, the fundamental domain might be a sector of the circle. If it's a translation group, like a lattice in the plane, the fundamental domain could be a parallelogram or a hexagon, depending on the lattice.

Wait, the image has a circle with an arrow. Maybe the group is the modular group or some other group acting on the plane. Alternatively, if it's a finite group of rotations, the fundamental domain would be a region that repeats under the group's action. For example, for the group of rotations of a regular polygon, the fundamental domain is a "wedge" or a sector.

But without more specific information about the group G, it's hard to be exact. However, the key idea is that a fundamental domain is a set such that the orbit of each point in the domain covers the entire space, and the domain is chosen to be as simple as possible. If the image shows a circle with an arrow, maybe the group is the group of rotations around the center of the circle, and the fundamental domain is a sector with central angle equal to 360 degrees divided by the order of the group. For example, if it's a rotation by 60 degrees, the fundamental domain would be a 60-degree sector.

Alternatively, if the group is a lattice in the plane, like Z², the fundamental domain is a rectangle or a hexagon, but the image has a circle, so maybe the group is acting on the circle, like rotations of the circle, and the fundamental domain is an arc. Wait, the image says "white and black circle with a gray arrow inside it". Maybe the circle is a fundamental domain for a group acting on the circle, like the modular group acting on the upper half-plane, but that's more complex.

Alternatively, think of the group as the symmetry group of a regular polygon. For a regular n-gon, the symmetry group is the dihedral group Dn, and the fundamental domain would be a region between two radii separated by an angle of 2π/n. So if the arrow is indicating a rotation, the fundamental domain would be a sector with angle 2π/n.

But since the image isn't provided, I need to make a general statement. The fundamental domain of a group G is a subset D of the space such that G·D = the entire space (the orbit of D under G covers the space), and D ∩ G·D' = empty for any other subset D' with the same property (or something like that, ensuring minimality). For example, in the case of a lattice in the plane, the fundamental domain is often a parallelogram, and for a rotation group, a sector.

Given the image description mentions a circle with an arrow, maybe the group is the cyclic group acting by rotation, and the fundamental domain is a sector of the circle. So the answer would involve describing the fundamental domain as a region (like a sector) that is mapped onto the entire space by the group's action, with the least area possible.</think><answer>To determine the fundamental domain of a group \( G \), we first recall that a fundamental domain (or region) is a subset of the space such that every point in the space is equivalent to exactly one point in the domain under \( G \)'s action, and the domain is chosen to be "minimal" (e.g., with the smallest area or simplest shape) to avoid overlap.  

For a group acting on a space (e.g., rotations, translations, or symmetries), the fundamental domain is constructed to cover the entire space without redundancy. If the image depicts a circle with an arrow (suggesting rotational symmetry), \( G \) is likely a cyclic group of rotations. For such a group, the fundamental domain is typically a sector of the circle with a central angle equal to \( \frac{2\pi}{|G|} \), where \( |G| \) is the order of the group. This sector is mapped onto the entire circle by the group's rotations, ensuring every point on the circle is equivalent to a point in the sector under \( G \)'s action.  

In summary, the fundamental domain for a rotational group acting on a circle is a sector with the appropriate central angle, chosen to minimize area while covering the space without overlap.  

The answer is \(<|begin_of_box|>\text{A sector of the circle with central angle } \frac{2\pi}{|G|}<|end_of_box|>\).</answer>