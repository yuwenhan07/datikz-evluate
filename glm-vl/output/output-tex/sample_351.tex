<think>Got, let's try to figure out the problem here. The question is about function space equality between L²-space and the homogeneous Sobolev space H⁰. First, I need to recall what these spaces are. 

L²-space is the space of square-integrable functions, so functions f such that the integral of |f|² over the domain is finite. Then H⁰, which is the homogeneous Sobolev space, usually for a domain Ω, H⁰(Ω) is the closure of C_c^∞(Ω) (smooth functions with compact support in Ω) under the H¹ norm. Wait, but sometimes H⁰ might be defined differently, but in one dimension, maybe? Wait, the homogeneous Sobolev space H^k_0(Ω) is the closure of C_c^∞(Ω) in H^k(Ω), but H⁰ might be the same as L² if the space is homogeneous? Wait, no, H^0 is usually the same as L², but maybe in some contexts, H^0 is the space of functions with zero trace, but that's H^1_0. Wait, maybe the question is about H^0 being equal to L². Let me think. 

In one dimension, say Ω = ℝ^n, the homogeneous Sobolev space H^k_0(Ω) is the closure of C_c^∞(Ω) in H^k(Ω). But if we take k=0, then H^0_0(Ω) would be the closure of C_c^∞(Ω) in L²(Ω), which is just the space of functions with compact support, but wait, no, the closure in L² is the space of functions with finite L² norm, but if we take the closure in H^0 (which is the same as L²), then maybe H^0 is equal to L². Wait, but the homogeneous version might have different properties. Wait, another approach: the Sobolev space H^k(Ω) consists of functions in L²(Ω) whose weak derivatives up to order k are also in L²(Ω). The homogeneous Sobolev space H^k_0(Ω) is the closure of C_c^∞(Ω) in H^k(Ω). So if we consider H^0_0(Ω), that's the closure of C_c^∞(Ω) in H^0(Ω), but H^0(Ω) is just L²(Ω), so the closure of C_c^∞(Ω) in L²(Ω) is the space of functions with compact support, but wait, no, the closure of C_c^∞(Ω) in L²(Ω) is the space of functions with compact support, but actually, in general, the closure of C_c^∞(Ω) in L²(Ω) is the space of functions with compact support, but if the domain is, say, the whole space ℝ^n, then C_c^∞(ℝ^n) is dense in L²(ℝ^n), so H^0_0(ℝ^n) = L²(ℝ^n). But if the domain is a bounded domain with boundary, then H^0(Ω) is the same as L²(Ω), and H^0_0(Ω) is the closure of C_c^∞(Ω) in L²(Ω), which is the space of functions with compact support, but wait, no, in a bounded domain, the closure of C_c^∞(Ω) in L²(Ω) is the space of functions in L²(Ω) with compact support, but actually, in a bounded domain, the space of functions with compact support is dense in L²(Ω), so H^0_0(Ω) = L²(Ω). Wait, but maybe the question is stating that H^0 is equal to L², which would make sense if H^0 is defined as L². Alternatively, maybe in some contexts, H^0 is the space of functions with zero mean or something, but I need to recall the standard definitions. 

Wait, another angle: the homogeneous Sobolev space H^k_0(Ω) is often used when dealing with boundary conditions, like Dirichlet boundary conditions, and it's a subspace of H^k(Ω). But if we take k=0, then H^0_0(Ω) is just the space of functions in L²(Ω) with compact support, but is that equal to L²(Ω)? No, because L²(Ω) includes functions without compact support, like constants. Wait, but if the domain is the whole space, then H^0_0(ℝ^n) is dense in L²(ℝ^n), but not equal. Hmm, maybe the question is in one dimension, or maybe the space is defined on a domain where H^0 is the same as L². Alternatively, maybe the problem is that H^0 is the same as L², so the equality is H^0 = L². But why would that be? Wait, the Sobolev space H^k is defined as the space of functions in L² with weak derivatives up to order k in L². For k=0, that's just L², so H^0(Ω) = L²(Ω). But then what is H^0_0(Ω)? That's the closure of C_c^∞(Ω) in H^0(Ω), which is L²(Ω), so H^0_0(Ω) is dense in L²(Ω), but not equal unless the domain is such that C_c^∞(Ω) is dense in L²(Ω), which it is, but the closure is the whole space? Wait, no, the closure of C_c^∞(Ω) in L²(Ω) is the space of functions with compact support, but in a bounded domain, functions with compact support are dense in L², but not equal. Wait, I might be confusing the notation here. Let me check standard references. In Adams' Sobolev Spaces, H^k(Ω) is the Sobolev space of order k, and H^0(Ω) is indeed L²(Ω). The homogeneous version H^k_0(Ω) is the closure of C_c^∞(Ω) in H^k(Ω). So H^0_0(Ω) would be the closure of C_c^∞(Ω) in L²(Ω), which is the space of functions with compact support, but that's not equal to L²(Ω) unless the domain is empty, which doesn't make sense. Wait, maybe the question is referring to H^0 as the space of functions with zero mean, but that's a different space. Alternatively, if the domain is a single point, but that's trivial. Wait, the user mentioned a white circle on a white background, maybe the image was supposed to show a graph or something, but since it's not visible, maybe the problem is a standard one where H^0 is equal to L². Given that H^0 is the same as L², then the equality is H^0 = L². But I need to confirm. If we take the definition that H^k(Ω) = W^{k,2}(Ω), then H^0(Ω) = W^{0,2}(Ω) = L²(Ω). So the homogeneous version H^0_0(Ω) would be the closure of C_c^∞(Ω) in L²(Ω), which is dense but not equal. But if the question is about H^0 without the homogeneous subscript, then H^0 = L². So the equality is H^0(Ω) = L²(Ω). Therefore, the function space equality is that the homogeneous Sobolev space H^0 is equal to the L² space.</think><answer>To determine the function space equality between the \( L^2 \)-space and the homogeneous Sobolev space \( H^0 \), we start by recalling their definitions:  

- The \( L^2 \)-space consists of square-integrable functions \( f \) such that \( \int_\Omega |f|^2 \, dx < \infty \), denoted \( L^2(\Omega) \).  
- The homogeneous Sobolev space \( H^k_0(\Omega) \) is the closure of \( C_c^\infty(\Omega) \) (smooth functions with compact support in \( \Omega \)) under the \( H^k \)-norm. For \( k = 0 \), \( H^0_0(\Omega) \) is the closure of \( C_c^\infty(\Omega) \) in \( L^2(\Omega) \).  

However, the standard Sobolev space \( H^0(\Omega) \) (without the homogeneous subscript) is defined as the space of functions in \( L^2(\Omega) \) whose weak derivatives up to order 0 (i.e., the function itself) are in \( L^2(\Omega) \). Since the 0th weak derivative of a function is the function itself, \( H^0(\Omega) \) is precisely \( L^2(\Omega) \).  

Thus, the equality \( H^0(\Omega) = L^2(\Omega) \) holds, as both spaces consist of square-integrable functions.  

The answer is \(<|begin_of_box|>L^2 = H^0<|end_of_box|>\).</answer>