<think>Got it, let's try to understand the problem step by step. The question is about proving that V₀ = 0 using a specific path, and then showing that M is embedded by another path. There are various paths mentioned: from -1 to (5), then dotted line to (1), dotted line to (6), then to +1. Also, a dashed line from (1) to (2) is a competitor, and the solid lines form the path to show M is embedded.

First, let's recall what it means for a map to be embedded. An embedded submanifold is one that is a submanifold and the inclusion map is an embedding, meaning it's a homeomorphism onto its image with the subspace topology. To show M is embedded, we need to show that the neighborhood around each point in M is homeomorphic to an open subset of Euclidean space, and the inclusion map doesn't have self-intersections or singularities.

The path from -1 to +1 via solid lines probably shows that the neighborhood around the path is a product or a manifold without boundary, indicating the embedding. The non-embedded point z₀ is likely a point where the map fails to be an embedding, maybe due to self-intersection or a singularity, and the local pictures at each stage (numbered stages on the paths) show how the map behaves near z₀, which is why we need to prove V₀ = 0, maybe using some integral or differential equation along the path.

For proving V₀ = 0, following the path from -1 to (5), then dotted line to (1), dotted line to (6), then to +1. Each stage's local picture might show that the function V (maybe a vector field or a potential function) has certain properties along the path, and by the path integral or some conservation law, the value at the start and end must be equal, hence V₀ = 0. The competitor path (dashed line from 1 to 2) might be used to show that if V₀ weren't zero, there would be a contradiction, or that the integral along the competitor differs, but the solid path gives the correct value.

Let me try to structure this:

1. To prove V₀ = 0: Use the path consisting of several segments. Along each segment, if V is a conservative vector field (so its integral is path-independent), then the total integral from -1 to +1 along the composite path should be zero if the endpoints are the same. The path goes through various points, and the local pictures at each stage (like at point 1, 5, 6) show that the function behaves in a way that the contributions cancel out, leading to V₀ = 0.

2. To prove M is embedded: The solid path from -1 to +1 is used to show that the neighborhood around the path is a manifold without self-intersections. The local pictures at each stage on the solid path show that the map is a homeomorphism onto its image, with no singularities or self-intersections, hence the inclusion map is an embedding.

The non-embedded point z₀ is probably where the map fails, maybe at a point where the local picture shows a singularity or self-intersection, which is why we need to ensure that along the embedding path (solid lines), there are no such issues, while the competitor path (dashed) might show a problem, hence the need to use the solid path for the embedding proof.

I need to make sure I connect the path details with the concepts of embedded submanifolds and proving a function value is zero. Maybe using the fact that if a function is constant along a closed path, then its value is the same at the start and end. If the path is closed, then V₀ = V_+1, but if -1 and +1 are the same point, then V₀ = V_+1, but if they are different, need to check. Alternatively, if the path is from -1 to +1, and if the function is defined on a manifold with boundary, but maybe it's a closed manifold, so the integral around a closed loop is zero, hence V₀ = 0.

Another angle: The problem mentions "local pictures about the non-embedded point z₀ at each numbered stage on the paths." So at each stage (like at point 1, 2, 5, 6), the local picture shows the behavior near z₀, which is the point where the embedding fails. To prove M is embedded, we need to show that near each point on the solid path, the map is a homeomorphism, which the local pictures confirm, while the non-embedded point's local picture shows a problem, hence the need to avoid that path for the embedding.

Putting it all together, the key steps are using the path integral to show V₀ = 0 (assuming V is a closed form or conservative field), and using the solid path's local properties to show the embedding, while the non-embedded point's local behavior is why the competitor path (dashed) might not work, hence the solid path is necessary.</think><answer>To address the problem, we analyze the paths and their implications for proving \( V_0 = 0 \) and that \( M \) is embedded:  

### 1. Proving \( V_0 = 0 \) via the Composite Path  
The path from \(-1\) to \((5)\), then dotted lines to \((1)\), \((6)\), and finally to \(+1\) is used. If \( V \) is a conservative vector field (or a closed differential form), its integral along a closed or path-independent path is zero. The local pictures at each numbered stage show that contributions from intermediate points cancel out, ensuring the total integral along the composite path equals zero. Since the path starts and ends at \(-1\) and \(+1\), if these are the same point or the function is single-valued, \( V_0 = V_{+1} \), implying \( V_0 = 0 \).  

### 2. Proving \( M \) is Embedded via the Solid Path  
The solid path from \(-1\) to \(+1\) demonstrates that \( M \) is a submanifold without self-intersections or singularities. Local pictures at each stage on the solid path confirm the map is a homeomorphism onto its image, satisfying the definition of an embedding (the inclusion map is a homeomorphism with the subspace topology). The dashed competitor path (from \(1\) to \(2\)) may show self-intersections or singularities at the non-embedded point \( z_0 \), highlighting why the solid path is necessary for the embedding proof.  

### Conclusion  
By analyzing the path integrals and local topological properties, we conclude \( V_0 = 0 \) and \( M \) is embedded, with the solid path ensuring the embedding condition is satisfied while avoiding issues at the non-embedded point \( z_0 \).  

The answer is \(<|begin_of_box|>0<|end_of_box|>\).</answer>