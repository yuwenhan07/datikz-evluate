<think>Got it, let's try to figure out what the illustration is about. The question mentions numerical fluxes used to update an intermediate cell, with a schematic cell having blue lines and black dots. First, I need to recall what numerical fluxes are in the context of computational fluid dynamics or finite volume methods. Numerical fluxes are used to update cell averages by accounting for fluxes across cell interfaces. 

The cell schematic probably shows the cell boundaries (blue lines) and maybe interfaces between cells, with black dots as interface points or control volumes. Let's think about the structure: a cell might be divided into subcells or have interfaces where fluxes are calculated. The blue lines could be the cell's edges or the boundaries of subcells, and the black dots might be the interface locations between adjacent cells. 

In finite volume methods, when updating a cell, you calculate the fluxes across each interface (between this cell and its neighbors) and then update the cell's average. So the illustration might show the cell with its neighboring cells, the interfaces (blue lines), and the points where fluxes are computed (black dots). The blue lines could represent the cell's boundaries or the interfaces between the current cell and adjacent cells, while the black dots are the interface points where numerical fluxes are evaluated. 

So, putting it together: the image is a schematic of a computational cell (like in a finite volume grid) with cell boundaries (blue lines) and interface points (black dots) between the current cell and its neighbors. The numerical fluxes are calculated across these interfaces (between the black dots, which are the interface locations) to update the cell's state. The blue lines might outline the cell's volume or the subcells within the main cell, and the black dots mark the locations where fluxes are computed, connecting adjacent cells. 

I need to make sure the description connects the elements: blue lines as cell boundaries or interfaces, black dots as interface points, and the purpose of numerical fluxes (calculating flux across interfaces to update the cell). So the illustration shows how fluxes are computed across the cell interfaces (marked by black dots, maybe) using the blue lines as the boundaries, and then using those fluxes to update the intermediate cell's state.</think><answer>To illustrate the numerical fluxes used to update an intermediate cell, the schematic likely depicts a computational cell (common in finite volume methods) with **blue lines** representing cell boundaries or interfaces between adjacent cells, and **black dots** marking specific interface points (e.g., locations where fluxes are computed between neighboring cells).  

In this context, numerical fluxes are calculated across the interfaces defined by the black dots (connecting the current cell to its neighbors). These fluxes account for mass, momentum, or energy transfer across the cell boundaries (blue lines), and their sum is used to update the intermediate cell’s state (e.g., cell average properties). The blue lines may outline the cell’s volume or subcell divisions, while the black dots identify the critical points where flux interactions with neighboring cells occur.  

In summary, the illustration visually represents how fluxes across cell interfaces (marked by black dots) are computed using the cell’s boundaries (blue lines) to update the intermediate cell’s state.  
So, the final answer is <|begin_of_box|>The illustration shows a computational cell with blue lines (cell boundaries/interfaces) and black dots (interface points), where numerical fluxes are calculated across the interfaces (between black dots) to update the cell’s state.<|end_of_box|></answer>