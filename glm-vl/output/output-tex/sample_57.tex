<think>Got it, let's try to figure out the positions within k-supertiles for k = 0 and 3. First, I need to recall what a supertile is. From what I remember, in the context of tiling patterns like the one described (maybe related to the Sierpiński triangle or similar fractal tilings), a supertile is a larger triangle made up of smaller tiles. The k-supertile would be a triangle of size 2^k, maybe? Wait, the problem mentions equilateral triangles with smallest at the center and largest at the edges. So maybe each supertile is a larger triangle composed of smaller ones. 

For k = 0, the 0-supertile would probably be the smallest triangle, the basic tile. Then for k = 3, the 3-supertile would be a large triangle made up of 3^2 = 9 small triangles? Wait, no, if each supertile is a triangle with side length 2^k, then the number of small triangles in a k-supertile is 3*(2^k)^2? Wait, maybe I need to think in terms of the Sierpiński triangle structure. In the Sierpiński triangle, each level k has 3^k triangles, and the supertile for level k would be a triangle that contains 3^k small triangles. Wait, maybe the positions within a k-supertile are the positions relative to the center of the supertile. 

Alternatively, considering that each supertile is a larger triangle, the positions within it can be categorized by their coordinates or their position in the fractal structure. For k = 0, the 0-supertile is a single small triangle, so positions are just the single position. For k = 3, the 3-supertile is a large triangle made up of many small triangles, and positions within it would be the positions relative to the center of the 3-supertile, maybe divided into three sub-supertiles (like the Sierpiński gasket), each corresponding to a corner of the large triangle. 

Wait, the problem says "positions within k-supertiles for k = 0, 3". So maybe for each k, we need to describe the positions (like coordinates or labels) within the k-supertile. Let's start with k = 0. The 0-supertile is the smallest triangle, so there's only one position, which is the center of the smallest triangle. For k = 3, the 3-supertile is a large triangle. In the Sierpiński triangle structure, each level k supertile is divided into three smaller supertiles (each of size k-1), so the 3-supertile would be divided into three 2-supertiles, each of which is divided into three 1-supertiles (which are 0-supertiles, the small triangles). So positions within the 3-supertile would be the positions in each of these sub-supertiles. But maybe more specifically, if we assign coordinates based on the fractal structure, each position in the k-supertile can be identified by its position in the hierarchy. For example, in the Sierpiński triangle, each point can be represented by a sequence of choices (left, center, right) at each level, but maybe for the supertiles, the positions are categorized by their level within the k-supertile. 

Alternatively, considering that the tiling has equilateral triangles with the smallest at the center and largest at the edges, the k-supertile's structure would have a central region with smaller triangles and larger ones at the edges. For k = 0, the supertile is a single small triangle, so position is the center of that small triangle. For k = 3, the supertile is a large triangle, and within it, there are positions that are in the central region (which might be a 2-supertile) or in the corner regions (which are 2-supertiles as well). But I need to make sure I'm not mixing up the levels. Wait, another approach: in the context of the Sierpiński gasket, each triangle of side length L has area proportional to L², and the number of triangles at level k is 3^k. So a k-supertile would be a triangle of side length 2^k, containing 3*(2^k)² small triangles? Wait, no, the number of small triangles in a triangle of side length n (in terms of the number of small triangles per side) is n², but in the Sierpiński triangle, each level k has 3^k triangles, each of side length 2^k. Wait, maybe the positions within a k-supertile are the positions relative to the center of the supertile, and for k = 0, it's the single position, for k = 3, there are three positions corresponding to the three sub-supertiles (top, bottom-left, bottom-right) each of which is a 2-supertile, and each 2-supertile has its own positions, but maybe the question is asking for the positions categorized by their position in the k-supertile structure. 

Alternatively, if we think of the tiling as a fractal where each supertile is a larger triangle made up of smaller ones, then the positions within a k-supertile can be described by their "depth" in the fractal, i.e., how many levels down from the k-supertile they are. For k = 0, depth 0 (the supertile itself is the tile), for k = 3, positions can be at depth 0 (inside the 3-supertile), depth 1 (inside a 2-supertile within the 3-supertile), depth 2 (inside a 1-supertile), depth 3 (inside a 0-supertile). But the question is about positions within the k-supertiles, so maybe for each k, list the positions. For k = 0, the only position is the center of the 0-supertile (the small triangle). For k = 3, the 3-supertile has positions that are in its three sub-supertiles (each a 2-supertile), so there are three types of positions: one in the central 2-supertile and two in the corner 2-supertiles? Wait, no, in the Sierpiński triangle, each level k supertile has three sub-supertiles, each of level k-1. So a 3-supertile has three 2-supertiles, each 2-supertile has three 1-supertiles, each 1-supertile has three 0-supertiles. So positions within the 3-supertile would be categorized by their position in one of the three 2-supertiles. But maybe the question is expecting a more specific answer, like the number of positions or their arrangement. Alternatively, considering that the image has a tiling with equilateral triangles, the positions within a k-supertile could be the vertices or centers of the triangles at each level. For k = 0, the position is the center of the smallest triangle. For k = 3, the positions would include the centers of the triangles at the 3rd level, which are larger triangles composed of smaller ones. But I need to make sure I'm not overcomplicating. The key points are probably that for k = 0, the supertile is a single small triangle with one position (its center), and for k = 3, the supertile is a large triangle divided into 3^3 = 27 small triangles? Wait, no, 3^k would be 3^3 = 27, but if each supertile is a triangle with side length 2^k, then the number of small triangles (side length 1) is 3*(2^k)^2 / 3? Wait, maybe I should recall that in the Sierpiński triangle, the number of triangles at level n is 3^n, and each has side length 2^n. So a k-supertile (which is a triangle of side length 2^k) contains 3^k triangles of side length 2^(k-1), and so on. Therefore, positions within the k-supertile would be the positions of the vertices or centers of these smaller triangles. For k = 0, the supertile is a triangle of side length 1 (smallest), so position is its center. For k = 3, the supertile is a triangle of side length 8 (if side length 1 is the smallest), and within it, there are positions corresponding to the centers of the triangles of side length 2, 4, and 8, but specifically, the positions within the k-supertile would be categorized by their position relative to the fractal structure, like being in the central region or a corner region. But without more specific details from the image, it's hard to say exactly, but the general idea is that for each k, the k-supertile has positions that are part of its hierarchical structure, with k = 0 being the basic tile and k = 3 being a larger structure with multiple sub-positions.</think><answer>To determine positions within k-supertiles for \( k = 0 \) and \( k = 3 \) in the described tiling of equilateral triangles, we analyze the fractal structure:  

- **For \( k = 0 \)**: The 0-supertile is the smallest equilateral triangle (basic tile). It contains one position, which is the center of this smallest triangle.  
- **For \( k = 3 \)**: The 3-supertile is a large equilateral triangle composed of smaller supertiles. It is divided into three 2-supertiles (each a smaller triangle), and each 2-supertile is further divided into three 1-supertiles, and so on. Positions within the 3-supertile include the centers of its sub-supertiles (2-supertiles) and the vertices/centers of the smaller triangles they contain, organized hierarchically.  

In summary, the positions within the k-supertiles follow the fractal hierarchy, with \( k = 0 \) having a single position and \( k = 3 \) having multiple positions corresponding to its sub-supertiles and smaller triangles.  

The answer is \(<|begin_of_box|>0, 3<|end_of_box|>\).</answer>