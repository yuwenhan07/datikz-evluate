<think>Got it, let's try to understand this problem step by step. First, the domain is the unit disk, which is a 2D set \( D = \{x \in \mathbb{R}^2; ||x|| < 1\} \). The velocity field is \( u = [-y, x]^T \), and we need a coarse triangulation where the normal component of the velocity (un) is zero. Wait, the problem says "satisfying un=0", so maybe the velocity field's normal component is zero on the edges? But the velocity field given is \( u = [-y, x]^T \), which is a rotational field, like a rigid body rotation with angular velocity 1 (since the curl is 2, maybe). 

First, let's recall that for a velocity field, the normal component \( u_n \) is the component of the velocity vector perpendicular to the edge. If we have a triangulation, each edge is part of a triangle, and the normal direction for an edge would be the direction perpendicular to the edge's tangent. For a triangle with edge vector \( \mathbf{v} = \mathbf{p}_2 - \mathbf{p}_1 \), the normal vector would be \( \mathbf{n} = \frac{\mathbf{v}^\perp}{||\mathbf{v}||} \), where \( \mathbf{v}^\perp \) is the perpendicular vector. The normal component of the velocity is then \( u_n = u \cdot \mathbf{n} \). 

The problem states that \( u_n = 0 \), so for each edge in the triangulation, the velocity vector is tangent to the edge, meaning the velocity doesn't push the edge in or out along its normal. But even with this, when we transport the edges (maybe using the velocity field to advect the edges), the blue edge gets transported out of the domain to the green edge. 

Let's think about the velocity field \( u = [-y, x]^T \). This is a counterclockwise rotation with angular velocity 1. So, points inside the unit disk are rotating around the origin. If we have a triangulation, say a coarse one, maybe with a triangle that has an edge on the boundary of the disk. Wait, the domain is the unit disk, so the boundary is the unit circle. If an edge is on the boundary, its normal is radial (pointing outward from the disk). The velocity field at a point on the boundary (||x||=1) would have a velocity vector perpendicular to the radius. For a point on the unit circle, the radius is \( [x, y]^T \), so the velocity is \( [-y, x]^T \), which is perpendicular to the radius (since their dot product is -xy + xy = 0), so the velocity is indeed tangent to the boundary. Wait, but if the velocity is tangent to the boundary, then the normal component of the velocity on the boundary is zero, which matches the condition \( u_n = 0 \). 

But the problem says "the blue edge gets transported out of the domain to the green edge". Maybe the triangulation has an edge that's not on the boundary, but inside the disk. If we transport the vertices of the edge using the velocity field, since the velocity is rotational, points inside the disk are moving in circles around the origin. If the edge is a chord of the disk, say from a point inside to another, then transporting each vertex along the velocity field would move them along circular paths. If the edge is close to the boundary, maybe the transported vertices could move outside the disk, but wait, the velocity field is defined inside the disk, but if we consider the transport as a flow, points inside the disk remain inside? Wait, no, the velocity field is defined everywhere, but the domain is the unit disk. If we have a point on an edge inside the disk, its velocity is \( [-y, x]^T \), which is a rotation, so it should stay within the disk if it starts inside. Wait, maybe the triangulation has an edge that's a diameter, say from (-1,0) to (1,0), but wait, the disk is ||x|| < 1, so the boundary is ||x|| = 1, so the diameter endpoints are on the boundary. If we take an edge from (-0.5, 0) to (0.5, 0), that's inside the disk. Transporting each endpoint along the velocity field: at point (0.5, 0), the velocity is [-0, 0.5]^T = [0, 0.5]^T, so moving in the positive y-direction. At (-0.5, 0), the velocity is [0, -0.5]^T, moving in the negative y-direction. So the edge from (-0.5, 0) to (0.5, 0) would be transported such that its endpoints move up and down, but the edge itself would rotate. However, if the edge is a triangle edge, maybe the triangle has vertices on different radii, and when transported, the edge might cross the boundary? Wait, but the velocity is tangent to the boundary, so points on the boundary move along the boundary, but points inside move in circles. If the edge is entirely inside, maybe the transported edge remains inside, but the problem says it goes out to the green edge, which might be on the boundary. 

Alternatively, maybe the triangulation has a very coarse structure, like a single triangle inscribed in the disk. The edges of this triangle would be chords of the disk. When transporting each edge with the velocity field, since the velocity is rotational, the edges would rotate, but if the original edge is a chord, after transport, it might become a chord closer to the boundary or cross the boundary? Wait, but the velocity field is smooth inside the disk, so the transport should keep points inside. Hmm, maybe the issue is that the normal component condition is satisfied, but the velocity's tangential component causes the edge to rotate in such a way that one of its endpoints exits the domain, but since the domain is the disk, exiting would mean going outside, but the velocity is defined everywhere, so maybe the transport is considered as a flow, and the edge is moved along the flow lines. If the flow lines are circles, then a chord would be transported into a different chord, possibly intersecting the boundary. 

Wait, let's think about the flow of the velocity field. The velocity field \( u = [-y, x]^T \) is a potential field with potential function \( \phi(x, y) = -xy \), since the gradient of \( \phi \) is \( [-y, x]^T \). The flow lines are the level sets of \( \phi \), which are circles centered at the origin. So, any point (x, y) moves along a circle centered at the origin with radius sqrt(x² + y²), moving counterclockwise. Therefore, if we have a triangle with vertices on different circles, their transport would move each vertex along its respective circle, and the edges would rotate. If the original edge is a chord of a circle inside the disk, after transport, the edge would be a chord of a different circle, possibly intersecting the boundary if the original chord was close to the boundary. 

The problem mentions a "blue edge" transported to a "green edge". Maybe the blue edge is a chord near the center, and after transport, it becomes a chord near the boundary, hence exiting the domain (but the domain is the interior, so exiting would mean reaching the boundary, which is allowed if the edge is on the boundary). Wait, but the original domain is the interior, so edges on the boundary are part of the domain's boundary. If the transported edge reaches the boundary, it becomes a boundary edge, hence the green edge is on the boundary. 

In summary, even though the normal component of the velocity is zero (so the velocity is tangent to the edges), the tangential component causes the edges to rotate, and due to the rotational nature of the velocity field, the edge's orientation changes such that one of its endpoints moves towards the boundary, and the edge becomes a boundary edge (green edge), which was originally a blue edge inside. The key is the rotational flow causing the edge to rotate and exit the domain's interior to the boundary.</think><answer>The problem involves a unit disk \( D = \{x \in \mathbb{R}^2; ||x|| < 1\} \) with a velocity field \( u = [-y, x]^T \), which is a rotational field (counterclockwise rigid rotation). The condition \( u_n = 0 \) ensures the velocity is tangent to edges, meaning no normal transport across edges. However, the rotational nature of \( u \) causes edges to rotate under advection. For a coarse triangulation, an edge (e.g., a chord inside the disk) is transported along circular flow lines centered at the origin. As the edge rotates, its orientation changes such that one endpoint moves toward the unit circle (the domain's boundary), transforming the edge into a boundary edge (green edge) from an interior edge (blue edge). Despite the zero normal component, the tangential rotational transport causes the edge to exit the interior domain and align with the boundary.  

The answer is \(<|begin_of_box|>1<|end_of_box|>\).</answer>