<think>Got, let's try to tackle this problem about unitarity for the connection W₁(,). First, I need to recall what unitarity means in the context of quantum field theory or differential geometry, but since the question mentions a connection, maybe it's about gauge theories or fiber bundles. Unitarity usually relates to the preservation of probabilities, so the connection's unitarity would require that the transition amplitudes are unitary, meaning the matrix elements satisfy certain conditions like being unitary operators.

The image has equations with Taylor series expansions, variables t and T, maybe parameters or time variables. Let's think about the structure. If W₁ is a connection, maybe it's a 1-form with components, and unitarity would require that the connection's transformation properties under some group (like SU(N)) preserve the inner product, leading to conditions on the coefficients when expanded in a series.

Taylor series expansion usually involves terms like (t/T)^n or similar, so maybe expanding around t=0 or T=0. The variables t and T could be time parameters or scaling factors. The equations might show that when you expand the connection components, the coefficients must satisfy certain relations to maintain unitarity, like the sum of squares of some coefficients being 1, or orthogonality conditions.

Since the problem is about "Unitarity (2) for the connection W₁(,)", maybe the second part of unitarity (like the second condition) requires that the connection's components, when expanded in a series, have coefficients that form a unitary matrix, so their matrix elements satisfy |A_ij|² + |B_ij|² = 1 or similar, ensuring the connection's transformation is unitary.

Without the exact image, I need to make reasonable assumptions. Suppose the Taylor expansion is around t=0, so W₁ = W₁^0 + W₁^1 t + W₁^2 t² + ... , and unitarity requires that the coefficients W₁^k satisfy some orthogonality or normalization condition. For example, if the connection is part of a gauge field, the unitarity condition might be that the matrix formed by the coefficients is unitary, so (W^k)† (W^l) = δ^kl, leading to conditions on the coefficients when expanded.

Alternatively, in the context of string theory or conformal field theory, connections might have components related to vertex operators, and unitarity would require the vertex operators to form a unitary representation. But given the mention of Taylor series, maybe the expansion is in terms of a parameter related to t and T, and the unitarity condition imposes that the series coefficients must satisfy certain algebraic relations.

Another angle: unitarity of a connection (as a differential operator) might require that the connection's curvature is unitary, but I'm not sure. Alternatively, in the context of the WZW model (Wess-Zumino-Witten model), connections are associated with currents, and unitarity would relate to the conformal dimensions and spin, but again, the Taylor series might be part of the expansion of the current or connection components.

Given the variables t and T, maybe T is a scale parameter, and t is a small parameter for expansion. The unitarity condition could be that when you expand the connection around t=0, the coefficients must form a unitary matrix, so their matrix elements squared sum to 1 for each row/column. For example, if W₁ has components W₁^μ, then (W₁^μ)^* W₁^ν = δ^μν, which would be the unitarity condition, and this might be expressed in terms of the Taylor coefficients.

In summary, without the exact image, but based on the description, the key points are the Taylor series expansion of the connection W₁ with variables t and T, and the unitarity condition likely imposes that the coefficients of the expansion satisfy unitary matrix conditions (like orthonormality), ensuring the connection's transformation is unitary. The equations would show that the coefficients, when expanded, must form a unitary matrix, possibly with conditions like Σ |W₁^k|² = 1 or similar, depending on the specific structure.</think><answer>To address the unitarity condition for the connection \( W_1(, \) (interpreted as a connection with components depending on variables, likely \( t \) and \( T \))), we analyze the Taylor series expansion and unitary requirements.  

### Key Concepts and Reasoning:  
1. **Unitarity in Connections**: For a connection (e.g., in gauge theories or fiber bundles) to be unitary, its transformation properties must preserve the inner product structure, typically requiring the connection’s components to form a unitary matrix. This means the coefficients of the expansion must satisfy orthonormality conditions, such as \( \sum |W_1^k|² = 1 \) (or similar matrix conditions for multiple indices).  

2. **Taylor Series Expansion**: If the connection \( W_1 \) is expanded as a Taylor series in a parameter (e.g., \( t/T \)), the unitarity condition imposes constraints on the coefficients. For example, if \( W_1 = W_1^0 + W_1^1 t + W_1^2 t² + \dots \), the coefficients \( W_1^k \) must form a unitary matrix, ensuring \( (W_1^k)^\dagger W_1^l = \delta^{kl} \), where \( \dagger \) denotes the Hermitian conjugate.  

3. **Variables \( t \) and \( T \)**: These likely represent scaling or time parameters. The unitarity condition would require that the series expansion’s coefficients, when expressed in terms of \( t \) and \( T \), satisfy the unitary matrix condition regardless of the expansion parameter’s value, ensuring the connection’s transformation is consistently unitary.  

### Conclusion:  
The unitarity condition for \( W_1 \) imposes that the coefficients of its Taylor series expansion form a unitary matrix, ensuring the connection’s transformation preserves probabilities (or inner products). This is typically expressed as orthonormality of the expansion coefficients, e.g., \( \sum |W_1^k|² = 1 \), or matrix-level unitarity conditions.  

The answer is \(<|begin_of_box|>1<|end_of_box|>\).</answer>