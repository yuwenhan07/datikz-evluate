<think>Got it, let's try to figure out how to align a circle and a rectangle with the same starting y-axis. First, I need to recall what the starting y-axis means—probably the y-coordinate of the top or bottom edge, or maybe the center? Wait, the problem says "same starting y-axis," so maybe both the circle and rectangle have their top or bottom edge aligned to the same y-coordinate. 

First, let's think about the elements: a circle and a rectangle. Let's assume the rectangle has a height and width, and the circle has a radius. To align them with the same starting y-axis, maybe their top edges are at the same y-coordinate, or their bottom edges, or their centers. Let's say we want their top edges to start at the same y-axis. So if the rectangle's top edge is at y = h, the circle's top edge should also be at y = h. 

Let's define coordinates. Suppose the rectangle has a top-left corner at (x1, y1) and bottom-right at (x2, y2), so its height is y2 - y1. The circle has center (cx, cy) and radius r, so its top edge is at cy - r, bottom at cy + r. To align the starting y-axis (maybe top edge) of both, set cy - r = y1 (if rectangle's top is y1) or cy + r = y2 (bottom), or maybe their centers have the same y-coordinate. Wait, the problem says "same starting y-axis"—maybe the y-coordinate of the starting point (like the top left corner of the rectangle and the top left of the circle, but the circle is a circle, so maybe the topmost point of the circle and the top edge of the rectangle are aligned). 

Alternatively, maybe the starting y-axis refers to the y-coordinate of the first element, so both have their top edge at the same y. Let's say the rectangle's top edge is at y = 0 (for simplicity), then the circle's top edge should also be at y = 0. The circle's top edge is center y minus radius, so center y = radius. The rectangle's height is, say, h, so its bottom edge is at y = h. The circle's bottom edge is center y + radius = radius + radius = 2r. So if the rectangle's height is h, then 2r should be less than or equal to h? Wait, maybe the rectangle is taller than the circle, so the circle fits inside the rectangle vertically. 

Wait, the image description is a white screen with black background, text "1 Node 2 Node 3 Node"—maybe the nodes are represented as shapes, like the rectangle and circle are nodes. To align them with same starting y-axis, maybe their top edges are aligned to the same y-coordinate. So if the rectangle's top edge is at y = 0, the circle's top edge is also at y = 0, meaning the circle's center is at y = r (since top edge is center minus radius). The rectangle's bottom edge is at y = h, so the circle's bottom edge is center y + radius = r + r = 2r, so 2r ≤ h, meaning the circle's diameter is less than or equal to the rectangle's height. 

Alternatively, if the starting y-axis is the bottom edge, then rectangle's bottom edge is y = 0, circle's bottom edge is center y + radius = 0, so center y = -r, top edge is center y - radius = -2r. But that might not make sense if the rectangle is above. 

Wait, maybe the key is to set the y-coordinate of the top of the rectangle equal to the y-coordinate of the top of the circle. So if the rectangle has a top y-coordinate of y, the circle's top y-coordinate (which is center y minus radius) is also y. So center y = y + radius. The rectangle's height is, say, H, so its bottom y-coordinate is y + H. The circle's bottom y-coordinate is center y + radius = y + radius + radius = y + 2r. So to have the circle fit inside the rectangle vertically, y + 2r ≤ y + H → 2r ≤ H, so the rectangle's height must be at least twice the circle's radius. 

So the steps would be: 1. Determine the desired y-coordinate for the starting edge (top or bottom). 2. For the rectangle, set its top or bottom edge to that y-coordinate. 3. For the circle, set its center's y-coordinate such that its top or bottom edge matches the rectangle's edge. 

If the starting y-axis is the top edge, then rectangle's top edge = circle's top edge. Circle's top edge = center y - radius. Rectangle's top edge = rectangle's top y. So center y = rectangle's top y + radius. If the rectangle's top y is, say, 0, then center y = radius. If the rectangle's height is H, then its bottom y is H, so circle's bottom edge is center y + radius = radius + radius = 2r, so 2r ≤ H. 

Alternatively, if the starting y-axis is the bottom edge, rectangle's bottom edge = circle's bottom edge. Circle's bottom edge = center y + radius. Rectangle's bottom edge = rectangle's bottom y. So center y = rectangle's bottom y - radius. If rectangle's bottom y is 0, center y = -radius, top edge is center y - radius = -2r. But maybe the rectangle is centered vertically, so its center y is H/2, then circle's center y should be H/2, so top edge of circle is H/2 - r, bottom is H/2 + r. If we want the top edge of the circle to align with the rectangle's top edge, then H/2 - r = rectangle's top edge. But this is getting a bit confusing. 

Wait, the image has text "1 Node 2 Node 3 Node"—maybe the nodes are arranged vertically, each with the same starting y-axis. So each node (circle and rectangle) has their top edge at the same y-coordinate. So if the first node (maybe the rectangle) has top edge at y = 0, the second (circle) also has top edge at y = 0. Then the rectangle's height is, say, 50 pixels, so bottom edge at 50. The circle's radius is 25 pixels, so center at y = 25, top edge 25 - 25 = 0, bottom edge 25 + 25 = 50. That way, both have top edge at 0, bottom at 50, same y-axis. 

So the alignment would involve setting the top edge of both shapes to the same y-coordinate, which requires adjusting the center of the circle (since it's a circle) so that its top edge matches the rectangle's top edge. For a rectangle, the top edge is the top y-coordinate of its top-left corner. For a circle, the top edge is the center y minus radius. So if the rectangle's top y is y0, then circle's center y = y0 + radius. 

In summary, to align a circle and a rectangle with the same starting y-axis (top edge, for example), set the top edge of the rectangle equal to the top edge of the circle. The top edge of the circle is center y minus radius, so center y = top edge + radius. The rectangle's top edge is its top y-coordinate, so center y = rectangle's top y + radius. This ensures both have the same starting y-axis (top edge) and their vertical positions are aligned.</think><answer>To align a circle and a rectangle with the same starting y-axis (e.g., their top edges), follow these steps:  

1. **Define the starting y-coordinate**: Choose a common y-coordinate (e.g., the top edge of both shapes).  
2. **Adjust the rectangle’s position**: Set the rectangle’s top edge to the chosen y-coordinate.  
3. **Adjust the circle’s center**: For the circle, its top edge is its center’s y-coordinate minus its radius. To match the rectangle’s top edge, set the circle’s center y-coordinate to \( \text{chosen y-coordinate} + \text{radius} \).  

For example, if the rectangle’s top edge is at \( y = 0 \) and the circle has a radius \( r \), the circle’s center y-coordinate becomes \( 0 + r = r \), making its top edge \( r - r = 0 \), matching the rectangle’s top edge. This ensures both shapes share the same starting y-axis.  

**Image Credit**: *A white screen with a black background displaying the text "1 Node 2 Node 3 Node" (representing aligned nodes with the same starting y-axis)*.  
So, the final answer is <|begin_of_box|>Aligning a circle and rectangle with the same starting y-axis involves setting their top edges to the same y-coordinate, achieved by positioning the circle’s center \( r \) units above the rectangle’s top edge (where \( r \) is the circle’s radius).<|end_of_box|></answer>