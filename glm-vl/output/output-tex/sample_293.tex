<think>Got it, let's try to tackle this problem step by step. First, let's recall some definitions to make sure I understand everything correctly. 

First, K₁,₃ is a complete tripartite graph with one vertex connected to three others, right? So it's a star graph with four vertices. Then, replacing an edge of K₁,₃ by a P₄ (which is a path of four vertices, so a path with three edges connecting four vertices) would result in a tree T. Wait, let's visualize K₁,₃. Let's say the vertices are v, u₁, u₂, u₃, where v is connected to u₁, u₂, u₃. If we replace one of the edges, say the edge between v and u₁, with a P₄. A P₄ is a path with four vertices, so maybe we replace the edge (v, u₁) with a path that has vertices like w, x, y, u₁, where v is connected to w, w connected to x, x connected to y, y connected to u₁. Wait, but then the resulting graph should be a tree. Let's check: K₁,₃ has 4 vertices, replacing one edge (which is an edge between the center and a leaf) with a P₄ would add three new vertices (since a P₄ has four vertices, so replacing one edge (two vertices) with four vertices, so total vertices become 4 - 2 + 4 = 6? Wait, wait, K₁,₃ has 4 vertices: 1 + 3 = 4. If we replace an edge, say between the center (v) and a leaf (u₁), then remove the edge (v, u₁), and add a P₄ which is a path of four vertices, say w, x, y, u₁. So now the vertices are v, w, x, y, u₂, u₃, u₁. Wait, but u₁ is part of the P₄, so the edges would be v-w, w-x, x-y, y-u₁, and the original edges u₂ connected to v, u₃ connected to v. So the graph now has vertices v, u₂, u₃, w, x, y, u₁, and edges: v-u₂, v-u₃, v-w, w-x, x-y, y-u₁. Is this a tree? Let's count the edges: original K₁,₃ had 3 edges (v-u₁, v-u₂, v-u₃). We removed one edge (v-u₁), added four edges (v-w, w-x, x-y, y-u₁), so total edges are 3 - 1 + 4 = 6. The number of vertices is 4 (original) - 1 (removed u₁ connection) + 4 (new P₄ vertices) = 4 - 1 + 4 = 7? Wait, original K₁,₃ has 4 vertices: v, u₁, u₂, u₃. Replacing edge (v, u₁) with P₄: the P₄ is a path of four vertices, so let's say the P₄ is a, b, c, u₁. So the vertices become v, u₂, u₃, a, b, c, u₁. Wait, u₁ is already there, so replacing edge (v, u₁) with a-b-c-u₁, so edges are v-a, a-b, b-c, c-u₁, and the original edges v-u₂, v-u₃. So total vertices: 7, edges: 4 (new) + 2 (original) = 6. A tree with 7 vertices has 6 edges, so that works. So T is a tree with 7 vertices.

Now, the problem states that T is also a TS₂-reconfiguration graph, and each node ab in T represents a size-2 stable set of G. Wait, G is probably the original graph, which is K₁,₃. A size-2 stable set in K₁,₃: a stable set is a set of vertices with no edges between them. In K₁,₃, the stable sets can be of size 0, 1, or 2. A size-2 stable set would be two vertices with no edge between them. In K₁,₃, the center vertex v is connected to all others, so the leaves u₁, u₂, u₃ are not connected to each other. So a size-2 stable set could be any two leaves, like {u₁, u₂}, {u₁, u₃}, {u₂, u₃}, or maybe including the center? Wait, the center v is connected to u₁, u₂, u₃, so a stable set can't include v and any leaf, because they are connected. So the size-2 stable sets in K₁,₃ are the pairs of leaves, i.e., each pair of the three leaves forms a size-2 stable set. So there are C(3,2) = 3 such sets.

Now, the tree T is a TS₂-reconfiguration graph. TS₂-reconfiguration graph usually refers to a graph where each node corresponds to a stable set of a certain size, and edges represent reconfigurations (i.e., changing one element of the stable set while keeping the rest, or some operation). But the problem says each node ab in T represents a size-2 stable set of G. Wait, maybe "ab" is a pair of vertices, so each node is a pair, and the tree structure allows moving between different pairs via reconfigurations.

Wait, let's recall that a TS_k-reconfiguration graph is a graph where each node is a k-stable set, and there is an edge between two nodes if one can be obtained from the other by a single reconfiguration step (i.e., adding or removing one vertex from the stable set, maintaining the stability). So for TS₂, each node is a 2-stable set, and edges connect sets that differ by one vertex (i.e., if S and S' differ by one vertex, then there's an edge between them).

Now, the original graph G is K₁,₃, which has 3 size-2 stable sets: {u₁, u₂}, {u₁, u₃}, {u₂, u₃}. When we replace an edge of K₁,₃ with a P₄, we get tree T. The tree T is supposed to be a TS₂-reconfiguration graph, meaning that the tree structure encodes the possible transitions between different size-2 stable sets via reconfigurations.

Let's think about the structure of T. If T is a tree with nodes representing size-2 stable sets, then the edges would represent the reconfigurations. For example, if we have a node representing {u₁, u₂}, maybe an edge leads to {u₁, u₃} or {u₂, u₃}, depending on the tree structure. But since T is obtained by replacing an edge of K₁,₃ with a P₄, maybe the tree T has a structure that allows moving from one stable set to another by traversing the tree.

Alternatively, let's consider the tree T as having a root or some structure where each node ab corresponds to a pair, and the edges correspond to adding or removing a vertex from the pair while maintaining stability. For example, if we have a node {u₁, u₂}, maybe an edge leads to {u₁, u₃} if there's a way to replace one vertex in the pair with another, but in the original graph, the only stable sets are the pairs of leaves, so maybe the tree T allows for intermediate steps when the graph is modified.

Wait, another approach: K₁,₃ has a center vertex connected to three leaves. When we replace an edge between the center and a leaf with a P₄, say the edge between center v and leaf u₁ is replaced by a path v-w-x-y-u₁. Now, in the new tree T, the vertices are v, w, x, y, u₂, u₃, u₁. The size-2 stable sets in T would need to be considered. But wait, T is a tree, so it's a connected acyclic graph. A stable set in T is a set of vertices with no edges between them. In T, the edges are v-w, w-x, x-y, y-u₁, v-u₂, v-u₃. So the stable sets in T would include pairs like {u₂, u₃} (since they are not connected), {w, x}, {w, y}, {x, y}, {u₁, u₂}, {u₁, u₃}, etc. Wait, but the problem states that each node ab in T represents a size-2 stable set of G (the original K₁,₃). So maybe the tree T is constructed in such a way that its nodes correspond to the size-2 stable sets of G, and the edges represent the reconfigurations between them.

Given that T is a TS₂-reconfiguration graph, it should have the property that between any two size-2 stable sets, there's a path in T corresponding to a sequence of reconfigurations. Since the original graph has three size-2 stable sets, and T is a tree, which is connected, it should have a structure that allows moving from any stable set to another through the tree edges.

Putting this all together, the key points are: replacing an edge of K₁,₃ (which is a star graph) with a P₄ (a path) results in a tree T. This tree T is a TS₂-reconfiguration graph, meaning it's a graph where each node is a size-2 stable set of G, and edges represent possible reconfigurations (changing the stable set by one vertex). Each node ab (a pair of vertices) in T corresponds to such a stable set, and the tree structure allows for the reconfiguration process between different stable sets.</think><answer>To address the problem, we start by analyzing the graph transformations and definitions:  

1. **Graph Definitions**:  
   - \( K_{1,3} \) is a complete tripartite graph with one central vertex connected to three leaves, forming a star graph with 4 vertices.  
   - Replacing an edge of \( K_{1,3} \) (e.g., between the center and a leaf) with a \( P_4 \) (a path of 4 vertices) results in a tree \( T \). This replacement adds three new vertices (for the \( P_4 \)) and maintains acyclicity, resulting in a tree with 7 vertices and 6 edges.  

2. **TS₂-Reconfiguration Graph**:  
   A TS₂-reconfiguration graph has nodes representing size-2 stable sets (sets of two vertices with no edges between them) and edges representing reconfigurations (changing one vertex in the set while preserving stability). For \( K_{1,3} \), the size-2 stable sets are pairs of leaves (e.g., \( \{u_1, u_2\} \), \( \{u_1, u_3\} \), \( \{u_2, u_3\} \)), totaling 3 such sets.  

3. **Tree Structure and Stable Sets**:  
   The tree \( T \) encodes these stable sets as nodes, with edges representing valid reconfigurations (e.g., transitioning between stable sets by modifying one vertex). Each node \( ab \) in \( T \) corresponds to a size-2 stable set of \( G \), and the tree’s acyclic, connected structure ensures all stable sets are reachable via reconfigurations.  

Thus, the resulting tree \( T \) serves as a TS₂-reconfiguration graph, where each node represents a size-2 stable set of \( G \), and edges denote valid reconfiguration steps.  

The answer is \(<|begin_of_box|>T<|end_of_box|>\).</answer>