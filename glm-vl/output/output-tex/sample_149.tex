<think>Got it, let's try to understand this step by step. First, the question is about de Sitter space, which has time-translation invariance. De Sitter space is a maximally symmetric space with a positive cosmological constant, so it's like a hyperboloid in Minkowski space with a metric signature (-, +, +, +) or something similar, depending on the convention. 

The poles of the spatial sphere—so in de Sitter space, the spatial slices are spheres, and the poles would be points on those spheres. Let's think of de Sitter space as a hyperboloid embedded in a higher-dimensional Minkowski space. The time-translation invariance means that the metric is invariant under time shifts, so we can use this symmetry to map two points on the poles to a global equal time slice. 

The points are on the poles, so maybe they are antipodal points on the spatial sphere? Wait, the poles would be the north and south poles, so two points on the poles could be, say, the north pole and the south pole of a spatial sphere. Then, using time-translation invariance, we can shift the time coordinates such that one point has time t_x = -t and the other has t_y = t, so that they are on the same "global" time slice? Wait, but the statement says "map two points on the poles of the spatial sphere to a global equal time slice with t_x = -t and t_y = t". So maybe the original coordinates of the two points have different times, but by applying a time translation, we can make them have times -t and t, so that they are on a slice where the time coordinate is some value, but maybe the key is that in de Sitter space, the geodesics between two points depend on their time separation. 

The statement says "There are no real geodesics between these two points unless t=0." So if we have two points with times t_x = -t and t_y = t, then the time difference between them is 2t. For there to be a geodesic connecting them, the time separation must satisfy some condition. In de Sitter space, the existence of a geodesic between two points depends on whether they can be connected by a timelike or null geodesic. If the time separation is such that the spatial distance is less than the maximum possible, then a geodesic exists. But if the time difference is too large, maybe the spatial separation becomes too big, and there's no geodesic. When t=0, then both points have the same time (t_x = 0, t_y = 0), so they are on the same time slice, and then there should be a geodesic connecting them, maybe a spatial geodesic on the sphere. But when t ≠ 0, the time difference is non-zero, and there's no real geodesic. 

Let me recall that in de Sitter space, the metric can be written in static coordinates as ds² = -dt² + e^(2t) (dx² + dy² + dz²), but maybe another coordinate system is better. Alternatively, using the Poincaré coordinates or the global coordinates. But the key idea is that time-translation invariance allows us to shift the time of one point by -2t to align with the other, but the geodesic connection requires that the points are within the causal diamond or something. If the two points are on the poles of the spatial sphere, their spatial separation is maximum (antipodal points), so the spatial distance is the diameter of the sphere. For a geodesic to connect them, the time separation must be such that the "light cone" allows it. But if they are on antipodal points with a time difference of 2t, then when t ≠ 0, the spatial distance is e^(2t) times the diameter, which might be larger than the maximum possible for a geodesic, hence no real geodesic. When t=0, the spatial distance is 1 (if we normalize), and then a geodesic exists. 

So putting it all together: using the time-translation symmetry, we can map the two points to a global time slice where their times are -t and t. The condition for a geodesic between them is that the time separation must satisfy certain conditions, and when t ≠ 0, the spatial separation becomes too large, so no real geodesic exists, but when t=0, they are on the same time slice, and a geodesic exists.</think><answer>To analyze the problem, we start by considering de Sitter space, which exhibits time-translation invariance. This symmetry allows us to use time shifts to map two points on the poles of a spatial sphere (antipodal points on a spherical spatial slice) to a global equal-time slice with coordinates \( t_x = -t \) and \( t_y = t \).  

In de Sitter space, the existence of a geodesic between two points depends on their temporal and spatial separation. For the antipodal points with time separation \( 2t \), the spatial distance between them is proportional to \( e^{2t} \) (in appropriate coordinates). When \( t \neq 0 \), this spatial separation becomes too large for a geodesic to connect the points, as the temporal separation prevents a causal connection. However, when \( t = 0 \), both points lie on the same time slice, and a geodesic (e.g., a spatial geodesic on the sphere) exists between them.  

Thus, the conclusion is that real geodesics exist only when the time separation \( t = 0 \), as non-zero time separation makes the spatial distance too large for a geodesic to connect the points.  

The answer is \(<|begin_of_box|>t = 0<|end_of_box|>\).</answer>