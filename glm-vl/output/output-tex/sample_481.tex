<think>Got, let's tackle this problem. First, the algebraic expression given is \( x^2 + 2xy + y^2 \). I need to figure out its expression graph, which is like a tree structure representing the operations and terms. 

First, let's recall that \( x^2 + 2xy + y^2 \) is a perfect square trinomial, right? It can be factored as \( (x + y)^2 \), but maybe the expression graph is about breaking down the original expression into its components. 

Let's start by identifying the main operations. The expression has three terms: \( x^2 \), \( 2xy \), and \( y^2 \), connected by addition. So the root of the tree would probably be the addition operation, with each term as its children. 

First term: \( x^2 \). That's a squaring operation, so maybe a node for "square" with variable x. Second term: \( 2xy \). That's a product of 2, x, and y. So maybe a product node with 2, x, y. Third term: \( y^2 \), similar to the first term, a square of y. 

So the tree structure would have the root as "addition" (or "+" symbol), with three children: the first child is "square" with x, the second is "product" with 2, x, y, and the third is "square" with y. 

Wait, let's think step by step. The original expression is \( x^2 + 2xy + y^2 \). Breaking it down:

- The first term is \( x \) squared, so that's a node for squaring x.
- The second term is 2 times x times y, so that's a product of 2, x, y.
- The third term is y squared, another squaring node.
- Then all three terms are added together. So the root is addition, with three children: the first child is the square of x, the second is the product of 2, x, y, the third is the square of y.

Alternatively, if we consider the factored form \( (x + y)^2 \), the tree would be a square node with a sum of x and y inside. But the question mentions the expression graph for the given expression, which is \( x^2 + 2xy + y^2 \), not the factored form. So probably the first structure: addition of three terms, each being a square or a product.

Let me confirm the structure. The expression is a sum of three terms: \( x^2 \), \( 2xy \), \( y^2 \). Each term is a product or a power. \( x^2 \) is x multiplied by x (or power), \( 2xy \) is 2 times x times y (product), \( y^2 \) is y times y (power). So the tree would have:

- Root: +
- Left child: x^2 (which is x * x, or a square node)
- Middle child: 2xy (which is 2 * x * y, product node with three operands)
- Right child: y^2 (y * y, square node)

Alternatively, if we consider the algebraic structure, the expression graph might represent the operations in a hierarchical way. The addition is the top-level operation, with each term as a sub-expression. Each term is either a monomial (like x^2, which is a product of x with itself) or a product of constants and variables (like 2xy, which is 2 * x * y). 

Another way: the expression can be seen as a sum of three monomials. The first monomial is x squared (degree 2), the second is 2xy (degree 2), the third is y squared (degree 2). So the tree would have the addition node at the top, with three monomial nodes as children. Each monomial node is either a power (for x^2 and y^2) or a product (for 2xy). 

For the power nodes (x^2 and y^2), the structure would be a "power" operation with exponent 2 and base x or y. For the product node (2xy), it's a "product" operation with operands 2, x, y. 

So putting it all together, the expression graph tree would have:

- Root: Addition (or + symbol)
  - Left child: Power (with exponent 2, base x)
  - Middle child: Product (with operands 2, x, y)
  - Right child: Power (with exponent 2, base y)

This structure shows the hierarchical organization of the algebraic expression, breaking down the main operation (addition) into its constituent terms, each of which is a product or power, and further breaking down those into their respective operations and operands. The image conveys order and organization by clearly separating each operation and variable, making the structure easy to follow.</think><answer>To determine the expression graph for the algebraic expression \( x^2 + 2xy + y^2 \), we analyze its structure by breaking down operations and terms into a hierarchical tree.  

1. **Identify the main operation**: The expression is a sum of three terms, so the root node is the addition operation (\(+\)).  
2. **Break down each term**:  
   - The first term \( x^2 \) is a power operation (squaring \( x \)), represented as a "power" node with base \( x \) and exponent 2.  
   - The second term \( 2xy \) is a product of three operands (2, \( x \), \( y \)), represented as a "product" node.  
   - The third term \( y^2 \) is another power operation (squaring \( y \)), represented as a "power" node with base \( y \) and exponent 2.  
3. **Construct the tree**: The root (\(+\)) has three children: the first child is the power node for \( x^2 \), the second is the product node for \( 2xy \), and the third is the power node for \( y^2 \).  

This tree structure clearly organizes the expression by showing the hierarchical relationship between addition, power, and product operations, with variables and constants as operands.  

The answer is \(<|begin_of_box|>(x + y)^2 (or x^2 + 2xy + y^2)<|end_of_box|>\).</answer>