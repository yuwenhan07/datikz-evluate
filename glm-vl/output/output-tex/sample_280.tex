<think>Got it, let's tackle this question step by step. First, I need to recall what gauge invariant Wilson operators are and the difference between long-range and local gauge symmetries. 

Starting with part (a): Gauge invariant Wilson operators in theories with long-range gauge symmetry. Long-range gauge symmetries are those that don't have a local gauge fixing, meaning the gauge symmetry is global or topological, like in abelian theories with magnetic monopoles or non-abelian theories with instantons. Wilson operators are typically constructed as products of gauge links around a loop, and gauge invariance means they transform trivially under gauge transformations. For long-range symmetries, maybe the Wilson loops that are topological, like those wrapping non-contractible loops in a compact space, or those associated with magnetic charges. Also, in theories with long-range symmetries, the Wilson operators might be related to the topological sectors, so operators like Wilson loops around non-contractible cycles, or operators that are invariant under the global symmetry. For example, in a theory with a U(1) gauge symmetry and magnetic monopoles, the Wilson loop around a monopole loop would be a gauge invariant operator, and since the monopole is a topological charge, the loop operator is invariant under gauge transformations that don't change the monopole charge. So such operators exist because the gauge symmetry is not local, so local gauge transformations can't change the Wilson loop's value, making them gauge invariant.

Now part (b): Additional gauge invariant Wilson operators in theories with local charged operators. Local charged operators are those that transform non-trivially under local gauge transformations, like fermions or charged scalars. In such theories, besides the Wilson loops from part (a), there might be Wilson operators that involve these local charged operators. For example, in a gauge theory with charged fermions, you can have Wilson lines (links) with fermions inserted, or Wilson loops with insertions of charged operators. But wait, Wilson operators are usually loops, but maybe more generally, operators that are products of gauge links and local charged operators. However, gauge invariance requires that the operator is invariant under gauge transformations. If you have a Wilson loop with a charged operator inserted, does that remain invariant? Wait, inserting a charged operator into a Wilson loop might break gauge invariance unless the operator is gauge invariant, but local charged operators are not gauge invariant. Hmm, maybe the additional operators are those that are constructed by combining Wilson loops with the local charged operators in a way that the gauge invariance is maintained. Alternatively, in theories with local charged operators, you can have Wilson operators that are associated with the charged operators' gauge charges. For example, in a theory with a charged scalar field, you might have Wilson loops with insertions of the scalar field, but to maintain gauge invariance, the scalar must be in a gauge invariant combination, like a topological charge or something. Wait, maybe the key is that in theories with local charged operators, you can have Wilson operators that are not just topological loops but also involve the local charges, such as operators that are products of Wilson loops and the local charged operators, but only if the combination is gauge invariant. Alternatively, think about the example of QED: the photon is a gauge field, and charged particles are the fermions. The Wilson loop is gauge invariant, but if you have a Wilson loop with a charged fermion inserted, like a fermion loop around the Wilson loop, that might be a gauge invariant operator. But I need to recall that in gauge theories with local charged operators, the gauge invariance of Wilson operators can be extended to include operators that involve the charged matter fields in a gauge invariant way. For example, the 't Hooft operator, which is a product of Wilson loops around a set of non-contractible cycles, can be combined with matter fields, but the matter fields must be in a gauge invariant state. Alternatively, the additional operators might be those that are constructed by taking the product of Wilson loops with the local charged operators, but only when the gauge symmetry allows for such combinations. Another angle: gauge invariant Wilson operators in general are those that are invariant under gauge transformations, which for local symmetries would require that the operator is a topological invariant (like a Wilson loop around a non-contractible loop) or a product of such loops. But when there are local charged operators, you can have operators that are combinations of Wilson loops and the charged operators, but only if the charged operators are in a gauge invariant way. Wait, maybe the additional operators are those that are not just topological loops but also involve the local charges, such as operators that are products of Wilson loops and the local charged operators, but only when the gauge symmetry allows for such operators. For example, in a theory with a U(1) gauge symmetry and charged fermions, the gauge invariant Wilson operators would include the Wilson loops (for the topological part) and also operators like the product of Wilson loops and the fermion number operator, but the fermion number operator is gauge invariant? Wait, the fermion number operator is gauge invariant if the theory is gauge invariant, but inserting a charged operator into a Wilson loop might not be gauge invariant. Hmm, maybe I need to think of the difference between long-range (global/topological) and local gauge symmetries. In long-range symmetries, the gauge transformations are not local, so Wilson loops that are non-contractible (in a compact space) are gauge invariant because local gauge transformations can't change the loop's topology. In local symmetries, Wilson loops can be made gauge invariant by being non-contractible, but also, with local charged operators, you can have operators that are products of Wilson loops and the charged operators, but only if the charged operators are in a gauge invariant combination. Alternatively, the additional operators in part (b) might be those that are not just Wilson loops but also include the local charged operators in a way that the gauge invariance is maintained, such as operators that are constructed by taking the product of Wilson loops with the local charged operators, but only when the gauge symmetry allows for such operators. For example, in a gauge theory with a charged scalar field, the gauge invariant Wilson operators would include the Wilson loops (as in part a) and also operators like the product of Wilson loops and the scalar field, but only if the scalar field is in a gauge invariant state. Wait, maybe the key is that in theories with local charged operators, you can have Wilson operators that are associated with the charged operators' gauge charges, such as operators that are products of Wilson loops and the local charged operators, but only when the gauge symmetry allows for such operators. Alternatively, think of the example of the 't Hooft operator, which is a product of Wilson loops, and in theories with local charged operators, you can have operators that are combinations of 't Hooft operators and the charged matter fields, but those would be gauge invariant if the matter fields are in gauge invariant states. But I need to structure this clearly.

For part (a), gauge invariant Wilson operators in long-range gauge theories: these are typically Wilson loops that are topological, meaning they wrap non-contractible cycles in a compact space, or are associated with topological charges (like monopoles, instantons) whose gauge charges are conserved. Since the gauge symmetry is long-range (not local), local gauge transformations can't change the topological charge, so the Wilson loop operator is gauge invariant. Examples include Wilson loops around monopole loops, instanton operators, or Wilson loops in a compact manifold with non-trivial topology. These operators exist because the gauge symmetry is not local, so the Wilson loop's value is invariant under gauge transformations that don't change the topological sector.

For part (b), additional gauge invariant Wilson operators in theories with local charged operators: these would involve the local charged matter fields in a gauge invariant way. For example, in a gauge theory with charged fermions, you can have Wilson operators that are products of Wilson loops and the fermion number operator, or operators that insert charged fermions into Wilson loops. However, to maintain gauge invariance, the operators must be constructed such that the gauge transformations cancel out. Another example is the 't Hooft operator combined with matter fields, but more generally, operators that are formed by taking the product of Wilson loops (from part a) with the local charged operators, but only when the gauge symmetry allows for such combinations. Alternatively, in theories with local charged operators, you can have Wilson operators that are associated with the charged operators' gauge charges, such as operators that are products of Wilson loops and the local charged operators, but only when the gauge symmetry is such that the combination is invariant. For instance, in QED, the photon is the gauge field, and electrons are charged. The gauge invariant Wilson operators include the photon's Wilson loops (as in part a), and additionally, operators like the product of Wilson loops with electron number operators, but the electron number operator is gauge invariant, so such combinations would be gauge invariant. However, more precisely, the additional operators might be those that are not just topological loops but also involve the local charges, such as operators that are constructed by taking the product of Wilson loops with the local charged operators, but only when the gauge symmetry allows for such operators. Another angle: in theories with local charged operators, the gauge invariance of Wilson operators can be extended to include operators that are associated with the charged operators' gauge charges, such as operators that are products of Wilson loops and the local charged operators, but only when the gauge symmetry is such that the combination is invariant. For example, in a theory with a U(1) gauge symmetry and a charged scalar field φ, the gauge invariant Wilson operators would include the Wilson loops (as in part a) and also operators like φ multiplied by Wilson loops, but φ must be in a gauge invariant state, like a topological charge. Wait, maybe the additional operators are those that are constructed by combining Wilson loops with the local charged operators in a way that the gauge invariance is maintained, such as operators that are products of Wilson loops and the local charged operators, but only when the gauge symmetry allows for such operators. Alternatively, think of the example of the 't Hooft operator, which is a product of Wilson loops, and in theories with local charged operators, you can have operators that are combinations of 't Hooft operators and the charged matter fields, but those would be gauge invariant if the matter fields are in gauge invariant states. However, the key difference is that in part (a), the operators exist due to long-range symmetries (no local gauge fixing), while in part (b), the additional operators exist because of the presence of local charged operators, which allow for more complex gauge invariant combinations involving both the gauge fields and the matter fields.</think><answer>To address the question, we analyze gauge invariant Wilson operators in theories with long-range and local gauge symmetries:  


### (a) Gauge invariant Wilson operators in theories with long-range gauge symmetry  
Long-range gauge symmetries (e.g., global, topological, or non-local) do not allow local gauge transformations. Wilson operators, constructed as products of gauge links around loops, are gauge invariant if they are associated with topological or global properties. Key examples include:  
- **Wilson loops around non-contractible cycles** in a compact space (e.g., monopole loops, instanton operators). These loops are gauge invariant because local gauge transformations cannot change the topological charge (e.g., magnetic monopole charge), and the loop’s value is preserved under non-local gauge transformations.  
- **Topological Wilson operators** tied to global symmetries (e.g., in abelian theories with magnetic monopoles, or non-abelian theories with instantons), which are invariant under the long-range gauge symmetry.  

These operators exist because long-range symmetries lack local gauge fixing, so Wilson loops’ values are unchanged by gauge transformations, ensuring gauge invariance.  

### (b) Additional gauge invariant Wilson operators in theories with local charged operators  
In theories with local charged operators (e.g., fermions, charged scalars), gauge invariance allows for more complex combinations involving both gauge fields and matter fields. Key examples include:  
- **Wilson operators with matter field insertions** (e.g., inserting charged fermions or scalars into Wilson loops). These operators remain gauge invariant if the matter fields are in gauge-invariant states (e.g., topological charge sectors or gauge-invariant combinations like fermion number).  
- **Combined Wilson-matter operators** (e.g., products of Wilson loops and local charged operators), such as the 't Hooft operator combined with matter fields. These operators are gauge invariant because the matter fields’ gauge charges cancel out under gauge transformations, preserving the overall operator’s invariance.  

These additional operators exist because local charged operators allow for gauge-invariant combinations with Wilson loops, extending the class of gauge-invariant operators beyond topological loops alone.  

In summary, long-range symmetries support Wilson operators tied to topological properties, while local charged operators enable more complex gauge-invariant combinations involving matter fields.  

The answer is <|begin_of_box|>(a) Wilson loops around non-contractible cycles (topological Wilson operators) in theories with long-range gauge symmetry.  
(b) Wilson operators with insertions of local charged matter fields (e.g., fermions or scalars) combined with Wilson loops, existing in theories with local charged operators.<|end_of_box|></answer>