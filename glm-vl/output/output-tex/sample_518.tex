<think>Got, let's try to figure out the substitution rule for the geometric progression of triangles. First, let's recall what a geometric progression is—each term is a constant multiple of the previous one. In the case of triangles, maybe the side length or area follows a geometric sequence. 

Looking at the image description: the first triangle is at the bottom left, largest at top right, arranged in a stair-like shape. So maybe each subsequent triangle has a side length that's a multiple of the previous. Let's assume the first triangle has side length r, then the next one is r times some ratio, say, each time increasing by a factor of k. But wait, the problem mentions a substitution rule, which might relate to the number of triangles or their positions. 

Alternatively, maybe the substitution rule is about replacing each triangle with smaller triangles in a pattern. For a geometric progression, if each term is multiplied by a common ratio, say, the side length of each triangle is multiplied by a factor each time. Suppose the first triangle (n=1) has side length a, then the next (n=2) has side length ar, then ar², etc. But the image shows a stair-like structure, so maybe the number of triangles in each step follows a geometric sequence. 

Wait, another approach: in a geometric progression of triangles, if each triangle is similar to the previous one, then their corresponding sides are in a ratio. Let's say the first triangle has side length 1, the next has side length r, then r², etc. But the arrangement is stair-like, so maybe each step adds a triangle with side length increasing by a factor. Alternatively, the substitution rule could be that each triangle is replaced by a smaller triangle and some other shapes, but since it's a geometric progression, maybe the substitution is such that each term is the previous term multiplied by a constant factor. 

Wait, the question is about the substitution rule for the geometric progression of triangles. Maybe the substitution rule is that each triangle is replaced by a number of smaller triangles in a geometric sequence. For example, the first triangle is replaced by 1 triangle (but that doesn't make sense), or maybe each triangle is divided into smaller triangles in a way that the number of triangles follows a geometric progression. Alternatively, considering the area: if each triangle has area proportional to the square of its side length, then if sides are in ratio r, areas are in ratio r². But the problem might be about the number of triangles in each step. 

Wait, let's think of the image as a fractal-like structure, where each triangle is divided into smaller triangles, and the number of triangles increases in a geometric progression. Suppose the first triangle (level 1) has 1 triangle, then level 2 has 3 triangles (1 + 2), level 3 has 7 triangles (3 + 4), which is a geometric progression? Wait, 1, 3, 7... the differences are 2, 4, which is doubling, so the number of triangles might follow a sequence where each term is previous plus 2^(n-1), but that's a different progression. Alternatively, if it's a geometric sequence, the number of triangles would be 1, r, r², r³, etc. 

But the key here is the substitution rule. In calculus, the substitution rule (u-substitution) is used for integrals, but the question mentions "substitution rule for 1. The image shows a geometric progression...", so maybe the substitution rule refers to replacing the triangles with a geometric sequence in terms of their dimensions. If each triangle's side length is multiplied by a factor when substituted, say, each time substituting a triangle with a smaller one scaled by 1/2, then the area would scale by 1/4, but I need to connect this to the image's stair-like structure. 

Alternatively, considering the image has triangles arranged in a sequence where each subsequent triangle is larger, forming a stair shape, the substitution rule might be that each step (each triangle) is a scaled version of the previous one, with the scaling factor being a constant. For example, if the first triangle has side length 1, the next has side length 2, then 4, etc., doubling each time, forming a geometric progression with ratio 2. The substitution rule could then be that each triangle is replaced by a triangle with side length multiplied by the common ratio, say, r, and the area would follow r². 

But since the problem is about a substitution rule for the geometric progression of triangles, and given that the image shows a continuous upward progression with triangles increasing in size, the substitution rule might involve replacing each triangle with a number of smaller triangles in a geometric sequence, or the side length following a geometric progression. Given that it's a stair-like shape, maybe each step adds a triangle with side length increasing by a factor, and the substitution rule is that each triangle is part of a sequence where each term is the previous term multiplied by a constant, say, the substitution rule is that the side length of each triangle is the previous side length multiplied by a factor k, leading to a geometric progression. 

Wait, another angle: in a geometric progression, the nth term is a*r^(n-1). If the triangles are arranged such that each has a side length of a*r^(n-1), then the substitution rule could be replacing a triangle with side length a with one with side length a*r, maintaining the geometric progression. But without more specific details from the image, it's a bit tricky, but given the common geometric progression substitution, maybe the rule is that each triangle is replaced by a triangle with side length multiplied by a constant ratio, say, 2, leading to a geometric sequence with ratio 2. Alternatively, if the first triangle has area A, the next has area A*r, etc. 

Given the problem mentions "substitution rule for 1" (maybe a typo, like substitution rule 1), and the image shows a geometric progression of triangles arranged in a stair-like shape with increasing size, the substitution rule is likely that each triangle is substituted by a triangle with side length in a geometric progression, say, each subsequent triangle has side length k times the previous one, where k is the common ratio. Therefore, the substitution rule would involve replacing a triangle with side length s with one with side length s*r, maintaining the geometric progression, and this could relate to the area or number of triangles following the geometric sequence.</think><answer>The problem involves a geometric progression of triangles arranged in a stair-like shape, where each triangle's size increases following a geometric sequence. The substitution rule for this progression likely involves replacing each triangle with a scaled version of the previous one, maintaining a constant ratio (common ratio) between consecutive triangles. For example, if the first triangle has a side length \( s \), the next has side length \( rs \), the next \( r^2s \), and so on, where \( r > 1 \) is the common ratio. This ensures the sequence of triangle sizes forms a geometric progression, with each subsequent triangle being \( r \) times larger than the previous one.  

The answer is \(<|begin_of_box|>2<|end_of_box|>\).</answer>