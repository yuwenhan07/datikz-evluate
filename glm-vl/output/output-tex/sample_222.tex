<think>Got it, let's try to figure out this MCR (Minimum Cost Routed) problem with 4 obstacles. First, I need to recall what MCR is. MCR is a problem where you have obstacles, and you need to find a path from start to goal with the minimum cost, considering the obstacles. The transformed graph has vertices as obstacle sets, each node has obstacle numbers, and edges connect obstacle sets that are connected in 2D.

First, let's understand part (a): the instance with 4 obstacles, start (green) and goal (red). The obstacles are probably regions that block the path. Then part (b) is the graph representation where each vertex is a set of obstacles, meaning a connected region formed by the partition along obstacle boundaries. So each vertex is a region, and edges exist between regions that are adjacent (can be connected without crossing obstacles).

To solve this, maybe we need to model the graph where each node is a region, and edges represent possible paths between regions. The start and goal would be in some regions, and we need to find the shortest path in this graph, considering the costs between regions. But wait, the problem might be about finding the minimum cost path through the regions, where the cost could be related to the number of obstacles or the distance between regions.

Wait, let's think step by step. First, identify the obstacle sets (regions) in part (a). Each obstacle is a region, and the partition along obstacle boundaries creates connected regions. For 4 obstacles, the number of regions can vary, but the graph in part (b) has vertices representing these regions. Each vertex has obstacle numbers inside, meaning the set of obstacles that form that region. For example, if a region is between obstacles 1 and 2, maybe the vertex has {1,2}, but need to check the partition.

Then, the graph has edges between regions that are adjacent (i.e., share a boundary without an obstacle). The start and goal are in specific regions, so we need to find the path from the start's region to the goal's region through adjacent regions, with the minimum cost. The cost might be the distance between regions or some other metric, but since it's MCR, maybe the cost is related to the number of obstacles crossed or the path length.

Alternatively, since it's transformed into a graph, each vertex is a connected component (region) formed by the obstacles, and edges connect components that are adjacent (can be connected without crossing obstacles). So the problem reduces to finding the shortest path in this graph from the start's component to the goal's component.

To solve this, we can use Dijkstra's algorithm if the costs are non-negative. First, we need to determine the adjacency of regions (edges in the graph) and the cost between each pair of adjacent regions. Then, apply Dijkstra's to find the shortest path.

But maybe the problem is more about understanding the graph structure. Let's assume that in part (a), the obstacles create regions, and part (b) shows the graph with vertices as regions (each with obstacle numbers) and edges between adjacent regions. The start and goal are in specific vertices, so we need to find the path through the graph.

Alternatively, if the MCR problem is transformed into a graph where each vertex is an obstacle set, and edges represent possible paths that don't cross obstacles, then the solution is the shortest path in this graph. But without the exact image, I need to make general assumptions.

Wait, the question is asking for part (a) and (b), but since I don't have the image, maybe the key is to explain the transformation. The MCR problem with obstacles is transformed into a graph where vertices are obstacle sets (connected regions), and edges connect regions that are adjacent (can be connected without crossing obstacles). The start and goal are mapped to their respective regions in the graph, and then the shortest path in the graph gives the minimum cost path in the original MCR problem.

So, for part (a), the instance has start (green) and goal (red) in specific regions among the 4 obstacles. Part (b) shows the graph with vertices as regions (each labeled with obstacle numbers), and edges between regions that are adjacent. The solution would involve finding the shortest path from the start's vertex to the goal's vertex in this graph.

If I were to solve it, I'd first identify all the regions (vertices) formed by the 4 obstacles. Each region is a connected area not containing any obstacles. Then, for each pair of adjacent regions, there's an edge. The start and goal are in two of these regions. Then, using Dijkstra's algorithm on the graph, starting from the start's region, find the shortest path to the goal's region, considering the edge weights (which could be the distance between regions or some cost metric).

But since the problem might expect a specific answer, maybe the minimum cost path goes through certain regions, but without the exact image, it's hard. However, the key idea is that the MCR problem is transformed into a graph where vertices are obstacle sets (regions), edges are between adjacent regions, and the shortest path in this graph gives the minimum cost path in the original problem.</think><answer>To solve the MCR (Minimum Cost Routed) problem with 4 obstacles, we first transform the problem into a graph where vertices represent connected regions formed by partitioning along obstacle boundaries, and edges connect adjacent regions (regions that can be connected without crossing obstacles).  

### Step 1: Identify Regions (Vertices)  
Each vertex in the graph corresponds to a connected region in the original MCR instance, labeled with the set of obstacles forming that region. For 4 obstacles, the number of regions depends on their arrangement, but each region is a 2D connected area not containing any obstacles.  

### Step 2: Define Adjacency (Edges)  
Edges exist between vertices (regions) if they share a boundary without crossing any obstacles. This adjacency graph represents possible paths that avoid obstacles.  

### Step 3: Map Start/Goal to Regions  
The start (green) and goal (red) locations in the original instance are mapped to their respective regions in the graph.  

### Step 4: Apply Shortest Path Algorithm  
Using Dijkstra’s algorithm (for non-negative edge weights, which typically represent path cost or distance), find the shortest path from the start’s region to the goal’s region in the adjacency graph. This path corresponds to the minimum cost path in the original MCR problem.  

For the specific instance, if the start and goal are in regions connected via adjacent regions with minimal cost edges, the shortest path would traverse the fewest regions with the lowest total cost. Assuming standard graph structure, the minimum cost path likely connects the start and goal through 2–3 intermediate regions, depending on obstacle placement.  

The answer is <|begin_of_box|>The minimum cost path in the graph corresponds to the shortest path from the start’s region to the goal’s region, which is found via Dijkstra’s algorithm on the adjacency graph. (Specific path details depend on the exact obstacle arrangement, but the method ensures the minimum cost route is identified.)<|end_of_box|></answer>