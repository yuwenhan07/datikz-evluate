<think>Got, let's try to figure out what the question is about. The user mentioned "Unitarity (1) for the connection W_1(,)." and there's an image with a white square on a white background, equations, etc. First, I need to recall what unitarity means in physics, especially in quantum mechanics or quantum field theory. Unitarity is about the conservation of probability, meaning the sum of probabilities for all possible outcomes is 1, which is related to the Hermitian conjugate of the S-matrix being its inverse, ensuring probabilities don't change over time.

But the question is about a connection W_1(,). Maybe this is related to a connection in a fiber bundle, like in gauge theories, or a connection in a Lie algebra? Wait, W_1 might refer to a WZW model (Wess-Zumino-Witten model) with level 1, which is a conformal field theory. The connection there would be related to the gauge field. Unitarity for such a model would require that the representation is unitary, meaning the operators are unitary, so the S-matrix is unitary, preserving probabilities.

But the image has a white square—maybe the square is a diagram representing the connection, like a graph with vertices and edges, or a tensor network? Alternatively, maybe the question is about verifying unitarity for a specific connection, like a Wilson line or a connection in a quantum group. But without more details on the image's equations, it's hard to say exactly. However, the key terms are "unitarity" and "connection W_1(,)".

Let me think step by step. First, unitarity in quantum mechanics: a unitary operator U satisfies U†U = UU† = I, so it preserves inner products, hence probabilities. For a connection, if it's part of a quantum theory, the connection's representation should be unitary. The W_1 connection might be a specific type, like a Chern-Simons type connection with level 1, and unitarity would require that the representation is irreducible and unitary, which for WZW models with integer levels is possible.

Alternatively, if this is in the context of quantum field theory, the connection could be a gauge connection, and unitarity would require that the S-matrix is unitary, which is related to the gauge group being compact (for example, SU(N) is compact, so the theory is unitary). But W_1 might be a specific case.

Wait, the image has a white square—maybe the square is a diagram where the connection is represented, and the equations around it are the mathematical conditions for unitarity. For example, the connection's components must satisfy certain equations (like the Yang-Mills equations) and the representation must be unitary, so the matrix elements of the connection operators must form a unitary representation.

But since the user's description is a bit vague, maybe the key is to explain that unitarity for a connection (like a gauge connection or a connection in a fiber bundle) requires that the connection's representation is unitary, meaning the operators associated with the connection (like the gauge fields) form a unitary representation of the gauge group. For a WZW model with level 1, the unitarity is ensured if the representation is unitary, which is possible for certain levels and groups.

Alternatively, if W_1 is a specific connection, like a Wilson line with some properties, unitarity would mean that the transition amplitudes are unitary, so the sum of probabilities is 1. But without more specific info, I need to make a general statement about unitarity for a connection, referencing the mathematical conditions.

So, putting it all together: Unitarity for a connection (like W_1) in a quantum theory requires that the connection's associated operators form a unitary representation, meaning the S-matrix (or transition operators) are unitary, preserving probability conservation. This typically involves the connection's components satisfying certain algebraic or differential equations (like the Yang-Mills equations) and the representation being irreducible and unitary under the gauge group's action. For specific models like WZW models with level 1, unitarity is ensured if the representation is chosen appropriately, ensuring the operators are unitary and the theory's probabilities are conserved.</think><answer>To address the question about unitarity for the connection \( W_1(\cdot) \), we first recall that **unitarity** in quantum theory requires that operators preserve probability (i.e., the sum of probabilities for all outcomes is 1), which is mathematically expressed by the condition \( U^\dagger U = UU^\dagger = I \) for a unitary operator \( U \). For a connection (such as \( W_1 \)), unitarity typically means the connection’s associated operators form a **unitary representation** of the underlying gauge group or algebra.  


### Key Steps for Analyzing Unitarity of \( W_1(\cdot) \):  
1. **Connection and Representation**: If \( W_1 \) is a gauge connection (e.g., in a fiber bundle or quantum field theory), unitarity requires that the operators corresponding to the connection’s components (like gauge fields) form a unitary representation of the gauge group. This ensures the S-matrix (or transition amplitudes) is unitary, preserving probability conservation.  
2. **Mathematical Conditions**: For connections in models like the Wess-Zumino-Witten (WZW) model with level 1 (where \( W_1 \) might denote a level-1 connection), unitarity is ensured if the representation is irreducible and unitary under the group’s action. This often involves the connection satisfying equations like the Yang-Mills equations and the representation being chosen such that the operators are unitary.  
3. **Image Context**: The image’s white square likely represents a diagrammatic or geometric structure related to the connection (e.g., a graph, tensor network, or fiber bundle), with surrounding equations detailing the connection’s properties. Unitarity would manifest as the equations ensuring the connection’s operators satisfy unitary conditions (e.g., Hermiticity, norm preservation).  


In summary, unitarity for \( W_1(\cdot) \) requires the connection’s operators to form a unitary representation, ensuring probability conservation and satisfying the algebraic/differential equations of the theory. For specific models (like level-1 WZW models), this is achievable with appropriately chosen representations.  

The answer is \(<|begin_of_box|>1<|end_of_box|>\).</answer>