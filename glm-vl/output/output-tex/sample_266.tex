<think>Got it, let's try to unpack this. The question is about a pseudo-periodic map's dual graph and its corresponding 3D mirror. First, let's recall some concepts. A pseudo-periodic map is a type of tiling or mapping with some periodic-like properties but not fully periodic. The dual graph of a tiling is a graph where each vertex corresponds to a tile, and edges connect adjacent tiles. For a pseudo-periodic map, the dual graph might have components with different genera (genus is a topological invariant related to the number of "holes" or handles, for a graph, it's the number of handles needed to embed it on a surface without crossings).

The problem states the dual graph has a weighted graph with one genus g₁ component connected by a genus zero component. So, the dual graph has two parts: one component (maybe a cycle or a more complex graph) with genus g₁, and another component (maybe a tree or a simpler graph) with genus zero, connected together. Genus zero usually means the component can be embedded on a sphere (or plane), like a tree or a planar graph. Genus g₁ would be a graph that needs a surface with g₁ handles, like a torus if g₁=1, etc.

Now, the corresponding 3D mirror. In mirror symmetry, a 3D Calabi-Yau manifold (or its mirror) is related to the geometry of the tiling. The dual graph's structure would influence the mirror's topology. If the dual graph has a genus g₁ component connected by a genus zero component, the mirror might have a 3D structure where one part has a certain handle structure (related to the genus) and the other part is simpler. Maybe the mirror has a component with g₁ handles and another part that's more "simple" (like a product with a sphere or something), connected in a way that reflects the connection in the dual graph.

Wait, let's think about the dual graph and the mirror. The mirror of a pseudo-periodic tiling is often a 3D polytope or a manifold, and the dual graph's properties (like genus, connectedness) correspond to the mirror's topology. If the dual graph has a genus g₁ component connected by a genus zero component, the mirror might have a 3D structure where one part is a manifold with genus g₁ (like a handlebody) and the other part is a simpler manifold (like a ball or a product with a sphere), connected along their boundaries. The connection between the components in the dual graph (the genus zero component connecting the genus g₁ component) would correspond to a "bridge" or a connection in the 3D mirror, maybe a tube or a cylindrical connection between the two parts.

Also, weighted graph—maybe the edges have weights, which could correspond to volumes or other geometric properties in the mirror. The genus g₁ component might have a more complex topology, while the genus zero component is simpler, and their connection is a boundary or a link.

So, putting it all together: the dual graph of the pseudo-periodic map has a structure with one component of genus g₁ (which would require a surface with g₁ handles to embed) and a genus zero component (embeddable on a sphere) connected by some edges. The corresponding 3D mirror would then have a 3D structure where one part is a manifold with g₁ handles (like a genus g₁ surface times a line, or a more complex 3D shape with g₁ handles) and the other part is a simpler 3D shape (like a ball or a cylinder) connected along their common boundary, reflecting the connection in the dual graph. The weighted aspects might correspond to the volumes or other invariants of the mirror's components.

I need to make sure I connect the dual graph's components to the mirror's structure. The dual graph's genus is related to the mirror's handle structure. If the dual graph has a genus g₁ component, the mirror might have a 3D manifold with g₁ handles, and the genus zero component might correspond to a simpler part, maybe a 3D ball or a product with a sphere, connected via a boundary. The connection between the two components in the dual graph (the edges connecting them) would correspond to a 3D "bridge" or a tube connecting the two parts of the mirror.

Another angle: in the context of mirror symmetry for toric varieties, the dual graph (the toric diagram) corresponds to the mirror. But here it's a pseudo-periodic map, maybe a non-toric case. The dual graph's genus components might correspond to different "cells" or regions in the mirror with different topologies. The genus zero component connected to the g₁ component could mean the mirror has a part with a more complex topology (g₁) and a simpler part (genus zero) linked together, with the connection reflecting the edges in the dual graph.

So, summarizing: The dual graph has a structure with one component of genus g₁ (requiring a surface with g₁ handles) and a genus zero component (embeddable on a sphere) connected by edges. The corresponding 3D mirror would have a 3D structure where one part is a manifold with g₁ handles (like a genus g₁ surface extended into 3D) and the other part is a simpler 3D shape (like a ball or a cylinder) connected along their common boundary, with the connection reflecting the edges connecting the two components in the dual graph. The weighted aspects of the graph might correspond to geometric invariants (like volumes, lengths) of the mirror's components.</think><answer>To analyze the relationship between the dual graph of a pseudo-periodic map and its corresponding 3D mirror, we start by interpreting the dual graph’s structure:  

1. **Dual Graph Structure**: The dual graph has a weighted component with one genus \( g_1 \) subgraph (requiring a surface with \( g_1 \) handles to embed, e.g., a torus if \( g_1 = 1 \)) and a genus 0 subgraph (embeddable on a sphere, like a tree or planar graph) connected by edges. Genus 0 implies the subgraph is topologically simple (e.g., a tree or a cycle with no "holes"), while genus \( g_1 \) indicates a more complex topology with \( g_1 \) handles.  

2. **3D Mirror Correspondence**: In mirror symmetry, the dual graph’s topology reflects the mirror’s 3D structure. The genus \( g_1 \) component of the dual graph corresponds to a 3D manifold with \( g_1 \) handles (e.g., a handlebody or a product of a genus \( g_1 \) surface with a line segment). The genus 0 component corresponds to a simpler 3D shape (e.g., a ball, cylinder, or product with a sphere). The edges connecting the two dual graph components correspond to a "bridge" or cylindrical connection in the 3D mirror, linking the two topologically distinct regions.  

3. **Weighted Graph Aspects**: The weights on the dual graph likely correspond to geometric invariants of the mirror (e.g., volumes, edge lengths, or other topological invariants), which are preserved or transformed under mirror symmetry. These weights would translate to properties of the 3D mirror’s components, such as their volumes or the geometry of their connection.  

In summary, the dual graph’s structure—with a genus \( g_1 \) component connected by a genus 0 component—corresponds to a 3D mirror with a complex \( g_1 \)-handled region linked to a simpler region via a connecting structure, reflecting the graph’s connectivity and genus properties.  

The answer is \(<|begin_of_box|>g_1<|end_of_box|>\).</answer>