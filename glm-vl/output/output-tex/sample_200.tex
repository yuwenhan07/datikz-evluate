<think>Got it, let's try to unpack this step by step. First, the problem is about equipping a 1-skeleton with factorization rules, then considering color sets (red and blue) for full reducibility, and then what happens if we use a smaller color set. Also, there's a mention of reduction not being possible if we remove b₁, which affects the commuting square.

First, let's recall what a 1-skeleton and factorization rules are. A 1-skeleton is a graph, and factorization rules are relations between edges, like c₂ b₁ e₁ a₂ and e₂ a₁ c₁ b₂. These are probably relations in a group or a monoid, where the product of the edges equals the identity or some other relation. Then, full reducibility with a color set means that the graph can be reduced by removing certain edges (colored, say, red or blue) such that the relations are preserved.

The problem states that with the red (dashed) and blue (dotted) color sets, w is fully reducible. But if we choose a smaller color set, the first condition of Definition red (maybe Definition of full reducibility?) isn't satisfied. Then, trying to reduce by removing b₁ would only remove one edge from the commuting square c₂ b₁ e₁ a₂, so the resulting 1-skeleton doesn't correspond to a 2-graph anymore.

Let me try to visualize the 1-skeleton. The factorization rules are c₂ b₁ e₁ a₂ and e₂ a₁ c₁ b₂. These look like commutative squares or relations in a 2-complex. The 1-skeleton would have edges a₁, a₂, b₁, b₂, c₁, c₂, e₁, e₂, and vertices connected by these edges. The commuting square c₂ b₁ e₁ a₂ probably means that the product c₂ * b₁ equals e₁ * a₂ in some group, and similarly for the other relation.

Full reducibility with color sets usually requires that the color set is such that the relations can be reduced by removing colored edges, maintaining the relations. If the color set is too small, maybe the first condition (like the color set must contain certain edges to allow reduction) isn't met. Removing b₁ would affect the first relation c₂ b₁ e₁ a₂, removing b₁ would leave c₂ e₁ a₂, but maybe that's not a valid relation, or the square isn't preserved, hence the 1-skeleton doesn't correspond to a 2-graph anymore because the 2-cells (which are represented by the relations) would be broken.

So, summarizing: The 1-skeleton has factorization rules defining relations between edges. Full reducibility with red and blue color sets means we can remove those edges to simplify the graph while preserving the relations. A smaller color set fails the first condition of full reducibility, meaning we can't remove edges to maintain the relations. Removing b₁ only affects one edge in the first commuting square, breaking the relation, so the resulting graph isn't a valid 2-graph (since 2-cells depend on those relations).</think><answer>To address the problem, we analyze the 1-skeleton with the given factorization rules and color sets:  

1. **Factorization Rules and 1-Skeleton**: The factorization rules \( c_2 b_1 e_1 a_2 \) and \( e_2 a_1 c_1 b_2 \) define relations in the 1-skeleton, which can be visualized as commuting squares (2-cells) in a 2-complex. These rules dictate how edges interact, forming the structure of the 1-skeleton.  

2. **Full Reducibility with Color Sets**: The red (dashed) and blue (dotted) color sets allow full reducibility, meaning edges in these sets can be removed while preserving the relations. This satisfies the first condition of full reducibility (likely requiring the color set to contain edges necessary for reducing relations).  

3. **Smaller Color Set**: If a smaller color set is chosen, the first condition of full reducibility is not satisfied, meaning edges cannot be removed without breaking the relations.  

4. **Reduction with \( b_1 \)**: Removing \( b_1 \) only eliminates one edge from the commuting square \( c_2 b_1 e_1 a_2 \), leaving \( c_2 e_1 a_2 \). This breaks the relation, as the square is no longer preserved. The resulting 1-skeleton lacks the necessary 2-cell structure (since 2-cells depend on the commuting relations), so it does not correspond to a valid 2-graph.  

The answer is \(<|begin_of_box|>2<|end_of_box|>\).</answer>