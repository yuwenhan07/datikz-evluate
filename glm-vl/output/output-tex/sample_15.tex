<think>Got it, let's try to parse this problem step by step. First, the question mentions that the interval \( I_n \) converges uniformly to a nonconstant flow line joining two critical points. Then, we need to repeat the arguments to \( (a_n, b_n') \) and \( (a_n', b_n) \). 

First, let's recall what a flow line is in the context of dynamical systems, maybe like a solution to a differential equation, which connects critical points (which are fixed points or equilibrium points). A nonconstant flow line would mean it's not just a single point, so it's a curve connecting two such points. Uniform convergence of intervals to a flow line suggests that as \( n \) increases, the intervals \( I_n \) get closer and closer to this flow line in a uniform way, meaning the maximum difference between the interval and the flow line goes to zero.

Now, the next part says "repeat the above arguments to \( (a_n, b_n') \) and \( (a_n', b_n) \)". So maybe \( a_n, b_n \) are endpoints of the interval \( I_n \), and \( a_n', b_n' \) are endpoints of another interval or related intervals. The arguments probably involve some properties of the flow line, critical points, maybe the behavior of the intervals as they converge, and perhaps applying similar reasoning to the other intervals \( (a_n, b_n') \) and \( (a_n', b_n) \).

I need to make sure I understand the setup. If \( I_n \) is an interval converging uniformly to a flow line between two critical points, then the endpoints \( a_n, b_n \) should converge to the critical points. Then, when we consider \( (a_n, b_n') \) and \( (a_n', b_n) \), maybe these are adjacent intervals or related to the flow line's structure. The repetition of arguments might involve showing that these other intervals also converge appropriately, perhaps to parts of the flow line or other critical points, maintaining the uniform convergence or similar properties.

Wait, maybe this is related to the concept of connecting critical points via flow lines, and using some kind of approximation or iterative process where each interval is refined, and the convergence is maintained. The key points here are uniform convergence, nonconstant flow line, critical points, and repeating the arguments for other intervals. Since the problem is a bit abstract without more context, but given the mention of repeating arguments, it's likely that the original arguments (maybe about the flow line's properties, critical points' behavior, interval convergence) can be applied analogously to the other intervals, leading to similar conclusions about their convergence or properties.

So, summarizing, the process involves showing that as \( n \) increases, the interval \( I_n \) gets closer to the flow line, then applying the same reasoning to the intervals \( (a_n, b_n') \) and \( (a_n', b_n) \), which might be intermediate intervals or related to the endpoints, to conclude their convergence or properties, maintaining the uniform convergence and flow line connection.</think><answer>The problem describes a sequence of intervals \( I_n \) that uniformly converges to a nonconstant flow line connecting two critical points. This implies the endpoints of \( I_n \) converge to the critical points as \( n \to \infty \). Repeating the arguments for intervals \( (a_n, b_n') \) and \( (a_n', b_n) \) involves applying the same reasoning about uniform convergence and flow line properties to these related intervals, likely to confirm their convergence behavior or connection to critical points.  

The answer is \(<|begin_of_box|>I_n \text{ converges uniformly to a nonconstant flow line joining two critical points, and the arguments repeat similarly for } (a_n, b_n') \text{ and } (a_n', b_n)<|end_of_box|>\).</answer>