<think>Got it, let's try to understand the problem here. The question is about backreacted de Sitter space at finite temperature, with Penrose diagrams at zero temperature (dotted lines) and then the finite temperature case with apparent horizon (black dot) and cosmological horizons (red dots). 

First, I need to recall what de Sitter space is. De Sitter space is a maximally symmetric solution to Einstein's equations with a positive cosmological constant, representing a universe with a positive curvature. At zero temperature, the Penrose diagram would typically show the infinite future and past, with the cosmological horizons (which are the boundaries of the de Sitter space, like the event horizons for a static observer). But when you have a black hole, backreaction means the presence of the black hole affects the spacetime geometry, so the de Sitter space is "backreacted" by the black hole's presence.

At finite temperature, the system is likely a black hole in de Sitter space, and the temperature would be related to the size of the black hole. The apparent horizon (black dot) is the boundary of the event horizon for the black hole, while the cosmological horizons (red dots) are the horizons of the de Sitter space itself. In the backreacted case, the cosmological horizons might be different from the zero-temperature case because the black hole's presence changes the geometry.

Penrose diagrams for de Sitter space at zero temperature usually have a diamond shape with the center being the big bang and big crunch, but with infinite future and past. When you have a black hole, the Penrose diagram would have the black hole's event horizon (apparent horizon) and the cosmological horizons. At finite temperature, the black hole is in a thermal state, so the Penrose diagram would show the black hole's horizon and the cosmological horizons, with the temperature affecting the size of the black hole and the horizons.

I need to make sure I connect the elements: backreaction means the black hole's mass affects the de Sitter space, so the cosmological constant is modified by the black hole's energy. The apparent horizon is the boundary of the black hole's event horizon, while cosmological horizons are the horizons of the entire spacetime. At finite temperature, the black hole emits Hawking radiation, so the temperature is related to its mass. The Penrose diagram would show the black hole's horizon (black dot) and the cosmological horizons (red dots) which are now closer or further depending on the temperature.

Wait, the problem mentions that the dotted lines are Penrose diagrams at zero temperature. So without the black hole (or with a de Sitter space without backreaction), the Penrose diagram is the dotted lines. When you add the backreaction (the black hole), the Penrose diagram changes, with the black dot being the apparent horizon (so the event horizon of the black hole) and red dots as cosmological horizons. At finite temperature, the black hole is in a thermal state, so the spacetime is a black hole in de Sitter space, and the horizons are affected by the temperature.

I should also recall that in de Sitter space, the cosmological horizon is the boundary beyond which events cannot affect an observer. When a black hole is present, the apparent horizon is the boundary of the black hole's event horizon, and the cosmological horizon is still the boundary of the de Sitter space. The backreaction would mean that the black hole's presence changes the cosmological horizon's position.

So, summarizing: Backreacted de Sitter at finite temperature means a black hole in de Sitter space with temperature, Penrose diagram shows apparent horizon (black dot) and cosmological horizons (red dots). The dotted lines are the zero-temperature de Sitter Penrose diagram, which is different from the finite temperature case with the black hole. The apparent horizon is the event horizon of the black hole, while cosmological horizons are the horizons of the entire spacetime. At finite temperature, the black hole's mass is such that it's in thermal equilibrium, so the horizons are positioned accordingly, with the apparent horizon being inside the cosmological horizon (or maybe not, depending on the black hole size relative to de Sitter).

I need to make sure the description connects the elements: backreaction (so the de Sitter space is modified by the black hole), finite temperature (so the black hole is radiating, hence the thermal state), Penrose diagrams (dotted vs solid, with horizons), apparent horizon vs cosmological horizons. The key points are the presence of the black hole causing backreaction, the temperature affecting the horizons, and the Penrose diagram elements.</think><answer>To understand the backreacted de Sitter space at finite temperature with the described Penrose diagram elements, let’s break down the key concepts:  

1. **De Sitter Space Basics**: De Sitter space is a maximally symmetric solution to Einstein’s equations with a positive cosmological constant (\(\Lambda > 0\)), representing a universe with positive curvature. At zero temperature, its Penrose diagram (dotted lines) typically shows infinite future/past with cosmological horizons (boundaries of the spacetime).  

2. **Backreaction**: When a black hole is present, its mass/energy modifies the de Sitter geometry (backreaction), altering the cosmological constant and spacetime structure. This changes the Penrose diagram from the zero-temperature case.  

3. **Finite Temperature and Black Hole**: A black hole in de Sitter space at finite temperature is in a thermal state (Hawking radiation), with temperature \(T \propto 1/M\), where \(M\) is the black hole mass. The apparent horizon (black dot) is the boundary of the black hole’s event horizon, while cosmological horizons (red dots) are the horizons of the entire backreacted de Sitter space.  

4. **Penrose Diagram Comparison**:  
   - **Dotted lines (zero temperature)**: Represent pure de Sitter space without a black hole, with cosmological horizons defining the spacetime boundaries.  
   - **Solid lines (finite temperature)**: Show the backreacted de Sitter space with a black hole. The apparent horizon (black dot) is inside the cosmological horizons (red dots), as the black hole’s event horizon is a subset of the larger de Sitter horizons.  

5. **Horizon Interpretation**:  
   - **Apparent Horizon (Black Dot)**: The boundary of the black hole’s event horizon, marking the region from which light cannot escape the black hole.  
   - **Cosmological Horizons (Red Dots)**: Boundaries of the entire backreacted de Sitter space, marking the region beyond which events cannot affect an observer.  

In summary, the backreacted de Sitter space at finite temperature combines the effects of a black hole with de Sitter geometry, resulting in a Penrose diagram with an apparent horizon (black dot) and cosmological horizons (red dots), distinct from the zero-temperature de Sitter diagram (dotted lines).  

The answer is \(<|begin_of_box|>\text{Described the backreacted de Sitter at finite temperature with Penrose diagram elements as outlined}<|end_of_box|>\).</answer>