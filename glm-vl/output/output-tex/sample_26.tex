<think>Got it, I need to illustrate the observable and non-observable points for the trajectory a(t) = (0, 0, 2t) in the xOz plane, with t1 and t2. First, let's understand the trajectory. The position vector is (0, 0, 2t), so in 3D space, that's along the z-axis, starting from the origin when t=0, moving upwards as t increases. But the problem mentions the xOz plane, which is the plane where y=0, so the trajectory is indeed in the xOz plane since the y-component is always 0.

Now, observable points would be those that can be "seen" from some point, maybe the origin or another reference point? Wait, the problem says "observable (green arc) and non-observable (dotted arc) points for the trajectory". Maybe it's about visibility from a certain point, like an observer at the origin. If we consider an observer at the origin, the line of sight from the origin to a point (0, 0, 2t) is along the z-axis. But if there's an obstacle or something blocking the view, but the problem doesn't mention obstacles. Wait, maybe it's about the trajectory being observable if the point is visible from some point, but perhaps the key is the direction of the trajectory and the angles.

Wait, the trajectory is a straight line along the z-axis. If we consider the xOz plane, which is the plane y=0, so the points on the trajectory are (0, 0, 2t), which lie on the z-axis. Now, if we're looking for observable points, maybe those points where the line from the observer (say, at the origin) to the point doesn't intersect any other part of the trajectory? But since the trajectory is a straight line, maybe the observable points are those where the angle is such that the point is not occluded by another point on the trajectory. Wait, but the trajectory is a single line, so if you have two points on the line, say at t1 and t2, the points between them might be observable if there's no obstruction, but maybe the problem is about the trajectory being a curve, but here it's a straight line.

Wait, another approach: in the xOz plane, the trajectory is a straight line with parametric equations x=0, z=2t, so z = 2t, x=0. If we consider the trajectory as a curve from t1 to t2, then the points on the curve are connected. Now, observable points might be those that are visible from a certain point, say the point (0,0,0), the origin. The line from the origin to a point (0,0,2t) is along the z-axis, so all points on the trajectory are visible from the origin because there's no obstruction in the xOz plane (assuming the plane is unobstructed). But maybe the problem is referring to the trajectory being observable if the point is not "behind" an observer's position, but I need more details.

Wait, the problem says "illustration of observable (green arc) and non-observable (dotted arc) points for the trajectory a(t) = (0,0,2t) for t1,2 in the xOz plane." Maybe the key is that the trajectory is a straight line, and if we consider two points at t1 and t2, the arc between them is observable, and the rest are non-observable? Or perhaps the observable points are those where the angle between the trajectory and the line of sight is such that they are visible, but I'm getting confused.

Alternatively, maybe the trajectory is a straight line, and the observable points are those that lie on the same side of a certain plane or line, making them visible, while others are occluded. But without more specific information about the observer's position or the nature of the occlusion, it's hard to say. Wait, the problem mentions "for t1,2", so maybe t1 and t2 are two specific times, and the trajectory between them is observable (green arc), while outside is non-observable (dotted arc). But the trajectory is a straight line, so if it's a line segment between t1 and t2, then all points on that segment are observable, and beyond are non-observable. But the problem says "points for the trajectory", so maybe the trajectory is a curve, but here it's a straight line, so the illustration would show a green arc (a segment) and dotted arc (the rest) on the z-axis in the xOz plane.

Wait, let's think about the parametric equations again. The trajectory is (0,0,2t), so in the xOz plane, it's a vertical line (since x=0, z increases with t). If we consider two points at t1 and t2, say t1 < t2, then the points between t1 and t2 form a line segment, which would be the observable part (green arc), and the points before t1 or after t2 are non-observable (dotted arc). But how to illustrate this? The xOz plane has x and z axes, so the trajectory is along the z-axis, starting from (0,0,2t1) to (0,0,2t2), with the segment between them as green arc and the rest dotted. But maybe the problem is about the trajectory being a curve that's not a straight line, but here it is a straight line, so the observable part is the line segment between t1 and t2, and non-observable are the rest.

Alternatively, if the observer is at a point not on the trajectory, say at (x,0,z0), then the visibility would depend on the line of sight, but without specific observer details, it's hard. Given the problem statement, maybe the key is that in the xOz plane, the trajectory is a straight line, and the observable points are those that form a connected arc (like a line segment) between t1 and t2, while the rest are dotted. So the illustration would show a green arc connecting (0,0,2t1) to (0,0,2t2) and dotted arcs extending beyond those points. But since it's a straight line, the arc would be a straight segment, and dotted arcs are the extensions.

Wait, another thought: if the trajectory is a straight line, then all points on the line are connected, but maybe the observable part is the part that's "in front" of the observer, while the rest are behind. If the observer is at the origin, then points with z > 0 are in front, but if the trajectory is along the positive z-axis, then all points are in front, but maybe the observable arc is between two points, and the rest are non-observable. Given the problem's mention of "for t1,2", maybe t1 and t2 are two specific times, and the arc between them is observable, while outside is non-observable. So the illustration would have a green arc from (0,0,2t1) to (0,0,2t2) and dotted arcs before and after that segment.

In summary, the trajectory is a straight line along the z-axis in the xOz plane. The observable points (green arc) would be the segment between the two points at t1 and t2, and non-observable (dotted arc) would be the rest of the line beyond those points. The illustration would show this line with the middle segment green and the ends dotted.</think><answer>The trajectory \( a(t) = (0, 0, 2t) \) lies along the z-axis in the xOz plane, with points parametrized by \( z = 2t \). For \( t_1 \) and \( t_2 \) (where \( t_1 < t_2 \)), the observable points form a green arc connecting \( (0, 0, 2t_1) \) to \( (0, 0, 2t_2) \), representing the segment of the trajectory between these times. Non-observable points, represented by dotted arcs, extend beyond \( t_1 \) and \( t_2 \) along the z-axis.  

The answer is \(<|begin_of_box|>(0, 0, 2t_1) \text{ to } (0, 0, 2t_2) \text{ (green arc), rest dotted}<|end_of_box|>\).</answer>