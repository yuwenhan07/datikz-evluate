<think>Got it, let's try to figure out the problem here. The question mentions strips H₁, H₂m, and subproblem A_i (shown in shade). There are two rectangles, white and gray, with heights h1 and h2, where h1=2 and h2=3. Wait, maybe we need to find the area or some relation between them? But the problem statement is a bit vague. Wait, let's parse the question again. It says "the image shows a pair of white and gray rectangles, each with a different height, positioned on a white background. The heights of the rectangles are represented by the variables h1 and h2, where h1 is the height of the smaller rectangle and h2 is the height of the larger rectangle. The dimensions of the rectangles are such that the smaller one has a height of 2, while the larger rectangle's height is 3." Maybe we need to find the area of one of the rectangles or the total area? Wait, but without knowing the width, maybe it's about the height difference or something else. Wait, the problem might be asking for the area of the larger rectangle minus the smaller one, or maybe the area of the shaded subproblem? Wait, the question is a bit unclear, but let's assume that we need to find the area of the larger rectangle if the smaller one's area is known, but wait, we need width. Wait, maybe the problem is simpler. If h1=2 and h2=3, and assuming the width is the same for both? Wait, no, the problem says "strips H₁, H₂m" – maybe H₁ is the height of the first strip, H₂m is the second strip, and the subproblem A_i is the shaded area. Wait, perhaps the key is that the height of the smaller rectangle is 2, larger is 3, so if they are placed side by side or stacked, but without more info, maybe the area of the larger rectangle is 3 times the width if the smaller is 2 times width, but wait, this is confusing. Wait, let's try to think differently. If the smaller rectangle has height 2 and larger has 3, and maybe they are in a configuration where the total height is 5, but the problem might be asking for the area of the shaded subproblem, which is the larger rectangle minus the smaller one, so area difference is (3w - 2w) = w, but without w, maybe the question is about the height difference, which is 1. Wait, but the user might have intended a specific calculation. Wait, given that h1=2, h2=3, and if we need to find the area of the larger rectangle if the smaller is, say, 2x w, then larger is 3x w, so the difference is x w. But without more info, maybe the problem is simply asking for the height of the larger rectangle minus the smaller one, which is 3-2=1, but the question mentions "subproblem A_i (shown in shade)" – maybe the shaded area is the difference in height, but I'm getting confused. Wait, let's start over. The problem states: heights h1=2, h2=3. If the rectangles are, for example, two rectangles with heights 2 and 3, and maybe the same width, then the area of the larger is 3w, smaller is 2w, difference is w. But without knowing w, maybe the question is about the height of the combined figure? Wait, the image shows a pair of rectangles, maybe stacked, so total height is 2+3=5, but the problem says "strips H₁, H₂m" – maybe H₁ is 2, H₂m is 3, so the subproblem A_i is the area between them, which would be 3* something minus 2* something. Wait, I think I need more information, but since the problem provides h1=2, h2=3, and maybe the area of the larger rectangle is 3 times the smaller's area if width is same, but without width, maybe the answer is the difference in height, which is 1, or the area difference if width is 1, then 1*1=1. But I'm not sure. Wait, the question might be a standard one where if two rectangles have heights h1 and h2, the area of the larger minus smaller is (h2 - h1)*width, but since width isn't given, maybe the problem assumes unit width, making the area difference 1*1=1. Alternatively, if the question is asking for the height of the larger rectangle, it's 3, but the problem says "the dimensions of the rectangles are such that the smaller one has a height of 2, while the larger rectangle's height is 3" – maybe the answer is 3, but the question might want the area of the shaded subproblem, which could be the larger rectangle's area minus the smaller one's, but without width, it's hard. Wait, given the information, maybe the answer is the height of the larger rectangle minus the smaller one, which is 3-2=1, or the area difference if width is 1, then 1. Alternatively, if the problem is about the total height of both rectangles, it's 2+3=5. But I need to make a best guess. Given that h1=2, h2=3, and the question might be asking for the height of the larger rectangle, which is 3, or the difference, which is 1. Wait, the problem says "subproblem A_i (shown in shade)" – maybe the shaded area is the larger rectangle, so its height is 3, or the area is 3 times width. But without more info, I'll assume the question is asking for the height of the larger rectangle, which is 3, or the difference in height is 1. Wait, let me check the problem statement again: "the image shows a pair of white and gray rectangles, each with a different height, positioned on a white background. The heights of the rectangles are represented by the variables h1 and h2, where h1 is the height of the smaller rectangle and h2 is the height of the larger rectangle. The dimensions of the rectangles are such that the smaller one has a height of 2, while the larger rectangle's height is 3." So the question might be asking for the height of the larger rectangle, which is 3, or maybe the area of the larger rectangle if we assume unit width, then 3. But the problem says "strips H₁,,H_2m" – maybe H₁ is 2, H_2m is 3, so the subproblem A_i is the area between them, which would be 3* something minus 2* something. But without more info, I think the key numbers are h1=2, h2=3, so the difference is 1, and maybe the answer is 1. Alternatively, if it's the area of the larger rectangle minus the smaller one, with same width, say w, then 3w - 2w = w, but without w, can't compute. Wait, maybe the question is a trick question, and the answer is 3-2=1, so the height difference is 1, or the area difference is 1 when width is 1. Given the information provided, I'll go with the height difference or the area difference as 1, or the larger height is 3. But since the problem mentions "subproblem A_i (shown in shade)", maybe the shaded area is the larger rectangle, so its height is 3, or area is 3. Given the possible answers, I think the intended answer is 3, but let me confirm. Wait, the problem says "the smaller one has a height of 2, while the larger rectangle's height is 3" – so the larger rectangle's height is 3, which is the answer if they're asking for h2, which is 3.</think><answer>The problem describes two rectangles with heights \( h_1 = 2 \) (smaller) and \( h_2 = 3 \) (larger). Assuming the question asks for the height of the larger rectangle, the answer is derived directly from the given dimensions.  

The answer is \(<|begin_of_box|>3<|end_of_box|>\).</answer>