<think>Got it, let's try to parse this question. The user is talking about relations among sources for a fixed continuous solution of a balance law (equation \eqref{EE}) under a non-degeneracy assumption (Assumption \eqref{ass:h}). They mention that when the Lagrangian source is continuous, it's the source term in all formulations. 

First, I need to understand the context. Balance laws often involve conservation laws, and sources can be Eulerian or Lagrangian. The Lagrangian perspective is common in fluid dynamics, where you track material elements. The non-degeneracy assumption probably ensures that certain conditions hold, like the Jacobian not being zero, which is important for the well-posedness of the problem.

The question seems to be asking about the nature of these relations—maybe how different formulations (Eulerian vs. Lagrangian) relate, or how the source term behaves under the given assumptions. Since the Lagrangian source is continuous and is the source term in all formulations, maybe the key point is that when the Lagrangian source is continuous, it serves as a common reference or that all formulations converge to it under the non-degeneracy condition.

I need to make sure I'm addressing the question correctly. The user might be asking for an explanation of the relations, the role of the Lagrangian source, or the implications of the non-degeneracy assumption. Since the question is a bit descriptive, maybe the answer should elaborate on how the relations are established, the role of the Lagrangian source as a common term, and how the non-degeneracy assumption affects these relations.

Wait, the original question is a bit of a statement, but maybe the user is asking for an explanation or elaboration. Let me check the structure: "We picture relations among the sources that we determine for a fixed continuous solution of the balance law~\eqref{EE} under the non-degeneracy Assumption~\eqref{ass:h}. When the Lagrangian source is continuous, it is `the' source term in all the formulations." So maybe the question is asking for an explanation of why the Lagrangian source is the source term in all formulations under those conditions, or what the relations among sources are.

Given that, the thought process would involve recalling that in balance laws, different coordinate systems (Eulerian and Lagrangian) have different expressions for the source term. The Lagrangian perspective tracks the material, so the source term there is often more fundamental. Under non-degeneracy (which might mean the transformation between coordinates is invertible, like the Jacobian is non-zero), the source terms in different formulations are related by the change of variables, and when the Lagrangian source is continuous, it serves as the common term that all other formulations (Eulerian, etc.) can be related to, hence being "the" source term in all formulations.

So, summarizing, the relations among sources are established through the change of variables under the non-degeneracy assumption, and the Lagrangian source, being continuous, serves as the unified source term across different formulations.</think><answer>To address the question about relations among sources for a fixed continuous solution of the balance law under the non-degeneracy assumption, we analyze the role of the Lagrangian source term:  

1. **Context of Balance Laws and Formulations**: Balance laws (e.g., conservation laws) can be expressed in different coordinate systems (Eulerian, Lagrangian). The Lagrangian perspective tracks material elements, while the Eulerian perspective focuses on fixed spatial regions.  
2. **Non-Degeneracy Assumption**: This typically ensures invertibility of transformations (e.g., the Jacobian is non-zero), which is critical for well-posedness and consistency between formulations.  
3. **Lagrangian Source as a Common Term**: When the Lagrangian source is continuous, it acts as a unified reference for all formulations. Different formulations (Eulerian, mixed, etc.) relate to this Lagrangian source via coordinate transformations, and under the non-degeneracy assumption, these transformations preserve the source term’s continuity and consistency.  

Thus, the relations among sources are governed by the coordinate transformations, and the continuous Lagrangian source serves as the fundamental term that all other formulations converge to or relate to, making it "the" source term in all contexts.  

The answer is <|begin_of_box|>\text{The Lagrangian source is the unified source term in all formulations under the non-degeneracy assumption, as it is continuous and relates to other formulations via coordinate transformations.}<|end_of_box|></answer>