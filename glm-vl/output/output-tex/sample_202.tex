<think>Got it, let's try to figure out this problem. First, the problem mentions UD=(RRRLLRLL)(LLRLRLRR). I think UD is a union or a combination of two sequences, maybe representing a path or a sequence of up and down movements? Wait, the problem says elevation is positive inside U, negative inside D, and zero for exactly three indices, which are marked with black dots. The dashed line is elevation zero.

First, let's parse the sequences. Let's split UD into U and D. Wait, UD is written as (RRRLLRLL)(LLRLRLRR). Maybe U is the first part, D the second? Let's check the length. Each R and L is a step. Let's count the number of steps. First part: RRRLLRLL has 8 steps (3 R, 2 L, 2 R, 2 L), second part: LLRLRLRR has 8 steps (2 L, 1 R, 2 L, 2 R). So total 16 steps. But the problem says "the zero elevation locations are marked with a black dot" and there are exactly three indices where elevation is zero. Wait, maybe we need to model the elevation as a function of the steps. Let's think of each step as changing elevation: R is +1, L is -1. Then the elevation after each step is the cumulative sum. The zero elevation points are where the cumulative sum is zero. The problem states there are exactly three such indices, which are the black dots on the dashed line (elevation zero).

Let's try to compute the cumulative elevation for the entire sequence UD. Let's first write out the sequence of steps: R, R, R, L, L, R, L, L, L, R, L, R, L, R, R. Wait, wait, let's parse UD correctly. The first part is (RRRLLRLL) which is R, R, R, L, L, R, L, L and the second part is (LLRLRLRR) which is L, L, R, L, R, L, R, R. So combining them, the full sequence is R, R, R, L, L, R, L, L, L, R, L, R, L, R, R. Wait, let's count the steps: first part has 8 steps, second part has 8 steps, total 16 steps. Now, let's compute the cumulative elevation step by step. Starting at 0.

Step 1: R → 0 + 1 = 1
Step 2: R → 1 + 1 = 2
Step 3: R → 2 + 1 = 3
Step 4: L → 3 - 1 = 2
Step 5: L → 2 - 1 = 1
Step 6: R → 1 + 1 = 2
Step 7: L → 2 - 1 = 1
Step 8: L → 1 - 1 = 0 → first zero at step 8
Step 9: L → 0 - 1 = -1
Step 10: R → -1 + 1 = 0 → second zero at step 10
Step 11: L → 0 - 1 = -1
Step 12: R → -1 + 1 = 0 → third zero at step 12
Step 13: L → 0 - 1 = -1
Step 14: R → -1 + 1 = 0 → wait, step 14 would be zero? Wait, let's check step 14: after step 12 (which is zero), step 13 is L: -1, step 14 is R: -1 +1 = 0 → fourth zero? But the problem says exactly three indices. Hmm, maybe I made a mistake.

Wait, let's do it again carefully, step by step, keeping track of cumulative elevation:

Start at 0 (before any step).

Step 1: R → elevation 0 + 1 = 1 (index 1: 1)
Step 2: R → 1 + 1 = 2 (index 2: 2)
Step 3: R → 2 + 1 = 3 (index 3: 3)
Step 4: L → 3 - 1 = 2 (index 4: 2)
Step 5: L → 2 - 1 = 1 (index 5: 1)
Step 6: R → 1 + 1 = 2 (index 6: 2)
Step 7: L → 2 - 1 = 1 (index 7: 1)
Step 8: L → 1 - 1 = 0 (index 8: 0) → first zero
Step 9: L → 0 - 1 = -1 (index 9: -1)
Step 10: R → -1 + 1 = 0 (index 10: 0) → second zero
Step 11: L → 0 - 1 = -1 (index 11: -1)
Step 12: R → -1 + 1 = 0 (index 12: 0) → third zero
Step 13: L → 0 - 1 = -1 (index 13: -1)
Step 14: R → -1 + 1 = 0 (index 14: 0) → fourth zero? Wait, now we have four zeros. But the problem states "exactly three indices" where elevation is zero. Hmm, maybe the problem is considering the indices after the first step? Or maybe I misread the sequence.

Wait, let's check the original problem statement again: "Elevation is positive inside U, negative inside D, and zero for exactly three indices; the zero elevation locations are marked with a black dot." So maybe U and D are two separate regions, and the zero elevation points are the boundaries between U and D? Wait, UD is the combination, so maybe U is the first part, D the second. Let's consider the elevation for U first: RRRLLRLL. Let's compute elevation for U:

U: R, R, R, L, L, R, L, L. Cumulative elevation:

Start at 0.

Step 1: R → 1
Step 2: R → 2
Step 3: R → 3
Step 4: L → 2
Step 5: L → 1
Step 6: R → 2
Step 7: L → 1
Step 8: L → 0 → so U ends at elevation 0.

Then D is LLRLRLRR. Let's compute elevation for D starting from U's end, which is at 0.

D: L, L, R, L, R, L, R, R. Starting from 0.

Step 1: L → 0 - 1 = -1
Step 2: L → -1 - 1 = -2
Step 3: R → -2 + 1 = -1
Step 4: L → -1 - 1 = -2
Step 5: R → -2 + 1 = -1
Step 6: L → -1 - 1 = -2
Step 7: R → -2 + 1 = -1
Step 8: R → -1 + 1 = 0 → so D ends at elevation 0.

Now, the total sequence UD is U followed by D, so the cumulative elevation for UD is the sum of U's elevation and D's elevation. But let's track the cumulative elevation for UD as a single sequence:

Starting from 0, then following U and D:

After U (8 steps), we have elevation 0. Then start D:

Step 9: L → 0 - 1 = -1
Step 10: L → -1 - 1 = -2
Step 11: R → -2 + 1 = -1
Step 12: L → -1 - 1 = -2
Step 13: R → -2 + 1 = -1
Step 14: L → -1 - 1 = -2
Step 15: R → -2 + 1 = -1
Step 16: R → -1 + 1 = 0

Wait, so the cumulative elevations for UD are:

Index 0: 0 (start)
Index 1: 1
Index 2: 2
Index 3: 3
Index 4: 2
Index 5: 1
Index 6: 2
Index 7: 1
Index 8: 0
Index 9: -1
Index 10: -2
Index 11: -1
Index 12: -2
Index 13: -1
Index 14: -2
Index 15: -1
Index 16: 0

So the zeros are at index 8, 10, 12, 16? Wait, index 0 is the start, but maybe they consider the first step as index 1. Wait, the problem says "exactly three indices" where elevation is zero. But according to this, there are four zeros: 8, 10, 12, 16 if we include the start, but maybe the start is not counted. Wait, let's check the problem statement again: "elevation is zero for exactly three indices; the zero elevation locations are marked with a black dot." So maybe the three indices are 8, 10, 12, but that's three? Wait, 8, 10, 12 are three, and 16 is the fourth. Hmm, maybe I made a mistake in the sequence parsing.

Wait, let's try another approach. The problem says "elevation is positive inside U, negative inside D, and zero for exactly three indices". So U is the region where elevation is positive, D is where it's negative, and the boundaries between them are the zero elevation points. If there are three zero indices, that would mean two boundaries, but usually, the number of boundaries is one less than the number of regions. Wait, if U and D are two separate regions, then there should be one boundary between them, which would be a zero elevation point. But the problem says three indices, so maybe U and D are each split into sub-regions? Wait, the sequence UD is (RRRLLRLL)(LLRLRLRR), which might be two separate paths, but combined as one. Alternatively, maybe the first part is U, which is a path that goes up and then down, and the second part is D, which is a path that goes down and then up, with their intersection having zero elevation points.

Alternatively, let's consider the cumulative sum as a function h(i) = sum of first i steps. We need to find where h(i) = 0, and there are exactly three such i. From the earlier calculation, h(8)=0, h(10)=0, h(12)=0, h(16)=0. So four zeros. But the problem says three, so maybe the indices are 8, 10, 12, but 16 is the end, maybe not counted. Or perhaps the problem considers the indices within the U and D regions. Wait, U is the first part, which ends at h=0, then D starts, and D ends at h=0. So the zeros are at the start of U, end of U, start of D, end of D? Wait, start of U is step 1, end of U is step 8 (h=0), start of D is step 9, end of D is step 16 (h=0). So zeros at step 8, 16, and maybe one in between? Wait, step 10 and 12 are also zeros. Hmm, this is confusing.

Wait, let's try to visualize the path. Each R is up, L is down. The elevation is the height above the dashed line (zero). Inside U, elevation is positive, so the path is above the dashed line, then it goes down into D, which is below the dashed line, then back up to zero. The zero elevation points are where it crosses the dashed line. If there are three crossings, that would mean two times going from positive to negative and one back, or vice versa. But with the sequence given, let's count the number of times the cumulative sum changes from positive to negative or vice versa.

Starting from 0, after step 1: +1 (positive), step 2: +2 (positive), step 3: +3 (positive), step 4: +2 (positive), step 5: +1 (positive), step 6: +2 (positive), step 7: +1 (positive), step 8: 0 (zero). Then step 9: -1 (negative), step 10: -2 (negative), step 11: -1 (negative), step 12: -2 (negative), step 13: -1 (negative), step 14: -2 (negative), step 15: -1 (negative), step 16: 0 (zero). So the elevation is positive from step 1 to step 7 (7 steps), then zero at step 8, then negative from step 9 to step 15 (7 steps), then zero at step 16. So the positive region is steps 1-7, negative is steps 9-15, with zeros at 8 and 16. That's two zero points, but the problem says three. Hmm, maybe I missed one.

Wait, let's check step 10: cumulative sum is -2, step 11: -1, step 12: 0. Oh, step 12 is zero. So zeros at 8, 10, 12, 16. That's four zeros. Maybe the problem has a typo, or I'm misinterpreting the sequence. Alternatively, maybe the sequence is UD = (RRRLLRLLRLLRRLRR) or something else, but the user wrote (RRRLLRLL)(LLRLRLRR), which is 8+8=16 steps.

Wait, another approach: the number of zero elevation points is equal to the number of times the cumulative sum returns to zero. Starting from zero, the first time it goes above zero is step 1, then stays positive until step 8 where it returns to zero. Then goes negative, returns to zero at step 16. But that's two returns to zero. To have three zeros, there must be two returns: start at zero, go positive, return to zero once, then negative, return to zero again, then positive? Wait, let's do the cumulative sum again carefully:

i: 0 (start) h=0
1: R → h=1
2: R → h=2
3: R → h=3
4: L → h=2
5: L → h=1
6: R → h=2
7: L → h=1
8: L → h=0 → first zero
9: L → h=-1
10: R → h=0 → second zero
11: L → h=-1
12: R → h=0 → third zero
13: L → h=-1
14: R → h=0 → fourth zero
15: L → h=-1
16: R → h=0 → fifth zero

Wait, now I see five zeros if we include the start, but the start is h=0 before any step. If we count from step 1 to 16, the zeros are at steps 8, 10, 12, 14, 16. That's five zeros. I must be making a mistake here. The problem states "exactly three indices" where elevation is zero, so maybe the indices are 4, 8, 12? Wait, let's check step 4: h=2, not zero. Step 8 is zero, step 12 is zero, step 16 is zero. Hmm.

Wait, going back to the problem statement: "Elevation is positive inside U, negative inside D, and zero for exactly three indices; the zero elevation locations are marked with a black dot." So U is the region where elevation is positive, D is where it's negative, and the boundary between them has the zero elevation points. If there are three zero indices, that would mean two boundaries, but usually, the number of boundaries is one less than the number of regions. If U and D are two separate regions, there should be one boundary, which is a single zero index. But the problem says three, so maybe U is divided into two sub-regions by a zero, and D is divided into two sub-regions by another zero, making three total? Not sure.

Alternatively, maybe the sequence UD is a single path, and the three zero points are where it crosses the dashed line, entering and exiting U and D. For example, starting at zero, going up (U), crossing zero once to enter U, then crossing zero again to exit U, then crossing zero again to enter D, then exit D. But that would be three crossings, which are the zero points. Wait, let's think of the path as starting at zero, then moving up (R) to +1, +2, +3, then down (L) to +2, +1, 0 (first zero), then down (L) to -1, -2, then up (R) to -1, 0 (second zero), then up (R) to 0, -1, -2, then up (R) to -1, 0 (third zero), then up (R) to 0. So the crossings are at the first zero (entering U), second zero (exiting U), third zero (entering D), and then exiting D at the end. But the problem says "inside U" is positive, "inside D" is negative, so the regions are between the crossings. If there are three zero crossings, then there are four regions, but the problem says exactly three indices where elevation is zero. I'm getting confused here.

Wait, let's try to count the number of times the cumulative sum is zero. From the earlier step-by-step, the cumulative sums are: 1,2,3,2,1,2,1,0,-1,-2,-1,-2,-1,-2,-1,0. The zeros are at positions 8, 10, 12, 16 (if we count from 0 to 16). If we exclude the start (position 0) and the end (position 16), then zeros at 8, 10, 12: three zeros. That must be it. Maybe the problem