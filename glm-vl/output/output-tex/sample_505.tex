<think>Got it, let's try to figure out what the E₂-page of the Atiyah-Hirzebruch spectral sequence for an Enriques surface over F_q would look like. First, I need to recall the general structure of the Atiyah-Hirzebruch spectral sequence (AHSS) for a space X. The AHSS is a tool in algebraic topology (and algebraic geometry) that relates the homology of X with coefficients in a ring (here, probably the coefficient ring is F_q, or maybe a field, but let's think) to the cohomology of X with coefficients in the same ring. Wait, the AHSS is usually for cohomology, right? The E₂-page is given by H^p(X, R^q X) where R^q X is the q-th right derived functor of the constant sheaf with coefficients in R. If R is a field, say F_q, then R^q X is zero for q > 0, because the higher direct images of a constant sheaf on a compact space (like a surface) vanish. Wait, but an Enriques surface is a compact complex surface, so maybe we're considering cohomology with coefficients in F_q, and the AHSS would have E₂^{p,q} = H^p(X, R^q X). If R is F_q, then R^q X = 0 for q ≥ 1, because the higher direct images of a constant sheaf on a compact space vanish. So then the E₂-page would be concentrated in q=0, meaning E₂^{p,0} = H^p(X, F_q), and the rest are zero. But wait, an Enriques surface has Betti numbers: the cohomology groups are H^0 = F_q, H^1 = F_q^2, H^2 = F_q, H^3 = F_q. Wait, but the AHSS for cohomology with coefficients in F_q would have E₂^{p,q} = H^p(X, R^q X). Since R^q X = 0 for q ≥ 1, then E₂ is just the cohomology groups shifted by q=0. So E₂^{p,0} = H^p(X, F_q), and the spectral sequence collapses at E₂. But wait, maybe the question is about homology? Wait, the user mentioned "E₂-page of the Atiyah-Hirzebruch spectral sequence of Enriques surface X over F_q". The AHSS can be for either homology or cohomology. Let's confirm: the Atiyah-Hirzebruch spectral sequence for a space X with coefficients in a ring R has E₂^{p,q} = H^p(X; R^q X), where R^q X is the q-th right derived functor of the constant sheaf R on X. If R is a field, then R^q X = 0 for q > 0, so E₂ is just the cohomology groups with coefficients in R. For an Enriques surface, which is a complex surface, its cohomology with coefficients in F_q is: H^0 = F_q, H^1 = F_q^2, H^2 = F_q, H^3 = F_q. So the E₂-page would have non-zero entries only in q=0, with E₂^{p,0} = H^p(X, F_q). The differentials d_r: E_r^{p,q} → E_r^{p+r, q-r+1} would start from E_2, but if R^q X = 0 for q ≥ 1, then all differentials must be zero, so the spectral sequence collapses at E_2. Therefore, the E₂-page is just the cohomology groups with coefficients in F_q, arranged in the E₂ table with p as the first index and q=0. But wait, maybe the user is considering homology? Let's check the AHSS for homology. The homology AHSS is E₂^{p,q} = H_q(X; R^p X), where R^p X is the p-th left derived functor of the constant sheaf R on X. For a field R, R^p X = 0 for p > 0, so E₂ is H_q(X; R^0 X) = H_q(X; R), and again, the spectral sequence collapses. So in either case, if the coefficients are a field, the E₂-page is just the cohomology or homology groups with coefficients in the field. For an Enriques surface, the cohomology groups are as I mentioned, so the E₂-page would have entries: E₂^{0,0} = F_q, E₂^{1,0} = F_q^2, E₂^{2,0} = F_q, E₂^{3,0} = F_q, and all other entries zero. The differentials would be zero because there's nothing to differentiate. But maybe the question is more about the structure when considering the Enriques surface's properties, like its Picard group or the action of the Galois group? Wait, the Enriques surface is a complex surface with trivial canonical bundle, and it's a double cover of P^1 branched at 6 points, or maybe a double cover of a curve of genus 1? Wait, an Enriques surface is a K3 surface with an involution, and the cohomology with coefficients in F_q would have H^1 = F_q^2, H^2 = F_q, etc. Alternatively, if we're considering the spectral sequence for the cohomology of the surface with coefficients in F_q, the E₂-page is just the cohomology groups, as the higher derived functors vanish. So the E₂-page would have the cohomology groups arranged in the E₂ table, with p from 0 to 3 (since the surface is 2-dimensional, cohomology up to degree 3), and q=0. So the E₂ page is a 4x1 matrix (if we consider p=0,1,2,3 and q=0) with entries corresponding to the cohomology groups. But maybe the user is referring to a more specific case, like the spectral sequence for the cohomology of the surface with coefficients in a local system? Wait, but the question mentions "over F_q", which is a field, so maybe the coefficients are F_q, and the spectral sequence is for the cohomology with local coefficients? Hmm, I need to recall that the Atiyah-Hirzebruch spectral sequence for a space X with coefficients in a sheaf F is E₂^{p,q} = H^p(X, R^q F), where R^q F is the q-th derived sheaf. If F is a constant sheaf with coefficients in a field, then R^q F = 0 for q > 0, so E₂ is just the cohomology groups with coefficients in the field. Therefore, for an Enriques surface X over F_q (as a scheme over F_q), the E₂-page of the AHSS for cohomology with coefficients in F_q would have E₂^{p,0} = H^p(X, F_q), and all other E₂ terms zero. The Betti numbers for an Enriques surface are: H^0 = 1, H^1 = 2, H^2 = 1, H^3 = 1 (in terms of dimension over F_q), so the E₂-page would have entries: (1, 0), (2, 0), (1, 0), (1, 0) in the rows p=0,1,2,3 and column q=0. The differentials would start from E_2, but since there's nothing to differentiate (as R^q X = 0 for q ≥ 1), the spectral sequence collapses immediately, so E_∞ = E_2. Therefore, the E₂-page is essentially the cohomology groups with coefficients in F_q, arranged in the standard AHSS format. I think that's the main point here. The key is recognizing that for a compact space like an Enriques surface with coefficients in a field, the higher derived functors vanish, so the E₂-page is just the cohomology groups, and the spectral sequence collapses.</think><answer>To determine the \( E_2 \)-page of the Atiyah-Hirzebruch spectral sequence (AHSS) for an Enriques surface \( X \) over \( \mathbb{F}_q \), we start by recalling the structure of the AHSS: for a space \( X \) with coefficients in a ring \( R \), the \( E_2 \)-page is given by \( E_2^{p,q} = H^p(X, R^q X) \), where \( R^q X \) denotes the \( q \)-th right derived functor of the constant sheaf \( R \) on \( X \).  

For a field \( R = \mathbb{F}_q \), the higher derived functors \( R^q X \) vanish for \( q > 0 \) (since \( X \) is a compact space, higher direct images of a constant sheaf vanish). Thus, the \( E_2 \)-page simplifies to \( E_2^{p,0} = H^p(X, \mathbb{F}_q) \), with all other entries zero.  

An Enriques surface is a compact complex surface with trivial canonical bundle. Its cohomology groups with coefficients in \( \mathbb{F}_q \) are:  
- \( H^0(X, \mathbb{F}_q) = \mathbb{F}_q \),  
- \( H^1(X, \mathbb{F}_q) = \mathbb{F}_q^2 \),  
- \( H^2(X, \mathbb{F}_q) = \mathbb{F}_q \),  
- \( H^3(X, \mathbb{F}_q) = \mathbb{F}_q \).  

Since the spectral sequence collapses at the \( E_2 \)-page (as there are no non-zero differentials when higher derived functors vanish), the \( E_2 \)-page is simply these cohomology groups arranged in the standard AHSS format, with non-zero entries only in \( q = 0 \).  

**Summary**: The \( E_2 \)-page for the AHSS of an Enriques surface over \( \mathbb{F}_q \) is concentrated in \( q = 0 \), with entries corresponding to the cohomology groups \( H^p(X, \mathbb{F}_q) \), and all differentials vanish, leading to a collapsed spectral sequence.  

The answer is \(<|begin_of_box|>E_2^{p,0} = H^p(X, \mathbb{F}_q) \text{ (with } E_2^{p,q} = 0 \text{ for } q > 0\text{)}<|end_of_box|>\).</answer>