<think>Got it, let's try to unpack this statement step by step. First, let's recall some concepts from quantum gravity and statistical mechanics. The question is about how the topology of a Wick-rotated configuration (which is related to the Euclidean path integral, since Wick rotation is the process of going from Minkowski time to imaginary time, turning the path integral into a Euclidean one) can differ from the non-gravitational case, and how this affects the partition function, especially regarding black hole states.

First, let's recall Wick rotation. In quantum field theory, when you do a Wick rotation, you replace the time coordinate with an imaginary one, so the metric signature changes from (-,+,+,+) to (+,+,+,+), turning the Lorentzian (Minkowski) spacetime into a Euclidean one. The path integral then becomes a sum over Euclidean field configurations.

In the non-gravitational case (like a scalar field in flat space), the Euclidean path integral is well-behaved, and the partition function (which is the sum over all states) doesn't have issues with divergences because the topology of the Euclidean manifold is usually simple (like a sphere or a torus, depending on the boundary conditions), and there are no topological defects like black holes (which are solutions in gravity but not in non-gravitational theories).

But in gravitational theories, when you do the Wick rotation, the Euclidean manifold can have more complex topologies. For example, in asymptotically AdS spaces, the Euclidean continuation might involve a manifold with a "cylinder" topology (like S^1 × R^n), but if you have black holes, the Euclidean section might have a topology with a "handle" or a different compactification. Wait, but the statement says the topology of the Wick-rotated configuration can differ from the non-gravitational case. So maybe in the non-gravitational case, the Euclidean manifold is topologically trivial (like a sphere with no handles), while in the gravitational case, it can have non-trivial topology, like having a black hole horizon, which changes the topology of the manifold.

Now, the partition function in quantum gravity is the sum over all gravitational states, including black holes. If the topology of the Wick-rotated configuration (the Euclidean manifold) suppresses the contribution of black hole states, then the partition function doesn't diverge. How does topology do that? Maybe because black holes are topological objects (their event horizon is a topological feature), and if the Euclidean manifold's topology doesn't allow for the formation of black holes (or their contributions are suppressed due to the topology), then the sum over states doesn't include divergent terms from black holes.

Wait, another angle: in the Euclidean path integral, the topology of the manifold affects the number of states. For example, in a compact Euclidean manifold, the number of states is related to the volume of the moduli space. If the topology is such that the moduli space of black holes (which are solutions with horizons) has a smaller volume or is somehow "forbidden" by the topology, then their contribution to the partition function is suppressed. Alternatively, maybe the Wick-rotated configuration in gravity can have a topology where the black hole states are not allowed, or their action is larger due to the topology, making their contribution negligible.

Also, the partition function in quantum gravity is often divergent because of the infinite number of states (like black holes of all sizes), but if the topology of the Wick-rotated configuration (the Euclidean manifold) restricts the possible states, then the sum doesn't diverge. For example, if the Euclidean manifold is such that only certain topologies allow for certain types of states, and black holes are not part of those, then their contribution is zero or suppressed.

Let me check some concepts. In the context of Euclidean quantum gravity, the path integral is over all metrics on a Euclidean manifold. The topology of the manifold (like the number of handles, the genus) can affect the partition function. For example, in 2D Euclidean gravity, the partition function is related to the Euler characteristic, and the topology (genus) changes the result. In higher dimensions, the topology of the Euclidean section (which is a compact manifold with boundary, often a sphere with some handles) can affect the behavior of black holes. If the Euclidean manifold has a topology that doesn't allow for black hole formation (maybe because the horizon would create a non-trivial topology that's not allowed), then the black hole states don't contribute, so the partition function is finite.

Another point: Wick rotation changes the sign of the gravitational action. In Lorentzian signature, the Einstein-Hilbert action has a negative sign for the Ricci scalar term (since R is positive, the action is - (1/16πG) ∫R d^4x), but in Euclidean signature, the Ricci scalar is positive, so the action becomes positive. However, black holes have negative action (since their mass is less than the horizon area times Newton's constant, by the Bekenstein-Hawking formula), so in Euclidean signature, black holes would have negative action, making their contribution to the path integral large (since exp(-S) would be large if S is negative). But if the topology of the Euclidean manifold restricts the possible configurations, maybe black holes cannot form, or their action is modified by the topology, leading to a suppression.

Wait, the statement says "the topology of the wick rotated configuration can differ from the non-gravitational case. This suppresses the contribution of gravitational single-particle states (black holes) so that the partition function does not diverge." So the key is that in the non-gravitational case, the Wick-rotated configuration (Euclidean manifold) has a certain topology, maybe simple, allowing for certain states, but in gravity, the topology is different, which suppresses black holes. So maybe in the non-gravitational case, the Euclidean manifold is like a sphere, and the partition function includes all states, but in gravity, the Euclidean manifold has a different topology (like a manifold with a "cylinder" or a different compactification) that doesn't allow for black holes, so their contribution is zero, making the partition function finite.

Alternatively, think about the partition function as a sum over all possible states, each with weight exp(-S), where S is the action. If black holes have a large action (negative, so exp(-S) is large), but if the topology of the Euclidean manifold restricts the number of black hole states or their action, then their contribution is suppressed. For example, if the Euclidean manifold has a topology where the horizon cannot form (maybe because the topology requires the horizon to have a certain structure that's not possible), then black holes don't contribute, so the partition function is finite.

In summary, the idea is that when you Wick rotate a gravitational theory, the Euclidean manifold's topology can be more complex than in the non-gravitational case. This complex topology either restricts the formation of black holes (which are topological objects) or modifies their action, leading to a suppression of their contribution to the partition function. As a result, the partition function, which would otherwise diverge due to infinite black hole states, becomes finite.</think><answer>To understand the statement, we analyze the role of Wick rotation and topology in gravitational theories:  

1. **Wick Rotation and Euclidean Path Integral**:  
   Wick rotation converts the Lorentzian (Minkowski) spacetime to a Euclidean one by replacing time with imaginary time. This changes the metric signature from \((-,+,+,+)\) to \((+,+,+,+)\), turning the path integral into a sum over Euclidean field configurations.  

2. **Topological Differences Between Gravitational and Non-Gravitational Cases**:  
   In non-gravitational theories (e.g., scalar fields in flat space), the Euclidean manifold typically has simple topology (e.g., a sphere or torus). In gravitational theories, the Wick-rotated configuration (Euclidean spacetime) can have more complex topology, such as manifolds with handles or non-trivial compactifications. This difference arises because gravity introduces topological features like black hole horizons, which alter the manifold’s structure.  

3. **Suppression of Black Hole Contributions**:  
   Black holes are "single-particle" states in gravitational theories, with contributions to the partition function proportional to \(e^{-S}\) (where \(S\) is the Euclidean action). If the Euclidean manifold’s topology restricts black hole formation (e.g., by preventing horizon topology or modifying the action), their contribution is suppressed. For example, if the topology does not allow for the formation of horizons, black holes cannot exist in the Euclidean path integral, eliminating their divergent contribution.  

4. **Partition Function Finiteness**:  
   Without black hole contributions, the partition function (sum over all states) avoids divergence. The complex topology of the Wick-rotated gravitational configuration effectively "forbids" or suppresses black hole states, ensuring the partition function remains finite.  

In summary, the topology of the Wick-rotated gravitational configuration differs from the non-gravitational case by introducing constraints that suppress black hole contributions, thereby preventing the partition function from diverging.  

The answer is \(<|begin_of_box|>\text{The topology of the Wick-rotated configuration suppresses black hole contributions, making the partition function finite}<|end_of_box|>\).</answer>