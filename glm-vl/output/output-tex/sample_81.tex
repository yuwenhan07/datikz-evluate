<think>Got, let's try to figure out this problem. First, the question is about partitioning an enlarged annulus PQu into parts P, Q₁, A(PQ₁), B, and C. Wait, maybe I need to recall what an annulus is—an annulus is a ring-shaped object, like the area between two concentric circles. So an enlarged annulus PQu probably has some labels related to the circles. The partitioning means dividing the annulus into those five parts: P, Q₁, A(PQ₁), B, C. 

First, let's parse the labels. PQu might be the original annulus, and when enlarged, it's partitioned. The parts are P, Q₁ (maybe a modified Q), A(PQ₁) which could be a region between P and Q₁, B and C as other regions. Since it's an annulus, which is between two circles, maybe P and Q are the two circles (inner and outer), and Q₁ is a new circle, so the annulus is between P and Q, then after enlargement, maybe between P and Q₁? Wait, the problem says "enlarged annulus PQu"—maybe P and Q are the original circles, and after enlargement, the outer circle is Q, and the inner is P, with Q₁ being a new circle? Hmm, maybe the partitioning creates regions: P is a part, Q₁ is another, A(PQ₁) is the area between P and Q₁, then B and C as other parts. 

Since the image has circles labeled A, B, C, Q, maybe the partitioning corresponds to those labels. The annulus is between two circles, say with centers at the same point. If we partition it into P (maybe a sector or a smaller annulus), Q₁ (another circle), A(PQ₁) the area between P and Q₁, B and C as other regions. But without the exact diagram, we need to think about standard partitioning of an annulus. An annulus can be divided into sectors, but here the partitioning includes P, Q₁, A(PQ₁), B, C. Maybe P is a smaller circle (inner), Q is the outer, Q₁ is a circle between P and Q, so the annulus is between P and Q, then after enlargement, maybe the outer is Q, inner is P, and Q₁ is a new circle, so the partitioning creates regions: P (maybe a sector), Q₁ (another circle), the area between P and Q₁ (A(PQ₁)), then B and C as other parts. 

Alternatively, since the problem mentions "enlarged annulus PQu", maybe P and Q are the original circles, and after enlargement, the new outer circle is Q, and the inner is P, with Q₁ being a circle in between. The partitioning would split the enlarged annulus into P (a region), Q₁ (a circle), the area between P and Q₁ (A(PQ₁)), then B and C as other regions. But without the exact diagram, we need to consider that the partitioning is into five parts: P, Q₁, A(PQ₁), B, C. So the key is to identify how these parts relate in the annulus. 

An annulus is a region between two concentric circles. If we have an enlarged annulus, maybe the outer radius is increased, making the area larger. Partitioning it would involve dividing the area into sectors or smaller annuli. If P is a smaller circle (inner), Q is the outer, Q₁ is a circle between P and Q, then the regions could be: P (a sector), Q₁ (a circle), the area between P and Q₁ (A(PQ₁)), then B and C as other sectors or regions. But since the problem says "partition of an enlarged annulus PQu into P, Q₁, A(PQ₁), B, and C", each of these is a part of the partition. So the partitioning divides the enlarged annulus into these five regions. 

Assuming that the annulus is between two circles, say with center O, then P and Q are circles with radii r and R, respectively. Enlarging the annulus might mean increasing R, making the annulus between r and R'. Q₁ could be a circle with radius r' between r and R', so the partitioning includes the circle P (radius r), Q₁ (radius r'), the area between P and Q₁ (which is a smaller annulus), then B and C as other regions. But without the diagram, it's a bit tricky, but the key is that the partitioning is into five parts, and the question might be asking for the relationship or the nature of the partition, but since the user probably expects a description or the partitioning process, but given the initial problem statement, maybe the answer is about identifying the regions as parts of the annulus. 

Wait, the question is "Partition of an enlarged annulus PQu into P, Q₁, A(PQ₁), B, and C." So the partitioning is done by dividing the enlarged annulus into these five components. The image has a diagram with circles labeled A, B, C, Q, so maybe the labels correspond: P is a circle, Q₁ is another, A(PQ₁) is the area between them, B and C are other regions. But since the problem is about partitioning, the answer might involve stating that the enlarged annulus is divided into these five parts, each being a region (like sectors or annular regions) connected by the labels. However, without more specific details from the diagram, the best approach is to describe the partitioning as dividing the enlarged annulus into the five specified regions, with each being a distinct part of the annulus, possibly sectors or annular segments.</think><answer>The problem involves partitioning an enlarged annulus labeled \( PQu \) into five distinct regions: \( P \), \( Q_1 \), \( A(PQ_1) \), \( B \), and \( C \). An annulus is a ring-shaped region between two concentric circles, so the enlarged annulus likely has an inner circle \( P \) and outer circle \( Q \) (or modified labels like \( Q_1 \) after enlargement). The partitioning divides the annulus into these regions, where \( P \) and \( Q_1 \) are likely circles (or sectors), \( A(PQ_1) \) is the annular region between \( P \) and \( Q_1 \), and \( B \), \( C \) are additional sectors or regions within the enlarged annulus. The diagram with interconnected circles and arrows suggests a schematic representation of these regions and their connections, with labels \( A, B, C, Q \) corresponding to the partitioned parts.  

The answer is \(<|begin_of_box|>5<|end_of_box|>\).</answer>