<think>Got it, let's try to parse this description. The user is talking about encircled elements on the same cycle of some group, maybe a cyclic group? The mention of _0^-1_1 might be a notation for a cycle, like in group theory or graph theory. The image has a white background with equations, three columns (red, white, black backgrounds) with numbers and variables. 

First, let's identify the key elements: encircled elements on the same cycle, the group notation _0^-1_1 (maybe a cycle labeled with 0, inverse, 1?), columns with different colors, equations in columnar format. The question might be asking about the structure, but since the user provided a description, maybe they want an explanation or analysis.

Wait, the user's message is a bit of a description, maybe a question about the image's mathematical structure. Let's break down the elements:

1. Encircled elements lie on same cycle of _0^-1_1: This likely refers to a cyclic group or a cycle in a graph, where elements are part of a cycle labeled with 0, inverse (maybe a generator inverse), 1 (identity or base element). So the cycle is a cyclic structure with elements connected in a cycle, and the encircled ones are part of that cycle.

2. Image features: white background, equations, three columns (red, white, black backgrounds) with numbers/variables. Columnar format, organized.

So, if we need to explain the mathematical concept, maybe the cycle is a cyclic group, and the encircled elements are part of a single cycle in that group's structure. The columns might represent different components or representations, with colors distinguishing them.

But since the user's input is a description, maybe they want a summary or an explanation of the mathematical elements. Let's structure the thoughts:

- The key mathematical concept is "cycle" in a group (cyclic group) or graph. Encircled elements are part of the same cycle, so they form a closed loop under some operation (like multiplication in a group, or adjacency in a graph).

- The notation _0^-1_1 could be a way to label the cycle, maybe with elements 0, inverse of some element, and 1 (like identity), so the cycle includes these elements.

- The image's layout: three columns with different background colors (red, white, black) for organization, equations in each column, possibly representing different aspects of the mathematical structure (like group elements, operations, or graph edges).

So, putting it all together, the encircled elements are part of a cyclic structure (cycle) in a mathematical object (group, graph, etc.), labeled with _0^-1_1, and the image's columnar format with colored backgrounds helps organize the equations and symbols related to this structure.</think><answer>To address the description, let’s break down the key mathematical and visual elements:  

### Mathematical Structure  
- **Encircled elements on the same cycle of _0^-1_1**: This likely refers to a *cyclic group* or a *cycle in a graph* (a closed loop of elements). The notation “_0^-1_1” probably labels the cycle, with “0” (possibly an identity or base element), “-1” (an inverse of an element), and “1” (another base/identity element) being part of the cycle. Encircled elements are part of this closed loop, meaning they are connected in a cyclic manner under some operation (e.g., group multiplication, graph adjacency).  

### Visual Layout  
- The image has a **white background** with a **columnar format** (three columns) for equations and symbols.  
- Columns have distinct background colors:  
  - First column: Red background.  
  - Second column: White background.  
  - Third column: Black background.  
- The equations and symbols are organized in a clean, structured way, likely to distinguish different components of the mathematical structure (e.g., group elements, operations, or graph edges).  


In summary, the encircled elements form a cyclic structure (labeled “_0^-1_1”) in a mathematical object (group/graph), and the image uses color-coded columns to organize related equations and symbols. The “cycle” here refers to a closed loop of elements under a defined operation, with the encircled elements belonging to the same such loop.</answer>