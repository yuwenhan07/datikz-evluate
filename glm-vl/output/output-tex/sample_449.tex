<think>Got it, let's try to tackle this problem step by step. First, let's understand what's being asked. The question is about a pictorial representation of "Col" from the receiver's point of view, with dashed arrows for conditional steps. There's a whiteboard with a black arrow pointing right, labeled with mathematical symbols showing a relationship between x, y, z: if x and y are odd, then z is also odd.

First, let's parse the components. The key elements are the conditional relationship (if x and y are odd, then z is odd), the pictorial representation (maybe a diagram with nodes and arrows), dashed arrows for conditional steps, and the whiteboard with the arrow and symbols.

First, let's recall that a conditional statement like "if A then B" can be represented in logic as A → B, which is equivalent to ¬A ∨ B. In terms of a diagram, maybe nodes for x, y, z, with arrows indicating the relationship. The receiver's point of view might mean the diagram is showing the flow from inputs (x, y) to output (z), with conditions.

The dashed arrows represent conditional steps, so maybe the arrow from x and y to z is dashed, indicating the condition. The black arrow pointing right on the whiteboard—maybe the arrow is part of the diagram, showing the direction of the relationship. The mathematical symbols would likely include the implication symbol, maybe something like x ∧ y → z, where ∧ is "and" and → is "if...then".

Let's structure the pictorial representation. Maybe a diagram with three nodes: x, y, z. There's a dashed arrow from x and y (connected, maybe with an "and" condition) to z, indicating that when x and y are both odd (the condition), z becomes odd. The black arrow pointing right could be the main arrow showing the direction of the relationship, maybe from the condition (x, y) to z.

Now, translating the conditional statement "if x and y are odd, then z is odd" into a logical expression: (x is odd ∧ y is odd) → z is odd. In terms of a diagram, the dashed arrow would connect the "x is odd" and "y is odd" conditions to "z is odd", with the arrow being dashed to indicate the conditional nature.

The whiteboard with the black arrow labeled with symbols—probably the implication symbol, maybe written as x ∧ y → z, with the arrow pointing right, indicating the direction of the implication. The dashed arrows would be part of the diagram showing the conditional steps from the inputs (x, y) to the output (z) under the condition.

So, putting it all together, the pictorial representation would have nodes for x, y, z, a dashed arrow from x and y (with an "and" condition) to z, indicating the conditional relationship, and a black arrow on the whiteboard pointing right with the mathematical symbols representing the implication. The key is to show the conditional steps (dashed arrows) and the relationship between the variables as described.</think><answer>To address the problem, we analyze the conditional relationship and pictorial representation:  

1. **Conditional Relationship**: The statement “if \( x \) and \( y \) are odd, then \( z \) is odd” is a logical implication: \( (x \text{ is odd} \land y \text{ is odd}) \rightarrow z \text{ is odd} \).  
2. **Pictorial Elements**:  
   - **Nodes**: Represent \( x \), \( y \), and \( z \) as distinct points.  
   - **Dashed Arrows (Conditional Steps)**: A dashed arrow connects the “\( x \) is odd” and “\( y \) is odd” nodes (via an “and” condition) to the “\( z \) is odd” node, indicating the conditional relationship.  
   - **Black Arrow on Whiteboard**: A solid (or black) arrow pointing right labels the implication with mathematical symbols (e.g., \( x \land y \rightarrow z \)), showing the direction of the relationship.  

The pictorial representation visually encodes the conditional logic, with dashed arrows for the conditional steps and the whiteboard arrow highlighting the implication between \( x, y, z \).  

The answer is \(<|begin_of_box|>x \land y \rightarrow z (if\ x\ and\ y\ are\ odd,\ then\ z\ is\ odd)<|end_of_box|>\).</answer>