<think>Got it, let's try to understand this problem step by step. First, the question is about modifying a quantum circuit or a geometric structure, maybe related to quantum error correction or topological quantum computing, where circuits are represented as diagrams with gates (like unitary operations) and regions (like annuli or holes). 

Starting with part (a): "We first remove gates from U on a ``hole" within A; we call the new circuit U'." So, we have a circuit U, and there's a hole (maybe a region without gates, or a region where gates are removed) inside another region A. Removing gates from U on that hole gives U', so U' is U with some gates removed from the hole area. 

Part (b): "We then deform A to A' A such that the boundary of the annulus A'BC is only partially covered." Hmm, deforming A into A'... maybe A is an annulus, and deforming it changes its shape so that the boundary of the new annulus A'BC is only partially covered by some other structure (like a boundary with gates or a region). The term "annulus" suggests a region between two concentric loops, so deforming A (maybe stretching or changing the shape) to A' such that the boundary of the annulus A'BC is only partially covered. So the boundary, which might have been fully covered before, now has some parts not covered, maybe by gates or another region.

Part (c): "We then further remove some of the gates in the vicinity of A', obtaining U''." So after deforming A to A', we remove more gates near A', resulting in U''. 

I need to make sure I'm interpreting the steps correctly. The process seems to be a sequence of modifying a circuit (U) by first removing gates from a hole in region A, then deforming region A to A' such that an annulus boundary is partially covered, then removing more gates near A', leading to U''. 

Maybe this is related to simplifying a circuit by removing gates in specific regions, possibly to make the circuit more compact or to apply certain theorems about circuit equivalence (like using the fact that certain gates can be removed without changing the unitary, or using properties of annular regions in quantum circuit diagrams). 

Since the question is presented as a description of steps (a), (b), (c), maybe the task is to explain what each step does or to recognize the process, but since the user says "put your final answer within \boxed{}", maybe it's a conceptual question where the final answer is a summary or a key point. But without more context, it's hard to tell. However, given the steps are about modifying a circuit by removing gates and deforming regions, perhaps the key is to describe the process, but since the question is in Chinese (the original might be, but the user's question is in English), maybe the answer is a statement of the process or a conclusion. 

Wait, the user might be asking for an explanation of these steps, but the way the question is phrased is a bit like a problem statement with parts (a), (b), (c), each describing a step. Maybe the question is asking for a summary or to state the result after these steps, but without more context, it's tricky. However, given the structure, perhaps the final answer is a description of the resulting circuit U'' after these steps, but since the question is more about the process, maybe the answer is a statement that these steps are part of a circuit simplification or deformation process, and the final circuit U'' is obtained by these operations. 

Alternatively, if this is related to a specific theorem or concept, like in quantum circuit equivalence, where certain deformations and gate removals preserve the unitary, then the final circuit U'' is equivalent to the original U up to the removed gates. But without more context, it's hard to be precise. However, given the steps are sequential, removing gates, deforming, removing more gates, the final circuit U'' would be a simplified version with gates removed near A', after the deformation. 

Since the user might expect a summary of the process or a conclusion, but given the instructions say to put the final answer in a box, maybe the answer is a statement like "The final circuit U'' is obtained by removing gates from U on a hole in A, deforming A to A' with a partially covered annulus boundary, and further removing gates near A'." But that's too vague. Alternatively, if this is a specific problem where the answer is a particular result, but without more info, it's hard. Wait, maybe the question is from a textbook or a specific context where these steps are part of a proof, and the final answer is a key result, like U'' is equivalent to U, or a specific form. But since I don't have that context, I'll have to make an educated guess. 

Alternatively, the question might be asking for the final circuit after these steps, but since the steps are about modifying U, the final answer could be U'', which is the circuit after all modifications. But the question is probably asking for an explanation of the process, but the user says "put your final answer within \boxed{}", which usually is a concise answer, maybe a statement or a formula. Given the lack of specific details, I'll assume that the process results in a simplified circuit, and the final answer is a description of U'' as the circuit after all steps, but since I need to box it, maybe just state the final circuit is U''. But that's not helpful. 

Wait, maybe the question is about a specific example, like removing gates in a hole, deforming an annulus, and removing more gates, leading to a circuit with certain properties. But without more info, I'll have to proceed with the understanding that the steps describe a process, and the final answer is the resulting circuit U'', which is obtained through these steps. So, the final answer is U'', boxed.</think><answer>The process involves modifying a circuit \( U \) through three sequential steps: first removing gates from a hole within region \( A \) to obtain \( U' \), then deforming region \( A \) into \( A' \) such that the boundary of the annulus \( A'BC \) is partially covered, and finally removing additional gates near \( A' \) to obtain \( U'' \). The final circuit after these modifications is \( U'' \), which results from the combined effects of gate removal and geometric deformation.  

The answer is \(<|begin_of_box|>U''<|end_of_box|>\).</answer>