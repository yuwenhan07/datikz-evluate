<think>Got it, let's try to figure out this problem. The user wants symmetric blend modes, meaning the order of drawing two semi-transparent regions doesn't affect the intersection's color. Also, they want to exclude the exclude blend mode, and the result should still show the intersection. 

First, let's recall what blend modes do. Common ones like multiply, screen, overlay, etc. But we need symmetry, so if we have layer A and layer B, blending A over B should be the same as B over A. 

Let's think about mathematical properties. A symmetric blend mode would satisfy f(A, B) = f(B, A). So we need a blend mode function that's commutative. 

Let's consider the color components. For each color channel (R, G, B), the blend mode should combine them in a way that's commutative. Let's look at standard blend modes:

- Multiply: (A * B) / 255, but wait, actually, multiply is A * B, but maybe normalized. Wait, in practice, blend modes often use (A * B) / (255 * 255) or similar, but maybe the key is the mathematical operation. 

Wait, let's think of the intersection as the overlap. If we have two semi-transparent layers, the intersection's color should be the combination of both, and the order shouldn't matter. So maybe a blend mode that's based on the minimum or maximum, but wait, minimum (darken) and maximum (lighten) are not commutative? Wait, no, minimum is commutative: min(A,B) = min(B,A), maximum is also commutative. But do those denote intersection? Maybe not directly. 

Wait, another approach: the intersection of two sets is commutative (A ∩ B = B ∩ A), so maybe a blend mode that represents the intersection's color. But how to translate that into a blend mode. 

Alternatively, think about the formula for blend modes. For two layers with opacity, the blend mode formula is usually something like (A * opacityA + B * opacityB * (1 - opacityA)) * blendFunction, but maybe the blend function itself needs to be commutative. 

Wait, let's consider the "lighten" or "darken" modes, but those are not symmetric in the way of intersection. Wait, the "average" blend mode? Let's check: (A + B)/2. Is that commutative? Yes, (A + B)/2 = (B + A)/2. But does that denote intersection? Maybe not directly. 

Wait, the "soft light" or "hard light" modes are not commutative. What about "overlay"? Overlay is a bit complex, but maybe not commutative. 

Wait, let's think of the mathematical operation that's commutative and represents the intersection. The intersection of two colors could be a color that's a combination of both, but symmetric. 

Another angle: the problem says "excluding blend modes like exclude". The exclude mode is exclusive or, which is not commutative (A exclude B is not the same as B exclude A, unless A = B). So we need to avoid that. 

Let's consider the "multiply" blend mode. Multiply is A * B, which is commutative (A*B = B*A). Does that represent intersection? Maybe, but how does the color look? If you multiply two semi-transparent layers, the result is darker, which might represent the intersection's color. But does it still denote the intersection? Maybe. 

Wait, let's test with two colors. Suppose layer A is red (255, 0, 0) with 50% opacity, layer B is green (0, 255, 0) with 50% opacity. If we multiply them, each channel is (255*0, 0*255, 0*0) = (0, 0, 0), which is black. If we do B then A, same result. So that's commutative. But does that denote the intersection? The intersection of red and green is black (no overlap), but maybe that's not the best example. Wait, if both layers are the same color, say red with 50% opacity, then multiply would be red * red = red, which is the same as doing it either way. 

Alternatively, "screen" mode is A * (1 - B) + B * (1 - A) + A * B? Wait, no, screen is (255 - (255 - A)*(255 - B))/255, which is commutative? Let's check: screen(A,B) = screen(B,A). Yes, because the formula is symmetric. But does screen denote intersection? Maybe not directly. 

Wait, the key is that the blend mode should represent the intersection, so maybe a mode that's like the minimum of the two colors, but in terms of opacity. Wait, the "darken" mode is min(A,B), which is commutative. If you darken A over B, it's min(A,B), darken B over A is also min(A,B). So that's symmetric. Does darken denote intersection? Maybe, because the intersection would be the darker part. But does the result still be recognizable as intersection? Maybe, but is there a better one? 

Wait, another blend mode: "color burn" or "color dodge" are not commutative. "linear burn" is commutative? Let's check. Linear burn is min(A + B - 255, 0), which is commutative. But does that denote intersection? 

Wait, let's think of the problem again: the intersection of two semi-transparent regions. The color of the intersection should be the same regardless of drawing order, and it should denote the intersection. So maybe a blend mode that combines the two colors in a way that's the intersection's color. 

Wait, the "intersect" blend mode? But that's not a standard blend mode. Wait, in some graphics software, there's a "intersect" mode, but maybe not. Alternatively, think of the formula for the intersection's color. If you have two layers, the intersection's color is the color that's present in both, so maybe the blend mode should be the minimum of the two colors, or the product, or something. 

Wait, let's consider the mathematical property of commutativity. The blend mode function f must satisfy f(A, B) = f(B, A) for any A, B. So let's list common blend modes and check commutativity:

- Normal: f(A,B) = A, which is commutative (A = B), but doesn't denote intersection.
- Dissolve: depends on random pixels, not commutative.
- Darken: f(A,B) = min(A,B) → commutative.
- Multiply: f(A,B) = A * B (in terms of color channels) → commutative.
- Screen: f(A,B) = A * B' + B * A' (where A' = 1 - A), which is commutative.
- Overlay: complex, not commutative.
- Soft light: not commutative.
- Hard light: not commutative.
- Color dodge: not commutative.
- Color burn: not commutative.
- Linear dodge (add): not commutative.
- Linear burn: commutative (min(A + B - 255, 0)), commutative.
- Difference: f(A,B) = |A - B|, which is commutative (|A-B|=|B-A|), but does it denote intersection? Maybe not, as difference is more about contrast.
- Exclusion: f(A,B) = A + B - 2AB, which is not commutative (unless A=B), so exclude is out.
- Hue, Saturation, Color, Luminosity: these are commutative? Hue blend mode is commutative, but does it denote intersection? Maybe not directly.

So among the commutative blend modes, darken, multiply, screen, linear burn are candidates. Now, which one denotes the intersection? The intersection of two regions would be the area where both are present, so the color should be a combination of both, but symmetric. 

If we use multiply, the result is darker, which could represent the intersection (since both layers are overlapping, the color is a product, which is darker). Darken is also darker, but maybe more extreme. Screen is lighter, which might not denote intersection. 

Wait, let's think of an example. Two semi-transparent circles: one red, one blue. The intersection is where they overlap. If we use multiply, the overlapping area is purple (red * blue), but if the opacity is 50%, maybe the result is darker purple. If we draw red first then blue, or blue then red, the result is the same. Does that denote the intersection? Maybe, as the overlapping area is a darker color. 

Alternatively, darken mode would make the overlapping area darker, maybe black if both are dark enough. But does that still be recognizable as intersection? Maybe, but is there a better one? 

Wait, the problem says "excluding blend modes like exclude", so exclude is out. Among the commutative ones, darken, multiply, screen, linear burn. Let's check if any of these are standard blend modes that denote intersection. 

In some contexts, "multiply" is used for shadows, which is the intersection of a light source and an object, so maybe that's a way. "Darken" is also used for shadows. But the question is about the intersection's color not depending on order, and still being recognizable as intersection. 

Another thought: the intersection of two sets is commutative, so a blend mode that's based on the intersection operation. If we model the color as a set, the intersection's color could be the color that's common to both, which in terms of blend modes might be the minimum (darken) or product (multiply). 

Given that multiply is commutative and represents a product (which could be the intersection's color), and darken is also commutative, but maybe multiply is more appropriate for representing the intersection's color as a combination of both, rather than just the darker one. 

Wait, let's test with two identical layers. If you have a red layer with 50% opacity, multiply it with itself, you get red (since 255*255 = 65025, divided by 255*255 is 1, so 255*255/255*255 = 1, so 255*255/(255*255) = 1, wait, no, multiply blend mode is (A * B) / 255^2, so (255*255)/255^2 = 1, so the color is red. If you darken it, you'd get 127.5, which is darker red. But if the layers are different, say red and green, multiply gives black, darken also gives black. 

Hmm. Alternatively, the "intersect" blend mode, if it existed, would do this, but since it's not a standard one, we need to pick from existing ones. Among the commutative ones, multiply, darken, screen, linear burn. 

The question says "excluding blend modes like exclude", so exclude is out. Now, which of the remaining commutative modes is best for denoting intersection? 

If we think of the intersection as the area where both layers overlap, the color should be a combination of both, so maybe "multiply" is better because it combines the two colors (like the intersection's color is the product), whereas darken is just the darker one. But I need to confirm if multiply is considered to denote intersection. 

Alternatively, "soft light" is not commutative, so no. "Overlay" is not commutative. "Linear dodge" is not. 

Wait, let's check the mathematical definition of commutativity for each:

- Darken: f(A,B) = min(A,B) → commutative.
- Multiply: f(A,B) = A * B (per channel) → commutative.
- Screen: f(A,B) = A * (1 - B) + B * (1 - A) + A * B? Wait, no, screen is (A + B - A*B)/255, which is commutative.
- Linear burn: f(A,B) = min(A + B - 255, 0) → commutative.
- Difference: f(A,B) = |A - B| → commutative.
- Hue, Saturation, Color, Luminosity: commutative.

Now, which of these is most likely to denote the intersection? The intersection of two sets is the area where both are present, so the color should be a color that's present in both, which in terms of color channels would be the minimum (darken) or the product (multiply). 

If we use darken, the intersection's color is the darker of the two, which could represent the overlap. If we use multiply, it's the product, which could represent the combined color of the overlap. 

Since the problem says "the color of the intersection of two semi-transparent regions does not depend on the order", both darken and multiply are commutative. But which one is more appropriate for denoting intersection? 

In graphic design, "multiply" is often used for creating shadows, which are the intersection of a light source and an object, so the intersection's color is the product of the two. "Darken" is also used for shadows, but maybe "multiply" is more specific to intersection. 

Alternatively, if we consider that the intersection's color should be a color that's a combination of both layers, multiply combines them, while darken just takes the darker one. 

Given that the question says "excluding blend modes like exclude", and among the commutative ones, multiply is a common blend mode that's commutative and could represent the intersection's color. 

Wait, but let's think of another angle. The problem says "the result should still be recognizable as denoting the intersection". If we use "multiply", the overlapping area is darker, which is recognizable as the intersection. If we use "darken", it's also darker, but maybe "multiply" is more distinct. 

Alternatively, "color burn" is not commutative, so no. "Linear burn" is commutative, but maybe not as common. 

After considering all, the most likely answer is "multiply" or "darken", but since multiply is more about combining colors (like intersection) and is commutative, and darken is about taking the darker, which is also commutative. But let's check standard blend modes. 

Wait, in Photoshop, the "Multiply" blend mode is described as "Multiplies the base color with the blend color. The result is a darker color." The "Darken" mode is "Choose the darker color between the base color and the blend color as the result color." Both are commutative. 

Which one denotes intersection better? The intersection is the area where both are present, so the color should be a color that's present in both, which would be the minimum (darken) or the product (multiply). If the layers are semi-transparent, the intersection's color would be a combination, so multiply might be better. But I'm a bit torn. 

Wait, let's think of the intersection as the set of points where both layers overlap. The color at each point is the color of the intersection, which should be the same regardless of drawing order. So the blend mode must be commutative. Among the commutative blend modes, "multiply" and "darken" are the most likely candidates. 

Since the question says "excluding blend modes like exclude", and "exclude" is an exclusive or, which is not commutative, so we need to pick from the rest. If we consider that "multiply" is a standard blend mode that's commutative and represents a combination (like intersection), then that's the answer. Alternatively, "darken" is also commutative. 

Wait, let's do a quick check with an example. Suppose layer A is a red circle (255,0,0) with 50% opacity, layer B is a green circle (0,255,0) with 50% opacity. If we multiply them, the overlapping area is (255*0, 0*255, 0*0) = (0,0,0) which is black. If we darken them, it's also black. If we use screen, it's (255*0, 0*255, 0*0) = (0,0,0) as well? Wait, screen is (A + B - A*B)/255, so (255 + 0 - 0)/255 = 255/255 = 1, (0 + 255 - 0)/255 = 255/255 = 1, (0 + 0 - 0)/255 = 0. Wait, no, screen for red and green would be (255 + 0 - 0)/255 = 1 (white), (0 + 255 - 0)/255 = 1 (white), (0 + 0 - 0)/255 = 0 (black). Wait, that's confusing. Maybe I made a mistake. 

Wait, screen blend mode formula is (A * (1 - B) + B * (1 - A) + A * B) / 255? No, the correct formula for screen is (255 - (255 - A)*(255 - B))/255. So for A=255 (red), B=255 (green), it's (255 - 0)/255 = 1 (white). For A=128, B=128, it's (255 - (127*127))/255 ≈ (255 - 16129)/255? Wait, no, 127*127 is 16129? Wait, 127*127 = 16129? Wait, 127*100=12700, 127*27=3429, total 12700+3429=16129. Then 255 - 16129 = negative, which doesn't make sense. I must have the formula wrong. 

The correct screen formula is (A * B) / (255 * 255) * 255? No, standard screen is (A + B - A*B)/255. Let's try A=128, B=128: (128 + 128 - 128*128)/255 = (256 - 16384)/255 ≈ -16128/255 ≈ -63.3, which is not right. I think I messed up the formula. The correct screen blend mode formula is (A * (1 - B) + B * (1 - A) + A * B) / 255? No, the correct formula is: for each channel, the screen blend is (A * B) / (255 * 255) * 255? No, I think the correct formula is screen(A, B) = 255 - (255 - A)(255 - B)/255. So for A