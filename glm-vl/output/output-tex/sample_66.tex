<think>Got it, let's try to figure out this problem step by step. First, the question is about choosing strings associated with anyons to simulate abelian exchange statistics, specifically mentioning phases of /2 for vertical and horizontal parts for semions. Wait, maybe I need to recall what abelian anyons and exchange statistics are. 

Abelian anyons have exchange statistics described by a phase factor when exchanging two anyons, which is a phase in the complex plane, like e^(iθ), and for non-abelian, it's more complicated. To simulate abelian exchange, maybe using semions or other anyons with specific string phases. 

The problem says "the phases associated to the anyons' strings are π/2 for the vertical and for the horizontal parts for semions." Wait, maybe the strings have different phases depending on their orientation (vertical/horizontal), and for semions, those phases are π/2. But how does this relate to abelian exchange? 

Abelian exchange usually has a phase that's a multiple of 2π, but maybe when you have strings with certain phases, the total phase when exchanging two anyons is a phase that's abelian. For example, if the string phase for vertical is π/2 and horizontal is π/2, then when you form a loop (exchanging two anyons), the total phase would be the sum around the loop. If the loop has vertical and horizontal parts, the total phase would be (π/2) + (π/2) = π, but wait, maybe that's for semions. Wait, semions have exchange statistics with phase π, which is abelian (since it's a phase, not a matrix). But the question is about choosing strings to simulate abelian exchange, so maybe the strings need to have phases that when combined give a phase that's a multiple of 2π, but maybe for semions, the string phases are π/2 in each direction, so when you have a loop, the total phase is 2*(π/2) = π, which is abelian. 

Wait, let's think about the general idea: to simulate abelian exchange, the anyons' strings should have phases that when you exchange two anyons (i.e., connect their strings and form a loop), the total phase is a phase factor e^(iθ) where θ is a multiple of 2π, but maybe θ is π, which is still abelian. The key is that the string phases depend on the path, but for abelian anyons, the exchange phase is independent of the path, so the string phases should be such that the total phase around any loop is a constant. If the vertical and horizontal string phases are π/2, then a loop going around a square (with two vertical and two horizontal parts) would have total phase 2*(π/2) + 2*(π/2) = 2π, which is e^(i2π)=1, but wait, maybe for semions, the exchange is π, so a loop around them would have phase π. Hmm, maybe the strings for semions have a phase of π/2 for each segment, so when you have a loop enclosing two semions, the total phase is 2*(π/2) = π, which is the exchange phase. But the question is about choosing strings to simulate abelian exchange, so maybe the strings need to have phases that are such that the exchange phase is a multiple of 2π, but semions have π, which is still abelian. 

Alternatively, maybe the strings are chosen such that when you have two anyons, their strings have phases that when you connect them (exchange), the total phase is e^(iθ), and for abelian, θ is a phase, not a matrix. So if the vertical string has phase π/2 and horizontal has π/2, then moving a string around a vertical loop would give π/2, but maybe for abelian, you need the phase to be 2π, so maybe the strings need to have phase π for each direction, but the question says π/2. Wait, the question states: "the phases associated to the anyons' strings are π/2 for the vertical and for the horizontal parts for semions." So semions have string phases of π/2 in both vertical and horizontal directions. Then, when you exchange two semions, you form a loop with vertical and horizontal parts. The total phase would be the sum of the phases along the loop. If the loop has, say, two vertical segments and two horizontal segments, each with π/2, then total phase is 2*(π/2) + 2*(π/2) = 2π, which is e^(i2π)=1, but that would be trivial. Wait, maybe a smaller loop, like a square with one vertical and one horizontal, total phase π/2 + π/2 = π, so the exchange phase is π, which is abelian. So the choice of strings with π/2 phases in each direction leads to an exchange phase of π, which is abelian. Therefore, the strings associated with semions (which have these phases) can simulate abelian exchange statistics with a phase of π.</think><answer>To simulate abelian exchange statistics using anyons with specific string phases, the key is to ensure the exchange phase is a single complex phase factor (not a matrix), which is characteristic of abelian anyons. For semions, the strings associated with them have a phase of \( \pi/2 \) for both vertical and horizontal segments. When exchanging two semions, the total phase around a loop formed by these strings is the sum of the phases along the loop. For a loop enclosing two semions with one vertical and one horizontal segment, the total phase is \( \pi/2 + \pi/2 = \pi \), which is an abelian phase factor (\( e^{i\pi} = -1 \), a multiple of \( 2\pi \) in the complex plane). This choice of string phases ensures the exchange statistics are abelian, as the phase is independent of the path and a simple complex number.  

The answer is \(<|begin_of_box|>\pi/2<|end_of_box|>\).</answer>