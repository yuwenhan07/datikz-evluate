<think>Got it, let's try to figure out how to approach this. The problem is about modifying a map of topics by removing unnecessary boxes (A, B, C, Duck) and replacing them with chapter titles (2, 3, 4, 5), while keeping the central box (1). 

First, I need to understand the structure. The initial map probably has boxes labeled A, B, C, Duck, and the central box 1. The goal is to shrink the overall picture by removing those unnecessary boxes and replacing them with the chapter titles. So, the steps would be:

1. Identify which boxes are unnecessary: A, B, C, Duck. These need to be removed.
2. Replace those positions with chapter titles 2, 3, 4, 5. Wait, need to make sure the positions correspond. Maybe the original map had A, B, C, Duck in certain positions, and we need to put 2,3,4,5 in those spots.
3. Ensure the central box (1) remains. So the central area is still box 1, and the surrounding areas that were A, B, C, Duck are now replaced with the chapter titles.
4. Make sure the overall layout is smaller, so maybe the new elements (2,3,4,5) are smaller or arranged to fit, reducing the total size of the map.

Let me think about the visual aspects. If the original map had multiple boxes around the central one, removing four boxes (A, B, C, Duck) and adding four chapter titles (2,3,4,5) would mean the total number of elements is the same (central + 4 around), but the labels change. So the key is to replace the old labels with the new ones in their respective positions, remove the old ones, and adjust the layout to make the whole map smaller if needed (maybe by resizing the chapter titles or adjusting spacing).

Also, need to make sure the chapter titles are clear and fit the visual style. Maybe the chapter titles are in a similar style to the original boxes but smaller, or adjusted to fit the space. The central box (1) is already present, so we don't touch that.

So, step-by-step plan:

- Locate boxes A, B, C, Duck in the initial map.
- Remove each of these boxes (delete them from the map).
- In the positions where A, B, C, Duck were, add the chapter titles 2, 3, 4, 5. Make sure they are placed in the same relative positions (e.g., if A was top-left, 2 goes there; B top-right, 3; C bottom-left, 4; Duck bottom-right, 5, or some arrangement that makes sense for the map's structure).
- Adjust the size or spacing of the new chapter titles and the central box (1) if needed to make the overall map smaller. Maybe reduce the font size of the chapter titles or the size of the boxes, or tighten the spacing between elements.
- Ensure the map is visually balanced, with the central box (1) as the focal point, surrounded by the chapter titles (2,3,4,5) in a clear, organized way.

If the original map had a grid or specific arrangement, the chapter titles should follow that arrangement. For example, if A, B, C, Duck were in a square around the central box, then 2,3,4,5 should be placed in the same square positions.

Also, check if there are any other elements that need adjustment, but the problem states only removing A, B, C, Duck and replacing with chapter titles, so focus on those changes.

In summary, the process involves removing four elements, adding four new elements in their places, adjusting layout and size to make the map smaller while keeping the central box and the new chapter titles.</think><answer>To create a visually appealing, smaller map by modifying the initial attempt, follow these steps:  

### Step 1: Identify and Remove Unnecessary Boxes  
Locate the boxes labeled **A, B, C, and Duck** in the initial map. These are the “unnecessary boxes” to remove.  

### Step 2: Replace with Chapter Titles  
In the positions where A, B, C, and Duck were located, add the chapter titles **2, 3, 4, and 5**. Ensure they are placed in the same relative positions (e.g., if A was top - left, place “2” there; if B was top - right, place “3” there; and so on for C and Duck).  

### Step 3: Adjust Layout and Size  
- Keep the central box (1) intact.  
- To make the overall map smaller, adjust the size of the chapter titles (2, 3, 4, 5) or reduce spacing between elements. Ensure the map remains balanced, with the central box (1) as the focal point and the chapter titles arranged clearly around it.  

### Step 4: Refine Visual Appeal  
Make sure the chapter titles match the visual style of the original map (e.g., same font, color scheme, or box style if applicable). Ensure the new layout is easy to read and understand, with clear connections between the central box (1) and the surrounding chapter titles (2, 3, 4, 5).  


By removing the unnecessary boxes and replacing them with chapter titles while adjusting the layout, the map will be smaller, more focused, and visually appealing.</answer>