<think>Got it, let's try to parse this problem step by step. First, let's understand the setup. We have a graph G_D with maximum degree D. Then G_D^2 is probably the square of the graph G_D, which usually means the graph where two vertices are adjacent if they are at distance at most 2 in G_D. But the problem mentions black vertices forming a clique of order 5D/2 in G_D^2. Wait, a clique of order 5D/2 means there are 5D/2 vertices, all connected to each other. Then, each pair of black vertices with no common white neighbor—so if two black vertices have no white neighbor in common, there are only 5 such pairs total. For each such pair, there's a gray neighbor that is adjacent only to them. Hmm, need to make sure I understand each part.

First, G_D has maximum degree D, so each vertex in G_D has degree at most D. The square G_D^2 would have edges between vertices at distance ≤2. The black vertices in G_D^2 form a clique of size 5D/2. A clique is a complete subgraph, so every pair of black vertices is connected. Now, among these black vertices, any two that have no common white neighbor—so if two black vertices u and v have no white neighbor in common, meaning the intersection of their neighborhoods (in G_D) is empty? Wait, or maybe in G_D^2, the neighborhoods? Wait, the problem says "each pair of black vertices with no common white neighbor (only 5 total)". So there are 5 pairs of black vertices where each pair has no white neighbor in common. For each such pair, there's a gray neighbor that is adjacent only to them. So for each of these 5 pairs, there's a gray vertex connected only to those two black vertices. So each such gray vertex has degree 2, and is only connected to those two black vertices.

Now, let's think about the structure. The black clique has 5D/2 vertices. Each pair in the 5 pairs has a unique gray neighbor. So there are 5 gray vertices, each with degree 2, each connected to two black vertices. Now, we need to consider the degrees of the black vertices. Each black vertex is in the clique, so each black vertex is connected to 5D/2 - 1 other black vertices. But also, in G_D^2, the square of G_D, the neighborhood of a vertex includes all vertices at distance ≤2. So the neighborhood of a black vertex in G_D^2 includes its neighbors in G_D, their neighbors in G_D, minus itself (since it's the square, maybe not including itself). Wait, the square G^2 has an edge between u and v if d_G(u,v) ≤ 2. So the neighborhood of a vertex u in G^2 is N_G^2(u) = N_G(u) ∪ (N_G(u) \ {u}) ∪ {v | ∃w, d_G(u,w)=1, d_G(w,v)=1, v ≠ u}. But maybe focusing on the black vertices: each black vertex is part of the clique, so connected to all other black vertices. Now, the problem says "each pair of black vertices with no common white neighbor"—so for two black vertices u and v, if they have no white neighbor in common, then there's a gray vertex adjacent only to u and v. So for each such pair, the gray vertex is connected to u and v, and maybe no one else. So each gray vertex has exactly two neighbors, both black, and no other edges.

Now, let's consider the total number of edges between black and gray vertices. Each gray vertex has two edges, so 5 gray vertices × 2 = 10 edges. Each edge is between a gray vertex and a black vertex, so there are 10 black-gray edges. Each black vertex can have edges to gray vertices. Let's say a black vertex is part of how many pairs with no common white neighbor? If there are 5 such pairs total, and each black vertex is in how many pairs? If the black clique has n = 5D/2 vertices, then the number of pairs is C(n, 2) = n(n-1)/2. The number of pairs with no common white neighbor is 5. So the fraction of pairs that have no common white neighbor is 5 / [n(n-1)/2] = 10 / [5D/2 (5D/2 - 1)] = 4 / [D(5D/2 - 1)]. Hmm, maybe not the best approach.

Alternatively, think about the degrees of black vertices. Each black vertex is in the clique, so degree within the clique is (5D/2 - 1). Additionally, each black vertex is connected to some gray vertices. Each gray vertex is connected to two black vertices, so the total number of black-gray edges is 5×2=10. If each black vertex has degree d in the black clique and e in the black-gray edges, then the total degree of a black vertex is d + e. But in G_D^2, the maximum degree could be related to D, but maybe we need to ensure that the number of gray neighbors per black vertex is limited. Wait, but the problem says "only a few such gray vertices are shown", implying that each black vertex might have at most a few gray neighbors, but we need to relate this to D.

Wait, let's try to find a relation for D. The black clique has 5D/2 vertices. Each vertex in the clique has degree D in G_D, but in G_D^2, the degree could be higher. However, the key might be that the number of pairs of black vertices with no common white neighbor is 5, and each such pair has a unique gray neighbor. So the number of such gray vertices is equal to the number of such pairs, which is 5. Each gray vertex is connected to two black vertices, so each gray vertex accounts for one pair. Therefore, the number of such pairs is equal to the number of gray vertices, which is 5. Now, for each black vertex, how many pairs does it participate in? A black vertex is paired with (n - 1) other black vertices, but how many of those pairs have no common white neighbor? Let's say a black vertex is in k pairs with no common white neighbor. Then total number of such pairs would be (n × k)/2, since each pair is counted twice. We have (n × k)/2 = 5. n is 5D/2, so (5D/2 × k)/2 = 5 → (5Dk)/4 = 5 → Dk = 4 → k = 4/D. Since k must be an integer (as it's the number of pairs per vertex), D must divide 4. Possible values for D could be 1, 2, 4. But D is the maximum degree of G_D, which is a graph, so D should be at least 1. If D=2, then k=2, so each black vertex is in 2 pairs with no common white neighbor, total pairs (5D/2 * 2)/2 = (5*2/2 * 2)/2 = (5 * 2)/2 = 5, which matches. If D=4, then k=1, total pairs (5*4/2 *1)/2 = (10 *1)/2=5, also matches. Wait, but need to check if the degrees make sense. If D=2, then G_D has maximum degree 2, so it's a union of cycles or paths. The square of such a graph would have certain properties, but maybe the key is that D must be even? Wait, 5D/2 must be an integer, so D must be even. So D is even, say D=2m, then 5D/2 = 5m, which is an integer. Then, going back, if D=2, then n=5, so the black clique has 5 vertices. Then the number of pairs is C(5,2)=10, and 5 of them have no common white neighbor, meaning 5 pairs, each with a gray neighbor. Each gray neighbor is connected to two black vertices, so 5 gray vertices, each with two edges, total 10 edges, which matches 5 pairs × 2 edges = 10. That works. If D=4, then n=10, C(10,2)=45 pairs, 5 pairs have no common white neighbor, so each gray vertex is connected to two black vertices, 5 gray vertices ×2=10 edges, but 5 pairs would need 5 gray vertices, each connected to two black vertices, so total edges 10, but 45-40=5 pairs left, which would need 5 gray vertices, total 10 gray vertices? Wait, maybe I messed up. Wait, if n=5D/2, for D=4, n=10, C(10,2)=45 pairs. If 5 of them have no common white neighbor, then the remaining 40 pairs must have at least one common white neighbor. Each gray vertex is connected to one pair, so 5 gray vertices account for 5 pairs, leaving 40 pairs, which would need 40/2=20 gray vertices, but the problem says "only a few such gray vertices are shown", implying that the number is small, so D=2 makes n=5, which is small, and 5 gray vertices is also small. So likely D=2, making n=5, and the black clique has 5 vertices, each pair among the 5 has a gray neighbor if they have no common white neighbor, and there are 5 such pairs, each with a unique gray neighbor, totaling 5 gray vertices, each connected to two black vertices, which fits. Therefore, the maximum degree D must be 2, or maybe another even number, but given that 5D/2 must be integer, D even, and the example with D=2 works.</think><answer>To analyze the graph \( G_D \) with maximum degree \( D \) and its square \( G_D^2 \), we note the black vertices form a clique of order \( \frac{5D}{2} \). Among these, 5 pairs of black vertices have no common white neighbor, each connected to a unique gray vertex (adjacent only to those two black vertices). For the clique size \( \frac{5D}{2} \) to be an integer, \( D \) must be even. Setting \( D = 2 \) gives a clique of size 5, with \( \binom{5}{2} = 10 \) total pairs, 5 of which have no common white neighbor. Each such pair connects to a gray vertex, resulting in 5 gray vertices (each with degree 2), matching the problem's conditions.  

The answer is \(<|begin_of_box|>2<|end_of_box|>\).</answer>