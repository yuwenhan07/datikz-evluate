<think>Got it, I need to understand the representation of different integration paths that lead to the inequality d(x,y) + d(y,z) ≥ d(x,z). This seems related to the triangle inequality in metric spaces, but maybe in the context of path integrals or something else. Wait, the question mentions integration paths, so perhaps it's about the length of paths in a space where the distance function isn't necessarily a metric, or maybe in a more general setting like Riemannian manifolds, but the inequality given is the standard triangle inequality for metrics. 

First, let's recall that in a metric space, the triangle inequality states that for any three points x, y, z, the distance from x to z is less than or equal to the sum of the distances from x to y and y to z. The inequality here is written as d(x,y) + d(y,z) ≥ d(x,z), which is exactly the triangle inequality. Now, the question is about representing different integration paths that lead to this inequality. Maybe considering paths from x to z that go through y, versus direct paths, and comparing their lengths.

If we think of the distance as the length of a path, then a path from x to z going through y would be the concatenation of a path from x to y and a path from y to z. The total length would be the sum of the lengths of these two paths. The direct path from x to z might have a shorter length, hence the inequality. But how does this relate to integration paths? Maybe in the context of line integrals, where the integral of a function along a path depends on the path, but if the function is related to the distance, then integrating along different paths might give different results, but the inequality would come from the fact that the shortest path is the direct one.

Alternatively, if we consider the distance as the integral of some metric tensor along a path, then the length of a path γ from x to y is ∫_γ ds, where ds is the line element. Then, the path from x to z through y would be γ1 from x to y and γ2 from y to z, so the total length is ∫_γ1 ds + �int_γ2 ds, which should be greater than or equal to the length of the direct path from x to z, which is ∫_γ3 ds. Hence, the inequality holds.

But the question is about representing the different integration paths. So, maybe drawing three points x, y, z, with paths from x to y, y to z, and x to z. The path from x to z through y is the combination of the first two, and the inequality is that the sum of the first two is at least the third. To represent this, one might draw a triangle with vertices x, y, z, and show the paths as the sides of the triangle, with the inequality indicating that the sum of two sides is at least the third side.

Another angle: if we consider the distance as the integral of a non-negative function (like the speed along the path), then the integral over the combined path is the sum of the integrals over each segment, and by the properties of integrals, the integral over the direct path should be less than or equal to the sum of the integrals over the segments, hence the inequality.

Wait, but the problem says "different integration paths that lead to the inequality". So maybe considering all possible paths from x to z, and showing that the minimum length is d(x,z), while any other path through y would have length at least d(x,y)+d(y,z). Therefore, the inequality holds because the direct path is the shortest, and any detour through y can't be shorter than the direct path.

To represent this, one could draw a diagram with points x, y, z, and show two paths: one direct from x to z, and another going through y (so x to y to z), and indicate that the length of the detour path is the sum of the two segments, which must be greater than or equal to the direct path's length. This visual representation would show the inequality as a consequence of the path lengths.

Alternatively, in a more mathematical sense, if we define the distance as the infimum of the lengths of all paths between two points, then d(x,z) is the infimum over all paths from x to z, while d(x,y) + d(y,z) is the length of a specific path (through y), which could be greater than or equal to the infimum, hence the inequality.

So, summarizing, the inequality d(x,y) + d(y,z) ≥ d(x,z) is a direct consequence of the triangle inequality in metric spaces, and the different integration paths (or paths in the space) can be represented by considering paths going through an intermediate point y versus direct paths, with the inequality stating that the sum of the lengths of the two segments is at least the length of the direct segment. The representation would involve showing these paths in a diagram, possibly a triangle with sides representing the paths, and indicating the inequality between their lengths.</think><answer>The inequality \( d(x,y) + d(y,z) \geq d(x,z) \) is a fundamental triangle inequality in metric spaces, representing the relationship between path lengths. To represent different integration paths, consider three points \( x, y, z \) with paths: a direct path from \( x \) to \( z \), and a detour path through \( y \) (concatenating paths \( x \to y \) and \( y \to z \)). The length of the detour path is the sum of the individual segment lengths, which must be at least the length of the direct path due to the triangle inequality. This can be visualized as a triangle with vertices \( x, y, z \), where the inequality indicates the sum of two side lengths (paths \( x \to y \) and \( y \to z \)) is greater than or equal to the third side (direct path \( x \to z \)).  

The answer is \(<|begin_of_box|>d(x,y) + d(y,z) \geq d(x,z)<|end_of_box|>\).</answer>