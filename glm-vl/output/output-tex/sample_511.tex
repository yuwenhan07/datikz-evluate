\documentclass{standalone}
\usepackage{tikz}
\usepackage{amsmath}
\begin{document}

\begin{tikzpicture}[scale=1.5, background color=white]
    % Define the vector field
    % For example, a vector field with a saddle point at (0,0)
    % F = (x - y, x + y)
    % Or a vortex: F = (y, -x)
    % Let's use F = (x, -y) which has a vortex at (0,0)
    % The streamlines command in TikZ can be used with the `streamlines` library
    % Or use pgfplots, but since the user asked for TikZ, let's try TikZ's approach
    % Alternatively, use the `vector` library, but maybe better to use pgfplots for stream plots
    % Wait, pgfplots has a `stream plot` function. Let's try that.
    % But the user wants TikZ, so maybe use TikZ's `streamlines` with a defined vector field.
    
    % Define the vector field as a function
    % For example, vector field components:
    % \def\vx#1#2{#1} % x component
    % \def\vy#1#2{-#2} % y component
    % Then use \draw[streamlines] plot[domain=-2:2, samples=10] (\x,\y) ...
    
    % Alternatively, use the `tikz` library's `streamlines` with a defined vector field.
    % The syntax is \draw[streamlines={vector field={\vx(\x,\y) \vy(\x,\y)}}] ...
    
    % Let's define the vector field as F = (x, -y)
    % So \vx(\x,\y) = \x; \vy(\x,\y) = -\y;
    % Then draw streamlines:
    \draw[streamlines={vector field={\x \ -\y}}] plot[domain=-2:2, samples=10, smooth, variable=\t] (\t, {0.5*(\t - 2)});
    % Wait, maybe better to use a more general approach.
    
    % Alternatively, use pgfplots for the stream plot, then include it in TikZ.
    % But the user asked for TikZ, so maybe stick to TikZ.
    
    % Another approach: use the `vector` library in TikZ, which allows drawing vector fields.
    % \usetikzlibrary{vector}
    % Then \draw[vector] plot[domain=-2:2, samples=10, smooth, variable=\t] (\t, {0.5*(\t - 2)});
    % But I need to define the vector field components.
    
    % Let's try a simple example with a vector field F = (x, y) (source at origin)
    % Then streamlines would radiate outwards.
    % Define the vector field as \vx = x, \vy = y
    \draw[streamlines={vector field={\x \ \y}}] plot[domain=-2:2, samples=10, smooth, variable=\t] (\t, {0.5*(\t - 2)});
    % Wait, maybe the domain should be a rectangle, so use \draw[streamlines={vector field={\x \ \y}}] plot[domain=-2:2, domain y=-2:2, samples=10, smooth] (\x,\y);
    
    % Now, add the equations. The first equation is in red, second in tanh (maybe a color like brown), third in white.
    % First equation (red): \(\mathbf{F} = (x, y)\)
    \node[red] at (0, -2) {\(\mathbf{F} = (x, y)\)};
    % Second equation (tanh, maybe a color like tan, which is a brownish color)
    \node[tan] at (0, -3) {\(\mathbf{F} = (\tanh(x), \tanh(y))\)};
    % Third equation (white, maybe just normal text, but on white background, it's white)
    \node[white] at (0, -4) {\(\mathbf{F} = (x - y, x + y)\)};
    
    % Add singularities, maybe a dot at the origin to indicate a singularity.
    \fill[red] (0,0) circle (2pt);
    % Or a different symbol, like a cross.
    \draw[red, thick] (-0.1,0) -- (0.1,0) (-0.1,0) -- (0.1,0);
    
    % Adjust the plot area
    \draw[thick,->] (-2.5, -2.5) -- (2.5, 2.5) node[anchor=south west] {};
    \draw[thick,->] (-2.5, -2.5) -- (-2.5, 2.5) node[anchor=east] {};
    % Add grid if needed
    \draw[gray, very thin] (-2,-2) grid (2,2);
\end{tikzpicture}

\end{document}