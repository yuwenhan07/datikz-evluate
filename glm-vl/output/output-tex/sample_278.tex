<think>Got it, let's try to figure out what the question is asking. The user mentioned arrows on the left part illustrating properties of the original Dyson-Schmidt dynamics on (0, ∞) as stated in Lemma~lemma:slower, and the right part is after logarithmic transformation f to R. 

First, I need to recall what the Dyson-Schmidt dynamics are. Dyson-Schmidt decomposition is a method used in random matrix theory, often related to the dynamics of eigenvalues. The original dynamics might be on the positive real line (0, ∞), and then a logarithmic transformation maps this to the real line R. 

The left part (original dynamics on (0, ∞)) probably shows how the dynamics behave, maybe like a flow or some kind of evolution, with arrows indicating direction or properties like slower growth or decay, as mentioned in Lemma~lemma:slower. The right part, after applying the logarithmic transformation f(x) = ln(x), maps (0, ∞) to R, so each point x > 0 is mapped to ln(x), which is a real number. 

So the question might be asking about the relationship between the original dynamics and the transformed dynamics, or how the properties from the lemma translate under the logarithmic transformation. Maybe the arrows on the right represent the transformed dynamics, showing how the original flow becomes a different kind of dynamics on R, perhaps linear or with different properties. 

I need to make sure I understand the transformation: if f(x) = ln(x), then the inverse is f^{-1}(y) = e^y. So the dynamics on R would correspond to the dynamics on (0, ∞) under the exponential map. If the original dynamics on (0, ∞) have some "slower" behavior (as per Lemma~lemma:slower), then after taking the logarithm, the dynamics on R might have different characteristics, maybe faster or more straightforward. 

Alternatively, the arrows on the left might indicate that the original dynamics have some kind of expansion or contraction, and after the logarithmic transformation, which is a conformal map (since it's a diffeomorphism), the dynamics might become more manageable, like a linear flow. 

Since the user is asking about the illustration, maybe the key points are the transformation from (0, ∞) to R via logarithm, and how the dynamics' properties change, with the lemma stating that the original dynamics are "slower" (maybe in some sense of growth rate), and after transformation, the dynamics on R might have a different rate or structure. 

I should also consider that in random matrix theory, the Dyson-Schmidt process often involves moving eigenvalues around, and the logarithmic transformation is used to turn the dynamics into a more standard form, maybe like a Brownian motion or some other process on the real line. 

Putting it all together, the question is likely about explaining the relationship between the original dynamics on (0, ∞) and the transformed dynamics on R via the logarithmic map, referencing Lemma~lemma:slower which probably states that the original dynamics have some "slower" property (maybe the rate of change is slower, or the flow is more gradual), and after the logarithmic transformation, the dynamics on R might have a different rate or be more straightforward. The arrows on the left show the original dynamics' behavior, and the right shows the transformed one, with the transformation preserving some properties but changing the scale.</think><answer>To analyze the relationship between the original Dyson-Schmidt dynamics on \((0, \infty)\) and the transformed dynamics on \(\mathbb{R}\) via the logarithmic transformation \(f(x) = \ln(x)\), we start by noting that the logarithmic map is a diffeomorphism (smooth with a smooth inverse) that transforms the positive real line \((0, \infty)\) to the entire real line \(\mathbb{R}\).  

For the original dynamics on \((0, \infty)\), Lemma~lemma:slower likely describes a "slower" growth or evolution property (e.g., a more gradual flow or reduced rate of change). Under the logarithmic transformation, each point \(x > 0\) is mapped to \(y = \ln(x)\), and the inverse map is \(x = e^y\). This transformation compresses the infinite interval \((0, \infty)\) to \(\mathbb{R}\), which can simplify the dynamics by converting multiplicative or exponential behaviors into additive ones.  

The arrows on the left (original dynamics) illustrate the flow or evolution of the system on \((0, \infty)\), possibly showing slower changes in the original scale. On the right (transformed dynamics), the arrows represent the corresponding flow on \(\mathbb{R}\), where the "slower" property from the lemma might translate into a more standard or manageable dynamics (e.g., linear flow or simpler rate of change). The transformation preserves the qualitative behavior of the dynamics while altering the scale, making the properties stated in the lemma more apparent in the real line setting.  

In summary, the logarithmic transformation \(f(x) = \ln(x)\) maps the original dynamics from \((0, \infty)\) to \(\mathbb{R}\), where the "slower" properties from Lemma~lemma:slower are re-expressed in a form that is easier to analyze, with the dynamics on \(\mathbb{R}\) reflecting the transformed behavior of the original system.  

The answer is \(<|begin_of_box|>\ln(x)<|end_of_box|>\).</answer>