<think>Got it, I need to find the supports of the polynomial f(X) = 1 + x⁻¹y⁻¹ + x⁻¹y¹ + x¹y⁻¹ + x¹y¹, and then its dilations f(X²) and f(X³). First, let's recall that the support of a polynomial is the set of all monomials present in it, which are the terms with non-zero coefficients. 

Starting with f(X) = 1 + x⁻¹y⁻¹ + x⁻¹y¹ + x¹y⁻¹ + x¹y¹. The terms here are 1 (which is x⁰y⁰), x⁻¹y⁻¹, x⁻¹y¹, x¹y⁻¹, x¹y¹. So the exponents for x are -1, -1, 1, 1, 0 (for the constant term) and for y are -1, 1, -1, 1, 0. Wait, the exponents for x in each term are -1, -1, 1, 1, 0 (since 1 is x⁰), and for y they are -1, 1, -1, 1, 0. So the support is the set of monomials x^a y^b where a ∈ {-1, 0, 1} and b ∈ {-1, 0, 1}, but wait, let's check each term:

First term: 1 is x^0 y^0. Second term: x⁻¹y⁻¹, so exponents (a, b) = (-1, -1). Third term: x⁻¹y¹, (a, b) = (-1, 1). Fourth term: x¹y⁻¹, (a, b) = (1, -1). Fifth term: x¹y¹, (a, b) = (1, 1). So the support is the set of monomials with exponents (a, b) where a is in {-1, 1} or 0, and b similarly. Wait, more precisely, the support is the set of monomials x^k y^l where k ∈ {-1, 0, 1} and l ∈ {-1, 0, 1}, but let's list all the exponents:

For x: the exponents are -1, -1, 1, 1, 0. So x can have exponents -1, 0, 1. For y: exponents are -1, 1, -1, 1, 0. So y can have exponents -1, 0, 1. Therefore, the support is the set of monomials x^a y^b where a ∈ {-1, 0, 1} and b ∈ {-1, 0, 1}, but not all combinations? Wait, let's check each term:

1 is x^0 y^0. Then x⁻¹y⁻¹, x⁻¹y¹, x¹y⁻¹, x¹y¹. So the monomials present are x^0 y^0, x^{-1} y^{-1}, x^{-1} y^1, x^1 y^{-1}, x^1 y^1. So the support is the set containing these five monomials. 

Now, for the dilation f(X²). When we dilate by X², each monomial x^k y^l becomes (X²)^k (Y²)^l = X^{2k} Y^{2l}. So let's apply this to each term in f(X):

1 becomes 1 (since 1 is the constant term, dilated by X²^0 Y²^0, which is 1). 

x⁻¹y⁻¹ becomes (X²)^{-1} (Y²)^{-1} = X^{-2} Y^{-2}.

x⁻¹y¹ becomes (X²)^{-1} (Y²)^1 = X^{-2} Y^2.

x¹y⁻¹ becomes (X²)^1 (Y²)^{-1} = X^2 Y^{-2}.

x¹y¹ becomes (X²)^1 (Y²)^1 = X^2 Y^2.

So f(X²) = 1 + X^{-2}Y^{-2} + X^{-2}Y^2 + X^2Y^{-2} + X^2Y^2. Therefore, the support of f(X²) is the set of monomials X^a Y^b where a ∈ {-2, 0, 2} and b ∈ {-2, 0, 2}, similar to the original support but with exponents doubled. Wait, let's check each term's exponents:

From the dilated terms, the exponents for X are -2, -2, 2, 2, 0 (from the constant term), and for Y similarly -2, 2, -2, 2, 0. So the support is monomials with X exponents even (since -2, 0, 2) and Y exponents even as well. So the support is {X^a Y^b | a, b ∈ {-2, 0, 2}}.

Next, f(X³). Similarly, dilating by X³ would replace each x^k with X^{3k} and y^l with Y^{3l}. Let's apply this to each term:

1 becomes 1.

x⁻¹y⁻¹ becomes (X³)^{-1} (Y³)^{-1} = X^{-3} Y^{-3}.

x⁻¹y¹ becomes (X³)^{-1} (Y³)^1 = X^{-3} Y^3.

x¹y⁻¹ becomes (X³)^1 (Y³)^{-1} = X^3 Y^{-3}.

x¹y¹ becomes (X³)^1 (Y³)^1 = X^3 Y^3.

So f(X³) = 1 + X^{-3}Y^{-3} + X^{-3}Y^3 + X^3Y^{-3} + X^3Y^3. Therefore, the support is the set of monomials X^a Y^b where a ∈ {-3, 0, 3} and b ∈ {-3, 0, 3}, with exponents being multiples of 3 for both X and Y.

Wait, let me confirm the support definition. The support of a polynomial is the set of monomials with non-zero coefficients. So for f(X), the monomials present are x⁰y⁰, x⁻¹y⁻¹, x⁻¹y¹, x¹y⁻¹, x¹y¹. So the support is {x⁰y⁰, x⁻¹y⁻¹, x⁻¹y¹, x¹y⁻¹, x¹y¹}. For f(X²), each monomial is replaced by (X²)^k (Y²)^l, so the exponents for X are 2k, Y are 2l, where k is -1, -1, 1, 1, 0 for the original terms. So 2k gives -2, -2, 2, 2, 0, and 2l gives -2, 2, -2, 2, 0. So the support is {X⁰Y⁰, X⁻²Y⁻², X⁻²Y², X²Y⁻², X²Y²}. Similarly, for f(X³), exponents are 3k, 3l, so k=-1, -1, 1, 1, 0 gives 3k = -3, -3, 3, 3, 0, and 3l similarly, so support is {X⁰Y⁰, X⁻³Y⁻³, X⁻³Y³, X³Y⁻³, X³Y³}.

I need to make sure there are no other terms. For f(X), the polynomial has five terms, so support has five monomials. For f(X²), each term's exponents are doubled, so the monomials are those with even exponents (for both X and Y) in the original support, but since the original support had exponents in x and y as -1,0,1, doubling them gives -2,0,2, which are even exponents. Similarly for f(X³), exponents are multiples of 3. So the supports are as above.</think><answer>To determine the supports of the polynomial \( f(X) = 1 + x^{-1}y^{-1} + x^{-1}y^1 + x^1y^{-1} + x^1y^1 \) and its dilations \( f(X^2) \) and \( f(X^3) \), we identify the monomials with non-zero coefficients in each polynomial.  

For \( f(X) \), the monomials are \( x^0y^0, x^{-1}y^{-1}, x^{-1}y^1, x^1y^{-1}, x^1y^1 \), so its support is the set of these monomials.  

For \( f(X^2) \), replacing \( x \) with \( X^2 \) and \( y \) with \( Y^2 \) gives the monomials \( X^0Y^0, X^{-2}Y^{-2}, X^{-2}Y^2, X^2Y^{-2}, X^2Y^2 \), so its support is the set of these monomials.  

For \( f(X^3) \), replacing \( x \) with \( X^3 \) and \( y \) with \( Y^3 \) gives the monomials \( X^0Y^0, X^{-3}Y^{-3}, X^{-3}Y^3, X^3Y^{-3}, X^3Y^3 \), so its support is the set of these monomials.  

The answer is \(<|begin_of_box|>\{x^0y^0, x^{-1}y^{-1}, x^{-1}y^1, x^1y^{-1}, x^1y^1\} (f(X)), \{X^0Y^0, X^{-2}Y^{-2}, X^{-2}Y^2, X^2Y^{-2}, X^2Y^2\} (f(X^2)), \{X^0Y^0, X^{-3}Y^{-3}, X^{-3}Y^3, X^3Y^{-3}, X^3Y^3\} (f(X^3))<|end_of_box|>\)</answer>