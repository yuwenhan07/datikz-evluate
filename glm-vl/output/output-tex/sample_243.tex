<think>Got it, let's try to understand the problem step by step. First, the question is about orthogonalized deflation applied to a rank-two tensor, specifically the tensor defined as \( \_i=1^2 \_i x_i^3 \), which I think is a rank-2 tensor with components \( \_1 x_1^3 \) and \( \_2 x_2^3 \), maybe? Wait, the notation might be a bit confusing. Let me parse the tensor definition again. The tensor is written as \( \_i=1^2 \_i x_i^3 \), which probably means a rank-2 tensor with entries \( \_1 x_1^3 \) and \( \_2 x_2^3 \), so the tensor is \( \begin{pmatrix} \_1 x_1^3 & \_1 x_1^3 \\ \_2 x_2^3 & \_2 x_2^3 \end{pmatrix} \) or something? Wait, no, rank-two tensor usually has components like \( T_{ij} = \_i x_i^3 \) for i,j=1,2? Maybe the tensor is \( T_{ij} = \_i x_i^3 \delta_{ij} \), but the description says "rank-two tensor _i=1^2 _i x_i^3" which might mean a tensor with two indices, each component being \( \_i x_i^3 \), so maybe \( T_{ij} = \_i x_i^3 \) for i,j=1,2. Then, orthogonalized deflation is applied to this tensor to get signal estimates \( u_1 \) and \( u_2 \) at first and second deflation steps.

The problem states that when the signals are orthogonal (i.e., \( x_1, x_2 = 0 \), maybe meaning \( x_1 \) and \( x_2 \) are orthogonal, like their inner product is zero), the signals are successfully recovered, but when they are correlated ( \( x_1, x_2 = 0.5 \), maybe the correlation coefficient is 0.5), the estimation is altered. Also, \( \_1 \) is fixed while varying \( \_2 \).

First, let's recall what orthogonalized deflation does. Deflation methods are used to separate components of a signal by subtracting previous estimates from the data, assuming orthogonality. If the components are orthogonal, deflation works well, but if they are correlated, the subtraction introduces errors, altering the estimates.

The tensor here is a rank-2 tensor, so maybe it's a matrix where each element is \( \_i x_i^3 \), so the matrix is \( \begin{pmatrix} \_1 x_1^3 & \_1 x_1^3 \\ \_2 x_2^3 & \_2 x_2^3 \end{pmatrix} \), but that might not make sense. Alternatively, if it's a tensor with two indices, each component is \( \_i x_i^3 \), so for i=1,2, j=1,2, \( T_{ij} = \_i x_i^3 \). Wait, maybe the tensor is \( T = \_1 x_1^3 \mathbf{e}_1 \mathbf{e}_1^T + \_2 x_2^3 \mathbf{e}_2 \mathbf{e}_2^T \), where \( \mathbf{e}_1, \mathbf{e}_2 \) are basis vectors. Then, orthogonalized deflation would first estimate the first component \( \_1 x_1^3 \mathbf{e}_1 \mathbf{e}_1^T \), then subtract it from T to get the residual, then estimate the second component from the residual.

If \( x_1 \) and \( x_2 \) are orthogonal (i.e., \( \mathbf{e}_1 \) and \( \mathbf{e}_2 \) are orthogonal), then the deflation should work, but if they are correlated, the residual might not capture the second component correctly, leading to altered estimates.

Given that \( \_1 \) is fixed and \( \_2 \) is varied, we need to see how the estimation of \( u_2 \) changes with \( \_2 \), especially when the signals are correlated versus orthogonal.

In the orthogonal case, the deflation should correctly separate the two components, so \( u_1 \) and \( u_2 \) would accurately represent the original signals. When they are correlated, the deflation might mix the components, causing the estimates to be less accurate, possibly with \( u_2 \) being distorted or having a different magnitude depending on \( \_2 \).

Also, the problem mentions "at first and second deflation steps" yielding \( u_1 \) and \( u_2 \). So first deflation step estimates \( u_1 \) from the initial tensor, then subtracts \( u_1 \) to get the residual, then estimates \( u_2 \) from the residual. If the original signals are orthogonal, subtracting \( u_1 \) (which is a multiple of \( \_1 x_1^3 \)) leaves the residual with only \( \_2 x_2^3 \), so \( u_2 \) should be a multiple of \( \_2 x_2^3 \). If they are correlated, the residual might still have a component from \( \_1 x_1^3 \), so \( u_2 \) would be a combination, altering its estimate.

Since \( \_1 \) is fixed, varying \( \_2 \) would change the relative strength of the second component. If \( \_2 \) is large, the second component is stronger, so even with correlation, the deflation might still capture it, but if \( \_2 \) is small, the correlation might make the estimation worse.

Putting it all together, the key points are the effect of orthogonality on deflation accuracy, how correlation between signals affects the deflation process, and how varying \( \_2 \) (with fixed \( \_1 \)) influences the estimation of \( u_2 \) in both orthogonal and correlated cases.</think><answer>To analyze the orthogonalized deflation applied to the rank-two tensor \( T_{ij} = \_i x_i^3 \) (for \( i,j = 1,2 \)), we consider the impact of signal orthogonality and correlation on signal recovery:  

1. **Orthogonal Case (\( x_1, x_2 \perp \))**:  
   When signals \( x_1 \) and \( x_2 \) are orthogonal, the tensor components \( \_1 x_1^3 \) and \( \_2 x_2^3 \) are uncorrelated. Orthogonalized deflation works by first estimating \( u_1 \) (corresponding to \( \_1 x_1^3 \)) and subtracting it from \( T \), leaving a residual with only \( \_2 x_2^3 \). This allows \( u_2 \) to be accurately estimated as a multiple of \( \_2 x_2^3 \), successfully recovering both signals.  

2. **Correlated Case (\( x_1, x_2 = 0.5 \))**:  
   When signals are correlated, the tensor components are not orthogonal. Deflation subtracts \( u_1 \) (estimated from the full tensor) from \( T \), but the residual still contains a component from \( \_1 x_1^3 \) due to correlation. This introduces errors in estimating \( u_2 \), causing \( u_2 \) to be a combination of both \( \_1 x_1^3 \) and \( \_2 x_2^3 \), altering its accuracy.  

3. **Effect of Varying \( \_2 \) (with Fixed \( \_1 \))**:  
   - When \( \_2 \) is large, the second component \( \_2 x_2^3 \) dominates, and even with correlation, deflation may still capture \( u_2 \) reasonably.  
   - When \( \_2 \) is small, the second component is weak, and correlation amplifies errors, making \( u_2 \) estimation more distorted.  

In summary, orthogonality ensures accurate deflation, while correlation degrades it, with \( \_2 \) modulating the severity of distortion.  

The answer is \(<|begin_of_box|>u_1, u_2 \text{ accurately recovered in orthogonal case, altered in correlated case}<|end_of_box|>\).</answer>