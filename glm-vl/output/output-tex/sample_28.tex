python
import matplotlib.pyplot as plt
from fractions import Fraction  # For handling fractions

# Define the fractions to display (as numerator/denominator tuples)
fractions = [(3, 20), (4, 9), (1, 3), (1, 2), (5, 12)]  # Example fractions
fraction_objects = [Fraction(n, d) for n, d in fractions]  # Convert to Fraction objects
fraction_decimals = [f.numerator / f.denominator for f in fraction_objects]  # Decimal equivalents

# Create the plot
fig, ax = plt.subplots(figsize=(10, 2))
ax.set_xlim(0, 1)  # Axis range from 0 to 1
ax.set_ylim(0, 0.1)  # Adjust y-axis to center the axis line
ax.axhline(y=0.05, color='black', linewidth=1)  # Horizontal axis line
ax.set_xlabel('Fractional Axis')

# Add vertical ticks at fractional positions
for value in fraction_decimals:
    ax.axvline(x=value, color='gray', linestyle='-', alpha=0.7)  # Vertical tick marks
    # Label each tick with the fraction
    ax.text(value, 0.05, f' {fraction_objects[fraction_decimals.index(value)].numerator}/{fraction_objects[fraction_decimals.index(value)].denominator}', 
            ha='center', va='center', fontsize=10)

# Set x-ticks to show fractional labels
ax.set_xticks(fraction_decimals)
ax.set_xticklabels([f'{n}/{d}' for n, d in fractions])  # Format as "numerator/denominator"

# Add grid lines (optional, for better readability)
ax.grid(axis='x', linestyle='--', alpha=0.5)

plt.tight_layout()
plt.show()