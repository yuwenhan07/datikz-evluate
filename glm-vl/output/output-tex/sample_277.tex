<think>Got it, I need to come up with an example of a convex and lower-semicontinuous (lsc) function that has a discontinuity at a point, say 0, and show some elements of its subdifferential at 0 as linear continuous tangent functionals, represented by dashed lines. 

First, let's recall some definitions. A convex function is lsc if it's convex and lower semicontinuous. A discontinuity in a convex function usually occurs at a boundary point or a point where the function isn't continuous, but for convex functions, the discontinuity would be a jump discontinuity, maybe a "corner" or a point where the function isn't continuous from the right or left. 

A common example of a convex function with a discontinuity is the function defined as f(x) = |x|, but that's continuous everywhere. Wait, maybe a function that has a "step" or a jump. For example, consider f(x) = max{0, x} + max{0, -x}, but that's just |x| again. Hmm, maybe a function that's not continuous at 0 but is convex. Let's think of a function defined piecewise. For example, f(x) = x for x ≥ 0, and f(x) = 0 for x < 0. Wait, is that convex? Let's check the convexity condition. For two points x1 < 0 and x2 ≥ 0, the line segment between them would have f(tx1 + (1-t)x2) = t f(x1) + (1-t)f(x2) if x1 < 0 and x2 ≥ 0. Wait, if x1 is negative, f(x1) = 0, and x2 is non-negative, f(x2) = x2. So f(tx1 + (1-t)x2) = f( t x1 + (1-t)x2 ). If t x1 + (1-t)x2 is negative, then f is 0; otherwise, it's equal to t x1 + (1-t)x2. But the convex combination would be t*0 + (1-t)x2 = (1-t)x2. However, if the point t x1 + (1-t)x2 is negative, then f of that is 0, which is less than (1-t)x2 if x2 > 0. So this function is not convex because the function value at the midpoint could be less than the average of the endpoints. So maybe that's not convex.

Another idea: consider a function that is convex but has a discontinuity at 0. Let's take f(x) = x for x ≥ 0, and f(x) = 0 for x < 0. Wait, let's check convexity. Take two points a < 0 and b > 0. The line segment between them is y = t a + (1-t) b for t in [0,1]. When t is between 0 and 1, y = (1-t)b + t a. If y ≥ 0, then f(y) = y; if y < 0, f(y) = 0. The convex combination is t f(a) + (1-t) f(b) = t*0 + (1-t)b = (1-t)b. For the function f(y) to be convex, we need f(y) ≥ t f(a) + (1-t) f(b) = (1-t)b. But if y = (1-t)b + t a < 0, then f(y) = 0, which is less than (1-t)b when b > 0. So this function is not convex. Hmm, maybe I need a different approach.

Wait, maybe a function that is convex but has a discontinuity in its subdifferential. Wait, the subdifferential of a convex function at a point contains all linear functionals that are tangent to the function at that point. If the function has a discontinuity at 0, maybe the subdifferential at 0 is not a singleton, and there are multiple linear functionals (dashed lines) representing different subgradients.

Let me consider the function f(x) = |x| for x ≠ 0, and f(0) = 0. Wait, no, |x| is continuous. Wait, maybe a function that is convex but has a "corner" at 0, but is discontinuous there. Wait, another example: f(x) = x for x ≥ 0, and f(x) = -x for x < 0. Wait, that's |x| again, continuous. Hmm.

Wait, let's try a function defined as f(x) = x for x ≥ 0, and f(x) = 0 for x < 0. Wait, we saw that's not convex. What if f(x) = x^2 for x ≥ 0, and f(x) = 0 for x < 0? Is that convex? The second derivative for x > 0 is 2, for x < 0, the function is 0, but the function is not differentiable at 0. Wait, let's check convexity. For any x1, x2 and λ ∈ [0,1], f(λx1 + (1-λ)x2) ≤ λ f(x1) + (1-λ) f(x2). If both x1 and x2 are non-negative, then f is convex there. If one is negative and the other is non-negative, say x1 < 0, x2 ≥ 0, then λx1 + (1-λ)x2 could be negative or non-negative. If it's negative, f(λx1 + (1-λ)x2) = 0 ≤ λ*0 + (1-λ)x2 = (1-λ)x2. If it's non-negative, f(λx1 + (1-λ)x2) = (λx1 + (1-λ)x2)^2 ≤ λ*0 + (1-λ)x2^2? Wait, no, that's not necessarily true. For example, take x1 = -1, x2 = 1, λ = 0.5. Then λx1 + (1-λ)x2 = -0.5 + 0.5 = 0, f(0) = 0. The right-hand side is 0.5*0 + 0.5*1 = 0.5. So 0 ≤ 0.5, which holds. Another example: x1 = -2, x2 = 2, λ = 0.5. Then λx1 + (1-λ)x2 = -1 + 1 = 0, f(0) = 0. RHS is 0.5*0 + 0.5*4 = 2. 0 ≤ 2, holds. But is this function convex? Wait, the function is f(x) = x^2 for x ≥ 0, 0 for x < 0. The epigraph would be the set of (x, t) such that t ≥ x^2 for x ≥ 0, and t ≥ 0 for x < 0. Is this epigraph a convex set? Let's see if the line segment between (x1, t1) and (x2, t2) is in the epigraph. If x1 and x2 are both non-negative, then yes, since the function is convex there. If one is negative and the other is non-negative, say x1 < 0, x2 ≥ 0, then the line segment between (x1, 0) and (x2, x2^2) would need to have t ≥ 0 for all x between x1 and x2. But if the line segment goes through x=0, then at x=0, t should be ≥ 0, which it is, but also for x > 0, t should be ≥ x^2. However, the line segment from (x1, 0) to (x2, x2^2) has t = [(x2^2 - 0)/(x2 - x1)](x - x1) if x1 ≠ x2. At x = 0, t = [x2^2 / (x2 - x1)](-x1). If x1 is negative, x2 is positive, then (x2 - x1) is positive, so t at x=0 is negative times x2^2 over positive, which is negative. But t needs to be ≥ 0 for x=0, which would require t ≥ 0, but here t is negative, which violates the epigraph condition. Therefore, this function is not convex.

Hmm, maybe I need a function that is convex but has a discontinuity in its subdifferential at 0. Wait, the subdifferential of a convex function at a point is non-empty, but if the function has a "corner" at 0, the subdifferential might contain multiple elements. For example, consider f(x) = |x|, which has subdifferential at 0 as [-1, 1], but it's continuous there. Wait, but the question is about a discontinuity at 0. Maybe a function that is not continuous at 0 but is convex. Let's try f(x) = x for x ≥ 0, and f(x) = -x for x ≤ 0. Wait, that's |x| again, continuous. Wait, maybe a function defined as f(x) = x for x ≥ 0, and f(x) = 0 for x < 0. Wait, we saw that's not convex. Alternatively, consider f(x) = x^2 for x ≥ 0, and f(x) = 0 for x < 0. Wait, is this function convex? Let's check the second derivative. For x > 0, f''(x) = 2 > 0, so convex there. For x < 0, the function is flat (f(x)=0), so the derivative is 0. At x=0, the left derivative is 0, the right derivative is 2. So the function is convex because the right derivative is increasing, and the left derivative is flat. Wait, does this function have a discontinuity? The function is continuous everywhere: as x approaches 0 from the right, f(x) approaches 0, and from the left, it's also 0. Wait, but the derivative has a jump discontinuity at 0. However, the function itself is continuous. So maybe the discontinuity is in the derivative, not the function. The question is about the function having a discontinuity at 0, not the derivative.

Wait, let's try a function that is not continuous at 0 but is convex. For example, f(x) = 0 for x ≤ 0, and f(x) = x for x > 0. Is this function convex? Let's check the epigraph. For x ≤ 0, the epigraph is the set of (x, t) with t ≥ 0. For x > 0, it's (x, t) with t ≥ x. The intersection of these two regions would require t ≥ max(0, x). Is this set convex? Take two points: (a, t1) where a ≤ 0, t1 ≥ 0, and (b, t2) where b > 0, t2 ≥ b. The line segment between them is (λa + (1-λ)b, λ t1 + (1-λ) t2). We need λ t1 + (1-λ) t2 ≥ max(0, λa + (1-λ)b). If λa + (1-λ)b ≤ 0, then max is 0, so need λ t1 + (1-λ) t2 ≥ 0. Since t1 ≥ 0 and (1-λ) t2 ≥ (1-λ)b ≥ 0 (because b > 0 and (1-λ) ≤ 1), so their sum is non-negative. If λa + (1-λ)b > 0, then max is λa + (1-λ)b, so need λ t1 + (1-λ) t2 ≥ λa + (1-λ)b. But t1 ≥ 0 = λ*0 + (1-λ)*0 = 0, and t2 ≥ b = (1-λ)b, so λ t1 + (1-λ) t2 ≥ λ*0 + (1-λ)b = (1-λ)b. But λa + (1-λ)b > 0 implies λa > - (1-λ)b. If a ≤ 0, then λa ≤ 0, so λa + (1-λ)b ≥ (1-λ)b - |λa| ≥ (1-λ)b - λ|a| (if a is negative). Hmm, this is getting complicated. Maybe this function is convex. But does it have a discontinuity at 0? The function is 0 for x ≤ 0, x for x > 0. At x=0, the left limit is 0, the right limit is 0, so the function is continuous there. Wait, but maybe I need a different function.

Wait, let's consider the function f(x) = x for x ≥ 0, and f(x) = -x for x ≤ 0. Wait, that's |x|, continuous. Another approach: take a convex function that has a "jump" in its subdifferential at 0. For example, the function f(x) = x for x ≥ 0, and f(x) = 0 for x < 0. Even though we saw earlier it might not be convex, let's check the subdifferential at 0. The subdifferential of a convex function at a point contains all subgradients, which are the slopes of the supporting lines at that point. For f(x) = x for x ≥ 0, the right derivative at 0 is 1. For x < 0, the function is 0, so the left derivative is 0. Therefore, the subdifferential at 0 should be the interval [0, 1], but wait, is that right? Wait, the subdifferential of f at 0 is the set of λ such that f(x) ≥ λ(x - 0) for all x. For x ≥ 0, f(x) = x ≥ λx, which requires λ ≤ 1. For x < 0, f(x) = 0 ≥ λx. Since x < 0, λx ≥ 0 implies λ ≤ 0. Wait, that's a contradiction. Wait, if x < 0, then λx ≥ 0 implies λ ≤ 0 (because x is negative). But for x ≥ 0, λ ≤ 1. So the intersection of these conditions is λ ≤ 0 and λ ≤ 1, so λ ≤ 0. But wait, let's test x = 0.5, f(0.5) = 0.5 ≥ λ*0.5 → λ ≤ 1. For x = -0.5, f(-0.5) = 0 ≥ λ*(-0.5) → 0 ≥ -0.5λ → λ ≥ 0. Wait, now I'm confused. Let's do it properly. The subdifferential at 0 is the set of λ such that f(x) ≥ λx for all x. For x > 0, f(x) = x ≥ λx → λ ≤ 1. For x < 0, f(x) = 0 ≥ λx. Since x < 0, λx ≥ 0 implies λ ≤ 0 (because x is negative, so λ must be ≤ 0 to make λx non-negative). For x = 0, it's trivial. So combining both, λ must be ≤ 0 and λ ≤ 1, so the subdifferential is (-∞, 0]. But wait, if λ = 0, then f(x) ≥ 0 for all x, which is true since f(x) is non-negative. If λ is negative, say λ = -a where a > 0, then f(x) ≥ -a x. For x > 0, this is x ≥ -a x → x(1 + a) ≥ 0, which is true. For x < 0, f(x) = 0 ≥ -a x. Since x is negative, -a x = a |x| ≥ 0, so 0 ≥ a |x| is not true unless |x| = 0, i.e., x = 0. Therefore, there's a mistake here. The correct condition should be that for all x, f(x) ≥ λx. For x > 0, f(x) = x ≥ λx → λ ≤ 1. For x < 0, f(x) = 0 ≥ λx. Since x < 0, λx ≥ 0 implies λ ≤ 0. For x = 0, it's okay. So the intersection is λ ≤ 0. But then for x < 0, if λ is negative, say λ = -1, then f(x) = 0 ≥ -1 * x. Since x is negative, -x is positive, so 0 ≥ positive number is false. Therefore, the subdifferential can't include λ = -1. Wait, this is getting too confusing. Maybe I should pick a simpler example. Let's consider the function f(x) = max{0, x} + max{0, -x}, which is |x|, but that's continuous. Alternatively, take f(x) = x for x ≥ 0, and f(x) = 0 for x < 0. Even if it's not convex, maybe the question allows for a function that's convex except at the discontinuity point, but I need to ensure convexity.

Wait, another idea: the function f(x) = x^2 for x ≥ 0, and f(x) = 0 for x ≤ 0. Is this convex? For x1, x2 ≥ 0, it's convex. For x1 ≤ 0, x2 ≥ 0, the line segment between them: if the midpoint is positive, then f of the midpoint is (midpoint)^2, and the average of f(x1) and f(x2) is (0 + x2^2)/2. We need (midpoint)^2 ≤ (0 + x2^2)/2. But midpoint is (x1 + x2)/2. If x1 is negative and x2 is positive, (x1 + x2)/2 could be positive or negative. If it's positive, then ( (x1 + x2)/2 )^2 ≤ (x2^2)/2? Not necessarily.