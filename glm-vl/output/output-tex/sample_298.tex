<think>Got it, let's try to understand this text step by step. First, the user is comparing two versions of a model (mgn baseline and a version with one-hot encoded edge types). The comparison is across three tasks, and they found no significant advantage in using the explicit edge type partitioning (which is mgn(1H), maybe a typo? Wait, the text says "mgn(1H)"—maybe mgn with one-hot? Wait, the first part says "mgn baseline with a version using the one-hot encoded edge types instead of an explicit edge type partitioning indicated by mgn(1H)". So mgn(1H) is the version with explicit partitioning? Wait, the text says "Both are compared for all three tasks and no significant advantage of the explicit edges partitioning could be found." So the explicit partitioning (mgn(1H)) didn't have an advantage. Then, "for this reason, ggns uses the one-hot encoding, because it is both conceptually simpler and requires less computational power." Wait, maybe "ggns" is a typo? Maybe "the model" or "the approach"? But regardless, the conclusion is that one-hot encoding is used because it's simpler and less computationally intensive. Then, "The mgn baseline still uses explicit edge type partitioning throughout this work, following pfaff2020learning." So the mgn baseline keeps using the explicit partitioning as in the paper by Pfaff et al. 2020.

Now, the question is probably asking for a summary or explanation of this comparison. But since the user provided the text, maybe they want an explanation of the comparison results and the reasoning for choosing one-hot encoding. Let's break down the key points:

1. Comparison between mgn baseline (explicit edge type partitioning) and a version with one-hot encoded edge types.
2. Comparison done across three tasks.
3. No significant advantage found for the explicit partitioning (mgn(1H) version? Wait, the text says "mgn(1H)"—maybe mgn with explicit partitioning is called mgn(1H), but the text says "the version using the one-hot encoded edge types instead of an explicit edge type partitioning indicated by mgn(1H)". Wait, maybe mgn(1H) is the one with explicit partitioning, and the other is the one with one-hot. So when they compared mgn baseline (which is mgn(1H)) with the one-hot version, they found no significant advantage in mgn(1H). Therefore, the new approach (ggns) uses one-hot because it's simpler and less computationally expensive. The mgn baseline continues using explicit partitioning as per Pfaff's work.

So the main points are: the comparison showed no benefit of explicit partitioning, so the new model uses one-hot (simpler, less compute), while the mgn baseline keeps using explicit partitioning following the original paper.

Now, if the question is to explain this comparison and the reasoning, the answer would summarize these points. But since the user provided the text, maybe they want a paraphrase or explanation. But since the user's input is the text, maybe they want a summary of the comparison and the conclusion.</think><answer>To summarize the comparison and reasoning:  

1. **Comparison Setup**: The mgn baseline (using explicit edge type partitioning) was compared with a version using one - hot encoded edge types across three tasks.  
2. **Results of Comparison**: No significant advantage was found for the explicit edge type partitioning (the version indicated as mgn(1H), presumably the one with explicit partitioning).  
3. **Reason for Choosing One - hot Encoding**: For the new approach (referred to as ggns, though likely a shorthand for the model/technique), one - hot encoding was chosen because it is conceptually simpler and requires less computational power.  
4. **Mgn Baseline’s Choice**: The mgn baseline continued using explicit edge type partitioning throughout the work, following the methodology from *pfaff2020learning*.  


In essence, the comparison showed that explicit edge partitioning did not outperform one - hot encoding, so the new method uses the simpler, less computationally intensive one - hot approach, while the mgn baseline retains the original explicit partitioning method.</answer>