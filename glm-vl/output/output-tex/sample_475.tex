<think>Got it, let's try to figure out why the I-stable set S must contain (4,0) given that it already contains (0,0), (0,1), (1,0). First, I need to recall what an I-stable set is. I think an I-stable set (or maybe a convex set?) has the property that if you have points in the set, then certain combinations or sums are also in the set. Wait, the problem mentions "I-stable" which might refer to a set that's stable under some operations, maybe like a cone or a convex set, but let's think step by step.

We have points (0,0), (0,1), (1,0) in S. Let's consider the properties of an I-stable set. If S is I-stable, then perhaps it's closed under addition of vectors or something. Wait, (0,0) + (0,1) = (0,1), which is already in S. (0,0) + (1,0) = (1,0), which is also in S. (0,1) + (1,0) = (1,1), but wait, do we have (1,1) in S? Wait, the problem says the set contains (0,0), (0,1), (1,0), and we need to show it contains (4,0). Maybe we need to use multiple additions.

Let's try adding (1,0) to itself. If S is stable under addition, then (1,0) + (1,0) = (2,0) should be in S. Then adding (2,0) + (1,0) = (3,0), then adding (3,0) + (1,0) = (4,0). Wait, but does the problem state that S is stable under addition of vectors? Or maybe it's a cone with vertex at (0,0), so all linear combinations with non-negative coefficients? But (4,0) would be 4*(1,0), which if S contains (1,0), then scalar multiples might be in S if it's a cone. But wait, the problem says "I-stable" – maybe I need to check the definition again. Alternatively, maybe the set is closed under the operation of taking sums of points, but with some conditions.

Alternatively, think about the points as vectors. The set S contains (0,0), (0,1), (1,0). Let's consider the Minkowski sum or the convex hull. The convex hull of these three points would include all points of the form α(0,0) + β(0,1) + γ(1,0) where α + β + γ = 1 and α, β, γ ≥ 0. But (4,0) would require γ = 4, which is impossible since α + β + γ = 1. So maybe it's not convex. Alternatively, if it's a cone, then it should contain all scalar multiples of its points. Since (1,0) is in S, then 4*(1,0) = (4,0) should be in S if S is a cone with apex at (0,0). But wait, the problem says "I-stable" – maybe a different type of stability.

Wait, another approach: in the proof of lem:X1Construction, the membership is demonstrated in order. The points are (0,0), (0,1), (1,0), then maybe (1,1), (2,0), etc. But how do we get (4,0)? Let's see, starting from (1,0), adding (1,0) gives (2,0), then adding (1,0) again gives (3,0), then adding (1,0) again gives (4,0). If each addition is allowed because S is stable under adding the point (1,0) to itself multiple times, then since (1,0) is in S, (1,0) + (1,0) = (2,0) must be in S, then (2,0) + (1,0) = (3,0) in S, then (3,0) + (1,0) = (4,0) in S. Therefore, S must contain (4,0). Alternatively, if the set is closed under the operation of taking sums of any two points, then (1,0) + (1,0) = (2,0), (1,0) + (2,0) = (3,0), etc., building up to (4,0). Since the problem states that S is I-stable, which likely implies closure under such operations, then (4,0) must be in S.</think><answer>To determine why an I-stable set \( S \) containing \((0,0)\), \((0,1)\), and \((1,0)\) must also contain \((4,0)\), we analyze the properties of I-stable sets. An I-stable set is typically closed under vector addition. Starting with \((1,0) \in S\), repeatedly adding \((1,0)\) to itself:  
- \((1,0) + (1,0) = (2,0) \in S\),  
- \((2,0) + (1,0) = (3,0) \in S\),  
- \((3,0) + (1,0) = (4,0) \in S\).  

Thus, by closure under vector addition, \((4,0)\) must be in \( S \).  

The answer is \(<|begin_of_box|>(4,0)<|end_of_box|>\).</answer>