<think>Got it, let's try to figure out how to approach proving Theorem Green-func-estim, Case 3. First, I need to recall what Green's function estimation usually involves—probably related to potential theory, harmonic functions, or boundary value problems. Case 3 might refer to a specific scenario, like a domain with certain properties, maybe a spiral domain? Wait, the image is a spiral, so maybe the domain is a spiral-shaped region. 

First, let's outline the general approach for Green's function estimates. Green's function G(x, y) for a domain D satisfies ΔG(x, y) = δ_x(y) (the Dirac delta), with G = 0 on ∂D. Estimates often involve bounds on G, like |G(x, y)| ≤ C / dist(x, ∂D), or similar, depending on the domain's geometry. For a spiral domain, the geometry is more complex, so the estimate might involve the distance from the point to the boundary, or some other geometric quantity.

Case 3 might be when the domain has a certain type of boundary, like a spiral with certain curvature properties, or maybe a domain with a spiral-shaped hole, or a multiply connected domain with a spiral outer boundary. The key would be to use the properties of the spiral's geometry to derive the estimate. 

Let's think about the proof structure. Usually, Green's function estimates are derived using the maximum principle, or by comparing with a simpler domain (like a ball or rectangle), or using integral estimates. For a spiral domain, maybe we can use a coordinate system adapted to the spiral, or use a conformal map if the domain is conformally equivalent to a simpler domain, but spirals might not be conformally equivalent to standard domains unless they have specific properties.

Alternatively, if the spiral is a sector with a logarithmic spiral, the distance from a point to the boundary might be related to the angle or the radius. Suppose the domain is a region bounded by a logarithmic spiral r = a e^{bθ} and a straight line or another curve. The distance from a point (r, θ) to the boundary could be estimated based on the spiral's equation. 

For the Green's function estimate, we might need to show that |G(x, y)| ≤ C / d(x, ∂D), where d(x, ∂D) is the distance from x to the boundary. If the spiral has a certain growth rate (like exponential in θ), the distance might be proportional to r or θ, leading to an estimate involving 1/d(x, ∂D). 

Another approach is to use the mean value property or Poisson integral formula if the domain is a region with a known Green's function. But for a general spiral domain, this might not be straightforward. Maybe the case 3 is when the domain is a spiral with a certain boundedness condition, like the spiral is contained within a sector of angle 2π, and the radius goes to infinity, making the domain unbounded. In that case, the Green's function might have different behavior, but estimates still need to hold.

Let's try to structure the proof steps:

1. Define the Green's function G(x, y) for the domain D (spiral-shaped) with Dirichlet boundary condition.
2. Use the maximum principle or comparison principle to relate G(x, y) to the distance from x to the boundary.
3. Estimate the Green's function by comparing it to a function defined on a simpler domain (like a sector or a rectangle) that contains D, and use the known estimates for that simpler domain.
4. Use the geometry of the spiral to bound the distance d(x, ∂D) in terms of the coordinates, then derive the estimate for G(x, y).

For example, if the spiral is a logarithmic spiral, the boundary is given by r = a e^{bθ}, and a point x inside has polar coordinates (r, θ), then the distance to the boundary might be approximately r - a e^{bθ} if x is near the center, but this is a rough idea. Alternatively, if the spiral is the boundary, then for a point inside, the distance to the boundary is the minimum distance to any point on the spiral, which could be estimated using the spiral's equation.

Another angle: Green's function for a domain with a smooth boundary satisfies |G(x, y)| ≤ C / d(x, ∂D) for x, y in D, x ≠ y. For a spiral domain, if the boundary is smooth enough (C^2, etc.), then this estimate holds. The constant C might depend on the domain's geometry, like the maximum curvature or the growth rate of the spiral.

If the spiral has a growth rate such that the radius increases exponentially with θ, then the distance from a point at radius r to the boundary (which is at radius a e^{bθ} when θ is the angle) might be roughly r - a e^{bθ} if θ is the angle of the point, but this requires specific relations. Alternatively, if the domain is a spiral with inner and outer boundaries, but the image shows a single spiral, maybe an unbounded domain.

In summary, the proof would likely involve:

- Establishing the Green's function properties for the spiral domain.
- Using geometric estimates of the distance from points to the boundary.
- Applying standard Green's function estimates (like the maximum principle or integral bounds) to derive the desired inequality, possibly with a constant depending on the spiral's parameters (like the growth rate of the spiral).

Since the image is a spiral, the key is to use the spiral's geometry to bound the distance function, then relate that to the Green's function estimate. Without more specific details on the theorem's exact statement, this is a general approach, but the core idea is connecting the domain's geometry (spiral) to the Green's function's decay rate, which is typically inversely proportional to the distance to the boundary.</think><answer>To prove Theorem *Green-func-estim* for Case 3 (a spiral-shaped domain), we follow these steps, leveraging the geometry of the spiral and standard Green's function estimates:  


### 1. **Define the Green's Function and Domain**  
Let \( D \) be a spiral domain with boundary \( \partial D \), and let \( G(x, y) \) be the Green's function for \( D \), satisfying:  
- \( \Delta G(x, y) = \delta_x(y) \) (Dirac delta at \( x \)),  
- \( G(x, y) = 0 \) for \( x \in \partial D \).  

For a spiral domain, the boundary can often be parameterized (e.g., a logarithmic spiral \( r = a e^{b\theta} \) in polar coordinates).  


### 2. **Relate Green's Function to Distance to Boundary**  
For harmonic functions and Green's functions, standard estimates show \( |G(x, y)| \leq C / d(x, \partial D) \), where \( d(x, \partial D) \) is the distance from \( x \) to the boundary. For a spiral domain, we first estimate \( d(x, \partial D) \) using its geometry.  

If the spiral has a parameterized boundary (e.g., logarithmic spiral), the distance from a point \( x = (r, \theta) \) inside the spiral to the boundary can be bounded. For example, if the spiral grows exponentially with angle (\( r = a e^{b\theta} \)), the distance to the boundary for a point at radius \( r \) and angle \( \theta \) is approximately \( r - a e^{b\theta} \) (for \( r < a e^{b\theta} \), which holds inside the spiral).  


### 3. **Apply Green's Function Estimates**  
Using the maximum principle or comparison with a simpler domain (e.g., a sector or rectangle containing the spiral), we bound \( G(x, y) \). For a smooth spiral boundary (with sufficient regularity, e.g., \( C^2 \)), the Green's function satisfies:  
\[
|G(x, y)| \leq \frac{C}{d(x, \partial D)}
\]  
where \( C \) depends on the spiral's geometry (e.g., its growth rate, curvature bounds).  

If the spiral is unbounded (e.g., a logarithmic spiral extending to infinity), the estimate may involve a logarithmic term or adjust the constant \( C \) to account for the domain's unboundedness, but the core decay rate remains inversely proportional to the distance to the boundary.  


### 4. **Conclusion**  
By relating the spiral's geometry to the distance function and applying standard Green's function estimates, we derive the desired inequality for Case 3. The constant \( C \) in the estimate typically depends on the spiral's parameters (e.g., the exponent \( b \) in the logarithmic spiral \( r = a e^{b\theta} \)) and the smoothness of the boundary.  

**Summary**: The proof leverages the spiral domain's geometry to bound the distance to the boundary, then applies Green's function properties to establish the estimate \( |G(x, y)| \leq C / d(x, \partial D) \), with \( C \) dependent on the spiral's structure.  

The answer is \(<|begin_of_box|>|G(x, y)| \leq \frac{C}{d(x, \partial D)}<|end_of_box|>\), where \( C \) depends on the spiral domain's geometry.</answer>