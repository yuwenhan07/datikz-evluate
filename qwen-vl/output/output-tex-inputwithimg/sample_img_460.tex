\documentclass{article}
\usepackage{tikz}
\usetikzlibrary{arrows.meta, positioning, quotes}

\begin{document}

\begin{tikzpicture}[node distance=1cm, auto]
    % Define node styles
    \tikzset{
        block/.style={
            rectangle,
            draw,
            fill=yellow!30,
            text width=5em,
            text centered,
            rounded corners,
            minimum height=4em
        },
        circle/.style={
            circle,
            draw,
            fill=white,
            inner sep=2pt
        },
        arrow/.style={
            thick,
            ->,
            >=Stealth
        }
    }

    % Nodes
    \node [circle] (h) {$h$};
    \node [circle, below=of h] (h') {$h'$};
    \node [block, right=of h'] (f) {$f_{\phi}$};
    \node [circle, right=of f] (z) {$z$};
    \node [circle, below=of z] (a) {$a$};
    \node [block, right=of a] (q) {$Q_{\omega}, \pi_{\nu}$};
    \node [circle, right=of q] (r) {$r$};
    \node [block, right=of a] (g) {$g_{\theta}$};
    \node [circle, right=of g] (z') {$z'$};

    % Edges
    \draw [arrow] (h) -- (f);
    \draw [arrow] (h') -- (f);
    \draw [arrow] (f) -- (z);
    \draw [arrow] (f) -- (a);
    \draw [arrow] (z) -- (q);
    \draw [arrow] (a) -- (q);
    \draw [arrow] (a) -- (g);
    \draw [arrow] (g) -- (z');
    \draw [arrow, dashed] (f) to[bend left] (z');
    \draw [arrow, dashed] (f) to[bend right] (z');

    % Caption
    \node at (current bounding box.north west) {\textbf{Architecture of our minimalist $\phi_L$ algorithm.}};
    \node at (current bounding box.south west) {The dashed edge indicates the stop-gradient operator; the undirected edges indicate learning from grounded signals of rewards or next latent states.};
\end{tikzpicture}

\end{document}