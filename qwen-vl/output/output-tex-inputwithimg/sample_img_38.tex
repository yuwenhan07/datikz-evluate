\documentclass{article}
\usepackage{tikz}
\usetikzlibrary{graphs, graphdrawing, quotes}
\usegdlibrary{trees}

\begin{document}

\begin{tikzpicture}[every node/.style={draw, ellipse}]
    \node (Type) {Type};
    \node (Size) [below left=of Type] {Size};
    \node (Weight) [below right=of Type] {Weight};
    \node (Texture) [below right=of Weight] {Texture};
    \node (Rigidity) [below right=of Texture] {Rigidity};
    \node (Gripper) [below left=of Type] {Gripper};
    \node (Target) [below right=of Gripper] {Target};
    \node (Success) [below right=of Target] {Success};
    \node (Robot) [below left=of Gripper] {Robot};
    \node (Goal) [below right=of Target] {Goal};

    \graph [tree layout, grow'=right, level distance=2cm, sibling distance=1.5cm]
    {
        Type -> [bend left] Size,
        Type -> [bend left] Weight,
        Type -> [bend left] Texture,
        Type -> [bend left] Rigidity,
        Size -> [bend left] Gripper,
        Weight -> [bend left] Gripper,
        Texture -> [bend left] Gripper,
        Rigidity -> [bend left] Gripper,
        Gripper -> Target,
        Target -> Success,
        Robot -> Gripper,
        Goal -> Target;
    };
\end{tikzpicture}

An example of the relationships between the operator-identified variables for a pick-and-place task. Each edge corresponds to a direct causal relationship between the connected variables. Changes in the source of an edge have a direct consequence on the target of an edge.
\end{document}