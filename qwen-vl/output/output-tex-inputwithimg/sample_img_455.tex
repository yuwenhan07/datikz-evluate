\documentclass{article}
\usepackage{tikz}
\usetikzlibrary{matrix}

\begin{document}

\begin{figure}[ht]
    \centering
    \begin{tikzpicture}
        % Define the size of the grid
        \def\gridsize{4}
        
        % Draw the grid for 1 antenna/array
        \matrix (m1) [matrix of nodes, nodes={draw, minimum size=0.5cm, anchor=center}, row sep=-\pgflinewidth, column sep=-\pgflinewidth] {
            & & & \\
            & & & \\
            & & & \\
            & & & \\
        };
        \foreach \i in {1,...,\gridsize} {
            \foreach \j in {1,...,\gridsize} {
                \node[fill=yellow!30] at (m1-\i-\j) {};
            }
        }
        
        % Draw the grid for 2 antennas/array
        \matrix (m2) [matrix of nodes, nodes={draw, minimum size=0.5cm, anchor=center}, row sep=-\pgflinewidth, column sep=-\pgflinewidth, right=of m1] {
            & & & \\
            & & & \\
            & & & \\
            & & & \\
        };
        \foreach \i in {1,...,\gridsize} {
            \foreach \j in {1,...,\gridsize} {
                \node[fill=yellow!30] at (m2-\i-\j) {};
            }
        }
        
        % Draw the grid for 4 antennas/array
        \matrix (m4) [matrix of nodes, nodes={draw, minimum size=0.5cm, anchor=center}, row sep=-\pgflinewidth, column sep=-\pgflinewidth, above=of m2] {
            & & & \\
            & & & \\
            & & & \\
            & & & \\
        };
        \foreach \i in {1,...,\gridsize} {
            \foreach \j in {1,...,\gridsize} {
                \node[fill=yellow!30] at (m4-\i-\j) {};
            }
        }
        
        % Draw the grid for 8 antennas/array
        \matrix (m8) [matrix of nodes, nodes={draw, minimum size=0.5cm, anchor=center}, row sep=-\pgflinewidth, column sep=-\pgflinewidth, right=of m4] {
            & & & \\
            & & & \\
            & & & \\
            & & & \\
        };
        \foreach \i in {1,...,\gridsize} {
            \foreach \j in {1,...,\gridsize} {
                \node[fill=yellow!30] at (m8-\i-\j) {};
            }
        }
        
        % Draw the grid for 16 antennas/array
        \matrix (m16) [matrix of nodes, nodes={draw, minimum size=0.5cm, anchor=center}, row sep=-\pgflinewidth, column sep=-\pgflinewidth, right=of m8] {
            & & & \\
            & & & \\
            & & & \\
            & & & \\
        };
        \foreach \i in {1,...,\gridsize} {
            \foreach \j in {1,...,\gridsize} {
                \node[fill=yellow!30] at (m16-\i-\j) {};
            }
        }
        
        % Add labels
        \node at (m1.north west) [above left] {1 antenna/array};
        \node at (m2.north west) [above left] {2 antennas/array};
        \node at (m4.north west) [above left] {4 antennas/array};
        \node at (m8.north west) [above left] {8 antennas/array};
        \node at (m16.north west) [above left] {16 antennas/array};
        \node at (m4.north) [above] {4 antennas/array};
        \node at (m8.north) [above] {8 antennas/array};
        \node at (m16.north) [above] {16 antennas/array};
    \end{tikzpicture}
    
    \caption{Examples of the evaluated rectangular \gls{hmd} antenna sub-array shapes (active antennas are marked yellow). For each sub-array, all possible locations within the full $4\times4$ array are included in the analysis.}
    \label{fig:antenna_sub_arrays}
\end{figure}

\end{document}