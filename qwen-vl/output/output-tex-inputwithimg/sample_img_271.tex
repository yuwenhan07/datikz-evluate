\documentclass{article}
\usepackage{amsmath}
\usepackage{tikz}
\usetikzlibrary{matrix}

\begin{document}

\begin{equation}
\begin{tikzpicture}[baseline=(current bounding box.center)]
    \matrix (m) [matrix of math nodes,row sep=3em,column sep=4em,minimum width=2em] {
        n(n+1)/2 & \mathcal{S} \\
        & p \\
        };
    \path[-stealth]
        (m-1-1) edge node [left] {$q$} (m-1-2)
                edge node [below] {$p$} (m-2-2)
        (m-1-2) edge node [above] {} (m-2-2);
    
    \node[right=2cm of m-1-2] (arrow) {};
    \draw[->] (arrow) -- ++(0.5,0) node[above] {};

    \matrix (n) [matrix of math nodes,row sep=3em,column sep=4em,minimum width=2em,right=2cm of arrow] {
        & \mathcal{Q} & q \\
        d & \mathcal{N} & n(n+1)/2 \\
        & \mathcal{P} & p \\
        };
    \path[-stealth]
        (n-1-2) edge node [above] {$D$} (n-2-2)
                edge node [below] {$D$} (n-3-2)
        (n-2-1) edge[out=180,in=90] node [left] {} (n-2-2)
                edge[out=180,in=-90] node [left] {} (n-2-2)
        (n-2-2) edge node [above] {} (n-2-3)
                edge node [below] {} (n-3-2)
        (n-3-2) edge node [below] {$p$} (n-3-3);
\end{tikzpicture}
\end{equation}

Diagrammatic representation of Eq.~\eqref{eq:TN} showing the decomposition of the full tensor $\mathcal{S}$ (left) into the three tensors $\mathcal{P}$, $\mathcal{N}$, and $\mathcal{Q}$ (right). Legs are labeled with their dimension.

\end{document}