\documentclass{article}
\usepackage{tikz}
\usetikzlibrary{decorations.pathmorphing}

\begin{document}

\begin{figure}[h]
    \centering
    \begin{tikzpicture}[scale=1.5]
        % Draw the AdS-Vaidya spacetime diagram
        \draw (0,0) -- (4,0) -- (4,4) -- (0,4) -- cycle;
        \draw[thick] (3.8,3.8) -- (0.2,0.2);
        
        % Add a wavy line at the top right corner
        \draw[decorate, decoration={snake, amplitude=1pt, segment length=5pt}] (3.9,3.9) -- (4,4);
        
        % Add a label for the integral
        \node at (3.5, 3.5) {$\int dx O(x,t)$};
    \end{tikzpicture}
    
    \caption{Vaidya geometry. Right side boundary denotes the asymptotic boundary of the AdS-Vaidya spacetime. Outside the mass shell is the black hole geometry. Inside the mass shell is Vacuum AdS in Poincaré patch.}
    \label{fig:vaidya_geometry}
\end{figure}

\end{document}