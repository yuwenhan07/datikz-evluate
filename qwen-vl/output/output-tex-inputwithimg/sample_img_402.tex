\documentclass{article}
\usepackage{tikz}
\usetikzlibrary{matrix}

\begin{document}

\begin{figure}[h]
    \centering
    \begin{tikzpicture}[scale=0.7]
        % Define styles for nodes
        \tikzset{
            mynode/.style={draw=red, fill=red!20, minimum width=1cm, minimum height=1cm},
            myedge/.style={draw=blue, thick},
            myrededge/.style={draw=red, thick},
            mycircle/.style={draw=blue, fill=blue!20, circle, inner sep=0pt, minimum size=0.5cm}
        }
        
        % Draw the initial graph
        \matrix (A) [matrix of nodes, row sep=0.5cm, column sep=0.5cm] {
            \mynode & & & \\
            & \mynode & & \\
            & & \mynode & \\
            & & & \mynode \\
            & & & \\
            & & & \\
            & & & \\
            & & & \\
            & & & \\
            & & & \\
        };
        \foreach \i in {1,...,9} {
            \draw[myedge] (A-\i-1) -- (A-\i-2);
            \draw[myedge] (A-\i-2) -- (A-\i-3);
            \draw[myedge] (A-\i-3) -- (A-\i-4);
        }
        
        % Draw the first transformation
        \matrix (B) [matrix of nodes, right=1cm of A, row sep=0.5cm, column sep=0.5cm] {
            \mynode & & & \\
            & \mynode & & \\
            & & \mynode & \\
            & & & \mynode \\
            & & & \\
            & & & \\
            & & & \\
            & & & \\
            & & & \\
        };
        \foreach \i in {1,...,4} {
            \draw[myrededge] (B-\i-1) -- (B-\i-2);
            \draw[myrededge] (B-\i-2) -- (B-\i-3);
            \draw[myrededge] (B-\i-3) -- (B-\i-4);
        }
        \draw[myedge] (B-5-1) -- (B-5-2);
        \draw[myedge] (B-5-2) -- (B-5-3);
        \draw[myedge] (B-5-3) -- (B-5-4);
        
        % Draw the second transformation
        \matrix (C) [matrix of nodes, right=1cm of B, row sep=0.5cm, column sep=0.5cm] {
            \mynode & & & \\
            & \mynode & & \\
            & & \mynode & \\
            & & & \mynode \\
            & & & \\
            & & & \\
            & & & \\
            & & & \\
            & & & \\
        };
        \foreach \i in {1,...,3} {
            \draw[myrededge] (C-\i-1) -- (C-\i-2);
            \draw[myrededge] (C-\i-2) -- (C-\i-3);
        }
        \draw[myedge] (C-4-1) -- (C-4-2);
        \draw[myedge] (C-4-2) -- (C-4-3);
        
        % Draw the third transformation
        \matrix (D) [matrix of nodes, right=1cm of C, row sep=0.5cm, column sep=0.5cm] {
            \mynode & & & \\
            & \mynode & & \\
            & & \mynode & \\
            & & & \mynode \\
            & & & \\
            & & & \\
            & & & \\
            & & & \\
            & & & \\
        };
        \foreach \i in {1,...,2} {
            \draw[myrededge] (D-\i-1) -- (D-\i-2);
        }
        \draw[myedge] (D-3-1) -- (D-3-2);
        
        % Draw the fourth transformation
        \matrix (E) [matrix of nodes, right=1cm of D, row sep=0.5cm, column sep=0.5cm] {
            \mynode & & & \\
            & \mynode & & \\
            & & \mynode & \\
            & & & \mynode \\
            & & & \\
            & & & \\
            & & & \\
            & & & \\
            & & & \\
        };
        \foreach \i in {1,...,1} {
            \draw[myrededge] (E-\i-1) -- (E-\i-2);
        }
        
        % Draw the final transformation
        \matrix (F) [matrix of nodes, right=1cm of E, row sep=0.5cm, column sep=0.5cm] {
            \mynode & & & \\
            & \mynode & & \\
            & & \mynode & \\
            & & & \mynode \\
            & & & \\
            & & & \\
            & & & \\
            & & & \\
            & & & \\
        };
        \foreach \i in {1,...,1} {
            \draw[myrededge] (F-\i-1) -- (F-\i-2);
        }
        
        % Draw the final circles
        \matrix (G) [matrix of nodes, right=1cm of F, row sep=0.5cm, column sep=0.5cm] {
            \mynode & & & \\
            & \mynode & & \\
            & & \mynode & \\
            & & & \mynode \\
            & & & \\
            & & & \\
            & & & \\
            & & & \\
            & & & \\
        };
        \foreach \i in {1,...,1} {
            \draw[mycircle] (G-\i-1) circle (0.25cm);
            \draw[mycircle] (G-\i-2) circle (0.25cm);
        }
        
        % Draw the arrows
        \draw[->] (A-5-4) -- node[above] {$\Rightarrow$} (B-5-4);
        \draw[->] (B-5-4) -- node[above] {$\Rightarrow$} (C-5-4);
        \draw[->] (C-5-4) -- node[above] {$\Rightarrow$} (D-5-4);
        \draw[->] (D-5-4) -- node[above] {$\Rightarrow$} (E-5-4);
        \draw[->] (E-5-4) -- node[above] {$\Rightarrow$} (F-5-4);
        \draw[->] (F-5-4) -- node[above] {$\Rightarrow$} (G-5-4);
        
        % Caption
        \node at (current bounding box.north west) [anchor=north west, align=center] {
            Shown is the upper-bounding construction with $k = 10$, $l = 3$, $\delta = 0.5$. In each step we replace the blue edges with as many red edges of $\frac{1}{1 + \delta}$ times the size as possible. Then we pick the $l$ red edges that the algorithm puts the most weight on, make those the new blue edges and repeat until only singleton edges are left.
        };
    \end{tikzpicture}
\end{figure}

\end{document}