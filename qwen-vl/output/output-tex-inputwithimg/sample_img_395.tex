\documentclass{article}
\usepackage{amsmath}
\usepackage{graphicx}
\usepackage{subcaption}
\usepackage{tikz}
\usetikzlibrary{positioning, arrows.meta}

\begin{document}

\begin{figure}[h]
    \centering
    \begin{subfigure}{0.32\textwidth}
        \centering
        \caption{(a)}
        \label{fig:sub1}
        \includegraphics[width=\linewidth]{example-image-a}
    \end{subfigure}
    \begin{subfigure}{0.32\textwidth}
        \centering
        \caption{(b)}
        \label{fig:sub2}
        \includegraphics[width=\linewidth]{example-image-b}
    \end{subfigure}
    \begin{subfigure}{0.32\textwidth}
        \centering
        \caption{(c)}
        \label{fig:sub3}
        \includegraphics[width=\linewidth]{example-image-c}
    \end{subfigure}
    
    \caption{Spanning tree matching decoder with a $[[85,1,7]]$ surface code. 
    a) Three $\M{Z}$ channel errors occur on the lattice. Exited ancillas are depicted in red.
    b) Two alternative \acp{MST} obtained with the nearest ghost ancilla to the left (above) and to the right (below) boundary, respectively.
    c) Resulting $\mathcal{E}$ from the tree matching procedure.}
    \label{fig:my_label}
\end{figure}

\end{document}