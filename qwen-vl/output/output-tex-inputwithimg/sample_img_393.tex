\documentclass{article}
\usepackage{amsmath}
\usepackage{tikz}
\usetikzlibrary{arrows.meta}

\begin{document}

\begin{figure}[h]
    \centering
    \begin{tikzpicture}[scale=0.8]
        % Draw the horizontal line
        \draw[->] (0,0) -- (12,0);
        
        % Mark the positions on the x-axis
        \foreach \x/\label in {0/0, 3/0+\frac{1}{m}, 6/1, 9/1-\frac{1}{m}, 12/1}
            \draw (\x,0.1) -- (\x,-0.1) node[below] {\label};
        
        % Draw the segments for b = 0 and b = 1
        \draw[->] (0,0.5) -- (6,0.5) node[midway,above] {$b \equiv 0$};
        \draw[->] (6,0.5) -- (12,0.5) node[midway,above] {$b \equiv 1$};
        
        % Draw the distribution of b = 0
        \foreach \x/\color in {0.5/green, 1.5/green, 2.5/green}
            \fill[\color] (\x,0.5) circle (0.1);
        
        % Draw the distribution of b = 1
        \foreach \x/\color in {7.5/blue, 8.5/blue, 9.5/blue}
            \fill[\color] (\x,0.5) circle (0.1);
        
        % Add the text for the distribution
        \node at (3,0.2) {$\{\textcolor{green}{*}\}$ - valid distributions of candidate values $b$ across different processes};
        \node at (9,0.2) {$\{\textcolor{blue}{*}\}$ - invalid distribution of the same values; ratios at different processes differ by more than $\frac{1}{m}$};
        
        % Add the text for the probability distribution
        \node at (6,1.5) {$b \sim \mathcal{B}\left(1,\frac{1}{7}\right)$};
    \end{tikzpicture}
    \caption{A picture explaining the thresholds in a single execution of the biased-majority-voting subroutine, see lines~\ref{line:if-1}-\ref{line:if-random} in Algorithm~\ref{alg:opt-omissions}. Different colors represent different outcomes (each obtained in a different epoch) of the counting of candidate values in preceding lines~\ref{line:group-bits-aggr} and~\ref{line:sum_ones_zeros}.}
    \label{fig:bias_majority_voting_thresholds}
\end{figure}

\end{document}