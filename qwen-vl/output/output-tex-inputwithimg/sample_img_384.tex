\documentclass{article}
\usepackage{amsmath}
\usepackage{amssymb}
\usepackage{tikz}
\usetikzlibrary{patterns}

\begin{document}

\begin{figure}[h]
    \centering
    \begin{tikzpicture}[scale=1.5]
        % Draw the axes
        \draw[->] (-3,0) -- (3,0) node[right] {$\beta^*$};
        \draw[->] (0,-3) -- (0,3) node[above] {$-\infty$};

        % Draw the shaded region
        \fill[pattern=north east lines, pattern color=gray!50] (-3,-3) rectangle (3,3);

        % Draw the boundary lines
        \draw (-3,0) -- (3,0);
        \draw (0,-3) -- (0,3);
        \draw (-3,3) -- (3,3);
        \draw (-3,-3) -- (3,-3);

        % Draw the curve
        \draw[red, thick, domain=-3:3, samples=100] plot (\x, {1/(\x+1)});

        % Mark points
        \node at (0, 0) [circle, fill, inner sep=1pt]{};
        \node at (0, 0) [below right] {$\beta(\alpha)$};
        \node at (-1, -1) [circle, fill, inner sep=1pt]{};
        \node at (-1, -1) [left] {$\mathcal{D}_{\varphi}$};
        \node at (-1, 0) [circle, fill, inner sep=1pt]{};
        \node at (-1, 0) [below left] {$\alpha^*$};

        % Annotations
        \node at (1.5, 1.5) [rectangle, draw, fill=white, inner sep=2pt] {$\beta(\alpha)$};
        \node at (1.5, -1.5) [circle, fill, inner sep=1pt]{};
        \node at (1.5, -1.5) [right] {$\beta^*$};
    \end{tikzpicture}
    \caption{In Lemma \ref{Lem_Aux4}, the construction of $\beta(\alpha)$ changes slightly compared to Lemma \ref{Lem_AuxL2}.}
    \label{fig:beta_alpha}
\end{figure}

\end{document}