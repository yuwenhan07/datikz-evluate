\documentclass{article}
\usepackage{amsmath}
\usepackage{tikz}
\usetikzlibrary{arrows.meta}

\begin{document}

\begin{figure}[h]
    \centering
    \begin{tikzpicture}[node distance=1cm, auto]
        % Define node styles
        \tikzset{
            vertex/.style={circle, draw, minimum size=0.5cm},
            arrow/.style={-{Stealth[scale=1.5]}, dashed, thick},
            message/.style={font=\footnotesize}
        }

        % Nodes
        \node[vertex] (v1) at (0, 0) {$v_1$};
        \node[vertex] (v2) at (2, 0) {$v_2$};
        \node[vertex] (v3) at (1, -1) {$v_3$};

        % Edges with messages
        \draw[arrow] (v1) -- node[above, message] {${}^+{}$} (v3);
        \draw[arrow] (v2) -- node[above, message] {${}^-{}$} (v3);

        % Messages
        \node[below left=0.2cm and 0.5cm of v1] (msg1) {\small ${}^+{} \otimes {}^+{} = {}^+$};
        \node[below right=0.2cm and 0.5cm of v2] (msg2) {\small ${}^+{} \otimes {}^-{} = {}^-$};
        \node[below=0.5cm of v3] (msg3) {\small ${}^+{} \oplus {}^-{} = {}?$};

        % Labels for the signs
        \node[left=0.5cm of msg1] {\small ${}^+{}$};
        \node[right=0.5cm of msg2] {\small ${}^+{}$};
        \node[right=0.5cm of msg3] {\small ${}^-$};
    \end{tikzpicture}
    \caption{An example for propagating the sign of \( v_1 \) to \( v_3 \). During the propagation, all other parents of \( v_3 \) (here only \( v_2 \)) are also taken into account. Both \( v_1 \) and \( v_2 \) are assigned the sign '\( + \)' and therefore \( v_3 \) receives the two messages \( \text{'\( + \)'} \otimes \text{'\( + \)'} = \text{'\( + \)'} \) and \( \text{'\( + \)'} \otimes \text{'\( - \)'} = \text{'\( - \)'} \), which are then combined using the sign addition operator to obtain \( \text{'\( + \)'} \oplus \text{'\( - \)'} = \text{'\( ? \)'} \) as a new sign for \( v_3 \).}
    \label{fig:sign_propagation}
\end{figure}

\end{document}