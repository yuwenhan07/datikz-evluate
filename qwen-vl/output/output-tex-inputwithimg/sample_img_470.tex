\documentclass{article}
\usepackage{tikz}
\usetikzlibrary{arrows.meta}

\begin{document}

\begin{figure}[h]
    \centering
    \begin{tikzpicture}[scale=1.5]
        % Draw the left hyperbolic quadrilateral
        \draw (0,0) -- (2,0) -- (2,2) -- (0,2) -- cycle;
        \draw[thick] (0,0) -- (0,2);
        \draw[thick] (2,0) -- (2,2);
        
        % Draw the right hyperbolic quadrilateral
        \draw (3,0) -- (5,0) -- (5,2) -- (3,2) -- cycle;
        \draw[thick] (3,0) -- (3,2);
        \draw[thick] (5,0) -- (5,2);
        
        % Draw the red arrows
        \draw[red, ultra thick, ->] (2,1) -- (4,1);
        
        % Label the vertices
        \node at (0.5, 0.5) {$Q''$};
        \node at (2.5, 0.5) {$Q'$};
        \node at (3.5, 0.5) {$P'$};
        \node at (1.5, 2.5) {$S'$};
        \node at (2.5, 2.5) {$R'$};
        \node at (0.5, 2.5) {$S''$};
        \node at (1.5, 0.5) {$P''$};
        \node at (2.5, 0.5) {$R''$};
        
        % Draw the conical singularities
        \draw[dotted, thick] (6,1) circle (0.5);
        \draw[thick] (6,1) -- (7,1);
        \draw[thick] (6,1) -- (6.5, 1.5);
        \draw[thick] (6,1) -- (6.5, 0.5);
        
        % Label the conical singularities
        \node at (6.5, 1.5) {$R$};
        \node at (6.5, 0.5) {$S$};
        
        % Draw the arrow indicating the identification
        \draw[->, thick] (5.5, 1) -- (7.5, 1);
        
        % Draw the hyperbolic disk
        \draw (8,0) circle (1);
        \node at (9, -0.5) {$B_L$};
        \node at (10, 0) {$B_R$};
        
        % Label the points on the hyperbolic disk
        \fill (8,0) circle (0.05);
        \fill (9,0) circle (0.05);
        \fill (10,0) circle (0.05);
        
        % Add labels for the angles
        \node at (1.5, 2.25) {$\beta$};
        \node at (2.5, 2.25) {$\alpha$};
        \node at (3.5, 2.25) {$\alpha$};
        \node at (4.5, 2.25) {$\beta$};
    \end{tikzpicture}
    \caption{The gravitational region of the wormhole geometry can be cut into $8n+1$ number of pieces. In which $8n$ pieces have the geometry of a hyperbolic disk with a cut connecting two conical singularities, one at point $R$ and another at point $S$. The conical singularity at $R$ has opening angle $2\alpha$ and the conical singularity at $S$ has opening angle $2\beta$. This geometry can be obtained by identifying the edges $P'R'$ and $P''R''$ and edges $Q'S'$ and $Q''S''$ of a pair of hyperbolic quadrilaterals. The angle at the vertices $R'$ and $R''$ are $\alpha$ and the angle at the vertices $S'$ and $S''$ are $\beta$.}
    \label{fig:wormhole_geometry}
\end{figure}

\end{document}