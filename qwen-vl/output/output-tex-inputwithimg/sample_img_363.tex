\documentclass{article}
\usepackage{amsmath}
\usepackage{tikz}
\usetikzlibrary{matrix}

\begin{document}

\begin{figure}[h]
    \centering
    \begin{subfigure}{0.3\textwidth}
        \centering
        \begin{tikzpicture}
            \matrix (m) [matrix of math nodes,
                         nodes={minimum width=2cm, minimum height=1cm, text centered},
                         column sep=-\pgflinewidth,
                         row sep=-\pgflinewidth]
            {
                & x_1 & y \\
                x_1 & w_h & w_l \\
                y & w_h & w_l \\
            };
            \draw[->] (m-2-2) -- (m-2-1);
            \draw[->] (m-3-2) -- (m-3-1);
            \draw[->] (m-2-1) -- (m-1-2);
            \draw[->] (m-3-1) -- (m-1-2);
        \end{tikzpicture}
        \caption{}
    \end{subfigure}
    \begin{subfigure}{0.3\textwidth}
        \centering
        \begin{tikzpicture}
            \matrix (m) [matrix of math nodes,
                         nodes={minimum width=2cm, minimum height=1cm, text centered},
                         column sep=-\pgflinewidth,
                         row sep=-\pgflinewidth]
            {
                & x_1 & x_2 & y \\
                x_1 & w_l & w_h & w_l \\
                x_2 & w_h & w_l & w_l \\
                y & w_h & w_l & w_l \\
            };
            \draw[->] (m-2-2) -- (m-2-1);
            \draw[->] (m-3-2) -- (m-3-1);
            \draw[->] (m-4-2) -- (m-4-1);
            \draw[->] (m-2-1) -- (m-1-2);
            \draw[->] (m-3-1) -- (m-1-2);
            \draw[->] (m-4-1) -- (m-1-2);
            \draw[->] (m-2-2) -- (m-2-3);
            \draw[->] (m-3-2) -- (m-3-3);
            \draw[->] (m-4-2) -- (m-4-3);
            \draw[->] (m-2-3) -- (m-1-4);
            \draw[->] (m-3-3) -- (m-1-4);
            \draw[->] (m-4-3) -- (m-1-4);
        \end{tikzpicture}
        \caption{}
    \end{subfigure}
    \begin{subfigure}{0.3\textwidth}
        \centering
        \begin{tikzpicture}
            \matrix (m) [matrix of math nodes,
                         nodes={minimum width=2cm, minimum height=1cm, text centered},
                         column sep=-\pgflinewidth,
                         row sep=-\pgflinewidth]
            {
                x_1 & x_2 & x_3 & y \\
                x_1, x_1 & w_l & w_h & w_l & w_l \\
                x_1, x_2 & w_l & w_l & w_h & w_l \\
                x_1, x_3 & w_h & w_l & w_l & w_l \\
                x_1, y & w_h & w_l & w_l & w_l \\
                \vdots & \vdots & \vdots & \vdots & \vdots \\
            };
            \draw[->] (m-2-2) -- (m-2-1);
            \draw[->] (m-3-2) -- (m-3-1);
            \draw[->] (m-4-2) -- (m-4-1);
            \draw[->] (m-5-2) -- (m-5-1);
            \draw[->] (m-2-1) -- (m-1-2);
            \draw[->] (m-3-1) -- (m-1-2);
            \draw[->] (m-4-1) -- (m-1-2);
            \draw[->] (m-5-1) -- (m-1-2);
            \draw[->] (m-2-2) -- (m-2-3);
            \draw[->] (m-3-2) -- (m-3-3);
            \draw[->] (m-4-2) -- (m-4-3);
            \draw[->] (m-5-2) -- (m-5-3);
            \draw[->] (m-2-3) -- (m-1-4);
            \draw[->] (m-3-3) -- (m-1-4);
            \draw[->] (m-4-3) -- (m-1-4);
            \draw[->] (m-5-3) -- (m-1-4);
            \draw[->] (m-2-4) -- (m-1-5);
            \draw[->] (m-3-4) -- (m-1-5);
            \draw[->] (m-4-4) -- (m-1-5);
            \draw[->] (m-5-4) -- (m-1-5);
        \end{tikzpicture}
        \caption{}
    \end{subfigure}
    \caption{Transition probability matrices for node (a), edge (b), and path ($l=2$) (c) hypotheses, where $w_h \geq w_l > 0$ denote transition probabilities. $x_i$ represents nodes in $\mathcal{G}$ satisfying the $i$-th node modifier on $\mathcal{P}$, while $y$ represents nodes not satisfying any node modifier on $\mathcal{P}$. (a) and (b) involve 1st-order random walks, whereas (c) involves 2nd-order random walks because the probability of selecting the next node depends on both the current and previous nodes.}
\end{figure}

\end{document}