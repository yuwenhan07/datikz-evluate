\documentclass{article}
\usepackage{amsmath}
\usepackage{tikz}
\usetikzlibrary{arrows.meta}

\begin{document}

\begin{figure}[h]
    \centering
    \begin{tikzpicture}[scale=1.5]
        % Axes
        \draw[->] (-3,0) -- (3,0) node[right] {$\text{Re}(z)=\tau$};
        \draw[->] (0,-1.5) -- (0,1.5) node[above] {$\text{Im}(z)=\pi$};
        \draw[->] (3,0) -- (4,0) node[right] {$\text{Im}(z)=0$};
        
        % Dashed lines
        \draw[dashed] (-2,0) -- (-2,1.5);
        \draw[dashed] (0,0) -- (0,1.5);
        \draw[dashed] (2,0) -- (2,1.5);
        
        % Curves
        \draw[blue, thick, domain=-2:2, smooth, variable=\x, samples=100] plot ({\x},{0.5*\x^2-1});
        \node at (1.5, 0.75) {$\nu_2$};
        
        \draw[red, thick, domain=-2:2, smooth, variable=\x, samples=100] plot ({\x},{0.25*\x^2-0.5});
        \node at (1.5, -0.25) {$\nu_1$};
        
        % Labels
        \node at (3.5, 0.75) {$V(a_n,b_{n+j},M_l)$};
        \node at (3.5, -0.25) {$\omega_{\lambda_{g,s}}$};
    \end{tikzpicture}
    \caption{The curve $\nu_2$ passes twice through $V(a_n,b_{n+j},M_l)$.}
    \label{fig:curves}
\end{figure}

\end{document}