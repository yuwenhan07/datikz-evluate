\documentclass{article}
\usepackage{tikz}
\usetikzlibrary{angles, quotes}

\begin{document}

\begin{figure}[h]
    \centering
    \begin{tikzpicture}[scale=1.5]
        % Draw the circle representing the drop
        \draw (0,0) circle (1);
        
        % Draw the center of the circle
        \fill (0,0) circle (0.05);
        
        % Draw the radius vector
        \draw[->] (0,0) -- (1,0) node[right] {$r$};
        
        % Draw the angle theta
        \pic [draw, ->, "$\theta$", angle eccentricity=1.2] {angle = 0--1--0};
        
        % Draw the vertical line from the center to the free surface
        \draw (0,-1) -- (0,0) node[left] {$z$};
        
        % Draw the x-axis
        \draw[dotted] (-1.5,0) -- (1.5,0) node[below] {$x$};
        
        % Draw the wavy free surface
        \draw[samples=100,domain=-1.5:1.5,smooth] plot (\x,{sin(\x r)});
        
        % Draw the dotted line representing the unperturbed free surface
        \draw[dotted] (-1.5,-1) -- (1.5,-1);
        
        % Label the origin
        \node at (0,-1.2) {Origin};
    \end{tikzpicture}
    \caption{Schematic of model domain denoting two frames of reference points. Origin for the full frame is Cartesian $(x,z)$ and sets the origin directly beneath the south pole of the drop and aligned to an unperturbed bath free surface. The droplet will be evolved in a moving frame of reference with polar coordinates $(r,\theta)$ centered at the center of mass of a drop whose free surface lies at $r=R_0$, and $\theta=0$ pointing vertically downwards towards the bath.}
    \label{fig:model_domain}
\end{figure}

\end{document}