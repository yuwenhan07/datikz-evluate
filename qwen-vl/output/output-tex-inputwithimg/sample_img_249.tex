\documentclass{article}
\usepackage{amsmath}

\begin{document}

\begin{table}[h]
\centering
\begin{tabular}{c|c}
\hline
\multicolumn{2}{c}{$U(4)$} \\
\hline
\multirow{2}{*}{$T^x_0 = \overrightarrow{\tau}$} & 
\multirow{2}{*}{$\begin{pmatrix}
1 & 0 & 0 & 0 \\
0 & 1 & 0 & 0 \\
0 & 0 & 1 & 0 \\
0 & 0 & 0 & 1
\end{pmatrix}$} \\
& 
$\begin{pmatrix}
\not\!\!{\tau} & 0 \\
0 & \not\!\!{\tau}
\end{pmatrix}$ \\
\hline
\multirow{2}{*}{$T^x_F = \frac{1}{\sqrt{2}} \begin{pmatrix}
1 & 0 & 0 & 0 \\
0 & -1 & 0 & 0 \\
0 & 0 & 1 & 0 \\
0 & 0 & 0 & -1
\end{pmatrix}$} &
\multirow{2}{*}{$\begin{pmatrix}
0 & 1 & 0 & 0 \\
1 & 0 & 0 & 0 \\
0 & 0 & 0 & 0 \\
0 & 0 & 0 & 1
\end{pmatrix}$} \\
& 
$\begin{pmatrix}
a & -i & 0 & 0 \\
-i & a & 0 & 0 \\
0 & 0 & 0 & 0 \\
0 & 0 & -i & 0
\end{pmatrix}$ \\
\hline
\multirow{2}{*}{$T^x_F = \frac{1}{2} \begin{pmatrix}
0 & 0 & 1 & 0 \\
0 & 0 & 0 & 0 \\
1 & 0 & 0 & 0 \\
0 & 0 & 0 & 0
\end{pmatrix}$} &
\multirow{2}{*}{$\begin{pmatrix}
0 & 0 & 0 & 0 \\
0 & 0 & 1 & 0 \\
0 & 1 & 0 & 0 \\
1 & 0 & 0 & 0
\end{pmatrix}$} \\
& 
$\begin{pmatrix}
0 & 0 & 0 & 0 \\
0 & 0 & 0 & 1 \\
0 & 0 & 0 & 0 \\
0 & 1 & 0 & 0
\end{pmatrix}$ \\
\hline
\multirow{2}{*}{$T^x_F = \frac{1}{2} \begin{pmatrix}
0 & 0 & 0 & i \\
0 & 0 & 0 & 0 \\
0 & 0 & 0 & 0 \\
0 & 0 & 0 & 0
\end{pmatrix}$} &
\multirow{2}{*}{$\begin{pmatrix}
0 & 0 & 0 & i \\
0 & 0 & 0 & 0 \\
0 & 0 & 0 & 0 \\
i & 0 & 0 & 0
\end{pmatrix}$} \\
& 
$\begin{pmatrix}
0 & 0 & 0 & 0 \\
0 & 0 & 0 & 0 \\
0 & 0 & 0 & i \\
0 & 0 & -i & 0
\end{pmatrix}$ \\
\hline
\multirow{2}{*}{$T^x_0 = \overrightarrow{\tau}$} &
\multirow{2}{*}{$\begin{pmatrix}
1 & 0 & 0 & 0 \\
0 & 1 & 0 & 0 \\
0 & 0 & -1 & 0 \\
0 & 0 & 0 & 0
\end{pmatrix}$} \\
& 
$\begin{pmatrix}
a & -i & 0 & 0 \\
-i & a & 0 & 0 \\
0 & 0 & 0 & 0 \\
0 & 0 & -i & 0
\end{pmatrix}$ \\
\hline
\multirow{2}{*}{$T^x_S = \overrightarrow{\tau}$} &
\multirow{2}{*}{$\begin{pmatrix}
1 & 0 & 0 & 0 \\
0 & 1 & 0 & 0 \\
0 & 0 & 0 & 0 \\
0 & 0 & 0 & -1
\end{pmatrix}$} \\
& 
$\begin{pmatrix}
\not\!\!{\tau} & 0 \\
0 & -\not\!\!{\tau}
\end{pmatrix}$ \\
\hline
\multirow{2}{*}{$T^x_S = \frac{1}{\sqrt{2}} \begin{pmatrix}
0 & 0 & 0 & 1 \\
0 & 0 & -1 & 0 \\
0 & -1 & 0 & 0 \\
0 & 0 & 0 & 0
\end{pmatrix}$} &
\multirow{2}{*}{$\begin{pmatrix}
0 & 0 & 0 & i \\
0 & 0 & 0 & 0 \\
0 & 0 & 0 & 0 \\
i & 0 & 0 & 0
\end{pmatrix}$} \\
& 
$\begin{pmatrix}
0 & 0 & 0 & 0 \\
0 & 0 & -i & 0 \\
0 & i & 0 & 0 \\
0 & 0 & 0 & 0
\end{pmatrix}$ \\
\hline
\end{tabular}
\qquad
\begin{tabular}{c|c}
\hline
\multicolumn{2}{c}{$U(2N_F)$} \\
\hline
\multirow{2}{*}{$T^x_0 = \overrightarrow{\tau}$} & 
\multirow{2}{*}{$\begin{pmatrix}
\not\!\!{\tau} & 0 \\
0 & \not\!\!{\tau}
\end{pmatrix}$} \\
& 
$\begin{pmatrix}
H_2 & 0 \\
0 & -H_2^\dagger
\end{pmatrix}$ \\
\hline
\multirow{2}{*}{$T^x_S = \overrightarrow{\tau}$} &
\multirow{2}{*}{$\begin{pmatrix}
\not\!\!{\tau} & 0 \\
0 & -\not\!\!{\tau}
\end{pmatrix}$} \\
& 
$\begin{pmatrix}
H_1 & 0 \\
0 & H_1^\dagger
\end{pmatrix}$ \\
\hline
\multirow{2}{*}{$T^x_F = \frac{1}{\sqrt{2}} \begin{pmatrix}
0 & 0 & 1 & 0 \\
0 & 0 & 0 & 0 \\
1 & 0 & 0 & 0 \\
0 & 0 & 0 & 0
\end{pmatrix}$} &
\multirow{2}{*}{$\begin{pmatrix}
0 & S \\
S^\dagger & 0
\end{pmatrix}$} \\
& 
$\begin{pmatrix}
0 & S \\
S^\dagger & 0
\end{pmatrix}$ \\
\hline
\multirow{2}{*}{$T^x_F = \frac{1}{2} \begin{pmatrix}
0 & 0 & 0 & i \\
0 & 0 & 0 & 0 \\
0 & 0 & 0 & 0 \\
0 & 0 & 0 & 0
\end{pmatrix}$} &
\multirow{2}{*}{$\begin{pmatrix}
0 & 0 & 0 & i \\
0 & 0 & 0 & 0 \\
0 & 0 & 0 & 0 \\
i & 0 & 0 & 0
\end{pmatrix}$} \\
& 
$\begin{pmatrix}
0 & 0 & 0 & 0 \\
0 & 0 & 0 & 0 \\
0 & 0 & 0 & i \\
0 & 0 & -i & 0
\end{pmatrix}$ \\
\hline
\multirow{2}{*}{$T^x_S = \overrightarrow{\tau}$} &
\multirow{2}{*}{$\begin{pmatrix}
0 & 0 & 0 & 1 \\
0 & 0 & -1 & 0 \\
0 & -1 & 0 & 0 \\
0 & 0 & 0 & 0
\end{pmatrix}$} \\
& 
$\begin{pmatrix}
0 & A \\
A^\dagger & 0
\end{pmatrix}$ \\
\hline
\end{tabular}
\caption{Generators of $U(2N_F)$. On the left we have explicit, properly normalised generators for $SU(4)$. To the right their general structure is given for arbitrary values of $N_F$. The matrices $H_{1,2}$ denote hermitian matrices. The matrix $S = S_R + i S_I$ denotes a complex, traceful, symmetric matrix. The matrix $A = A_R + i A_I$ denotes a complex, anti-symmetric matrix. All matrices are defined with respect to the Nambu-Gorkov basis \eqref{eq:dark_strong_quark_lagrangian_nambu_gorkov}.}
\end{table}

\end{document}