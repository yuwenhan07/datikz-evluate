\documentclass{article}
\usepackage{amsmath}
\usepackage{tikz}
\usetikzlibrary{arrows.meta}

\begin{document}

\begin{figure}[h]
    \centering
    \begin{tikzpicture}[scale=0.8]
        % Draw axes
        \draw[->, thick] (0,0) -- (16,0) node[right] {$\mu_1$};
        \draw[->, thick] (0,0) -- (0,16) node[above] {$\mu_2$};
        
        % Draw horizontal line at y=15
        \draw[cyan, thick] (0,15) -- (16,15);
        
        % Draw diagonal lines
        \foreach \i in {0,...,14} {
            \draw[cyan, thick] (\i, 15-\i) -- (15-\i, \i);
        }
        
        % Mark points on the grid
        \foreach \x in {0,...,15} {
            \foreach \y in {0,...,15} {
                \filldraw[black] (\x,\y) circle (1pt);
            }
        }
        
        % Label the top right point
        \node at (16,15) [right] {$\widehat{\mu}_1$};
        
        % Label the left side
        \node at (0,15) [left] {$\mu_2$};
        
        % Label the bottom right corner
        \node at (16,0) [below] {$\mu_1$};
    \end{tikzpicture}
    \caption{The recurrence coefficients of $(\mu_1,\mu_2)$ can be computed via the CC algorithm (Theorem~\ref{thm:NNCC}). On the upper boundary we find the recurrence coefficients of $\widehat{\mu}_1$.}
    \label{fig:recurrence_coefficients}
\end{figure}

\end{document}