\documentclass{article}
\usepackage{tikz}
\usetikzlibrary{arrows.meta}

\begin{document}

\begin{figure}[h]
    \centering
    \begin{tikzpicture}[scale=1.5]
        % Draw the horizontal lines representing the CFTs
        \draw[thick] (-4,0) -- (4,0);
        \node at (-4,-0.2) {CFT${}_{\text{I}}$};
        \node at (4,-0.2) {CFT${}_{\text{II}}$};
        
        % Draw the vertical lines representing the interfaces
        \draw[red, thick] (-3,0) -- (-3,1) node[midway, right] {$x_{\text{I}}$};
        \draw[red, thick] (3,0) -- (3,1) node[midway, right] {$x_{\text{II}}$};
        
        % Draw the arrows indicating the interfaces
        \draw[->, red, ultra thick] (-3,1) -- (-3,1.5) node[midway, above] {$u_{\text{I}}$};
        \draw[->, red, ultra thick] (3,1) -- (3,1.5) node[midway, above] {$u_{\text{II}}$};
        
        % Draw the blue curve representing the interface brane Q
        \draw[blue, thick, dotted] (-2,0) .. controls (0,2) and (2,2) .. (4,0) node[midway, above] {$Q$};
        \node at (-2.5,2.5) {$(N_{\text{I}}, g^{\text{I}}_{ab})$};
        \node at (3.5,2.5) {$(N_{\text{II}}, g^{\text{II}}_{ab})$};
    \end{tikzpicture}
    \caption{Cartoon plot for the setup. The left and right bulk, denoted as $N_{\text{I}}$ and $N_{\text{II}}$ respectively, are dual to CFT${}_{\text{I}}$ and CFT${}_{\text{II}}$ respectively. The interface brane $Q$ is in blue.}
    \label{fig:cartoon_plot}
\end{figure}

\end{document}