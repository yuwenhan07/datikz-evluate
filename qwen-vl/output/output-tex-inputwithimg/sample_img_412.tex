\documentclass{article}
\usepackage{amsmath}
\usepackage{forest}
\usetikzlibrary{positioning}

\begin{document}

\section*{(a) Phylogenetic species tree \( T \), used to generate gene genealogies for \( n_{\underline{1}} \) and \( n_{\underline{2}} \) individuals sampled from species 1 and 2 respectively.}

\begin{figure}[h]
    \centering
    \begin{forest}
        for tree={
            grow'=0,
            l sep+=5mm,
            s sep+=5mm,
            parent anchor=south,
            child anchor=north,
            align=center,
            font=\small,
            draw,
            fill=white,
            text width=1cm,
            minimum height=1cm,
            inner sep=0pt,
            outer sep=0pt,
            anchor=center,
            edge={thick,->},
            edge path={
                \noexpand\path [\forestoption{edge}]
                    (!u.parent anchor) -- +(0,-5mm) -| (.child anchor)\forestoption{edge label};
            },
        }
        [n_{\underline{1}}, name=A
            [n_{\overline{1}}, name=B
                [n_{\underline{1}}]
                [r_{\underline{1}}]
            ]
            [n_{\overline{2}}, name=C
                [n_{\underline{2}}]
                [r_{\underline{2}}]
            ]
        ]
        [n_{\underline{2}}, name=D
            [n_{\overline{1}}, name=E
                [n_{\underline{1}}]
                [r_{\underline{1}}]
            ]
            [n_{\overline{2}}, name=F
                [n_{\underline{2}}]
                [r_{\underline{2}}]
            ]
        ]
    \end{forest}
\end{figure}

\section*{(b) Graph \( G \) for the graphical model associated with the evolution of a binary trait on a gene tree drawn from the multispecies coalescent model. \( G \) is a DAG with 2 sources (roots) \( n_{\underline{1}} \) and \( n_{\underline{2}} \), and 2 sinks (leaves) \( r_{\underline{1}} \) and \( r_{\underline{2}} \). The model restricted to variables \( n_{\overline{e}} \) and \( n_{\underline{e}} \) (in black) can be described by the subgraph \( G_n \) whose nodes and edges are in black. It is a tree similar to \( T \) but with reversed edge directions.}

\begin{figure}[h]
    \centering
    \begin{forest}
        for tree={
            grow'=0,
            l sep+=5mm,
            s sep+=5mm,
            parent anchor=south,
            child anchor=north,
            align=center,
            font=\small,
            draw,
            fill=white,
            text width=1cm,
            minimum height=1cm,
            inner sep=0pt,
            outer sep=0pt,
            anchor=center,
            edge={thick,->},
            edge path={
                \noexpand\path [\forestoption{edge}]
                    (!u.parent anchor) -- +(0,-5mm) -| (.child anchor)\forestoption{edge label};
            },
        }
        [n_{\underline{1}}, name=A
            [n_{\overline{1}}, name=B
                [n_{\underline{1}}]
                [r_{\underline{1}}]
            ]
            [n_{\overline{2}}, name=C
                [n_{\underline{2}}]
                [r_{\underline{2}}]
            ]
        ]
        [n_{\underline{2}}, name=D
            [n_{\overline{1}}, name=E
                [n_{\underline{1}}]
                [r_{\underline{1}}]
            ]
            [n_{\overline{2}}, name=F
                [n_{\underline{2}}]
                [r_{\underline{2}}]
            ]
        ]
    \end{forest}
\end{figure}

\section*{(c) Clique tree \( \mathcal{U} \) for \( G \). Note that the 6-variable clique is overparametrized because \( n_{\overline{1}} + n_{\overline{2}} = n_{\underline{3}} \) and \( r_{\overline{1}} + r_{\overline{2}} = r_{\underline{3}} \), but reflects the symmetry of the model.}

\begin{figure}[h]
    \centering
    \begin{forest}
        for tree={
            grow'=0,
            l sep+=5mm,
            s sep+=5mm,
            parent anchor=south,
            child anchor=north,
            align=center,
            font=\small,
            draw,
            fill=white,
            text width=1cm,
            minimum height=1cm,
            inner sep=0pt,
            outer sep=0pt,
            anchor=center,
            edge={thick,->},
            edge path={
                \noexpand\path [\forestoption{edge}]
                    (!u.parent anchor) -- +(0,-5mm) -| (.child anchor)\forestoption{edge label};
            },
        }
        [n_{\underline{1}}, name=A
            [n_{\overline{1}}, name=B
                [n_{\underline{1}}]
                [r_{\underline{1}}]
            ]
            [n_{\overline{2}}, name=C
                [n_{\underline{2}}]
                [r_{\underline{2}}]
            ]
        ]
        [n_{\underline{2}}, name=D
            [n_{\overline{1}}, name=E
                [n_{\underline{1}}]
                [r_{\underline{1}}]
            ]
            [n_{\overline{2}}, name=F
                [n_{\underline{2}}]
                [r_{\underline{2}}]
            ]
        ]
    \end{forest}
\end{figure}

\end{document}