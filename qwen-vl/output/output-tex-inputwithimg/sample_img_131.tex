\documentclass{article}
\usepackage{tikz}
\usetikzlibrary{matrix, arrows.meta}

\begin{document}

\begin{figure}[h]
    \centering
    \begin{tikzpicture}[node distance=2cm, auto]
        % Define nodes
        \node[draw, circle] (u) {$u$};
        \node[draw, circle, right of=u] (v) {$v$};
        \node[draw, circle, right of=v] (w) {$w$};

        % Draw edges with labels
        \path[->, thick]
            (u) edge[bend left=45, above] node[above] {20@$[1,3]$} (v)
            (v) edge[bend left=45, below] node[below] {15@$[4,5]$} (u)
            (v) edge[bend left=45, above] node[above] {25@$[4,6]$} (w)
            (w) edge[bend left=45, below] node[below] {25@$[4,6]$} (v);
        
        % Draw asset labels
        \node[draw, rectangle, inner sep=2pt, minimum width=8mm, below=of u] (asset_u) {\euro 30};
        \node[draw, rectangle, inner sep=2pt, minimum width=8mm, below=of v] (asset_v) {\euro 20};
        \node[draw, rectangle, inner sep=2pt, minimum width=8mm, below=of w] (asset_w) {\euro 10};
        
        % Position asset labels relative to nodes
        \coordinate[below=of asset_u] (aux_u);
        \coordinate[below=of asset_v] (aux_v);
        \coordinate[below=of asset_w] (aux_w);
        
        \draw (u) -- (aux_u) -- (asset_u);
        \draw (v) -- (aux_v) -- (asset_v);
        \draw (w) -- (aux_w) -- (asset_w);
    \end{tikzpicture}
    
    \caption{A simple instance of the Interval Debt Model (IDM). Numbers in square boxes represent the assets of the node (for example, \euro 30 for node $u$), directed edges represent debts, and the label on an edge represents the terms of the associated debt (for example, $u$ must pay $v$ \euro 20 between time 1 and time 3).}
    \label{fig:idm_example}
\end{figure}

\end{document}