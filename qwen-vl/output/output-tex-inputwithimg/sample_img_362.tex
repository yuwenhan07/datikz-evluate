\documentclass{article}
\usepackage{amsmath}
\usepackage{graphicx}

\begin{document}

\begin{figure}[h]
    \centering
    \includegraphics[width=0.8\textwidth]{path_to_your_image.png}
    \caption{
        \textit{Left:} the dressed propagator $\mathcal{D}^{p'p}_{N}$. The double line represents the non-perturbative treatment of the coupling between charged particles and the background.
        \textit{Right:} LSZ reduction on the matter lines in $\mathcal{D}^{p'p}_{N}$ yields the tree-level off-shell current $\mathcal{A}^{p'p}_{N}$, wherein the photon momenta $k_i$ are not necessarily on-shell. One can compute the corresponding tree-level amplitude from $\mathcal{A}^{p'p}_{N}$ by simply setting the photons to be on-shell and transverse, i.e., $k_i^2 = 0 = \varepsilon_{i} \cdot k_{i}$.
    }
    \label{fig:dressed_propagator}
\end{figure}

\end{document}