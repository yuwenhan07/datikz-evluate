\documentclass{article}
\usepackage{tikz}
\usetikzlibrary{automata,positioning}

\begin{document}

\begin{figure}[h]
    \centering
    \begin{tikzpicture}[node distance=2cm, auto]
        % State 1-cycles
        \node[state] (q_1) {$Q_1$};
        \path[->] 
            (q_1) edge [loop above] ();
        
        % State 2-cycles
        \node[state, right=of q_1] (q_2) {$Q_2$};
        \path[->] 
            (q_1) edge[bend left] node {} (q_2)
            (q_2) edge[bend left] node {} (q_1);
        
        % Divider
        \draw (current bounding box.north east) -- (current bounding box.south east);
        
        % State n-cycles
        \node[state, right=of q_2] (q_n) {$Q_n$};
        \node[state, below=of q_n] (q_1') {$Q_1$};
        \node[state, below=of q_1'] (q_2') {$Q_2$};
        
        \path[->] 
            (q_n) edge[bend left] node {} (q_1')
            (q_1') edge[bend left] node {} (q_2')
            (q_2') edge[bend left] node {} (q_n);
        
        \node[right=of q_n, xshift=1cm] {$n-2$ times};
    \end{tikzpicture}
    
    \caption{$Q_i$ are elements of the state space. On the left, 1-cycles (which is an equilibrium) and a 2-cycle are frequently observed and behave deterministically. On the right, an $n$-cycle with $n > 2$ is never observed deterministically and stochastically will exist with probability 0.}
\end{figure}

\end{document}