\documentclass{article}
\usepackage{tikz}

\begin{document}

\begin{figure}[h]
    \centering
    \begin{tikzpicture}[scale=0.5]
        % Draw the grid
        \draw[help lines] (0,0) grid (4,4);
        
        % Draw the diagonal pattern
        \foreach \x in {0,...,3} {
            \foreach \y in {0,...,3} {
                \ifnum\x=\y
                    \draw (\x,\y) -- ++(1,1) -- ++(-1,-1);
                \else
                    \draw (\x,\y) -- ++(1,-1) -- ++(-1,1);
                \fi
            }
        }
        
        % Draw the second row of patterns
        \foreach \x in {0,...,4} {
            \node at (\x*1.5, -2) {\includegraphics[width=1cm]{example-image-x}};
        }
    \end{tikzpicture}
    
    \caption{The diagonal transformation that is part of Pinsky's combinatorial theorem. The choice of the next step being among \(\{(0,1),(1,0),(-1,0),(0,-1)\}\) is equivalent to a diagonal version of the encoding being in the set \(\{(+1,+1),(+1,-1),(-1,-1),(-1,+1)\}\) as shown.}
    \label{fig:diagonal_transformation}
\end{figure}

\end{document}