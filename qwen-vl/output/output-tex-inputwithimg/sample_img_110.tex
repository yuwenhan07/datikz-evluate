\documentclass{article}
\usepackage{amsmath}
\usepackage{tikz}
\usetikzlibrary{arrows.meta}

\begin{document}

\begin{figure}[h]
    \centering
    \begin{tikzpicture}[scale=1.5]
        % Draw the horizontal lines
        \draw[blue, thick] (-2, 0) -- (4, 0);
        \draw[blue, thick] (-2, 1) -- (4, 1);
        \draw[blue, thick] (-2, -1) -- (4, -1);
        
        % Draw the vertical line segments
        \draw[dotted, thick] (-2, 0) -- (-1.5, 0);
        \draw[dotted, thick] (-2, 1) -- (-1.5, 1);
        \draw[dotted, thick] (-2, -1) -- (-1.5, -1);
        
        % Draw the curved path
        \draw[blue, thick, dashed] (-1.5, 0) .. controls (-1, 0.5) and (-0.5, 0.5) .. (-0.5, 1);
        \draw[blue, thick, dashed] (-1.5, 1) .. controls (-1, 0.5) and (-0.5, 0.5) .. (-0.5, 0);
        \draw[blue, thick, dashed] (-1.5, -1) .. controls (-1, -0.5) and (-0.5, -0.5) .. (-0.5, 0);
        
        % Add the labels
        \node at (4.5, 0) {$-$};
        \node at (4.5, 1) {$-$};
        \node at (4.5, -1) {$-$};
    \end{tikzpicture}
    \caption{Sticky snapping out Brownian motion is a Feller process on $\ka$ copies of $[0,\infty]$ (here $\ka =3$), which on the $i$th copy behaves like a one-dimensional sticky Brownian motion with stickiness coefficient $a_i/b_i$. After spending enough time at $(0,i)$ the process jumps to one of the points $(0,j), j\not =i$ to continue its motion on the corresponding copy of $[0,\infty]$, and so on. Times between jumps are governed by parameters $c_i$.}
    \label{fig:sticky_brownian_motion}
\end{figure}

\end{document}