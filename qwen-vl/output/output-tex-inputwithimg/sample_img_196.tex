\documentclass{article}
\usepackage{tikz}
\usetikzlibrary{shapes.geometric}

\begin{document}

\begin{figure}[h]
    \centering
    \begin{tikzpicture}[scale=1.5]
        % Draw the O5+ branes
        \draw[thick] (-2,0) -- (-1,0);
        \draw[thick] (2,0) -- (1,0);
        \draw[thick] (-1,0) -- (-1,-1);
        \draw[thick] (1,0) -- (1,-1);
        \draw[thick] (-1,-1) -- (-2,-2);
        \draw[thick] (1,-1) -- (2,-2);
        \draw[thick] (-2,-2) -- (-1,-3);
        \draw[thick] (2,-2) -- (1,-3);
        \draw[thick] (-1,-3) -- (-2,-4);
        \draw[thick] (1,-3) -- (2,-4);
        
        % Draw the O5- branes
        \draw[dashed, thick] (-2,-2) -- (-1,-3);
        \draw[dashed, thick] (2,-2) -- (1,-3);
        \draw[dashed, thick] (-1,-3) -- (-2,-4);
        \draw[dashed, thick] (1,-3) -- (2,-4);
        
        % Label the O5 branes
        \node at (-2.5,0) {O5$^+$};
        \node at (2.5,0) {O5$^+$};
        \node at (0,-2) {O5$^-$};
        
        % Draw the toric nodes
        \foreach \x in {-2,-1,0,1,2} {
            \filldraw (\x,-2) circle (1pt);
        }
        \foreach \y in {-1,-3} {
            \filldraw (0,\y) circle (1pt);
        }
        \foreach \x in {-2,-1,0,1,2} {
            \filldraw (\x,-3) circle (1pt);
        }
        \foreach \y in {-2,-4} {
            \filldraw (0,\y) circle (1pt);
        }
        \foreach \x in {-2,-1,0,1,2} {
            \filldraw (\x,-4) circle (1pt);
        }
    \end{tikzpicture}
    \caption{The brane construction of $SO(8)$ gauge group with toric nodes indicated by the brane construction. Notice that O$5^+$ branes are replaced by 3 nodes instead of 1. See \cite{Hayashi:2023boy} for details.}
    \label{fig:so8_brane}
\end{figure}

\end{document}