\documentclass{article}
\usepackage{amsmath}
\usepackage{tikz}
\usetikzlibrary{arrows.meta}

\begin{document}

\begin{figure}[h]
    \centering
    \begin{tikzpicture}[scale=1.5]
        % Draw the horizontal line
        \draw (-4,0) -- (4,0);
        
        % Mark points
        \filldraw [black] (0,0) circle (1pt) node[anchor=north] {$(0,0)$};
        \filldraw [black] (1,0) circle (1pt) node[anchor=north] {$(r,r)$};
        \filldraw [black] (3,2) circle (1pt) node[anchor=south] {$(n,n)$};
        
        % Draw the thick segments
        \draw[line width=1mm, color=black!50] (1,0) -- (2,0);
        \draw[line width=1mm, color=black!50] (2,0) -- (3,0);
        
        % Label the thick segments
        \node at (1.5,-0.5) {$\xi[\lambda]\text{-directed}$};
        \node at (2.5,1.5) {$-\xi[\eta]\text{-directed}$};
        
        % Arrows
        \draw[-{Stealth[scale=1.5]}] (1.5,0) -- (2.5,0.5);
        \draw[-{Stealth[scale=1.5]}] (1.5,0) -- (2.5,-0.5);
        
        % Grid lines
        \draw[dashed] (0,0) -- (0,2);
        \draw[dashed] (1,0) -- (1,2);
        \draw[dashed] (3,0) -- (3,2);
    \end{tikzpicture}
    \caption{An illustration for choosing the parameters described in Section \ref{fix_para}. The two thick segments represent the intervals $[r+\lfloor a_1 t r^{2/3} \rfloor, r+\lfloor a_2 t r^{2/3} \rfloor]$ and $[r+\lfloor b_1 t (n-r)^{2/3} \rfloor, r+\lfloor b_2 t (n-r)^{2/3} \rfloor]$.}
    \label{fig:parameters}
\end{figure}

\end{document}