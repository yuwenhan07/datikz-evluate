\documentclass{article}
\usepackage{amsmath}
\usepackage{tikz-feynman}

\begin{document}

\begin{center}
\begin{tikzpicture}[baseline={([yshift=-.5ex]current bounding box.center)}]
    \begin{feynman}
        \vertex (a);
        \vertex [right=of a] (b);
        \vertex [above right=of b] (c);
        \vertex [below right=of b] (d);
        \vertex [below left=of d] (e);
        \vertex [left=of e] (f);
        \vertex [below left=of f] (g);
        \vertex [below right=of c] (h);
        \vertex [above right=of h] (i);

        \diagram* {
            (a) -- [fermion, edge label'=\(\gamma,\text{Z},\text{W}^\pm\)] (b),
            (b) -- [boson, edge label'=\(\Psi\)] (c),
            (b) -- [boson, edge label'=\(\Psi\)] (d),
            (d) -- [fermion, edge label'=\(\gamma,\text{Z},\text{W}^\pm\)] (e),
            (e) -- [fermion, edge label'=\(\chi\)] (f),
            (f) -- [boson, edge label'=\(\Psi\)] (g),
            (g) -- [fermion, edge label'=\(\Psi\)] (h),
            (h) -- [boson, edge label'=\(\Psi\)] (i),
            (i) -- [fermion, edge label'=\(\chi\)] (c),
            (c) -- [boson, edge label'=\(S\)] (h),
        };
    \end{feynman}
\end{tikzpicture}
\end{center}

\end{document}