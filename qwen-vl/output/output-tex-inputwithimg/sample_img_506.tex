\documentclass{article}
\usepackage{tikz}
\usetikzlibrary{matrix}

\begin{document}

\begin{figure}[h]
    \centering
    \begin{tikzpicture}[node distance=1cm, auto]
        \matrix (m) [matrix of nodes, column sep=3em, row sep=3em] {
            \node[draw] (teacher) {Teacher \\ pattern $\xi^*$}; & & \node[draw] (examples) {Examples $\sigma$}; & & \node[draw] (student) {Student \\ pattern $\xi$}; \\
            & & & & \\
            \node[draw] (data_teacher) {Data $x$ \\ with known \\ labels $y$}; & & \node[draw] (weights) {Weights $w$}; & & \node[draw] (data_student) {Data $x$ \\ with predicted \\ labels $y'$}; \\
        };
        
        \path[->]
            (teacher) edge node {$H(\sigma|\xi^*)$} (examples)
            (examples) edge node {$H(\xi^b|\sigma)$} (student)
            (teacher) edge[dashed] (data_teacher)
            (student) edge[dashed] (data_student)
            (data_teacher) edge[bend left=45] node[below] {Backward \\ propagation} (weights)
            (weights) edge[bend left=45] node[below] {Forward \\ propagation} (data_student);
    \end{tikzpicture}
    \caption{The first row of this diagram sketches how a $p$-body Hopfield network in the teacher-student setting can reconstruct an incomplete pattern $\xi^b$ to match the teacher pattern $\xi^*$ by relying on the examples $\sigma$ obtained from $\xi^*$. The second row summarizes how a dense neural network trained by Krotov can recover the labels $y'$ of the data $x$ given the weights $w$ learned from $x$ \cite{krotov2016dense}. Both models tackle similar tasks using an approach where $\sigma$ and $\xi^b$ respectively play the same roles as $w$ and $(x, y')$.}
    \label{fig:teacher_student}
\end{figure}

\end{document}