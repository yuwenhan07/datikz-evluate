\documentclass{article}
\usepackage{amsmath}
\usepackage{bm}
\usepackage{xcolor}
\usepackage{tikz}
\usetikzlibrary{arrows.meta}

\begin{document}

\begin{figure}[h]
    \centering
    \begin{tikzpicture}[scale=1.5]
        % Draw the axes
        \draw[->] (-3,0) -- (3,0);
        \draw[->] (0,-1) -- (0,1);

        % Draw the blue curve
        \draw[thick, blue] plot [smooth,tension=1] coordinates {(-2.5,0.5) (-1.5,1) (0,0) (1.5,1) (2.5,0.5)};
        
        % Draw the dashed brown curve
        \draw[dashed, brown] plot [smooth,tension=1] coordinates {(-3,0.5) (-2,1) (0,0) (2,1) (3,0.5)};
        
        % Draw the red arrows
        \draw[red, ultra thick, -Stealth] (0,0.5) -- (0,1.5);
        \draw[red, ultra thick, -Stealth] (0,-0.5) -- (0,-1.5);
        
        % Label the regions
        \node at (0,1.5) {$\Omega^+$};
        \node at (0,-1.5) {$\Omega^+$};
        \node at (-1.5,0.5) {$\Omega^0$};
        \node at (1.5,0.5) {$\Omega^0$};
        
        % Draw the vertical line
        \draw[thick, black] (0,0) -- (0,2);
        
        % Repeat the same drawing on the right side
        \begin{scope}[xshift=4cm]
            % Draw the blue curve
            \draw[thick, blue] plot [smooth,tension=1] coordinates {(0.5,0.5) (1.5,1) (3,0) (4.5,1) (5.5,0.5)};
            
            % Draw the dashed brown curve
            \draw[dashed, brown] plot [smooth,tension=1] coordinates {(0,0.5) (1,1) (3,0) (4,1) (5,0.5)};
            
            % Draw the red arrows
            \draw[red, ultra thick, -Stealth] (3,0.5) -- (3,1.5);
            \draw[red, ultra thick, -Stealth] (3,-0.5) -- (3,-1.5);
            
            % Label the regions
            \node at (3,1.5) {$\Omega^+$};
            \node at (3,-1.5) {$\Omega^+$};
            \node at (1.5,0.5) {$\Omega^0$};
            \node at (4.5,0.5) {$\Omega^0$};
        \end{scope}
        
        % Draw the blue dashed line
        \draw[dashed, blue] (0,0) -- (4,0);
        
        % Caption
        \captionsetup{justification=centering}
        \caption{
            Examples of singular free boundary points where the function $u-\psi$ has strict quadratic growth. We expect quadratic growth in the direction of the \textcolor{red}{red} arrows. The case on the right is more degenerate than the case on the left, and worse for the approximation of the free boundary $\Gamma$ (depicted in \textcolor{blue}{blue}). In both cases we have that $\dim \ker\bm{\mathbf{A}} = 1$. A discrete free boundary $\Gamma_\Triang$, which is at a distance $\mathcal{O}(\delta(h)^{1/2})$ is depicted in \textcolor{brown}{dashed brown}.
        }
    \end{tikzpicture}
\end{figure}

\end{document}