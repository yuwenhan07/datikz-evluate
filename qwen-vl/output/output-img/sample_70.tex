\documentclass{article}
\usepackage{tikz}
\usetikzlibrary{angles,quotes}

\begin{document}

\begin{figure}[h]
    \centering
    \begin{tikzpicture}[scale=2]
        % Draw the base of the triangle
        \draw (0,0) -- (1,0);
        
        % Draw the height from the top vertex to the base
        \draw (0.5,0) -- (0.5,1);
        
        % Draw the slanted line from the top vertex to the midpoint of the base
        \draw (0.5,1) -- (0.75,0.5);
        
        % Label the sides
        \node at (0.5,-0.1) {$1$};
        \node at (0.75,0.5) {$\frac{s}{2}$};
        \node at (0.5,1) {$\frac{s}{2}$};
        
        % Draw the angles
        \pic [draw, angle radius=1cm, "$\alpha$", angle eccentricity=1.3] {angle = 0--0.5--1};
        \pic [draw, angle radius=1cm, "$\beta$", angle eccentricity=1.3] {angle = 0--0.5--0.75};
        
        % Add the equations
        \node at (2.5,0.8) {$\beta = \sqrt{1 - \left(\frac{s}{2}\right)^2}$};
        \node at (2.5,0.4) {$\alpha = 1 - \beta$};
        \node at (2.5,0.0) {$s' = \sqrt{\alpha^2 + \left(\frac{s}{2}\right)^2}$};
    \end{tikzpicture}
    \caption{Illustration of the geometric relationships in the given problem.}
    \label{fig:sample_161}
\end{figure}

\end{document}