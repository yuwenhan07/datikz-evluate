It seems you're describing a construction process for a graph \( T \) derived from a 3-uniform hypergraph \( H \), where the edges are represented with specific directions and connections. Let's break down the details:

### Definitions:
- **3-uniform hypergraph \( H \)**: A hypergraph where each edge (hyperedge) contains exactly 3 vertices.
- **Vertices**: \( v_a, v_b, \ldots \)
- **Edges**: Represented by arcs in the constructed graph \( T \).

### Construction Process:
1. **Vertices in \( T \)**:
   - The vertices in \( T \) correspond to the vertices in \( H \). So, if \( H \) has vertices \( v_1, v_2, \ldots, v_n \), then \( T \) will have vertices \( v_1, v_2, \ldots, v_n \).

2. **Arcs in \( T \)**:
   - **Red Arcs (Going Down)**:
     - These arcs represent the edges in \( H \) that include the vertex \( v_1 \).
     - For example, if \( H \) has an edge \( \{v_1, v_2, v_3\} \), then there would be an arc from \( v_1 \) to \( v_2 \) and another arc from \( v_1 \) to \( v_3 \) in \( T \).
   - **Arcs Going Up**:
     - These arcs connect vertices \( v_a, i \) to vertices \( v_b, j \) for \( b \neq a \).
     - Specifically, for any vertex \( v_a, i \), there is an arc from \( v_a, i \) to every vertex \( v_b, j \) where \( b \neq a \).

### Example:
Let's consider a simple 3-uniform hypergraph \( H \) with the following edges:
\[ E(H) = \{\{v_1, v_2, v_3\}, \{v_1, v_4, v_5\}\} \]

The corresponding graph \( T \) would have the following arcs:
- From \( v_1 \) to \( v_2 \) and \( v_3 \) (red arcs going down).
- From \( v_1 \) to \( v_4 \) and \( v_5 \) (red arcs going down).
- From \( v_2 \) to \( v_1, v_4, v_5 \) (arcs going up).
- From \( v_3 \) to \( v_1, v_4, v_5 \) (arcs going up).
- From \( v_4 \) to \( v_1, v_2, v_3 \) (arcs going up).
- From \( v_5 \) to \( v_1, v_2, v_3 \) (arcs going up).

### Summary:
- Red arcs (going down) represent the edges in \( H \) involving a specific vertex (e.g., \( v_1 \)).
- Blue arcs (going up) connect all other vertices to the specific vertex (e.g., \( v_1 \)).

This construction ensures that the structure of the original 3-uniform hypergraph \( H \) is preserved in the directed graph \( T \), with the additional arcs connecting all vertices except those involved in the original edges.