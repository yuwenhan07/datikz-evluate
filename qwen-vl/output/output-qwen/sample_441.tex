The description you've provided seems to be referring to a discrete-time representation of a continuous function, often used in simulations or numerical analysis. Here's how we can interpret this:

1. **Discrete-Time Representation**: The graph you're describing is likely a discrete-time representation of a continuous function \( V(t) \), where \( t \) represents time and \( V(t) \) represents the value of the function at that time.

2. **Sample Path**: A sample path of a stochastic process (like the waiting time \( V(t) \)) refers to a specific realization of the process over time. In other words, it's one possible trajectory of the function \( V(t) \) as time progresses.

3. **Graph Interpretation**: The dots on the graph represent specific values of \( V(t) \) at certain discrete points in time. The lines connecting these dots are used to visually interpolate between these points, giving an idea of how the function might behave between the sampled times.

4. **Behavior Over Time**: By examining the pattern of the dots and lines, one can infer the general trend of the function \( V(t) \). For example:
   - If the dots are generally increasing, it suggests that the waiting time is growing over time.
   - If they are decreasing, it suggests the opposite.
   - If they are fluctuating, it indicates variability in the waiting time.

5. **Example**: Suppose \( V(t) \) represents the waiting time for a customer in a queue. The sample path could show how long a customer has waited at different points in time. If the dots are consistently increasing, it might indicate that the queue is getting longer over time.

In summary, the graph you described is a visual representation of a sample path of the waiting time \( V(t) \), showing how the waiting time evolves over time through a series of discrete observations. This type of visualization is useful for understanding trends and patterns in the data.