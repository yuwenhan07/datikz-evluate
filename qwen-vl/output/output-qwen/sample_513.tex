It seems you're referring to a "pcf-array," which is not a standard term in mathematics or science. However, based on the description of the image, it sounds like you might be describing a type of graph or diagram that could be related to concepts from various fields such as computer science, physics, or mathematics.

If this is indeed a scientific or mathematical diagram, here are some possibilities:

1. **Probability Density Function (PDF) or Cumulative Distribution Function (CDF):** These are often represented using graphs where the x-axis represents values and the y-axis represents probability density or cumulative probability. Overlapping gray boxes could represent different distributions or regions of interest within the data.

2. **Heat Map or Contour Plot:** These are used to visualize data where the intensity of color or shading corresponds to the value of a variable. Overlapping boxes could represent different levels of the variable being measured.

3. **Signal Processing or Image Analysis:** In these contexts, grayscale images can represent different intensities or levels of a signal or feature. Overlapping boxes might indicate different segments or regions of interest.

4. **Statistical Analysis:** If the diagram is part of a statistical analysis, the boxes could represent confidence intervals, error bars, or different categories of data.

5. **Algorithm Visualization:** In algorithm visualization, the boxes might represent different states or steps in an algorithm's execution.

Without more specific details about the context or the exact nature of the data being represented, it's challenging to provide a precise interpretation. If you have additional information or if you can provide more details about what the diagram is supposed to represent, I'd be happy to help further!