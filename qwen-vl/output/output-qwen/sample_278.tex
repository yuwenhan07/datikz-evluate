It seems you're referring to a mathematical or theoretical physics context, possibly involving dynamical systems and transformations. Let's break down your statement into more manageable parts:

1. **Left Part: Dyson-Schmidt Dynamics on (0,)**
   - The Dyson-Schmidt dynamics is a concept often used in quantum mechanics and statistical physics to describe the evolution of a system over time.
   - The notation \((0,\infty)\) typically refers to the positive real numbers, which could be the domain of the system's state variables.
   - Lemma~lemma:slower likely refers to a specific lemma that describes how certain properties evolve slower than others under this dynamics.

2. **Right Part: Logarithmic Transformation \(f\) to \(R\)**
   - The transformation \(f\) maps the original space \((0,\infty)\) to another space \(R\), which could be the real numbers.
   - This transformation might be applied for simplification, analysis, or to make certain properties of the system easier to study.

3. **Arrows Illustrating Properties**
   - The arrows on the left part seem to indicate the flow or evolution of properties under the Dyson-Schmidt dynamics.
   - On the right part, the arrows might show how these properties transform or change under the logarithmic transformation \(f\).

### Example Context

Let's consider an example where the system evolves according to some function \(g(t, x)\) where \(x \in (0,\infty)\). The Dyson-Schmidt dynamics might describe how \(g(t, x)\) changes with time \(t\).

- **Left Part**: If we have a property \(P(x)\) evolving under the dynamics, the arrows might show how \(P(x)\) changes over time, e.g., \(P(x) = P_0 e^{-\lambda t}\) for some constant \(\lambda\).
  
- **Right Part**: After applying the logarithmic transformation \(f(x) = \ln(x)\), the property \(P(x)\) transforms to \(P(f(x)) = P(e^{\ln(x)}) = P(x)\). However, if we consider the transformed dynamics, the arrows might show how the transformed property \(Q(y) = P(e^y)\) evolves under the new dynamics.

### Summary

The arrows on the left part illustrate the evolution of properties under the original Dyson-Schmidt dynamics on the interval \((0,\infty)\). The arrows on the right part illustrate how these properties transform and evolve under the logarithmic transformation \(f\) to the real numbers \(R\).

If you need further clarification or have specific questions about the notations or transformations involved, feel free to ask!