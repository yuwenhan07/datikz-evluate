The description you've provided seems to be about visual patterns created by hexagonal shapes with varying sizes and colors. Here's a breakdown of what you might be referring to:

1. **Hexagonal Patterns**: Hexagons are geometric shapes with six sides. In your case, there are two distinct patterns made up of hexagons:
   - A larger pattern composed of hexagons that are approximately 4x4 units.
   - A smaller pattern consisting of hexagons that are roughly 2x2 units.

2. **Color and Size Differences**:
   - The larger hexagons are colored green.
   - The smaller hexagons are colored yellow.
   - Both sets of hexagons are set against a white background.

3. **Visual Distinction**:
   - The larger green hexagons form a more intricate or complex pattern due to their size.
   - The smaller yellow hexagons create a simpler or more scattered pattern because they are smaller.

### Iterative Substitution:
In mathematics and computer science, iterative substitution is a technique used to generate fractal-like patterns. This process involves repeatedly applying a set of rules to an initial shape (or pattern) to create a more complex structure over time. Here’s how it could apply to your hexagonal patterns:

- **Initial Pattern**: Start with a simple arrangement of hexagons, such as a single large hexagon.
- **Substitution Rules**: Apply rules to replace each hexagon with a smaller version of the pattern (e.g., replacing a large hexagon with four smaller ones arranged in a specific way).
- **Iteration**: Repeat the substitution process multiple times to create a more detailed and complex pattern.

### Example:
If we were to apply this to your patterns:
- **Green Hexagons**: If we start with a large green hexagon, we could replace it with four smaller green hexagons arranged in a specific pattern (e.g., a cross or a star-like formation). Repeating this process would create a more intricate green pattern.
- **Yellow Hexagons**: Similarly, if we start with a large yellow hexagon, we could replace it with four smaller yellow hexagons arranged in a specific pattern. Repeating this process would create a more intricate yellow pattern.

### Conclusion:
Your description suggests a visual distinction between two hexagonal patterns based on their size and color. Iterative substitution could be used to generate these patterns by repeatedly applying rules to replace larger hexagons with smaller versions of the same pattern. This method can lead to complex and visually interesting designs, especially when applied iteratively.