The statement you've made touches on an important aspect of conformal field theory and its applications in quantum field theory, particularly in the context of holography and the AdS/CFT correspondence. Let's break down the key points:

1. **Boundary Global Symmetry**: In a holographic setup, such as the AdS/CFT correspondence, the boundary of the Anti-de Sitter (AdS) space is described by a conformal field theory (CFT). The global symmetries of this CFT are typically represented by transformations that act on the boundary degrees of freedom.

2. **Entanglement Wedges**: In the context of holography, entanglement wedges are regions in the bulk spacetime that correspond to the entanglement entropy between two regions of the boundary CFT. These wedges are defined by the causal structure of the bulk spacetime and are crucial for understanding the Ryu-Takayanagi formula, which relates the entanglement entropy to the area of the minimal surface in the bulk.

3. **Action of Boundary Global Symmetry**: The boundary global symmetry transformations are restricted to act only within these entanglement wedges. This means that if an operator is localized outside of an entanglement wedge, it will not be affected by the global symmetry transformations that are confined to that wedge.

4. **Charged Operator**: When you mention a "supposedly charged operator," you're referring to an operator that is expected to carry some charge under the global symmetry. However, if this operator is localized outside of the entanglement wedge where the global symmetry transformation acts, then the operator will not be affected by those transformations.

To summarize, the boundary global symmetry transformations are indeed restricted to the union of the entanglement wedges, and any operator that is localized outside of these wedges will not be acted upon by these transformations. This restriction is a fundamental aspect of how holographic dualities work and helps to maintain the consistency of the theory across different regions of the boundary CFT.