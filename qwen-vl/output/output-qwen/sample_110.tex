To determine the 1-neighborhoods of the black vertex for each type of grid, we need to understand what a 1-neighborhood means. A 1-neighborhood of a vertex is the set of vertices that are directly adjacent to it.

Let's analyze each case step by step:

### (a) Square Grid

In a square grid, each vertex has exactly 4 neighbors (up, down, left, right). If we assume the black vertex is at the center of the grid, its 1-neighborhood would be the four vertices directly surrounding it.

### (b) King Grid

In a king grid (also known as a chessboard), each vertex has exactly 8 neighbors (up, down, left, right, diagonally up-left, up-right, down-left, down-right). If we assume the black vertex is at the center of the grid, its 1-neighborhood would be the eight vertices directly surrounding it.

### (c) Triangular Grid

In a triangular grid, each vertex has exactly 6 neighbors (3 directly above, below, and to the sides). If we assume the black vertex is at the center of the grid, its 1-neighborhood would be the six vertices directly surrounding it.

Since the problem does not provide specific coordinates or a visual representation of the black vertex in the images, we can only describe the general neighborhoods based on the type of grid. Here are the answers:

(a) For the square grid, the 1-neighborhood of the black vertex consists of 4 vertices.
(b) For the king grid, the 1-neighborhood of the black vertex consists of 8 vertices.
(c) For the triangular grid, the 1-neighborhood of the black vertex consists of 6 vertices.

So the final answer is:
\[
\boxed{(4, 8, 6)}
\]