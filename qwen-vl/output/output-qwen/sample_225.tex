It appears you're discussing concepts from string theory, specifically related to the compactification of extra dimensions and the stabilization of moduli fields. Let's break down the key components of your description:

1. **Strict Fundamental Domain \( F_i \)**:
   - This refers to a specific region in the complex plane that represents the allowed values for a particular modulus \( T_i \). In the context of string theory, this modulus often corresponds to the Kähler modulus, which controls the size and shape of the internal compactified dimensions.
   - The "strict" part indicates that this domain is defined by the constraints imposed by the theory, such as the requirement that the metric on the compact space be positive definite.

2. **Fixed Points**:
   - The dots within the domain represent fixed points under some symmetry operation or transformation. These points are special because they remain unchanged under the action of the symmetry group.

3. **Stabilized Value \( T_{i1.23i} \)**:
   - This is a specific value of the modulus \( T_i \) that is often quoted in the literature on heterotic string theory. It is a stabilized value, meaning it is the value that the modulus tends to take when the theory is compactified on a Calabi-Yau manifold. The subscript "1.23i" suggests a specific choice of parameters or a particular configuration of the compactification.

4. **Reduced Fundamental Domain**:
   - The green shaded region represents the reduced fundamental domain after identifying \( \text{Re}(T_i) = -\text{Re}(T_i) \). This means that the real part of \( T_i \) is identified with its negative counterpart, effectively folding the domain over itself along the real axis. This identification is a common technique used in the study of discrete symmetries and orbifolds.

5. **Green Contour**:
   - The solid green contour is the boundary of the reduced fundamental domain combined with the vertical line where the imaginary part of \( T_i \) is greater than 1. This contour helps visualize the structure of the domain and how it is affected by the identifications and symmetries involved.

In summary, the image and text describe a specific region in the complex plane (the fundamental domain) for a modulus \( T_i \), with particular attention to its symmetries and the stabilized value of the modulus. The reduced fundamental domain and the green contour provide a visual representation of these properties.