To solve this problem, we need to understand what it means for two partitions of an annulus \(X\) to be topologically equivalent but geometrically distinct. This implies that the partitions have the same connectivity and structure but differ in their specific spatial arrangement.

Given:
- \(X\) is an annulus.
- The annulus \(X\) is partitioned into regions labeled \(A\), \(B\), \(C\), and \(D\).
- There are two different partitions: \(X = ABC\) and \(X = A'B'C'\).

The task is to determine if these partitions are topologically equivalent but geometrically distinct.

### Step-by-Step Analysis:

1. **Understanding Topological Equivalence**:
   - Two partitions are topologically equivalent if they have the same number of connected components and the same connectivity structure. In other words, the way the regions are connected should be the same, even if their spatial arrangement differs.

2. **Analyzing the Partitions**:
   - Partition \(X = ABC\): This means the annulus is divided into three regions, \(A\), \(B\), and \(C\).
   - Partition \(X = A'B'C'\): This means the annulus is also divided into three regions, \(A'\), \(B'\), and \(C'\).

3. **Connectivity Check**:
   - For both partitions, the annulus is divided into three regions. The connectivity structure should be the same. If the annulus is divided into three regions, there must be at least one region that shares a boundary with the other two regions. This is consistent for both partitions since they both divide the annulus into three regions.

4. **Geometric Distinctness**:
   - The partitions \(ABC\) and \(A'B'C'\) can differ in their spatial arrangement. For example, the regions \(A\), \(B\), and \(C\) could be arranged differently from the regions \(A'\), \(B'\), and \(C'\). The specific way these regions overlap or are positioned relative to each other can vary, making them geometrically distinct.

### Conclusion:
Since both partitions divide the annulus into three regions with the same connectivity structure but differ in their spatial arrangement, they are topologically equivalent but geometrically distinct.

Thus, the answer is:
\[
\boxed{\text{Yes}}
\]