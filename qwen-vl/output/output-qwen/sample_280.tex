### (a) Gauge Invariant Wilson Operators That Exist in Any Theory with Long-Range Gauge Symmetry

Wilson operators, also known as Wilson loops, are fundamental objects in quantum field theory that describe the effect of a closed path in spacetime on the gauge fields. They are gauge-invariant and can be defined for any gauge theory, regardless of whether the gauge symmetry is local or non-local.

#### Definition:
A Wilson loop \( W(C) \) around a closed contour \( C \) in spacetime is given by:
\[
W(C) = \text{Tr} \left( \mathcal{P} \exp \left( i \int_C A_\mu(x) dx^\mu \right) \right)
\]
where \( \mathcal{P} \) denotes the path-ordering operator, \( A_\mu(x) \) is the gauge potential, and \( dx^\mu \) is the line element along the contour \( C \).

#### Properties:
1. **Gauge Invariance**: The Wilson loop is gauge-invariant because it is constructed from the gauge potential, which transforms under gauge transformations. The path-ordering ensures that the order of the gauge fields along the contour is preserved, making the expression independent of the choice of gauge.
2. **Long-Range Symmetry**: Wilson loops can be defined in theories with long-range gauge symmetries, such as Yang-Mills theories with massless gauge bosons. In these cases, the gauge fields extend over large distances, and the Wilson loop captures the behavior of the gauge field configuration along the closed contour.

#### Examples:
- **Yang-Mills Theory**: In this case, the gauge potential \( A_\mu(x) \) is a matrix-valued function, and the Wilson loop is a trace of the path-ordered exponential of the gauge potential along a closed contour.
- **Chern-Simons Theory**: This is a topological quantum field theory where the gauge potential is a connection on a principal bundle, and the Wilson loop is a phase factor that depends on the topology of the contour.

### (b) Additional Gauge Invariant Wilson Operators That Only Exist in Gauge Theories with Local Charged Operators

In addition to the standard Wilson loops described above, there are other types of gauge-invariant Wilson operators that are specific to gauge theories with local charged operators. These operators are more complex and often involve additional degrees of freedom or interactions.

#### Examples:

1. **Loop Operators with Fermionic Charges**:
   - **Fermion Loop Operators**: In gauge theories with fermions, one can consider Wilson loops that enclose both gauge fields and fermions. These operators are gauge-invariant and can be used to study the interaction between the gauge fields and the fermions.
   - **Example**: Consider a Wilson loop that encloses a closed contour \( C \) and includes a fermion field \( \psi \). The operator would be:
     \[
     W_{\text{fermion}}(C) = \text{Tr} \left( \mathcal{P} \exp \left( i \int_C A_\mu(x) dx^\mu + i \int_C \bar{\psi}(x) \gamma^\mu \partial_\mu \psi(x) dx^\mu \right) \right)
     \]
     Here, \( \bar{\psi} \) and \( \psi \) are the Dirac conjugate and the spinor field, respectively, and \( \gamma^\mu \) are the Dirac matrices.

2. **Non-Abelian Wilson Loops with Matter Fields**:
   - **Non-Abelian Wilson Loops**: In non-Abelian gauge theories, the gauge fields are represented by matrices, and the Wilson loop involves the trace of the product of these matrices along the contour.
   - **Example**: For a non-Abelian gauge theory with matter fields, the Wilson loop might include the matter fields as well:
     \[
     W_{\text{non-Abelian}}(C) = \text{Tr} \left( \mathcal{P} \exp \left( i \int_C A_\mu(x) dx^\mu + i \int_C \bar{\psi}(x) \gamma^\mu \partial_\mu \psi(x) dx^\mu \right) \right)
     \]
     Here, the trace is taken over the gauge group and the matter fields.

3. **Topological Wilson Loops**:
   - **Chern-Simons Theory**: In Chern-Simons theory, the gauge potential is a connection on a principal bundle, and the Wilson loop is a phase factor that depends on the topology of the contour. This type of Wilson loop is particularly important in topological quantum field theories.
   - **Example**: The Chern-Simons Wilson loop is given by:
     \[
     W_{\text{CS}}(C) = \exp \left( i \frac{k}{4\pi} \int_C \epsilon^{\mu\nu\rho} F_{\mu\nu} dx^\rho \right)
     \]
     where \( k \) is an integer, \( F_{\mu\nu} \) is the field strength tensor, and \( \epsilon^{\mu\nu\rho} \) is the Levi-Civita symbol.

These additional Wilson operators provide deeper insights into the structure and dynamics of gauge theories, especially those involving fermions and non-Abelian gauge fields. They are crucial for understanding phenomena such as confinement, chiral symmetry breaking, and the role of topological effects in quantum field theories.