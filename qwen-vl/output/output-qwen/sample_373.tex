It seems like you're referring to a geometric figure that might be a square, but it's not explicitly stated whether this square is the unit square required for applying Lemma lemma-fiberstructure1. Let's break down the situation:

1. **Lemma lemma-fiberstructure1**: This lemma likely pertains to some specific property or structure in geometry or topology, where the unit square plays a crucial role. For instance, it might involve properties related to projections, fibers, or other geometric transformations.

2. **The Parallelogram**: You mentioned that the parallelogram should ideally be the unit square. However, the image provided is a black and white picture of a square, which could represent either a square hole or a square object.

3. **Context Dependence**: The interpretation of the square as a square hole or an object depends on the context. If it's a square hole, it might be part of a larger structure or diagram where the unit square is necessary for the lemma to hold true. If it's an object, the lemma might still apply if the object can be mapped to a unit square through some transformation.

Given these points, here’s how we can approach the problem:

- **Assumption**: Assume the square in the image is meant to be the unit square unless otherwise specified.
- **Application of Lemma**: If the square is indeed the unit square, then Lemma lemma-fiberstructure1 can be directly applied. If the square is not the unit square, additional information or a transformation might be needed to map it to the unit square.

If you provide more details about the context or any additional information about the square (e.g., its dimensions, relationships with other shapes, etc.), I can offer a more precise analysis.