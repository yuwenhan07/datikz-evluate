It seems like you're referring to a graphical representation of a Matrix Product State (MPS), which is often used in quantum computing and condensed matter physics to describe the state of a one-dimensional quantum system. An MPS can be visualized as a chain of matrices, where each matrix represents the state of a single qubit.

In this context, the "white line" you mentioned could be a simplified or abstract representation of the MPS. Here's how you might interpret it:

1. **Matrix Representation**: Each square in the line could represent a matrix \(A^{(i)}\), which is associated with a qubit at position \(i\). These matrices are typically \(d \times d\) matrices where \(d\) is the dimension of the Hilbert space of the qubit (e.g., 2 for a qubit).

2. **Chain Structure**: The line itself represents the chain structure of the MPS. Each square is connected to the next, forming a chain that extends along the length of the system.

3. **Color or Symbol**: If there are any colors or symbols within the squares, they might indicate different states or parameters of the qubits. For example, different colors could represent different values of the matrices, or symbols could denote specific operations or transformations applied to the qubits.

4. **White Background**: The white background suggests that the focus is on the line itself, which is the MPS representation.

If you have a clearer image or a more detailed description, I can provide a more accurate interpretation. However, based on your description, the image likely shows a simplified or abstract representation of an MPS, where the line is the MPS itself and the squares represent the matrices \(A^{(i)}\).