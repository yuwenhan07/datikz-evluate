The image you've described seems to be illustrating concepts from geometry, particularly related to the dual simplicial complex. In this context:

- **Tetrahedron**: A three-dimensional geometric shape with four triangular faces, six edges, and four vertices (A, B, C, and D). It's a type of polyhedron.

- **Dual Simplicial Complex**: This refers to a concept in topology where each face of one polyhedron corresponds to a vertex in another polyhedron. For example, if we have a tetrahedron, its dual would be a tetrahedron where each vertex of the original tetrahedron becomes a face of the dual, and vice versa.

In your description:
- The tetrahedron is a three-dimensional object.
- The smaller triangular shape labeled as XYZ appears to be a projection or a representation of a face of the tetrahedron onto a two-dimensional plane. This could be a perspective view or a cross-section of the tetrahedron.

If XYZ represents a face of the tetrahedron, then XYZ would correspond to one of the triangular faces of the original tetrahedron (ABC, ABD, ACD, or BCD).

The vertex state might refer to the labeling of the vertices of the tetrahedron, which you've already mentioned: A, B, C, and D. These labels can be used to identify specific points on the tetrahedron for further analysis or calculations.

If you're looking to understand more about the dual simplicial complex or need help with specific calculations involving these shapes, feel free to ask!