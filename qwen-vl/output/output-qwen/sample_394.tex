It seems like you're describing a visual representation or diagram rather than a specific mathematical or physical problem. However, based on your description, it appears that you might be referring to a scenario where the "neck" \( n^i_n \) is being replaced with "sub-necks" that are more tailored to certain "marked points." In this context, the "marked points" could be the small white dots on the white wall.

Here's how we can interpret and visualize this:

1. **Original Neck (\( n^i_n \)):** This likely refers to a general region or area on the wall.
2. **Sub-Necks:** These are smaller, more specific regions that are better suited for the marked points (the white dots).
3. **Shaded Regions:** These represent the new sub-necks, which are visually distinct from the original neck and are positioned around the marked points.

Given the description:
- The wall is white.
- There is a series of small white dots arranged in a line.
- The dots are evenly spaced.
- The dots cover the entire length of the wall.

If we were to draw this out, we would have a white wall with a series of evenly spaced white dots running along its length. The shaded regions (sub-necks) would then be areas around each dot, highlighting the regions that are specifically adapted to these marked points.

In summary, the image you described is a simple illustration of a white wall with evenly spaced white dots, and the shaded regions represent the sub-necks that are better adapted to these marked points. If you need further clarification or assistance with a specific mathematical or physical problem related to this description, please provide more details!