The scenario you've described seems to be related to the concept of indirect measurement in quantum mechanics, where a meter (which could represent a detector or an observable) interacts with a quantum system to extract information about the system's state. This is a fundamental aspect of quantum measurement theory.

In this context, the whiteboard likely contains equations that describe the interaction between the meter and the quantum state, as well as the process of measurement. Here’s a breakdown of what might be present:

1. **Quantum State Representation**: The quantum state of the system is often represented by a vector in a Hilbert space. For example, if the system has two possible states (like spin-up and spin-down), the state can be written as:
   \[
   |\psi\rangle = \alpha |0\rangle + \beta |1\rangle
   \]
   where \(|\alpha|^2\) and \(|\beta|^2\) are the probabilities of finding the system in states \(|0\rangle\) and \(|1\rangle\), respectively.

2. **Interaction Hamiltonian**: The interaction between the meter and the quantum system is often described by a Hamiltonian \(H_{int}\). This Hamiltonian typically depends on the state of the meter and the state of the quantum system. For instance, if the meter is a spin detector, the interaction Hamiltonian might involve terms like:
   \[
   H_{int} = g S_z \sigma_z
   \]
   where \(S_z\) is the spin operator of the system and \(\sigma_z\) is the Pauli z-matrix for the meter.

3. **Measurement Process**: When the system interacts with the meter, the total Hamiltonian becomes:
   \[
   H_{total} = H_{sys} + H_{int}
   \]
   where \(H_{sys}\) is the Hamiltonian of the system alone. The time evolution of the combined system is governed by \(H_{total}\).

4. **Evolution and Measurement Outcome**: The system evolves according to Schrödinger's equation until the meter is read. The measurement outcome \(m\) is determined by the state of the meter after the interaction. This could be a classical observable like the position or momentum of the meter, which is then recorded.

5. **Mathematical Symbols and Equations**: The whiteboard might contain equations such as:
   - The Schrödinger equation: 
     \[
     i\hbar \frac{\partial}{\partial t} |\psi(t)\rangle = H_{total} |\psi(t)\rangle
     \]
   - The Born rule for calculating probabilities:
     \[
     P(m) = |\langle m | \psi_f \rangle|^2
     \]
     where \(|\psi_f\rangle\) is the final state of the system after the interaction, and \(|m\rangle\) is the eigenstate of the meter corresponding to the measurement outcome \(m\).

6. **Circuit Diagrams**: The circles with lines connecting them could represent a circuit diagram illustrating the interaction between the meter and the quantum system. Each circle might represent a component of the system (e.g., a qubit, a detector, etc.), and the lines indicate the flow of energy or information between these components.

In summary, the whiteboard likely contains a combination of quantum mechanical equations and circuit diagrams to illustrate the process of indirect measurement in quantum systems. The equations would describe the dynamics of the system and the meter, while the circuit diagrams would provide a visual representation of how these components interact.