It seems like you're describing a geometric figure that could be interpreted in several ways depending on the context. Given the details provided:

- A white triangle on a black background.
- The triangle appears to represent a slice of pie.
- There are three lines of equal length extending from a common point (which might suggest an equilateral triangle).
- The triangle has labels, including a curve labeled "u(x)".

Let's break down the elements:

1. **White Triangle on a Black Background**: This is likely a visual representation of a geometric shape, possibly a slice of pie or a sector of a circle.

2. **Slice of Pie**: If this is indeed a slice of pie, it could be a sector of a circle. The slice would have two radii and an arc forming its boundary.

3. **Three Lines of Equal Length Extending from a Common Point**: This suggests an equilateral triangle if we consider the slice as a sector of a circle. In a circle, the radii are equal, and if the slice is such that the three lines are equal, it could imply symmetry.

4. **Curve Labeled "u(x)":** This could represent a function or a mathematical relationship. In the context of a pie slice, "u(x)" might describe the angle or some other variable related to the slice.

Given these observations, let's interpret the case mathematically:

### Case Analysis:
- **k > 0**: This indicates a positive constant, which could relate to the size or scale of the slice.
- **y_1 > u(θ)**: This inequality suggests that there is a value \( y_1 \) that is greater than the value of the function \( u(\theta) \) at some angle \( \theta \).

### Possible Interpretation:
If we assume the slice of pie is a sector of a circle, then:
- \( u(x) \) could represent the angle \( x \) in radians.
- \( y_1 \) could be a specific height or radius-related measurement within the slice.

The inequality \( y_1 > u(\theta) \) implies that the height \( y_1 \) is greater than the angle \( \theta \) in radians for some specific angle \( \theta \).

### Conclusion:
The image likely represents a sector of a circle where the angle \( u(x) \) is being compared to a height \( y_1 \). The condition \( k > 0 \) and \( y_1 > u(\theta) \) suggests that the height \( y_1 \) is greater than the angle \( \theta \) in radians, possibly indicating a specific geometric or trigonometric relationship within the slice of pie.