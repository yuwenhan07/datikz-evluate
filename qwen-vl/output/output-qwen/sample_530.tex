The graph you've described appears to be a visualization of how the final sparsity percentage (\( s_f \)) varies with the size of the models and their Mean Absolute Error (MAE). Here's a breakdown of what this graph might be illustrating:

1. **X-Axis (Model Size):** This axis represents the size of the models, which could be measured in terms of parameters, number of layers, or some other metric that scales with the complexity of the model. The range is from 20 to 87.5, suggesting a logarithmic scale or a specific range of model sizes.

2. **Y-Axis (Final Sparsity Percentage):** This axis measures the percentage of the model that remains after pruning or optimization processes. A higher percentage indicates a sparser model, which can lead to better efficiency and potentially lower computational costs.

3. **Color-Coded Lines:** Each line represents a different model, allowing for easy comparison of how sparsity changes across different models as their size increases. Different colors help distinguish between these models, making it possible to see trends and patterns more clearly.

4. **Relationship Between Model Size and Final Sparsity:** The graph likely shows that as the size of the models increases, the final sparsity percentage also increases. This suggests that larger models may be more amenable to pruning or optimization techniques, leading to a higher percentage of sparsity while maintaining good performance.

5. **Mean Absolute Error (MAE):** Although not explicitly mentioned in the description, MAE is often used as a measure of the accuracy of predictions made by a model. It is possible that the graph includes MAE as an additional metric, perhaps shown as a secondary axis or as a separate set of data points, to provide a comprehensive view of the trade-off between model size, sparsity, and accuracy.

In summary, the graph provides insights into how the final sparsity of models changes with their size, and it may also indicate the impact of these changes on the model's accuracy (as measured by MAE). This kind of analysis is crucial for understanding the balance between model complexity, computational efficiency, and predictive performance.