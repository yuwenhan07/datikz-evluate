The setting you've described seems to be related to a mathematical or geometric context, particularly one involving polar coordinates and the behavior of certain functions or transformations in that space. Here's how we can interpret this:

1. **Circle and Radius**: The circle has a radius of 2, which means it extends 2 units outwards from its center. The circle is filled with a dark color, suggesting it might represent a region of interest in some analysis.

2. **Polar Coordinates**: The labels on the circle, such as \(t\), are likely indicating points along the circle in terms of their polar coordinates. In polar coordinates, a point is defined by its radial distance (radius) from the origin and its angle (theta) from a reference direction (often the positive x-axis).

3. **Lemma lem:A_z_explicit**: This suggests that there is a lemma or proposition named "lem:A_z_explicit" which might describe a specific property or result about the function or transformation \(A_z\) evaluated at a particular point \(z\) within the circle. The context provided indicates that this point \(z\) could be specified using the polar coordinate \(t\).

### Possible Interpretation:
- **Function \(A_z\)**: This could be a function that depends on the position \(z\) within the circle. For example, \(A_z\) might represent an amplitude, a potential, or another physical quantity that varies depending on the location.
- **Explicit Form**: The term "explicit" implies that the lemma provides a clear, direct formula or expression for \(A_z\) when \(z\) is specified by the polar coordinate \(t\).

### Example:
If \(A_z\) represents the amplitude of a wave at a point \(z\) in the circle, then the lemma might state something like:
\[ A_z = f(t) \]
where \(f(t)\) is a function of the angle \(t\) (or equivalently, the polar coordinate angle). This function \(f(t)\) would describe how the amplitude changes as the point \(z\) moves around the circle.

### Conclusion:
The setting of the lemma involves a circle of radius 2 in a plane, where the points on the circle are labeled with polar coordinates \(t\). The lemma "lem:A_z_explicit" likely describes a specific value or behavior of the function \(A_z\) at these points, providing an explicit formula or expression for \(A_z\) in terms of \(t\).