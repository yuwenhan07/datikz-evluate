The scenario you've described involves a specific configuration in the context of AdS/CFT correspondence, particularly focusing on the geometry of Euclidean Anti-de Sitter (AdS) space and its relation to conformal field theories (CFTs).

### Key Elements:

1. **Euclidean Poincaré AdS_3**:
   - This refers to a three-dimensional Anti-de Sitter space with Euclidean signature. In this context, the metric is typically given by:
     \[
     ds^2 = \frac{dr^2}{R^2 - r^2} + r^2 d\Omega_2^2,
     \]
     where \(d\Omega_2^2\) represents the line element for a two-sphere.

2. **ETW Brane (Gray)**:
   - The "ETW" likely stands for "Eguchi-Townsend-Witten" brane, which is a specific type of brane that appears in certain string theory and M-theory configurations. It is often used as a boundary condition or a reference point in the study of AdS/CFT correspondence.
   - The brane is centered at \(x=0\) and lies at a constant radius \(r=R\) in the AdS space. This means it is located at a distance \(R\) from the origin in the radial direction.

3. **Minimal Surface (Blue)**:
   - A minimal surface in this context is a surface that minimizes its area subject to certain boundary conditions. Here, the minimal surface is anchored to \(x=L\) and forms a portion of a semi-circle in the \(x-z\) plane.
   - The minimal surface meets the ETW brane orthogonally, meaning that at the points of intersection, the normal vectors of the minimal surface and the brane are perpendicular to each other.

4. **Orthogonal Intersection**:
   - The fact that the minimal surface intersects the brane orthogonally suggests that the minimal surface is a solution to the Dirichlet problem in the AdS space, where the boundary conditions are specified on the brane.

### Visualization:
- The brane is a flat, circular surface in the \(x-z\) plane at \(r=R\).
- The minimal surface is a curved surface that starts at \(x=L\) and extends radially outward, forming a portion of a semi-circle in the \(x-z\) plane.
- The minimal surface meets the brane at points where the tangent planes are perpendicular, indicating that the minimal surface is a geodesic in the AdS space that is orthogonal to the brane.

This setup is often studied in the context of AdS/CFT correspondence, where the minimal surface can be interpreted as a dual object in the CFT side, representing a particular state or configuration of the conformal field theory. The orthogonal intersection condition is crucial for ensuring that the minimal surface is a valid solution to the boundary value problem in the AdS space.