It seems you're referring to a Gauss diagram, which is a tool used in knot theory to represent a knot or link. A Gauss diagram consists of a circle with chords connecting points on the circle, where each chord represents a crossing in the knot or link. The numbers on the circle indicate the order of the crossings as one traverses the circle.

The conditions you mentioned (1)--(3) likely refer to specific properties that must be satisfied for a Gauss diagram to correspond to a realizable knot or link. These conditions might include:

1. **Orientation**: Each chord should be oriented consistently, meaning if you follow the orientation of the chord, it should form a consistent direction around the circle.
2. **Crossing Information**: The numbers on the circle should correctly indicate the over/under information at each crossing point.
3. **Consistency**: The diagram should be consistent with the rules of knot theory, such as the fact that no two chords can cross more than once and that the diagram should be able to be drawn without any self-intersections outside the circle.

If the Gauss diagram you described does not satisfy these conditions, then it would indeed be considered non-realizable. This means there is no actual knot or link that corresponds to this diagram because it violates some fundamental property of knots or links.

To determine whether a given Gauss diagram is realizable, one typically checks:
- That the orientation is consistent.
- That the crossing information is correct.
- That the diagram can be drawn without self-intersections outside the circle.

If any of these checks fail, the diagram is non-realizable. If all checks pass, the diagram could potentially represent a realizable knot or link, though further analysis may be required to confirm its existence.