The performance of denoising algorithms like Noise2Clean can indeed be influenced by the size of the training set used for their training phase. Generally, increasing the size of the training set can lead to better generalization and potentially improved denoising performance due to more diverse examples being available for learning.

Here's a breakdown of how this might work:

1. **Noise2Clean**: This is a specific type of denoising algorithm that typically involves a deep learning model trained on a dataset of noisy images paired with their corresponding clean versions. As the training set size increases, the model has access to more examples of noisy-to-clean transformations. This can help the model learn more robust features and patterns that generalize well to unseen noise types or levels.

2. **Dataset-Free Methods**: These methods, such as wavelet-based denoising or certain types of filtering techniques, do not rely on a training set. Instead, they operate based on predefined rules or filters. The performance of these methods is often considered constant because they do not benefit from additional training data. Their effectiveness depends on the choice of parameters and the inherent properties of the method itself.

### Key Points:
- **Training Set Size**: Larger training sets generally lead to better performance in supervised learning tasks like denoising.
- **Generalization**: More data allows the model to learn more robust representations that generalize better to new, unseen data.
- **Noise2Clean Performance**: As the training set size grows, the denoising performance of Noise2Clean should improve, assuming the model architecture and hyperparameters remain fixed.
- **Dataset-Free Methods**: These methods maintain a consistent performance level regardless of the training set size because they don't rely on learning from data but rather on fixed algorithms or heuristics.

### Practical Considerations:
- **Resource Constraints**: Larger training sets require more computational resources (memory and processing power) and time to train the model.
- **Overfitting**: While more data can help, there's a risk of overfitting if the model becomes too complex relative to the amount of training data. Regularization techniques may be necessary to prevent this.

In summary, scaling up the training set for Noise2Clean can enhance its denoising performance, while dataset-free methods will continue to perform at a relatively constant level.