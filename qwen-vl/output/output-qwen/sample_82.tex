It seems like you're describing a visual representation of a mathematical problem involving curves and their coordinates. Let's break down your description into parts:

### Part (a): Local Picture of \(_i\) on the Surface

This likely refers to a specific point or segment of the curve represented by the equation \(_i\). Since the equations are labeled as coordinates of a curve, \(_i\) could be a particular index or identifier for a specific point or section of the curve.

### Part (b): Polygon Containing \(_i\)

This might refer to a geometric shape (such as a polygon) that encloses or contains the point or segment identified by \(_i\). This could be useful in understanding the spatial relationship between the curve and other elements in the problem, such as boundaries or regions of interest.

### General Context

Given that there are three equations representing the coordinates of a curve, it suggests that the problem involves analyzing the curve in some way. Here’s how you might interpret this setup:

1. **Equation 1 and Equation 2**: These are likely simpler forms of the curve's parametric or implicit equations. They might describe basic properties or segments of the curve.
   
2. **Equation 3**: This is more complex and is placed below the others, possibly indicating that it represents a more intricate aspect of the curve, such as a transformation, a special case, or a boundary condition.

### Possible Interpretations

- **Curve Analysis**: You might be asked to analyze the curve defined by these equations, perhaps finding its points of intersection, its curvature, or its behavior within certain regions.
  
- **Geometric Constraints**: The polygon containing \(_i\) could be used to define constraints or regions where the curve must lie. For example, if \(_i\) is a point on the curve, the polygon might define a region where the curve cannot extend beyond.

### Steps to Solve the Problem

1. **Understand the Equations**: Analyze each equation to understand what they represent. If they are parametric equations, express \(x\) and \(y\) in terms of a parameter \(t\).

2. **Identify Key Points**: Determine the coordinates of the point or segment \(_i\) using the equations. If necessary, solve for specific values of the parameter \(t\).

3. **Visualize the Curve**: Plot the curve based on the equations. Use graphing tools if available to visualize the curve and the polygon.

4. **Analyze the Relationship**: Examine how the point or segment \(_i\) relates to the polygon. Does it lie inside the polygon? Is it on the boundary? How does the polygon constrain the curve?

5. **Solve for Specific Cases**: If the problem involves solving for specific conditions (e.g., finding the area under the curve within the polygon), use appropriate calculus techniques.

If you provide the actual equations, I can help you with more specific steps and calculations!