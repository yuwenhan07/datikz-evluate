To provide a clear understanding of the schematic picture you're referring to, let's break down the components step by step:

### Components of the Schematic Picture

1. **Points and Branch Cuts**:
   - The schematic shows the configuration of points and branch cuts for an integral involving complex variables \(u\) and \(t\), where \(0 < u, t < 1\).
   - The branch cuts are represented as intervals \([c_1, c_2]\) and \([d_1, d_2]\).

2. **Poles**:
   - The poles of the integral are located at specific points \(a_1, a_2, b_1,\) and \(b_2\).

3. **Roots of Polynomials**:
   - The points \(c_1, c_2, d_1,\) and \(d_2\) are roots of the polynomial \(Q_1\).
   - The points \(a_1, a_2, b_1,\) and \(b_2\) are roots of the polynomial \(Q_2\).

4. **Theorem Reference**:
   - The roots of the polynomials \(Q_1\) and \(Q_2\) are derived from Theorem `thm:rewrite`.

### Configuration in the Schematic

- **Branch Cuts**:
  - The branch cut \([c_1, c_2]\) is a line segment on the complex plane.
  - The branch cut \([d_1, d_2]\) is another line segment on the complex plane.

- **Poles**:
  - The poles \(a_1\) and \(a_2\) are located within the region bounded by the branch cuts.
  - The poles \(b_1\) and \(b_2\) are also located within the same region.

### Assumptions:
- The schematic assumes that \(u\) and \(t\) are sufficiently small, which influences the positions of the points and the behavior of the integral.

### Summary:
The schematic provides a visual representation of the integral's domain, highlighting the locations of the poles and the branch cuts. The roots of the polynomials \(Q_1\) and \(Q_2\) determine the exact positions of these elements in the complex plane. This configuration is crucial for understanding the behavior of the integral under different conditions of \(u\) and \(t\).

If you need further clarification or have additional questions about this setup, feel free to ask!