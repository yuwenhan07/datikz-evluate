The whiteboard in the image appears to contain various mathematical elements, including shapes and curves that could represent different functions or concepts in mathematics. However, the specific question asks for a counterexample related to a bounded viscosity coefficient.

A viscosity coefficient is often used in the context of partial differential equations, particularly in the study of fluid dynamics and heat transfer. A bounded viscosity coefficient means that the coefficient does not exceed a certain upper limit and does not fall below a certain lower limit within the domain of interest.

To provide a counterexample, let's consider a scenario where the viscosity coefficient is not bounded. Suppose we have a function \( u(x) \) that describes the velocity of a fluid at position \( x \). If the viscosity coefficient \( \mu \) were unbounded, it would mean that \( \mu \) could take arbitrarily large values or become zero at some points, which would lead to singularities or undefined behavior in the governing equations (e.g., Navier-Stokes equations).

For instance, if we consider a situation where the viscosity coefficient \( \mu \) becomes zero at a point \( x_0 \), this would imply that the fluid has no resistance to shear at that point, which is physically unrealistic. This kind of singularity can be a counterexample to the assumption of a bounded viscosity coefficient.

In summary, a counterexample for a bounded viscosity coefficient would involve a situation where the viscosity coefficient is allowed to become zero or take arbitrarily large values at certain points, leading to undefined or singular behavior in the mathematical model. This is a theoretical construct rather than a visual one, but it aligns with the mathematical concept being discussed.