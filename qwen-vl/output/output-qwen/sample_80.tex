The second step in encryption typically involves the application of a specific algorithm that transforms the data into an unreadable format. Given your description of the image as an artistic representation of a network or structure, it could be interpreted as a visual metaphor for a complex system, such as a digital network or a mathematical graph.

Here’s how you might proceed with encryption:

1. **Choose an Encryption Algorithm**: Select an encryption algorithm that suits your needs. Common algorithms include AES (Advanced Encryption Standard), RSA (Rivest-Shamir-Adleman), or DES (Data Encryption Standard). For simplicity, let's assume we're using AES.

2. **Key Generation**: Generate a key for the encryption process. This key will be used to encrypt the data and must be kept secret. AES requires a 128-bit, 192-bit, or 256-bit key.

3. **Data Preparation**: Convert the image into a format suitable for encryption. This often involves converting the image into a binary format where each pixel can be represented by its RGB values or grayscale values if the image is black and white.

4. **Encryption Process**:
   - **Split the Image**: Divide the image into smaller chunks or blocks.
   - **Encrypt Each Block**: Apply the AES algorithm to each block using the generated key. This will transform each block into a cipher text block.
   - **Combine the Encrypted Blocks**: Reassemble the encrypted blocks back into a single encrypted image.

5. **Output the Encrypted Image**: The final output will be an encrypted version of the original image, which cannot be directly understood without the decryption key.

If you have the image file, you can use software tools like GIMP, Photoshop, or specialized encryption software to perform these steps. If you need further assistance with the technical details or specific software recommendations, feel free to ask!