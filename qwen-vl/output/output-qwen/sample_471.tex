It seems like you're describing a conceptual or theoretical framework rather than a specific mathematical structure. However, I can provide some context on the six Stabilizer Operator Classes (SLOCC) in quantum information theory, which might be what you're referring to.

In quantum information theory, stabilizer operators play a crucial role in understanding the structure of quantum states, particularly those that can be efficiently manipulated using local operations and classical communication (LOCC). The six SLOCC classes are:

1. **Product States**: These are states that can be written as a tensor product of single-qubit states. For example, \( |0\rangle \otimes |0\rangle \).

2. **Bell States**: These are maximally entangled two-qubit states. They are represented by:
   - \( |\Phi^+\rangle = \frac{1}{\sqrt{2}}(|00\rangle + |11\rangle) \)
   - \( |\Phi^-\rangle = \frac{1}{\sqrt{2}}(|00\rangle - |11\rangle) \)
   - \( |\Psi^+\rangle = \frac{1}{\sqrt{2}}(|01\rangle + |10\rangle) \)
   - \( |\Psi^-\rangle = \frac{1}{\sqrt{2}}(|01\rangle - |10\rangle) \)

3. **GHZ States**: Greenberger-Horne-Zeilinger states are a class of three-qubit entangled states. An example is:
   - \( |\text{GHZ}\rangle = \frac{1}{\sqrt{2}}(|000\rangle + |111\rangle) \)

4. **W States**: W states are another class of three-qubit entangled states. An example is:
   - \( |\text{W}\rangle = \frac{1}{\sqrt{3}}(|001\rangle + |010\rangle + |100\rangle) \)

5. **BC States**: Bell Cluster states are a generalization of GHZ states to an arbitrary number of qubits. They are entangled states where one qubit is in a maximally entangled state with the rest of the system.

6. **ABT States**: Anyons Bound Together (ABT) states are a class of states that describe the entanglement of anyons in a topological quantum computation setting. These states are more abstract and typically involve higher-dimensional Hilbert spaces.

These classes represent different types of entanglement structures that can be achieved through LOCC. Each class has its own properties and applications in quantum computing and quantum information processing.

If the "tree" you mentioned refers to a hierarchical structure of these classes, it would likely show how each class can be decomposed into simpler ones, starting from product states and moving up to more complex entangled states like GHZ and W states. This kind of hierarchical representation helps in understanding the relationships and transformations between different types of entangled states.