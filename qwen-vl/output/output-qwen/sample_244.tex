The inequality \( d(x, y) + d(y, z) \geq d(x, z) \) is known as the triangle inequality in the context of metric spaces. This inequality holds for any three points \( x, y, \) and \( z \) in a metric space, where \( d \) denotes the distance function.

To represent the different integration paths that lead to this inequality, we need to understand that the triangle inequality is not an integral but rather a fundamental property of metrics. However, if you're looking at how this inequality can be visualized or understood through paths, here's a conceptual approach:

### Geometric Interpretation:
1. **Triangle \( XYZ \)**: Consider a triangle with vertices \( X, Y, \) and \( Z \). The sides of the triangle are the distances between these points: \( XY = d(x, y) \), \( YZ = d(y, z) \), and \( XZ = d(x, z) \).

2. **Path from \( X \) to \( Z \)**: There are two possible paths from \( X \) to \( Z \):
   - Direct path: \( X \to Z \)
   - Indirect path: \( X \to Y \to Z \)

3. **Inequality Visualization**:
   - The direct path length is \( d(x, z) \).
   - The indirect path length is \( d(x, y) + d(y, z) \).

The triangle inequality states that the sum of the lengths of any two sides of a triangle must be greater than or equal to the length of the remaining side. Therefore, in our case:
\[ d(x, y) + d(y, z) \geq d(x, z) \]

This means that the total distance traveled along the indirect path (via \( Y \)) cannot be less than the direct distance from \( X \) to \( Z \).

### Mathematical Representation:
If we were to represent this mathematically using integrals, it would be more appropriate to consider the integral of the distance function over a path. For instance, if we have a curve \( \gamma \) connecting points \( x \) and \( z \), the integral of the distance function along this curve would give us the length of the path. However, the triangle inequality itself does not involve an integral but rather a comparison of distances.

### Conclusion:
The triangle inequality \( d(x, y) + d(y, z) \geq d(x, z) \) is a fundamental property of metrics and does not directly involve integration paths. It is a geometric property that can be visualized by considering the lengths of the sides of a triangle formed by the points \( x, y, \) and \( z \). The inequality ensures that the shortest path between two points is a straight line (or a curve that is shorter than any other path connecting them).