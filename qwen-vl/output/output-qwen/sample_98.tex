The description you provided suggests that we're dealing with a geometric figure, specifically a tetrahedron, which is a three-dimensional shape composed of four triangular faces. In this context, the red and blue lines seem to represent connections (edges) between certain vertices of the tetrahedron.

In a tetrahedron, there are 4 vertices and 6 edges. If we were to label these vertices as \(A\), \(B\), \(C\), and \(D\), then the edges would connect these vertices in pairs. For example, some possible edges could be \(AB\), \(AC\), \(AD\), \(BC\), \(BD\), and \(CD\).

Given the mention of "spanned by points of an absolute hexagon," it's important to note that an absolute hexagon is a concept from projective geometry, where a hexagon is considered "absolute" if opposite sides are parallel in a certain sense. However, for a standard tetrahedron, we don't typically use the term "absolute hexagon."

If the red and blue lines are meant to represent specific subsets of the edges of the tetrahedron, they might be highlighting particular relationships or properties within the structure. For instance, the red lines could represent edges that form a smaller substructure within the tetrahedron, such as a triangle, while the blue lines could represent the remaining edges.

Without more specific information about the exact arrangement of the red and blue lines, it's difficult to provide a precise interpretation. However, if we assume that the red lines form a smaller triangle within the tetrahedron, one possible configuration could be:

- Red lines: \(AB\), \(AC\), \(BC\) (forming a triangle)
- Blue lines: \(AD\), \(BD\), \(CD\)

This would divide the tetrahedron into two parts: a smaller tetrahedron formed by the red lines and a larger tetrahedron formed by the blue lines.

In summary, the red and blue lines in the image likely represent specific edges of a tetrahedron, possibly highlighting a smaller substructure within the larger tetrahedron. The exact configuration depends on the specific relationships being studied, but without further details, we can only speculate based on common geometric configurations.