It seems like you're describing an image that might be a visual representation of Feynman diagrams, which are graphical tools used in quantum field theory to represent interactions between particles. However, your description suggests that the image is not clear or well-defined, with blue and red lines appearing scrambled and creating an impression of distorted text.

In reality, Feynman diagrams are typically drawn as clear, organized graphs where vertices (dots) represent particle interactions, and lines represent the propagation of particles. Each diagram corresponds to a specific term in the perturbative expansion of the scattering amplitude, which can contribute to various physical processes.

If you're trying to understand how Feynman diagrams contribute classically to \(A_4\), it's important to note that \(A_4\) could refer to several things depending on the context:

1. **\(A_4\) as a group**: In mathematics, \(A_4\) refers to the alternating group on four elements, which has 12 elements. This group is relevant in symmetry considerations in physics, particularly in particle physics and condensed matter physics.
   
2. **\(A_4\) as a scattering amplitude**: In quantum field theory, \(A_4\) could refer to a specific scattering amplitude, such as the amplitude for a process involving four particles.

To clarify further, if you're interested in Feynman diagrams contributing to a classical scattering amplitude \(A_4\), you would need to consider the following:

- **Classical Contributions**: Classical contributions to scattering amplitudes are often neglected in quantum field theory because they are typically much smaller than quantum corrections. However, in some cases, classical contributions can be significant, especially in certain limits or in simplified models.

- **Feynman Diagrams**: Each Feynman diagram represents a term in the perturbative expansion of the scattering amplitude. For example, a one-loop diagram contributes a term proportional to \(\alpha^2\), where \(\alpha\) is the fine-structure constant.

If you have a specific context or a particular model in mind, please provide more details so I can give a more precise answer. If you're looking at a specific diagram or a set of diagrams, it would be helpful to describe them in detail or provide an image.