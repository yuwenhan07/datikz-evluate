It sounds like you're describing a specific type of function behavior that involves a "spike" and a "discontinuity," which can be visualized in a graph. Let's break down the scenario step by step:

1. **Function \( R \)**:
   - **Spike on the right end**: This means there is a sharp increase or peak in the value of the function as it approaches the point \( b \).
   - **Discontinuity at the left end**: This indicates that the function is not continuous at the point \( a \), meaning the limit from the left does not equal the limit from the right at \( a \).

2. **Flipping the Island**:
   - Flipping the island essentially means swapping the roles of \( a \) and \( b \). After flipping, the function \( R \) will now have a sharp decrease or "spike" at \( a \) and a discontinuity at \( b \).

3. **Tilted Functions \( R \) and \( B' \)**:
   - The function \( R \) is described as the maximum on vertical lines.
   - The function \( B' \) is described as the minimum on vertical lines.

Given these descriptions, let's visualize this situation with a simple example:

### Example Function \( R(x) \):
- For \( x < a \): \( R(x) = 0 \)
- At \( x = a \): \( R(x) \) is undefined or has a discontinuity (e.g., \( R(a^-) = 0 \) and \( R(a^+) \) could be very large).
- For \( a < x < b \): \( R(x) \) increases sharply to a peak at \( x = b \).
- At \( x = b \): \( R(x) \) is undefined or has a discontinuity (e.g., \( R(b^-) \) could be very large and \( R(b^+) = 0 \)).
- For \( x > b \): \( R(x) = 0 \)

### Flipped Function \( R'(x) \):
- For \( x < b \): \( R'(x) = 0 \)
- At \( x = b \): \( R'(x) \) is undefined or has a discontinuity (e.g., \( R'(b^-) = 0 \) and \( R'(b^+) \) could be very large).
- For \( b < x < a \): \( R'(x) \) decreases sharply to a trough at \( x = a \).
- At \( x = a \): \( R'(x) \) is undefined or has a discontinuity (e.g., \( R'(a^-) \) could be very large and \( R'(a^+) = 0 \)).
- For \( x > a \): \( R'(x) = 0 \)

### Tilted Functions \( R \) and \( B' \):
- \( R(x) \) is the maximum on vertical lines because for any vertical line \( x = c \) where \( a < c < b \), the maximum value of \( R(c) \) is the peak at \( b \).
- \( B'(x) \) is the minimum on vertical lines because for any vertical line \( x = c \) where \( a < c < b \), the minimum value of \( B'(c) \) is the trough at \( a \).

In summary, the function \( R \) has a sharp increase at \( b \) and a discontinuity at \( a \), while the flipped function \( R' \) has a sharp decrease at \( a \) and a discontinuity at \( b \). The function \( R \) is the maximum on vertical lines, and \( B' \) is the minimum on vertical lines.