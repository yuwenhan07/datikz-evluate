To solve this problem, we need to understand the relationship between a genus two Riemann surface and its associated cut curve, as well as the resulting weighted graph.

1. **Understanding the Cut Curve**: A genus two Riemann surface can be visualized as a surface with two "holes" or handles. A cut curve on such a surface is a simple closed curve that divides the surface into two connected components. For a genus two surface, there is exactly one such cut curve.

2. **Effect of the Cut**: When you cut along this unique cut curve, the surface is divided into two separate pieces. Each piece is now a topological disk with two boundary components (the original boundary of the surface plus the new boundary created by the cut).

3. **Weighted Graph Representation**: The cut curve can be represented as a weighted graph where each vertex represents a point on the cut curve, and each edge represents a segment of the cut curve. The weight of each edge can represent various properties, but in this context, it's often used to indicate the number of times a particular path around the surface intersects the cut curve.

Given these points, let's summarize:

- The cut curve for a genus two Riemann surface is unique.
- After cutting along this curve, the surface is divided into two components.
- The weighted graph representation of the cut system shows the structure of the cut curve, with vertices and edges corresponding to the points and segments of the cut curve.

The final answer, encapsulating the key points, is:

\[
\boxed{\text{There is only one cut curve for a genus two Riemann surface, which divides the surface into two components.}}
\]