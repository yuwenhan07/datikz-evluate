It seems like you're referring to a figure or diagram that illustrates the state transition graph (STG) and the master-gene (MG) for a toggle switch model, which is a common motif in systems biology used to describe bistable gene regulatory networks.

### Left Panel:
- **STG (State Transition Graph)**: This graph represents the dynamics of the toggle switch system. Each node in the graph corresponds to a possible state of the system, and the edges represent transitions between these states. The STG is typically derived from the differential equations governing the toggle switch, such as those described by the DSGRN (Deterministic Systems Gene Regulatory Network) parameters.
- **Rectangular Domains Dividing Phase Space**: These domains are regions in the phase space where the system's behavior can be characterized. The phase space is a space where each point represents a possible state of the system. The toggle switch model often has two stable states, and the rectangular domains help visualize how the system transitions between these states based on the DSGRN parameters.

### Right Panel:
- **MG (Master-Gene)**: This is a simplified representation of the toggle switch network. It typically consists of a single gene (the master gene) that can be in one of two states (on or off). The MG captures the essence of the toggle switch behavior but omits the details of the underlying molecular interactions. The MG is useful for understanding the qualitative behavior of the toggle switch without delving into the complexities of the full model.

### Summary:
The left panel shows the STG, which provides a detailed view of the system's dynamics, including the transitions between different states. The right panel shows the MG, which simplifies the toggle switch model to its core behavior, making it easier to understand the qualitative aspects of the system's bistability.

If you have specific questions about the parameters, the transitions, or any other aspect of the toggle switch model, feel free to ask!