The image you've described appears to be a tree diagram where the nodes represent variables and the edges represent relationships or dependencies between these variables. Given the structure and labels provided:

- The tree has four branches.
- Each branch represents a different variable: \(X_{a_2}\), \(Y_{a_2}\), \(X_{a_2}\), and another variable not explicitly named but implied by the context.
- The variables are labeled with letters (a, b, c, and d) and numbers (1, 2, 3, and 4).
- The branches are arranged such that they all meet at the same point, the center of the tree.
- Numbers 1 and 2 are closer to the root, while 3 and 4 are further away.
- Variables \(a\) and \(b\) are on the left side of the tree, and \(c\) and \(d\) are on the right side.

To determine the DS (Dependency Structure) result component contributing to \(\mu^3\), we need to understand how the variables interact and influence each other. Since the problem involves \(\mu^3\), it suggests a third-order dependency or interaction among the variables.

Given the structure:
- \(X_{a_2}\) and \(Y_{a_2}\) are likely related as they share the same subscript \(a_2\).
- \(X_{a_2}\) might also have some dependency on \(a\) or \(b\) since they are on the left side of the tree.
- \(c\) and \(d\) are on the right side and could potentially interact with \(a\) and \(b\) through the central node.

Considering the tree structure and the labels:
- If \(X_{a_2}\) and \(Y_{a_2}\) are directly dependent on each other, then their interaction could contribute to \(\mu^3\).
- The variables \(a\) and \(b\) being on the left side suggest they might influence \(X_{a_2}\) and \(Y_{a_2}\) independently or together.
- \(c\) and \(d\) being on the right side imply they might have an indirect effect on \(X_{a_2}\) and \(Y_{a_2}\) through the central node.

Thus, the DS result component contributing to \(\mu^3\) would likely involve the interaction between \(X_{a_2}\) and \(Y_{a_2}\), possibly influenced by \(a\) and \(b\) and potentially moderated by \(c\) and \(d\).

In summary, the key components contributing to \(\mu^3\) would be the interaction between \(X_{a_2}\) and \(Y_{a_2}\), with potential influences from \(a\), \(b\), \(c\), and \(d\).