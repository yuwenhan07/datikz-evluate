The statement you've made touches on an important aspect of quantum gravity and the role of topology in gravitational theories. Let's break it down step by step:

1. **Wick Rotation and Topology**: Wick rotation is a mathematical technique used to transform a real-time path integral into a complex-time one. In the context of quantum field theory, this transformation can help in dealing with divergences, particularly those related to ultraviolet (UV) and infrared (IR) behaviors.

2. **Gravitational Partition Function**: The partition function in quantum gravity is a crucial object that encodes information about the system at thermal equilibrium. It is defined as the sum over all possible states weighted by their Boltzmann factors. For a gravitational system, this includes contributions from black holes, which are singularities in spacetime.

3. **Divergences and Black Holes**: In the absence of Wick rotation, the partition function for a gravitational system can diverge due to the presence of black hole states. These states contribute to the partition function in a way that can lead to UV divergences, making the theory non-renormalizable.

4. **Role of Wick Rotation**: By performing Wick rotation, we can introduce a complex time parameter \( \tau = it \), where \( t \) is the real time. This transformation can help in suppressing certain types of divergences, particularly those associated with black hole states. The complex time parameter allows us to explore the behavior of the system in a regime where the influence of black holes is reduced or even suppressed.

5. **Topological Effects**: The topology of the Wick-rotated configuration can indeed differ from the non-gravitational case. In some cases, the topology can be such that it suppresses the contribution of gravitational single-particle states (like black holes). This suppression is often achieved through the introduction of a cosmological constant or other topological features that modify the effective action.

6. **Partition Function and Divergences**: With the suppression of black hole contributions, the partition function no longer diverges. This is because the dominant contributions come from the physical degrees of freedom rather than the singularities. The modified partition function can then be analyzed within a renormalizable framework.

In summary, the Wick rotation and the resulting change in topology play a crucial role in stabilizing the partition function in quantum gravity. By suppressing the contributions of black hole states, these techniques allow for a more manageable and physically meaningful description of the system, avoiding divergences and ensuring the theory remains renormalizable.