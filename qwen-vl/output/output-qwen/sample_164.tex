To understand the relationship between \(d(k, l)\) and \(d(j, j-1)\), let's break down the problem step by step.

### Step 1: Understanding Linear Combination
A linear combination of a set of terms means that each term in the sum is multiplied by a constant coefficient and then all these products are added together. In this context, \(d(k, l)\) is expressed as a linear combination of \(d(j, j-1)\) values from a specific region.

### Step 2: Identifying the Blue Shaded Region
The blue shaded region refers to the area where the values of \(j\) and \(j-1\) are such that they contribute to the calculation of \(d(k, l)\). This means that for each pair \((k, l)\), there is a specific range or set of pairs \((j, j-1)\) that influence \(d(k, l)\).

### Step 3: Identifying the Orange Region
The orange region refers to the area where a specific value of \(d(j, j-1)\) contributes to the calculation of \(d(k, l)\). This means that for each value of \(d(j, j-1)\), there is a specific range or set of pairs \((k, l)\) that it influences.

### Step 4: Relationship Between \(d(k, l)\) and \(d(j, j-1)\)
Given the above points, we can summarize the relationship as follows:
- Each term \(d(k, l)\) is a linear combination of specific values of \(d(j, j-1)\) from the blue shaded region.
- A specific value of \(d(j, j-1)\) contributes to the calculation of \(d(k, l)\) only within the orange region.

### Example
Let's consider a simple example to illustrate this:

Suppose we have a function \(d(k, l)\) defined over a grid, and we want to express \(d(3, 4)\) as a linear combination of \(d(j, j-1)\):

\[ d(3, 4) = a_1 d(1, 0) + a_2 d(2, 1) + a_3 d(3, 2) \]

Here:
- The blue shaded region would be the set of pairs \((j, j-1)\) that influence \(d(3, 4)\), which could be \((1, 0)\), \((2, 1)\), and \((3, 2)\).
- The orange region would be the set of pairs \((k, l)\) that are influenced by a specific value of \(d(j, j-1)\). For instance, \(d(3, 4)\) is influenced by \(d(3, 2)\) but not by \(d(1, 0)\) or \(d(2, 1)\) directly.

In summary, the blue shaded region specifies which \(d(j, j-1)\) values are used in the linear combination to form \(d(k, l)\), while the orange region specifies how each \(d(j, j-1)\) value contributes to the final value of \(d(k, l)\).