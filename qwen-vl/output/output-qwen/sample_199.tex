To solve the problem, we need to understand the structure of the ladder surface and how it relates to the given image. A ladder surface is a topological space obtained by identifying opposite edges of a polygon. In this case, the polygon is a square, so the ladder surface is a torus (a doughnut-shaped surface).

The key insight here is that the dots in the image form a path on the torus that wraps around both the "hole" and the "body" of the torus. This means that the path must be able to traverse the entire surface of the torus without lifting off the surface.

Let's analyze the image step-by-step:

1. **Identify the Path**: The dots form a continuous path that starts at one point and ends at another point after traversing the entire surface of the torus. This path must wrap around both the "hole" and the "body" of the torus exactly once each.

2. **Visualize the Path**: Imagine the torus as a square with opposite sides identified. If you start at a point on the left edge and move right, you will eventually reach the right edge. If you then move up, you will wrap around the "hole" of the torus and end up on the bottom edge. From there, if you move left, you will wrap around the "body" of the torus and end up back on the top edge. Finally, if you move down, you will return to your starting point.

3. **Count the Dots**: The number of dots in the image corresponds to the number of times the path wraps around the "hole" and the "body" of the torus. Since the path wraps around both the "hole" and the "body" exactly once each, the total number of dots is 4.

Therefore, the number of dots in the image is \(\boxed{4}\).