It seems like you're describing a topological or geometric structure involving a surface \( M \) with specific properties related to cylinders and singularities. Let's break down the information provided:

1. **Surface \( M \)**: This is the main object we're considering.
2. **Cylinder**: There is a cylinder on the surface \( M \) with a circumference \( c \).
3. **Boundary Components**: The boundary of the cylinder intersects with the surface \( M \). These intersections are labeled as singularities denoted by \( \_i \) and \( \_j \).
4. **Edges Not Labeled by**: These edges are not directly connected to the singularities \( \_i \) and \( \_j \), but they are connected to another part of the surface \( M \) denoted as \( M_C \).

Given this description, it sounds like you might be dealing with a surface that has a cylindrical hole (or handle) and some additional structure around the boundary of the cylinder.

### Possible Interpretation:
- **Surface with a Cylinder**: Imagine a surface \( M \) that has a cylindrical hole in it. The circumference \( c \) of the cylinder represents the length of the boundary of this hole.
- **Singularities**: The singularities \( \_i \) and \( \_j \) likely represent points where the surface \( M \) has some special behavior, such as a cusp or a self-intersection.
- **Edges Not Labeled by**: These edges are part of the surface \( M \) but do not connect to the singularities \( \_i \) and \( \_j \). They could be part of the rest of the surface or connected to another part of the surface \( M \) denoted as \( M_C \).

### Example:
Consider a surface \( M \) that looks like a torus (a doughnut shape) with a cylinder removed from it. The cylinder's circumference \( c \) would be the length of the boundary of the hole left after removing the cylinder. The singularities \( \_i \) and \( \_j \) could be points where the surface has a cusp or a self-intersection along the boundary of the cylinder.

If you have more specific details or a diagram, it would help in providing a more precise interpretation or further analysis.