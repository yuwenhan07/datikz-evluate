It sounds like you're describing a visual representation of a binary tree using colored dots to indicate different levels or nodes within the tree. This type of visualization can be very useful for understanding the hierarchical structure of data in a binary tree.

In iteration 3.1, you seem to be focusing on obtaining a specific coloring, denoted as `_2'`. This could refer to a particular set of rules or criteria for assigning colors to the nodes in the tree. The goal might be to achieve a certain level of uniformity or to highlight specific patterns within the tree's structure.

Here’s how you might approach this:

### Steps to Obtain Coloring `_2'`:
1. **Define the Criteria**: Clearly define what `_2'` means in the context of your binary tree. This could involve specifying which nodes should be colored which way based on their position, value, or some other attribute.
   
2. **Apply the Rules**: Use these criteria to color each node in the tree. For example, if `_2'` is about alternating colors (e.g., red and blue), you would alternate the colors as you traverse the tree.

3. **Verify the Result**: Check that the coloring adheres to the defined criteria. Ensure that no node violates the rules specified by `_2'`.

4. **Document the Process**: Keep track of the steps taken and any decisions made during the process. This documentation will be useful for future reference and for others who may need to understand the coloring method.

5. **Iterate if Necessary**: If the initial attempt does not meet the desired outcome, go back and refine the criteria or the application of the rules until the desired coloring `_2'` is achieved.

### Example:
If `_2'` specifies an alternating color scheme where every second node is colored differently (e.g., red and blue):

- Start at the root node.
- Color the root node red.
- Move to the left child of the root node and color it blue.
- Move to the right child of the root node and color it red.
- Continue this pattern down the tree.

This ensures that the tree is colored according to the specified criteria.

By following these steps, you can systematically obtain the desired coloring `_2'` for your binary tree visualization.