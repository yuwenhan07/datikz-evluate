The Atiyah-Hirzebruch spectral sequence (AHSS) is a powerful tool in algebraic topology used to compute the homology or cohomology of spaces. It is constructed from a filtration on the cochain complex of a space, and it converges to the homology or cohomology groups.

The \( E_2 \)-page of the AHSS is the second page of the spectral sequence, which is often the most important for computations because it provides a good approximation to the final answer. On this page, the entries are given by the tensor product of the cohomology of the base space with the cohomology of the fiber, modulo certain relations.

In the context of your question, you mentioned that the image is a black and white picture featuring a pattern of squares. This could be a visual representation of the \( E_2 \)-page of the AHSS, where each square represents an entry in the spectral sequence. The pattern might indicate some structure or symmetry in the cohomology groups, such as periodicity or vanishing patterns.

To interpret the pattern more precisely, one would need to know the specific details of the space and the filtration used in the construction of the AHSS. However, in general, the \( E_2 \)-page can be quite complex and may not have a simple geometric interpretation like a pattern of squares. It is typically represented as a table or grid of numbers or symbols, where each entry corresponds to a cohomology group.

If you have a specific example or more details about the space and the filtration, I could provide a more detailed explanation or even a simplified version of the \( E_2 \)-page based on those specifics.