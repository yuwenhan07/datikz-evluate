To illustrate the observable and non-observable regions for the given trajectory \( \mathbf{a}(t) = (0.5t, 0.5t, 0) \), we need to understand the constraints on the observation region. Typically, this involves a sensor or observer that can only observe points within a certain distance from the trajectory.

Let's assume the sensor can observe points up to a distance \( r \) from the trajectory. The trajectory lies in the \( xOy \) plane, so the observable region will be a cylinder around the trajectory with radius \( r \).

### Step-by-Step Illustration:

1. **Trajectory Equation:**
   The trajectory is given by:
   \[
   \mathbf{a}(t) = (0.5t, 0.5t, 0)
   \]
   This is a straight line in the \( xOy \) plane along the line \( y = x \).

2. **Observable Region:**
   The observable region is a cylinder with radius \( r \) centered along the trajectory. The equation of the cylinder in 3D space is:
   \[
   (x - 0.5t)^2 + (y - 0.5t)^2 \leq r^2
   \]
   Since the trajectory lies in the \( xOy \) plane, the \( z \)-coordinate is always zero, and the observable region in the \( xOy \) plane is a circle of radius \( r \) centered at \( (0.5t, 0.5t) \).

3. **Non-observable Region:**
   The non-observable region is everything outside the observable region. For the trajectory \( \mathbf{a}(t) = (0.5t, 0.5t, 0) \), the non-observable region will be the area outside the circle of radius \( r \) centered at each point \( (0.5t, 0.5t) \) along the trajectory.

### Visualization:

- **Observable Region:** 
  - Draw a straight line in the \( xOy \) plane along the line \( y = x \).
  - Around this line, draw circles of radius \( r \) at each point \( (0.5t, 0.5t) \). These circles represent the observable region at different times \( t \).

- **Non-observable Region:**
  - The area outside these circles represents the non-observable region. This will be the area where the sensor cannot observe any points due to the distance constraint \( r \).

### Final Illustration:

1. **Draw the Trajectory Line:**
   - Plot the line \( y = x \) in the \( xOy \) plane.

2. **Draw Circles:**
   - At each point \( (0.5t, 0.5t) \) along the trajectory, draw a circle of radius \( r \).

3. **Shade the Regions:**
   - The green area represents the observable region, which is the area inside the circles.
   - The red area represents the non-observable region, which is the area outside the circles.

This visualization will show how the observable and non-observable regions change as \( t \) varies, maintaining the circular shape around the trajectory line.