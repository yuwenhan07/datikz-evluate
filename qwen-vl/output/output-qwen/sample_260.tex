It seems like you're describing a visual representation that could be part of a mathematical or educational diagram, possibly related to geometry or set theory. Here's how we can interpret this:

1. **Concentric Circles**: These are circles that share the same center but have different radii.
2. **Descending Order of Labels**: The labels on the circles start from 'b' and go down to 'K'. This suggests an ordered sequence, which might represent a hierarchy or a specific pattern.

Given the context, here’s a possible interpretation:

- Each circle represents a set or a region.
- The labels (b, c, d, K) could denote different sets or regions within these circles.
- The order of the labels (descending) might indicate a hierarchical relationship where 'b' is the largest set and 'K' is the smallest set.

### Example Interpretation:
- Circle 'b' could represent the largest set or region.
- Circle 'c' could represent a smaller set within 'b'.
- Circle 'd' could represent an even smaller set within 'c'.
- Circle 'K' could represent the smallest set or region.

This kind of diagram is often used in set theory to illustrate relationships between sets, such as inclusion or exclusion. If you need further clarification or if there's a specific question about this diagram, feel free to ask!