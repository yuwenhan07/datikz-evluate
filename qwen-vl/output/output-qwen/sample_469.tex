It seems like you're describing a visual representation that could be part of a mathematical or algorithmic diagram, possibly related to graph theory or recursion. Here's how we can interpret and understand this:

1. **Graph Theory Context**: In graph theory, "W" having an outgoing edge to "U" suggests a directed graph where there is a connection from node W to node U. This could represent a flow or relationship between these nodes.

2. **Nested Boxes and Green Circles**: The series of nested white boxes with green circles in the middle might represent a recursive structure. Each box could symbolize a level of recursion, with the inner boxes being smaller versions of the outer ones. The green circles act as separators, indicating distinct levels or stages within the recursion.

3. **Possible Interpretation**:
   - **Recursive Function Call**: The boxes could represent function calls in a recursive algorithm. The green circles could denote the base case or the stopping condition for the recursion.
   - **Tree Structure**: The boxes could also represent a tree structure, where each node (box) has children (inner boxes). The green circles could separate the parent-child relationships.

4. **Example**: If we consider a simple example of a recursive function that calculates the factorial of a number, the boxes could represent the function calls at different levels of recursion. The green circles would separate the current call from its child calls.

Given your description, it looks like you're dealing with a visual representation of a recursive process or a tree-like structure. If you have more specific details about what you're trying to model or analyze, I can provide a more precise interpretation!