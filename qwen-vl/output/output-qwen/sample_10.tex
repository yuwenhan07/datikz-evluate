The graph you described is likely a scatter plot or a line graph that visualizes the relationship between sentence length and performance metrics such as matched words and exact sentences. Here's a step-by-step interpretation of what this type of graph might look like:

1. **Axes**: 
   - The x-axis typically represents the independent variable, which in this case is the sentence length.
   - The y-axis represents the dependent variable, which could be the performance metrics such as the number of matched words or the number of exact sentences.

2. **Lines**:
   - The blue line represents the performance metrics. This line would show how the performance metrics change as the sentence length varies.
   - The red line represents the sentence length. This line would show how the sentence length changes across different data points.

3. **Relationship**:
   - If there is a positive correlation, the blue line would generally increase as the red line increases, suggesting that longer sentences lead to better performance metrics.
   - If there is a negative correlation, the blue line would decrease as the red line increases, suggesting that longer sentences lead to worse performance metrics.
   - If there is no correlation, the blue line would remain relatively flat regardless of changes in the red line.

4. **Trends**:
   - The trend of the blue line can help identify patterns. For example, if the blue line shows an upward trend, it suggests that performance improves with increasing sentence length.
   - Conversely, if the blue line shows a downward trend, it suggests that performance decreases with increasing sentence length.

5. **Data Points**:
   - Each point on the graph represents a specific combination of sentence length and performance metric. By examining these points, one can see how individual data points align with the overall trend.

6. **Interpretation**:
   - The graph helps in understanding whether there is a direct relationship between sentence length and performance metrics. It can also provide insights into the optimal sentence length for achieving the best performance metrics.

In summary, the graph you described is a useful tool for analyzing the relationship between sentence length and performance metrics. By examining the trends and patterns in the graph, one can make informed decisions about how sentence length impacts performance.