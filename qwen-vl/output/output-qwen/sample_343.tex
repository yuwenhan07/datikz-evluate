It seems like you're describing a visual representation that might be related to theoretical physics, specifically in the context of string theory or quantum field theory, where such diagrams can appear. However, the description provided does not directly correspond to the standard depiction of Witten diagrams for gluon exchanges.

Witten diagrams are Feynman diagrams used in quantum field theory to describe scattering processes. For gluon exchanges, which are fundamental in the strong interaction described by Quantum Chromodynamics (QCD), the diagrams typically involve vertices representing the strong force and lines representing gluons.

In the context of eight-point Witten diagrams, these would represent a process involving eight particles interacting via gluons. These diagrams are complex and often involve multiple loops and vertices, reflecting the intricate nature of particle interactions at the quantum level.

The image you've described, with a circle divided into seven equal parts and a wave pattern, doesn't align with the typical visual representation of Witten diagrams. Instead, it might be a creative or abstract representation unrelated to the specific scientific content you're asking about.

If you're looking for a more accurate representation of eight-point Witten diagrams, they would likely involve multiple vertices connected by lines, with each vertex representing a particle and the lines representing the exchange of gluons. The exact structure would depend on the specific scattering process being depicted.

If you need further clarification or have additional questions regarding Witten diagrams or other aspects of theoretical physics, feel free to ask!