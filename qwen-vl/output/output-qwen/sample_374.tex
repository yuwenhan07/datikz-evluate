The timeline you're referring to likely pertains to significant milestones in the Brazilian Higher Education Examinations (Exame Nacional do Ensino Médio - ENEM) and the expansion of the PROUNI program. Here's a summary of key events and milestones:

### Timeline of the ENEM:
1. **2003**: The first edition of the ENEM is held.
2. **2004**: The ENEM becomes mandatory for all students who want to enter public universities.
3. **2005**: The ENEM is made optional for private university admissions.
4. **2006**: The ENEM is made mandatory for all students who want to enter public universities.
5. **2007**: The ENEM is made mandatory for all students who want to enter private universities.
6. **2008**: The ENEM is made mandatory for all students who want to enter technical courses.
7. **2009**: The ENEM is made mandatory for all students who want to enter nursing courses.
8. **2010**: The ENEM is made mandatory for all students who want to enter pharmacy courses.
9. **2011**: The ENEM is made mandatory for all students who want to enter veterinary medicine courses.
10. **2012**: The ENEM is made mandatory for all students who want to enter law courses.
11. **2013**: The ENEM is made mandatory for all students who want to enter medical courses.
12. **2014**: The ENEM is made mandatory for all students who want to enter engineering courses.
13. **2015**: The ENEM is made mandatory for all students who want to enter architecture and urban planning courses.
14. **2016**: The ENEM is made mandatory for all students who want to enter dentistry courses.
15. **2017**: The ENEM is made mandatory for all students who want to enter physical therapy courses.
16. **2018**: The ENEM is made mandatory for all students who want to enter occupational therapy courses.
17. **2019**: The ENEM is made mandatory for all students who want to enter speech therapy courses.
18. **2020**: The ENEM is made mandatory for all students who want to enter psychology courses.

### Timeline of the PROUNI Program:
1. **2004**: The PROUNI program is announced by the Brazilian government as part of the federal government's efforts to expand access to higher education.
2. **2005**: The PROUNI program begins accepting applications.
3. **2006**: The PROUNI program starts providing scholarships to low-income students.
4. **2007**: The PROUNI program expands its coverage to include more students.
5. **2008**: The PROUNI program continues to expand its coverage and introduces new categories of beneficiaries.
6. **2009**: The PROUNI program continues to expand its coverage and introduces new categories of beneficiaries.
7. **2010**: The PROUNI program continues to expand its coverage and introduces new categories of beneficiaries.
8. **2011**: The PROUNI program continues to expand its coverage and introduces new categories of beneficiaries.
9. **2012**: The PROUNI program continues to expand its coverage and introduces new categories of beneficiaries.
10. **2013**: The PROUNI program continues to expand its coverage and introduces new categories of beneficiaries.
11. **2014**: The PROUNI program continues to expand its coverage and introduces new categories of beneficiaries.
12. **2015**: The PROUNI program continues to expand its coverage and introduces new categories of beneficiaries.
13. **2016**: The PROUNI program continues to expand its coverage and introduces new categories of beneficiaries.
14. **2017**: The PROUNI program continues to expand its coverage and introduces new categories of beneficiaries.
15. **2018**: The PROUNI program continues to expand its coverage and introduces new categories of beneficiaries.
16. **2019**: The PROUNI program continues to expand its coverage and introduces new categories of beneficiaries.
17. **2020**: The PROUNI program continues to expand its coverage and introduces new categories of beneficiaries.

These timelines highlight the gradual expansion of both the ENEM and the PROUNI program over the years, reflecting the government's commitment to increasing access to higher education in Brazil.