The corner growth process is a mathematical model that describes how a system evolves over time by adding new elements at its corners. It is often used to study the dynamics of growing interfaces and patterns in various physical systems.

The Totally Asymmetric Simple Exclusion Process (TASEP) is a specific type of stochastic model that can be interpreted as a special case of the corner growth process. In TASEP, particles move along a one-dimensional lattice in a single direction, with each particle moving to the right (or left) at a rate determined by the state of neighboring sites. Particles cannot occupy the same site simultaneously, which introduces the "exclusion" aspect of the process.

In the context of the corner growth process, the arrows you described could represent the movement of particles in the TASEP model. Each arrow's length and orientation would correspond to the probability or rate of a particle moving in a particular direction at a given time step. The overlapping nature of the arrows suggests that multiple particles may be involved in the process simultaneously, leading to complex interactions and evolving patterns.

The visual representation of the arrows provides a way to visualize the dynamics of the system and understand how the particles move and interact over time. By analyzing the patterns formed by the arrows, researchers can gain insights into the behavior of the TASEP model and other related processes.