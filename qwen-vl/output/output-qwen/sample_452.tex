It seems there might be some confusion here. The description you provided does not match the image you mentioned. You described a right triangle inscribed in a circle, but the image you referred to is an equilateral triangle inscribed in a circle.

If we consider the scenario where an equilateral triangle is inscribed in a circle (which is a common geometric configuration), we can explore some properties:

1. **Properties of the Equilateral Triangle Inscribed in a Circle:**
   - All three sides of the triangle are equal.
   - Each angle of the triangle is 60 degrees.
   - The center of the circle is also the centroid, circumcenter, and orthocenter of the triangle.

2. **Relationship Between the Side Length of the Triangle and the Radius of the Circle:**
   - If \( s \) is the side length of the equilateral triangle and \( r \) is the radius of the circumscribed circle, then:
     \[
     r = \frac{s}{\sqrt{3}}
     \]
   - Conversely, if \( r \) is known, the side length \( s \) can be calculated as:
     \[
     s = r \sqrt{3}
     \]

3. **Area of the Equilateral Triangle:**
   - The area \( A \) of an equilateral triangle with side length \( s \) is given by:
     \[
     A = \frac{\sqrt{3}}{4} s^2
     \]

4. **Perimeter of the Equilateral Triangle:**
   - The perimeter \( P \) of the triangle is simply three times the side length:
     \[
     P = 3s
     \]

If you have specific values or further details about the problem, please provide them so I can assist you more accurately.