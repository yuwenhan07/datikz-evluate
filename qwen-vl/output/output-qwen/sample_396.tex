It seems like you're referring to a specific part of a mathematical or theoretical physics problem involving a triangular array and its summation. Let's break down the context and the details provided:

### Context:
The equation `eq:triangular-array-sum-color` likely refers to a summation over a triangular array where each element \(a_{i,j}\) is indexed by two indices \(i\) and \(j\). This type of summation often appears in combinatorial mathematics, statistical mechanics, or other areas where triangular arrays are used to represent certain types of data or configurations.

### Specific Case: \(m = 4\)
For \(m = 4\), we are dealing with a triangular array that has 4 rows. The total number of elements in such an array is given by the sum of the first 4 natural numbers, which is \(\frac{4(4+1)}{2} = 10\).

### Calculation of \(a_6\):
The notation \(a_6\) suggests that we are looking at the 6th element in some ordered sequence derived from the triangular array. Given the structure of the triangular array, the 6th element could be located in the third row (since the first three rows contain 1, 3, and 6 elements respectively).

### Summation Terms:
The summation in question, `eq:triangular-array-sum-color`, involves summing over different terms based on the value of \(k\). For \(m = 4\), there will be 5 terms corresponding to \(k = 1, 2, 3, 4, 5\).

### Diagram Explanation:
The diagram shows the relationship between the indices \(i\) and \(j\) and the canonical ordering \((q, p)\). Specifically, it illustrates the \(k = 2\) term, which means we are considering the contribution of the second row of the triangular array to the calculation of \(a_6\).

- **Indices \(i\) and \(j\)**: These indices typically range over the rows and columns of the triangular array.
- **Canonical Ordering**: This is a specific way of ordering the elements of the triangular array, often used to simplify summations or to ensure a consistent order when performing operations like summation or integration.

### Example:
If we were to write out the summation explicitly for \(k = 2\), it might look something like this:
\[ a_6 = \sum_{i=1}^{3} \sum_{j=1}^{i} a_{i,j} \]
This is because the 6th element is in the third row, and we need to sum over all elements up to and including the third row.

### Conclusion:
The diagram and the text suggest that the \(k = 2\) term is being highlighted as part of a larger summation process. The exact form of the summation would depend on the specific rules governing the triangular array and the definition of \(a_{i,j}\). If you have more details about the specific rules or the nature of the triangular array, I can provide a more detailed explanation or derivation.