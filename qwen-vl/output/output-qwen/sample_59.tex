To understand the 3-edge coloring of \(G_n\) for even integers \(n \geq 6\), we need to consider two cases: when \(n/2\) is even and when \(n/2\) is odd.

### Case 1: When \(n/2\) is even

Let's denote \(n = 2k\) where \(k\) is an integer. Here, \(k\) is also even because \(n/2\) is even. We can construct a graph \(G_n\) as follows:

- Consider a complete bipartite graph \(K_{k,k}\). This graph has two sets of vertices, each containing \(k\) vertices, and every vertex in one set is connected to every vertex in the other set.
- The edges of this graph can be colored with three colors such that no two adjacent edges share the same color. This is possible because the graph is bipartite and the number of edges between the two sets is \(k^2\), which is divisible by 3 when \(k\) is even.

### Case 2: When \(n/2\) is odd

Let's denote \(n = 2k+1\) where \(k\) is an integer. Here, \(k\) is odd because \(n/2\) is odd. We can construct a graph \(G_n\) as follows:

- Consider a complete bipartite graph \(K_{k+1,k}\). This graph has two sets of vertices, one containing \(k+1\) vertices and the other containing \(k\) vertices, and every vertex in one set is connected to every vertex in the other set.
- The edges of this graph can be colored with three colors such that no two adjacent edges share the same color. This is possible because the graph is bipartite and the number of edges between the two sets is \((k+1)k\), which is divisible by 3 when \(k\) is odd.

In both cases, the graph \(G_n\) can be 3-edge colored. The key point is that the number of edges in each case is divisible by 3, allowing us to use three colors without any two adjacent edges sharing the same color.

Thus, the answer to the problem is:

\[
\boxed{\text{The graph } G_n \text{ can be 3-edge colored for both cases.}}
\]