It seems like you're describing a transformation from an initial graph \( E_1 \) to a new graph \( E_2 \). Let's break down the process step by step:

1. **Initial Graph \( E_1 \)**:
   - \( E_1 \) is a directed weighted graph.
   - There is exactly one edge in \( E_1 \).
   - The weight of this single edge is 1.

2. **Transformation to \( E_2 \)**:
   - In \( E_2 \), the single edge from \( E_1 \) is duplicated.
   - Both edges now have the same weight, which remains 1.

So, the transformation can be summarized as follows:
- Start with \( E_1 \): One edge with weight 1.
- End up with \( E_2 \): Two edges (duplicates of the original edge) each with weight 1.

This transformation does not change the overall structure of the graph significantly in terms of connectivity or the total weight of the edges, but it introduces redundancy in the edge weights. If we were to represent this mathematically, let's denote the vertices involved in the edge as \( u \) and \( v \).

For \( E_1 \):
\[ E_1 = \{(u, v)\} \text{ with weight } 1 \]

For \( E_2 \):
\[ E_2 = \{(u, v), (u, v)\} \text{ with weight } 1 \text{ for both edges} \]

In summary, the transformation from \( E_1 \) to \( E_2 \) involves duplicating the single edge present in \( E_1 \) while keeping the weight of each edge at 1. This results in \( E_2 \) having two identical edges between the same pair of vertices, each with a weight of 1.