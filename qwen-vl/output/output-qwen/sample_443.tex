An ear decomposition (ED) of a graph is a sequence of paths \( P_1, P_2, P_3, \ldots, P_k \) such that:

1. Each path \( P_i \) is simple (i.e., it does not contain any self-loops or repeated vertices).
2. The union of all these paths forms the graph.
3. For each \( i > 1 \), the path \( P_i \) has exactly one endpoint in common with the previous path \( P_{i-1} \).

In your case, you have an exemplary ear decomposition \( ED = \{P_1, P_2, P_3, P_4\} \) of a graph, where:
- \( P_1 \) is a cycle with 10 edges.
- \( P_2 \) is a path with 4 edges.
- \( P_3 \) is a path with 3 edges.
- \( P_4 \) is a path with 2 edges.

Let's break down how this ear decomposition works:

### Step-by-Step Breakdown

1. **Cycle \( P_1 \)**: This is the first path in the sequence. It is a cycle with 10 edges. A cycle is a closed path where the starting vertex is also the ending vertex, and no edge is repeated.

2. **Path \( P_2 \)**: This path connects to one of the vertices in \( P_1 \). Since \( P_2 \) has 4 edges, it means there are 5 vertices in total in \( P_2 \). One end of \( P_2 \) must be connected to one of the vertices in \( P_1 \), and the other end of \( P_2 \) will be a new vertex not in \( P_1 \).

3. **Path \( P_3 \)**: This path connects to one of the vertices in \( P_2 \). Since \( P_3 \) has 3 edges, it means there are 4 vertices in total in \( P_3 \). One end of \( P_3 \) must be connected to one of the vertices in \( P_2 \), and the other end of \( P_3 \) will be a new vertex not in \( P_2 \).

4. **Path \( P_4 \)**: This path connects to one of the vertices in \( P_3 \). Since \( P_4 \) has 2 edges, it means there are 3 vertices in total in \( P_4 \). One end of \( P_4 \) must be connected to one of the vertices in \( P_3 \), and the other end of \( P_4 \) will be a new vertex not in \( P_3 \).

### Visual Representation

To visualize this, imagine the following structure:

- Start with a cycle \( P_1 \) (10 edges).
- Add a path \( P_2 \) (4 edges) that connects to one vertex in \( P_1 \).
- Add a path \( P_3 \) (3 edges) that connects to one vertex in \( P_2 \).
- Add a path \( P_4 \) (2 edges) that connects to one vertex in \( P_3 \).

The resulting graph will be a combination of these paths, ensuring that each path \( P_i \) for \( i > 1 \) shares exactly one vertex with the previous path \( P_{i-1} \).

This ear decomposition ensures that the graph can be constructed by successively adding these paths, maintaining the properties of an ear decomposition.