The image you've described seems to illustrate a sequence of geometric shapes that could be related to the concept of Pachner graphs in topology. Pachner graphs are used to describe how different triangulations of a manifold can be transformed into one another through a series of elementary moves.

In this context, let's break down the sequence:

1. **Hexagon**: This is likely the starting point.
2. **Square**: This could represent a transformation where the hexagon is replaced by a square.
3. **Triangle**: This might indicate a further transformation where the square is replaced by a triangle.
4. **Combination of Hexagon and Square**: This could be a more complex structure formed by combining the hexagon and the square.
5. **Hexagon Inside a Square**: This suggests a nested structure where the hexagon is inside the square.
6. **Hexagon Inside a Triangle**: Here, the hexagon is placed inside the triangle.
7. **Hexagon Inside a Square**: This is repeated, reinforcing the nested structure within the square.

### Pachner Graph Moves:
Pachner graph moves are specific types of topological transformations that allow for the conversion between different triangulations of a manifold. There are two main types of Pachner moves:

1. **Type I (1-4)**: This move involves replacing four tetrahedra with one tetrahedron. It is often used to simplify a triangulation by reducing the number of tetrahedra.
2. **Type II (4-1)**: This move involves replacing one tetrahedron with four tetrahedra. It is used to increase the complexity of a triangulation.

In your sequence, it appears that the shapes are being transformed in a way that could be analogous to these Pachner moves. For example, the transition from a hexagon to a square might represent a Type I move, while the transition from a square to a triangle might represent a Type II move.

However, without explicit information on the exact nature of the transformations (e.g., whether they involve adding or removing vertices, edges, or faces), it's challenging to definitively identify which Pachner moves are being applied. The visual appeal and the progression suggest a deliberate and systematic transformation, but the precise nature of the moves would require a more detailed analysis or a description of the rules governing the sequence.

If you have a specific set of rules or a mathematical framework that governs the sequence, please provide that information so we can better understand the relationship to Pachner graph moves.