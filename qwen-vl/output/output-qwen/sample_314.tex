The dual graph is a concept in graph theory that represents the relationship between vertices and edges in a graph. In this case, the dual graph for periodic maps can be represented by a tree-like structure with circles as nodes.

In the given image, there are six circles representing mathematical symbols, which could represent different elements or vertices in the dual graph. These circles are positioned at various heights and locations on the tree, indicating the connections between them. The white background helps to emphasize the circles and their placement on the tree.

The dual graph for periodic maps is often used in the study of dynamical systems, where it helps to understand the behavior of the system over time. It can also be used to analyze the properties of the system, such as its stability and periodicity. The dual graph provides a visual representation of the relationships between the different elements in the system, making it easier to identify patterns and trends.