It seems like you're describing a transformation or mapping \( G' \) from a graph \( G \) where \( c = 3 \). Let's break down the details:

1. **Graph \( G \)**:
   - You mentioned that there are two different depictions of a geometric shape.
   - The first depiction is a triangle with a red dot in the middle.
   - The second depiction is a red triangle with dots in the middle.

2. **Transformation \( G' \)**:
   - The transformation \( G' \) is applied to \( G \).
   - The parameter \( c = 3 \) might indicate a specific type of transformation or a rule for how the graph is modified.

3. **Possible Interpretation**:
   - If we consider the transformation \( G' \), it could be a mapping that changes the structure of the graph while preserving some properties.
   - For instance, if \( G \) is a triangle with a red dot in the middle, and \( G' \) is another triangle with dots in the middle, the transformation might involve moving the dot around the triangle or changing its position relative to the vertices.

4. **Mathematical Connection**:
   - The dots within the triangles are labeled with numbers, which suggests a possible mathematical relationship between the two shapes.
   - This could be related to the coordinates of the dots, their distances from each other, or any other geometric property.

### Example Scenario:
Let's assume the original triangle \( G \) has vertices at points \( A, B, \) and \( C \), and the red dot is at the centroid (the average of the vertices' coordinates).

- **Triangle \( G \)**:
  - Vertices: \( A, B, C \)
  - Red dot at the centroid: \( D \)

- **Transformation \( G' \)**:
  - Suppose the transformation moves the red dot to a new position, say \( E \), which is also the centroid but in a different location within the triangle.

- **New Triangle \( G' \)**:
  - Vertices: \( A, B, C \)
  - Red dot at the new centroid: \( E \)

The transformation \( G' \) could be a simple translation, rotation, or scaling of the original triangle \( G \), keeping the centroid as a key point.

### Conclusion:
Without more specific details about the transformation rules or the exact nature of the dots and their labels, it's challenging to provide a precise interpretation. However, the general idea is that \( G' \) is a transformed version of \( G \) with \( c = 3 \) possibly indicating a specific type of transformation or rule. The dots and their labels suggest a deeper mathematical connection, such as coordinate relationships or geometric properties.