The term "unitarity" in the context of physics often refers to a property of quantum systems where the total probability of all possible outcomes must sum up to one. In the context of gauge theories, such as Yang-Mills theory, unitarity ensures that the scattering amplitudes are well-defined and do not produce probabilities greater than one.

However, without more specific details about the mathematical equations and symbols surrounding the white square, it's difficult to provide a precise interpretation or analysis. If you could provide more information or clarify what these equations and symbols represent, I would be happy to help further. For instance, if they are related to a specific gauge theory or a particular aspect of quantum field theory, additional context can guide the analysis towards a more accurate understanding.