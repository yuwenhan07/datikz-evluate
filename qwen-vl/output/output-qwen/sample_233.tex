To determine the supports of the polynomials \( f(X) \), \( f(X^2) \), and \( f(X^3) \), we need to understand what the support of a polynomial is. The support of a polynomial \( f(X) \) is the set of all monomials that appear with non-zero coefficients in the polynomial.

First, let's write down the polynomial \( f(X) \):
\[ f(X) = 1 + x^{-1}y^{-1} + x^{-1}y + xy^{-1} + xy. \]

The support of \( f(X) \) is:
\[ \{1, x^{-1}y^{-1}, x^{-1}y, xy^{-1}, xy\}. \]

Next, we consider the dilation \( f(X^2) \). This means replacing each variable \( X \) in \( f(X) \) with \( X^2 \):
\[ f(X^2) = 1 + (x^2)^{-1}(y^2)^{-1} + (x^2)^{-1}(y^2) + (x^2)(y^2)^{-1} + (x^2)(y^2). \]
Simplifying the exponents, we get:
\[ f(X^2) = 1 + x^{-2}y^{-2} + x^{-2}y^2 + x^2y^{-2} + x^2y^2. \]
The support of \( f(X^2) \) is:
\[ \{1, x^{-2}y^{-2}, x^{-2}y^2, x^2y^{-2}, x^2y^2\}. \]

Finally, we consider the dilation \( f(X^3) \). This means replacing each variable \( X \) in \( f(X) \) with \( X^3 \):
\[ f(X^3) = 1 + (x^3)^{-1}(y^3)^{-1} + (x^3)^{-1}(y^3) + (x^3)(y^3)^{-1} + (x^3)(y^3). \]
Simplifying the exponents, we get:
\[ f(X^3) = 1 + x^{-3}y^{-3} + x^{-3}y^3 + x^3y^{-3} + x^3y^3. \]
The support of \( f(X^3) \) is:
\[ \{1, x^{-3}y^{-3}, x^{-3}y^3, x^3y^{-3}, x^3y^3\}. \]

Thus, the supports of the polynomials are:
\[ \boxed{\{1, x^{-1}y^{-1}, x^{-1}y, xy^{-1}, xy\}, \{1, x^{-2}y^{-2}, x^{-2}y^2, x^2y^{-2}, x^2y^2\}, \{1, x^{-3}y^{-3}, x^{-3}y^3, x^3y^{-3}, x^3y^3\}}. \]