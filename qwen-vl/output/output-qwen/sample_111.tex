To solve the problem, we need to understand what it means for a set to be cyclically ordered and how cutting points divide this set into two intervals.

1. **Cyclically Ordered Set**: A set is said to be cyclically ordered if there is a circular arrangement of its elements such that each element is related to the next in a specific order. For example, consider the set \(\{a, b, c\}\). We can arrange these elements in a circle as \(a \rightarrow b \rightarrow c \rightarrow a\), where the arrow indicates the cyclic order.

2. **Cutting Points**: Cutting points are points in the cyclic arrangement that divide the set into two intervals. These intervals are also cyclically ordered. For instance, if we have the set \(\{a, b, c, d\}\) arranged as \(a \rightarrow b \rightarrow c \rightarrow d \rightarrow a\), and we choose the cutting point at \(b\), then the set is divided into two intervals: \(\{a, b\}\) and \(\{c, d\}\).

Let's illustrate with an example:

Consider the set \(\{a, b, c, d, e\}\) arranged cyclically as \(a \rightarrow b \rightarrow c \rightarrow d \rightarrow e \rightarrow a\).

If we choose the cutting point at \(c\), the set is divided into two intervals:
- The first interval is \(\{a, b, c\}\), which is cyclically ordered as \(a \rightarrow b \rightarrow c \rightarrow a\).
- The second interval is \(\{d, e\}\), which is cyclically ordered as \(d \rightarrow e \rightarrow d\).

If we choose the cutting point at \(e\), the set is divided into two intervals:
- The first interval is \(\{a, b, c, d, e\}\), which is the entire set.
- The second interval is empty.

If we choose the cutting point at \(d\), the set is divided into two intervals:
- The first interval is \(\{a, b, c, d\}\), which is cyclically ordered as \(a \rightarrow b \rightarrow c \rightarrow d \rightarrow a\).
- The second interval is \(\{e\}\), which is cyclically ordered as \(e \rightarrow e\).

Therefore, the answer to the problem is that a set which is cyclically ordered and two "cutting points" dividing the cyclically ordered set into two intervals can be illustrated with any cyclically ordered set and any choice of cutting points. The specific example provided above shows one possible division.

The final answer is:
\[
\boxed{\text{A set which is cyclically ordered and two "cutting points" dividing the cyclically ordered set into two intervals}}
\]