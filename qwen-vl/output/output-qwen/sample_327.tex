It seems like you're describing an image that involves a white cable with beads of different colors arranged in a linear sequence. However, your question also mentions an "edge-color-avoiding connected graph" which doesn't seem directly related to the image description.

If we focus on the image description:

1. **Graph Theory Context**: An edge-color-avoiding connected graph refers to a graph where no two edges share the same color, and the graph remains connected (i.e., there's a path between any two vertices). For a graph with 8 vertices, the minimum number of edges required to ensure connectivity is 7 (forming a tree).

2. **Coloring**: If we need to use exactly 4 colors for the edges while ensuring the graph remains edge-color-avoiding and connected, we can consider a specific type of graph known as a "rainbow tree." A rainbow tree is a tree where each edge has a unique color, and since we have 7 edges in a minimal spanning tree for 8 vertices, we would indeed need 7 distinct colors if we were to strictly adhere to the edge-color-avoiding condition.

However, given that you mentioned only 4 colors, it suggests that some edges might share the same color, but the graph still needs to be connected and avoid having any two edges of the same color.

Given these points, let's summarize:
- The graph must be a tree with 7 edges.
- Since we have only 4 colors, some edges will necessarily share the same color, but the graph will remain connected and edge-color-avoiding within those constraints.

The image you described seems unrelated to this mathematical context unless it's part of a metaphorical or artistic representation of such a concept. If you're looking for a mathematical solution, the key is to find a way to color the edges of a tree with 7 edges using only 4 colors while ensuring no two edges share the same color. This is not possible without violating the edge-color-avoiding condition, so the problem as stated may not have a valid solution under the given constraints.