python
import numpy as np
import matplotlib.pyplot as plt

# Parameters
K_max = 50
h = 5
num_instances = 100

# Function to generate random test instances
def generate_random_instances(num_instances):
    return np.random.rand(num_instances)

# Function to calculate average relative loss and mean absolute deviation
def calculate_loss_and_deviation(K, instances):
    # Placeholder for the calculation of relative loss and mean absolute deviation
    # Replace this with your actual calculation logic
    relative_loss = np.random.normal(0, 1, num_instances)
    mean_absolute_deviation = np.mean(np.abs(relative_loss))
    return np.mean(relative_loss), mean_absolute_deviation

# Main loop to run the experiment
results = []
for K in range(1, K_max + 1):
    instances = generate_random_instances(num_instances)
    avg_relative_loss, mean_abs_deviation = calculate_loss_and_deviation(K, instances)
    results.append((K, avg_relative_loss, mean_abs_deviation))

# Convert results to arrays for plotting
Ks, avg_losses, devs = zip(*results)

# Plotting
plt.figure(figsize=(10, 6))
plt.plot(Ks, avg_losses, marker='o', linestyle='-', label='Average Relative Loss')
plt.fill_between(Ks, avg_losses - devs, avg_losses + devs, alpha=0.2, label='Mean Absolute Deviation')
plt.xlabel('K')
plt.ylabel('Relative Loss')
plt.title('Average Relative Loss vs K with Mean Absolute Deviation')
plt.legend()
plt.grid(True)
plt.show()