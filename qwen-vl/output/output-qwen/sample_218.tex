It seems like you're describing a concept related to graph theory, specifically focusing on embeddings and partitions. Let's break down the key points:

### Embeddable Word and Embedding
An "embeddable word" typically refers to a sequence of elements that can be embedded into a larger structure, such as a tree or a graph. An "embedding" is the process of placing these elements into the structure while preserving certain properties.

### Example 1: Graph Representation
The example you provided seems to be describing a tree \( T_{(5,9,10)} \). Here’s a breakdown of the notation and concepts used:

1. **Edges and Labels**:
   - Edges are labeled with red.
   - Edges without a label are marked with a dot (\(\cdot\)).

2. **Binomial Numbers**:
   - These numbers correspond to the number of extensions given by their root-to-leaf path.
   - The \(k\) indicates the size of the extended edges.

3. **Path Root-to-Leaf**:
   - The path from the root to a leaf in \( T_{(5,9,10)} \) given by \((7,2), (7,1,3), (7,1,4)\):
     - This path represents extensions of the initial tuple \((5,9,10)\).
     - It includes the elements 2 and 3 but excludes 7, 1, and 4.
     - The labels on the leaf vertices represent the cardinalities of the partition sets.

4. **Partition Sets**:
   - Each set in \( B_i \) corresponds to a partition of the original set.
   - In this example, each set in \( B_i \) has two vertices related to it, one of which is a leaf vertex.
   - Generally, there might be more than two vertices per set in \( B_i \).

### Interpretation
The example illustrates how a tree structure can be used to represent partitions of a set. The tree nodes and edges encode information about the relationships between different subsets of the original set. The binomial numbers and paths help in counting and identifying specific configurations within the tree.

### Generalization
This type of representation is often used in combinatorial mathematics and computer science for problems involving partitions, such as in algorithms for generating all possible partitions of a set or for studying the structure of partitions.

If you need further clarification or have additional questions, feel free to ask!