The image you've described seems to be an artistic representation rather than a mathematical one. However, I can provide some insights into the concept of "convex points" for hypersurfaces and their relation to equidistant hypersurfaces.

In mathematics, particularly in differential geometry, a convex point on a hypersurface refers to a point where the hypersurface is locally convex. This means that at such a point, the hypersurface bulges outwards like a sphere or a convex polyhedron. For a hypersurface \( M \) embedded in a higher-dimensional space, a point \( p \in M \) is convex if there exists a neighborhood around \( p \) such that any two points in this neighborhood lie on the same side of the tangent hyperplane at \( p \).

An equidistant hypersurface is a surface that is at a constant distance from another given hypersurface. If we consider a hypersurface \( M \) and its equidistant hypersurface \( M' \), the existence of convex points on \( M \) depends on the specific properties of \( M \) and how it interacts with \( M' \).

For example, if \( M \) is a convex hypersurface (like a sphere or a convex polyhedron), then every point on \( M \) will be a convex point because the hypersurface bulges outwards uniformly. On the other hand, if \( M \) is concave, then not all points on \( M \) will be convex points.

If the equidistant hypersurface \( M' \) is also considered, the convexity of points on \( M \) might be influenced by the geometric relationship between \( M \) and \( M' \). For instance, if \( M \) is convex and \( M' \) is its outer equidistant hypersurface, then every point on \( M \) will still be a convex point because the bulging nature of \( M \) ensures that it remains convex relative to \( M' \).

However, if \( M \) is concave and \( M' \) is its inner equidistant hypersurface, then the points on \( M \) near the boundary of \( M \) may no longer be convex points because the bulging nature of \( M \) is now being counteracted by the inward bulge of \( M' \).

In summary, the existence of convex points on a hypersurface supported on an equidistant hypersurface depends on the specific geometric properties of both the original hypersurface and the equidistant hypersurface. If the original hypersurface is convex, then most points on it will remain convex even when considering an equidistant hypersurface. Conversely, if the original hypersurface is concave, the convexity of points may be affected by the equidistant hypersurface.