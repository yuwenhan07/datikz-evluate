The image you've described appears to be a visual representation of some kind of data distribution, likely related to English scores of Vietnamese students in 2019. The use of a blue and green color scheme with two distinct curves suggests that there are two separate datasets being compared or analyzed.

Here's a breakdown of what this might look like:

- **Blue Curve**: This could represent one dataset, perhaps the distribution of English scores for a specific group of students (e.g., those from a particular school, grade level, or demographic).
- **Green Curve**: This represents another dataset, potentially contrasting with the first one. It could show a different group of students, such as those who took an alternative assessment or those from a different educational setting.

### Key Features:
- **Spectrum**: The term "spectrum" in your question might refer to the range of scores on the x-axis, which could be the English test scores.
- **Color Coding**: Blue and green are used to differentiate between the two datasets, making it easier to compare their distributions visually.
- **White Background**: This provides a clean and neutral backdrop that helps the curves stand out clearly.

### Possible Interpretations:
1. **Performance Comparison**: The curves might show how two groups of students performed differently in terms of their English scores.
2. **Engagement Levels**: If the curves represent different aspects of student engagement (e.g., pre-test vs. post-test), they could indicate changes in performance over time.
3. **Demographic Analysis**: The curves might be broken down by age, gender, or other demographic factors to see if certain groups perform better than others.

Without more specific details about the axes or labels, it's challenging to provide a precise interpretation. However, the visual setup strongly suggests a comparative analysis of some form of student performance data.