It seems like you're describing a process related to graph theory, specifically local complementation. Let's break down the steps involved:

### Local Complementation:
Local complementation is an operation on a graph \( G \) that involves selecting a vertex \( u \) and then complementing the subgraph induced by the neighbors of \( u \). Here’s how it works:

1. **Identify the Vertex \( u \)**: You have chosen a vertex \( u \) in the graph \( G \).

2. **Identify the Neighbors of \( u \)**: The neighbors of \( u \), denoted as \( N(u) \), are all vertices \( v_i \) such that there is an edge between \( u \) and \( v_i \). In your case, \( v_1 \) and \( v_3 \) are neighbors of \( u \).

3. **Complement the Subgraph Induced by \( N(u) \)**: The subgraph induced by \( N(u) \) consists of the vertices \( v_1, v_2, v_3 \) and the edges among them. If there is an edge between two vertices \( v_i \) and \( v_j \) in this subgraph, then after complementation, there will be no edge between them; if there is no edge, then there will be one.

4. **Apply the Operation**:
   - Initially, the edges in the subgraph \( \{v_1, v_2, v_3\} \) are given as follows:
     - \( v_1v_3 \in E \)
     - \( v_1v_2 \notin E \)
     - \( v_2v_3 \notin E \)

   - After complementation, the edges will change as follows:
     - Since \( v_1v_3 \in E \), after complementation, \( v_1v_3 \notin E \).
     - Since \( v_1v_2 \notin E \), after complementation, \( v_1v_2 \in E \).
     - Since \( v_2v_3 \notin E \), after complementation, \( v_2v_3 \in E \).

So, the new edges in the subgraph \( \{v_1, v_2, v_3\} \) will be:
- \( v_1v_2 \in E \)
- \( v_2v_3 \in E \)
- \( v_1v_3 \notin E \)

### Summary:
- The original edges in the subgraph \( \{v_1, v_2, v_3\} \) were \( v_1v_3 \) (present), \( v_1v_2 \) (absent), and \( v_2v_3 \) (absent).
- After local complementation at \( u \), the edges become \( v_1v_2 \) (present), \( v_2v_3 \) (present), and \( v_1v_3 \) (absent).

This operation can be repeated for different vertices in the graph to study its properties or to transform the graph into another form.