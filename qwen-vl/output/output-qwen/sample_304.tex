The graph you described appears to be a visualization of the correlation between non-functional property keywords and their actual targeted fitness in 305 repository papers. Here's a breakdown of what this might look like:

1. **Graph Structure**: The graph likely has two axes:
   - The x-axis could represent the non-functional property keyword categories such as time, energy, quality, memory, etc.
   - The y-axis could represent the actual targeted fitness or effectiveness of these properties.

2. **Color Coding**: Different colors on the graph represent different categories of property optimization. For example:
   - Red dots might represent "time" optimization.
   - Blue dots might represent "energy" optimization.
   - Green dots might represent "quality" optimization.
   - Yellow dots might represent "memory" optimization.
   - Purple dots might represent "other" categories.

3. **Distribution of Dots**: The spread of the dots across the graph indicates the distribution of how often each category is targeted in the repository papers. The density and clustering of dots within certain areas may suggest which categories are more frequently addressed in the literature.

4. **Interpretation**:
   - If there is a clear cluster of dots in one area, it suggests that a particular category (e.g., time) is more commonly targeted in the papers.
   - The spread of dots across multiple areas might indicate a diverse range of non-functional properties being optimized in the studies.

5. **Trends and Insights**:
   - By analyzing the graph, researchers can identify which non-functional properties are most frequently optimized in the context of the repository papers.
   - This can help in understanding the current trends and priorities in software development and system design regarding non-functional properties.

6. **Further Analysis**:
   - To gain deeper insights, one could perform statistical analyses such as correlation coefficients to quantify the relationship between the categories and their targeted fitness.
   - Additionally, clustering algorithms could be applied to group similar categories together based on their performance metrics.

In summary, the graph provides a visual representation of the distribution and frequency of non-functional property categories in the repository papers, helping researchers understand the current focus and potential gaps in optimizing these properties.