It seems like you're referring to a visual issue rather than a mathematical one. However, if we're discussing function spaces in mathematics, particularly the L^2-space and the homogeneous Sobolev space H^0, here's some relevant information:

The L^2-space (or Lebesgue space) is a space of square-integrable functions. Formally, for a measurable set \( \Omega \), the L^2-space consists of all functions \( f: \Omega \to \mathbb{R} \) such that:
\[
\int_{\Omega} |f(x)|^2 \, dx < \infty.
\]

The homogeneous Sobolev space \( H^0(\Omega) \) is a subset of the L^2-space. It consists of functions whose weak derivatives up to a certain order are also in L^2. For simplicity, let's consider the case where \( H^0(\Omega) \) includes functions with zero boundary conditions or functions that vanish at infinity.

In many contexts, especially in the study of partial differential equations, the L^2-space and the homogeneous Sobolev space \( H^0 \) are closely related. For instance, the solution to certain elliptic PDEs can be shown to lie in \( H^0 \).

If you have a specific problem or question about these spaces, feel free to provide more details!