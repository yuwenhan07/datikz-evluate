To understand the possible structures for a critical component \( K \) of \( H + C \), let's break down the given information step by step:

1. **Graph Representation**:
   - \( H + C \) represents the graph obtained by adding the edges of a path-cycle cover \( C \) to a graph \( H \).
   - The edges in \( H \) are divided into three categories: thick edges in the matching \( M \), dashed edges also in \( M \), and thin edges not in \( M \).

2. **Vertices and Edges**:
   - Vertices can be classified as being in \( V(M) \) or not in \( V(M) \).
   - Gray vertices are 2-anchors.
   - Red vertices are critical.

3. **Structures of \( K \)**:
   - Each structure shows a specific configuration of vertices and edges within \( K \).
   - The fraction on the right side of each structure represents the ratio \( s(K) / \text{opt}(K) \), where \( s(K) \) is the size of some solution and \( \text{opt}(K) \) is the optimal size of a solution for \( K \).

4. **Possible Structures**:
   - The structures provided likely represent different configurations of \( K \) that could arise from the constraints described above. Each structure will have a specific number of vertices and edges, with certain vertices marked as red (critical), gray (2-anchors), and others as either in \( V(M) \) or not in \( V(M) \).

5. **Interpretation**:
   - For each structure, the fraction \( s(K) / \text{opt}(K) \) indicates how well the current structure performs compared to the optimal solution for \( K \).
   - The goal is to analyze these structures to understand the behavior of the critical component \( K \) under the given constraints.

Without the actual structures provided in the problem statement, we cannot list them here. However, the general approach would involve:

- Identifying the vertices and edges in each structure.
- Determining which vertices are red (critical), gray (2-anchors), and in \( V(M) \) or not in \( V(M) \).
- Calculating the size of the solution \( s(K) \) for each structure.
- Comparing \( s(K) \) to the optimal size \( \text{opt}(K) \) to find the fraction \( s(K) / \text{opt}(K) \).

If you provide the specific structures, I can help you analyze them further.