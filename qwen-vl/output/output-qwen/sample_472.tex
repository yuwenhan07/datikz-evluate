It seems like you're describing a complex system involving both graph theory and binary coding, possibly related to the representation of a fractal-like structure within a cube. Let's break down your description into more manageable parts:

### Graph Theory
- **Fully-Connected Layered Graph**: This refers to a type of graph where every node in one layer is connected to every node in the next layer.
- **h = 3 Intermediate Layers**: There are three layers between the input and output layers.
- **r = 3 Nodes at Each Layer**: Each layer has three nodes.

### Binary Code Representation
- **Black and White Picture of a Cube**: This could be interpreted as a 3D binary image where black pixels represent 1s and white pixels represent 0s.
- **Smaller Cubes Arranged in a Pattern of 1s and 0s**: These smaller cubes form a binary code that represents a fractal-like structure.

### Combining Concepts
Given these descriptions, it appears you might be dealing with a scenario where a 3D binary image (a cube) is represented using a graph model. Here’s how we can conceptualize this:

1. **Graph Representation**:
   - **Input Layer**: Represents the initial state or the starting point of the binary code.
   - **Intermediate Layers**: Represent the transformations or states of the binary code as it progresses through the layers.
   - **Output Layer**: Represents the final state or the result of the transformation.

2. **Binary Code**:
   - Each node in the graph can be associated with a specific binary value (0 or 1).
   - The arrangement of these values in the graph can represent a fractal-like structure, which is often used in computer graphics and data compression algorithms.

### Example Scenario
Imagine you have a 3x3x3 cube (27 smaller cubes), and each smaller cube is either black (1) or white (0). You can represent this cube as a 3D array of binary values. Now, if you want to process this cube using a graph model, you can create a graph where:
- Each node in the graph corresponds to a smaller cube in the 3D cube.
- Edges connect nodes based on some rule (e.g., connecting adjacent cubes).

The fully-connected layered graph would then process this binary cube by transforming it through the intermediate layers according to certain rules or algorithms.

### Conclusion
Your description suggests a sophisticated system combining graph theory and binary coding to represent and process a fractal-like structure within a 3D binary cube. The exact nature of the transformations and the rules governing the connections between nodes would depend on the specific application or algorithm being used. If you need further clarification or assistance with a specific aspect of this system, feel free to ask!