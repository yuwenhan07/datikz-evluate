The problem you're referring to involves understanding the mean squared linking number of two disjoint \( n \)-cycles in a random book embedding of \( K_{2n} \), where \( K_{2n} \) is the complete graph on \( 2n \) vertices.

Firstly, let's clarify some concepts:
- A book embedding of a graph is an embedding of the graph into a book (a collection of parallel planes or pages) such that all edges lie on one side of each page.
- In a book embedding, cycles can be linked together if they cross each other in a way that cannot be separated without crossing another edge.
- The linking number measures how many times one cycle is linked around another cycle.

For a random book embedding of \( K_{2n} \), the mean squared linking number of two disjoint \( n \)-cycles can be analyzed using probabilistic methods and combinatorial arguments. However, the specific details of this calculation are quite complex and typically involve advanced techniques from graph theory and probability.

Given the geometric progression graph you mentioned, it seems like you might be referring to a different context or a simplified version of the problem. If we consider the geometric progression graph as a representation of some sequence or data points, it doesn't directly relate to the mean squared linking number in the context of graph embeddings.

If you have more specific details about the problem or if there's a particular aspect you'd like to explore further, please provide additional information so I can assist you better.