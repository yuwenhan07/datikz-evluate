The Wadge hierarchy \( W_X \) is a fundamental concept in descriptive set theory, particularly in the study of Borel and analytic sets. The hierarchy classifies subsets of a topological space \( X \) based on their complexity, with \( W_X \) being the collection of all Borel subsets of \( X \).

When considering the Wadge hierarchy \( W_X \) on an arbitrary zero-dimensional Polish space \( X \), we often encounter various levels of the hierarchy that correspond to different ordinals. These ordinals represent the complexity of the sets in the hierarchy.

In the context of your question, you mention that \( W_X \) is isomorphic neither to \( W_N \) (the Wadge hierarchy on the Cantor space \( N \)) nor to \( W_C \) (the Wadge hierarchy on the real line \( C \)). This implies that the structure of \( W_X \) is distinct from these well-known hierarchies.

The ordinal \( \omega_1 \) represents the first uncountable ordinal, which is the smallest ordinal greater than any countable ordinal. If \( W_X \) is not isomorphic to either \( W_N \) or \( W_C \), it suggests that the complexity of the sets in \( W_X \) is intermediate between those in \( W_N \) and \( W_C \). Specifically, the ordinal \( \omega_1 \) is the smallest ordinal that cannot be realized as the complexity of a Borel subset of \( X \) if \( X \) is not homeomorphic to the Cantor space or the real line.

Thus, the ordinal \( \omega_1 \) is the smallest ordinal that \( W_X \) can be isomorphic to under the given conditions. Therefore, we have:

\[
\omega_1 \leq \omega_X < \omega_2
\]

where \( \omega_X \) is the least ordinal such that \( W_X \) is isomorphic to \( W_{\omega_X} \). Since \( \omega_1 \) is the smallest ordinal greater than all countable ordinals, and \( \omega_2 \) is the next ordinal after \( \omega_1 \), the inequality \( \omega_1 \leq \omega_X < \omega_2 \) holds true.

In summary, the ordinal \( \omega_X \) satisfies:

\[
\omega_1 \leq \omega_X < \omega_2
\]