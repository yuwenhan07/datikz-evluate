The image you described does not provide enough information to determine if there is an insecure process involved. However, I can provide an example of an insecure process in a general context.

Insecure processes can occur in various fields such as software development, cybersecurity, and data management. Here's an example:

### Example: Insecure Software Development Process

Imagine a software development team that follows a traditional waterfall model for developing applications. This model involves a linear sequence of stages from requirements gathering to deployment. However, this process may be insecure due to several reasons:

1. **Lack of Security Testing**: The team might not include security testing as part of their development cycle. This means that vulnerabilities could be introduced into the application without being detected until it is too late.
   
2. **Insufficient Code Review**: If code reviews are not thorough or performed by developers who lack expertise in security, critical flaws may slip through unnoticed.

3. **No Regular Updates**: The team might not regularly update the application to patch known vulnerabilities. This leaves the application exposed to attacks that exploit known weaknesses.

4. **Inadequate Documentation**: If the documentation is incomplete or outdated, it can lead to miscommunication among team members, which might result in insecure practices being implemented unintentionally.

5. **Lack of User Education**: The team might not educate users about how to use the application securely. For instance, they might not inform users about the importance of using strong passwords or updating the application regularly.

6. **Insecure Configuration Management**: If the configuration settings of the application are not properly secured, attackers might be able to manipulate them to gain unauthorized access.

### Mitigation Strategies:
To mitigate these risks, the team should adopt more secure practices such as:

- **Continuous Integration/Continuous Deployment (CI/CD)**: Automating the testing and deployment processes can help catch issues early.
- **Security Testing**: Incorporating security testing at every stage of the development lifecycle.
- **Thorough Code Reviews**: Conducting code reviews by security experts.
- **Regular Updates**: Keeping the application up-to-date with the latest security patches.
- **Detailed Documentation**: Maintaining comprehensive and up-to-date documentation.
- **User Training**: Educating users on best security practices.
- **Secure Configuration Management**: Ensuring that configurations are secure and cannot be easily manipulated.

By implementing these strategies, the software development process can become more secure and resilient against potential threats.