The image you've described seems to depict a Hasse diagram, which is a type of graph used to represent the elements of a partially ordered set (poset) and their relationships. In a Hasse diagram, each element of the poset is represented by a point, and an upward arrow between two points indicates that one element is greater than the other according to the partial order.

Here's a breakdown of what you might be seeing:

1. **Hasse Diagrams**: These are graphical representations of lattices, which are special types of posets where every pair of elements has a unique least upper bound (join) and greatest lower bound (meet). The Hasse diagram simplifies the structure by removing the transitive relations, making it easier to visualize the structure of the lattice.

2. **Lattice P**: This could refer to a lattice \( P \), which is a poset where every pair of elements has a unique join and meet. The Hasse diagram for \( P \) would show the elements of \( P \) and the relationships between them.

3. **Interval Lattice Int\,P**: This refers to the interval lattice of \( P \), which consists of all intervals within \( P \). An interval in a poset is a subset of elements that are pairwise comparable. The interval lattice of \( P \) is a sublattice of \( P \) formed by these intervals.

4. **bP and Int bP**: Here, "b" likely stands for "bounded," so \( bP \) could denote the bounded part of \( P \), meaning the elements of \( P \) that have both a least upper bound and a greatest lower bound. The interval lattice of the bounded part of \( P \) (Int bP) would then be the interval lattice of this bounded part.

Given your description of a "black and white drawing of a tree with various branches and leaves," it sounds like you're looking at a simplified or stylized version of a Hasse diagram. Trees in this context often represent hierarchical structures, but in the context of lattices, they can also represent the structure of the lattice itself.

If you have more specific details about the diagram, such as the number of elements, the relationships between them, or any particular patterns, it might help in identifying the exact lattice or interval lattice being depicted. If you need further clarification or analysis, feel free to provide more details!