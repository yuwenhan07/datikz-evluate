The illustration you've described seems to be related to a mathematical or computational context, possibly involving graph theory or combinatorics. Let's break down the elements:

1. **Tree Structure**: The central element appears to be a tree, which is a type of graph where there are no cycles and each node (or vertex) has at most one parent. Trees are often used in various algorithms and data structures.

2. **Labels**: The blue and orange labels on the nodes could represent different values or variables within the tree. In some contexts, these might denote specific properties or states of the nodes, such as whether they are part of a certain set or have been processed in an algorithm.

3. **Equations**: The presence of equations around the tree suggests that the tree might be part of a larger mathematical problem or proof. These equations could be defining relationships between the nodes, or they might be part of the conditions under which the tree is constructed or analyzed.

4. **Thm:mainshellable**: This likely refers to a theorem named "mainshellable." In mathematics, especially in areas like combinatorial geometry or discrete mathematics, theorems often describe properties or characteristics of certain structures. The term "shellable" typically refers to a property of simplicial complexes or posets (partially ordered sets), where the complex can be decomposed into simpler pieces in a specific order.

Given this information, it seems the illustration is part of a proof or explanation for a theorem about shellable structures, possibly involving trees or graphs. The labels and equations provide additional context and constraints necessary to understand the theorem fully.

If you need further clarification or assistance with understanding the specific details of the theorem or the illustration, feel free to ask!