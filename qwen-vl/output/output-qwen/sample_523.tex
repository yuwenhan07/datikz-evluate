The blue curvy line you're describing sounds like it could be a helicoid, which is a surface in three-dimensional space that is generated by moving a straight line along a helical path. A helicoid is a type of minimal surface, meaning it has the least possible area for its boundary.

In mathematical terms, a helicoid can be described as the set of points \((x, y, z)\) in \(\mathbb{R}^3\) that satisfy the parametric equations:
\[ x = u \cos(v), \]
\[ y = u \sin(v), \]
\[ z = v, \]
where \(u\) and \(v\) are parameters.

This surface is characterized by its continuous bending and twisting, much like the shape you've described. It's often used in various fields such as architecture, engineering, and even in the study of soap films and other physical phenomena where minimal surfaces play a role.

If you have any specific questions about helicoids or need further clarification on their properties, feel free to ask!