It sounds like you're describing a Cayley tree with \( n = 45 \). A Cayley tree (or Cayley graph) is a type of tree structure where each vertex has the same number of edges, which is called the degree of the tree. In this case, since it's described as a circle of trees, we can assume that the degree is 2, meaning each tree is connected to two other trees.

Here’s a step-by-step breakdown:

1. **Understanding the Structure**: A Cayley tree with \( n = 45 \) means there are 45 vertices (trees), and each vertex is connected to exactly 2 other vertices. This forms a cycle or a ring-like structure where each tree is connected to its immediate neighbors.

2. **Visual Representation**: If you visualize this, you would see a circular arrangement of 45 trees, where each tree is connected to the two trees directly adjacent to it. This creates a continuous loop, resembling a chain of connected trees.

3. **Mathematical Properties**:
   - The total number of edges in such a structure is \( 45 \) because each edge connects two vertices.
   - The degree of each vertex is 2, meaning each tree is connected to exactly two other trees.

4. **Complexity**: The complexity of the network increases as the number of vertices grows. With 45 trees, the network becomes quite intricate, with many interconnected branches and lines forming a visually striking pattern.

If you have any specific questions about the properties of this Cayley tree or need further clarification on how to visualize or analyze it, feel free to ask!