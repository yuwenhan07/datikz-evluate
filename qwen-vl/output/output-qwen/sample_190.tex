It seems like you're referring to a scenario where a square (let's call it \( K \)) is placed inside a circle (the white circular boundary) and the square does not satisfy some specific property \({}^*\). However, without knowing what the property \({}^*\) is, I can provide an example based on common geometric properties.

One common property that might be relevant here could be related to the area or the relationship between the shapes' dimensions. For instance, if the property \({}^*\) is "the square's area is less than half the area of the circle," then we can check if this condition holds for our square \( K \).

Let's assume:
- The side length of the square \( K \) is \( s \).
- The radius of the circle is \( r \).

The area of the square \( K \) is given by:
\[ \text{Area of } K = s^2 \]

The area of the circle is given by:
\[ \text{Area of the circle} = \pi r^2 \]

For the property \({}^*\) to hold, we need:
\[ s^2 < \frac{1}{2} \pi r^2 \]

If the square \( K \) is placed such that it fits entirely within the circle but does not meet this inequality, then \( K \) would not satisfy the property \({}^*\).

For example, if the radius \( r \) of the circle is 5 units, then the area of the circle is:
\[ \text{Area of the circle} = \pi \times 5^2 = 25\pi \approx 78.54 \]

If the side length \( s \) of the square is 6 units, then the area of the square is:
\[ \text{Area of } K = 6^2 = 36 \]

Since \( 36 < \frac{1}{2} \times 78.54 \approx 39.27 \), the square with side length 6 units does not satisfy the property \({}^*\).

Therefore, an example of a square \( K \) that does not satisfy the property \({}^*\) is a square with a side length greater than approximately 6.24 units when the radius of the circle is 5 units. This ensures that the area of the square is more than half the area of the circle.