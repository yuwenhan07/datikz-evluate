To understand the context provided in your question, we need to delve into some concepts from knot theory and diagrammatic calculus, particularly focusing on the study of knot signatures and their changes under specific operations.

### Key Concepts

1. **Knot Signature**: The signature of a knot is an invariant that can be computed from its Seifert matrix or from a diagram of the knot. It is denoted by \(\sigma\).

2. **Alternating Diagrams**: An alternating diagram of a knot is one where the crossings alternate between over and under as you follow the knot in a consistent direction. Alternating knots have a special property related to their signature.

3. **Final Runs**: In the context of alternating diagrams, a "final run" refers to a sequence of consecutive crossings that do not change the sign of the crossing (i.e., they are either all overcrossings or all undercrossings).

4. **Partial \(A\)-Resolutions**: These are operations on a knot diagram that involve resolving certain crossings while keeping others unresolved. The notation \(A\) typically refers to a specific type of resolution, often used in the context of alternating diagrams.

5. **Positive Crossings**: Positive crossings are those where the overstrand passes over the understrand in a counterclockwise direction when viewed from above.

6. **Lemma \ref{lem:signaturerecursion}**: This lemma likely describes how the signature of a knot changes under certain operations, such as replacing final runs with other types of crossings.

### Interpretation of the Question

The question asks about the final steps in a process involving alternating diagrams and partial \(A\)-resolutions, specifically focusing on the effect on the knot's signature. Here’s a breakdown:

1. **Final Runs**: These are sequences of consecutive crossings that are either all overcrossings or all undercrossings.
2. **Alternating Diagrams**: The diagram is assumed to be alternating, which simplifies the analysis of the signature.
3. **Positive Crossings**: These are shaded in the diagram, indicating their importance in the calculation of the signature.
4. **Partial \(A\)-Resolutions**: These are operations that modify the diagram but keep it alternating.
5. **Signature Change (\(\Delta \sigma\))**: This represents the difference in the knot's signature before and after the final runs are replaced by other crossings through the partial \(A\)-resolution.

### Example Scenario

Consider an alternating diagram of a knot with several final runs. Suppose we replace these final runs with other crossings using a partial \(A\)-resolution. The key point is to determine how this replacement affects the knot's signature.

For instance, if we replace a final run of overcrossings with a sequence of undercrossings, the signature will generally decrease because the overcrossings contribute positively to the signature, while the undercrossings contribute negatively.

### Conclusion

The question is asking for a detailed analysis of how the signature of a knot changes when final runs are replaced by other crossings through partial \(A\)-resolutions in an alternating diagram. The positive crossings are important because they directly influence the signature. The exact nature of the change in signature depends on the specific configuration of the final runs and the replacements made during the partial \(A\)-resolution.

In summary, the answer would involve a step-by-step analysis of the diagram, identifying the final runs, determining the effect of the partial \(A\)-resolution on the crossings, and calculating the resulting change in the knot's signature. The positive crossings are crucial in this process as they determine the overall sign contribution to the signature.