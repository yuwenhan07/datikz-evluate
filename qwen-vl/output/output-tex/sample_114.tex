To understand the transitions from \(\sigma\) to \(\hat{\sigma}\) used in the proof of Lemma \ref{lemma:before_still_selected}, we need to carefully analyze the context and the specific steps involved in the proof. However, since the lemma and its proof are not provided in your question, I will assume that you are referring to a common type of proof in combinatorial optimization or graph theory where such transitions might be used.

Let's consider a typical scenario where \(\sigma\) represents an initial state (e.g., a permutation of vertices in a graph) and \(\hat{\sigma}\) represents a transformed state (e.g., a new permutation after some operation). The transitions could involve operations like swaps, insertions, deletions, or other transformations that preserve certain properties of the state.

Here is a general outline of how such transitions might be used in a proof:

1. **Initial State**: Start with the initial state \(\sigma\), which could be any permutation of vertices in a graph.
2. **Transformation Operation**: Apply a transformation operation to \(\sigma\) to get a new state \(\hat{\sigma}\). This operation could be something like:
   - Swapping two elements in the permutation.
   - Inserting an element at a specific position.
   - Deleting an element from the permutation.
   - Applying a specific rule or algorithm that transforms the permutation in a way that preserves certain properties.
3. **Verification**: Verify that the transformation operation preserves the property of interest (e.g., the number of selected vertices remains the same).
4. **Induction Hypothesis**: Assume that for all states before the current transition, the property holds.
5. **Conclusion**: Conclude that the property also holds for the new state \(\hat{\sigma}\).

For example, if the lemma is about a selection process in a graph where the number of selected vertices is preserved through a series of operations, the transitions might look like this:

- Let \(\sigma\) be an initial permutation of vertices in a graph.
- Apply a swap operation to \(\sigma\) to get \(\hat{\sigma}\).
- Verify that the number of selected vertices in \(\sigma\) is the same as in \(\hat{\sigma}\).
- Use induction to show that the number of selected vertices remains constant throughout the sequence of operations.

Without the specific lemma and its proof, it's difficult to provide a more detailed example. However, the general approach would be similar to the one outlined above.

If you can provide the exact lemma and its proof, I can give a more precise answer.