To address the question of possible UV completions for \(\mathcal{O}_3^f\) involving the fermionic dark matter (DM) candidate \(\chi\) (red lines), as well as BSM (Beyond Standard Model) particles \(\Psi\) and \(S\) (grey lines), we need to consider how these elements can be integrated into a consistent UV complete theory. Here's a step-by-step approach:

### 1. **Identify the Lagrangian Elements:**
   - \(\mathcal{O}_3^f\): This operator likely involves three fermions, possibly including the DM candidate \(\chi\).
   - \(\chi\): Fermionic DM candidate.
   - \(\Psi\): BSM fermion.
   - \(S\): Another BSM scalar or fermion.

### 2. **Construct the UV Complete Lagrangian:**
   To construct a UV complete theory, we need to ensure that all interactions are renormalizable and that the theory remains consistent at high energies. This typically involves adding higher-dimensional operators and ensuring that the dimensionality of the operators does not exceed the maximum allowed by the renormalization group flow.

### 3. **Possible UV Completions:**

#### **Option 1: Adding Higher-Dimensional Operators:**
One common way to achieve UV completion is by adding higher-dimensional operators that involve more fields. For example, if \(\mathcal{O}_3^f\) involves three fermions, we could consider adding an operator involving four or five fields.

- **Example Operator**: Consider an operator like \(\mathcal{O}_4 = (\bar{\chi} \chi \Psi S)\). This operator involves four fields and is dimensionally consistent with the renormalizability condition in theories with higher-dimensional operators.

#### **Option 2: Including Additional Fields:**
Another approach is to include additional fields that can mediate the interactions between the DM candidate and the BSM particles. For instance, introducing a new gauge boson \(Z'\) that couples to both \(\chi\) and \(\Psi\) can provide a mechanism for the interactions.

- **Example Interaction**: Consider the interaction term \(L_{int} = g' Z' \cdot (\bar{\chi} \chi \Psi + h.c.)\), where \(g'\) is the coupling constant and \(h.c.\) denotes the Hermitian conjugate.

#### **Option 3: Including Scalar Mediators:**
In some models, scalar mediators can play a crucial role in mediating interactions between fermions. For example, a scalar field \(H\) can couple to the DM candidate and the BSM particles through Yukawa couplings.

- **Example Interaction**: Consider the interaction term \(L_{int} = y \bar{\chi} \chi H + y' \bar{\Psi} \Psi H + h.c.\), where \(y\) and \(y'\) are Yukawa couplings.

### 4. **Consistency Checks:**
After constructing the UV complete Lagrangian, it is essential to check the consistency of the theory:
- **Renormalizability**: Ensure that all interactions are renormalizable.
- **Unitarity**: Check that the theory remains unitary at all orders.
- **Stability**: Verify that the theory is stable under renormalization group flow.

### 5. **Conclusion:**
The UV completion of \(\mathcal{O}_3^f\) can be achieved by adding higher-dimensional operators, including additional fields, or using scalar mediators. The specific choice will depend on the details of the model and the desired properties of the UV complete theory. The key is to ensure that the interactions remain consistent and renormalizable at all energy scales.

If you have a specific model or set of constraints, you can refine this general approach to fit those specifics.