To solve the problem, we need to understand what the action of \( f^m \) means in the context of a transformation \( f \) applied repeatedly. The problem states that for some \( m \in \mathbb{N} \), the action of \( f^m \) on the square bounded by \( x = x_1 \), \( x = x_2 \), \( y = y_1 \), and \( y = y_2 \) determines the red region.

Let's break this down step by step:

1. **Identify the Initial Square**: The initial square is defined by the boundaries \( x = x_1 \), \( x = x_2 \), \( y = y_1 \), and \( y = y_2 \). This square is the domain of the transformation \( f \).

2. **Transformation \( f \)**: The transformation \( f \) maps points from one region to another. When we apply \( f \) once, the image of the square will be a new region. If we apply \( f \) again, the image of the new region will be yet another region, and so on. The process continues until we apply \( f \) \( m \) times.

3. **Action of \( f^m \)**: The notation \( f^m \) represents the composition of the transformation \( f \) with itself \( m \) times. So, \( f^m \) is the transformation that results from applying \( f \) \( m \) times in succession.

4. **Determine the Red Region**: The red region is the result of applying the transformation \( f^m \) to the initial square. In other words, the red region is the image of the square under the transformation \( f^m \).

Therefore, the red region is the image of the square bounded by \( x = x_1 \), \( x = x_2 \), \( y = y_1 \), and \( y = y_2 \) after applying the transformation \( f \) \( m \) times.

The final answer is:
\[
\boxed{\text{The red region is the image of the square bounded by } x = x_1, x = x_2, y = y_1, \text{ and } y = y_2 \text{ under the transformation } f^m.}
\]