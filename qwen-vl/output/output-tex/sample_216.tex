The statement you've provided seems to be discussing the behavior of dynamical systems in terms of cycles within their state space. Let's break it down:

### Left Side: 1-Cycles and 2-Cycles

- **1-Cycles (Equilibrium):** These are fixed points where the system remains unchanged over time. In other words, if \( Q_i \) is a 1-cycle, then applying the dynamics of the system to \( Q_i \) results in \( Q_i \) again. Equilibria are deterministic because they represent stable states where the system does not change.

- **2-Cycles:** These are periodic orbits where the system returns to its initial state after two iterations. For example, if \( Q_i \) is part of a 2-cycle, then applying the dynamics twice brings the system back to \( Q_i \). Like equilibria, 2-cycles are deterministic because the system follows a predictable path.

### Right Side: n-Cycles with \( n > 2 \)

- **n-Cycles with \( n > 2 \):** These are periodic orbits where the system returns to its initial state after \( n \) iterations. For instance, if \( Q_i \) is part of an \( n \)-cycle with \( n > 2 \), then applying the dynamics \( n \) times brings the system back to \( Q_i \).

The statement suggests that:
- **Deterministic Behavior:** An \( n \)-cycle with \( n > 2 \) is never observed deterministically. This implies that such cycles do not exist as stable states or fixed points in the system's dynamics.
  
- **Stochastic Existence:** Stochastically, an \( n \)-cycle with \( n > 2 \) exists with a probability of 0. This means that while the system might exhibit such cycles under certain conditions or due to random perturbations, these cycles are not part of the deterministic attractor basin of the system. The probability of observing such a cycle is effectively zero under typical deterministic conditions.

### Summary

In summary, the left side describes deterministic behaviors (1-cycles and 2-cycles) that are frequently observed, while the right side describes a type of cycle (\( n \)-cycles with \( n > 2 \)) that is never observed deterministically but can exist stochastically with a probability of 0. This distinction highlights the difference between deterministic and stochastic behaviors in complex dynamical systems.