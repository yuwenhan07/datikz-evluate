To create a conceptual map illustrating the electron-mirror duality within the context of a thermal-particle-creation triangle, we can break down the components and relationships into a clear and structured format. Here’s a textual representation of such a map:

---

### Thermal-Particle-Creation Triangle

#### Lower Leg: Electron-Mirror Duality

**Node 1: Classical/Quantum Model of the Electron**
- **Description:** The electron is described using both classical and quantum mechanics.
- **Mathematical Identity Map:** The relationship between these two models is exact, meaning they are equivalent under certain conditions.

**Node 2: Moving Mirror Analogy**
- **Description:** The moving mirror has been used as an analogy for black hole radiation for over half a century.
- **Historical Context:** This analogy has provided insights into the behavior of black holes and their radiation.

**Node 3: Black Holes and Electrons**
- **Description:** The link between black holes and electrons is the least developed side of the triangle.
- **Experimental Challenges:** Investigating black holes experimentally is highly challenging due to their extreme conditions.
- **Electron Results:** Experimental results on electrons provide tractable means for studying the classical quantum aspects of relativistic thermodynamics.

**Node 4: Relativistic Thermodynamics**
- **Description:** A field that has struggled with a severe paucity of experimental evidence.
- **Reorientation of Inquiry:** Solutions derived from electron-mirror duality help reorient the direction of research in this field.

**Node 5: Linographic Utility**
- **Description:** The electron-mirror duality encodes physical content in 3D space into 1D space, making it a powerful tool for analysis.
- **Analytic Examples:** Provides concrete examples that illustrate how the duality can be applied to understand complex systems.

---

### Relationships Between Nodes

1. **Classical/Quantum Model of the Electron → Moving Mirror Analogy**
   - The classical/quantum model of the electron provides the foundation for understanding its behavior, which is then mirrored by the moving mirror analogy.

2. **Moving Mirror Analogy → Black Holes and Electrons**
   - The moving mirror serves as a useful analogy for understanding black hole radiation, but the direct link to electrons remains less explored.

3. **Black Holes and Electrons → Relativistic Thermodynamics**
   - The experimental challenges of black holes and the tractability of electron experiments highlight the need for new approaches in relativistic thermodynamics.

4. **Relativistic Thermodynamics → Linographic Utility**
   - Solutions from the electron-mirror duality help reorient the direction of research in relativistic thermodynamics, providing a linographic utility that simplifies complex 3D phenomena into 1D representations.

---

This conceptual map outlines the connections and relationships between the nodes, highlighting the electron-mirror duality as a key component in bridging the gap between classical/quantum models, moving mirrors, black holes, and relativistic thermodynamics. It also underscores the importance of the linographic utility in making complex systems more manageable through the duality.