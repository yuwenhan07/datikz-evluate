The scenario you're describing is related to solving the wave equation in an expanding spacetime, often encountered in cosmology or general relativity. Let's break down the key components:

1. **Wave Equation in Expanding Spacetime**:
   The wave equation in curved spacetime is typically written as:
   \[
   (\Box - m^2) \psi = 0,
   \]
   where \(\Box\) is the d'Alembertian operator (or wave operator), \(m\) is the mass of the scalar field, and \(\psi\) is the scalar field.

2. **Expanding Region**:
   The region of interest is an expanding spacetime, often modeled by a Friedmann-Lemaître-Robertson-Walker (FLRW) metric. The FLRW metric describes a homogeneous and isotropic universe, and it expands over time.

3. **Solving the Wave Equation**:
   To solve the wave equation in such a spacetime, one typically uses initial and boundary conditions. The initial condition might be specified at some initial time slice, and the boundary conditions are imposed on the spatial boundaries of the region under consideration.

4. **Finite Problems**:
   Since the problem is infinite-dimensional, one solves a sequence of finite problems. This involves considering solutions \(\psi_R\) on a sequence of spatial slices \(\Sigma_R\), where each \(\Sigma_R\) is a hypersurface at a distance \(R\) from a central point (e.g., the origin in a comoving coordinate system).

5. **Prescribed Data on \(\Sigma_R\)**:
   On each \(\Sigma_R\), the wave function \(\psi_R\) is given initial or boundary data. This data could include the value of the wave function and its derivatives at the surface \(\Sigma_R\).

6. **Limiting Behavior**:
   As \(R \to \infty\), the solutions \(\psi_R\) are expected to converge to a limiting solution \(\psi_\infty\) in a suitable energy space. This convergence is crucial for understanding the behavior of the wave function in the asymptotic future of the spacetime.

7. **Hypersurface \(\Sigma_{r_0}\)**:
   The process is stopped at a certain hypersurface \(\Sigma_{r_0}\), where \(r_0 > r_{\mathcal{C}}\). Here, \(r_{\mathcal{C}}\) is the radius of the Cauchy horizon, which is the boundary beyond which causal information cannot be received from the past. By stopping at \(\Sigma_{r_0}\), one ensures that the solutions do not enter the region beyond the Cauchy horizon, thus avoiding potential singularities or pathological behaviors.

In summary, the process involves solving the wave equation on a sequence of spatial slices \(\Sigma_R\) with prescribed data, and then showing that the solutions converge to a well-defined limit \(\psi_\infty\) in an appropriate energy space. The choice of \(\Sigma_{r_0}\) ensures that the analysis remains within a physically meaningful region of the spacetime.