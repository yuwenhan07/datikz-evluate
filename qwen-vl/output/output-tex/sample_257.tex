It seems like you're discussing a topic related to automated theorem proving or logic programming, specifically focusing on how to handle premises and resolutions in a proof system. Let's break down the components of your statement:

1. **Case 2**: This likely refers to a specific case within a broader context, such as a proof strategy or a particular rule application in a logical system.

2. **Lowering the premises for a resolution inference**: In the context of automated theorem proving, "lowering" premises often means simplifying or transforming them into a form that makes it easier to apply certain inference rules, such as resolution. Resolution is a fundamental inference rule used in many theorem provers, particularly in the context of first-order logic.

3. **Unlabeled leaves**: These typically refer to the initial clauses that have not yet been processed or resolved upon. In a proof tree, these are the starting points from which the proof is built up.

4. **Initial clauses from \( f(\Gamma, h) \setminus \Gamma \)**: Here, \( f(\Gamma, h) \) might represent a function that generates new clauses based on the current set of clauses \(\Gamma\) and some heuristic \(h\). The set difference \( \setminus \Gamma \) indicates that we are considering only those clauses generated by \( f \) that are not already part of \(\Gamma\).

5. **Included earlier in \( P_0 \)**: This suggests that these clauses were previously considered or processed in an earlier phase of the proof, possibly as part of the initial setup or preprocessing step.

Putting it all together, the statement appears to be describing a process where:
- You start with a set of clauses \(\Gamma\) and a heuristic \(h\).
- You generate new clauses using \( f(\Gamma, h) \), but only consider those that are not already in \(\Gamma\).
- These new clauses are then included in the proof tree (or some other data structure representing the proof), but they are not yet labeled.
- The goal is to use these unlabeled clauses to simplify or transform the premises in order to apply a resolution inference rule more effectively.

In summary, the statement is likely describing a step in a proof procedure where new clauses are generated and simplified to facilitate the application of resolution inference, with the focus being on the initial clauses that have not yet been processed.