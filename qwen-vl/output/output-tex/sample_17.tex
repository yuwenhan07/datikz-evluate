To sketch the construction of a short odd cycle containing \( v \) in the case \( d = 2 \), where both \( v \) and its blue neighbour \( v' \) lie outside \( Q_{2r} \), we need to follow these steps:

1. **Identify the structure**: We are working in a graph with a specific structure, likely a grid or a similar lattice, where \( Q_{2r} \) is a subgraph (possibly a square or a rectangle) and \( v \) and \( v' \) are vertices outside this subgraph.

2. **Use Claim \ref{claim:paths}**: This claim likely provides a method to find paths between certain vertices. We will use it to find paths from \( v \) and \( v' \) to some vertex \( q'_{t'} \) that lies within \( Q_{2r} \).

3. **Construct the paths**:
   - Start at \( v \). Use Claim \ref{claim:paths} to find a path from \( v \) to some vertex \( q'_{t'} \) inside \( Q_{2r} \).
   - Similarly, start at \( v' \). Use Claim \ref{claim:paths} to find a path from \( v' \) to the same vertex \( q'_{t'} \) inside \( Q_{2r} \).

4. **Close the cycle**:
   - Once we have paths from \( v \) and \( v' \) to \( q'_{t'} \), we can close the cycle by connecting \( q'_{t'} \) back to \( v \) and \( v' \) using the edges of the graph. Since \( v \) and \( v' \) are already connected to \( q'_{t'} \), we can form a cycle by traversing the edges in a way that ensures the cycle is odd.

5. **Verify the cycle**:
   - Ensure that the cycle formed is indeed odd. This typically involves counting the number of edges in the cycle. If the cycle has an odd number of edges, then it is an odd cycle.

The final answer, in a boxed format, would be:

\[
\boxed{\text{The cycle is constructed by using Claim \ref{claim:paths} to find paths from } v \text{ and } v' \text{ to } q'_{t'} \text{ and then closing the cycle by connecting these paths.}}
\]

This sketch outlines the high-level steps involved in constructing the desired cycle. The exact details of the paths and the verification of the cycle's length depend on the specific structure of the graph and the properties of the claim mentioned.