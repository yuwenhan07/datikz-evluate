The image you've described appears to be comparing different phase portraits for a system governed by certain Hamiltonians \( H_-, H, \) and \( H_+ \). Each phase portrait is associated with a specific set of equations derived from these Hamiltonians. The trajectories in red on each graph represent the evolution of the system's state over time, specifically the position \( z \) and its derivative \( \dot{z} \).

Here’s a breakdown of what this comparison might illustrate:

### Left: Phase Portrait for \( H_- \)
- **Hamiltonian \( H_- \)**: This could represent a Hamiltonian that describes a system in a lower energy state or under some specific boundary conditions.
- **Trajectories**: The red trajectory in the left graph shows how the system evolves according to the dynamics defined by \( H_- \).
- **Lines**: The three lines within the phase portrait correspond to different values of \( h_0 < h < h_1 \), where \( h \) is likely a parameter related to the energy level of the system.

### Middle: Phase Portrait for \( H \)
- **Hamiltonian \( H \)**: This represents the standard Hamiltonian without any specific modifications or constraints.
- **Trajectories**: The middle graph shows the trajectory of the system when it is governed by the standard Hamiltonian \( H \).
- **Lines**: Again, the three lines within the phase portrait correspond to the same range of \( h_0 < h < h_1 \).

### Right: Phase Portrait for \( H_+ \)
- **Hamiltonian \( H_+ \)**: This could represent a Hamiltonian that describes a system in a higher energy state or under some other specific boundary conditions.
- **Trajectories**: The right graph shows the trajectory of the system according to the dynamics defined by \( H_+ \).
- **Lines**: The three lines within the phase portrait again correspond to the same range of \( h_0 < h < h_1 \).

### Key Observations:
1. **Energy Levels**: The lines \( h_0 < h < h_1 \) indicate different energy levels of the system. The behavior of the trajectories can change depending on which energy level the system is at.
2. **Dynamics**: The shape and orientation of the trajectories provide insights into the nature of the system's dynamics. For example, closed loops might indicate periodic orbits, while open paths might suggest chaotic behavior.
3. **Comparison**: By comparing the trajectories across the three graphs, one can observe how the system behaves differently under different Hamiltonians. This comparison helps in understanding the influence of the Hamiltonian on the system's dynamics and stability.

In summary, the image is illustrating how the phase portraits and trajectories change as the system transitions between different Hamiltonians (\( H_-, H, H_+ \)) while keeping the energy level within a specified range (\( h_0 < h < h_1 \)). This comparison is crucial for understanding the system's behavior under varying conditions.