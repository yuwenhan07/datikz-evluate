To solve the problem, we need to understand the structure of the transition graph \(\graph{\Pi_1}\) and identify the nodes that belong to some solution of length 4. Here's a step-by-step breakdown:

1. **Identify the Transition Graph**: The transition graph \(\graph{\Pi_1}\) is a directed graph where each node represents a state in the temporal program \(\Pi_1\) and each edge represents a possible transition from one state to another.

2. **Determine the Solutions of Length 4**: A solution of length 4 means a sequence of 4 states (nodes) connected by edges such that there is a path from the first state to the fourth state. We need to find all such sequences and identify the nodes that appear in these sequences.

3. **Highlight the Nodes**: The nodes that belong to at least one solution of length 4 will be highlighted with a gray background. These nodes are part of the core of the graph that can form such a sequence.

4. **Represent the Transitions**: The transitions of the nodes that belong to these solutions will be represented by normal arrows. The other transitions will be represented by dashed arrows.

Since the specific details of the graph \(\graph{\Pi_1}\) are not provided, I'll illustrate the process with a hypothetical example. Let's assume the graph has the following nodes and edges:

- Nodes: \(A, B, C, D, E\)
- Edges: \(A \to B\), \(B \to C\), \(C \to D\), \(D \to E\), \(E \to A\)

In this case, the only solution of length 4 is \(A \to B \to C \to D \to E\). Therefore, the nodes \(A, B, C, D, E\) are all part of this solution. The transitions of these nodes will be represented by normal arrows, and the other transitions will be represented by dashed arrows.

The final answer would look like this:

\[
\begin{array}{c}
\text{Nodes: } A, B, C, D, E \\
\text{Edges: } A \to B, B \to C, C \to D, D \to E, E \to A \\
\text{Highlighted nodes: } A, B, C, D, E \\
\text{Normal arrows: } A \to B, B \to C, C \to D, D \to E, E \to A \\
\text{Dashed arrows: } \text{(all other edges)}
\end{array}
\]

So, the final answer is:

\boxed{\text{The nodes } A, B, C, D, E \text{ are highlighted with a gray background, and the transitions } A \to B, B \to C, C \to D, D \to E, E \to A \text{ are represented by normal arrows, while the other arrows are dashed.}}