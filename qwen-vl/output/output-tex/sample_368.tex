The problem you've described involves the Singular Value Decomposition (SVD) of a matrix \( \boldsymbol{J} \), which is a common technique used in various fields such as data analysis, machine learning, and signal processing. The SVD of a matrix \( \boldsymbol{J} \) is given by:

\[
\boldsymbol{J} = \boldsymbol{U}_J \boldsymbol{\Sigma}_J \boldsymbol{V}_J^\top
\]

where:
- \( \boldsymbol{U}_J \) is an \( n \times r_J \) orthogonal matrix whose columns are the left singular vectors.
- \( \boldsymbol{\Sigma}_J \) is an \( r_J \times r_J \) diagonal matrix containing the singular values.
- \( \boldsymbol{V}_J \) is an \( p \times r_J \) orthogonal matrix whose rows are the right singular vectors.

Given that \( r_J = 2 \), \( n = 3 \), and \( p = 4 \), we have the following dimensions for the matrices involved:
- \( \boldsymbol{U}_J \) is \( 3 \times 2 \).
- \( \boldsymbol{\Sigma}_J \) is \( 2 \times 2 \).
- \( \boldsymbol{V}_J \) is \( 4 \times 2 \).

Each of the joint components has an \( n \)-dimensional score vector (a column of \( \boldsymbol{U}_J \)) and a \( p \)-dimensional loading vector (a row of \( \boldsymbol{V}_J^\top \)). Each subward has an \( r_J \)-dimensional score vector (a row of \( \boldsymbol{U}_J \)), and each feature has a \( r_J \)-dimensional loading vector (a column of \( \boldsymbol{V}_J^\top \)).

The singular values in \( \boldsymbol{\Sigma}_J \) are distinct and ordered from largest to smallest. This means that the singular values are unique and can be used to identify the importance of each component in the decomposition.

The decomposition is identifiable up to multiplication of components by -1. Specifically, we can multiply any column of \( \boldsymbol{U}_J \) and the corresponding column of \( \boldsymbol{V}_J \) (row of \( \boldsymbol{V}_J^\top \)) by -1 without changing the value of \( \boldsymbol{U}_J \boldsymbol{\Sigma}_J \boldsymbol{V}_J^\top \).

This property allows us to interpret the signs of the singular vectors in a meaningful way, but it does not affect the overall structure or the interpretation of the decomposition itself. The signs of the singular vectors can be adjusted based on the context or the specific application of the decomposition.

In summary, the SVD of \( \boldsymbol{J} \) with \( r_J = 2 \), \( n = 3 \), and \( p = 4 \) provides a way to decompose the matrix into its constituent parts, with the singular values indicating the relative importance of each component, and the orthogonal matrices \( \boldsymbol{U}_J \) and \( \boldsymbol{V}_J \) providing the directions of these components in the original space. The decomposition is identifiable up to sign changes, which can be adjusted as needed.