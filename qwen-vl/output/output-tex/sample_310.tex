To parametrize the \( Y \)-variables in terms of the \( q \)-Weyl algebra generators in the vicinity of the crossing \( i \) (center) of the wiring diagram, we need to understand the structure of the symmetric butterfly quiver and how the \( Y \)-variables interact with the \( q \)-Weyl algebra generators.

The symmetric butterfly quiver has a central crossing \( i \) and four vertices (north, east, south, west) connected to it. Each vertex can be assigned a factor based on its position relative to the crossing \( i \).

Let's denote the \( q \)-Weyl algebra generators by \( x_i \) and \( y_i \), where \( x_i \) and \( y_i \) are the canonical variables associated with the crossing \( i \). The \( Y \)-variables at each vertex will be expressed as exponentials of linear combinations of these generators.

### Parametrization of \( Y \)-variables

1. **Vertex North of \( i \)**:
   - The \( Y \)-variable at the north vertex acquires a factor \( \mathrm{e}^{a_i + w_i} \).
   - Therefore, the \( Y \)-variable at the north vertex is given by:
     \[
     Y_{\text{north}} = \mathrm{e}^{a_i + w_i} x_i y_i
     \]

2. **Vertex East of \( i \)**:
   - The \( Y \)-variable at the east vertex acquires a factor \( \mathrm{e}^{b_i - u_i - w_i} \).
   - Therefore, the \( Y \)-variable at the east vertex is given by:
     \[
     Y_{\text{east}} = \mathrm{e}^{b_i - u_i - w_i} x_i y_i
     \]

3. **Vertex South of \( i \)**:
   - The \( Y \)-variable at the south vertex acquires a factor \( \mathrm{e}^{c_i + w_i} \).
   - Therefore, the \( Y \)-variable at the south vertex is given by:
     \[
     Y_{\text{south}} = \mathrm{e}^{c_i + w_i} x_i y_i
     \]

4. **Vertex West of \( i \)**:
   - The \( Y \)-variable at the west vertex acquires a factor \( \mathrm{e}^{d_i - u_i - w_i} \).
   - Therefore, the \( Y \)-variable at the west vertex is given by:
     \[
     Y_{\text{west}} = \mathrm{e}^{d_i - u_i - w_i} x_i y_i
     \]

5. **Vertex at \( i \)**:
   - The \( Y \)-variable at the central crossing \( i \) acquires a factor \( \mathrm{e}^{e_i + 2u_i} \).
   - Therefore, the \( Y \)-variable at the central crossing \( i \) is given by:
     \[
     Y_i = \mathrm{e}^{e_i + 2u_i} x_i y_i
     \]

### Summary

The \( Y \)-variables at each vertex are expressed as follows:

- **North**: \( Y_{\text{north}} = \mathrm{e}^{a_i + w_i} x_i y_i \)
- **East**: \( Y_{\text{east}} = \mathrm{e}^{b_i - u_i - w_i} x_i y_i \)
- **South**: \( Y_{\text{south}} = \mathrm{e}^{c_i + w_i} x_i y_i \)
- **West**: \( Y_{\text{west}} = \mathrm{e}^{d_i - u_i - w_i} x_i y_i \)
- **Central Crossing \( i \)**: \( Y_i = \mathrm{e}^{e_i + 2u_i} x_i y_i \)

These expressions show how the \( Y \)-variables are parametrized using the \( q \)-Weyl algebra generators \( x_i \) and \( y_i \) and the parameters \( a_i, b_i, c_i, d_i, e_i, u_i, w_i \) that depend on the specific positions and interactions within the wiring diagram.