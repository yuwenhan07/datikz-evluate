The notation \(\mathcal{M}(p, 4)\) typically refers to a specific class of modular categories or fusion categories that have been studied in the context of conformal field theory (CFT) and topological quantum field theory (TQFT). These categories often arise from the study of defects in CFTs, particularly those involving duality defects.

In the context of the figure you mentioned, it seems to be discussing the classification of these categories based on their spin contents and quantum dimensions, and how they influence the renormalization group (RG) flow. The RG flow is a way to describe how the properties of a system change as we vary some parameter, such as temperature or coupling constant.

Here's a breakdown of what the figure might be illustrating:

1. **Spin Contents**: In CFTs, the spin content of a defect line can be either integer or half-integer. Integer spin defects are more symmetric and tend to preserve the duality defect lines under RG flow. Half-integer spin defects, on the other hand, can break the symmetry and allow for more complex RG flows.

2. **Quantum Dimensions**: Quantum dimensions are numerical invariants associated with the objects in a modular category. They play a crucial role in determining the behavior of defects under RG flow. For example, defects with certain quantum dimensions may be more stable under RG flow than others.

3. **Dotted Arrows**: The dotted arrows in the figure likely represent possible RG flows that do not preserve the duality defect lines. These flows are associated with half-integer spin defects, which can break the symmetry of the duality defect lines. This means that the properties of the system can change in a more complex manner when these flows occur.

4. **Renormalization Group Flow**: The RG flow is constrained by the properties of the defects, such as their spin content and quantum dimensions. The figure shows how different types of defects (classified by their spin contents and quantum dimensions) can influence the RG flow. The constraints imposed by these defects can lead to different patterns of RG flow, which are represented by the solid and dotted arrows in the figure.

In summary, the figure is likely illustrating how the properties of defects (classified by their spin contents and quantum dimensions) influence the RG flow of a system. The dotted arrows represent RG flows that are possible but do not preserve the duality defect lines, which are associated with half-integer spin defects. The solid arrows represent RG flows that preserve the duality defect lines, which are associated with integer spin defects.