The inclusion relations between various classes of commutative rings, particularly those involving ideals and divisibility properties, are fundamental in algebraic geometry and number theory. Let's discuss some of these classes and their relationships:

1. **Fields**: A field is a commutative ring in which every non-zero element has a multiplicative inverse.
2. **Integral Domains**: An integral domain is a commutative ring with no zero divisors.
3. **Unique Factorization Domains (UFDs)**: A UFD is an integral domain where every non-zero, non-unit element can be uniquely factored into irreducible elements.
4. **Principal Ideal Domains (PIDs)**: A PID is an integral domain in which every ideal is principal, i.e., generated by a single element.
5. **Euclidean Domains**: A Euclidean domain is a PID that also satisfies the Euclidean algorithm for division with remainder.
6. **Bézout Domains**: A Bézout domain is an integral domain in which every finitely generated ideal is principal.
7. **Elementary Divisor Domains (Edd)**: An Edd is a ring in which every matrix over the ring has an elementary divisor form.

The inclusion relations among these classes are as follows:
- Every field is an integral domain.
- Every integral domain is a unique factorization domain if it is a UFD.
- Every UFD is a principal ideal domain if it is a PID.
- Every PID is a Euclidean domain.
- Every Euclidean domain is a Bézout domain.
- Every Bézout domain is an elementary divisor domain.

The strictness of these inclusions means that there exist examples of rings that satisfy the conditions for the lower class but not for the higher class. For instance:
- The integers \(\mathbb{Z}\) are a PID but not a Euclidean domain.
- The Gaussian integers \(\mathbb{Z}[i]\) are a Euclidean domain but not a PID.

The exception mentioned in your question is the relationship between Bézout domains and elementary divisor domains. It is known that every Bézout domain is an elementary divisor domain, but the converse is not true. There exist elementary divisor domains that are not Bézout domains. This is a subtle point in the structure of these rings and highlights the complexity of the relationships between these classes.

To summarize, the inclusion relations between these classes of commutative rings are strict except for the one between Bézout domains and elementary divisor domains, where the inclusion is known to be proper.