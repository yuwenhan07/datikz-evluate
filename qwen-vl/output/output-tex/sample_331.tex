The image you've described appears to be discussing the concept of polysemy and how it can be represented using a semantic tree or graph structure. Polysemy refers to a word having multiple meanings, often related but distinct from each other.

### Top: Representation of Polysemous Meanings

In the top part of the image, we see a semantic tree graph that represents the different meanings of a word \( w \). Each branch of the tree represents a different sense or meaning of the word. This type of visualization is useful for understanding the relationships between different meanings of a word and how they might be interconnected.

### Bottom: Graph Representation of "mouse"

The bottom part of the image shows a graph representation where "mouse" is the root node. This graph is likely generated by applying some algorithm or method to analyze the English Wikipedia corpus. The nodes represent different senses or uses of the word "mouse," and the edges represent relationships between these senses, such as co-occurrence in text, similarity in usage, or other semantic connections.

This kind of graph analysis can help linguists, lexicographers, and computational linguists understand the nuances and variations in the use of words like "mouse," which can have multiple meanings depending on context. For example, "mouse" can refer to a small rodent, a computer input device, or even slang terms.

### Example of "mouse" in Context

1. **Rodent**: "The mouse ran through the forest."
2. **Computer Input Device**: "Click the mouse to open the file."
3. **Slang**: "He's a real mouse; he never stands up for himself."

Each of these uses of "mouse" has its own context and connotations, and the graph helps visualize these relationships.

### Conclusion

The image effectively illustrates how complex words with multiple meanings can be analyzed and visualized using semantic graphs. Such representations are crucial for natural language processing tasks, including machine translation, sentiment analysis, and information retrieval.