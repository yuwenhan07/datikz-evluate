The statement you've made seems to be referring to the concept of fairness in machine learning models, particularly in the context of group fairness. When we talk about "worst group accuracy," it typically refers to the performance of a model on the group that performs the worst in terms of accuracy or some other metric.

In the context of fairness, the parameter \(\alpha\) often represents a threshold for the ratio of the accuracy of the worst-performing group to the best-performing group. If \(\alpha \geq 1\), it means that the accuracy of the worst-performing group must be at least as high as the accuracy of the best-performing group.

Here's a breakdown of what this might mean:

1. **Worst Group Accuracy**: This is the lowest accuracy achieved by any group in the dataset.
2. **\(\alpha \geq 1\)**: This condition ensures that the accuracy of the worst-performing group is not worse than the accuracy of the best-performing group. In other words, if \(\alpha = 1\), then the worst group's accuracy is exactly the same as the best group's accuracy. If \(\alpha > 1\), then the worst group's accuracy is at least as good as the best group's accuracy, but potentially better.

### Example:
Suppose you have three groups (A, B, and C) with accuracies of 80%, 90%, and 70% respectively. If \(\alpha = 1\), then the worst group (C) has an accuracy of 70%, which is exactly the same as the best group (B). However, if \(\alpha = 1.5\), then the worst group (C) must have an accuracy of at least 135% of the best group (B)'s accuracy, which is not possible since accuracy cannot exceed 100%. Therefore, \(\alpha = 1.5\) would not be feasible under these conditions.

### Impact on Different Datasets:
The impact of \(\alpha \geq 1\) on different datasets can vary significantly depending on the distribution of accuracies across groups. In some datasets, it may be easy to achieve \(\alpha \geq 1\) because the differences in accuracy between groups are small. In other datasets, achieving such a condition might be very challenging or even impossible due to large disparities in group performances.

### Conclusion:
The statement suggests that setting \(\alpha \geq 1\) can lead to a rapid increase in the worst group's accuracy. This is because it forces the model to perform at least as well on the worst-performing group as it does on the best-performing group. However, the feasibility of this condition depends heavily on the specific dataset and the distribution of accuracies among different groups. In some cases, it might be necessary to relax the condition slightly to make the model practical and effective.