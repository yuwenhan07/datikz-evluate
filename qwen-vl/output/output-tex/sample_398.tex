The statement you provided describes a method for solving the wave equation in a spacetime region that extends to the cosmological horizon. This approach is often used in the study of gravitational waves and other phenomena in general relativity. Let's break down the key components:

1. **Wave Equation**: The wave equation is a partial differential equation that describes how waves propagate through space and time. In the context of general relativity, it can be applied to describe gravitational waves or other perturbations in the spacetime metric.

2. **Spacetime Region**: The region of interest includes the entire spacetime, extending from some initial time \( r_0 \) to infinity. This means considering the evolution of the wave equation from an initial surface \( \Sigma_{r_0} \) (which could be a Cauchy surface) to future times.

3. **Cosmological Horizons**: Cosmological horizons are regions in spacetime where light cannot escape due to the expansion of the universe. They are important in cosmology and general relativity, particularly in the study of the early universe and the structure formation.

4. **Finite Problems**: The wave equation is solved in a series of finite problems. Each problem involves prescribing initial and boundary conditions on a specific surface \( \Sigma_{r_0}^T \), which is a surface at a finite time \( T \). Here, \( C_T \) represents the causal past of this surface, and \( \overline{C}_T^c \) represents the complement of the causal past, which includes the future and the rest of the spacetime.

5. **Prescribed Data**: On the surface \( \Sigma_{r_0}^T \cup C_T^c \cup \overline{C}_T^c \), the wave equation is solved with prescribed initial and boundary data. This means specifying the values of the wave function and its derivatives at these surfaces.

6. **Limit as \( T \to \infty \)**: As the time \( T \) approaches infinity, the solutions \( \psi_T \) are considered. The goal is to show that there exists a limit \( \lim_{T \to \infty} \psi_T \) in a suitable energy space. This limit is expected to represent the true solution to the wave equation in the entire spacetime region, including the cosmological horizon.

7. **Energy Space**: The energy space refers to a mathematical framework that ensures the solutions are well-behaved and physically meaningful. It typically involves norms that control the size and behavior of the wave function and its derivatives over the entire spacetime region.

In summary, the method described is a way to construct solutions to the wave equation in a spacetime region that extends to the cosmological horizon by solving a sequence of finite problems and showing that the solutions converge to a limit in an appropriate energy space as the time parameter goes to infinity. This approach is crucial in the study of gravitational waves and other phenomena in cosmology and general relativity.