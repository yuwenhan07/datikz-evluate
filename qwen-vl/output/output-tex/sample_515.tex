The notation you've provided is commonly used in graph theory, particularly when discussing trees or directed acyclic graphs (DAGs). Let's break down the components:

- \( k \in \mathbb{V} \): This means that node \( k \) belongs to the set of vertices \( \mathbb{V} \).

- \( i_1, i_2, \ldots, i_p \in \mathcal{C}(k) \): The set \( \mathcal{C}(k) \) represents the set of children of node \( k \). Therefore, each \( i_j \) (for \( j = 1, 2, \ldots, p \)) is a child of node \( k \).

- \( j = \mathcal{P}(k) \): The function \( \mathcal{P}(k) \) denotes the parent of node \( k \). So, \( j \) is the parent of node \( k \).

In summary:
- \( k \) is a vertex in the graph.
- \( i_1, i_2, \ldots, i_p \) are the children of \( k \).
- \( j \) is the parent of \( k \).

This structure is fundamental in representing hierarchical relationships, such as in family trees, organizational charts, or computer science applications like tree data structures.