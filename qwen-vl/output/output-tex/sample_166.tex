To understand the wedge \( C \times \mathbb{R} \) where \( C \) is defined as the graph of \( \log(x+1) \) for \( x \geq 0 \) and \( 0 \) for \( x < 0 \), we need to first visualize the set \( C \).

The set \( C \) consists of two parts:
1. For \( x \geq 0 \), the point \((x, \log(x+1))\) lies on the curve \( y = \log(x+1) \).
2. For \( x < 0 \), the point \((x, 0)\) lies on the line \( y = 0 \).

So, \( C \) is the union of the curve \( y = \log(x+1) \) for \( x \geq 0 \) and the x-axis (the line \( y = 0 \)) for \( x < 0 \). This can be written as:
\[ C = \{(x, \log(x+1)) \mid x \geq 0\} \cup \{(x, 0) \mid x < 0\}. \]

Now, the wedge \( C \times \mathbb{R} \) is the Cartesian product of the set \( C \) with the real numbers \( \mathbb{R} \). This means that each element in \( C \times \mathbb{R} \) is an ordered pair \((c, r)\) where \( c \in C \) and \( r \in \mathbb{R} \).

To get a better understanding, let's consider some specific points in \( C \times \mathbb{R} \):
- If \( c = (x, \log(x+1)) \) for \( x \geq 0 \), then \( c \times \mathbb{R} = \{(x, \log(x+1), r) \mid r \in \mathbb{R}\} \).
- If \( c = (x, 0) \) for \( x < 0 \), then \( c \times \mathbb{R} = \{(x, 0, r) \mid r \in \mathbb{R}\} \).

Thus, the wedge \( C \times \mathbb{R} \) is the union of all such sets for each \( c \in C \). In other words, it is the set of all points \((x, y, z)\) where either \( y = \log(x+1) \) for \( x \geq 0 \) or \( y = 0 \) for \( x < 0 \), and \( z \) can be any real number.

Therefore, the wedge \( C \times \mathbb{R} \) can be described as:
\[ C \times \mathbb{R} = \{(x, \log(x+1), z) \mid x \geq 0\} \cup \{(x, 0, z) \mid x < 0\} \cup \{(0, 0, z) \mid z \in \mathbb{R}\}. \]

In simpler terms, it is the union of the surface \( y = \log(x+1) \) for \( x \geq 0 \) and the plane \( y = 0 \) for \( x < 0 \), both extended infinitely along the \( z \)-axis.

The final answer is:
\[ \boxed{C \times \mathbb{R}} \]