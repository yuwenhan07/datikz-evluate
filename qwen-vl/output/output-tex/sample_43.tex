The trend of the mean Cross-Entropy (mCE) and Soft-Accuracy (SA) with respect to the rate parameter can provide insights into how these metrics change as the rate parameter varies. Here, we'll discuss what these terms typically represent and how they might behave under different conditions.

### Mean Cross-Entropy (mCE)
Mean Cross-Entropy is a measure of the average difference between the predicted probabilities and the true labels. It's commonly used as a loss function in classification tasks. Lower values of mCE indicate better model performance because it suggests that the model's predictions are closer to the true labels.

### Soft-Accuracy (SA)
Soft-Accuracy is a metric that combines both accuracy and confidence. It is calculated by taking the weighted sum of the accuracy and the confidence score. The confidence score is often derived from the softmax output of the model, where higher values indicate greater confidence in the prediction. SA is useful for evaluating models that not only need to be accurate but also confident in their predictions.

### Rate Parameter
In the context of your question, the "rate parameter" likely refers to the learning rate or some other hyperparameter that controls the step size at which the model updates its weights during training. This could also refer to the rate at which data is sampled or processed, such as the batch size or the number of epochs.

### Behavior with Respect to the Rate Parameter

1. **Learning Rate (if applicable):**
   - **Low Learning Rate:** With a low learning rate, the model may converge slowly but may achieve lower mCE and higher SA because the updates are small and more precise.
   - **High Learning Rate:** With a high learning rate, the model may converge faster but may overshoot the optimal solution, leading to higher mCE and potentially lower SA due to instability in the training process.

2. **Batch Size:**
   - **Small Batch Size:** Smaller batches can lead to more frequent updates and potentially higher variance in the gradient estimates, which might result in higher mCE and lower SA.
   - **Large Batch Size:** Larger batches can lead to smoother gradients and potentially lower mCE and higher SA, but this comes at the cost of increased memory usage and longer training times.

3. **Number of Epochs:**
   - **Fewer Epochs:** Fewer epochs might lead to lower mCE and higher SA if the model has enough capacity to learn quickly. However, if the model is underfitting, it might have higher mCE and lower SA.
   - **More Epochs:** More epochs can help the model learn more complex patterns, potentially leading to lower mCE and higher SA, but there is a risk of overfitting.

### Training with Auxiliary Features (AFA)
When training with Auxiliary Features (AFA), the model is trained on an additional dataset that provides extra information to improve generalization. This can sometimes lead to better performance on the main task, especially if the auxiliary features are relevant and informative.

### Summary
- **mCE:** Generally decreases with increasing rate parameters (learning rate, batch size, etc.) as the model learns more effectively. However, if the rate is too high, it might lead to unstable training and higher mCE.
- **SA:** Typically increases with increasing rate parameters as the model becomes more confident in its predictions. However, if the rate is too high, it might lead to overconfidence and lower SA.

In conclusion, the trends of mCE and SA with respect to the rate parameter depend on the specific settings and the nature of the problem. It's important to experiment with different rates and monitor the performance metrics to find the optimal balance for your model.