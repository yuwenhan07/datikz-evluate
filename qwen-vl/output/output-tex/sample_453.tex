To understand the plot you described, let's break it down step by step.

1. **Understanding the Axes:**
   - The \( x \)-axis represents the approximation ratio. This is a measure of how close the solution produced by the algorithm is to the optimal solution. For example, if the approximation ratio is 2, then the solution found by the algorithm is at most twice as large as the optimal solution.
   - The \( y \)-axis represents the base of the exponent in the running time. This means that the running time of the algorithm is given by \( d^k \cdot n^{\Oh(1)} \), where \( d \) is the value on the \( y \)-axis and \( k \) is some constant that does not depend on \( d \).

2. **Interpreting the Points:**
   - Each point \((\beta, d)\) on the plot corresponds to a specific running time of the form \( d^k \cdot n^{\Oh(1)} \). Here, \(\beta\) is the approximation ratio, and \( d \) is the base of the exponent in the running time.
   - For example, if a point is at \((2, 3)\), it means that for an approximation ratio of 2, the running time of the algorithm is \( 3^k \cdot n^{\Oh(1)} \).

3. **Example:**
   - Suppose we have a point \((2, 3)\) on the plot. This means that for an approximation ratio of 2, the running time of the algorithm is \( 3^k \cdot n^{\Oh(1)} \).
   - If another point is \((3, 4)\), this means that for an approximation ratio of 3, the running time of the algorithm is \( 4^k \cdot n^{\Oh(1)} \).

So, the plot essentially shows how the running time of the algorithm changes with different approximation ratios. The higher the value of \( d \) (the base of the exponent), the faster the running time grows as the size of the input \( n \) increases, but only up to a constant factor \( n^{\Oh(1)} \).

The final answer is:

\[
\boxed{\text{The plot shows the running time of the algorithm as a function of the approximation ratio, with the running time being } d^k \cdot n^{\Oh(1)} \text{ for a given approximation ratio } \beta.}
\]