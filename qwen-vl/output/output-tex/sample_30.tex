The colored S6V (Six-Vertex Model) is a statistical mechanical model that has been extensively studied, particularly in the context of lattice models and their applications to combinatorics and quantum physics. The transformation of the domain in the context of Lemma~\ref{l.S_N symmetry} likely refers to a specific transformation or reconfiguration of the lattice structure used in the model.

In the colored S6V model, the vertices of the lattice are typically colored with six different colors, and each vertex is associated with a weight that depends on the configuration of its neighbors. The spectral parameter \( z_i \) mentioned in your description likely refers to the spectral parameter associated with the weights at each vertex, which can be thought of as a complex number that influences the behavior of the model.

The transformation of the domain could involve several steps:

1. **Reordering of Vertices**: The vertices might be reordered in some systematic way, such as cyclically shifting them or rearranging them based on certain rules. This could affect how the weights are assigned and how the interactions between vertices are computed.

2. **Rescaling of Weights**: The weights at each vertex might be rescaled by a factor that depends on the spectral parameter \( z_i \). This rescaling can change the relative strengths of the interactions and thus alter the overall behavior of the model.

3. **Symmetry Operations**: The transformation might involve applying symmetry operations to the lattice. For example, the lattice could be rotated, reflected, or translated in a way that preserves the symmetry of the model but changes the arrangement of the vertices and their weights.

4. **Domain Restriction**: The domain of the model might be restricted to a smaller subset of the original lattice. This could be done to focus on a particular region of interest or to simplify the analysis of the model.

5. **Parameter Adjustment**: The spectral parameters \( z_i \) might be adjusted in a way that affects the overall symmetry properties of the model. This could involve changing the values of these parameters or their relationships to each other.

Without more specific details about the transformation and the context of Lemma~\ref{l.S_N symmetry}, it's challenging to provide a precise description of the transformation. However, the general idea is that the transformation aims to preserve certain symmetries or properties of the model while altering its structure in a controlled manner.

If you have a specific diagram or further details about the transformation, please provide them, and I can offer a more detailed explanation.