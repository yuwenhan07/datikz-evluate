To provide an accurate response to your question about the UCI-121 test set accuracy for a given train step using different methods (SVE, ENT, and SR), we would need specific data or results from experiments conducted on these methods. Since you haven't provided any specific data, I'll outline how you might approach this problem and what kind of information you would typically expect.

### Steps to Analyze UCI-121 Test Set Accuracy:

1. **Data Collection**: You would need the test set accuracy values for each method (SVE, ENT, SR) at various train steps. This data is usually obtained through training a model on the UCI-121 dataset and evaluating its performance on the test set after each train step.

2. **Train Step**: The train step refers to the number of iterations or epochs completed during training. For example, if you train for 1000 steps, then the train step could range from 1 to 1000.

3. **Accuracy Values**: For each train step, you would have a corresponding accuracy value for each method. These values are typically expressed as percentages or in terms of accuracy improvement over time.

4. **Formatting**: The accuracy values are often reported in thousands, meaning that an accuracy of 98% would be reported as 98000.

### Example Data Format:
If you had the following data:

- Train Step: 1, 2, 3, ..., 1000
- Accuracy for SVE: 98000, 98500, 99000, ..., 99800
- Accuracy for ENT: 97000, 97500, 98000, ..., 99700
- Accuracy for SR: 96000, 96500, 97000, ..., 99600

### Analysis:
- **Trend Analysis**: You can plot the accuracy values against the train step to see how the accuracy changes over time for each method.
- **Comparison**: Compare the trends and final accuracies of the three methods to determine which one performs better.
- **Stability**: Check the stability of the accuracy improvements over time for each method.

### Conclusion:
Without specific data, it's not possible to provide exact accuracy values or draw conclusions. However, if you provide the actual data, I can help you analyze it and provide insights into the performance of the different methods on the UCI-121 test set.