To solve this problem, we need to understand the context of the vector \( w' \) and its relationship with the sequence of support vectors \( (p_n)_{n \in \mathbb{N}} \) approaching the limit \( p \). Let's break it down step by step.

1. **Understanding Support Vectors**: In the context of convex sets and their supporting hyperplanes, a support vector is a point on the boundary of the set such that the hyperplane tangent to the set at that point is perpendicular to the direction vector \( w' \).

2. **Direction Vector \( w' \)**: The vector \( w' \) represents the direction from which the sequence of support vectors \( (p_n)_{n \in \mathbb{N}} \) approaches the limit \( p \). This means that as \( n \to \infty \), the support vectors \( p_n \) get closer and closer to the point \( p \) in the direction described by \( w' \).

3. **Largest Component of \( w' \)**: The problem states that the point \( x \) has the largest \( w' \) component among all points on the corresponding face of the set \( C \). This implies that the point \( x \) is the farthest along the direction \( w' \) from the origin or some reference point within the set \( C \).

4. **Conclusion**: Since \( x \) is the point on the face of \( C \) that has the largest \( w' \) component, it must be the point that is most aligned with the direction \( w' \) among all points on that face. Therefore, the point \( x \) is the point on the face of \( C \) that is closest to the direction \( w' \) in the sense of having the largest projection onto \( w' \).

Thus, the answer is:
\[
\boxed{x}
\]