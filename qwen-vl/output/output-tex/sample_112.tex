To solve the problem, we need to understand the definitions of the functions \( f \) and \( g \) and how they relate to the states of the DFA.

1. **Understanding \( f \):**
   - The function \( f: \Psi^* \rightarrow \{-1, 1\} \) assigns a value to each word in the language based on whether it starts with a red (dashed) or blue (solid) symbol.
   - If a word starts with a red symbol, \( f \) returns 1.
   - If a word starts with a blue symbol, \( f \) returns -1.
   - Therefore, \( f \) is positive for words that start with either a red or a blue symbol.

2. **Understanding \( g \):**
   - The function \( g: \Psi^* \rightarrow \{-1, 1\} \) assigns a value to each word in the language based on whether it does not start with a red symbol.
   - If a word starts with a blue symbol, \( g \) returns 1.
   - If a word starts with a red symbol, \( g \) returns -1.
   - Therefore, \( g \) is positive for words that do not start with a red symbol, i.e., words that start with a blue symbol.

Given these definitions, let's analyze the implications for the DFA:

- A word is positive under \( f \) if it starts with either a red or a blue symbol.
- A word is positive under \( g \) if it starts with a blue symbol.

Since every word must start with either a red or a blue symbol, the set of words that are positive under \( f \) is exactly the same as the set of words that are positive under \( g \). This means that the DFA must have a state where both conditions are satisfied simultaneously.

In a DFA, a state can be considered an "intersection" of conditions if it represents a state where both \( f \) and \( g \) are positive. However, since \( f \) and \( g \) are complementary (one is positive for words starting with red and the other is positive for words starting with blue), there cannot be a single state that satisfies both conditions simultaneously unless the DFA is designed such that it has a state that is both a red corner and a blue corner, which is not possible in a standard DFA without additional structure like a pushdown automaton.

Therefore, the only way to satisfy both conditions is if the DFA has a state that is both a red corner and a blue corner, which is not possible in a standard DFA. Hence, the DFA must have a state that is both a red corner and a blue corner, which is a contradiction unless the DFA is designed in a specific way that allows for this intersection.

The answer is that the DFA must have a state that is both a red corner and a blue corner, which is not possible in a standard DFA without additional structure like a pushdown automaton. However, in the context of the problem, the correct interpretation is that the DFA must have a state that is both a red corner and a blue corner, which is a contradiction unless the DFA is designed in a specific way that allows for this intersection.

Thus, the answer is:
\[
\boxed{\text{The DFA must have a state that is both a red corner and a blue corner}}
\]