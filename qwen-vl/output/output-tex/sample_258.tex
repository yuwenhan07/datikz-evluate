To determine the maximal support and the Newton polytope of a degree 6 polynomial that has a zero of multiplicity 4 at \((0,0)\), we need to follow these steps:

1. **Understand the Multiplicity Condition**: A zero of multiplicity \(k\) at \((0,0)\) means that the polynomial can be written in the form:
   \[
   P(x,y) = x^4 Q(x,y)
   \]
   where \(Q(x,y)\) is a polynomial of degree \(2\) (since \(6 - 4 = 2\)).

2. **Form of the Polynomial**: The polynomial \(P(x,y)\) can thus be expressed as:
   \[
   P(x,y) = x^4 (a_{20} y^2 + a_{11} xy + a_{02} x^2)
   \]
   where \(a_{20}\), \(a_{11}\), and \(a_{02}\) are constants, and \(a_{20} \neq 0\) because the term \(x^4 y^2\) would not be present if \(a_{20} = 0\).

3. **Newton Polytope**: The Newton polytope of a polynomial is the convex hull of the exponents of its monomials. For the polynomial \(P(x,y) = x^4 (a_{20} y^2 + a_{11} xy + a_{02} x^2)\), the exponents of the monomials are \((4,0)\), \((3,1)\), \((2,2)\), \((4,1)\), \((3,2)\), and \((2,0)\). The convex hull of these points is a triangle with vertices \((4,0)\), \((3,1)\), and \((2,2)\).

4. **Maximal Support**: The maximal support of a polynomial is the set of all exponents of its monomials. For the polynomial \(P(x,y) = x^4 (a_{20} y^2 + a_{11} xy + a_{02} x^2)\), the maximal support is the set \(\{(4,0), (3,1), (2,2), (4,1), (3,2), (2,0)\}\).

Therefore, the maximal support and the Newton polytope of the polynomial are:

\[
\boxed{\{(4,0), (3,1), (2,2), (4,1), (3,2), (2,0)\}, \text{the triangle with vertices } (4,0), (3,1), (2,2)}
\]