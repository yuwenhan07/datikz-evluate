The statement you've provided highlights an interesting aspect of how an automated system like OptiMUS operates in the context of solving optimization problems, particularly those from the NL4OPT dataset. Here's a breakdown of the key points:

1. **Simple Problems**: For simpler problems within the NL4OPT dataset, OptiMUS is designed to require only one call per agent to solve the problem. This suggests that these problems are straightforward enough for the system to handle with minimal human intervention.

2. **Complex Problems**: As the complexity of the problems increases, OptiMUS's performance becomes more dependent on human intervention. This means that while the system can identify and highlight errors, it may not always be able to resolve them autonomously. The reliance on human intervention increases as the problems become more complex.

3. **Formulation Errors**: The statement also notes that OptiMUS rarely improves by fixing formulation errors. This implies that when the system identifies issues related to the problem formulation (e.g., incorrect constraints, objective functions), it might not have the capability or the necessary information to correct these errors automatically. Instead, it often requires human expertise to address such issues effectively.

In summary, OptiMUS is effective at handling simpler problems with minimal human intervention but struggles with more complex problems where human expertise is required to resolve both algorithmic and formulation errors. This highlights the limitations of purely automated systems in dealing with highly complex optimization tasks and underscores the importance of human oversight in such scenarios.