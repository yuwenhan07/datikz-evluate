To solve the problem, we need to understand the context and the notation used. The notation \(x_3 \in X_3\) suggests that \(x_3\) is an element of a set \(X_3\). The phrase "is the a middle vertex of some \(P_5(u_1, u_2)\)" indicates that \(x_3\) is a specific vertex in a path \(P_5(u_1, u_2)\), which is a path graph with 5 vertices.

A path graph \(P_n\) with \(n\) vertices has vertices labeled \(v_1, v_2, \ldots, v_n\) such that there is an edge between \(v_i\) and \(v_{i+1}\) for \(i = 1, 2, \ldots, n-1\). For \(P_5(u_1, u_2)\), the vertices are \(u_1, u_2, x_3, u_3, u_4\).

The middle vertex of a path graph with an odd number of vertices (like \(P_5\)) is the vertex that is exactly halfway through the sequence of vertices. In this case, since there are 5 vertices, the middle vertex is the third vertex in the sequence. Therefore, \(x_3\) must be the middle vertex of the path \(P_5(u_1, u_2)\).

So, the answer is:

\[
\boxed{x_3}
\]