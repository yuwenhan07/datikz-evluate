To solve this problem, we need to understand the sets and their intersections as described in the problem statement. Let's break it down step by step.

1. **Identify the Sets:**
   - \( X_0 = N(x) \cap A_0 \): This set contains all vertices that are neighbors of \( x \) and also belong to \( A_0 \).
   - \( X_1 = N(x) \cap (A_1 \setminus A_0) \): This set contains all vertices that are neighbors of \( x \) but do not belong to \( A_0 \) and belong to \( A_1 \).

2. **Red Pair of Vertices:**
   - The red pair of vertices has two common neighbors in \( S(x, 2) \). This means there are at least two vertices in \( S(x, 2) \) that are neighbors of both vertices in the red pair.
   - These vertices are in \( M' \), which is a specific subset of vertices that satisfy this condition.

3. **Yellow Pair of Vertices:**
   - The yellow pair of vertices has exactly one common neighbor in \( S(x, 2) \). This means there is exactly one vertex in \( S(x, 2) \) that is a neighbor of both vertices in the yellow pair.
   - These vertices are not in \( M' \).

Given the problem statement, we need to determine what the red and yellow pairs of vertices represent in terms of their positions relative to each other and the sets \( X_0 \) and \( X_1 \).

Since the red pair has two common neighbors in \( S(x, 2) \) and the yellow pair has only one common neighbor in \( S(x, 2) \), the red pair must be in \( X_0 \) because they have two common neighbors, which implies they are in \( A_0 \). The yellow pair, on the other hand, does not have two common neighbors, so they cannot be in \( X_0 \) and must be in \( X_1 \).

Therefore, the red pair of vertices is in \( X_0 \) and the yellow pair of vertices is in \( X_1 \).

The final answer is:
\[
\boxed{X_0 \text{ and } X_1}
\]