The text you provided seems to be discussing a specific scenario in a mathematical or computational context, likely related to graph theory or a similar field where "red" and "blue" sites represent different states or configurations. Here's a breakdown of the key points:

1. **Protection of the Origin**: This refers to a strategy to protect a central point (the origin) from being "occupied" by red sites. The protection mechanism is described in Lemma~\ref{protection lemma}, which outlines how this protection can be achieved.

2. **Initial Configuration**: Red sites are initially confined to a specific area that is shaded red in the figure or diagram. This shaded area represents the initial state where red sites are located.

3. **Blocking Configurations**: Blue rectangles represent the first two layers of successful blocking configurations. Each blue rectangle has a blue site within its activation region, and a blue line traverses through these rectangles horizontally until it encounters a red site. This indicates a sequence of steps where blue sites are strategically placed to block the spread of red sites.

4. **Distance Considerations**:
   - The closest vertical lines containing red points are at a distance proportional to \( \frac{1}{p} \) from the y-axis.
   - The vertices of the cones are at a distance proportional to \( \frac{1}{\sqrt{p}} \) from the x-axis. These distances suggest a geometric arrangement where the red points and the cones are distributed in a way that influences the blocking strategy.

In summary, the text describes a method for protecting the origin from red sites using a series of blue blocking configurations. The blue rectangles and their positions relative to the red sites and cones provide a structured approach to achieve this protection, with specific distance constraints that guide the placement of blue sites.