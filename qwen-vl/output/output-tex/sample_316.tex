The scenario you've described involves the analysis of singular free boundary points in the context of partial differential equations, particularly those related to free boundary problems. Let's break down the key components:

### Key Concepts:
1. **Free Boundary Points**: These are points where the solution \( u \) of a given PDE changes its behavior, such as from being smooth to having a discontinuity or a singularity.
2. **Quadratic Growth**: This refers to the rate at which the function \( u - \psi \) grows near the free boundary. If the growth is quadratic, it means that the difference between \( u \) and a certain function \( \psi \) behaves like \( |x|^2 \) near the boundary.
3. **Red Arrows**: These indicate the direction in which the quadratic growth is expected to occur. This is typically along the normal direction to the free boundary.
4. **Degeneracy**: The case on the right is described as more degenerate compared to the case on the left. Degeneracy here likely refers to the complexity or the nature of the free boundary, which can affect the accuracy of approximations.
5. **Dimension of Kernel (\(\dim \ker \mathbf{A}\))**: The dimension of the kernel of the operator \(\mathbf{A}\) is mentioned as 1. This could be related to the multiplicity of eigenvalues or the number of linearly independent solutions to the homogeneous equation associated with \(\mathbf{A}\).
6. **Discrete Free Boundary (\(\Gamma_{\text{Triang}}\))**: This is an approximation of the true free boundary \(\Gamma\), depicted using dashed brown lines. It is assumed to be at a distance \(\mathcal{O}(\delta(h)^{1/2})\) from the true boundary, indicating that the approximation error scales with the square root of the mesh size \(h\).

### Examples of Singular Free Boundary Points:
- **Elliptic Regularity Theory**: For elliptic PDEs, singular free boundary points often arise when the solution exhibits a sharp transition. For instance, in the obstacle problem, the free boundary is the set where the solution touches the obstacle. At these points, the solution may not be smooth, and the behavior of the solution near the boundary can be analyzed using techniques like blow-up arguments.
- **Quadratic Growth**: In the context of free boundary problems, if the solution \( u \) has a quadratic growth near the free boundary, it suggests that the solution is behaving like a parabolic function in the vicinity of the boundary. This type of growth is often observed in problems involving minimal surfaces or interfaces.

### Degeneracy and Approximation:
- **Degeneracy**: The more degenerate case on the right might involve a more complex geometry of the free boundary, making it harder to approximate accurately. This could be due to the presence of singularities, cusps, or other irregularities in the boundary.
- **Approximation Accuracy**: The quality of the approximation \(\Gamma_{\text{Triang}}\) depends on how well it captures the true shape of the free boundary \(\Gamma\). The fact that it is at a distance \(\mathcal{O}(\delta(h)^{1/2})\) suggests that the approximation improves as the mesh size \(h\) decreases, but the degeneracy of the boundary can still pose challenges.

### Conclusion:
In summary, the examples you've described are typical in the study of free boundary problems, where the behavior of the solution near the boundary is crucial. The quadratic growth and the degeneracy of the boundary are important features that influence the accuracy of numerical approximations. The dimension of the kernel of the operator \(\mathbf{A}\) being 1 indicates a specific type of linear dependence among the solutions, which can be relevant for understanding the structure of the free boundary.