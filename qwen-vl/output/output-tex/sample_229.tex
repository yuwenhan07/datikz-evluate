The schematic you've described outlines a computational setup for modeling the dynamics of a droplet in a fluid bath, where the droplet's motion is analyzed using two different coordinate systems: a fixed Cartesian system and a moving polar system.

### Fixed Cartesian System:
- **Origin**: The origin is placed directly beneath the south pole of the droplet.
- **Axes**: The axes are \( (x, z) \).
- **Purpose**: This system is used as a reference frame to describe the initial position and orientation of the droplet before any motion occurs. It helps in setting up the problem and defining the initial conditions.

### Moving Polar System:
- **Origin**: The origin is located at the center of mass of the droplet.
- **Axes**: The radial axis is \( r \), and the angular axis is \( \theta \).
- **Positioning**:
  - The radial axis \( r = R_0 \) corresponds to the radius at which the droplet's free surface intersects the plane of interest.
  - The angle \( \theta = 0 \) points vertically downwards, which is typically the direction of gravity or the vertical axis in this context.
- **Purpose**: This system is used to describe the motion of the droplet relative to its center of mass. It simplifies the analysis by aligning the coordinate system with the droplet's center of mass, making it easier to track the droplet's deformation and motion under external forces such as gravity, surface tension, and fluid flow.

### Summary:
- **Fixed Cartesian System**: Provides a global reference frame for the initial conditions and overall geometry.
- **Moving Polar System**: Provides a local reference frame that follows the droplet's center of mass, simplifying the description of its motion and deformation.

This dual-coordinate-system approach is common in computational fluid dynamics (CFD) simulations, particularly when dealing with droplets or bubbles in a fluid environment. It allows for a more accurate and efficient modeling of the droplet's behavior under various conditions.