The function you've provided is:

\[
\varphi_\Dt(x) = B_1 - \left(\frac{\diffc^2}{2} + B_2\right)x^{2(\alpha-1)} + \frac{b(0)}{x} + b(0)\left(\diffc^2 + B_2\right)x^{2\alpha-3}\Dt.
\]

To analyze the behavior of this function, we need to consider its components and how they interact with each other as \(x\) varies and \(\Dt\) changes.

### 1. **Constant Term \(B_1\)**:
This term is a constant and does not depend on \(x\) or \(\Dt\). It simply shifts the entire function up or down by \(B_1\).

### 2. **Quadratic Term \(-\left(\frac{\diffc^2}{2} + B_2\right)x^{2(\alpha-1)}\)**:
This term is a quadratic function in \(x\) with a negative coefficient. The exponent \(2(\alpha-1)\) determines the curvature of the parabola. If \(\alpha > 1\), the term will be a downward-opening parabola (since the exponent is positive but less than 2). If \(\alpha = 1\), it becomes a linear term, and if \(\alpha < 1\), it will be an upward-opening parabola (since the exponent is negative).

### 3. **Term \(\frac{b(0)}{x}\)**:
This term is inversely proportional to \(x\). As \(x\) increases, \(\frac{b(0)}{x}\) decreases, and as \(x\) approaches zero from the right, \(\frac{b(0)}{x}\) becomes very large (approaches infinity).

### 4. **Term \(b(0)\left(\diffc^2 + B_2\right)x^{2\alpha-3}\Dt\)**:
This term is a product of a constant \(b(0)\left(\diffc^2 + B_2\right)\) and a power-law term \(x^{2\alpha-3}\Dt\). The exponent \(2\alpha - 3\) determines the curvature of this term. If \(\alpha > 1.5\), the term will be a downward-opening parabola (since the exponent is positive but less than 2). If \(\alpha = 1.5\), it will be a linear term, and if \(\alpha < 1.5\), it will be an upward-opening parabola (since the exponent is negative).

### Summary of Behavior:
- **As \(x \to 0^+\)**: The term \(\frac{b(0)}{x}\) dominates, making \(\varphi_\Dt(x)\) approach \(\infty\) if \(b(0) > 0\).
- **As \(x \to \infty\)**: The term \(-\left(\frac{\diffc^2}{2} + B_2\right)x^{2(\alpha-1)}\) dominates if \(\alpha > 1\), making \(\varphi_\Dt(x)\) approach \(-\infty\).
- **At \(x = 0\)**: The function is undefined due to the term \(\frac{b(0)}{x}\).
- **Effect of \(\Dt\)**: The term \(b(0)\left(\diffc^2 + B_2\right)x^{2\alpha-3}\Dt\) scales the function by \(\Dt\), which can be used to adjust the overall magnitude of the function.

### Conclusion:
The behavior of \(\varphi_\Dt(x)\) depends critically on the values of \(\alpha\), \(\diffc\), \(B_2\), \(b(0)\), and \(\Dt\). For specific values of these parameters, the function may exhibit different asymptotic behaviors at \(x = 0\) and \(x \to \infty\). The term \(\frac{b(0)}{x}\) introduces a singularity at \(x = 0\), while the other terms determine the overall shape of the function for \(x > 0\).