To solve the problem, we need to understand the construction of the trajectories and the relationship between the points \( z_+ \) in \( Q^+ \), \( z_- \) in \( Q^- \), and the intermediate point \( z_0 \) in \( Q^0 \).

The curves \(\Gamma_+\) and \(\Gamma_-\) are defined as follows:
- \(\Gamma_+\) connects any point \( z_+ \in Q^+ \) to some intermediate point \( z_0 \in Q^0 \).
- \(\Gamma_-\) connects any point \( z_- \in Q^- \) to some intermediate point \( z_0 \in Q^0 \).

This means that for each point \( z_+ \) in the set \( Q^+ \), there is a unique path \(\Gamma_+\) that leads from \( z_+ \) to an intermediate point \( z_0 \) in \( Q^0 \). Similarly, for each point \( z_- \) in the set \( Q^- \), there is a unique path \(\Gamma_-\) that leads from \( z_- \) to an intermediate point \( z_0 \) in \( Q^0 \).

The key here is to recognize that the intermediate point \( z_0 \) is common to both paths \(\Gamma_+\) and \(\Gamma_-\). This implies that the construction of these trajectories involves a mapping or a function that associates each point in \( Q^+ \) with a specific point in \( Q^0 \) via \(\Gamma_+\), and similarly for points in \( Q^- \) via \(\Gamma_-\).

Thus, the construction of the trajectories involves the following steps:
1. Identify the set \( Q^+ \) and the set \( Q^- \).
2. For each point \( z_+ \in Q^+ \), find the corresponding point \( z_0 \in Q^0 \) such that the path \(\Gamma_+\) connects \( z_+ \) to \( z_0 \).
3. For each point \( z_- \in Q^- \), find the corresponding point \( z_0 \in Q^0 \) such that the path \(\Gamma_-\) connects \( z_- \) to \( z_0 \).

The final answer, summarizing the construction of the trajectories, is:

\[
\boxed{\text{The trajectory } \Gamma_+ \text{ connects any point } z_+ \in Q^+ \text{ to some intermediate point } z_0 \in Q^0, \text{ and the trajectory } \Gamma_- \text{ connects any point } z_- \in Q^- \text{ to some intermediate point } z_0 \in Q^0.}
\]