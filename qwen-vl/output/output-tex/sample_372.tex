To understand the general representation of induced noncausal dependence in a conditional graph, let's break down the concepts step by step.

### Conditional Graphs and Causal Links

A **conditional graph** is a directed acyclic graph (DAG) that represents the causal relationships between variables, but it is conditioned on some specific values of other variables. In such a graph, the edges represent direct causal influences from one variable to another.

### Collider Fan

A **collider** in a DAG is a node that has two or more incoming edges. When conditioning on a collider, the graph can induce noncausal dependencies among the variables that are not directly connected by an edge in the original graph.

The **collider fan** refers to the set of nodes that are descendants of the collider when considering all possible paths through the collider. These nodes are influenced by the collider in a way that can lead to noncausal dependencies.

### Example: Conditional Graph Given \( S = s \)

Let's consider the example you mentioned, which is the conditional graph given \( S = s \) based on the graph DM in Figure~\ref{fig:modifying-DAG}B. Here’s how we can represent this:

1. **Identify the Collider**: Suppose the collider in the graph is node \( C \). This means there are two incoming edges to \( C \), say from nodes \( A \) and \( B \).

2. **Conditioning on \( S = s \)**: When we condition on \( S = s \), we are essentially fixing the value of \( S \). This can change the structure of the graph and introduce new dependencies.

3. **Collider Fan**: The collider fan includes all nodes that are descendants of \( C \) when considering all possible paths through \( C \). For instance, if \( C \) has descendants \( D \) and \( E \), then the collider fan would include \( D \) and \( E \).

4. **Induced Noncausal Dependence**: Conditioning on \( S = s \) can create noncausal dependencies among the nodes in the collider fan. For example, if \( D \) and \( E \) were not directly connected before conditioning, they might become dependent due to the collider \( C \).

### Different Ways to Show the Collider Fan

There are different ways to show the collider fan in a conditional graph:

- **Marking the Collider**: Simply mark the collider \( C \) in the graph.
- **Drawing the Collider Fan**: Draw the nodes in the collider fan explicitly, showing the paths through the collider.
- **Using Conditional Independence Notation**: Use notation like \( D \perp E \mid S = s \) to indicate that \( D \) and \( E \) are conditionally independent given \( S = s \).

### Summary

In summary, the general representation of induced noncausal dependence in a conditional graph involves identifying the collider, marking it, and then showing the nodes in the collider fan. This helps in understanding how conditioning on certain variables can introduce new dependencies that are not present in the original graph.

If you have a specific figure or graph in mind, you can provide it, and I can give a more detailed analysis!