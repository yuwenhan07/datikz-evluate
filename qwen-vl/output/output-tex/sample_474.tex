The description you've provided seems to be related to a computational experiment involving quantum configurations and their minimal Hamming distances. Here's a breakdown of what this might entail:

1. **Quantum Configurations**: These are likely referring to specific states or configurations in a quantum system, possibly derived from symplectic polar spaces. Symplectic polar spaces are geometric structures that can be used to model certain types of quantum systems.

2. **Minimal Hamming Distances**: In the context of quantum computing, the Hamming distance is often used to measure the similarity or dissimilarity between two quantum states. The minimal Hamming distance refers to the smallest Hamming distance among all pairs of distinct states in a given set.

3. **Heuristic Method**: This suggests that some algorithmic approach is being used to compute these minimal Hamming distances, and it is optimized to run efficiently on multiple threads.

4. **200 Threads and 20 Cores**: This indicates that the computation is being parallelized across 20 cores, with each core potentially handling up to 10 threads (since 200 threads divided by 20 cores equals 10 threads per core).

5. **Intel(R) Core(TM) i7-12700H Processor**: This specifies the hardware platform, which is an Intel processor with 12 physical cores and 20 logical threads.

6. **Three- to Six-Qubit Symplectic Polar Spaces**: This specifies the range of qubit numbers (from 3 to 6) and the type of symplectic polar spaces being considered. These spaces are important in quantum information theory as they provide a framework for understanding and constructing quantum codes.

### Key Points:
- The experiment is likely aiming to understand how the minimal Hamming distances behave as the number of iterations increases.
- The use of a heuristic method and parallelization suggests an attempt to optimize performance for large-scale computations.
- The focus on three- to six-qubit symplectic polar spaces indicates a study of small to medium-sized quantum systems, which are often used as testbeds for quantum error correction and other quantum information processing tasks.

### Possible Interpretation:
The graph you're describing would show the minimal Hamming distances on the y-axis and the number of iterations on the x-axis. As the number of iterations increases, the minimal Hamming distances should ideally decrease, indicating better performance or more efficient error correction strategies. The fact that the computation is being done on a multi-threaded, multi-core processor suggests that the goal is to achieve high throughput and efficiency in processing these quantum configurations.

If you have access to the actual data or the graph itself, you could provide more specific insights into the trends observed during the iterations.