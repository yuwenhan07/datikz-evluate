To provide a clear understanding of the potential function \(\phi_t\) defined in the context you've described, let's break down the example step by step.

### Definitions and Context

Given:
- \( k = 3 \)
- \( t = 8 \)
- \(\mathcal{R}_t = \{R_1, R_2, R_3, R_4, R_5\}\)

The potential function \(\phi_t(i, j)\) is defined based on the values of the elements in \(\mathcal{R}_t\). Specifically, it involves taking the maximum value among a subset of these elements, where the subset depends on the indices \(i\) and \(j\).

### Example Calculation

Let's calculate \(\phi_t(2, 0)\) and \(\phi_t(2, 3)\) as per the given example:

#### Step 1: Calculate \(\phi_t(2, 0)\)

According to the definition, \(\phi_t(2, 0)\) is calculated using the max operator over the set \(\{R_1, R_3\}\):

\[
\phi_t(2, 0) = \max_{R \in \{R_1, R_3\}} R
\]

From the figure (which is not provided here but assumed to show the values of \(R_1\) and \(R_3\)), let's assume the following values:
- \(R_1 = 4\)
- \(R_3 = 6\)

Thus,

\[
\phi_t(2, 0) = \max(4, 6) = 6
\]

So, \(\phi_t(2, 0) = 6\).

#### Step 2: Calculate \(\phi_t(2, 3)\)

Similarly, \(\phi_t(2, 3)\) is calculated using the max operator over the set \(\{R_1, R_4\}\):

\[
\phi_t(2, 3) = \max_{R \in \{R_1, R_4\}} R
\]

Again, from the figure (assuming the values of \(R_1\) and \(R_4\)):
- \(R_1 = 4\)
- \(R_4 = 3\)

Thus,

\[
\phi_t(2, 3) = \max(4, 3) = 4
\]

So, \(\phi_t(2, 3) = 4\).

### Summary

- For \(\phi_t(2, 0)\), the maximum value among \(R_1\) and \(R_3\) is 6, so \(\phi_t(2, 0) = 6\).
- For \(\phi_t(2, 3)\), the maximum value among \(R_1\) and \(R_4\) is 4, so \(\phi_t(2, 3) = 4\).

This example illustrates how the potential function \(\phi_t(i, j)\) is computed by taking the maximum value from a specific subset of the elements in \(\mathcal{R}_t\), depending on the indices \(i\) and \(j\).