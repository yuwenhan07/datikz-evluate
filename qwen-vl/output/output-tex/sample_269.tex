To understand the semilattice of idempotents of \( FI_1 \), we need to first clarify what \( FI_1 \) is and then identify its idempotents. Here, \( FI_1 \) typically refers to the free inverse monoid on one generator, which is the set of all finite sequences of elements from a set with inverses, including the empty sequence.

The free inverse monoid \( FI_1 \) can be generated by a single element, say \( u \). An idempotent in a monoid is an element that, when multiplied by itself, gives itself. In the context of \( FI_1 \), this means that if \( e \) is an idempotent, then \( e^2 = e \).

Given \( u = (-a, 3, b) \), we need to find the idempotents in the submonoid generated by \( u \), denoted as \( \langle u \rangle \). The idempotents in \( \langle u \rangle \) are the elements that, when squared, give themselves.

Let's denote the idempotents of \( \langle u \rangle \) by \( e_1, e_2, \ldots \). Since \( u \) is an idempotent in \( \langle u \rangle \), we have \( u^2 = u \). We need to determine other possible idempotents in \( \langle u \rangle \).

One important property of idempotents in monoids is that they form a semilattice under the operation of multiplication. This means that for any two idempotents \( e_i \) and \( e_j \), their product \( e_i e_j \) is also an idempotent, and the set of idempotents forms a lattice under this operation.

In the case of \( \langle u \rangle \), the idempotents are particularly simple because \( u \) itself is an idempotent. Therefore, the only idempotent in \( \langle u \rangle \) is \( u \) itself. This is because any other element in \( \langle u \rangle \) would not satisfy the idempotent condition unless it is a power of \( u \), but since \( u \) is already an idempotent, no other idempotent exists in \( \langle u \rangle \).

Thus, the semilattice of idempotents of \( \langle u \rangle \) is simply the set containing the single element \( u \).

The answer is:
\[
\boxed{\{u\}}
\]