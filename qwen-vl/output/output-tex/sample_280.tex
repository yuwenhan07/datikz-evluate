To understand the resolvability conditions in \( T^3 \) (the three-dimensional torus) depending on the distance \( d_T(u, v) \), we need to consider how points \( u \) and \( v \) can be resolved by other points in the torus. Specifically, we need to determine the positions of point \( x \) such that \( d_T(x, u) < d_T(x, v) \).

### Step-by-Step Explanation:

1. **Understanding \( T^3 \)**:
   - \( T^3 \) is a three-dimensional torus, which can be visualized as a cube with opposite faces identified.
   - The distance \( d_T(u, v) \) between two points \( u \) and \( v \) in \( T^3 \) is the shortest path along the edges of the torus.

2. **Resolvability Condition**:
   - A point \( x \) resolves \( u \) and \( v \) if \( d_T(x, u) < d_T(x, v) \).
   - This means that \( x \) must lie closer to \( u \) than to \( v \).

3. **Possible Positions of \( x \)**:
   - To find all possible positions of \( x \) that resolve \( u \) and \( v \), we need to consider the geometry of the torus.
   - In \( T^3 \), the distance between any two points \( u \) and \( v \) can vary from 0 to 6 (since the torus is a cube with side length 1 and opposite faces identified).

4. **Visualizing the Torus**:
   - Imagine a cube where each face is a square of side length 1, and opposite faces are identified.
   - The distance \( d_T(u, v) \) is the minimum number of steps required to move from \( u \) to \( v \) along the edges of this cube.

5. **Resolving Points**:
   - For a given \( d_T(u, v) \), the set of points \( x \) that resolve \( u \) and \( v \) forms a region in the torus.
   - This region can be visualized as a "strip" or a "band" around the line segment connecting \( u \) and \( v \) in the torus.

6. **Red Vertices**:
   - The red vertices in the figure represent the positions of \( x \) that resolve \( u \) and \( v \).
   - These vertices are typically located at the midpoints of the edges of the cube, forming a grid-like pattern.

### Conclusion:

The resolvability condition in \( T^3 \) depends on the distance \( d_T(u, v) \). For a given \( d_T(u, v) \), the set of points \( x \) that resolve \( u \) and \( v \) forms a region in the torus. The red vertices in the figure depict these positions, showing the possible locations of \( x \) that satisfy the resolvability condition.

\[
\boxed{\text{The red vertices in the figure depict the positions of } x \text{ that resolve } u \text{ and } v \text{ in the torus } T^3.}
\]