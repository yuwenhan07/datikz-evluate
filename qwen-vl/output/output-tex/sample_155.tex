It seems like you're describing a process involving Euclidean and Lorentzian geometries in the context of theoretical physics, possibly related to quantum field theory or statistical mechanics. Here's a breakdown of what you've described:

1. **Euclidean Geometry**: This is a type of space-time that is often used in quantum field theory and statistical mechanics. It is a space where time is treated on equal footing with space, and it is typically used for path integrals and the study of quantum systems at finite temperature.

2. **Lorentzian Evolution**: This refers to the evolution of a system in Minkowski space-time, which is the space-time framework of special relativity. In this context, the evolution is typically described using the Schrödinger equation or the Heisenberg picture.

3. **Deformed Thermofield Initial State**: This likely refers to a specific initial state used in the context of thermofield dynamics, a formalism that allows one to describe quantum systems at finite temperature by considering two copies of the system in antipodal points of a circle.

4. **Normalization Factor \( Z \)**: This is a common term in statistical mechanics and quantum field theory, representing the partition function or the normalization constant of a system. It is crucial for computing thermal expectation values (vevs).

5. **Thermal Expectation Value (vev)**: This is the average value of an observable in a thermal state. It is computed by inserting the operator \( O(t) \) into the thermal state and then taking the trace over all possible states.

The left figure appears to show the lower half of the Euclidean geometry, which is used to generate the deformed thermofield initial state. The right figure shows the full Euclidean evolution, which can be used to compute the normalization factor \( Z \) or the thermal expectation value of operators with an appropriate insertion of the operator \( O(t) \).

If you have any specific questions about these concepts or need further clarification, feel free to ask!