To understand the concept of well-splitting in the context you've described, let's break it down step by step.

### Definitions and Context

1. **Red Points**: These are labeled as \( A - B \). This means that these points belong to set \( A \) but do not belong to set \( B \).
2. **Blue Points**: These are labeled as \( B - A \). This means that these points belong to set \( B \) but do not belong to set \( A \).
3. **Violet Points**: These are labeled as \( A \cap B \). This means that these points belong to both sets \( A \) and \( B \).

### Well-Splitting

In the context of set theory and topology, a well-splitting is a way to partition a set into subsets such that each subset can be mapped to another set in a specific manner. The well-splitting you're referring to seems to be related to the structure of a graph and the semi-coarse space induced by this graph.

### Semi-Coarse Space

A semi-coarse space is a topological space where the coarse structure is defined by a family of coverings. In simpler terms, it's a way to describe how the space is divided into regions or neighborhoods.

### Graph Induced Semi-Coarse Space

When we talk about a graph-induced semi-coarse space, we are considering a graph where the vertices represent points in the space, and the edges represent some form of adjacency or relationship between these points. The semi-coarse space is then constructed based on this graph.

### Example

Let's consider a simple example to illustrate this:

- Suppose we have two sets \( A = \{a_1, a_2, a_3\} \) and \( B = \{b_1, b_2, b_3\} \).
- The red points \( A - B \) could be \( \{a_1, a_2\} \), meaning \( a_1 \) and \( a_2 \) are in \( A \) but not in \( B \).
- The blue points \( B - A \) could be \( \{b_1, b_2\} \), meaning \( b_1 \) and \( b_2 \) are in \( B \) but not in \( A \).
- The violet points \( A \cap B \) could be \( \{a_3\} \), meaning \( a_3 \) is in both \( A \) and \( B \).

### Graph Representation

Now, let's assume we have a graph where:
- \( a_1 \) and \( a_2 \) are connected to each other.
- \( b_1 \) and \( b_2 \) are connected to each other.
- \( a_3 \) is connected to \( b_3 \).

The semi-coarse space induced by this graph would be a way to describe the neighborhoods or regions around these points based on their connections.

### Conclusion

In summary, the well-splitting you're referring to involves partitioning the points into three subsets based on their membership in \( A \) and \( B \). The semi-coarse space induced by the graph provides a structured way to describe the relationships between these points, which can be visualized through the graph's connections.

If you have a specific graph or more details about the sets \( A \) and \( B \), I can provide a more detailed example or further explanation.