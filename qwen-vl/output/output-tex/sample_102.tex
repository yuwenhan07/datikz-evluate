It seems like you're discussing a comparison between two approaches in the context of cardiovascular disease (CVD) modeling. Here's a breakdown of what you've mentioned:

1. **Models Trained on All Features**: This suggests that the machine learning or statistical models used for CVD prediction are trained using a comprehensive set of features. These features could include various clinical, demographic, lifestyle, and genetic factors that might influence the risk of developing cardiovascular diseases.

2. **DCE Optimization on Age, Weight, and Height**: DCE stands for Data-Driven Cardiovascular Epidemiology, which is a method that uses data-driven approaches to optimize cardiovascular health metrics. In this case, the optimization process is being performed only on three specific features: age, weight, and height. These are likely chosen because they are fundamental demographic characteristics that can significantly impact cardiovascular health.

### Key Points:
- **Comprehensive vs. Limited Features**: The first approach considers a wide range of features, which can provide a more nuanced understanding of the risk factors involved in CVD. However, it may also be computationally intensive and complex.
- **Focused Optimization**: The second approach focuses on optimizing just age, weight, and height. This simplification can make the model easier to interpret and potentially more efficient in terms of computational resources. However, it might miss out on other important features that could contribute to the prediction of CVD.

### Potential Implications:
- **Model Complexity and Interpretability**: A model trained on all features might be more complex but also more interpretable if the features are well-understood. On the other hand, a model optimized only on age, weight, and height might be simpler but less comprehensive in capturing all relevant risk factors.
- **Performance**: The performance of both models would need to be evaluated on a validation dataset to determine which one performs better in predicting cardiovascular outcomes.

In summary, the choice between these two approaches depends on the specific goals of the study, the available computational resources, and the trade-off between model complexity and interpretability.