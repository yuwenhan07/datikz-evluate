To understand the three cases described for how the volume of the unbounded component restricted to a cube can change, let's break down each scenario:

### Case (i): Topology Change within the Cube
In this case, the topology within the cube \( B_w \) changes. This means that the structure of the unbounded component within \( B_w \) is altered, possibly by the introduction or removal of a hole or a new path that connects different parts of the unbounded component. The volume of the unbounded component can change significantly due to these topological changes.

The volume of the unbounded component is bounded by 1 in this case. This implies that the maximum possible change in the volume of the unbounded component, considering the topological changes, is limited to a value of 1. This could be due to the fact that the volume of the unbounded component is constrained by the volume of the cube itself, or it could be a theoretical upper bound based on some specific mathematical framework.

### Case (ii): No Topological Change within the Cube
Here, there is no topological change within the cube \( B_w \) or in any cube that is connected to \( B_w \) by a finite excursion component. This means that the structure of the unbounded component within \( B_w \) remains unchanged. The volume of the unbounded component can only change if there is a change in the volume of the solid grey region in the figure.

The volume of the unbounded component is bounded by the volume of the solid grey region in the figure. This suggests that the maximum possible change in the volume of the unbounded component, considering the constraints imposed by the grey region, is equal to the volume of the grey region itself. This could be because the grey region represents a fixed volume that the unbounded component cannot exceed, either due to physical constraints or theoretical limitations.

### Case (iii): Topological Change Outside the Cube
In this case, there is a topological change outside the cube \( B_w \) that affects which components inside \( B_w \) are part of the unbounded component. This means that the structure of the unbounded component within \( B_w \) can change due to the changes happening outside the cube. The volume of the unbounded component can change significantly depending on these external changes.

The volume of the unbounded component is again bounded by 1 in this case. Similar to case (i), this implies that the maximum possible change in the volume of the unbounded component, considering the topological changes outside the cube, is limited to a value of 1. This could be due to the same reasons as in case (i), such as the volume being constrained by the volume of the cube or a theoretical upper bound.

### Summary
- **Case (i)**: Topology change within the cube, bounded by 1.
- **Case (ii)**: No topological change within the cube, bounded by the volume of the solid grey region.
- **Case (iii)**: Topology change outside the cube, bounded by 1.

Each case describes a different scenario where the volume of the unbounded component within a cube can change, with the changes being bounded by specific values based on the nature of the topological changes involved.