To provide a geometrical visualization of the construction in the proof of Theorem \ref{thm:elliptic_decomposition} for the case where \( b_i \) is repelling and \( b_{i+1} \) is attracting, we need to understand the context of the theorem and the dynamics involved.

### Context and Setup

Theorem \ref{thm:elliptic_decomposition} typically deals with the decomposition of a dynamical system or a map into simpler components, often involving elliptic points (fixed points with eigenvalues on the unit circle). In this specific case, we are considering a sequence of points \( b_1, b_2, \ldots, b_n \) where \( b_i \) is repelling and \( b_{i+1} \) is attracting.

### Geometrical Interpretation

1. **Repelling Fixed Point (\( b_i \)):**
   - A repelling fixed point means that the derivative of the map at \( b_i \) has an absolute value greater than 1. Geometrically, this implies that the map stretches the space around \( b_i \), moving points away from it.
   - Think of \( b_i \) as a "source" where trajectories move outward.

2. **Attracting Fixed Point (\( b_{i+1} \)):**
   - An attracting fixed point means that the derivative of the map at \( b_{i+1} \) has an absolute value less than 1. Geometrically, this implies that the map contracts the space around \( b_{i+1} \), moving points towards it.
   - Think of \( b_{i+1} \) as a "sink" where trajectories move inward.

### Construction and Visualization

To visualize the construction, consider the following steps:

1. **Initial Setup:**
   - Start with a dynamical system or a map \( f \) defined on some domain \( D \).
   - Identify the points \( b_1, b_2, \ldots, b_n \) such that \( b_i \) is repelling and \( b_{i+1} \) is attracting.

2. **Local Dynamics Near Repelling Points:**
   - Near each repelling fixed point \( b_i \), the map \( f \) will stretch the space. This can be visualized by drawing small circles around \( b_i \) and observing how they expand under iteration of \( f \).

3. **Local Dynamics Near Attracting Points:**
   - Near each attracting fixed point \( b_{i+1} \), the map \( f \) will contract the space. This can be visualized by drawing small circles around \( b_{i+1} \) and observing how they shrink under iteration of \( f \).

4. **Decomposition:**
   - The theorem decomposes the map \( f \) into simpler components by focusing on the behavior near these repelling and attracting points.
   - The repelling points act as sources, and the attracting points act as sinks. The map can be thought of as a series of local transformations around these points.

5. **Geometrical Representation:**
   - To represent this geometrically, you can draw a series of concentric circles around each \( b_i \) and \( b_{i+1} \). The circles around the repelling points will expand, while the circles around the attracting points will contract.
   - The map \( f \) can be visualized as a series of transformations that stretch and contract the space between these points.

### Conclusion

The geometrical visualization involves drawing concentric circles around the repelling and attracting points, showing how the space expands around the repelling points and contracts around the attracting points. This helps to understand the local dynamics and the overall structure of the map as described by Theorem \ref{thm:elliptic_decomposition}.

\[
\boxed{\text{The map decomposes into local transformations around repelling and attracting points, visualized by expanding and contracting circles around these points.}}
\]