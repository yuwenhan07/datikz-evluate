The diagram you're referring to likely illustrates the process of how individual preferences (constituent normal mixing distributions) combine to form the overall distribution of consumer preferences, which is then used in mixed logit models and the HAVAN model.

Here's a simplified explanation of the process:

1. **Individual Preferences (Constituent Distributions):**
   - Consumers have different preferences for various attributes or features of a product.
   - These preferences can be modeled using normal distributions, where each distribution represents the probability that a consumer prefers one attribute over another.
   - The parameters of these distributions (mean and standard deviation) capture the average preference and variability among consumers.

2. **Mixed Logit Models:**
   - Mixed logit models combine these individual distributions into a single model.
   - The model estimates the parameters of these constituent distributions (mean and standard deviation).
   - This allows the model to account for both the average preferences and the variability across consumers when predicting choices.

3. **Heterogeneous Aggregate Value Added Network (HAVAN) Model:**
   - The HAVAN model takes a different approach.
   - Instead of estimating parameters for individual distributions, it predicts the parameters for the aggregate distribution of preferences.
   - It does this based on model inputs such as market data, consumer characteristics, and product attributes.
   - The HAVAN model then uses these predicted parameters to simulate the overall distribution of consumer preferences.

### Diagram Representation:
- **Step 1:** Individual Consumers' Preferences (Normal Distributions)
  - Each consumer has their own normal distribution representing their preferences.
  - Parameters: Mean (average preference) and Standard Deviation (variability).

- **Step 2:** Mixed Logit Model
  - Combines the individual distributions into a single model.
  - Estimates the parameters (mean and standard deviation) for these distributions.
  - Predicts the choice probabilities based on these parameters.

- **Step 3:** HAVAN Model
  - Predicts the parameters for the aggregate distribution of preferences.
  - Uses inputs like market data, consumer characteristics, and product attributes.
  - Simulates the overall distribution of consumer preferences.

### Summary:
- **Mixed Logit Models** focus on estimating the parameters of individual distributions and combining them to predict consumer choices.
- **HAVAN Model** focuses on predicting the parameters of the aggregate distribution directly from model inputs, providing a more aggregated view of consumer preferences.

This distinction highlights the difference between individual-level and aggregate-level approaches in modeling consumer preferences.