To construct an IDM (Integer Decision Maker) instance corresponding to the Equal Cardinality Partition problem with \( k = 16 \), we need to translate the problem into a graph-theoretic representation where the vertices represent elements of the set \( S = \{a_1, \ldots, a_n\} \) and the edges represent relationships between these elements.

The Equal Cardinality Partition problem asks us to partition the set \( S \) into \( k \) subsets such that each subset has the same number of elements. In this case, since \( k = 16 \), we need to partition the set \( S \) into 16 subsets, each containing the same number of elements.

Here is a step-by-step construction of the IDM instance:

1. **Vertices**: Each vertex in the graph represents an element of the set \( S \). So, if \( S \) has \( n \) elements, there will be \( n \) vertices.
2. **Edges**: We need to add edges to ensure that each subset has exactly 16 elements. To do this, we can use a complete graph where every pair of vertices is connected by an edge. This ensures that any subset of 16 vertices will have all possible edges among them, which is useful for the IDM to make decisions based on the connectivity of the graph.

The complete graph \( K_n \) has \( \binom{n}{2} \) edges, but since we only need to ensure that any subset of 16 vertices is connected, we can simplify the construction by considering the structure of the complete graph and how it can be partitioned.

3. **Practically Infinite Bankrupting Debts**: In the context of the IDM, "practically infinite" bankrupting debts typically refer to edges that are not part of the solution but are used to ensure the graph's properties. For the Equal Cardinality Partition problem, we can consider adding edges that are not part of the solution but are necessary to maintain the connectivity required for the IDM to make decisions.

Given the above, the IDM instance can be represented as follows:
- **Vertices**: \( V = \{v_1, v_2, \ldots, v_n\} \)
- **Edges**: \( E = \{(v_i, v_j) \mid 1 \leq i < j \leq n\} \)

This forms a complete graph \( K_n \).

Thus, the IDM instance corresponding to the Equal Cardinality Partition problem with \( k = 16 \) is a complete graph \( K_n \) with \( n \) vertices and \( \binom{n}{2} \) edges.

The final answer is:
\[
\boxed{K_n}
\]