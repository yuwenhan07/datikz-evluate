To solve the problem, we need to understand what \( C'_{12} \) represents and the significance of the automorphism group \( Aut(C'_{12}) = \mathbb{Z}_3 \).

First, let's clarify what \( C'_{12} \) is. Typically, \( C'_{12} \) refers to the complement of the cycle graph \( C_{12} \), which is a 12-cycle graph. The complement of a graph \( G \) is a graph with the same vertex set as \( G \) but with edges between vertices that were not connected in \( G \). So, \( C'_{12} \) would be a graph on 12 vertices where two vertices are connected if they were not connected in the original 12-cycle graph.

The automorphism group of a graph is the set of all permutations of its vertices that preserve the edge structure of the graph. In this case, we are given that \( Aut(C'_{12}) = \mathbb{Z}_3 \), which means the automorphism group of \( C'_{12} \) is isomorphic to the cyclic group of order 3. This implies that there are exactly three automorphisms of \( C'_{12} \).

Now, let's consider the properties of \( C'_{12} \). Since \( C'_{12} \) is the complement of a 12-cycle graph, it is a 12-vertex graph where each vertex has degree 6 (since in a 12-cycle, each vertex has degree 2, and in the complement, each vertex will have degree 10). The automorphism group of \( C'_{12} \) being isomorphic to \( \mathbb{Z}_3 \) suggests that the graph has a high degree of symmetry, but only three specific symmetries.

Given the constraints and the nature of the problem, we can infer that \( C'_{12} \) must be a highly symmetric graph with a small number of automorphisms. One such graph that fits these criteria is the complement of the 12-cycle graph, which is known to have an automorphism group isomorphic to \( \mathbb{Z}_3 \).

Therefore, the graph \( C'_{12} \) is the complement of the 12-cycle graph, and the automorphism group of \( C'_{12} \) is indeed \( \mathbb{Z}_3 \).

The answer is:
\[
\boxed{C'_{12}}
\]