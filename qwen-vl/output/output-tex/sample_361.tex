To solve the problem, we need to determine the number of lattice points on the boundary of a path whose endpoint is \((4 \cdot 3, 4 \cdot 2) = (12, 8)\) and which contains \(g+1 = 5\) lattice points in total.

First, let's understand what it means for a path to have 5 lattice points. If a path has 5 lattice points, it must be a straight line segment connecting two lattice points because any other type of path would not necessarily have exactly 5 lattice points unless it is a degenerate case or a very specific configuration that we can't assume without more information. Therefore, the path is a straight line segment from \((0,0)\) to \((12,8)\).

The number of lattice points on a straight line segment between two lattice points \((x_1, y_1)\) and \((x_2, y_2)\) is given by \(|x_2 - x_1| + |y_2 - y_1| + 1\). Here, \((x_1, y_1) = (0,0)\) and \((x_2, y_2) = (12,8)\), so we have:
\[ |12 - 0| + |8 - 0| + 1 = 12 + 8 + 1 = 21. \]
However, this calculation is incorrect because it assumes the path goes through all intermediate points, but we only need to count the endpoints and the points on the line segment itself. The correct way to count the lattice points on the line segment is to use the formula for the greatest common divisor (gcd) of the differences in the coordinates. The number of lattice points on the line segment is given by:
\[ \text{gcd}(12, 8) + 1 = 4 + 1 = 5. \]
This confirms that the path indeed has 5 lattice points, which matches the given condition.

Therefore, the number of lattice points on the boundary of the path is \(\boxed{5}\).