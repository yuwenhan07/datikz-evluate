To solve this problem, let's break it down step by step.

1. **Understanding the Lattice Diagram**: The lattice diagram represents a two-dimensional grid where each vertex (or node) is connected to its nearest neighbors. In this context, the central vertex is surrounded by four other vertices, which we can label as \(A\), \(B\), \(C\), and \(D\) in a clockwise manner starting from the top.

2. **Flipping the Qubits**: When we flip the qubits of the four vertices surrounding the central vertex, we need to consider the effect on the phase difference. The phase difference is typically determined by the state of these qubits. Let's assume that the qubits at vertices \(A\), \(B\), \(C\), and \(D\) have states \(a\), \(b\), \(c\), and \(d\) respectively. Flipping these qubits means changing their states to \(\bar{a}\), \(\bar{b}\), \(\bar{c}\), and \(\bar{d}\).

3. **Phase Difference Calculation**: The phase difference is often calculated based on the product of the states of the qubits. For simplicity, let's assume the phase difference is given by the product of the states of the qubits around the central vertex. Before flipping, the phase difference is:
   \[
   \text{Phase Difference} = abc d
   \]
   After flipping, the phase difference becomes:
   \[
   \text{Phase Difference} = \bar{a} \bar{b} \bar{c} \bar{d}
   \]

4. **Simplifying the Phase Difference**: Using the property of complements (\(\bar{\bar{x}} = x\)), we can rewrite the flipped phase difference as:
   \[
   \bar{a} \bar{b} \bar{c} \bar{d} = \overline{abcd}
   \]
   This shows that flipping all the qubits around the central vertex results in the phase difference being the complement of the original phase difference.

Therefore, the final answer is:
\[
\boxed{\text{The phase difference is the complement of the original phase difference.}}
\]