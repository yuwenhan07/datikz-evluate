To understand the given statement about an \(\mathsf{FL}[3]\)-frame \(\frk{F}\), let's break it down step by step.

1. **\(\mathsf{FL}[3]\)-Frame**: This refers to a frame that satisfies the axioms of the logic \(\mathsf{FL}[3]\). The logic \(\mathsf{FL}[3]\) is a many-valued logic that extends the basic fuzzy logic \(\mathsf{FL}\) with additional connectives and axioms. It is often used in the context of fuzzy set theory and fuzzy logic.

2. **Relativization to Each Layer**: In the context of frames, relativization typically means considering the frame in relation to a specific subset or "layer" of its universe. For example, if \(\frk{F}\) is a frame over a universe \(U\), then relativizing \(\frk{F}\) to a subset \(A \subseteq U\) gives a new frame \(\frk{F}|_A\) over \(A\).

3. **Locally Finite \(\mathsf{S5}_2\)-Algebra**: A \(\mathsf{S5}_2\)-algebra is a structure that generalizes the notion of a Boolean algebra and is used in modal logic. Specifically, \(\mathsf{S5}_2\) algebras are those that satisfy the axioms of the modal logic \(\mathsf{S5}\) and have a two-element domain. Being "locally finite" means that every finite subalgebra of the algebra is finite.

4. **\(\frk{F}^*\)**: This notation is not standard in the literature on frames and modal logic. However, if we interpret it as the dual of \(\frk{F}\) (which is common in the context of Stone duality for Boolean algebras), then \(\frk{F}^*\) would be the dual space of \(\frk{F}\). If \(\frk{F}\) is a frame, then \(\frk{F}^*\) is a locale, and vice versa.

Given these points, the statement can be interpreted as follows:

- \(\frk{F}\) is a \(\mathsf{FL}[3]\)-frame.
- When \(\frk{F}\) is relativized to any layer (subset of its universe), the resulting structure is a locally finite \(\mathsf{S5}_2\)-algebra.
- However, when we consider the dual space \(\frk{F}^*\) (or some other interpretation of \(\frk{F}^*\)), this dual space is not locally finite.

This statement highlights a property of the frame \(\frk{F}\) where the behavior of the frame changes significantly when moving from the original frame to its dual. This kind of property is important in understanding the relationship between different logics and their corresponding algebraic structures.