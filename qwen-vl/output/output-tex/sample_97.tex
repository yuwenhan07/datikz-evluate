To understand the monoid \( M = \langle a, b \mid ba = aa \rangle \) and its properties, let's break down the information given:

1. **Monoid Definition**: The monoid \( M \) is defined by the generators \( a \) and \( b \) subject to the relation \( ba = aa \). This means that in any word formed using these generators, we can replace \( ba \) with \( aa \).

2. **Up-Hopfian Property**: The term "up-hopfian" refers to a property in algebraic structures where every homomorphism from the structure to itself is injective (one-to-one). In simpler terms, it means that there are no non-trivial homomorphisms from the structure to itself.

3. **Core \( M_2 \)**: The core \( M_2 \) is a specific substructure or part of the monoid \( M \). In this context, it seems to be related to the rank two case mentioned in \cref{thm:rank_two}.

Given these points, let's analyze the monoid \( M \):

### Step-by-Step Analysis

1. **Generators and Relations**:
   - Generators: \( a \) and \( b \)
   - Relation: \( ba = aa \)

2. **Simplification**:
   - Using the relation \( ba = aa \), we can simplify any word involving \( ba \) to \( aa \).
   - Therefore, any word in \( M \) can be reduced to a word consisting only of \( a \)'s and possibly some \( b \)'s, but with the constraint that \( ba \) can be replaced by \( aa \).

3. **Structure of Words**:
   - Any word in \( M \) can be written as a sequence of \( a \)'s and \( b \)'s.
   - Since \( ba = aa \), any occurrence of \( ba \) in a word can be replaced by \( aa \).

4. **Up-Hopfian Property**:
   - To determine if \( M \) is up-hopfian, we need to check if every homomorphism from \( M \) to itself is injective.
   - Consider a homomorphism \( \phi: M \to M \). If \( \phi \) is not injective, then there exist distinct elements \( x, y \in M \) such that \( \phi(x) = \phi(y) \).
   - Given the relation \( ba = aa \), any homomorphism must respect this relation. However, since \( M \) is generated by \( a \) and \( b \), and the relation \( ba = aa \) does not introduce new elements beyond those already present in words consisting of \( a \)'s and \( b \)'s, any homomorphism must map \( b \) to an element that respects the relation when combined with \( a \).
   - Since \( M \) is finite (it can be shown to be a finite monoid), any homomorphism from \( M \) to itself must be injective because there are no non-trivial relations that would allow for non-injective mappings.

5. **Core \( M_2 \)**:
   - The core \( M_2 \) likely refers to a specific submonoid or part of the monoid \( M \) that captures the essential structure of rank two.
   - For a monoid of rank two, the core typically involves the interaction between the two generators \( a \) and \( b \).

### Conclusion

The monoid \( M = \langle a, b \mid ba = aa \rangle \) is up-hopfian because any homomorphism from \( M \) to itself must be injective due to the finite nature of the monoid and the constraints imposed by the relation \( ba = aa \). The core \( M_2 \) likely refers to a specific submonoid that captures the essential structure of rank two, which in this case is the interaction between \( a \) and \( b \) under the given relation.

Thus, the monoid \( M \) is up-hopfian, and its core \( M_2 \) is a submonoid that reflects the rank two structure of the monoid.