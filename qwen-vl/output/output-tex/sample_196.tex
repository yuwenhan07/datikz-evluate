The statement you've provided is referring to a specific construction in string theory, particularly related to the gauge group \( SO(8) \). Let's break down the key points:

1. **Brane Construction**: In string theory, branes are higher-dimensional objects that can carry charges and can interact with other branes or strings. The brane construction refers to a way of building a compactification space from lower-dimensional branes.

2. **\( SO(8) \) Gauge Group**: This is an eight-dimensional orthogonal group, which plays a significant role in various string theory models, especially those involving M-theory or heterotic string theories.

3. **Toric Nodes**: Toric geometry is a powerful tool used in string theory to describe the compactification spaces. Toric nodes refer to singularities in the compactified space that can be resolved using toric geometry techniques.

4. **O\( 5^+ \) Branes**: These are special types of branes in M-theory that can be used to construct certain gauge groups. The notation \( O5^+ \) indicates a specific type of five-brane.

5. **Replacement of \( O5^+ \) Branes**: The statement mentions that in this particular construction, three nodes replace one \( O5^+ \) brane. This is likely a reference to a specific configuration where multiple nodes (which represent singularities) are used to replace a single \( O5^+ \) brane, simplifying the brane construction while maintaining the same gauge group structure.

6. **Reference**: The citation \(\texttt{Hayashi:2023boy}\) suggests that this construction is detailed in a paper by Hayashi et al., published in 2023. It would be beneficial to consult this paper for a more comprehensive understanding of the brane construction and its implications.

In summary, the statement describes a specific brane construction for the \( SO(8) \) gauge group, where the usual \( O5^+ \) brane is replaced by three nodes, as detailed in the referenced paper. This construction is likely part of a broader study in string theory, focusing on the properties and implications of such brane configurations.