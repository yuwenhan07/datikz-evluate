To understand \(\mathcal{C}_3(P_3)\), we need to first define what \(P_3\) is and then consider its 3-coloring graph.

1. **Definition of \(P_3\)**:
   \(P_3\) (or \(P_3(n)\)) is a path graph with 3 vertices. It can be visualized as a straight line with three vertices labeled \(v_1, v_2, v_3\) such that there is an edge between \(v_1\) and \(v_2\) and another edge between \(v_2\) and \(v_3\).

2. **3-Coloring of \(P_3\)**:
   A 3-coloring of a graph is an assignment of one of three colors to each vertex such that no two adjacent vertices share the same color. For \(P_3\), we need to find all possible ways to assign three colors (say, red, blue, and green) to the vertices \(v_1, v_2, v_3\) such that no two adjacent vertices have the same color.

3. **Counting the 3-colorings**:
   Let's denote the colors by \(R\) (red), \(B\) (blue), and \(G\) (green). We will count the number of valid colorings:
   - If \(v_1\) is colored \(R\), then \(v_2\) must be either \(B\) or \(G\).
     - If \(v_2\) is colored \(B\), then \(v_3\) must be \(G\).
     - If \(v_2\) is colored \(G\), then \(v_3\) must be \(B\).
   - Similarly, if \(v_1\) is colored \(B\), then \(v_2\) must be either \(R\) or \(G\).
     - If \(v_2\) is colored \(R\), then \(v_3\) must be \(G\).
     - If \(v_2\) is colored \(G\), then \(v_3\) must be \(R\).
   - If \(v_1\) is colored \(G\), then \(v_2\) must be either \(R\) or \(B\).
     - If \(v_2\) is colored \(R\), then \(v_3\) must be \(B\).
     - If \(v_2\) is colored \(B\), then \(v_3\) must be \(R\).

   Therefore, there are exactly 6 valid 3-colorings for \(P_3\):
   1. \(v_1 = R, v_2 = B, v_3 = G\)
   2. \(v_1 = R, v_2 = G, v_3 = B\)
   3. \(v_1 = B, v_2 = R, v_3 = G\)
   4. \(v_1 = B, v_2 = G, v_3 = R\)
   5. \(v_1 = G, v_2 = R, v_3 = B\)
   6. \(v_1 = G, v_2 = B, v_3 = R\)

4. **Graph Representation**:
   The 3-coloring graph \(\mathcal{C}_3(P_3)\) is a graph where each vertex represents a 3-coloring of \(P_3\), and edges connect colorings that differ in exactly one vertex's color. In this case, since there are only 6 colorings, \(\mathcal{C}_3(P_3)\) is a complete graph on 6 vertices, denoted as \(K_6\).

Thus, the 3-coloring graph \(\mathcal{C}_3(P_3)\) is a complete graph with 6 vertices, which is \(K_6\).

The final answer is:
\[
\boxed{K_6}
\]