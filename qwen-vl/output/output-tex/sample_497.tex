The notation \( G^a_g(B) \) seems to refer to a specific type of cubic graph, but the exact meaning of the parameters and the structure of the graph is not fully specified in your query. Let's break down the components:

1. **Cubic Graph**: A cubic graph is a graph where every vertex has exactly three edges incident to it. This means that each vertex in the graph has a degree of 3.

2. \( B = \{b, b_0, \dots, b_a\} \): This appears to be a set of vertices or nodes labeled from \( b \) to \( b_a \).

3. \( G^a_g(B) \): The notation here is somewhat ambiguous without more context. It could be referring to a specific construction or a particular type of graph with these parameters. For example:
   - \( g \) might denote a specific graph operation or transformation.
   - \( a \) could be an index or parameter related to the construction of the graph.
   - \( B \) is the set of vertices, and \( G \) is the graph itself.

### Possible Interpretations

Given the lack of explicit details, let's consider a few possible interpretations:

#### 1. **Graph Construction Based on Set \( B \)**:
If \( G^a_g(B) \) refers to a graph constructed based on the set \( B \), one common way to construct such a graph is to form a complete graph on the set \( B \). In a complete graph, every pair of distinct vertices is connected by a unique edge. Since \( B \) contains \( a+1 \) vertices, the complete graph \( K_{a+1} \) would have \( \binom{a+1}{2} = \frac{(a+1)a}{2} \) edges.

However, since we are dealing with a cubic graph (each vertex has degree 3), this interpretation does not directly apply because a complete graph \( K_{a+1} \) is not necessarily cubic unless \( a+1 \leq 4 \) (i.e., \( a \leq 3 \)).

#### 2. **Specific Graph Operation**:
If \( g \) represents a specific graph operation or transformation, then \( G^a_g(B) \) could describe a graph obtained by applying \( g \) to the set \( B \). For instance, if \( g \) is a function that constructs a cubic graph from a given set of vertices, then \( G^a_g(B) \) would be the resulting cubic graph.

#### 3. **Cubic Graph with Specific Properties**:
Another possibility is that \( G^a_g(B) \) describes a specific type of cubic graph with certain properties. For example, it could be a cubic graph with a particular number of vertices or edges, or it could be a graph with a specific symmetry or structure.

### Conclusion

Without additional context or a clear definition for \( g \) and \( a \), it is challenging to provide a precise description of the graph \( G^a_g(B) \). If you can provide more details about the construction or the specific properties you are interested in, I can offer a more accurate and detailed explanation.