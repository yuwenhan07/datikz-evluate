The iterative process to compute self-consistent solutions of the \textsc{SIESTA}-BdG (Band Density-Functional Theory) secular equation involves several key steps, including the mixing of the density matrix and the Hamiltonian. Here's a schematic overview of this process:

1. **Initialization**: Start with an initial guess for the density matrix \( \rho_0 \) and the Hamiltonian \( H_0 \).

2. **Self-Consistency Loop**:
   - **Density Update Step**:
     - Compute the Green's function \( G \) using the current density matrix \( \rho_n \).
     - Calculate the self-energy \( \Sigma \) from the Green's function.
     - Update the density matrix \( \rho_{n+1} \) by solving the Kohn-Sham equations with the updated self-energy.
   - **Hamiltonian Update Step**:
     - Use the updated density matrix \( \rho_{n+1} \) to construct the effective Hamiltonian \( H_{\text{eff}} \).
     - Solve the BdG (Bogoliubov-de Gennes) equations to obtain the eigenvalues and eigenvectors corresponding to the effective Hamiltonian.

3. **Convergence Check**:
   - Compare the new density matrix \( \rho_{n+1} \) with the previous one \( \rho_n \) to check for convergence.
   - If the change in the density matrix is below a certain threshold, the iteration is considered converged, and the final density matrix and Hamiltonian are obtained.
   - If not, repeat the self-consistency loop starting from the density update step.

4. **Output**:
   - Once convergence is achieved, output the final density matrix and the effective Hamiltonian.

### Detailed Steps

#### Density Update Step
- **Green's Function Calculation**:
  \[
  G(\mathbf{k}, \omega) = \left( \omega + \mu - \epsilon_{\mathbf{k}} - \Sigma(\mathbf{k}, \omega) \right)^{-1}
  \]
  where \( \mu \) is the chemical potential, \( \epsilon_{\mathbf{k}} \) is the kinetic energy, and \( \Sigma(\mathbf{k}, \omega) \) is the self-energy.

- **Self-Energy Calculation**:
  \[
  \Sigma(\mathbf{k}, \omega) = \sum_{\mathbf{k}'} V(\mathbf{k}, \mathbf{k}') G(\mathbf{k}', \omega)
  \]
  where \( V(\mathbf{k}, \mathbf{k}') \) is the interaction potential.

- **Density Matrix Update**:
  \[
  \rho_{n+1}(\mathbf{k}) = \frac{1}{\pi} \int d\omega \, \text{Im} \left[ G(\mathbf{k}, \omega) \right]
  \]

#### Hamiltonian Update Step
- **Effective Hamiltonian Construction**:
  The effective Hamiltonian \( H_{\text{eff}} \) is constructed based on the density matrix \( \rho_{n+1} \):
  \[
  H_{\text{eff}} = H_0 + \sum_{\mathbf{k}} \sum_{\sigma} \sum_{\mathbf{k}'} \sum_{\sigma'} \rho_{n+1}(\mathbf{k}) V(\mathbf{k}, \mathbf{k}') G(\mathbf{k}', \omega) \sigma' \otimes \sigma
  \]

- **BdG Equations Solution**:
  Solve the BdG equations:
  \[
  \begin{pmatrix}
    H_{\text{eff}} & \Delta \\
    \Delta^* & -H_{\text{eff}}
  \end{pmatrix}
  \begin{pmatrix}
    \psi_{\uparrow} \\
    \psi_{\downarrow}
  \end{pmatrix} = E
  \begin{pmatrix}
    \psi_{\uparrow} \\
    \psi_{\downarrow}
  \end{pmatrix}
  \]
  where \( \Delta \) is the pairing term representing the superconducting gap.

### Mixing
During the update steps, the density matrix and the Hamiltonian are mixed to ensure that the self-consistency is maintained. This can be done through various methods such as the Fermi golden rule or other mixing schemes designed to converge the solution efficiently.

By iterating through these steps, the \textsc{SIESTA}-BdG method aims to find the self-consistent solution that satisfies both the Kohn-Sham equations and the BdG equations simultaneously.