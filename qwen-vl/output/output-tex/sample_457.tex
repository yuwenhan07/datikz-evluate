The illustration you've described seems to be related to a clinical trial or an observational study where the treatment effect is being evaluated across different strata defined by the levels of two variables, \(x_{ij}\) and \(u_{ij}\). Here's a breakdown of what this might represent:

### Case Allocation and Treatment Effect

In a case²-study (also known as a case-control study), the goal is often to compare the treatment effect between two groups: those who received the treatment (\(T\)) and those who did not (\(C\)). The illustration appears to show how the cases are allocated based on their treatment status.

#### Left Panel: All Subjects Treated
- This panel shows the allocation of cases when all subjects are treated.
- The numbers \(h_{NN}, h_{MN}, h_{NM}, h_{MM}\) likely represent the expected values for the number of cases in each stratum:
  - \(h_{NN}\): Number of cases where both \(x_{ij}\) and \(u_{ij}\) are in the "N" (non-treatment) stratum.
  - \(h_{MN}\): Number of cases where \(x_{ij}\) is in the "M" (treatment) stratum but \(u_{ij}\) is in the "N" stratum.
  - \(h_{NM}\): Number of cases where \(x_{ij}\) is in the "N" stratum but \(u_{ij}\) is in the "M" stratum.
  - \(h_{MM}\): Number of cases where both \(x_{ij}\) and \(u_{ij}\) are in the "M" stratum.

#### Right Panel: All Subjects Untreated
- This panel shows the allocation of cases when all subjects are untreated.
- Similar to the left panel, these numbers represent the expected values for the number of cases in each stratum under the assumption that no treatment was given.

### Treatment Effect Metrics

The metrics \(\theta_{Tij}/\theta_{Cij}\) and \((1 - \theta_{C_{ij}})/(1 - \theta_{T_{ij}})\) are used to assess the treatment effect:
- \(\theta_{Tij}\) and \(\theta_{Cij}\) are the probabilities of a positive outcome for the treatment group and control group, respectively, for subject \(i\) in stratum \(j\).
- \(\theta_{Tij}/\theta_{Cij}\) measures the relative risk of the treatment compared to the control.
- \((1 - \theta_{C_{ij}})/(1 - \theta_{T_{ij}})\) measures the odds ratio of the treatment compared to the control.

### Interpretation

The illustration helps visualize how the treatment effect varies across different strata. By comparing the expected values in the left and right panels, one can infer whether the treatment has a differential effect depending on the levels of \(x_{ij}\) and \(u_{ij}\).

For example, if \(h_{MM}\) is significantly higher than \(h_{NN}\) in the left panel, it suggests that the treatment may have a stronger effect in the "M" stratum for both \(x_{ij}\) and \(u_{ij}\).

This type of analysis is crucial for understanding the heterogeneity of treatment effects and can inform further subgroup analyses or stratified randomization in future studies.