To solve this problem, we need to understand the context and the definitions provided. Let's break it down step by step.

1. **Understanding the Non-Symmetric Space**: The solid lines in the figure represent a non-symmetric space with \( N = 3 \) and \( L_* = 4 \). This means the space has 3 vertices and the longest path (in terms of the number of edges) is 4. The ordering of vertices is also given, but for the purpose of this problem, we don't need the specific ordering to determine if an edge is removable.

2. **Definition of Removable Edge**: An edge is removable if it can be added or removed without changing the conductive uniformity and the \( L_* \)-uniform scaling property. In other words, adding or removing such an edge should not affect the overall structure in a way that violates these properties.

3. **General Property of Removable Edges**: According to the problem statement, any edge which connects vertically aligned points is removable. This means that if two vertices are directly above or below each other, the edge connecting them can be added or removed without affecting the conductive uniformity and the \( L_* \)-uniform scaling property.

Given this information, we can conclude that any edge connecting vertically aligned points is removable. Therefore, the answer to the problem is:

\[
\boxed{\text{Any edge connecting vertically aligned points is removable.}}
\]