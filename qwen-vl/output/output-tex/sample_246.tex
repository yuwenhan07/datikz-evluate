To solve this problem, we need to understand the context and the relationships between the variables and the network's properties as described in the problem statement.

1. **Understanding the Network Size and Throughput**:
   - The network size is \( O(\log q) \). This means the network size grows logarithmically with respect to \( q \).
   - The maximum throughput of the network is given as \( \frac{29}{55} \).

2. **Scaling the Throughput**:
   - The throughput values shown on the figure should be scaled by a factor of \( \frac{64}{55} \) to get the value at equilibrium. This scaling factor is important for understanding the relationship between the throughput at different points in time or under different conditions.

3. **Binary Encoding**:
   - The binary encoding of \( p = 29 \) is \( 011101 \).
   - The binary encoding of \( q = 26 \) is \( 011010 \).
   - The binary encoding of \( 2^k - q = 9 \) is \( 001001 \).

4. **Network Representation**:
   - The figure (labeled as \cref{fig:reversed-optimal-capacity-network}(a)) provides an alternative representation of the same network. This suggests that the network can be visualized or analyzed differently but still represents the same underlying structure and properties.

Given these points, let's summarize the key information:

- The network size is \( O(\log q) \), where \( q = 26 \).
- The maximum throughput of the network is \( \frac{29}{55} \).
- The throughput values need to be scaled by \( \frac{64}{55} \) to find the equilibrium throughput.
- The binary encodings provided give us specific numerical values related to the parameters \( p \) and \( q \).

The final answer, based on the problem statement and the information provided, is:

\[
\boxed{\frac{29}{55}}
\]

This is the maximum throughput of the network before scaling. The scaling factor \( \frac{64}{55} \) would be used to find the equilibrium throughput if needed.