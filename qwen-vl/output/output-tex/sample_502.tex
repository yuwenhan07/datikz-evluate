To solve this problem, we need to understand what the graph is showing and what the eigenvalues and the consecutive numbers represent. However, since the problem does not provide specific details about the functions \(\mu_0(\alpha)\), \(\mu_2(\alpha)\), and \(\mu_3(\alpha)\) or the sequence \(s_n\), I will provide a general explanation based on typical scenarios that such graphs might represent.

### Step-by-Step Explanation:

1. **Eigenvalues \(\mu_0(\alpha)\), \(\mu_2(\alpha)\), and \(\mu_3(\alpha)\):**
   - These are typically the eigenvalues of some matrix or operator depending on the parameter \(\alpha\).
   - The function \(\mu_0(\alpha)\) could be the smallest eigenvalue, \(\mu_2(\alpha)\) could be the second smallest eigenvalue, and \(\mu_3(\alpha)\) could be the third smallest eigenvalue.
   - The graph would show how these eigenvalues change as \(\alpha\) varies. Typically, you would expect these eigenvalues to be real numbers and they could either increase or decrease with \(\alpha\).

2. **Consecutive Numbers \(s_{20} < s_{13} < s_{21} < s_{14}\):**
   - These are likely indices from some sequence or set of values.
   - The inequality \(s_{20} < s_{13} < s_{21} < s_{14}\) suggests that the values at these indices are in ascending order, but not necessarily in the order of their indices.

### General Graph Interpretation:
- If the eigenvalues \(\mu_0(\alpha)\), \(\mu_2(\alpha)\), and \(\mu_3(\alpha)\) are plotted against \(\alpha\), the graph would show three curves representing each eigenvalue.
- The consecutive numbers \(s_{20} < s_{13} < s_{21} < s_{14}\) suggest that the values at these indices are in ascending order, but not necessarily in the order of their indices. This could imply that the sequence \(s_n\) is not strictly increasing or decreasing, but rather has some other pattern.

### Final Answer:
Since the problem does not provide specific functions or sequences, the most accurate answer based on the information given is:

\[
\boxed{\text{The graph shows the eigenvalues } \mu_0(\alpha), \mu_2(\alpha), \text{ and } \mu_3(\alpha) \text{ changing as } \alpha \text{ varies, and the consecutive numbers } s_{20} < s_{13} < s_{21} < s_{14} \text{ are in ascending order but not necessarily in the order of their indices.}}
\]