To solve the problem, we need to understand the context and the given information. The problem involves a set of functions \( f \in \mathcal{I}_2 \) and a specific value of \( m = 7 \). The sequence \( |\lambda_f| \) for these functions is given, and we know that the maximum value in this sequence is 10, which corresponds to the function \( x_5x_6 \). Additionally, the sequence \( S_j \) is symmetric and given as \( 1,1,2,2,3,3,3,2,2,1,1 \).

Let's break down the problem step by step:

1. **Understanding the Sequence \( S_j \)**:
   The sequence \( S_j \) is symmetric and consists of the values 1, 1, 2, 2, 3, 3, 3, 2, 2, 1, 1. This means it has 11 terms, and the middle term (the 6th term) is 3. The sequence can be written as:
   \[
   S_j = [1, 1, 2, 2, 3, 3, 3, 2, 2, 1, 1]
   \]

2. **Interpreting the Maximum Value**:
   The maximum value in the sequence \( |\lambda_f| \) is given as 10, and it corresponds to the function \( x_5x_6 \). This means that the value of \( |\lambda_{x_5x_6}| \) is 10.

3. **Symmetry in the Sequence**:
   The symmetry in the sequence \( S_j \) implies that the sequence is mirrored around its middle term. This symmetry helps us understand the distribution of the values in the sequence.

4. **Conclusion**:
   The problem asks for the sequence \( S_j \), which we have already determined. The sequence is:
   \[
   S_j = [1, 1, 2, 2, 3, 3, 3, 2, 2, 1, 1]
   \]

Therefore, the final answer is:
\[
\boxed{[1, 1, 2, 2, 3, 3, 3, 2, 2, 1, 1]}
\]