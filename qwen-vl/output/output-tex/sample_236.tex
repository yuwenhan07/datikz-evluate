To provide a visual representation of the configuration \(\mathcal{C}\) from Example~\ref{expl:0111001001110010} and its associated Möbius function \(\hat{\mu}_{\mathbbm{B}_S,\mathcal{C}}\) beside each node, we need to first understand the structure of the configuration and then apply the Möbius function.

### Step 1: Understanding the Configuration \(\mathcal{C}\)
The configuration \(\mathcal{C}\) is given by the sequence \(0111001001110010\). This sequence represents a binary string where each '1' indicates a colored node in a graph. The length of this sequence is 14, which means there are 14 nodes in the graph.

### Step 2: Visual Representation of \(\mathcal{C}\)
We can represent the configuration as a sequence of nodes where each '1' is a colored node:
\[ \mathcal{C} = 0111001001110010 \]
This can be visualized as:
\[ \text{Node 1: } 0, \text{ Node 2: } 1, \text{ Node 3: } 1, \text{ Node 4: } 1, \text{ Node 5: } 0, \text{ Node 6: } 0, \text{ Node 7: } 1, \text{ Node 8: } 0, \text{ Node 9: } 0, \text{ Node 10: } 1, \text{ Node 11: } 1, \text{ Node 12: } 1, \text{ Node 13: } 0, \text{ Node 14: } 0 \]

### Step 3: Applying the Möbius Function
The Möbius function \(\hat{\mu}_{\mathbbm{B}_S,\mathcal{C}}\) is defined on the Boolean lattice \(\mathbbm{B}_S\) of subsets of the set \(S = \{1, 2, \ldots, 14\}\). The Möbius function \(\hat{\mu}(A)\) for a subset \(A \subseteq S\) is given by:
\[ \hat{\mu}(A) = (-1)^{|A|} \frac{|\mathcal{C}|!}{\prod_{i \in A} |\mathcal{C}|_i!} \]
where \(|\mathcal{C}|_i\) is the number of times the \(i\)-th element appears in \(\mathcal{C}\).

For simplicity, let's assume that the Möbius function values are already computed and provided for each subset of \(\mathcal{C}\). We will denote these values as \(\hat{\mu}_{\mathbbm{B}_S,\mathcal{C}}(A)\) for each subset \(A \subseteq S\).

### Step 4: Visual Representation with Möbius Function Values
We will place the Möbius function value beside each colored node (i.e., each '1' in the sequence). For example, if the Möbius function value for the subset corresponding to the first three nodes is \(-1\), we would write \(-1\) beside the third node.

Since the exact values of the Möbius function are not provided here, we will assume hypothetical values for illustration purposes. Let's say the values are as follows (these are just examples):
- \(\hat{\mu}_{\mathbbm{B}_S,\mathcal{C}}(\{1, 2, 3\}) = -1\)
- \(\hat{\mu}_{\mathbbm{B}_S,\mathcal{C}}(\{4, 5, 6\}) = 1\)
- \(\hat{\mu}_{\mathbbm{B}_S,\mathcal{C}}(\{7, 8, 9\}) = -1\)
- \(\hat{\mu}_{\mathbbm{B}_S,\mathcal{C}}(\{10, 11, 12\}) = 1\)
- \(\hat{\mu}_{\mathbbm{B}_S,\mathcal{C}}(\{13, 14\}) = -1\)

Thus, the visual representation would look like:
\[ \text{Node 1: } 0, \text{ Node 2: } -1, \text{ Node 3: } -1, \text{ Node 4: } 1, \text{ Node 5: } 0, \text{ Node 6: } 0, \text{ Node 7: } -1, \text{ Node 8: } 0, \text{ Node 9: } 0, \text{ Node 10: } 1, \text{ Node 11: } 1, \text{ Node 12: } 1, \text{ Node 13: } -1, \text{ Node 14: } 0 \]

This is the final visual representation of the configuration \(\mathcal{C}\) and its associated Möbius function values beside each node.