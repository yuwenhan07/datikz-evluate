The left-hand side of your description refers to the singularity structure of the complex projective plane \(\mathbb{CP}^2\) with \(N\) singularities, denoted as \(\mathbb{CP}^2_{\boldsymbol{N}}\). This structure is often studied in the context of string theory and algebraic geometry, particularly in relation to Calabi-Yau manifolds.

### Singularity Structure:
In this context, the \(S^5\) moment polytope is used to describe the singularity structure. The vertices and edges of the polytope represent different types of singularities, and the labels on these vertices and edges indicate the degree of the singularity. The grey labels on the facets of the polytope denote the different types of singularities that can be present in the manifold.

### Branching Structure:
On the right-hand side, you are referring to the branching structure of \(S^5_{\boldsymbol{\alpha}}\), which is related to the \(S^5\) moment polytope. Here, the branching structure describes how the manifold branches or splits at certain points, typically along the edges of the polytope.

- **Vertices**: These represent the different branch points.
- **Edges**: These represent the connections between branch points.
- **Facets**: These represent the different types of branching behavior.

The black labels on the vertices and edges denote the branch index, which indicates the nature of the branching at those points. The grey labels on the facets again denote the different types of branching structures that can occur.

### Summary:
- **Left-hand side**: Describes the singularity structure of \(\mathbb{CP}^2_{\boldsymbol{N}}\) using the \(S^5\) moment polytope, where vertices and edges have black labels indicating the degree of the singularity, and grey labels indicate the type of singularity.
- **Right-hand side**: Describes the branching structure of \(S^5_{\boldsymbol{\alpha}}\) using the \(S^5\) moment polytope, where vertices and edges have black labels indicating the branch index, and grey labels indicate the type of branching.

These descriptions are crucial for understanding the geometric and topological properties of these manifolds, especially in the context of string theory and M-theory.