To understand the graph \( E_{(m_1, m_2, m_3, m_4, m_5)} \) of the Cuntz pentagon \(\mathcal{P}_{(m_1, m_2, m_3, m_4, m_5)}\), we need to delve into the structure and properties of the Cuntz algebra and its associated graphs.

### Step-by-Step Explanation:

1. **Cuntz Algebra**:
   The Cuntz algebra \(\mathcal{O}_n\) is a unital \(C^*\)-algebra generated by \(n\) isometries \(S_1, S_2, \ldots, S_n\) satisfying the relations:
   \[
   S_i^* S_j = \delta_{ij} I \quad \text{and} \quad \sum_{i=1}^n S_i S_i^* = I.
   \]
   Here, \(I\) is the identity operator on the Hilbert space.

2. **Cuntz-Pimsner Algebra**:
   The Cuntz-Pimsner algebra \(\mathcal{P}_{(m_1, m_2, m_3, m_4, m_5)}\) is constructed from the Cuntz algebra \(\mathcal{O}_n\) and a sequence of positive integers \((m_1, m_2, m_3, m_4, m_5)\). This algebra is a generalization of the Cuntz algebra and is used to study higher-dimensional analogues of the Cuntz-Krieger algebras.

3. **Graph Representation**:
   The graph \( E_{(m_1, m_2, m_3, m_4, m_5)} \) is a directed graph that encodes the structure of the Cuntz-Pimsner algebra. The vertices of the graph correspond to the generators of the algebra, and the edges represent the relations between these generators.

4. **Vertices**:
   The vertices of the graph \( E_{(m_1, m_2, m_3, m_4, m_5)} \) are labeled by the indices \(1, 2, 3, 4, 5\).

5. **Edges**:
   The edges in the graph are determined by the relations in the Cuntz-Pimsner algebra. Specifically, there is an edge from vertex \(i\) to vertex \(j\) if and only if the generator \(S_i\) can be used to express the generator \(S_j\) through the relations involving the sequence \((m_1, m_2, m_3, m_4, m_5)\).

6. **Example**:
   For simplicity, let's consider a specific example where \(m_1 = m_2 = m_3 = m_4 = m_5 = 1\). In this case, the Cuntz-Pimsner algebra reduces to the Cuntz algebra \(\mathcal{O}_5\). The graph \( E_{(1, 1, 1, 1, 1)} \) would be a complete directed graph with 5 vertices, where each vertex has an edge to every other vertex. This is because any generator \(S_i\) can be expressed as a product of the generators \(S_j\) for \(j \neq i\) using the relations \(S_i^* S_j = \delta_{ij} I\).

### Final Answer:
The graph \( E_{(m_1, m_2, m_3, m_4, m_5)} \) of the Cuntz pentagon \(\mathcal{P}_{(m_1, m_2, m_3, m_4, m_5)}\) is a directed graph with vertices labeled by \(1, 2, 3, 4, 5\). There is an edge from vertex \(i\) to vertex \(j\) if and only if the generator \(S_i\) can be used to express the generator \(S_j\) through the relations involving the sequence \((m_1, m_2, m_3, m_4, m_5)\).

For the specific case where \(m_1 = m_2 = m_3 = m_4 = m_5 = 1\), the graph is a complete directed graph with 5 vertices, where each vertex has an edge to every other vertex.