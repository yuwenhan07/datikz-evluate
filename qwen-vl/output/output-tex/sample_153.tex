To understand the "partition of partial sums" and its conjugate in the context of the given partition \(\sum_{j=1}^k j \cdot z_{-i_j}\) where \(i_k > i_{k-1} > \cdots > i_1\), we need to break down the problem into manageable parts.

### Step 1: Understanding the Partition
The partition \(\sum_{j=1}^k j \cdot z_{-i_j}\) is a weighted sum where each term \(z_{-i_j}\) is multiplied by \(j\). The weights \(j\) are in increasing order, starting from 1 up to \(k\).

### Step 2: Definition of Partition of Partial Sums
The partition of partial sums for a sequence \(a_1, a_2, \ldots, a_n\) is a way to express the sum as a sum of smaller sums, where each smaller sum is a subset of the original sequence. For our specific case, we can think of it as partitioning the indices \(i_1, i_2, \ldots, i_k\) into subsets such that the sum of the weights (which are the indices themselves) over each subset equals a certain value.

### Step 3: Conjugate Partition
The conjugate partition of a partition \(\lambda = (\lambda_1, \lambda_2, \ldots, \lambda_m)\) is another partition \(\mu = (\mu_1, \mu_2, \ldots, \mu_n)\) such that the number of parts of size at least \(i\) in \(\lambda\) is equal to the number of parts of size at least \(i\) in \(\mu\). In other words, if \(\lambda\) has \(r_i\) parts of size at least \(i\), then \(\mu\) should have \(r_i\) parts of size at least \(i\).

### Step 4: Applying to Our Specific Case
For the partition \(\sum_{j=1}^k j \cdot z_{-i_j}\), the conjugate partition would be constructed by considering how many times each weight (index) appears in the original partition. Specifically, if the index \(i\) appears \(r_i\) times in the original partition, then in the conjugate partition, the index \(i\) will appear \(r_i\) times.

### Final Answer
The conjugate partition of the partition \(\sum_{j=1}^k j \cdot z_{-i_j}\) is the partition where each index \(i\) appears \(r_i\) times, where \(r_i\) is the number of times the index \(i\) appears in the original partition. This can be written as:
\[
\boxed{\sum_{j=1}^k r_j \cdot z_{-i_j}}
\]
where \(r_j\) is the number of times the index \(i_j\) appears in the original partition.