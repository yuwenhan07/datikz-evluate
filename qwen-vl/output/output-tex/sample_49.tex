To determine the parking outcome for the given preference list \((7,5,3,3,2)\), we need to follow these steps:

1. **Understand the Preference List**: Each number in the list represents a car and its preferred parking spot. For example, the first car (number 7) prefers to park at spot 7, the second car (number 5) prefers to park at spot 5, and so on.

2. **Initial Setup**: We start with an empty parking lot and cars lined up in the order of their original places in line. So, the initial setup looks like this:
   - Car 7 wants to park at spot 7.
   - Car 5 wants to park at spot 5.
   - Car 3 wants to park at spot 3.
   - Car 3 wants to park at spot 3.
   - Car 2 wants to park at spot 2.

3. **Parking Process**:
   - **Car 7**: It can park at spot 7 because it is the only spot available.
   - **Car 5**: It can park at spot 5 because it is the only spot available.
   - **Car 3**: It can park at spot 3 because it is the only spot available.
   - **Car 3**: It cannot park at spot 3 because it is already occupied by car 3. It will have to find another spot. Since all other spots are occupied, it will not be able to park.
   - **Car 2**: It can park at spot 2 because it is the only spot available.

4. **Final Outcome**: After parking, the parking lot will look like this:
   - Spot 1: Empty
   - Spot 2: Car 2
   - Spot 3: Car 3
   - Spot 4: Empty
   - Spot 5: Car 5
   - Spot 6: Empty
   - Spot 7: Car 7

So, the final parking outcome is:
\[
\boxed{(2, 3, \text{Empty}, \text{Empty}, 5, \text{Empty}, 7)}
\]