To solve the problem, we need to understand the structure and properties of the infinite decision tree and how it can be embedded into a \((3,3)\)-balancer.

1. **Understanding the Infinite Decision Tree:**
   The infinite decision tree for sampling uniformly over a three-element set \(\{a, b, c\}\) is a tree where each node has exactly three children. This tree can be visualized as follows:
   - The root node has three children labeled \(a\), \(b\), and \(c\).
   - Each of these nodes also has three children, and so on, infinitely.
   - The probability of reaching any leaf node (which corresponds to an element in \(\{a, b, c\}\)) is \(\frac{1}{3}\).

2. **Embedding into a \((3,3)\)-Balancer:**
   A \((3,3)\)-balancer is a network with three inputs and three outputs such that the sum of the probabilities of the paths leading to any output is at most 1 on any arc. This means that if we have three inputs and we want to distribute them to three outputs in a way that respects the uniform distribution, we can use a \((3,3)\)-balancer.

3. **Sum of Probabilities on Any Arc:**
   In the context of the infinite decision tree, the sum of the probabilities of the paths leading to any output (which corresponds to any element in \(\{a, b, c\}\)) is exactly 1. This is because each path from the root to a leaf node represents a unique outcome, and the total probability of all outcomes must sum to 1.

4. **Constructing the Balancer:**
   To embed the infinite decision tree into a \((3,3)\)-balancer, we can use the fact that the sum of the probabilities of the paths leading to any output is at most 1 on any arc. This property ensures that the network is a valid balancer. Specifically, we can use the fact that the sum of the probabilities of the paths leading to any output is exactly 1, which is a stronger condition than the requirement for a balancer.

Therefore, the infinite decision tree can be embedded into a \((3,3)\)-balancer, and the sum of the probabilities of the paths leading to any output is at most 1 on any arc, which shows that this network is indeed a simple balancer.

The final answer is:
\[
\boxed{\text{The infinite decision tree can be embedded into a } (3,3)\text{-balancer.}}
\]