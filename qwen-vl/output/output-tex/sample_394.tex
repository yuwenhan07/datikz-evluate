To understand the birational equivalence of \(\mathbb{P}(L \oplus \mathbb{C})\) to a weighted projective plane, we need to delve into some advanced concepts in algebraic geometry and orbifolds. Let's break it down step by step.

### Step 1: Understanding the Setup
- **\(\mathbb{P}(L \oplus \mathbb{C})\)**: This is the projectivization of the direct sum of a line bundle \(L\) over an orbi-sphere (which is a quotient space of a sphere by a finite group action) and the complex numbers \(\mathbb{C}\).
- **Weighted Projective Plane**: A weighted projective plane \(\mathbb{P}(a:b:c)\) is defined as the quotient of \(\mathbb{A}^2 \setminus \{0\}\) by the action of the group \((\mathbb{C}^*)^3\) where \((z_0, z_1, z_2) \mapsto (\lambda^{a} z_0, \lambda^{b} z_1, \lambda^{c} z_2)\).

### Step 2: Birational Equivalence
Two varieties are birationally equivalent if there exists a rational map between them that has a rational inverse. In other words, they have the same field of rational functions.

### Step 3: Constructing the Birational Map
The key idea here is to use the fact that the orbi-sphere can be resolved to a smooth manifold, and then the line bundle \(L\) can be lifted to this resolution. The birational map can be constructed using the blow-up process at the singular points of the orbi-sphere.

### Step 4: Blow-Up Process
Consider the orbi-sphere \(S\) with a finite group \(G\) acting on it. The blow-up of \(S\) at the singular points will resolve the singularities, resulting in a smooth manifold. Let's denote this resolution by \(S'\). Now, consider the line bundle \(L\) over \(S\). We can lift \(L\) to a line bundle \(L'\) over \(S'\).

### Step 5: Projectivization
The projectivization \(\mathbb{P}(L' \oplus \mathbb{C})\) over \(S'\) will be a variety that is birationally equivalent to the weighted projective plane \(\mathbb{P}(a:b:c)\) for some weights \(a, b, c\).

### Step 6: Determining the Weights
The weights \(a, b, c\) are determined by the action of the group \(G\) on the orbi-sphere. Specifically, the weights are related to the orders of the stabilizers of the points in the orbi-sphere. If the orbi-sphere has \(n\) singular points with stabilizer groups \(G_1, G_2, \ldots, G_n\), then the weights are given by the orders of these stabilizer groups.

### Conclusion
Thus, the birational equivalence of \(\mathbb{P}(L \oplus \mathbb{C})\) to a weighted projective plane \(\mathbb{P}(a:b:c)\) is established through the blow-up process at the singular points of the orbi-sphere, lifting the line bundle \(L\) to a line bundle over the resolution, and then taking the projectivization of the direct sum of this line bundle and the complex numbers.

The final answer is:
\[
\boxed{\mathbb{P}(L \oplus \mathbb{C}) \text{ is birationally equivalent to } \mathbb{P}(a:b:c)}
\]
where \(a, b, c\) are determined by the orders of the stabilizer groups of the singular points of the orbi-sphere.