To understand the birational equivalence between \(\left( \mathbb{C} P^1 \times \mathbb{C} P^1 \right) / \mathbb{Z}_c\) and a weighted projective plane, we need to delve into some advanced concepts in algebraic geometry. Let's break it down step by step.

### Step 1: Understanding the Quotient Space
The space \(\left( \mathbb{C} P^1 \times \mathbb{C} P^1 \right) / \mathbb{Z}_c\) is obtained by quotienting the product of two complex projective lines (each being \(\mathbb{C} P^1\)) by the cyclic group \(\mathbb{Z}_c\). This means that we identify points in \(\mathbb{C} P^1 \times \mathbb{C} P^1\) that differ by an element of \(\mathbb{Z}_c\).

### Step 2: Weighted Projective Plane
A weighted projective plane \(\mathbb{P}(a:b:c)\) is defined as the quotient of \(\mathbb{C}^3 - \{0\}\) by the action of \((z_0, z_1, z_2) \mapsto (\lambda^{a} z_0, \lambda^{b} z_1, \lambda^{c} z_2)\) for \(\lambda \in \mathbb{C}^*\), where \(a\), \(b\), and \(c\) are positive integers. The weights \(a\), \(b\), and \(c\) determine how the coordinates transform under the action of \(\mathbb{C}^*\).

### Step 3: Birational Equivalence
To show that \(\left( \mathbb{C} P^1 \times \mathbb{C} P^1 \right) / \mathbb{Z}_c\) is birationally equivalent to a weighted projective plane, we need to find a way to map this quotient space to a weighted projective plane in a birational manner.

#### Key Insight:
The quotient \(\left( \mathbb{C} P^1 \times \mathbb{C} P^1 \right) / \mathbb{Z}_c\) can be realized as a blow-up of a weighted projective plane at certain points. Specifically, if we consider the quotient of \(\mathbb{C}^2\) by the action of \(\mathbb{Z}_c\) given by \((x, y) \mapsto (\omega x, \omega y)\) where \(\omega = e^{2\pi i / c}\), then the quotient space is birationally equivalent to a weighted projective plane with specific weights.

#### Explicit Construction:
Consider the quotient \(\mathbb{C}^2 / \mathbb{Z}_c\). This quotient can be thought of as a blow-up of the weighted projective plane \(\mathbb{P}(1:1:1)\) at the origin. The blow-up introduces a exceptional divisor, which corresponds to the quotient space \(\left( \mathbb{C} P^1 \times \mathbb{C} P^1 \right) / \mathbb{Z}_c\).

### Conclusion:
Thus, the quotient space \(\left( \mathbb{C} P^1 \times \mathbb{C} P^1 \right) / \mathbb{Z}_c\) is birationally equivalent to a weighted projective plane. The specific weights depend on the order \(c\) of the cyclic group \(\mathbb{Z}_c\).

In summary, the birational equivalence can be stated as:
\[
\left( \mathbb{C} P^1 \times \mathbb{C} P^1 \right) / \mathbb{Z}_c \sim_{bir} \mathbb{P}(a:b:c)
\]
where the weights \(a\), \(b\), and \(c\) are determined by the order \(c\) of the cyclic group \(\mathbb{Z}_c\).

\boxed{\left( \mathbb{C} P^1 \times \mathbb{C} P^1 \right) / \mathbb{Z}_c \sim_{bir} \mathbb{P}(a:b:c)}