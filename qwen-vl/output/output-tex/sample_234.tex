To understand the Newton polygon of \( Q = P[r(x)] \) and some elements that appear in the proof of Lemma~\ref{le:sum-zero}, we need to delve into some concepts from algebraic geometry, specifically related to the Newton polygon and its applications.

### Newton Polygon

The Newton polygon of a polynomial \( f(x) \) with coefficients in a field \( K \) is a geometric object that encodes information about the valuations of the roots of \( f(x) \). It is constructed as follows:

1. **Valuation**: Consider a valuation \( v \) on \( K \). The valuation of a polynomial \( f(x) = a_n x^n + a_{n-1} x^{n-1} + \cdots + a_0 \) is defined by \( v(f(x)) = \min(v(a_n), v(a_{n-1}), \ldots, v(a_0)) \).

2. **Newton Polygon Construction**:
   - Plot the points \( (i, v(a_i)) \) for each coefficient \( a_i \) of the polynomial.
   - Connect these points with line segments to form a polygon.
   - The Newton polygon is the convex hull of these points.

For a polynomial \( Q = P[r(x)] \), where \( P(x) \) is another polynomial and \( r(x) \) is a rational function, the Newton polygon can be used to analyze the behavior of the roots of \( Q(x) \).

### Elements in the Proof of Lemma~\ref{le:sum-zero}

Let's assume Lemma~\ref{le:sum-zero} is about the sum of certain elements being zero. Without the exact statement of the lemma, I'll provide a general outline of how such a proof might involve the Newton polygon.

#### Step-by-Step Outline:

1. **Define the Polynomial and Rational Function**:
   - Let \( P(x) \) be a polynomial and \( r(x) \) be a rational function.
   - Define \( Q(x) = P[r(x)] \).

2. **Newton Polygon of \( P(x) \)**:
   - Compute the Newton polygon of \( P(x) \).
   - This polygon will give you information about the exponents and valuations of the coefficients of \( P(x) \).

3. **Newton Polygon of \( r(x) \)**:
   - Compute the Newton polygon of \( r(x) \).
   - This polygon will give you information about the exponents and valuations of the coefficients of \( r(x) \).

4. **Newton Polygon of \( Q(x) \)**:
   - Compute the Newton polygon of \( Q(x) = P[r(x)] \).
   - This polygon will give you information about the exponents and valuations of the coefficients of \( Q(x) \).

5. **Summing Elements**:
   - If the lemma involves summing certain elements of \( Q(x) \) and their sums being zero, you would use the properties of the Newton polygon to analyze the contributions of different terms.
   - For example, if the Newton polygon of \( Q(x) \) has specific vertices or slopes, it might indicate that certain terms cancel out in a way that their sum is zero.

6. **Conclusion**:
   - Based on the analysis of the Newton polygons and the specific structure of \( Q(x) \), conclude that the sum of certain elements is indeed zero.

### Example

Consider \( P(x) = x^2 + 2x + 1 \) and \( r(x) = \frac{1}{x} \). Then \( Q(x) = P\left(\frac{1}{x}\right) = \left(\frac{1}{x}\right)^2 + 2\left(\frac{1}{x}\right) + 1 = \frac{1}{x^2} + \frac{2}{x} + 1 \).

- The Newton polygon of \( P(x) = x^2 + 2x + 1 \) is a triangle with vertices at \( (0,0) \), \( (1,1) \), and \( (2,0) \).
- The Newton polygon of \( r(x) = \frac{1}{x} \) is a ray starting at \( (1,-1) \) and extending to infinity.
- The Newton polygon of \( Q(x) = \frac{1}{x^2} + \frac{2}{x} + 1 \) is a triangle with vertices at \( (0,0) \), \( (1,1) \), and \( (2,0) \).

In this case, the Newton polygon of \( Q(x) \) shows that the terms involving negative powers of \( x \) (like \( \frac{1}{x^2} \) and \( \frac{2}{x} \)) balance out with the constant term \( 1 \), leading to the sum being zero.

Thus, the proof of Lemma~\ref{le:sum-zero} would involve similar steps, using the properties of the Newton polygons to show that certain terms cancel out, resulting in the sum being zero.

\[
\boxed{\text{The proof would involve analyzing the Newton polygons of } P(x), r(x), \text{ and } Q(x) \text{ to show that certain terms cancel out, leading to the sum being zero.}}
\]