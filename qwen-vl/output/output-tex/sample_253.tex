To understand the problem, let's break it down step by step.

1. **Understanding the Graph Transformation:**
   - We start with a graph \(\mathcal{G}\).
   - We swap the sets \(B\) and \(U \setminus A\) to obtain a new graph \(\mathcal{G}'\).

2. **Identifying Independent Transversals:**
   - An independent transversal in a bipartite graph is a set of vertices such that each vertex in the set belongs to a different part of the bipartition.
   - Given the sets \(S_1 = \{u_0, u_1, u_2, u_3\}\) and \(S_2 = \{v_0, v_1, y_2, y_3\}\), we need to check if these sets are independent transversals in \(\mathcal{G}'\).

3. **Checking Independence:**
   - For \(S_1\) to be an independent transversal, no two vertices in \(S_1\) should be adjacent.
   - For \(S_2\) to be an independent transversal, no two vertices in \(S_2\) should be adjacent.

4. **Drawing Elements from \(M(S_1)\) and \(M(S_2)\):**
   - \(M(S_1)\) represents the maximum matching in the subgraph induced by \(S_1\).
   - \(M(S_2)\) represents the maximum matching in the subgraph induced by \(S_2\).
   - We need to draw one element from \(M(S_1)\) and one element from \(M(S_2)\) in \(\mathcal{G}\).

5. **Example Construction:**
   - Let's assume \(\mathcal{G}\) is a bipartite graph with parts \(A\) and \(B\), and after swapping \(B\) and \(U \setminus A\), we get \(\mathcal{G}'\).

6. **Constructing \(\mathcal{G}'\):**
   - Suppose \(A = \{a_0, a_1, a_2, a_3\}\) and \(B = \{b_0, b_1, b_2, b_3\}\).
   - After swapping, \(U \setminus A = \{c_0, c_1, c_2, c_3\}\) becomes \(B'\) and \(B\) becomes \(U \setminus A'\).

7. **Independent Transversals:**
   - If \(S_1 = \{u_0, u_1, u_2, u_3\}\) and \(S_2 = \{v_0, v_1, y_2, y_3\}\) are independent transversals in \(\mathcal{G}'\), then:
     - No two vertices in \(S_1\) are adjacent.
     - No two vertices in \(S_2\) are adjacent.

8. **Drawing Elements from \(M(S_1)\) and \(M(S_2)\):**
   - Suppose \(M(S_1)\) includes edges \((u_0, b_0)\), \((u_1, b_1)\), \((u_2, b_2)\), and \((u_3, b_3)\).
   - Suppose \(M(S_2)\) includes edges \((v_0, c_0)\), \((v_1, c_1)\), \((y_2, c_2)\), and \((y_3, c_3)\).

9. **Conclusion:**
   - The elements drawn from \(M(S_1)\) and \(M(S_2)\) in \(\mathcal{G}\) are \((u_0, b_0)\) and \((v_0, c_0)\).

Thus, the final answer is:

\[
\boxed{(u_0, b_0) \text{ and } (v_0, c_0)}
\]