The image you've described is related to quantum field theory, specifically focusing on the dressed propagator and its reduction to obtain the tree-level off-shell current. Let's break down the key points:

### Left: The Dressed Propagator \(\mathcal{D}^{p'p}_{N}\)

1. **Double Line Representation**: The double line in the dressed propagator symbolizes the non-perturbative treatment of the coupling between charged particles and the background. This means that the propagator includes all possible loop corrections due to the interaction with the background fields.

2. **Charged Particles and Background Interaction**: The double line indicates that the propagator accounts for the full strength of the interaction between the charged particles and the background fields, which is not just a perturbative expansion but rather a complete treatment of the interaction.

### Right: LSZ Reduction and Tree-Level Off-Shell Current \(\mathcal{A}^{p'p}_{N}\)

1. **LSZ Reduction**: The left side of the image shows the dressed propagator \(\mathcal{D}^{p'p}_{N}\). The LSZ (Lippmann-Schwinger-Kadanoff-Zwanziger) reduction is applied to this propagator on the matter lines. This process effectively reduces the propagator to a form that involves the tree-level off-shell current \(\mathcal{A}^{p'p}_{N}\).

2. **Tree-Level Off-Shell Current**: The tree-level off-shell current \(\mathcal{A}^{p'p}_{N}\) is obtained after applying the LSZ reduction. This current describes the interaction of the charged particles with the background fields at the tree level, where the photons (or other gauge bosons) are not necessarily on-shell.

3. **Photon Momenta**: In the context of the current \(\mathcal{A}^{p'p}_{N}\), the photon momenta \(k_i\) are not required to be on-shell. This is because the current is derived from the dressed propagator, which already includes the effects of the loop corrections.

4. **On-Shell and Transverse Conditions**: To compute the corresponding tree-level amplitude, one can set the photons to be on-shell and transverse. This means:
   - \(k_i^2 = 0\): The photon momentum squared is zero, indicating that the photon is on-shell.
   - \(\varepsilon_i \cdot k_i = 0\): The polarization vector \(\varepsilon_i\) is orthogonal to the photon momentum \(k_i\), ensuring that the photon is transverse.

### Summary

- The dressed propagator \(\mathcal{D}^{p'p}_{N}\) includes all loop corrections due to the interaction with the background fields.
- Applying the LSZ reduction to the matter lines of \(\mathcal{D}^{p'p}_{N}\) yields the tree-level off-shell current \(\mathcal{A}^{p'p}_{N}\).
- To compute the tree-level amplitude, one sets the photons to be on-shell and transverse, i.e., \(k_i^2 = 0\) and \(\varepsilon_i \cdot k_i = 0\).

This process allows for a systematic way to handle interactions in quantum field theory, particularly when dealing with non-perturbative effects and loop corrections.