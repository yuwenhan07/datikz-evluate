To understand the polytope \( P_1 \) corresponding to \(\mathbb{P}^2\) (the projective plane) with vertices \( v_1 = (1,0) \), \( v_2 = (0,1) \), and \( v_3 = (-1,-1) \), we need to consider the geometric and algebraic properties of this configuration.

### Step-by-Step Explanation:

1. **Vertices in Projective Space**:
   - The vertices \( v_1 = (1,0) \), \( v_2 = (0,1) \), and \( v_3 = (-1,-1) \) are points in the projective plane \(\mathbb{P}^2\). In projective coordinates, these points can be represented as homogeneous coordinates.
   - For example, \( v_1 = [1:0] \), \( v_2 = [0:1] \), and \( v_3 = [-1:-1] \).

2. **Polytope Definition**:
   - A polytope is a bounded region in Euclidean space that is the convex hull of a finite set of points.
   - The convex hull of the vertices \( v_1, v_2, v_3 \) in \(\mathbb{P}^2\) forms a triangle in the projective plane.

3. **Homogeneous Coordinates**:
   - To work with these points in a more familiar setting, we can convert them to Cartesian coordinates by choosing a suitable reference point. However, since we are dealing with projective geometry, it's important to note that the choice of coordinates does not affect the intrinsic properties of the polytope.

4. **Convex Hull**:
   - The convex hull of three non-collinear points in the projective plane is a triangle. This means that the polytope \( P_1 \) is a triangle in the projective plane.

5. **Projective Triangle**:
   - In projective geometry, a triangle is a fundamental object. The triangle formed by the vertices \( v_1, v_2, v_3 \) is a projective triangle, which is a closed figure in the projective plane.

### Conclusion:
The polytope \( P_1 \) corresponding to \(\mathbb{P}^2\) with vertices \( v_1 = (1,0) \), \( v_2 = (0,1) \), and \( v_3 = (-1,-1) \) is a projective triangle. It is the convex hull of these three points in the projective plane.

Thus, the final answer is:
\[
\boxed{\text{A projective triangle}}
\]