To provide a detailed explanation and visualization of the Laakso diamond space, we need to understand the construction process step by step.

### Step 1: Initial Square
Start with a square \( S_0 \) of side length 1. This is our initial shape.

### Step 2: Replacement Rule
The Laakso diamond space is constructed using a symmetric replacement rule. Here's how it works:

1. **Divide the Square**: Divide the square \( S_0 \) into four smaller squares of equal size. Each smaller square has a side length of \( \frac{1}{2} \).

2. **Remove the Center Square**: Remove the central square from each of these four smaller squares. This leaves us with eight smaller squares, each with a side length of \( \frac{1}{4} \).

3. **Replace the Removed Squares**: Replace each removed square with a smaller square of side length \( \frac{1}{4} \). This results in a total of 8 smaller squares, each with a side length of \( \frac{1}{4} \).

### Step 3: Iteration
Repeat this process for each of the resulting squares. At each iteration, divide each square into four smaller squares, remove the center square from each, and replace the removed square with a smaller square of half the side length.

### Visualization
Let's visualize the first two steps of the replacement rule:

#### Step 1:
- Start with a square \( S_0 \) of side length 1.
- Divide \( S_0 \) into 4 smaller squares, each of side length \( \frac{1}{2} \).
- Remove the center square from each of these 4 smaller squares.
- Replace each removed square with a smaller square of side length \( \frac{1}{4} \).

This results in 8 smaller squares, each with a side length of \( \frac{1}{4} \).

#### Step 2:
- Now, take each of the 8 squares from Step 1 and repeat the process.
- Divide each of these 8 squares into 4 smaller squares, each of side length \( \frac{1}{8} \).
- Remove the center square from each of these 32 smaller squares.
- Replace each removed square with a smaller square of side length \( \frac{1}{16} \).

This results in 32 smaller squares, each with a side length of \( \frac{1}{16} \).

### Final Note
The Laakso diamond space is the limit of this infinite process. It is a fractal space that is self-similar at all scales and has a non-trivial topological structure, including a non-empty interior and a non-empty boundary.

If you have access to a specific figure or diagram, you can refer to that for a visual representation of the steps described above. If not, the above explanation should help you understand the construction of the Laakso diamond space.