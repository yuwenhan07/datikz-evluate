To understand the given statement and its implications, let's break down the key components:

1. **Coupling Vectors \(\gamma_i\)**: These are the teal vectors in the figure, which represent the directions and magnitudes of the interactions or forces acting on a system.

2. **Convex Hull**: The orange filled shape represents the convex hull generated by these coupling vectors. A convex hull is the smallest convex set that contains all the points (vectors in this case).

3. **Distance from the Origin \(\gamma_\infty\)**: This is the purple vector, which represents the direction and magnitude of the vector that is farthest from the origin among the coupling vectors.

4. **Asymptotic Behavior**: The statement mentions "asymptotically," which implies that we are considering the behavior of the system as time goes to infinity.

5. **Minimal-Distance Convex Hull**: The minimal-distance convex hull refers to the convex hull that has the smallest possible distance from the origin. This is the convex hull that is closest to the origin among all possible convex hulls that can be formed by the coupling vectors.

6. **Late-Time Attractor**: In the context of dynamical systems, an attractor is a set of states (points in the phase space) toward which the system tends to evolve, for a wide variety of starting conditions of the system. The late-time attractor is the attractor that the system approaches as time goes to infinity.

The statement suggests that asymptotically, only the subset of potentials (which are represented by the coupling vectors) that generate the minimal-distance convex hull are relevant to determining the late-time attractor. This means that the dynamics of the system, particularly its long-term behavior, is determined by the directions and magnitudes of the coupling vectors that form the minimal-distance convex hull.

In summary, the key point is that the minimal-distance convex hull, which is the convex hull that is closest to the origin, is the most relevant for determining the late-time attractor of the system. This is because it captures the dominant interactions that drive the system towards its long-term behavior.

Thus, the answer to the question is:

\[
\boxed{\text{The subset of coupling vectors that generate the minimal-distance convex hull is relevant to determining the late-time attractor.}}
\]