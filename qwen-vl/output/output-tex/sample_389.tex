Questo sembra essere una descrizione di come si divide un intervallo in base al valore di \( x \) rispetto a \( \frac{b}{2} \). Ecco una spiegazione più dettagliata:

1. **Intervallo definito**: Supponiamo che stiamo considerando un intervallo relativo a \( b \), ad esempio \( [a, b] \).

2. **Caso 1: \( x \leq \frac{b}{2} \)**:
   - Questo caso si applica quando \( x \) è inferiore o uguale alla metà di \( b \).
   - In questo caso, \( x \) può assumere qualsiasi valore compreso tra \( a \) e \( \frac{b}{2} \).

3. **Caso 2: \( x > \frac{b}{2} \)**:
   - Questo caso si applica quando \( x \) è superiore alla metà di \( b \).
   - In questo caso, \( x \) può assumere qualsiasi valore compreso tra \( \frac{b}{2} \) e \( b \).

In sintesi, il punto di divisione è \( \frac{b}{2} \), e l'intervallo viene diviso in due parti:
- A sinistra di \( \frac{b}{2} \): \( x \leq \frac{b}{2} \)
- A destra di \( \frac{b}{2} \): \( x > \frac{b}{2} \)

Se stai lavorando con una funzione o un problema specifico, potrebbe essere utile specificare la funzione o il problema per fornire una risposta più precisa.