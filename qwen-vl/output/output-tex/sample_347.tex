It seems like you're describing a diagram or a visual representation of some kind of computational or logical process involving traces and rules. Let's break down the elements you've mentioned:

1. **$(n,a,b)$-resilient trace $\tau$**: This likely refers to a sequence of states or configurations that is resilient to certain types of perturbations or attacks up to a certain level $a$ and $b$. The notation suggests it might be related to formal verification or security protocols.

2. **$(n-1,a-d_i,b)$-resilient reaction trace $\tau'$**: This is similar to the first one but with a slightly different resilience parameter. It might represent a modified version of the original trace where the resilience has been adjusted by subtracting a value $d_i$ from the original resilience parameter $a$.

3. **Blue arrows**: These typically represent the application of system rules. In the context of formal methods or computer science, these could be transitions between states or steps in a computation.

4. **Red arrow**: This usually indicates an update rule application. This could be a specific operation or transformation applied to the current state or configuration.

5. **Green configurations**: These are labeled as "goal configurations." In the context of formal verification or logic, these might represent the desired or expected final states or outcomes of the process being described.

Given this information, it appears you're dealing with a diagram that illustrates a process involving multiple steps (traces), each step applying certain rules, and the process is designed to reach a specific set of configurations that are considered goals. The resilience parameters suggest that the process is robust against certain types of disturbances or errors, which is a common requirement in formal verification and security protocols.

If you have more details about the specific context or the purpose of this diagram, please provide them so I can give a more precise interpretation or explanation.