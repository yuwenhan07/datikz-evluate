This statement highlights an important concept in systems engineering and operations management, particularly in scenarios where there is a need to balance input and output capacities within a system. Let's break down the key points:

1. **Input Capacities and Effective Throughput:**
   - Increasing the input capacity (e.g., by adding more resources or increasing the speed of processing at the input) can indeed lead to increased throughput at the input stage.
   - However, if the system has limited processing capacity downstream, the increased input throughput might not translate into a proportional increase in overall system throughput. This is because the system might become bottlenecked at the point where the increased input capacity meets the limited processing capacity.
   - For example, consider a production line where the input rate increases but the subsequent stages (like assembly or testing) have a fixed capacity. The increased input rate will only be effectively utilized up to the point where the bottleneck capacity allows it.

2. **Output Capacities and Effective Throughput:**
   - Similarly, increasing the output capacity (e.g., by improving the efficiency of the final stages or adding more output channels) can increase the throughput at the output end.
   - But if the system is already constrained by the input capacity or other internal bottlenecks, increasing the output capacity alone won't necessarily increase the overall system throughput. The system might still be limited by the slower or less efficient parts of the process.
   - For instance, imagine a supply chain where the output capacity is increased but the input rate remains low. The increased output capacity won't help if the system is unable to receive and process the increased output efficiently due to insufficient input rates.

3. **Competition for Resources:**
   - When both input and output capacities are increased, there can be a situation where the system becomes more complex and potentially more prone to bottlenecks. If the increased input capacity leads to a higher demand on the system, and the output capacity is also increased, the system might experience a form of "resource competition" where the increased demands from both ends of the system can overwhelm the system's ability to manage these demands efficiently.
   - This can lead to situations where the effective throughput on some inputs or outputs is reduced because the system is struggling to handle the increased load without proper balancing and optimization.

In summary, while increasing input and output capacities can generally improve system performance, it's crucial to ensure that these changes are balanced and coordinated with the rest of the system. Proper capacity planning and optimization are essential to avoid bottlenecks and ensure that the system operates at its optimal throughput. This often involves analyzing the entire system flow, identifying bottlenecks, and making strategic decisions about where to allocate resources to achieve the best overall performance.