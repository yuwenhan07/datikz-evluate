The Interval Debt Model (IDM) is a mathematical framework used to model and analyze the dynamics of debt among nodes in a network over time. It is particularly useful in financial networks where debts have specific terms and conditions, such as interest rates, maturity dates, and payment schedules.

Let's consider a simple example to illustrate how the IDM works:

### Nodes and Assets:
- Node \( u \): \(\textup{\euro}30\)
- Node \( v \): \(\textup{\euro}40\)

### Directed Edges and Debts:
- There is a directed edge from \( u \) to \( v \) with a label indicating that \( u \) owes \( v \) \(\textup{\euro}20\) between time 1 and time 3.

### Time Periods:
We will consider three time periods: \( t_1 \), \( t_2 \), and \( t_3 \).

### Initial Conditions:
At time \( t_1 \):
- \( u \) has \(\textup{\euro}30\).
- \( v \) has \(\textup{\euro}40\).

### Debt Dynamics:
- At time \( t_1 \), \( u \) owes \( v \) \(\textup{\euro}20\) between \( t_1 \) and \( t_3 \).

### Time \( t_2 \):
- \( u \) still has \(\textup{\euro}30\).
- \( v \) still has \(\textup{\euro}40\).
- The debt of \(\textup{\euro}20\) from \( u \) to \( v \) remains unchanged.

### Time \( t_3 \):
- \( u \) still has \(\textup{\euro}30\).
- \( v \) still has \(\textup{\euro}40\).
- The debt of \(\textup{\euro}20\) from \( u \) to \( v \) remains unchanged.

### Summary:
In this simple example, the IDM shows that the debt between nodes \( u \) and \( v \) remains constant over the specified time period. The initial assets and the terms of the debt do not change during the given time intervals.

This basic example can be extended to more complex scenarios involving multiple nodes, varying debts, and different time periods, allowing for a detailed analysis of the financial dynamics within a network.