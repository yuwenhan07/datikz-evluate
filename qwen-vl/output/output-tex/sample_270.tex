The description you've provided refers to a visualization technique used in the study of random dynamical systems (RDS), particularly focusing on the forward dynamics and the associated Doeblin graph. Let's break down the key components:

### Key Components:
1. **Random Dynamical System (RDS):**
   - An RDS is a mathematical framework for describing the evolution of a system over time under the influence of random perturbations.
   - In this context, the system evolves according to a Markov chain with states in the state space \(\mathds{X} = \mathds{N}_0\) (the non-negative integers).

2. **Trajectories:**
   - The trajectories represent how the system evolves over time. Each trajectory starts at some initial state \(x\) and follows the rules of the RDS to move to subsequent states.

3. **Time Axis:**
   - The \(x\)-axis represents time, indicating the sequence of steps or iterations through which the system evolves.

4. **State Space:**
   - The \(y\)-axis represents the state space \(\mathds{X} = \mathds{N}_0\), which consists of all non-negative integers.

5. **Transition Arrows:**
   - An arrow from \((\theta_{-n} \omega, x)\) to \((\theta_{-n+1} \omega, y)\) signifies that the transformation \(\varphi_{\theta_{-n}\omega}^1\) maps the state \(x\) to the state \(y\). Here, \(\theta_{-n}\omega\) represents the state at time step \(-n\) under the measure \(\omega\).

6. **Thick Line:**
   - The thick line represents the value of \(m_n(\omega)\), which could be a measure of the probability or importance of the state at time step \(n\) under the measure \(\omega\).

7. **Doeblin Graph:**
   - The graph on the set of nodes \(\mathds{Z} \times \mathds{X}\) is known as the Doeblin graph. This graph is named after the mathematician David Doeblin, who made significant contributions to the theory of Markov chains and random processes.
   - The Doeblin graph provides a visual representation of the transition probabilities between states in the state space \(\mathds{X}\).

### Visualization:
- The visualization typically shows a series of arrows connecting nodes in the form \((\theta_{-n} \omega, x)\) to \((\theta_{-n+1} \omega, y)\), illustrating the evolution of the system over time.
- The thick line might highlight certain states or transitions that are of particular interest, such as those with higher probabilities or those that are critical to the long-term behavior of the system.

### Example Context:
The example likely refers to a specific instance where the RDS is defined by a Markov chain with states in \(\mathds{N}_0\). The Doeblin graph would then provide a detailed picture of how the system transitions between these states over time, with the thick line possibly indicating the most probable paths or states.

This type of visualization is crucial in understanding the long-term behavior of complex stochastic systems, such as those found in physics, biology, economics, and other fields where randomness plays a significant role in the evolution of systems.