The notation you've provided seems to be related to the concept of cyclic codes in coding theory, specifically Reed-Solomon codes. Here, \(\overrightarrow{C}(n; t, k)\) typically represents a cyclic code with length \(n\), dimension \(k\), and minimum distance \(t+1\). However, the specific values you've given (12, 1, 4) and (12, 1, 2) suggest that these might be parameters for two different cyclic codes.

Let's break it down:

1. The first broadcast \(f_1\) is on a cyclic code \(\overrightarrow{C}(12; 1, 4)\).
2. The second broadcast \(f_2\) is on a cyclic code \(\overrightarrow{C}(12; 1, 2)\).

In both cases, the length \(n\) of the code is 12. The parameter \(t\) represents the minimum distance of the code, which is the smallest number of positions in which any two distinct codewords differ. For the first code, \(t = 4 - 1 = 3\), meaning the minimum distance is 3. For the second code, \(t = 2 - 1 = 1\), meaning the minimum distance is 1.

These codes are independent in the sense that they operate on separate sets of data or messages, and their parameters do not directly interact with each other. They could be used in parallel or sequentially depending on the context of the broadcasting system.

If you need further clarification or have additional questions about these codes or their properties, feel free to ask!