To understand the given example and the notation used, let's break down the components step by step.

### Definitions:
1. **Tree Component \( T \)**: A tree is an undirected graph that is connected and has no cycles.
2. **Vertices**: The nodes in the tree.
3. **Branching Vertices**: These are vertices with degree greater than 2 (i.e., they have more than two edges connected to them).
4. **Red Vertices (\( D \setminus F \))**: These are vertices that are not part of the set \( F \) but are still red.
5. **Orange Vertices (\( F \))**: These are vertices that are part of the set \( F \) and are orange.
6. **Blue Vertices**: These are the branching vertices, which are blue by definition.

### Notation:
- \( \mathsf{br}_\psi(T) \): This represents the number of branching vertices in the tree \( T \).

### Given Information:
- The tree \( T \) has 4 branching vertices.
- The set \( D \setminus F \) contains 3 red vertices.
- The set \( F \) contains 1 orange vertex.
- By definition, \( \mathsf{br}_\psi(T) = 4 \).

### Calculation:
The formula provided is \( \mathsf{br}_\psi(T) = 1 + |D \setminus F| \).

Let's verify this:
- \( |D \setminus F| \) is the number of red vertices that are not in \( F \). According to the problem, \( |D \setminus F| = 3 \).
- Plugging this into the formula gives us:
  \[
  \mathsf{br}_\psi(T) = 1 + 3 = 4
  \]

Thus, the number of branching vertices \( \mathsf{br}_\psi(T) \) is indeed 4, which matches the given information.

### Conclusion:
The example correctly demonstrates that the number of branching vertices in the tree \( T \) is equal to 1 plus the number of red vertices that are not in \( F \), which is 4. Therefore, the statement is verified as correct.