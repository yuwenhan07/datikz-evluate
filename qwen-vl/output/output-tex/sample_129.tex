It seems like you're discussing concepts from representation theory and group theory, particularly focusing on the special unitary group \( SU \) and its associated Lie algebra \( \mathfrak{su} \). Let's break down the key points:

1. **Special Unitary Group (\( SU \))**:
   - The special unitary group \( SU(n) \) consists of \( n \times n \) unitary matrices with determinant 1.
   - These matrices preserve the inner product in \( \mathbb{C}^n \).

2. **Lie Algebra (\( \mathfrak{su} \))**:
   - The Lie algebra \( \mathfrak{su}(n) \) is the set of skew-Hermitian matrices (matrices that satisfy \( A^\dagger = -A \)).
   - The dimension of \( \mathfrak{su}(n) \) is \( n^2 - 1 \), since it has \( n^2 \) entries but the skew-Hermitian condition reduces the number of independent entries.

3. **Exponential Map**:
   - The exponential map \( \exp: \mathfrak{su}(n) \to SU(n) \) maps each element of the Lie algebra to an element of the group.
   - This map is well-defined for matrices in \( \mathfrak{su}(n) \) because they have purely imaginary eigenvalues, ensuring that their exponentials remain unitary.

4. **Invariance under SWAP**:
   - The term "SWAP" typically refers to the swap operation or permutation matrix in the context of quantum computing and group theory.
   - If we consider the group \( SU(n) \) and restrict its elements to be invariant under some specific permutation (like the SWAP operation), we might be looking at a subgroup of \( SU(n) \).
   - However, the standard definition of \( SU(n) \) does not inherently include such restrictions unless explicitly stated.

5. **Projection (\( \text{pi} \))**:
   - The notation \( \text{pi} \) suggests a projection or restriction operator.
   - In this context, \( \text{pi} \mathfrak{su} \) likely denotes the projection of the Lie algebra onto a subspace, possibly related to the SWAP invariance.
   - Similarly, \( \text{pi} SU \) would denote the projection of the group onto a subset that respects the SWAP invariance.

6. **Commuting Diagram**:
   - A commuting diagram in this context might represent the relationship between the group, its Lie algebra, and the projections.
   - For example, the diagram could show how the projection of the Lie algebra \( \text{pi} \mathfrak{su} \) generates the projection of the group \( \text{pi} SU \) via the exponential map.

Given these points, let's summarize the key steps:

- Start with the Lie algebra \( \mathfrak{su}(n) \).
- Apply the projection \( \text{pi} \) to obtain \( \text{pi} \mathfrak{su} \).
- Use the exponential map to generate \( \text{pi} SU \).
- Restrict the elements of \( SU(n) \) to be invariant under the SWAP operation to create \( \text{pi} SU \).

If you have a specific commuting diagram or further details, please provide them so I can give a more precise interpretation!