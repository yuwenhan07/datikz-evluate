To solve the problem of designing an automaton that recognizes strings over \(\{0,1\}^3\) which represent the binary addition of two numbers \(m\) and \(n\) modulo 2, we need to understand the structure of the input and the output.

The input string is a sequence of three columns, each representing a bit of the numbers \(m\) and \(n\). The output should be the result of the binary addition of these two numbers, also represented as a sequence of three bits.

Let's denote the input string as:
\[ \begin{pmatrix} x_1 & y_1 & z_1 \\ x_2 & y_2 & z_2 \\ x_3 & y_3 & z_3 \end{pmatrix} \]
where each row represents the bits of \(m\) and \(n\) respectively, and the third row represents the bits of \(m + n\).

The binary addition of two numbers \(m\) and \(n\) can be computed using the following rules:
- If both bits are the same (both 0 or both 1), the sum is that bit.
- If the bits are different (one 0 and one 1), the sum is 1.

We can use a finite state machine (FSM) to implement this logic. Here is a step-by-step construction of the FSM:

1. **States**: We need states to represent the different combinations of the bits from \(m\) and \(n\). There are \(2^3 = 8\) possible combinations of three bits, so we have 8 states. Let's label them as \(Q_0, Q_1, \ldots, Q_7\).

2. **Initial State**: The initial state is \(Q_0\), which corresponds to the state where both \(m\) and \(n\) are 0.

3. **Transition Function**: The transition function will depend on the current state and the input bit. For example, if we are in state \(Q_0\) and the input is \(x_1y_1z_1\), then the next state will be determined by the value of \(x_1y_1z_1\):
   - If \(x_1y_1z_1 = 000\), go to \(Q_0\).
   - If \(x_1y_1z_1 = 001\), go to \(Q_1\).
   - If \(x_1y_1z_1 = 010\), go to \(Q_2\).
   - If \(x_1y_1z_1 = 011\), go to \(Q_3\).
   - If \(x_1y_1z_1 = 100\), go to \(Q_4\).
   - If \(x_1y_1z_1 = 101\), go to \(Q_5\).
   - If \(x_1y_1z_1 = 110\), go to \(Q_6\).
   - If \(x_1y_1z_1 = 111\), go to \(Q_7\).

4. **Final States**: The final state is the state corresponding to the all-zero input, which is \(Q_0\). This is because the all-zero input corresponds to the case where both \(m\) and \(n\) are 0, and the sum is 0.

Here is the transition table for the FSM:

| Current State | Input \(x_1y_1z_1\) | Next State |
|---------------|----------------------|------------|
| \(Q_0\)       | 000                  | \(Q_0\)    |
| \(Q_0\)       | 001                  | \(Q_1\)    |
| \(Q_0\)       | 010                  | \(Q_2\)    |
| \(Q_0\)       | 011                  | \(Q_3\)    |
| \(Q_0\)       | 100                  | \(Q_4\)    |
| \(Q_0\)       | 101                  | \(Q_5\)    |
| \(Q_0\)       | 110                  | \(Q_6\)    |
| \(Q_0\)       | 111                  | \(Q_7\)    |

The FSM will accept the input if it ends up in state \(Q_0\). Therefore, the automaton will recognize the strings that represent the binary addition of two numbers \(m\) and \(n\) modulo 2.

The answer is:
\[
\boxed{\text{The automaton accepts the input if it ends up in state } Q_0.}
\]