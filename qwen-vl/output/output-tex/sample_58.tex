The figures you've described seem to be related to a specific optimization or machine learning context, possibly involving a genetic algorithm or another type of iterative optimization process. Let's break down what each part might represent:

### Part (a): Value of R-IGD

1. **R-IGD (Relative Inverted Generational Distance)**:
   - This metric is often used in evolutionary algorithms to evaluate the performance of a population after an iteration.
   - It measures how well the new generation of solutions (predictions) has improved upon the previous generation.
   - The value of R-IGD is typically calculated based on the distance between the current population and the best solutions found so far.

2. **Population from the Last Iteration**:
   - This refers to the set of solutions that have been generated at the end of the most recent iteration of the algorithm.

3. **Predictions Given by the Model**:
   - These are the solutions that the model predicts for the next iteration based on the current population and other factors like fitness values.

4. **Area within One Standard Deviation**:
   - This shaded area likely represents the spread or variability of the R-IGD values around the mean. It gives an idea of the consistency or stability of the R-IGD values across different iterations.

### Part (b): Variance of Decision Variables of the Predictions

1. **Decision Variables**:
   - These are the parameters or features that define the solutions being optimized. For example, if you're optimizing a neural network, these could be the weights and biases.

2. **Variance**:
   - The variance measures how much the decision variables fluctuate or vary from their mean value. A smaller variance indicates that the decision variables are more consistent or less variable.

3. **Convergence to a Small Area**:
   - As \(\gamma\) (which could be a parameter controlling the rate of convergence or some other factor) increases, the variance of the decision variables decreases. This suggests that the model's predictions become more stable and less variable as \(\gamma\) increases.

### Summary

- **Part (a)** shows the R-IGD values for the population and predictions, with a shaded area indicating the variability around the mean. This helps assess the improvement and stability of the solutions over iterations.
  
- **Part (b)** illustrates the variance of the decision variables in the predictions. As \(\gamma\) increases, the variance decreases, implying better convergence and more stable predictions.

These visualizations are useful for understanding the behavior of the optimization process and the effectiveness of the model in predicting solutions. If you need further clarification or have additional details about the context, feel free to ask!