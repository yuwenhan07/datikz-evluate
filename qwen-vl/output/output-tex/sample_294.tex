To understand why the condition \( g(\Gamma) \ge 5 \) is necessary in Proposition \ref{cota2}, we need to delve into the context of the proposition and the graph theory concepts involved.

### Step-by-Step Reasoning:

1. **Graph Theory Basics**:
   - \( g(\Gamma) \) typically refers to the genus of a graph \( \Gamma \). The genus of a graph is the minimum number of handles required on a surface (like a sphere, torus, etc.) to embed the graph without any edges crossing.
   - A graph with genus \( g \ge 5 \) means it cannot be embedded on a sphere or a torus but requires a more complex surface like a 5-holed torus or higher.

2. **Proposition \ref{cota2}**:
   - Proposition \ref{cota2} likely involves some form of upper bound or constraint on certain properties of the graph \( \Gamma \), such as its size, connectivity, or other structural properties.
   - The necessity of \( g(\Gamma) \ge 5 \) suggests that this condition is essential for the validity or correctness of the statement in the proposition.

3. **Necessity of the Condition**:
   - If \( g(\Gamma) < 5 \), then the graph can be embedded on a sphere or a torus. This might allow for different behaviors or properties that do not hold when \( g(\Gamma) \ge 5 \).
   - For example, if the proposition involves a property that is only valid for graphs requiring a more complex surface (genus 5 or higher), then \( g(\Gamma) < 5 \) would violate this property.

4. **Example**:
   - Consider a specific property that Proposition \ref{cota2} might be stating, such as the maximum degree of vertices in the graph. If the proposition states that the maximum degree of vertices in \( \Gamma \) is bounded by some function of the genus, then for \( g(\Gamma) < 5 \), this bound might not hold, whereas for \( g(\Gamma) \ge 5 \), the bound does hold.

### Conclusion:
The condition \( g(\Gamma) \ge 5 \) is necessary in Proposition \ref{cota2} because it ensures that the graph \( \Gamma \) requires a more complex surface for embedding, which is a prerequisite for the specific property or behavior described in the proposition to hold true.

Thus, the final answer is:

\[
\boxed{g(\Gamma) \ge 5}
\]