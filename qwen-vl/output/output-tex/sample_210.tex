It seems like you've provided an expression that appears to be related to the concept of partitions in combinatorics or number theory, but it's not entirely clear what specific problem or context this expression is meant to represent. Let's break down and analyze the given expression step by step:

The expression is:
\[
\sum_{k=1}^{R} (\beta_k + m_{k-1} + 1)(R - k + 1) - R
\]

### Step-by-Step Analysis:

1. **Understanding the Summation:**
   The summation is over \( k \) from 1 to \( R \). For each \( k \), we evaluate the term inside the sum:
   \[
   (\beta_k + m_{k-1} + 1)(R - k + 1)
   \]
   Here, \( \beta_k \) and \( m_{k-1} \) are some sequences or variables that depend on \( k \).

2. **Breaking Down the Term:**
   The term inside the summation can be expanded as:
   \[
   (\beta_k + m_{k-1} + 1)(R - k + 1) = \beta_k (R - k + 1) + m_{k-1} (R - k + 1) + (R - k + 1)
   \]
   This can be further simplified to:
   \[
   \beta_k (R - k + 1) + m_{k-1} (R - k + 1) + (R - k + 1)
   \]

3. **Summing Over \( k \):**
   When we sum this expression from \( k = 1 \) to \( k = R \), we get:
   \[
   \sum_{k=1}^{R} \left( \beta_k (R - k + 1) + m_{k-1} (R - k + 1) + (R - k + 1) \right)
   \]
   This can be split into three separate sums:
   \[
   \sum_{k=1}^{R} \beta_k (R - k + 1) + \sum_{k=1}^{R} m_{k-1} (R - k + 1) + \sum_{k=1}^{R} (R - k + 1)
   \]

4. **Simplifying Each Sum:**
   - The first sum is:
     \[
     \sum_{k=1}^{R} \beta_k (R - k + 1)
     \]
     This can be rewritten as:
     \[
     \sum_{k=1}^{R} \beta_k (R - k + 1) = \sum_{k=1}^{R} \beta_k (R - k + 1)
     \]
   - The second sum is:
     \[
     \sum_{k=1}^{R} m_{k-1} (R - k + 1)
     \]
     This can be rewritten as:
     \[
     \sum_{k=1}^{R} m_{k-1} (R - k + 1) = \sum_{k=0}^{R-1} m_k (R - (k+1) + 1) = \sum_{k=0}^{R-1} m_k (R - k)
     \]
   - The third sum is:
     \[
     \sum_{k=1}^{R} (R - k + 1)
     \]
     This can be simplified as:
     \[
     \sum_{k=1}^{R} (R - k + 1) = \sum_{k=1}^{R} R - \sum_{k=1}^{R} k + \sum_{k=1}^{R} 1
     \]
     Which further simplifies to:
     \[
     R \cdot R - \frac{R(R+1)}{2} + R = R^2 - \frac{R^2 + R}{2} + R = \frac{2R^2 - R^2 - R + 2R}{2} = \frac{R^2 + R}{2}
     \]

5. **Combining All Parts:**
   Putting it all together, we have:
   \[
   \sum_{k=1}^{R} \beta_k (R - k + 1) + \sum_{k=0}^{R-1} m_k (R - k) + \frac{R^2 + R}{2}
   \]

6. **Final Expression:**
   Subtracting \( R \) from the entire expression:
   \[
   \sum_{k=1}^{R} \beta_k (R - k + 1) + \sum_{k=0}^{R-1} m_k (R - k) + \frac{R^2 + R}{2} - R
   \]

This is the final form of the expression after breaking it down and simplifying. If you have more context or specific values for \( \beta_k \) and \( m_{k-1} \), please provide them so I can help further with numerical evaluation or specific applications.