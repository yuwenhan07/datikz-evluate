To solve this problem, we need to understand the relationship between the original graphs \( G_1 \) and \( G_2 \), their complements \( G_1^c \) and \( G_2^c \), and the subgraphs formed by adding specific edges.

Let's break down the problem step by step:

1. **Original Graphs and Subgraphs:**
   - We have two graphs \( G_1 \) and \( G_2 \).
   - Consider the subgraphs \( G_1' \) and \( G_2' \) obtained by adding the edge \( (u_i, v_j) \) to \( G_1 \) and the edge \( (w_k, x_\ell) \) to \( G_2 \), respectively.
   - The problem states that these two subgraphs \( G_1' \) and \( G_2' \) are isomorphic.

2. **Complements of the Original Graphs:**
   - The complement of \( G_1 \), denoted as \( G_1^c \), is the graph where there is an edge between any two vertices if and only if there was no edge between them in \( G_1 \).
   - Similarly, the complement of \( G_2 \), denoted as \( G_2^c \), is the graph where there is an edge between any two vertices if and only if there was no edge between them in \( G_2 \).

3. **Removing Edges from Complements:**
   - When we remove the edge \( (u_i, v_j) \) from \( G_1 \) to form \( G_1' \), we are essentially creating \( G_1^c \) because the edge \( (u_i, v_j) \) is not present in \( G_1^c \).
   - Similarly, when we remove the edge \( (w_k, x_\ell) \) from \( G_2 \) to form \( G_2' \), we are essentially creating \( G_2^c \) because the edge \( (w_k, x_\ell) \) is not present in \( G_2^c \).

4. **Isomorphism of Complements:**
   - Since \( G_1' \) and \( G_2' \) are isomorphic, it means that the structure of the graphs after adding the respective edges is identical.
   - This implies that the structure of the graphs before removing the respective edges must also be identical, because the addition and removal of edges should preserve the isomorphism property.

Therefore, the graphs \( G_1^c \) and \( G_2^c \) are isomorphic.

The final answer is:
\[
\boxed{G_1^c \text{ and } G_2^c \text{ are isomorphic}}
\]