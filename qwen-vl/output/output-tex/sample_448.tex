To solve the problem, we need to understand the structure of the spanning \( r \)-arborescence \( T \) in the directed graph \( G = (V, E) \). Here's a step-by-step breakdown:

1. **Definition of Spanning \( r \)-Arborescence**: A spanning \( r \)-arborescence is a directed tree rooted at \( r \) such that every vertex in \( V \setminus \{r\} \) has exactly one incoming arc from its parent in the tree.

2. **Non-Leaf Vertices and Their Out-Neighborhoods**: Non-leaf vertices are those vertices with more than one child in the arborescence. Each non-leaf vertex has an out-neighborhood consisting of all its children. These out-neighborhoods form a partition of \( V \setminus \{r\} \).

3. **Colorful Frames and Number of Leaves**: The colorful frames around the out-neighborhoods of the non-leaf vertices indicate the number of leaves that are connected to each non-leaf vertex through its out-neighborhood. The number of leaves connected to a non-leaf vertex is equal to the number of children it has in the arborescence.

4. **Summing Up the Numbers Below the Out-Neighborhods**: The number of leaves of the arborescence can be calculated by summing up the numbers written below the out-neighborhoods of the non-leaf vertices. This is because each leaf is counted exactly once in this sum.

Therefore, the number of leaves of the spanning \( r \)-arborescence \( T \) is given by the sum of the numbers written below the out-neighborhoods of the non-leaf vertices.

The final answer is:
\[
\boxed{\text{The number of leaves is the sum of the numbers written below the out-neighborhoods of the non-leaf vertices.}}
\]