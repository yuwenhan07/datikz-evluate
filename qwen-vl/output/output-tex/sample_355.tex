In the context of symplectic geometry and billiards, a "symplectic billiard bounce" refers to the transformation that occurs when a particle bounces off the boundary of a strictly convex domain \(\Omega\). This transformation is described by the symplectic form, which is a fundamental concept in Hamiltonian mechanics.

Let's break down the given notation:

- \((p_1, p_3)\) represents the initial momentum components of the particle.
- \((p_2, p_3)\) represents the final momentum components after the bounce.

The symplectic billiard bounce can be understood as follows:

1. **Symplectic Form**: The symplectic form \(\omega\) on the phase space \(T^*M\) (where \(M\) is the configuration space) is a closed, non-degenerate 2-form. For a Hamiltonian system, it is typically written as:
   \[
   \omega = dp \wedge dq
   \]
   where \(p\) and \(q\) are the momenta and coordinates respectively.

2. **Hamilton's Equations**: The equations of motion for a Hamiltonian system are given by Hamilton's equations:
   \[
   \dot{q} = \frac{\partial H}{\partial p}, \quad \dot{p} = -\frac{\partial H}{\partial q}
   \]
   where \(H\) is the Hamiltonian function.

3. **Bounce Transformation**: When a particle bounces off the boundary of a strictly convex domain \(\Omega\), the momentum components perpendicular to the boundary change sign, while the parallel components remain unchanged. This transformation can be described using the reflection properties of the boundary.

4. **Symplectic Reflection**: The symplectic reflection at the boundary can be expressed as a transformation that preserves the symplectic form. For a strictly convex domain, the reflection can be written as:
   \[
   (p', q') = (p, q) + 2 \left( n(q) \cdot (p, q) \right) n(q)
   \]
   where \(n(q)\) is the unit normal vector to the boundary at point \(q\).

5. **Momentum Update**: In terms of the momenta, this transformation implies that the component of the momentum perpendicular to the boundary changes sign, while the parallel components remain the same. Mathematically, if we denote the perpendicular and parallel components of the momentum as \(p_\perp\) and \(p_\parallel\) respectively, then:
   \[
   (p_\perp, p_\parallel) \mapsto (-p_\perp, p_\parallel)
   \]

6. **Final Momentum Components**: Given the initial momentum components \((p_1, p_3)\), the final momentum components after the bounce will be:
   \[
   (p_2, p_3) = (p_1, p_3)
   \]
   because the parallel component \(p_3\) remains unchanged, and the perpendicular component \(p_1\) changes sign.

Therefore, the symplectic billiard bounce \((p_1, p_3) \mapsto (p_2, p_3)\) in a strictly convex domain \(\Omega\) results in the final momentum components being the same as the initial momentum components, with the perpendicular component \(p_1\) changing sign. The final answer is:
\[
(p_2, p_3) = (p_1, p_3)
\]