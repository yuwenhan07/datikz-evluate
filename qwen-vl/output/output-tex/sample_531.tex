It sounds like you're discussing an event study design for analyzing the impact of European Union (EU) accession on the trade patterns of new member states (NMS). Specifically, you're focusing on how the immediate preferential access to third-party markets via previously signed EU trade agreements affected the NMS after their accession in 2004.

Here’s a structured way to approach this:

### Event Study Design

1. **Objective**: To assess the impact of EU accession on the trade patterns of NMS, particularly focusing on the preferential access to third-party markets via previously signed EU trade agreements.

2. **Hypothesis**:
   - The immediate preferential access to third-party markets via previously signed EU trade agreements will lead to significant changes in trade flows and tariffs for the NMS in 2004.

3. **Data**:
   - **Trade Data**: Import/export data from the NMS before and after EU accession.
   - **Tariff Data**: Product-level bilateral tariffs between the NMS and third-party countries.
   - **FTAs**: Information on any free trade agreements (FTAs) signed by the EU with third-party countries prior to 2004.

4. **Trade Shock Measure**:
   - **Product-Level Bilateral Tariff Variation ($\Delta \tau_{sdip,2004}$)**: This measures the change in tariffs for products traded between the NMS and third-party countries in 2004. It is calculated as the difference in tariffs before and after EU accession.

5. **Constructing the Trade Shock**:
   - **Baseline Tariffs**: Calculate the average tariffs for each product category between the NMS and third-party countries before EU accession.
   - **Post-EU Accession Tariffs**: Calculate the average tariffs for the same product categories after EU accession.
   - **Difference in Tariffs**: The difference between the post-EU accession tariffs and the pre-EU accession tariffs represents the trade shock.

6. **Example**:
   - Suppose the EU signed an FTA with Mexico in 2000, but the NMS joined the EU in 2004. In 2004, the NMS would immediately adhere to the previously negotiated tariff schedule under this FTA.
   - For a specific product category, if the average tariff before EU accession was 10% and after EU accession it dropped to 5%, the trade shock would be a reduction of 5 percentage points.

7. **Analysis**:
   - **Panel Regression**: Use panel regression models to analyze the impact of the trade shock on trade volumes and tariffs.
   - **Control Variables**: Include control variables such as GDP growth, population size, and other economic indicators to account for confounding factors.
   - **Robustness Checks**: Perform robustness checks by including different control variables or using alternative measures of the trade shock.

8. **Conclusion**:
   - Based on the analysis, determine whether the trade shock significantly influenced trade flows and tariffs for the NMS.

### Summary

By constructing the trade shock through the product-level bilateral tariff variation, you can effectively measure the impact of EU accession on the preferential access to third-party markets for NMS. This approach allows you to isolate the effect of the EU's preferential trade agreements on the NMS' trade patterns, providing valuable insights into the economic implications of EU membership.