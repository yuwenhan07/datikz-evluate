To understand the graph of the function \( v \) in the context of the expectation \(\Expec{\nu_{\beta, t}([a, 1])}\), we need to interpret the given information carefully.

The notation \(\nu_{\beta, t}([a, 1])\) typically refers to a measure or distribution associated with some parameter \(\beta\) and time \(t\). The expectation \(\Expec{\nu_{\beta, t}([a, 1])}\) is then the expected value of this measure over the interval \([a, 1]\).

The statement "particles that contribute to \(\Expec{\nu_{\beta, t}([a, 1])}\) are 'near' \(v(\beta)at\)" suggests that the measure \(\nu_{\beta, t}\) is concentrated around the point \(v(\beta)at\). This implies that the function \(v\) plays a crucial role in determining the location of the measure's support.

Given this interpretation, the graph of the function \(v\) would show how the expected position of the measure \(\nu_{\beta, t}\) scales with \(\beta\) and \(t\). Specifically, the graph would be a plot of \(v(\beta)\) on the y-axis against \(\beta\) on the x-axis.

Therefore, the graph of the function \(v\) would look like:

\[
\boxed{y = v(\beta)}
\]

This graph represents the relationship between the parameter \(\beta\) and the expected position of the measure \(\nu_{\beta, t}\) scaled by \(at\). The exact form of \(v(\beta)\) would depend on the specific details of the problem, but the general shape of the graph would be a curve or a line depending on the nature of the function \(v\).