The UO invariants you're referring to likely pertain to the study of unitary orthogonal (UO) transformations, which are rotations in complex vector spaces. These invariants are useful in various fields such as quantum mechanics, representation theory, and group theory.

Let's break down the information provided:

### For \( n = 1 \):
- **A**: This typically refers to the simplest case where \( n = 1 \). In this context, there is only one invariant, often denoted as \( I_1 \).

### For \( n = 2 \):
- **B1**: This could be the first invariant for \( n = 2 \).
- **B2**: This could be the second invariant for \( n = 2 \).
- **B3**: This could be the third invariant for \( n = 2 \).

### For \( n = 3 \):
- **C1**: This could be the first invariant for \( n = 3 \).
- **C2**: This could be the second invariant for \( n = 3 \).
- **C3**: This could be the third invariant for \( n = 3 \).
- **C4**: This could be the fourth invariant for \( n = 3 \).
- **C5**: This could be the fifth invariant for \( n = 3 \).

### Indistinguishability:
The remark about sets like (B2 and B3), (C2 and C3), and (C4 and C5) being indistinguishable in the pure orthogonal case suggests that these pairs of invariants are essentially the same under certain conditions. This might occur due to symmetries or redundancies in the invariants themselves.

### Example Interpretation:
In the context of UO transformations, the invariants \( I_n \) are polynomials in the generators of the UO group. The number of invariants grows with \( n \), reflecting the complexity of the group structure. For instance, for \( n = 1 \), there is only one invariant because the group is trivially simple. As \( n \) increases, the number of independent invariants also increases, reflecting the higher-dimensional nature of the group.

### Conclusion:
The UO invariants provide a way to classify and understand the structure of unitary orthogonal groups. The specific invariants mentioned (A, B1-B3, C1-C5) are part of a larger set of polynomials that describe the group's properties. The indistinguishability of some pairs indicates that these invariants are not all independent, and they can be reduced to a smaller set of fundamental invariants through algebraic manipulations.

If you have more specific details or need further clarification on any aspect of these invariants, feel free to ask!