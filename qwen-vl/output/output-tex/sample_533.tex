To understand the given statement and its implications, let's break it down step by step.

The statement is about an estimate for \( i \in \mathcal{J}_2 \), where \(\mathcal{J}_2\) is likely a set of indices or a partition of some interval. The key part of the statement is:

From the fact that the value \( v_{i-1} \) survives up to time \( 2^{-k} \), we deduce that the area of the yellow triangle can be controlled by the \( L^1 \) norm of all the \( v_q^p \) in the interval \( (x_{i-1}, x_i) \).

Let's analyze this step-by-step:

1. **Survival of \( v_{i-1} \)**: The phrase "the value \( v_{i-1} \) survives up to time \( 2^{-k} \)" suggests that the function \( v_{i-1} \) remains non-zero or significant within the interval of interest until time \( 2^{-k} \). This implies that there is some persistence or continuity of the function \( v_{i-1} \) over this time period.

2. **Yellow Triangle**: The reference to a "yellow triangle" is not explicitly defined in the text, but it is likely a geometric figure or region associated with the function \( v \). The area of this yellow triangle is what we are trying to control.

3. **\( L^1 \) Norm**: The \( L^1 \) norm of a function \( f \) over an interval \( [a, b] \) is defined as:
   \[
   \| f \|_{L^1([a, b])} = \int_a^b |f(x)| \, dx.
   \]
   In our context, the \( L^1 \) norm of \( v_q^p \) over the interval \( (x_{i-1}, x_i) \) is:
   \[
   \| v_q^p \|_{L^1((x_{i-1}, x_i))} = \int_{x_{i-1}}^{x_i} |v_q^p(x)| \, dx.
   \]

4. **Control of the Area**: The statement claims that the area of the yellow triangle can be controlled by the \( L^1 \) norm of \( v_q^p \) over the interval \( (x_{i-1}, x_i) \). This means that if we know the \( L^1 \) norm of \( v_q^p \) over this interval, we can bound the area of the yellow triangle.

In summary, the statement is suggesting that the persistence of the value \( v_{i-1} \) up to time \( 2^{-k} \) allows us to use the \( L^1 \) norm of \( v_q^p \) over the interval \( (x_{i-1}, x_i) \) to control the area of the yellow triangle. This is a common technique in analysis and partial differential equations, where norms and integrals are used to estimate quantities like areas or volumes.

Thus, the final answer is:

\[
\boxed{\text{The area of the yellow triangle can be controlled by the } L^1 \text{ norm of all the } v_q^p \text{ in the interval } (x_{i-1}, x_i).}
\]