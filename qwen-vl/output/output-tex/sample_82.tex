The problem you've described involves a triangle in the coordinate plane with vertices labeled as \(0\), \(1\), and \(2\) (and an additional vertex \(3\)). Let's break down the coordinates of these points:

- \(0 = (0, 0)\)
- \(1 = (2\epsilon_n, 0)\)
- \(2 = (\epsilon_n, \sqrt{3}\epsilon_n)\)
- \(3 = (\epsilon_n, -\sqrt{3}\epsilon_n)\)

It appears that the coordinates for the vertices \(1\), \(2\), and \(3\) are given in terms of \(\epsilon_n\). The value of \(\epsilon_n\) is not specified, but it seems to be a scaling factor.

To understand the shape and properties of this triangle, we can calculate the lengths of its sides and check if it forms a specific type of triangle, such as an equilateral or right triangle.

### Step 1: Calculate the lengths of the sides

The distance between two points \((x_1, y_1)\) and \((x_2, y_2)\) is given by:
\[ d = \sqrt{(x_2 - x_1)^2 + (y_2 - y_1)^2} \]

#### Side \(01\):
\[ d_{01} = \sqrt{(2\epsilon_n - 0)^2 + (0 - 0)^2} = \sqrt{(2\epsilon_n)^2} = 2\epsilon_n \]

#### Side \(02\):
\[ d_{02} = \sqrt{(\epsilon_n - 0)^2 + (\sqrt{3}\epsilon_n - 0)^2} = \sqrt{\epsilon_n^2 + 3\epsilon_n^2} = \sqrt{4\epsilon_n^2} = 2\epsilon_n \]

#### Side \(03\):
\[ d_{03} = \sqrt{(\epsilon_n - 0)^2 + (-\sqrt{3}\epsilon_n - 0)^2} = \sqrt{\epsilon_n^2 + 3\epsilon_n^2} = \sqrt{4\epsilon_n^2} = 2\epsilon_n \]

#### Side \(12\):
\[ d_{12} = \sqrt{(\epsilon_n - 2\epsilon_n)^2 + (\sqrt{3}\epsilon_n - 0)^2} = \sqrt{(-\epsilon_n)^2 + 3\epsilon_n^2} = \sqrt{\epsilon_n^2 + 3\epsilon_n^2} = \sqrt{4\epsilon_n^2} = 2\epsilon_n \]

#### Side \(13\):
\[ d_{13} = \sqrt{(\epsilon_n - 2\epsilon_n)^2 + (-\sqrt{3}\epsilon_n - 0)^2} = \sqrt{(-\epsilon_n)^2 + 3\epsilon_n^2} = \sqrt{\epsilon_n^2 + 3\epsilon_n^2} = \sqrt{4\epsilon_n^2} = 2\epsilon_n \]

#### Side \(23\):
\[ d_{23} = \sqrt{(\epsilon_n - \epsilon_n)^2 + (-\sqrt{3}\epsilon_n - \sqrt{3}\epsilon_n)^2} = \sqrt{0 + (-2\sqrt{3}\epsilon_n)^2} = \sqrt{4 \cdot 3 \epsilon_n^2} = \sqrt{12\epsilon_n^2} = 2\sqrt{3}\epsilon_n \]

### Step 2: Determine the type of triangle

From the calculations above, we see that all three sides \(01\), \(02\), and \(03\) have length \(2\epsilon_n\), and side \(23\) has length \(2\sqrt{3}\epsilon_n\).

Since all three sides are equal, the triangle \(012\) is an equilateral triangle. The additional point \(3\) does not affect the equilateral nature of triangle \(012\).

Thus, the triangle \(012\) is an equilateral triangle with side length \(2\epsilon_n\).

The final answer is:
\[
\boxed{\text{Equilateral}}
\]