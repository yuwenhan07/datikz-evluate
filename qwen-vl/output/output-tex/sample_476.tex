The description you've provided refers to a specific type of flux function that exhibits interesting properties near the origin. Let's break down the key points:

1. **Strictly Convex Near the Origin**: The flux function \( f \) is strictly convex in a neighborhood around the origin. This means that for any two points \( x_1 \) and \( x_2 \) close to the origin, and for any \( t \in [0, 1] \), the following inequality holds:
   \[
   f(tx_1 + (1-t)x_2) < tf(x_1) + (1-t)f(x_2).
   \]
   This property ensures that the function has no "bumps" or "dents" near the origin.

2. **Not Uniformly Convex**: Despite being strictly convex near the origin, the function is not uniformly convex. Uniform convexity would require that there exists a constant \( c > 0 \) such that for all \( x_1, x_2 \) in some neighborhood of the origin,
   \[
   f(tx_1 + (1-t)x_2) \leq tf(x_1) + (1-t)f(x_2) - c t(1-t) \|x_1 - x_2\|^2.
   \]
   The lack of uniform convexity implies that the rate of convexity can vary depending on the distance from the origin.

3. **Smoothness**: The flux function \( f \) is infinitely differentiable (\( C^\infty \)) everywhere on \( \mathbb{R} \). This means that all its derivatives exist and are continuous, including at the origin.

4. **Vanishing Derivatives at the Origin**: All derivatives of \( f \) vanish at the origin. This means that \( f'(0) = 0 \), \( f''(0) = 0 \), \( f'''(0) = 0 \), and so on. In other words, the Taylor series expansion of \( f \) around the origin starts with the second-order term.

5. **Non-Analytic**: Although \( f \) is smooth and all its derivatives vanish at the origin, it is not analytic. A function is analytic if it can be represented by a convergent power series in a neighborhood of every point in its domain. The fact that \( f \) is not analytic suggests that the function cannot be expressed as a convergent power series around the origin, even though it is infinitely differentiable.

### Example of Such a Function

A classic example of a flux function that fits this description is the function \( f(x) = |x|^{2+\epsilon} \) for some small positive \( \epsilon \). Specifically, consider \( f(x) = |x|^{2+\epsilon} \):

- For \( x \neq 0 \), \( f(x) = x^{2+\epsilon} \) is clearly smooth and infinitely differentiable.
- At \( x = 0 \), all derivatives of \( f \) up to order \( 2 \) are zero: \( f'(0) = 0 \), \( f''(0) = 0 \), etc.
- However, \( f \) is not analytic because the higher-order terms in the Taylor series do not converge to the function itself. The function \( f(x) = |x|^{2+\epsilon} \) grows too fast for a convergent power series representation around \( x = 0 \).

This example demonstrates a flux function that is smooth, strictly convex near the origin, but not uniformly convex, and not analytic.