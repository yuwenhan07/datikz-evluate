The process you're describing involves several steps in string theory and M-theory, which are theoretical frameworks used to describe the fundamental nature of the universe. Let's break down each step:

1. **Dimensionally Reducing Along \(\phi_3\)**:
   - Dimensional reduction is a technique used in string theory to reduce the number of spatial dimensions from 10 (or 11 for M-theory) to a lower number, typically 4, which corresponds to our observed universe.
   - When we perform a dimensional reduction along \(\phi_3\), we are essentially compactifying one of the extra dimensions. This means that the extra dimension is curled up into a small, compact space, often referred to as a "circle" or "torus."
   - The parameter \(\phi_3\) typically represents the coordinate along this compactified dimension.

2. **Performing a TST Along \(\phi_1\)**:
   - TST stands for Type IIA String Theory, which is one of the five consistent string theories in ten dimensions.
   - In the context of Type IIA string theory, performing a T-duality transformation along \(\phi_1\) is a specific operation that can be performed on the theory.
   - T-duality is a symmetry of string theory that relates different string theories with different compactifications. It exchanges the radius of a circle with the momentum along that circle.
   - So, when we perform a T-duality transformation along \(\phi_1\), we are essentially exchanging the radius of the circle along \(\phi_1\) with the momentum along that direction.

3. **Uplifting Along \(\phi_3\)**:
   - After performing the T-duality transformation, we might want to "uplift" the theory back to the original higher-dimensional space.
   - Uplifting is the process of reversing the compactification by unrolling the compactified dimension. This is done by increasing the size of the compactified dimension, effectively "unwrapping" it.
   - In the context of \(\phi_3\), this would mean increasing the radius of the circle along \(\phi_3\) back to its original size.

Putting these steps together, the sequence of operations you described is:

1. Dimensionally reduce the theory along \(\phi_3\).
2. Perform a T-duality transformation along \(\phi_1\).
3. Uplift the theory along \(\phi_3\).

This sequence of operations can be useful in various contexts within string theory, such as when studying the properties of compactified dimensions or when exploring different dualities between different string theories.