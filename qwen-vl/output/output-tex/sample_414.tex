To solve this problem, we need to understand the definitions and properties of matchings and perfect matchings in hypergraphs, particularly focusing on the given conditions.

A hypergraph \( \mathcal{H} \) is a generalization of a graph where an edge (or hyperedge) can connect more than two vertices. A matching in a hypergraph is a set of edges such that no two edges share a vertex. A perfect matching in a hypergraph is a matching that covers all vertices exactly once.

Given:
- Two matchings \( \mathcal{M}_1 \subseteq \mathcal{H}_1 \) and \( \mathcal{M}_2 \subseteq \mathcal{H}_2 \) in the hypergraph \( \mathcal{H} \).
- The union of these matchings forms a \( P \)-perfect matching \( \mathcal{M} \).
- Here, \( p = 2 \), \( q = 4 \), and \( r = 2 \).

The notation \( P \)-perfect matching suggests that the matching \( \mathcal{M} \) is a perfect matching with some specific properties related to the parameters \( p \), \( q \), and \( r \). However, without additional context or a specific definition for \( P \)-perfect matching, we can infer that the problem is asking us to determine the nature of the matchings and their union under the given conditions.

Since \( \mathcal{M}_1 \) and \( \mathcal{M}_2 \) are matchings in the hypergraph \( \mathcal{H} \) and their union \( \mathcal{M} \) is a perfect matching, it implies that each vertex in the hypergraph is covered exactly once by the edges in \( \mathcal{M} \). This is the fundamental property of a perfect matching.

Therefore, the answer to the problem, based on the given conditions, is that the union of the two matchings \( \mathcal{M}_1 \) and \( \mathcal{M}_2 \) forms a perfect matching \( \mathcal{M} \) in the hypergraph \( \mathcal{H} \).

The final answer is:
\[
\boxed{\text{The union of } \mathcal{M}_1 \text{ and } \mathcal{M}_2 \text{ forms a perfect matching in } \mathcal{H}.}
\]