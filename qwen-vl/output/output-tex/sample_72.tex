To address the statement about saddle moves in the context of twist regions, let's break it down step by step.

1. **Understanding Saddle Moves**: A saddle move is a specific type of braid move that involves two strands crossing over each other in a way that resembles a saddle shape. This move can be used to transform one braid into another while preserving the overall structure of the braid up to isotopy (a continuous deformation without cutting or gluing).

2. **Twist Regions**: In the context of braids, a twist region refers to a segment of the braid where all the strands are involved in a single cyclic permutation. For example, if we have three strands \(a, b, c\) and they form a twist region, then the strands might be permuted as \(a \to b \to c \to a\), \(b \to c \to a \to b\), etc., depending on the specific twist.

3. **Saddle Moves at the Start and End of Twist Regions**: The statement mentions that there are two saddle moves at the start and end of consecutive twist regions. This means that before entering a new twist region, there is a saddle move that transforms the braid configuration, and similarly, after exiting a twist region, there is another saddle move that transforms the braid back to its original configuration.

4. **Crossings Matching Our Setting**: The statement also mentions that the crossings match our setting. This implies that the specific configuration of the saddle moves and the twist regions in the given figure from \cite[Figure 3]{BKLMR} align with the specific conditions or constraints of our particular problem or setting.

Given these points, the statement is essentially describing a specific pattern of braid moves that occur at the boundaries of twist regions in a braid diagram. These moves ensure that the braid remains consistent within each twist region but changes appropriately at the transitions between different twist regions.

Therefore, the answer to the question is:

\[
\boxed{\text{The statement describes a specific pattern of saddle moves at the start and end of consecutive twist regions in a braid diagram, ensuring consistency within each twist region and appropriate transitions between them.}}
\]