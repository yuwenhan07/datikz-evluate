To solve this problem, we need to understand the geometric configuration described and use the given conditions to derive the necessary conclusion.

1. **Understanding the Setup:**
   - We have a sequence of disks \( B(\hat{q}_k) \) centered at points \( \hat{q}_k \).
   - These disks cover the segment connecting two points on a curve \( c \), specifically from \( c((\gamma_k + \beta)s) \) to \( c((\gamma_k + \beta + t\alpha)s) \).
   - There exists a smallest integer \( l_0 \) such that the distance between \( q_k \) and \( \hat{q}_{l_0} \) is at least \( (\beta - t\alpha)s \).

2. **Interpreting the Distance Condition:**
   - The condition \( d(q_k, \hat{q}_{l_0}) \geq (\beta - t\alpha)s \) means that the point \( q_k \) is at least \( (\beta - t\alpha)s \) away from the center of the disk \( \hat{q}_{l_0} \).

3. **Analyzing the Covering Property:**
   - Since the disks \( B(\hat{q}_k) \) cover the segment from \( c((\gamma_k + \beta)s) \) to \( c((\gamma_k + \beta + t\alpha)s) \), every point on this segment must be within some disk \( B(\hat{q}_k) \).
   - This implies that there must be at least one disk \( B(\hat{q}_k) \) that contains the entire segment.

4. **Conclusion:**
   - Given that \( d(q_k, \hat{q}_{l_0}) \geq (\beta - t\alpha)s \), and knowing that the disks cover the segment, it follows that the disk \( B(\hat{q}_{l_0}) \) must contain the segment.
   - Therefore, the smallest \( l_0 \) such that \( d(q_k, \hat{q}_{l_0}) \geq (\beta - t\alpha)s \) ensures that the disk \( B(\hat{q}_{l_0}) \) covers the segment, which is consistent with the covering property of the disks.

Thus, the final answer is:

\[
\boxed{l_0}
\]