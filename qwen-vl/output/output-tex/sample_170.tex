The image you've described seems to be illustrating concepts from formal language theory, specifically related to regular languages and their representations using automata or finite state machines. Let's break down the components:

1. **Primitive \( b \)-cycles**: These are cycles in a finite state machine that start at specific states and have certain properties. The term "primitive" here likely refers to a cycle that cannot be decomposed into smaller cycles.

2. **Accepting States**: These are states in the finite state machine that indicate the end of a successful computation or path through the machine. In your description, these are denoted by double circles.

3. **Words**: These are sequences of symbols (in this case, binary digits \(0\) and \(1\)) that represent paths through the finite state machine. The top cycle has an associated word \(0011\), and the bottom cycle has an associated word \(001\).

4. **Starting States**: The cycles start at different states, labeled as \(0\) for the top cycle and \(\alpha\) for the bottom cycle.

5. **Product of Cycles**: The right side of the image shows the result of combining (or concatenating) the two cycles. This is done by taking the word associated with one cycle and appending it to the word associated with the other cycle. The operation used here appears to be a special kind of concatenation, possibly denoted by \(\odot\). The resulting word is \(001101111011\).

6. **Lemma \ref{lm:primitive product}**: This lemma suggests that the product of two primitive cycles is also a primitive cycle. This is confirmed by the fact that the resulting word \(001101111011\) is indeed primitive according to the lemma.

In summary, the image illustrates how two primitive cycles can be combined to form a new primitive cycle, which is a fundamental concept in the study of regular languages and automata theory. The lemma mentioned ensures that the resulting cycle retains the property of being primitive, which is crucial for understanding the structure and behavior of these cycles within the context of formal languages.