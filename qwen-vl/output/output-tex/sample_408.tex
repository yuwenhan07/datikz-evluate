The phase diagram you've described pertains to the staggered six-vertex model, which is a lattice model in statistical mechanics. This model is particularly interesting because it exhibits a rich phase structure that can be understood through the use of duality transformations and the identification of different phases based on the values of certain parameters.

### Key Points:

1. **Staggered Six-Vertex Model:**
   - The staggered six-vertex model is a generalization of the six-vertex model where the vertices are staggered, meaning they alternate in orientation.
   - It is defined on a square lattice and has a set of rules for how the vertices interact.

2. **Anisotropy Parameter \( \gamma \):**
   - The parameter \( \gamma \) represents the anisotropy in the model. When \( 0 < \gamma \leq \pi \), the model exhibits different phases depending on the value of \( \gamma \).

3. **Phase Diagram:**
   - The phase diagram is divided into three distinct phases (I, II, and III) based on the values of the staggering parameters \( \alpha \) and \( \pi - \alpha \).
   - These phases are identified using a duality transformation, which relates the staggered six-vertex model to another model with different symmetry properties.

4. **Scaling Limits:**
   - **Phase I:** In this phase, the scaling limit is described by a compact free boson, similar to the homogeneous limit when \( \alpha \to 0 \). This means that the low-energy excitations are described by a single compactified boson.
   - **Phase II:** Here, the critical degrees of freedom include one massless compact boson and two Majorana fermions. This indicates a more complex structure involving both bosonic and fermionic excitations.
   - **Phase III:** In this phase, the low-energy excitations are identified with those of the black hole coset model, which includes a non-compact degree of freedom. This suggests a more exotic phase structure involving non-compact excitations.

5. **Duality Transformation:**
   - The duality transformation mentioned in the reference \cite{FrMa12, KoLu23} is crucial for identifying the phases. It provides a mapping between the staggered six-vertex model and another model, allowing for the identification of the different phases based on the symmetry properties of the transformed model.

### Summary:
The phase diagram of the staggered six-vertex model with quasi-periodic boundary conditions in the critical regime for \( 0 < \gamma \leq \pi \) reveals a rich structure with three distinct phases. Each phase is characterized by different low-energy excitations, ranging from a simple compact free boson in Phase I to a more complex structure involving both bosons and fermions in Phase II, and finally to a non-compact degree of freedom in Phase III. The duality transformation plays a key role in understanding these phases and their transitions.