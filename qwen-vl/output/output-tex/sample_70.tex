To provide a detailed explanation, let's break down the components mentioned in your question:

1. **Contour \(\Gamma_\lambda\)**:
   - This contour is typically used in complex analysis to define a path around which we might integrate or apply some transformation.
   - In the context of spectral theory, such contours are often used to enclose a specific region of the complex plane where the spectrum of an operator lies.

2. **Region Enclosed by \(\Gamma_\lambda\)**:
   - The region enclosed by \(\Gamma_\lambda\) is denoted as \(D_\lambda\).
   - This region is specified in \cref{assu:Filter}, which likely refers to a set of assumptions or conditions related to filtering or spectral properties.

3. **Dashed Interval \([0, \kappa^2]\)**:
   - This interval represents a range on the real axis from 0 to \(\kappa^2\).
   - It is mentioned that this interval contains the spectrum of two operators, \(T\) and \(T_X\).

### Summary:
- The contour \(\Gamma_\lambda\) is a closed curve in the complex plane that encloses a region \(D_\lambda\).
- The region \(D_\lambda\) corresponds to the domain of definition for some mathematical object (e.g., a filter or a spectral function).
- The dashed interval \([0, \kappa^2]\) is a subset of the real line and it is stated that the spectrum of both operators \(T\) and \(T_X\) lies within this interval.

### Illustration:
If you were to illustrate this, you would typically draw:
- A closed contour \(\Gamma_\lambda\) in the complex plane.
- The region \(D_\lambda\) inside this contour.
- A horizontal line segment representing the interval \([0, \kappa^2]\), with a dashed line to indicate it is not part of the contour but rather a reference interval.

The illustration should clearly show how the contour \(\Gamma_\lambda\) encloses the region \(D_\lambda\), and how the interval \([0, \kappa^2]\) is positioned relative to the real axis, indicating the spectrum location.

Would you like further clarification or additional details?