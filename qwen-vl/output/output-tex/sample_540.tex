It seems like you're referring to a mathematical or computational context where \(a_3\) and \(b_1\) are being compared, but their values are unknown or indeterminate at this point. Let's break down the situation step by step:

1. **Identify the Variables**: 
   - \(a_3\) and \(b_1\) are two variables that need to be compared.
   
2. **Comparison Context**:
   - The statement suggests that \(a_3\) and \(b_1\) are being drawn in the same place because their relative sizes are unknown.

3. **Reasoning**:
   - If we don't know whether \(a_3\) is greater than, less than, or equal to \(b_1\), it makes sense to represent them in the same location on a graph or diagram. This could be due to the fact that their exact values are not provided or are not yet determined.

4. **Conclusion**:
   - Since the values of \(a_3\) and \(b_1\) are not known, they are placed in the same location to indicate that their relationship (whether one is larger, smaller, or equal) is undetermined.

Therefore, the output of \ref{mainint} would show \(a_3\) and \(b_1\) in the same place because their relative sizes are unknown. This is a common practice in mathematics and computer science when dealing with uncertain or incomparable values.