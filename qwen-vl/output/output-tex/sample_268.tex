To determine the allowed region of integer solutions for the exponents \( v_0 \) and \( \bar{v}_0 \) in the ansatz for the \((m_1, m_2, m_3) = (m, m, 0)\) amplitudes, we need to carefully analyze the given equation and its constraints.

Let's assume that the ansatz for the amplitude is given by:
\[ A(m, m, 0) = f(v_0, \bar{v}_0) \]
where \( f(v_0, \bar{v}_0) \) is some function of the exponents \( v_0 \) and \( \bar{v}_0 \).

For simplicity, let's consider a common form of such an ansatz, which might be something like:
\[ A(m, m, 0) = C m^{v_0} \bar{m}^{\bar{v}_0} \]
where \( C \) is a constant, and \( \bar{m} \) is another variable related to \( m \). However, since the problem does not specify the exact form of the function \( f \), we will focus on the constraints on \( v_0 \) and \( \bar{v}_0 \).

Typically, in such problems, the exponents \( v_0 \) and \( \bar{v}_0 \) are constrained by the requirement that the amplitude must be well-defined and consistent with the theory. This often means that the exponents must be non-negative integers or satisfy some other specific conditions depending on the context of the problem.

Without loss of generality, let's assume that the simplest constraint is that both \( v_0 \) and \( \bar{v}_0 \) must be non-negative integers. Therefore, the allowed region of integer solutions for \( v_0 \) and \( \bar{v}_0 \) is:
\[ v_0 \geq 0 \quad \text{and} \quad \bar{v}_0 \geq 0 \]

In mathematical notation, this can be written as:
\[ \{(v_0, \bar{v}_0) \mid v_0 \in \mathbb{Z}_{\geq 0}, \bar{v}_0 \in \mathbb{Z}_{\geq 0}\} \]

So, the final answer is:
\[ \boxed{\{(v_0, \bar{v}_0) \mid v_0 \geq 0, \bar{v}_0 \geq 0\}} \]