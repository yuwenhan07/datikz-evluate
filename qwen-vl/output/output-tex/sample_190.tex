To understand the example trajectories of the \(\nu\)-integral poles in Eq.~(\ref{sec:theory:eq:gW_box}), we need to consider the contributions from different propagators in the context of quantum field theory, particularly in the calculation of the vertex function \(g_W\) for the box diagram.

### 1. Nuclear Propagators in \(T_3\)

The nuclear propagators typically refer to the propagators of nucleons (protons and neutrons). In the context of the box diagram, these propagators can contribute to the \(\nu\)-integral poles when they are involved in the loop integrals. The poles arise due to the singularities in the propagators at certain momentum transfers.

For the nuclear propagators, the poles are usually associated with the resonances or bound states of the nucleon system. These poles can be labeled as \(\mathcal{N}\), where \(\mathcal{N}\) represents the nucleon propagators.

### 2. Photon Propagator

The photon propagator is a fundamental part of the electromagnetic interaction. It contributes to the \(\nu\)-integral poles when it is involved in the loop integrals of the box diagram. The poles in this case are associated with the virtual photons that can be exchanged between the external lines.

The poles in the photon propagator are typically associated with the massless nature of the photon. However, in the context of quantum electrodynamics (QED), these poles are often regularized and renormalized through the process of dimensional regularization or other techniques to ensure the convergence of the integrals.

### 3. Electron Propagator

The electron propagator also contributes to the \(\nu\)-integral poles in the box diagram. The poles arise due to the singularities in the propagator at certain momentum transfers, which can be associated with the virtual electron loops.

The poles in the electron propagator are associated with the mass of the electron. In QED, the electron mass is typically taken to be zero, but in more general contexts, it can be non-zero. The poles are thus associated with the virtual electron masses.

### Summary

- **Nuclear Propagators (\(\mathcal{N}\))**: Poles arise from the resonances or bound states of the nucleon system.
- **Photon Propagator (\(\gamma\))**: Poles arise from the virtual photons exchanged between the external lines, which are typically massless.
- **Electron Propagator (e)**: Poles arise from the virtual electron loops, which can have non-zero masses depending on the context.

These poles are important in the calculation of the vertex function \(g_W\) for the box diagram, as they determine the behavior of the amplitude at high energies and provide insights into the underlying physics of the system. The specific trajectories of these poles depend on the details of the theory and the regularization scheme used.