It looks like you're referring to a scenario where an optimal control problem is being solved numerically using different levels of discretization or approximation, denoted by \( N \). Here's a breakdown of what this might mean:

1. **Optimal Trajectory (\( w^\star = \operatorname{col}(u^\star, x^\star) \))**:
   - This represents the true optimal solution to the control problem. It consists of the optimal control input \( u^\star \) and the corresponding state trajectory \( x^\star \).

2. **Approximate Optimal Trajectories (\( w^N = \operatorname{col}(u^N, x^N) \))**:
   - These are numerical approximations of the optimal trajectory obtained by solving the problem with increasing levels of discretization or approximation, denoted by \( N \).
   - For example:
     - \( w^2 \): An approximation obtained with \( N = 2 \).
     - \( w^3 \): An approximation obtained with \( N = 3 \).
     - \( w^4 \): An approximation obtained with \( N = 4 \).
     - \( w^5 \): An approximation obtained with \( N = 5 \).

3. **Dashed Black Line**:
   - The dashed black line represents the true optimal trajectory \( w^\star \), which is the reference against which the approximate solutions are compared.

### Interpretation:
- As \( N \) increases, the approximations \( w^N \) should get closer to the true optimal trajectory \( w^\star \). This is because higher values of \( N \) typically lead to more accurate numerical solutions.
- The difference between \( w^\star \) and \( w^N \) can be used to assess the accuracy of the numerical method used to solve the optimal control problem.

### Example Context:
This could be relevant in fields such as aerospace engineering, robotics, or any field that involves optimal control problems. For instance, in aerospace, one might want to find the optimal trajectory for a spacecraft to reach a target orbit with minimal fuel consumption. The true optimal trajectory would be the ideal path, while the approximations would be paths found through numerical methods with varying degrees of precision.

If you have specific details about the system or the numerical method used, it would help provide a more precise interpretation.