The description provided outlines a construction of a set \( A' \) based on the iterated application of a set word map \(\Upsilon\) over a sequence of sets, specifically \(\Gamma_{2k}\) and \(\Gamma_{4k}\). Let's break down the components and their roles:

### Components of \( A' \)

1. **Standard Basis \( B \)**:
   - \( B = \{x_1, x_2, \ldots, x_{2k}\} \)
   - This is the standard basis of \(\mathcal{F}_{2k}\), where each element \( x_i \) represents a generator of the free group.

2. **Words \( w_i \)**:
   - These words play two roles:
     - They are the right-side vertices in \(\Gamma_{2k}\).
     - They are also the words \( w_i(B) \) induced by the set word map \(\Upsilon_{\Gamma_{2k}}\), which constitute the set \( D \).

3. **Images \( v_i \)**:
   - The \( v_i \)'s are the images of the set word map \( A \) applied on the tuple \( D = (w_i(B))_{i=1}^{k} \).
   - These images constitute the set \( E \).

4. **Set \( E \)**:
   - \( E = \{v_1, v_2, \ldots, v_{4k}\} \)
   - The set \( E \) plays the role of the left side of \(\Gamma_{4k}\).

5. **Images \( y_i \)**:
   - The \( y_i \)'s are the right-hand-side vertices in \(\Gamma_{4k}\).
   - They are also the images of \(\Upsilon_{\Gamma_{4k}}\), which constitute the set \( F \).

6. **Set \( F \)**:
   - \( F = \{y_1, y_2, \ldots, y_{2k}\} \)

### Summary of \( A' \):
\[ A' = B \cup E \cup F = \{x_1, x_2, \ldots, x_{2k}, v_1, v_2, \ldots, v_{4k}, y_1, y_2, \ldots, y_{2k}\} \]

### Properties of \(\Gamma_{2k}\) and \(\Gamma_{4k}\):

- Both \(\Gamma_{2k}\) and \(\Gamma_{4k}\) are supposed to be \( d \)-left regular.
- The figure provided illustrates 3-regularity, but the actual construction is for \( d \)-left regular graphs.

### Conclusion:

The set \( A' \) is constructed by iteratively applying the set word map \(\Upsilon\) over the sets \(\Gamma_{2k}\) and \(\Gamma_{4k}\), resulting in a set that includes elements from the standard basis, the images of these bases under the set word map, and the images of the resulting sets. The structure ensures that \( A' \) captures the necessary vertices and edges to represent the iterated application of the set word map in the context of the given lemma.