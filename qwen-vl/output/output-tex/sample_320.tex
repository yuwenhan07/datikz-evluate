The concept you're describing is related to the theory of Stokes phenomena in complex analysis and its applications in mathematical physics, particularly in the context of asymptotic expansions and the behavior of solutions to differential equations near singular points.

### Stokes Phenomenon:
The Stokes phenomenon refers to the sudden change in the asymptotic behavior of solutions to certain differential equations as one crosses a Stokes line. These lines are typically oriented in the complex plane and are determined by the singularities of the differential equation.

### Stokes Arrows:
Stokes arrows are used to describe the transition across these Stokes lines. They indicate the direction of the jump in the asymptotic expansion when crossing from one sector to another. Each Stokes arrow connects the midpoints of two adjacent sectors and points in the direction of the jump.

### Legendre Transform and Path Interpretation:
The Legendre transform is a mathematical operation that can be used to transform one set of variables into another. In the context of the Stokes phenomenon, it can be used to map paths in the complex plane to other paths, often simplifying the analysis of the problem.

When you mention "a corresponding path on the left-hand side," you are likely referring to the transformation of the path in the complex plane under the Legendre transform. This transformation can sometimes simplify the analysis or provide a different perspective on the problem.

### Path Correspondence:
The path on the left-hand side, after the Legendre transform, may not necessarily correspond to a Stokes arrow because the Legendre transform can alter the geometric properties of the path. The original path in the complex plane might have been a straight line or a curve that crosses multiple Stokes lines, but after the Legendre transform, the resulting path could be more complex or simpler depending on the specific transformation applied.

### Summary:
- **Stokes Arrows**: Connect midpoints of sectors and indicate the direction of the jump in the asymptotic expansion.
- **Legendre Transform**: A mathematical tool that can transform paths in the complex plane, potentially altering their geometric properties.
- **Path Correspondence**: The transformed path on the left-hand side may not directly correspond to a Stokes arrow due to the nature of the Legendre transform.

Understanding these concepts requires a deep dive into complex analysis and asymptotic methods, which are fundamental in many areas of mathematical physics and engineering.