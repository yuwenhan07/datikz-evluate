It seems you're describing a type of graph that is being constructed or analyzed in a specific context, possibly related to time series data or causal inference. Here's a breakdown of what your description might mean:

1. **Graph Construction**: The graph is being built based on some form of estimation or analysis, likely from data where the relationships between variables are being inferred.

2. **Bold Edges**: These represent directed edges (arrows) between repeated measurements. This suggests that there is a known or expected relationship between these repeated measures, such as a cause-and-effect relationship. For example, if you have repeated measurements of a treatment and an outcome over time, you might expect to see a bold edge from the treatment to the outcome, indicating that the treatment causes the outcome.

3. **Dotted Edges**: These dotted lines represent edges that are implausible because they contradict the known time ordering. This implies that the direction of the edge does not align with the expected temporal sequence of events. For instance, if you know that a treatment is given before the outcome occurs, then any edge from the outcome back to the treatment would be marked as dotted, as it goes against the established time order.

### Example Scenario:
Suppose you are analyzing the effect of a drug (treatment) on blood pressure (outcome). You have multiple measurements of both the drug dosage and blood pressure at different times. In this case:
- A bold edge from "Drug Dosage" to "Blood Pressure" would indicate that the drug dosage is expected to influence blood pressure.
- A dotted edge from "Blood Pressure" to "Drug Dosage" would be implausible because blood pressure cannot cause its own dosage (it's not a self-causal relationship).

This kind of graph can be used in various fields like epidemiology, economics, or social sciences to understand causal relationships and test hypotheses about the directionality of effects.