To solve the problem of finding linear realizations for the given sets of numbers, we need to understand what a "linear realization" means in this context. Typically, a linear realization refers to a way to arrange the numbers in a sequence such that the sum of any two consecutive numbers is a constant or follows a specific pattern.

Let's start with the first set: \(\{1^6, 7^{18}\}\).

Since \(1^6 = 1\) and \(7^{18}\) is a very large number, it is clear that these two numbers cannot be arranged in a linear sequence where the sum of any two consecutive numbers is a constant. This is because the difference between \(7^{18}\) and 1 is so large that no other number can fit in between them while maintaining a constant sum with either 1 or \(7^{18}\). Therefore, there is no linear realization for the set \(\{1, 7^{18}\}\).

Now let's consider the second set: \(\{1^7, 7^{17}, 8^3\}\).

First, calculate the values:
\[1^7 = 1,\]
\[7^{17},\]
\[8^3 = 512.\]

We need to find a way to arrange these numbers in a linear sequence such that the sum of any two consecutive numbers is a constant. Let's denote the sequence by \(a_1, a_2, a_3\) where \(a_1 = 1\), \(a_2 = 7^{17}\), and \(a_3 = 512\). We need to find a constant \(k\) such that:
\[a_1 + a_2 = k,\]
\[a_2 + a_3 = k.\]

Substituting the known values, we get:
\[1 + 7^{17} = k,\]
\[7^{17} + 512 = k.\]

From these equations, we see that:
\[1 + 7^{17} = 7^{17} + 512 - 511,\]
which simplifies to:
\[1 + 7^{17} = 7^{17} + 512 - 511.\]

This confirms that the sum of any two consecutive numbers is indeed a constant. Therefore, one possible linear realization is:
\[1, 7^{17}, 512.\]

So, the linear realization for the set \(\{1^7, 7^{17}, 8^3\}\) is:
\[
\boxed{1, 7^{17}, 512}
\]