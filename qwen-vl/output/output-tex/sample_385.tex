To address the request for the transfer matrices corresponding to weak injectivity and strong injectivity, we need to first understand what these terms mean in the context of linear systems or networks.

### Weak Injectivity
Weak injectivity is a property that ensures that if two different inputs produce the same output, then there must be at least one input component that is different between those two inputs. In other words, it guarantees that the system can distinguish between different inputs based on their outputs, but not necessarily in a unique way.

The transfer matrix \( T \) for a system with weak injectivity can be represented as:
\[ T = \begin{pmatrix}
1 & 0 \\
0 & 1
\end{pmatrix} \]
This matrix indicates that each input component is mapped to a unique output component without any overlap or ambiguity. However, this is a simplified example, and in more complex systems, the transfer matrix might look different depending on the specific structure and connections within the system.

### Strong Injectivity
Strong injectivity is a stronger form of injectivity where each input component is mapped to a unique output component, and no two different inputs can produce the same output. This means that the system is able to uniquely identify each input based on its output.

The transfer matrix \( T \) for a system with strong injectivity can be represented as:
\[ T = \begin{pmatrix}
1 & 0 \\
0 & 1
\end{pmatrix} \]
Again, this is a simplified example. In more complex systems, the transfer matrix might look different depending on the specific structure and connections within the system.

### General Formulas
For a general system, the transfer matrix \( T \) can be written as:
- For weak injectivity: \( T = \begin{pmatrix}
T_{11} & T_{12} \\
T_{21} & T_{22}
\end{pmatrix} \)
where \( T_{ij} \) represents the mapping from the \( i \)-th input to the \( j \)-th output. The key point here is that the matrix should be such that if two different inputs produce the same output, then there must be at least one input component that is different between those two inputs.

- For strong injectivity: \( T = \begin{pmatrix}
T_{11} & T_{12} \\
T_{21} & T_{22}
\end{pmatrix} \)
where the matrix should be such that each input component is mapped to a unique output component, and no two different inputs can produce the same output. This typically implies that the matrix is invertible, meaning that \( T^{-1} \) exists and is also a valid transfer matrix.

### Conclusion
The exact forms of the transfer matrices depend on the specific details of the system being considered. The examples provided above are simplified and illustrative. For a more precise answer, you would need to know the specific structure and connections of the system in question.

The final answer, in a boxed format, is:
\[
\boxed{
\begin{pmatrix}
1 & 0 \\
0 & 1
\end{pmatrix}
}
\]
for both weak and strong injectivity, assuming the simplest case.