The illustration you're referring to likely depicts a scenario where a semi-supervised classification-incremental learning (SS-CIL) framework is being applied to a series of tasks. In this context, the Average Confidence Score (ACS) is a metric used to assess the quality of predictions made on unlabeled data. Here's a breakdown of what this might look like and why it suggests a potential issue with using a fixed high threshold:

### What is ACS?
- **Definition**: ACS is calculated by averaging the maximum probability confidence scores of predictions made on unlabeled data.
- **Calculation**: For each unlabeled data point, the model predicts probabilities for each class. The maximum probability score for each data point is then averaged over all unlabeled data points.

### Observed Decaying Trend
- **Trend Description**: The trend shows a decrease in ACS as the number of tasks increases.
- **Implication**: This decay suggests that the model's confidence in its predictions on unlabeled data is decreasing over time.

### Why a Fixed High Threshold Might Not Be Suitable
- **Threshold Issue**: A fixed high threshold in SS-CIL means that only data points with a confidence score above this threshold are considered for training. As the ACS decreases, fewer data points meet this threshold, leading to a significant reduction in the amount of unlabeled data used for training.
- **Impact on Feature Learning**: With less data being used for training, the model may not effectively learn features that generalize well across tasks. This can lead to suboptimal performance on new tasks.

### Example Scenario
Imagine you have a sequence of tasks where the model starts with a high ACS but gradually sees a decline due to increasing complexity or domain shift between tasks. If a fixed high threshold is used, the model will only consider a small subset of the unlabeled data for each task, which could result in:
1. **Reduced Data Utilization**: Fewer data points are used for training, leading to slower learning and potentially worse performance.
2. **Feature Learning Issues**: The model might not capture enough information from the unlabeled data to adapt to the changing nature of the tasks.

### Recommendations
To address this issue, one might consider:
- **Adaptive Thresholds**: Instead of a fixed threshold, use an adaptive threshold that adjusts based on the current ACS. This could help maintain a reasonable amount of data for training even as the ACS decreases.
- **Dynamic Data Selection**: Implement a mechanism that dynamically selects more data points when the ACS drops below a certain threshold, ensuring that the model continues to have sufficient data for learning.

By addressing these issues, the model can better utilize the unlabeled data across tasks, potentially improving its performance and adaptability.