It seems like you're referring to a geometric or combinatorial proof involving specific configurations of sets and their intersections, possibly related to covering radii in the context of coding theory or geometry. Let's break down the elements mentioned:

1. **\( K \) (black)**: This is likely one of the main sets involved in the proof. It could be a convex body or a region in space that is central to the theorem.

2. **\(-K\) (dashed)**: This represents the negation or reflection of \( K \) across some axis or center. If \( K \) is symmetric about the origin, then \(-K\) would be the same set but with all points reflected across the origin.

3. **\(\overline{M}_p(K, -K)\) (blue)**: This term suggests a specific operation or construction involving the sets \( K \) and \(-K\). The notation \(\overline{M}_p\) might refer to a particular type of Minkowski sum or a modified version of it, where \( p \) could be an exponent or parameter that modifies the operation. The blue color indicates this is a key part of the construction.

4. **\(\underline{M}_q(K, -K)\) (orange)**: Similar to the previous point, this term involves another operation on the sets \( K \) and \(-K\). The notation \(\underline{M}_q\) suggests a different type of Minkowski sum or modification, possibly with a different exponent or parameter \( q \). The orange color signifies its importance in the proof.

### Context of the Theorem

Given the parameters \( n = 2 \), \( p = 2 \), \( q = 1 \), and \( R = 3 \), we can infer that the theorem is dealing with two-dimensional spaces (\( n = 2 \)), and the operations involve specific exponents and radii. The covering radius \( R = 3 \) implies that the sets are considered within a certain distance from each other, which is crucial for understanding the geometric relationships between \( K \), \(-K\), \(\overline{M}_p(K, -K)\), and \(\underline{M}_q(K, -K)\).

### Possible Interpretation

In the context of covering radii, these sets might represent regions around a central point or a set of points, and the operations \(\overline{M}_p\) and \(\underline{M}_q\) could be used to construct larger regions that cover the original sets within a specified radius \( R \). The blue and orange regions might represent the resulting regions after applying these operations, ensuring that they cover the original sets \( K \) and \(-K\) within the given radius.

### Conclusion

The proof of Theorem~\ref{covering radii forth combination first theorem} likely involves showing that the combined regions \(\overline{M}_p(K, -K)\) and \(\underline{M}_q(K, -K)\) cover the original sets \( K \) and \(-K\) within the specified radius \( R = 3 \). The visual representation helps in understanding the spatial relationships and the construction of these regions.

If you have more details or a specific reference to the theorem, it would help in providing a more precise interpretation and further insights into the proof.