The expression \( K_t = K^+_t \cup L_t \cup K^-_t \) describes the union of three sets, which is a common way to define a curve or a path in mathematical analysis and geometry. Let's break down each component:

1. **\( K^+_t \)**: This typically represents the positive part of the curve at time \( t \). It could be a segment of the curve that is oriented in a specific direction (e.g., moving from left to right).

2. **\( L_t \)**: This often denotes a line segment or a straight line at time \( t \). It might represent a transition point where the curve changes its direction or nature.

3. **\( K^-_t \)**: This stands for the negative part of the curve at time \( t \). Similar to \( K^+_t \), it could be a segment of the curve that is oriented in the opposite direction (e.g., moving from right to left).

### Interpretation:
- The union \( \cup \) indicates that all points belonging to any one of these sets are included in \( K_t \).
- Together, \( K^+_t \), \( L_t \), and \( K^-_t \) form the complete curve \( K_t \) at time \( t \).

### Example:
Imagine a simple curve that moves along a straight line and then turns back. At time \( t \):
- \( K^+_t \) could be the segment of the curve moving forward.
- \( L_t \) could be the straight line segment where the curve changes direction.
- \( K^-_t \) could be the segment of the curve moving backward after the turn.

In this case, \( K_t \) would be the entire path traced by the curve from the start to the end at time \( t \).

This type of decomposition is useful in various fields such as differential geometry, control theory, and computer graphics, where understanding the components of a complex path can provide insights into its behavior and properties.