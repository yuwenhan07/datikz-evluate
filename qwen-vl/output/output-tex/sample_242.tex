To solve this problem, we need to understand the definitions of the terms used in the problem statement. Let's break it down step by step.

### Definitions:
1. **$(1,1,4,4)$-packing colorable**: A graph is said to be $(1,1,4,4)$-packing colorable if there exists a coloring of its vertices such that each vertex is colored with one of four colors, and the following conditions are satisfied:
   - Each color class contains exactly 1 vertex.
   - Each color class contains exactly 1 vertex.
   - Each color class contains exactly 4 vertices.
   - Each color class contains exactly 4 vertices.
   
2. **$1$-saturated graph**: A graph is $1$-saturated if adding any new edge would create a matching of size greater than 1. In other words, the maximum matching in the graph has size 1.

3. **$(3,2)$-saturated subcubic graph**: A graph is $(3,2)$-saturated subcubic if it is a subcubic graph (each vertex has degree at most 3) and adding any new edge would create a matching of size greater than 2. In other words, the maximum matching in the graph has size 2.

### Problem Analysis:
The problem states that there is a non $(1,1,4,4)$-packing colorable $1$-saturated graph on the left, and a non $(1,1,3,3)$-packing colorable $(3,2)$-saturated subcubic graph on the right. We need to find these graphs.

#### Left Graph: Non $(1,1,4,4)$-packing colorable $1$-saturated graph
A $1$-saturated graph cannot have a matching of size 2 or more because adding any edge would create a matching of size 2 or more. Therefore, the maximum matching in a $1$-saturated graph has size 1. This means that the graph must be such that no two edges can form a matching of size 2. One example of such a graph is a graph where every vertex is connected to exactly one other vertex, forming a perfect matching. However, since the graph is not $(1,1,4,4)$-packing colorable, it must be a graph where this perfect matching condition is violated in some way. An example could be a graph where the perfect matching is disrupted by an additional edge that does not form a matching of size 2 but still prevents the graph from being $(1,1,4,4)$-packing colorable.

#### Right Graph: Non $(1,1,3,3)$-packing colorable $(3,2)$-saturated subcubic graph
A $(3,2)$-saturated subcubic graph must be a subcubic graph where the maximum matching has size 2. This means that the graph must be such that no two edges can form a matching of size 3 or more. One example of such a graph is a graph where the maximum matching is exactly 2, and adding any new edge would disrupt this matching. An example could be a graph where the maximum matching is disrupted by an additional edge that does not form a matching of size 3 but still prevents the graph from being $(1,1,3,3)$-packing colorable.

### Conclusion:
The specific graphs are not provided in the problem statement, but based on the definitions and the requirements, the left graph could be a graph like a cycle of length 5 (a pentagon), and the right graph could be a graph like a cycle of length 6 (a hexagon) with an additional edge added to disrupt the maximum matching of size 2.

Thus, the answer is:

\boxed{\text{The left graph is a non $(1,1,4,4)$-packing colorable $1$-saturated graph, and the right graph is a non $(1,1,3,3)$-packing colorable $(3,2)$-saturated subcubic graph.}}