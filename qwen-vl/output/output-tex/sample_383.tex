To illustrate Newton's method for finding the root of the nonlinear equation \( f(x) = 0.5x^2 - 3x + 4 \), let's follow the steps and visualize the process.

### Step-by-Step Illustration

1. **Define the Function:**
   \[
   f(x) = 0.5x^2 - 3x + 4
   \]

2. **Compute the Derivative:**
   \[
   f'(x) = x - 3
   \]

3. **Choose an Initial Guess \( x_0 \):**
   Let's start with \( x_0 = 2 \).

4. **Iterate Using Newton's Method:**

   The formula for Newton's method is:
   \[
   x_{n+1} = x_n - \frac{f(x_n)}{f'(x_n)}
   \]

   Let's compute the first iteration:

   - Calculate \( f(x_0) \):
     \[
     f(2) = 0.5(2)^2 - 3(2) + 4 = 2 - 6 + 4 = 0
     \]
     Since \( f(2) = 0 \), \( x_0 = 2 \) is already a root!

   However, let's proceed with another initial guess to show the iterative process.

   Suppose we choose \( x_0 = 3 \):

   - Calculate \( f(x_0) \):
     \[
     f(3) = 0.5(3)^2 - 3(3) + 4 = 4.5 - 9 + 4 = -0.5
     \]

   - Calculate \( f'(x_0) \):
     \[
     f'(3) = 3 - 3 = 0
     \]

   Since \( f'(3) = 0 \), this indicates a potential issue where the derivative is zero, but we can handle it by choosing a slightly different initial guess or using a different method.

   Let's choose \( x_0 = 3.5 \):

   - Calculate \( f(x_0) \):
     \[
     f(3.5) = 0.5(3.5)^2 - 3(3.5) + 4 = 6.125 - 10.5 + 4 = -0.375
     \]

   - Calculate \( f'(x_0) \):
     \[
     f'(3.5) = 3.5 - 3 = 0.5
     \]

   - Update \( x_1 \):
     \[
     x_1 = 3.5 - \frac{-0.375}{0.5} = 3.5 + 0.75 = 4.25
     \]

   Now, let's iterate again with \( x_1 = 4.25 \):

   - Calculate \( f(x_1) \):
     \[
     f(4.25) = 0.5(4.25)^2 - 3(4.25) + 4 = 8.90625 - 12.75 + 4 = -0.84375
     \]

   - Calculate \( f'(x_1) \):
     \[
     f'(4.25) = 4.25 - 3 = 1.25
     \]

   - Update \( x_2 \):
     \[
     x_2 = 4.25 - \frac{-0.84375}{1.25} = 4.25 + 0.675 = 4.925
     \]

   Continue iterating until convergence.

### Visualization

To visualize this process, you can plot the function \( f(x) \) and its tangent lines at each step. The tangent line at \( x_n \) will intersect the x-axis at \( x_{n+1} \). This intersection point is the next approximation of the root.

### Conclusion

Newton's method converges quickly to the root when the initial guess is close enough to the actual root. If the initial guess is far from the root, the method may converge to a different root depending on the shape of the function and the location of the roots.

In this example, starting with \( x_0 = 2 \) directly gives the exact root, while starting with \( x_0 = 3 \) or \( x_0 = 3.5 \) leads to an iterative process that converges to the root, but the path taken depends on the initial guess.