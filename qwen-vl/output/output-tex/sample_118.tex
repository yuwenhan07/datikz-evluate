To address the given problem, let's break down the key points and analyze them step-by-step:

1. **Understanding the Sets Involved:**
   - \( K \) is a convex set.
   - \( -K \) is the set obtained by reflecting \( K \) across the origin.
   - \( \underline{M}_0(K, -K) \) represents the Minkowski sum of \( K \) and \( -K \).
   - \( \overline{M}_1(K, -K) \) represents the Minkowski sum with a specific scaling factor.
   - \( \overline{M}_3(K, -K) \) and \( \overline{M}_\infty(K, -K) \) represent other Minkowski sums with different scaling factors.

2. **Common Boundary Points:**
   - The common boundary points of \( K \) and \( -K \) are not boundary points of \( \underline{M}_0(K, -K) \). This means that the boundary of \( \underline{M}_0(K, -K) \) does not include any point that is on the boundary of both \( K \) and \( -K \).

3. **Vertices of \( \overline{M}_1(K, -K) \):**
   - The vertices of \( \overline{M}_1(K, -K) \) are smooth boundary points of \( \overline{M}_\infty(K, -K) \). This implies that the vertices of \( \overline{M}_1(K, -K) \) are points where the boundary of \( \overline{M}_\infty(K, -K) \) is smooth.

4. **Supporting Lines:**
   - According to Lemma~\ref{common boundary points lemma}~(ii), for \( p > 1 \), \( \overline{M}_p(K, -K) \) is supported at each of these points by exactly one respective line that also supports \( K \) and \( -K \). This means that there is a unique supporting line from \( \overline{M}_p(K, -K) \) at each of these points that is also a supporting line for \( K \) and \( -K \).

5. **Conclusion:**
   - The points where the vertices of \( \overline{M}_1(K, -K) \) lie are smooth boundary points of \( \overline{M}_\infty(K, -K) \).
   - These points are supported by lines that are also supporting lines for \( K \) and \( -K \).
   - However, these points do not necessarily belong to \( K \) or \( -K \). They can be part of the boundary of \( \overline{M}_\infty(K, -K) \) but not necessarily part of the boundaries of \( K \) or \( -K \).

Thus, the final answer is:
\[
\boxed{\text{The points where the vertices of } \overline{M}_1(K, -K) \text{ lie are smooth boundary points of } \overline{M}_\infty(K, -K) \text{ and are supported by lines that are also supporting lines for } K \text{ and } -K, \text{ but they do not necessarily belong to } K \text{ or } -K.}
\]