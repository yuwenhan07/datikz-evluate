To address the problem, we need to understand the context of the theorem and the conditions it imposes on the trees \( T_i \) and the sets \( S_i \). Let's break down the problem step by step.

### Step 1: Understanding the Theorem
Theorem \ref{neccsuff} likely provides necessary and sufficient conditions for a certain property or relationship to hold between the trees \( T_i \) and the sets \( S_i \). These conditions are typically formulated in terms of distances, connectivity, or other graph-theoretic properties.

### Step 2: Identifying the Failure Condition
The problem states that all conditions of Theorem \ref{neccsuff} hold true except for one specific condition \( i \). This means there is a particular distance \( d_T(u,v) = i \) where the condition fails for some pair of vertices \( u \) and \( v \).

### Step 3: Analyzing the Failure
The failure of condition \( i \) implies that there exists at least one pair of vertices \( u \) and \( v \) in the tree \( T_i \) such that the distance \( d_T(u,v) = i \) does not satisfy the condition specified in Theorem \ref{neccsuff}. This could be due to various reasons depending on the specific condition \( i \) being violated.

### Step 4: Example of a Possible Violation
Without knowing the exact form of Theorem \ref{neccsuff}, we can provide a general example of how this might occur. Suppose Theorem \ref{neccsuff} involves the property that for any two vertices \( u \) and \( v \) with distance \( d_T(u,v) = i \), the set \( S_i \) must contain at least one vertex from each of the connected components formed by removing \( u \) and \( v \) from \( T_i \).

If the condition fails for a pair \( u \) and \( v \) with \( d_T(u,v) = i \), it means that when \( u \) and \( v \) are removed, the remaining tree does not have the required property regarding the set \( S_i \). For instance, the set \( S_i \) might not intersect both connected components formed by the removal of \( u \) and \( v \).

### Conclusion
The failure of condition \( i \) for the pair \( u \) and \( v \) with \( d_T(u,v) = i \) indicates that the specific condition \( i \) of Theorem \ref{neccsuff} is not satisfied for these vertices. This could be due to various reasons depending on the exact nature of the condition \( i \).

Thus, the answer to the problem is:

\[
\boxed{\text{The condition } i \text{ fails for the pair } u \text{ and } v \text{ with } d_T(u,v) = i.}
\]