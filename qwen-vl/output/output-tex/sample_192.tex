It looks like you're describing a process related to mesh refinement and coarsening, commonly used in finite element analysis or computational simulations. Here's a breakdown of what each part might represent:

1. **Left: Initial Refinement Level**
   - This typically refers to the starting point of a mesh where the elements are relatively fine, meaning they are small and numerous. This is often done to ensure high accuracy but can be computationally expensive.

2. **Middle: Coarsening Step**
   - In this step, the mesh is coarsened, meaning the number of elements is reduced by combining them into larger elements. This is usually done to reduce computational cost while maintaining acceptable accuracy. The image shows that all four element families have been coarsened, indicating that the mesh has been simplified across multiple types of elements.

3. **Right: Second Coarsening Step**
   - This further reduces the number of elements by coarsening the mesh again. The resulting family of elements from the first coarsening step is being refined even more. This could be necessary if the initial coarsening did not achieve the desired balance between accuracy and computational efficiency.

### Example Context:
- **Finite Element Analysis (FEA):** In FEA, meshes are refined to capture complex geometries and material properties accurately. Coarsening is then used to optimize the simulation for faster computation without significantly compromising accuracy.
- **Computational Fluid Dynamics (CFD):** Similar principles apply in CFD, where meshes are refined near boundaries and regions of high flow gradients and coarsened in less critical areas to save computational resources.

If you need further clarification or have specific questions about these steps, feel free to ask!