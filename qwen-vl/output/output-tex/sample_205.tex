The symbols \(\Delta\), \(\Delta^-\), \(\Delta_0\), and \(\Delta^+\) often appear in various contexts in physics, particularly in particle physics where they represent different types of resonances or particles. However, without additional context, it's challenging to provide a precise schematic picture. Let's break down what these symbols might represent:

1. **\(\Delta\) (Delta Resonance)**: This is a resonance in particle physics that appears in the decay of baryons like the proton and neutron. It is a collective excitation of quarks within the baryon.

2. **\(\Delta^-\)**: This could be a specific state of the \(\Delta\) resonance with negative parity. In particle physics, the parity of a particle is a measure of how its properties change under spatial inversion.

3. **\(\Delta_0\)**: This might refer to a specific state of the \(\Delta\) resonance with zero orbital angular momentum. The subscript 0 indicates this.

4. **\(\Delta^+\)**: This could be another state of the \(\Delta\) resonance with positive parity.

Given the notation \(r = \frac{\tau}{2n}\) and \(\delta = \frac{1}{2m}\), these parameters might be related to the resolution or the scale at which you are analyzing the \(\Delta\) states. For instance, if \(n\) and \(m\) are large, then \(r\) and \(\delta\) become small, indicating high precision in your analysis.

If you are dealing with a specific theoretical framework or model, such as quantum chromodynamics (QCD), these parameters might be used to describe the energy levels or the coupling constants in the system.

Here’s a simplified schematic representation of these states:

- **\(\Delta\)**: A general \(\Delta\) resonance.
- **\(\Delta^-\)**: A \(\Delta\) resonance with negative parity.
- **\(\Delta_0\)**: A \(\Delta\) resonance with zero orbital angular momentum.
- **\(\Delta^+\)**: A \(\Delta\) resonance with positive parity.

The parameters \(r\) and \(\delta\) would be used to specify the energy levels or the resolution of the analysis in a more detailed diagram or table.

If you have a specific context or a reference paper, please provide more details so I can give a more accurate and detailed schematic.