The posterior mean of the total \( R^2 \) over time is a measure that quantifies the proportion of variance in the data explained by the model across all time points. This metric is particularly useful in assessing the overall performance of a model, especially in dynamic or time-series contexts.

In the context of your question, you are comparing different priors for Bayesian regression models. These priors include:

1. **ARR2 prior with flat concentration**: This prior is designed to be non-informative, meaning it does not strongly influence the posterior distribution unless there is strong evidence from the data.
2. **ARR2 prior with Minnesota concentration**: This prior is more informative than the flat concentration version, incorporating a specific form of concentration parameter that can help stabilize the posterior distribution.
3. **Sparsity-inducing concentration**: This prior is specifically designed to induce sparsity in the model coefficients, which can be beneficial when dealing with high-dimensional data where many predictors may have no significant effect on the response variable.
4. **Minnesota-type prior**: This is another type of prior that is known for its ability to provide a balance between shrinkage and flexibility, making it suitable for a wide range of applications.
5. **Regularized horseshoe prior (RHS)**: This is a sparsity-inducing prior that has been shown to perform well in high-dimensional settings, where many predictors are likely to have zero effects on the response variable.

Each of these priors will influence the posterior mean of the total \( R^2 \) differently due to their varying levels of informativeness and sparsity-inducing properties. The posterior mean of the total \( R^2 \) under each prior can be calculated using Bayesian inference techniques, such as Markov Chain Monte Carlo (MCMC), and then averaged over the posterior samples.

To summarize, the posterior mean of the total \( R^2 \) over time provides a comprehensive measure of how well the model explains the data across all time points, and the comparison of different priors helps to understand how these choices affect the model's performance. The priors listed above are designed to balance between flexibility and sparsity, with some being more informative than others, and the choice of prior can significantly impact the resulting posterior distributions and thus the posterior mean of the total \( R^2 \).