The image you've described seems to be related to an algorithm called "Greedy" which is likely used in the context of a specific problem, possibly involving data structures or algorithms for managing and querying a set of keys (or elements) that form rectangles. The red points represent the keys that are being searched for, while the blue points represent the keys that have been touched by the Greedy algorithm at any given time \( i \). These blue points ensure that all the rectangles are arborally satisfied at time \( i \).

To provide more detail, let's break down the components:

1. **Red Points**: These points indicate the keys that are currently being searched for by the Greedy algorithm. This could mean that these keys are part of the current query or operation being performed.

2. **Blue Points**: These points represent the keys that the Greedy algorithm has interacted with up to time \( i \). The interaction here means that the algorithm has either added these keys to its structure or has used them to satisfy certain conditions (in this case, ensuring that all rectangles are arborally satisfied).

3. **Arborally Satisfied**: This term suggests that the algorithm is maintaining some kind of tree-like structure where each node (key) satisfies certain properties relative to its parent nodes. For example, in a binary search tree, each node is greater than all nodes in its left subtree and less than all nodes in its right subtree. In the context of rectangles, it might mean that the structure formed by the keys ensures that certain geometric or logical conditions are met.

Given the context, the Greedy algorithm is likely trying to optimize some aspect of the key arrangement or query processing. It might be a variant of a greedy algorithm used in computational geometry, database management systems, or other areas where efficient querying and updating of data structures are crucial.

If you need further clarification on how the Greedy algorithm works or what the specific problem is, please provide more details about the context or the specific problem you're referring to.