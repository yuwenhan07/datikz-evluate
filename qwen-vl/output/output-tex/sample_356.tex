It seems like you're referring to a piecewise function or a situation where a variable \( x \) is divided into two cases based on its value relative to a constant \( b \). Let's break it down step by step.

### Case 1: \( x \leq \frac{b}{2} \)

In this case, we consider the behavior of the function when \( x \) is less than or equal to half of \( b \).

### Case 2: \( x > \frac{b}{2} \)

In this case, we consider the behavior of the function when \( x \) is greater than half of \( b \).

### Example

Let's consider a simple example of a piecewise function:

\[ f(x) = \begin{cases} 
x^2 & \text{if } x \leq \frac{b}{2} \\
2x - 1 & \text{if } x > \frac{b}{2}
\end{cases} \]

Here:
- For \( x \leq \frac{b}{2} \), the function is \( f(x) = x^2 \).
- For \( x > \frac{b}{2} \), the function is \( f(x) = 2x - 1 \).

### Summary

The left side (\( x \leq \frac{b}{2} \)) and the right side (\( x > \frac{b}{2} \)) represent different intervals for the variable \( x \). The function behaves differently in each interval, as defined by the piecewise conditions.

If you have a specific problem or need further clarification on how to apply these cases, please provide more details!