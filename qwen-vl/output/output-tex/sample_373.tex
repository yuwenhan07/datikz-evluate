Your description outlines a comprehensive approach to autonomous navigation and path planning in a rover context, leveraging advanced technologies such as Graph-Based Simultaneous Localization and Mapping (SLAM) and 3D perception. Here’s a breakdown of the key components and steps involved:

### 1. **Global Traversability Map Construction**
   - **Graph-Based SLAM**: This involves building a map of the environment by estimating the robot's position and orientation while simultaneously mapping the surrounding area.
   - **3D Perception**: Utilizing sensors like LiDAR or stereo cameras to create a detailed 3D representation of the environment.

### 2. **Traversability Scoring and Frontier Detection**
   - **Traversability Scores**: Based on the rover's capabilities (e.g., wheelbase, ground clearance, etc.), the system assigns traversability scores to different areas of the map. These scores reflect how easily the rover can navigate through those areas.
   - **Thresholding**: Applying thresholds to these scores helps identify regions that are more navigable than others.
   - **Frontier Detection**: Frontiers are the boundaries between explored and unexplored regions. The system identifies these boundaries where the rover can potentially expand its exploration.

### 3. **Goal Definition and Ranking**
   - **Goals Definition**: For each frontier identified, goals are defined. These could be points of interest, strategic locations, or objectives that the rover needs to reach.
   - **Information Gain**: Goals are ranked based on their potential information gain. This means that the system prioritizes goals that provide the most new information about the environment or are critical for achieving the mission objectives.

### 4. **Path Safety Evaluation**
   - **Predicted Perception Entropies**: To ensure safe navigation, the system evaluates the path safety for each goal. This involves predicting the uncertainty (entropy) in the rover's perception at various points along the path. Higher entropy indicates higher uncertainty, which might require additional sensor data or slower speeds to ensure safety.
   - **Safety Constraints**: Depending on the mission constraints (e.g., time limits, energy levels, safety protocols), the system adjusts the path to ensure it adheres to these constraints while maintaining safety.

### 5. **Final Path Selection and Execution**
   - **Path Selection**: After evaluating all the factors, the system selects the optimal path that balances exploration, safety, and mission objectives.
   - **Execution**: The selected path is then executed by the rover, with continuous updates from the SLAM and perception systems to ensure real-time adjustments if necessary.

### Summary
This framework provides a robust and adaptive approach to autonomous navigation and path planning. By integrating SLAM, 3D perception, and sophisticated path planning algorithms, it ensures that the rover can efficiently explore its environment while adhering to safety and mission constraints. This method is particularly useful in complex and dynamic environments where real-time decision-making is crucial.