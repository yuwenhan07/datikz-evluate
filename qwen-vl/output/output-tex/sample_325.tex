To provide you with a nice tree decomposition of the Partial Ancestral Function (PAF) of Example~\ref{ex:paf} and the corresponding tables for computing \( P^{\mathit{ext}}_{com}(\{a, c, e\}) \), we need to first understand the structure of the PAF and the tree decomposition.

### Example PAF Structure

Let's assume Example~\ref{ex:paf} is a PAF that represents a specific ancestral relationship among nodes in a graph. For simplicity, let's consider a small example where the PAF is defined over a set of nodes \(\{a, b, c, d, e\}\).

The PAF can be represented as a function \( P \) such that:
\[ P(x, y) = \begin{cases} 
1 & \text{if } x \text{ is an ancestor of } y \\
0 & \text{otherwise}
\end{cases} \]

For instance, if the PAF is given by:
\[ P(a, b) = 1, \quad P(b, c) = 1, \quad P(c, d) = 1, \quad P(c, e) = 1 \]
\[ P(d, e) = 1, \quad P(e, f) = 1 \]
and so on, then this would mean that \(a\) is an ancestor of \(b\), \(b\) is an ancestor of \(c\), and so forth.

### Tree Decomposition

A tree decomposition of a graph is a tree where each node (called a bag) contains a subset of the vertices of the original graph, and the bags satisfy certain conditions:
1. The union of all bags covers all vertices.
2. For every edge \((u, v)\) in the original graph, there exists a bag containing both \(u\) and \(v\).
3. For every vertex \(v\), the bags containing \(v\) form a connected subtree.

For our example, let's assume the tree decomposition is as follows:

- Bag 1: \(\{a, b, c\}\)
- Bag 2: \(\{b, c, d\}\)
- Bag 3: \(\{c, d, e\}\)

This tree decomposition covers all nodes and satisfies the conditions mentioned above.

### Computing \( P^{\mathit{ext}}_{com}(\{a, c, e\}) \)

The function \( P^{\mathit{ext}}_{com} \) computes the probability of a set of nodes being connected through common ancestors. Given the tree decomposition and the PAF, we can compute this probability step-by-step.

#### Step 1: Identify Relevant Bags
We need to find the bags that contain at least one of the nodes in the set \(\{a, c, e\}\). In our example:
- Bag 1: \(\{a, b, c\}\)
- Bag 2: \(\{b, c, d\}\)
- Bag 3: \(\{c, d, e\}\)

#### Step 2: Compute \( P^{\mathit{ext}}_{com} \) for Each Pair of Nodes
We will compute \( P^{\mathit{ext}}_{com} \) for each pair of nodes in \(\{a, c, e\}\):

1. **\( P^{\mathit{ext}}_{com}(\{a, c\}) \):**
   - Bag 1: \(\{a, b, c\}\)
   - Bag 2: \(\{b, c, d\}\)
   - Bag 3: \(\{c, d, e\}\)
   - Both \(a\) and \(c\) are in Bag 1, so \( P^{\mathit{ext}}_{com}(\{a, c\}) = 1 \).

2. **\( P^{\mathit{ext}}_{com}(\{a, e\}) \):**
   - Bag 1: \(\{a, b, c\}\)
   - Bag 2: \(\{b, c, d\}\)
   - Bag 3: \(\{c, d, e\}\)
   - \(a\) is in Bag 1, and \(e\) is in Bag 3. There is no common bag containing both \(a\) and \(e\), so \( P^{\mathit{ext}}_{com}(\{a, e\}) = 0 \).

3. **\( P^{\mathit{ext}}_{com}(\{c, e\}) \):**
   - Bag 1: \(\{a, b, c\}\)
   - Bag 2: \(\{b, c, d\}\)
   - Bag 3: \(\{c, d, e\}\)
   - Both \(c\) and \(e\) are in Bag 3, so \( P^{\mathit{ext}}_{com}(\{c, e\}) = 1 \).

#### Step 3: Combine Results
Finally, we combine these results using the formula for \( P^{\mathit{ext}}_{com} \):
\[ P^{\mathit{ext}}_{com}(\{a, c, e\}) = \sum_{S \subseteq \{a, c, e\}} (-1)^{|S| + 1} P^{\mathit{ext}}_{com}(S) \]

For our example:
\[ P^{\mathit{ext}}_{com}(\{a, c, e\}) = (-1)^{1+1} P^{\mathit{ext}}_{com}(\{a\}) + (-1)^{2+1} P^{\mathit{ext}}_{com}(\{c\}) + (-1)^{3+1} P^{\mathit{ext}}_{com}(\{a, c\}) + (-1)^{2+1} P^{\mathit{ext}}_{com}(\{a, e\}) + (-1)^{1+1} P^{\mathit{ext}}_{com}(\{c, e\}) + (-1)^{3+1} P^{\mathit{ext}}_{com}(\{a, c, e\}) \]

Since \( P^{\mathit{ext}}_{com}(\{a\}) = 0 \), \( P^{\mathit{ext}}_{com}(\{c\}) = 0 \), and \( P^{\mathit{ext}}_{com}(\{a, e\}) = 0 \), the expression simplifies to:
\[ P^{\mathit{ext}}_{com}(\{a, c, e\}) = 1 + 0 + 1 + 0 + 1 + 0 = 3 \]

However, since probabilities must be between 0 and 1, we normalize this result:
\[ P^{\mathit{ext}}_{com}(\{a, c, e\}) = \frac{3}{8} \]

Thus, the final answer is:
\[ \boxed{\frac{3}{8}} \]