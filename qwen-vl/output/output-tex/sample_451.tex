A universal quantum ansatz function for a system of \( n = 3 \) qubits can be constructed using a combination of single-qubit rotations (specifically, \( R_y \)) and controlled-NOT (CNOT) gates. The goal is to create a function that can approximate any unitary transformation on the three-qubit system.

Here's an example of such a universal quantum ansatz function:

\[
\hat{U}({\bm \lambda}) = \prod_{i=1}^{m} \left( R_y(\lambda_{i,0}) \otimes I_2 \right) \otimes \left( I_2 \otimes R_y(\lambda_{i,1}) \right) \otimes \left( I_2 \otimes I_2 \otimes R_y(\lambda_{i,2}) \right) \cdot \text{CNOT}(1,2) \cdot \text{CNOT}(2,3)
\]

Where:
- \( m \) is the number of terms in the product.
- \( \lambda_{i,j} \) are parameters that control the rotation angles around the \( y \)-axis for each qubit.
- \( R_y(\lambda_{i,j}) \) represents a single-qubit rotation by angle \( \lambda_{i,j} \) about the \( y \)-axis.
- \( I_2 \) is the identity operator on a two-qubit system.
- \( \text{CNOT}(a,b) \) is a controlled-NOT gate where the first qubit (controlled qubit) is \( a \) and the second qubit (target qubit) is \( b \).

### Explanation:
1. **Single-Qubit Rotations**: Each term in the product involves a single-qubit rotation \( R_y(\lambda_{i,j}) \) applied to one of the qubits. This allows us to manipulate the state of individual qubits independently.
2. **Controlled-NOT Gates**: The CNOT gates are used to entangle the qubits. Specifically, we use two CNOT gates: one between qubits 1 and 2, and another between qubits 2 and 3. These gates ensure that the states of the qubits are correlated in a way that cannot be achieved with only single-qubit operations.

### Example with Specific Parameters:
For simplicity, let's consider a specific case with \( m = 2 \):

\[
\hat{U}({\bm \lambda}) = \left( R_y(\lambda_{1,0}) \otimes I_2 \right) \otimes \left( I_2 \otimes R_y(\lambda_{1,1}) \right) \otimes \left( I_2 \otimes I_2 \otimes R_y(\lambda_{1,2}) \right) \cdot \text{CNOT}(1,2) \cdot \text{CNOT}(2,3)
\]

\[
\cdot \left( R_y(\lambda_{2,0}) \otimes I_2 \right) \otimes \left( I_2 \otimes R_y(\lambda_{2,1}) \right) \otimes \left( I_2 \otimes I_2 \otimes R_y(\lambda_{2,2}) \right) \cdot \text{CNOT}(1,2) \cdot \text{CNOT}(2,3)
\]

This ansatz function includes four single-qubit rotations and two CNOT gates, which together provide a flexible way to manipulate the state of the three-qubit system.

### Generalization:
To make this ansatz more general, you could increase the number of terms \( m \) and include additional single-qubit rotations and CNOT gates as needed. For instance, you might add more terms like:

\[
\hat{U}({\bm \lambda}) = \cdots \cdot \left( R_y(\lambda_{3,0}) \otimes I_2 \right) \otimes \left( I_2 \otimes R_y(\lambda_{3,1}) \right) \otimes \left( I_2 \otimes I_2 \otimes R_y(\lambda_{3,2}) \right) \cdot \text{CNOT}(1,2) \cdot \text{CNOT}(2,3)
\]

This would give you a total of six single-qubit rotations and three CNOT gates, making the ansatz even more powerful.

By varying the parameters \( \lambda_{i,j} \), you can approximate any unitary transformation on the three-qubit system, thus making this ansatz universal.