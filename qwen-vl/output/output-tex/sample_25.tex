To solve the problem, we need to understand the structure of the cluster \(\mathcal{G}(P)\) and how its perimeter is calculated. The perimeter of a cluster in a grid is the number of edges that form the boundary of the cluster. In this case, the cluster has 11 vertices and a perimeter of 19.

Let's break down the problem step by step:

1. **Vertices and Edges**: A cluster with 11 vertices can be visualized as a connected graph with 11 nodes. The number of edges in a connected graph with \(n\) vertices is at least \(n-1\). For 11 vertices, the minimum number of edges is 10 (a tree), but since the cluster is not necessarily a tree, it could have more edges. However, the exact number of edges is not needed to determine the perimeter.

2. **Perimeter Calculation**: The perimeter of the cluster is the number of edges that form the boundary of the cluster. These are the edges that connect the vertices on the outermost layer of the cluster. If the cluster were a simple rectangle or square, the perimeter would be straightforward to calculate. However, since the cluster is not specified as a simple shape, we need to consider the general case where the perimeter is given as 19.

3. **Grid Structure**: The problem states that edges of the cluster running along the exterior of the grid do not contribute to the perimeter. This means that if the cluster extends beyond the grid boundaries, only the edges that form the boundary of the cluster itself are counted in the perimeter.

Given that the perimeter is 19, we can infer that the cluster must be a complex shape with many edges forming the boundary. The exact shape of the cluster is not specified, but the perimeter calculation is clear.

Therefore, the perimeter of the cluster \(\mathcal{G}(P)\) is \(\boxed{19}\).