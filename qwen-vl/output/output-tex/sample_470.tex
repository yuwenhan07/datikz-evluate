To solve this problem, we need to understand the structure of the wormhole geometry and how it is formed from the hyperbolic quadrilaterals.

First, let's break down the problem step-by-step:

1. **Understanding the Geometry**: The gravitational region of the wormhole geometry can be divided into \(8n+1\) pieces. Out of these, \(8n\) pieces have the geometry of a hyperbolic disk with a cut connecting two conical singularities, one at point \(R\) and another at point \(S\). The conical singularity at \(R\) has an opening angle \(2\alpha\) and the conical singularity at \(S\) has an opening angle \(2\beta\).

2. **Hyperbolic Quadrilaterals**: These hyperbolic disks are formed by identifying the edges of hyperbolic quadrilaterals. Specifically, the edges \(P'R'\) and \(P''R''\) are identified, as well as the edges \(Q'S'\) and \(Q''S''\). The angles at the vertices \(R'\) and \(R''\) are \(\alpha\) and the angles at the vertices \(S'\) and \(S''\) are \(\beta\).

3. **Counting the Pieces**: We need to determine how many of these \(8n+1\) pieces are hyperbolic disks with the specified conical singularities. Since each hyperbolic disk with the specified conical singularities is one of the \(8n\) pieces, we need to identify how many such pieces there are in total.

4. **Conclusion**: The problem states that out of the \(8n+1\) pieces, \(8n\) pieces have the geometry of a hyperbolic disk with the specified conical singularities. Therefore, the number of such pieces is exactly \(8n\).

Thus, the answer is \(\boxed{8n}\).