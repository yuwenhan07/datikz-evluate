The notation you've provided refers to the affine Dynkin diagrams associated with specific Lie algebras, which are fundamental in the study of quantum field theory and statistical mechanics. Let's break down the components:

1. **Affine Dynkin Diagrams**: These are generalizations of the finite Dynkin diagrams that appear in the classification of simple Lie algebras. Affine Dynkin diagrams are used to describe the structure of affine Lie algebras, which are important in the context of conformal field theories and integrable systems.

2. **New Scattering Theories**: These refer to scattering theories that have been recently discovered or proposed, often in the context of integrable models. These theories can be described using the underlying Lie algebra structures encoded in the Dynkin diagrams.

3. **Nodes**:
   - **Blue Node ($R$)**: This typically represents a "right" node, which could correspond to a particular type of particle or excitation in the scattering theory.
   - **Red Node ($L$)**: This represents a "left" node, similar to the blue node but possibly representing a different type of particle or excitation.
   - **Empty Node (Magnons)**: This represents magnons, which are excitations in the system that do not correspond to particles but rather to the collective behavior of the system.

4. **Green Links**: These denote "shifted universal kernels" by \(\alpha\). In the context of scattering theories, these kernels represent the interaction strengths between different types of excitations (nodes). The parameter \(\alpha\) is a scaling factor that modifies these interaction strengths.

### Specific Cases (\(p = 3, 4, 5, 6, 7\)):

- **\(p = 3\)**: This corresponds to the affine Dynkin diagram of \(A_2^{(1)}\), which is related to the \(SU(3)\) Lie algebra.
- **\(p = 4\)**: This corresponds to the affine Dynkin diagram of \(D_4^{(1)}\), which is related to the \(SO(8)\) Lie algebra.
- **\(p = 5\)**: This corresponds to the affine Dynkin diagram of \(E_6^{(1)}\), which is related to the \(E_6\) Lie algebra.
- **\(p = 6\)**: This corresponds to the affine Dynkin diagram of \(E_7^{(1)}\), which is related to the \(E_7\) Lie algebra.
- **\(p = 7\)**: This corresponds to the affine Dynkin diagram of \(E_8^{(1)}\), which is related to the \(E_8\) Lie algebra.

Each of these affine Dynkin diagrams encodes the structure of the corresponding affine Lie algebra, which in turn describes the symmetries and interactions in the scattering theory. The specific nodes and their interactions (encoded by the green links) determine the detailed properties of the scattering amplitudes and the nature of the excitations in the system.

In summary, the notation you've provided is a way to visualize and understand the structure of certain scattering theories through the lens of affine Dynkin diagrams and their associated Lie algebras. Each diagram provides a blueprint for the symmetries and interactions within the theory.