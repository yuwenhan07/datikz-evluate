The figures you've described seem to be related to theoretical physics, specifically in the context of string theory or particle physics, where concepts like the convex hull, distance from the origin, and couplings play significant roles.

### Top Figure:
The top figure illustrates the concept of a convex hull, which is a geometric structure that encloses a set of points. In this case, the convex hull is formed by potential couplings (\(\gamma_i\)) and axion couplings (\(\lambda_r\)). The distance from the origin \(\gamma_\infty\) is represented as a purple vector, and it is defined such that it is perpendicular to the plane spanned by the \(\gamma_i\) and \(\lambda_r\) vectors. The condition \(\gamma_\infty \cdot \lambda_r < 0\) indicates that the vector \(\gamma_\infty\) points in a direction that is opposite to the direction of the \(\lambda_r\) vectors when projected onto the plane defined by \(\gamma_i\) and \(\lambda_r\).

This setup is often used in the context of string theory or effective field theories to understand the constraints on the couplings and their impact on the physical observables, such as the late-time \(\epsilon\)-parameter.

### Bottom Figure:
The bottom figure appears to show a relevant part of the canonical-scalar theory. This typically refers to a scalar field theory where the Lagrangian is written in terms of a canonical scalar field \( \phi \). The canonical scalar theory is often used to describe the dynamics of fields in various contexts, including cosmology and particle physics.

In the context of the late-time \(\epsilon\)-parameter, the \(\epsilon\)-parameter is a measure of the deviation from the standard model predictions for the expansion rate of the universe at late times. It is particularly important in the context of inflationary models and the early universe.

### Relationship Between the Figures:
The top figure's convex hull and the conditions on the couplings suggest that there are constraints on the possible values of the couplings \(\gamma_i\) and \(\lambda_r\). These constraints are likely related to the stability and consistency of the theory. The bottom figure, showing the canonical-scalar theory, is the framework within which these couplings are embedded. The late-time \(\epsilon\)-parameter is computed using the dynamics of the scalar field, and the constraints from the convex hull might influence the allowed values of the parameters in the canonical-scalar theory.

In summary, the top figure provides a geometric constraint on the couplings, while the bottom figure describes the theoretical framework within which these couplings are studied. The interplay between these two aspects helps in understanding the physical implications of the couplings on the late-time behavior of the universe.