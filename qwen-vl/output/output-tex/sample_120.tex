To analyze the number of steps taken by the RBQL agent to find the goal in a randomly generated grid world maze of size \(50 \times 50\), we need to understand the data presented in the image. The image provides a visual representation of the distribution of the number of steps taken by the agent.

1. **Identify the Key Elements:**
   - The light green area represents the range of the highest and lowest step counts, excluding the highest and lowest two.
   - The green line represents the average performance of the agent.

2. **Interpret the Light Green Area:**
   - The light green area is a box plot that shows the interquartile range (IQR) of the step counts. The IQR is the range between the first quartile (Q1) and the third quartile (Q3). This means that the middle 50% of the step counts fall within this range.
   - The lower edge of the light green area is Q1, and the upper edge is Q3. The median (Q2) is typically represented by a line inside the box, but since it's not explicitly shown, we can infer it from the position of the box.

3. **Interpret the Green Line:**
   - The green line represents the average number of steps taken by the agent. It gives us an idea of the typical performance of the agent across multiple trials.

4. **Conclusion:**
   - The light green area gives us an idea of the variability in the number of steps taken by the agent. If the light green area is narrow, it indicates that the agent consistently performs around the same number of steps. If the light green area is wide, it indicates a higher variability in the number of steps taken.
   - The green line gives us the central tendency of the performance, which helps us understand the typical number of steps taken by the agent.

Since the image does not provide specific numerical values for Q1, Q3, or the median, we can only describe the general characteristics based on the visual representation. However, if we assume that the image is typical and follows standard practices in data visualization, we can infer that the agent's performance is relatively consistent with some variability.

Therefore, the answer to the question "What is the number of steps taken to find the goal in a randomly generated grid world maze of size \(50 \times 50\)?" is best described by the green line, which represents the average performance. Without specific numerical values, we can only say that the average number of steps taken by the agent is around the value indicated by the green line.

The final answer is:
\[
\boxed{\text{The average number of steps taken by the agent is around the value indicated by the green line.}}
\]