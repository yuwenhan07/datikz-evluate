To summarize the given information graphically for the case where the integer point \((0,4)\) is missing from \(\mathrm{Newt}(\Omega)\) and we are dealing with the minimal resolution \(\mathbb{F}_4 \to \mathbb{P}(1,1,4)\):

1. **Identify the Minimal Resolution**: The minimal resolution of \(\mathbb{P}(1,1,4)\) is denoted as \(\mathbb{F}_4\).

2. **Exceptional Divisor**: There is an exceptional divisor associated with this resolution.

3. **Moving the Edge**: The edge corresponding to the exceptional divisor is moved normally inwards until it reaches an integer point. This movement is such that the new edge has an affine length of 4.

4. **Intersection Multiplicity**: The strict transform of the branch curve intersects the contracted \(-4\)-curve \(C_1\) with total multiplicity 4. This means that at each intersection point, the branch curve and the \(-4\)-curve intersect with a multiplicity of 4.

5. **Affine Distance**: The affine distance from the missing point \((0,4)\) to the new edge is 1. This implies that the curve \(C_1\) appears in \(\mathrm{tot}_{\tilde{Y}}(B)\) with multiplicity 1.

The graphical representation can be summarized as follows:

- Start with the projective space \(\mathbb{P}(1,1,4)\).
- Apply the minimal resolution \(\mathbb{F}_4 \to \mathbb{P}(1,1,4)\), which introduces an exceptional divisor.
- Move the edge corresponding to the exceptional divisor inwards until it reaches an integer point, resulting in a new edge with an affine length of 4.
- The strict transform of the branch curve intersects the contracted \(-4\)-curve \(C_1\) with a total multiplicity of 4.
- The affine distance from the missing point \((0,4)\) to the new edge is 1, indicating that the curve \(C_1\) appears in \(\mathrm{tot}_{\tilde{Y}}(B)\) with multiplicity 1.

In terms of a diagram, you would typically draw the projective space \(\mathbb{P}(1,1,4)\) and then show the minimal resolution \(\mathbb{F}_4\). You would then indicate the exceptional divisor and the movement of the edge, showing the new position of the edge after the movement. Finally, you would illustrate the intersection of the branch curve with the \(-4\)-curve \(C_1\) and mark the multiplicity of the intersection points.

Since this is a textual summary, the final answer is:

\[
\boxed{\text{The strict transform of the branch curve intersects the contracted \(-4\)-curve \(C_1\) with total multiplicity 4, and the curve \(C_1\) appears in \(\mathrm{tot}_{\tilde{Y}}(B)\) with multiplicity 1.}}
\]