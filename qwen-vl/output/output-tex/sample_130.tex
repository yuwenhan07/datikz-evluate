To understand the depiction of a Bernoulli path \( h_0 \) and the associated particle configuration \( \eta_{h_0} \), let's break down the components and their relationships:

1. **Bernoulli Path \( h_0 \)**:
   - A Bernoulli path is a random walk on a lattice where each step is either up or down with equal probability.
   - In a one-dimensional context, it can be visualized as a sequence of steps that can either increase or decrease by 1 at each step.

2. **Particle Configuration \( \eta_{h_0} \)**:
   - The particle configuration \( \eta_{h_0} \) is a way to represent the state of the system at a given time.
   - Each point in the lattice is occupied by a particle, and the configuration tells us which points are occupied and which are not.
   - For a Bernoulli path, the particle configuration can be thought of as a binary string where each bit represents whether a particle is present (1) or absent (0) at a particular position along the path.

3. **Dashed Portions of the Path**:
   - The dashed portions of the path indicate the definition of the new height function.
   - The new height function is typically defined based on the particle configuration. For example, if we have a particle at a certain position, the height at that position might be increased by some value, and if there is no particle, the height remains unchanged.
   - This new height function is used to define the dynamics of the system, such as the movement of particles or the evolution of the path itself.

4. **Relationship Between Path and Particle Configuration**:
   - The particle configuration \( \eta_{h_0} \) provides the initial state of the system, which influences the evolution of the path \( h_0 \).
   - As the path evolves according to the rules of the system (e.g., particle movements), the particle configuration changes, and this change is reflected in the new path and the new height function.

In summary, the Bernoulli path \( h_0 \) is a random walk that defines the initial state of the system, and the particle configuration \( \eta_{h_0} \) describes the state of the system at that initial time. The dashed portions of the path illustrate how the new height function is defined based on the particle movements, which in turn influence the evolution of the path over time.

The final answer is:

\[
\boxed{\text{The Bernoulli path } h_0 \text{ is a random walk that defines the initial state of the system, and the particle configuration } \eta_{h_0} \text{ describes the state of the system at that initial time. The dashed portions of the path illustrate how the new height function is defined based on the particle movements, which in turn influence the evolution of the path over time.}}
\]