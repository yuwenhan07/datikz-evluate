It seems like you're referring to different types of graphs in graph theory, but the notation is not standard or widely recognized. Let's break down each term to see if we can identify what they might be:

1. **\( K_{1,n}^* \)**: This could refer to a modified version of the complete bipartite graph \( K_{1,n} \). The standard \( K_{1,n} \) has one vertex connected to all \( n \) vertices in the other part, but the superscript asterisk (\( ^* \)) suggests some modification. Without more context, it's hard to specify exactly what this modification is.

2. **\( K_n^* \)**: This could also be a modified version of the complete graph \( K_n \), which is a graph with \( n \) vertices where every pair of distinct vertices is connected by a unique edge. Again, the asterisk suggests some modification, but without further information, it's unclear what that modification is.

3. **\( C\!K_n \)**: This looks like a combination of "C" and "K_n". If "C" stands for "cycle," then \( C\!K_n \) could be interpreted as a cycle graph with \( n \) vertices, which is a simple closed path where each vertex is connected to two others. However, the notation is unconventional, so it's possible there's a typo or a specific meaning intended that isn't clear from the standard graph theory notation.

4. **\( B\!S_n^p \)**: This appears to be a combination of "B" and "S_n", possibly indicating a specific type of graph related to permutations or symmetric groups. The superscript \( p \) suggests a parameterized version of this structure. Without more context, it's difficult to determine the exact nature of this graph.

5. **\( F_n \)**: This could be a specific type of graph, such as a Fibonacci graph, which is a graph constructed using the Fibonacci sequence. Alternatively, it could be a different type of graph depending on the context in which it is used.

Given the lack of standard notation, it would be helpful to have more context or a reference to understand these graphs better. If you have a specific source or a particular field of study in mind, please provide more details, and I can help you interpret these graphs more accurately.