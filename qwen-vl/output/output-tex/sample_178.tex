To solve the problem, we need to understand the structure of the vector bundle \(W = \Sym^3 V \oplus \Sym^2 V \otimes \det V\) and its Newton polygon \(\Lambda^u\) at a point \(u \in \mathbb{P}^1\) with given vanishing orders \(r_1 = 2\) and \(r_2 = 0\).

The Newton polygon \(\Lambda^u\) is a geometric representation of the vanishing orders of the sections of the vector bundle \(W\) at the point \(u\). The vanishing orders \(r_1 = 2\) and \(r_2 = 0\) indicate that the section corresponding to \(\Sym^3 V\) has a vanishing order of 2 at \(u\), while the section corresponding to \(\Sym^2 V \otimes \det V\) has a vanishing order of 0 at \(u\).

The Newton polygon \(\Lambda^u\) is constructed by plotting the points \((i, r_i)\) for each term in the direct sum, where \(i\) is the degree of the symmetric power and \(r_i\) is the vanishing order at \(u\). For our bundle, this gives us the points \((3, 2)\) and \((2, 0)\).

The Newton polygon is the convex hull of these points. In this case, the Newton polygon is a line segment connecting the points \((3, 2)\) and \((2, 0)\).

The short normal vectors (dashed) represent the canonical Newton polygon \(\beta^{\rm can}\), which is the Newton polygon obtained by considering the degrees of the symmetric powers without regard to their multiplicities. Here, it is the same as the Newton polygon \(\Lambda^u\) because there are no other terms contributing to the Newton polygon.

The longer normal vectors (dotted) represent the residual Newton polygon \(\beta^{\rm res}\), which is the Newton polygon obtained by subtracting the canonical Newton polygon from the actual Newton polygon. However, since the canonical Newton polygon is the same as the actual Newton polygon in this case, the residual Newton polygon is the zero vector.

Therefore, the Newton polygon \(\Lambda^u\) is simply the line segment connecting \((3, 2)\) and \((2, 0)\).

The final answer is:
\[
\boxed{(3, 2) \text{ to } (2, 0)}
\]