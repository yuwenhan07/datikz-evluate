It seems you're discussing a topic related to numerical analysis, specifically concerning the computation of recurrence coefficients for probability measures using the Conjugate Convolution (CC) algorithm. The notation and context suggest that you might be dealing with a specific type of measure or distribution.

To provide more clarity and detail, let's break down the key points:

1. **Recurrence Coefficients**: These are parameters in the recurrence relation for orthogonal polynomials associated with a given measure. They play a crucial role in spectral theory and numerical methods for solving differential equations.

2. **Conjugate Convolution (CC) Algorithm**: This is a method used to compute the recurrence coefficients of a measure from its moments. It is particularly useful when dealing with measures on the real line.

3. **Upper Boundary**: In the context of numerical analysis, the "upper boundary" could refer to the boundary condition or the highest point in a certain range of values. Here, it likely refers to the upper limit of the interval over which the measure is defined.

4. **${\widehat{\mu}}_1$**: This appears to denote a transformed or modified version of the original measure $\mu_1$. The hat symbol often indicates a transformation or a different representation of the measure.

Given these points, the statement suggests that the recurrence coefficients for the measure ${\widehat{\mu}}_1$, which is presumably obtained by some transformation or modification of $\mu_1$, can be computed using the CC algorithm. Specifically, the coefficients for ${\widehat{\mu}}_1$ are found on the upper boundary of the interval where the measure is defined.

If you have a specific reference or theorem (e.g., Theorem~\ref{thm:NNCC}) that you are referring to, it would be helpful to provide more details so I can give a more precise answer. However, based on the general context, the statement is likely correct if the CC algorithm is indeed applicable to the transformed measure ${\widehat{\mu}}_1$ and the upper boundary condition is correctly interpreted.