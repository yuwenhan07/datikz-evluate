The graph you've described is likely a comparison between the predicted and actual percentage changes in area reduction for a design, as determined by ROVER (a tool used for area estimation) during logic synthesis with the goal of minimizing delay. Here’s a breakdown of what each part of your description means:

1. **ROVER's Predicted Percentage Change vs. Actual Percentage Change**: This indicates how well ROVER predicts the actual area reduction achieved through logic synthesis.

2. **Minimum Delay Target**: The synthesis process aims to achieve the lowest possible delay while optimizing other parameters like area.

3. **Points Above/Below the Diagonal**: 
   - Points above the diagonal suggest that ROVER over-predicts the area reduction.
   - Points below the diagonal suggest that ROVER under-predicts the area reduction.

4. **Omitting MCM Results**: This implies that the results from the Multi-Chip Module (MCM) are not included in this analysis. MCM results might be different due to additional complexities and interactions between chips.

5. **Red Lines Representing Synthesis Noise Window**: These lines define a range within which the actual area reduction is expected to fall, given the inherent variability or "noise" in the synthesis process. This helps in understanding the reliability of the predictions made by ROVER.

In summary, this graph is a visual representation of how accurate ROVER's predictions are compared to the actual outcomes of the logic synthesis process, with an emphasis on the area reduction achieved. The red lines help in assessing the confidence interval around these predictions.