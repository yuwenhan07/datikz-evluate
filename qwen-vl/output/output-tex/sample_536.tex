To solve the problem of constructing intersecting families whose union is bigger than the union of stars, we need to understand the definitions and properties of the sets involved. Let's break down the problem step by step.

1. **Understanding the Sets:**
   - The red boxes represent the parts of the ground set \( S \). Let's denote the sizes of these parts as \( |A_1|, |A_2|, \ldots, |A_k| \).
   - The green ellipses represent 2-element sets that define the family \( \mathcal{T}_i \). Each \( \mathcal{T}_i \) is a collection of 2-element subsets of \( S \).
   - The blue vertex \( y \) and the set \( H_y^i \) define the family \( \mathcal{HM}_y^i \).

2. **Constructing Intersecting Families:**
   - We need to construct two intersecting families \( \mathcal{F} \) and \( \mathcal{G} \) such that the union of these families is larger than the union of the stars of the elements in \( S \).

3. **Stars of Elements:**
   - The star of an element \( x \in S \), denoted by \( \text{star}(x) \), is the union of all sets in \( \mathcal{T}_i \) that contain \( x \).
   - The union of all stars of the elements in \( S \) is the union of all sets in \( \mathcal{T}_i \).

4. **Intersecting Families:**
   - An intersecting family is a family where any two sets in the family have at least one element in common.
   - To ensure that the union of our intersecting families is larger than the union of the stars, we need to carefully choose the sets in \( \mathcal{F} \) and \( \mathcal{G} \).

5. **Example Construction:**
   - Suppose \( S = \{a, b, c, d\} \) and \( \mathcal{T}_i = \{\{a, b\}, \{b, c\}, \{c, d\}\} \).
   - The stars of the elements are:
     - \( \text{star}(a) = \{a, b\} \)
     - \( \text{star}(b) = \{a, b, c\} \)
     - \( \text{star}(c) = \{b, c, d\} \)
     - \( \text{star}(d) = \{c, d\} \)
   - The union of all stars is \( \{a, b, c, d\} \).

   - We can construct two intersecting families \( \mathcal{F} \) and \( \mathcal{G} \) as follows:
     - \( \mathcal{F} = \{\{a, b\}, \{b, c\}\} \)
     - \( \mathcal{G} = \{\{b, c\}, \{c, d\}\} \)

   - The union of \( \mathcal{F} \) and \( \mathcal{G} \) is \( \{\{a, b\}, \{b, c\}, \{c, d\}\} \), which has size 3.
   - The union of the stars is \( \{a, b, c, d\} \), which has size 4.

Therefore, the union of the intersecting families \( \mathcal{F} \) and \( \mathcal{G} \) is smaller than the union of the stars. However, this example does not satisfy the condition. We need to find a different construction.

6. **General Construction:**
   - Consider the ground set \( S = \{a_1, a_2, \ldots, a_n\} \) with sizes \( |A_1|, |A_2|, \ldots, |A_n| \).
   - Construct \( \mathcal{F} \) and \( \mathcal{G} \) such that each set in \( \mathcal{F} \) intersects with each set in \( \mathcal{G} \) and the union of \( \mathcal{F} \) and \( \mathcal{G} \) is larger than the union of the stars.

The final answer is:

\[
\boxed{\text{Construct two intersecting families } \mathcal{F} \text{ and } \mathcal{G} \text{ such that the union of } \mathcal{F} \text{ and } \mathcal{G} \text{ is larger than the union of the stars.}}
\]