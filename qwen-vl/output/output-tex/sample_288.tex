To illustrate the recursive encryption algorithm described, let's break down the process step by step.

### Initial Setup:
We have a \(3 \times 3\) pixel grid. Let's label the pixels as follows:

\[
\begin{array}{ccc}
A & B & C \\
D & E & F \\
G & H & I \\
\end{array}
\]

The event \(\mathbf{E}_c\) is triggered at the center pixel \(E\).

### First Layer of Recursion:
In this layer, we synthesize noise \(\mathbf{N}^{(1)}_i\) in the four spatially adjacent pixels (horizontally or vertically) to \(\mathbf{E}_c\). These pixels are \(A, B, D,\) and \(F\).

- For each of these pixels, the noise \(\mathbf{N}^{(1)}_i\) is synthesized such that its magnitude is equal to the magnitude of \(\mathbf{E}_c\). This means if \(\mathbf{E}_c\) has a certain intensity, then \(\mathbf{N}^{(1)}_i\) will also have the same intensity.

The resulting set of pixels after the first layer of recursion is:

\[
\begin{array}{ccc}
A & B & C \\
D & E & F \\
G & H & I \\
\end{array}
\]

where the noise \(\mathbf{N}^{(1)}_i\) is added to the pixels \(A, B, D,\) and \(F\).

### Second Layer of Recursion:
In the second layer, the algorithm synthesizes new noise \(\mathbf{N}^{(2)}_i\) based on the adjacent events from the first layer. The adjacent pixels to \(\mathbf{E}_c\) now include the newly added noise pixels from the first layer, specifically \(A, B, D,\) and \(F\).

The algorithm is blind to the specific values of \(\mathbf{N}^{(1)}_i\) and \(\mathbf{E}_c\), but it synthesizes \(\mathbf{N}^{(2)}_i\) based on the presence of these events. The resulting set of pixels after the second layer of recursion is:

\[
\begin{array}{ccc}
A' & B' & C' \\
D' & E' & F' \\
G' & H' & I' \\
\end{array}
\]

where \(A', B', D', F'\) are the pixels that were originally \(A, B, D,\) and \(F\) plus the noise \(\mathbf{N}^{(1)}_i\), and \(E'\) is the original \(E\) plus the noise \(\mathbf{N}^{(1)}_i\).

The algorithm continues this process recursively, adding more layers of noise synthesis until the desired level of encryption is achieved.

### Summary:
1. **First Layer**: Synthesize noise \(\mathbf{N}^{(1)}_i\) in the four adjacent pixels to \(\mathbf{E}_c\).
2. **Second Layer**: Synthesize new noise \(\mathbf{N}^{(2)}_i\) based on the presence of the events from the first layer.
3. **Recursive Layers**: Continue this process until the desired level of encryption is reached.

This recursive approach ensures that the noise is distributed across the entire grid, making it difficult to identify the original event \(\mathbf{E}_c\).