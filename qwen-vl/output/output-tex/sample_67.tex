The idea of using gradients to assess the validity of a local linearization is indeed a common approach in image processing and computer vision. Let's break down the reasoning step by step:

1. **Local Linearization**: In the context of image processing, a local linearization is often used to approximate the behavior of an image intensity function \( I(\mathbf{s}) \) around a point \(\mathbf{s}_0\). This approximation is typically done using a Taylor series expansion up to the first order:
   \[
   I(\mathbf{s}) \approx I(\mathbf{s}_0) + \nabla I(\mathbf{s}_0)^T (\mathbf{s} - \mathbf{s}_0)
   \]
   where \(\nabla I(\mathbf{s}_0)\) is the gradient of the image intensity at the point \(\mathbf{s}_0\).

2. **Assumption of Constant Gradient**: The local linearization assumes that the gradient \(\nabla I(\mathbf{s})\) is approximately constant over a small neighborhood around \(\mathbf{s}_0\). If the gradient changes significantly within this neighborhood, the linear approximation will become less accurate.

3. **Observations from Gradients**: The gradients \(\nabla I_i(\mathbf{s})\) for different points \(\mathbf{s}\) (e.g., \(\mathbf{s}_0, \mathbf{s}_1, \mathbf{s}_2\)) provide information about how the image intensity changes locally. If these gradients are similar, it suggests that the gradient is relatively constant in the neighborhood, supporting the validity of the local linearization.

4. **Validity Check**: To check the validity of the local linearization, one can compare the gradients at different points. If the gradients \(\nabla I_0(\mathbf{s})\), \(\nabla I_1(\mathbf{s})\), and \(\nabla I_2(\mathbf{s})\) are significantly different, it indicates that the gradient is not constant over the region, and thus the linearization may not be valid.

5. **Dotted Line Consideration**: The dotted line in your diagram represents a path or region where you want to ensure the gradient is constant for the linearization to hold. By examining the gradients along this line, you can determine whether the assumption of constant gradient is reasonable.

In summary, comparing the gradients at different points is a reasonable way to assess the validity of a local linearization. If the gradients are consistent across the region of interest, the linear approximation should be reliable. If there are significant differences in the gradients, the linearization may need to be refined or re-evaluated.