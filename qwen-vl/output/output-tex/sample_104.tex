To solve the problem, we need to understand the context and the meaning of the statement "The curve \(\nu_2\) passes twice through \(V(a_n, b_{n+j}, M_l)\)." This statement suggests that the point \(V(a_n, b_{n+j}, M_l)\) is a point on the curve \(\nu_2\) and it appears twice as a point on this curve.

Let's break down the components of the statement:

1. **Curve \(\nu_2\)**: This is a mathematical curve defined by some equation or set of parametric equations.
2. **Point \(V(a_n, b_{n+j}, M_l)\)**: This is a specific point in three-dimensional space with coordinates \((a_n, b_{n+j}, M_l)\).
3. **Passes twice**: This means that the point \(V(a_n, b_{n+j}, M_l)\) is a repeated point on the curve \(\nu_2\), i.e., there are two distinct parameter values (or points along the curve) that correspond to the same point in space.

Given these components, the statement can be interpreted as follows: There exist two different parameter values \(t_1\) and \(t_2\) such that the curve \(\nu_2\) evaluated at these parameters gives the point \(V(a_n, b_{n+j}, M_l)\). Mathematically, this can be written as:
\[
\nu_2(t_1) = V(a_n, b_{n+j}, M_l) \quad \text{and} \quad \nu_2(t_2) = V(a_n, b_{n+j}, M_l)
\]
where \(t_1 \neq t_2\).

Therefore, the answer to the problem is:
\[
\boxed{\text{The curve } \nu_2 \text{ has two distinct parameter values corresponding to the point } V(a_n, b_{n+j}, M_l).}
\]