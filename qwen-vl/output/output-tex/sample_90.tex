To compare the performance of the \gls{pcnc} scheme with the scheme presented in~\cite{antonioli2023mixed}, we need to consider several aspects such as computational complexity, accuracy, and efficiency. However, since you've specified parameters ($L=10$, $M=5$, $K=5$, and $N=1$), let's focus on how these parameters might influence the comparison.

### Parameters Explanation:
- \( L = 10 \): This likely refers to the length or size of the problem instance.
- \( M = 5 \): This could be related to the number of blocks or subproblems.
- \( K = 5 \): This might represent the number of iterations or steps in an algorithm.
- \( N = 1 \): This is typically a dimensionality parameter, often indicating that the problem is one-dimensional.

### Performance Comparison:
1. **Computational Complexity**:
   - The \gls{pcnc} scheme and the scheme from~\cite{antonioli2023mixed} may have different complexities depending on their algorithms. For example, if the \gls{pcnc} scheme uses a divide-and-conquer approach, it might have a lower complexity than a more iterative method.
   - The specific complexity would depend on the details of both schemes, but generally, the \gls{pcnc} scheme might be more efficient due to its parallel nature, especially when \( L \) is large.

2. **Accuracy**:
   - Both schemes aim to solve the same problem, so their accuracy should be comparable if they are designed to achieve similar levels of precision. However, the \gls{pcnc} scheme might offer better accuracy due to its ability to handle larger problems efficiently.

3. **Efficiency**:
   - Efficiency can be measured by how quickly the schemes converge to a solution. If the \gls{pcnc} scheme has fewer iterations (or steps) to reach a solution, it would be more efficient.
   - Given \( N = 1 \), the one-dimensional nature of the problem might not significantly impact the comparison unless there are specific optimizations in the \gls{pcnc} scheme that leverage this property.

### Conclusion:
Without the actual implementation details of both schemes, it's challenging to provide a precise comparison. However, based on the parameters provided:

- If the \gls{pcnc} scheme is designed to be highly parallel and efficient, it might outperform the scheme in~\cite{antonioli2023mixed} in terms of computational time and possibly accuracy.
- The specific performance will depend on the underlying algorithms and optimizations used in each scheme.

For a definitive comparison, you would need to run both schemes under the given parameters and measure their performance metrics such as runtime, accuracy, and resource usage.