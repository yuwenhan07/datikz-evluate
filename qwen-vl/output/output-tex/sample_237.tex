It seems like you're describing a graph structure involving paths and cycles, with some vertices colored black to represent an FTD-code (which might stand for "Feedback Transient Dynamics" or something similar). Let's break down the components:

1. **Paths ($P_n$)**: A path is a sequence of vertices connected by edges where no vertex is repeated. For example, \(P_n\) could be a path from \(v_1\) to \(v_n\), which would look like \(v_1 - v_2 - v_3 - \cdots - v_{n-1} - v_n\).

2. **Cycles ($C_n$)**: A cycle is a closed path where the first and last vertices are the same. For example, \(C_n\) could be a cycle that starts and ends at \(v_1\), such as \(v_1 - v_2 - v_3 - \cdots - v_{n-1} - v_1\).

3. **Joining Vertices**: In both cases, you mention joining \(v_1\) and \(v_n\) by an edge. This means adding an edge between these two endpoints.

4. **Black Vertices**: The set of black vertices represents an FTD-code. This term isn't standard in graph theory, so it might refer to a specific concept within a particular context, such as coding theory or network dynamics.

### Example:
Let's consider a simple example with \(n = 5\):

- **Path \(P_5\)**: \(v_1 - v_2 - v_3 - v_4 - v_5\)
- **Cycle \(C_5\)**: \(v_1 - v_2 - v_3 - v_4 - v_5 - v_1\)

If we join \(v_1\) and \(v_5\) by an edge in both cases, we get:

- **Path \(P_5'\)**: \(v_1 - v_2 - v_3 - v_4 - v_5 - v_1\) (a cycle now)
- **Cycle \(C_5'\)**: \(v_1 - v_2 - v_3 - v_4 - v_5 - v_1\) (remains a cycle)

The black vertices in this context could represent certain properties or states of the vertices in the graph, but without more information about what an FTD-code signifies, it's hard to provide a precise interpretation.

### Conclusion:
To fully understand the role of the black vertices and the FTD-code, more context or a definition of what an FTD-code means in your specific scenario would be necessary. If you have additional details or a specific problem to solve, please provide them, and I can help further!