It seems like you're describing an example of a lower bound construction for a theorem related to query complexity in preference elicitation, specifically focusing on a scenario where \( m = 18 \). Here’s a structured breakdown of what this might look like:

### Context:
- **Theorem**: Theorem~\ref{thm:0/1_valued_1_query_lower_bound}
- **Parameters**:
  - \( m = 18 \)
  - \( t = 4 \)
  - \( n = t^2 = 16 \)

### Profile Construction:
Given \( t = 4 \) and \( n = 16 \), we need to construct a profile \( P \) with 16 agents. Each agent ranks 18 alternatives.

### Key Elements:
1. **Ranking Structure**:
   - Each agent's ranking is represented by horizontal bars.
   - The top-ranked alternative is highlighted in each ranking.
   - Alternatives 11 and 12 are marked in every ranking to demonstrate symmetry in the profile.
   - Alternatives 17 and 18 are shown for two agents from each of the four cohorts.

2. **Symmetry**:
   - The marking of alternatives 11 and 12 ensures that the profile has a certain level of symmetry across all rankings.
   - This symmetry helps in understanding how the preferences are distributed among the agents.

3. **Position Consistency**:
   - Alternatives 17 and 18 appear in the exact same positions in every agent's ranking.
   - This consistency is crucial for the lower bound construction, as it ensures that the preferences are not overly varied or complex.

### Example Representation:
Since the actual visual representation isn't provided here, let's describe it conceptually:

#### Agent Rankings:
Each agent's ranking can be visualized as a horizontal bar where:
- The length of the bar represents the number of alternatives.
- The position of the top-ranked alternative is marked at the beginning of the bar.
- Alternatives 11 and 12 are marked at specific positions within the bar.
- Alternatives 17 and 18 are consistently positioned at the end of the bar for two agents from each cohort.

#### Symmetry:
- The marking of alternatives 11 and 12 ensures that the profile is symmetric across all rankings.
- The consistent positioning of alternatives 17 and 18 across all rankings further reinforces the symmetry and uniformity of the profile.

### Conclusion:
This profile \( P \) is designed to demonstrate a specific lower bound in the context of query complexity. The symmetry and consistency in the rankings help in proving that a certain number of queries is necessary to elicit the preferences accurately. The example showcases how such a profile can be constructed and analyzed to establish the lower bound.

If you have a specific visualization or more detailed description of the profile, please provide it so I can offer a more precise analysis or further explanation.