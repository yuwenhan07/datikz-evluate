External economies of agglomeration refer to the benefits that firms and workers gain from being located in close proximity to other firms and workers, which can lead to increased productivity and efficiency. These benefits arise from various factors such as specialized skills, access to information, shared infrastructure, and reduced transaction costs.

When a firm experiences external economies of agglomeration due to an increase in the size of a city (from \( N \) to \( N' \)), it typically means that the firm benefits from the larger scale of the city, which can include:

1. **Specialized Skills and Labor Market**: As the city grows, there is likely to be a greater diversity and specialization in the labor market. This means that the firm can more easily find skilled workers with specific expertise, which can enhance its production process and innovation capabilities.

2. **Access to Information**: A larger city often has a more vibrant and interconnected business community, which can facilitate the exchange of ideas and information. This can help the firm stay updated on the latest technologies, market trends, and best practices.

3. **Shared Infrastructure**: With a larger population, cities tend to invest more in infrastructure such as transportation networks, communication systems, and public utilities. These improvements can reduce the costs associated with logistics and operations for the firm.

4. **Reduced Transaction Costs**: As the city grows, the cost of transactions between different parties (e.g., suppliers, customers, and other firms) tends to decrease. This can lead to more efficient supply chains and better coordination within the firm's operations.

5. **Economies of Scale**: Larger cities often offer economies of scale in terms of production, distribution, and marketing. The firm may benefit from these economies if it can leverage them effectively, potentially reducing costs and increasing efficiency.

6. **Innovation and Collaboration**: A larger city provides more opportunities for collaboration and innovation. Firms can engage in knowledge spillovers, where new ideas and innovations spread quickly throughout the city, benefiting all firms involved.

7. **Market Access**: With a larger population, the city offers a broader market for the firm's products or services. This can lead to increased sales and revenue, which can further support the firm's growth and development.

8. **Government Support and Policies**: Larger cities often have more robust government support mechanisms, including incentives for businesses, research and development funding, and regulatory frameworks that can foster a conducive environment for business growth.

In summary, when a firm experiences external economies of agglomeration due to an increase in the size of a city, it benefits from a range of positive effects that can enhance its productivity, innovation, and overall competitiveness. These benefits can manifest through improved access to specialized skills, better information flow, reduced transaction costs, and other advantages that come with a larger urban environment.