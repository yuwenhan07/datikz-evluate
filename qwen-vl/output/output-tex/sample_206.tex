To understand the intervals that satisfy the condition \( h_{x,\lambda,\eta}^+(\phi_s g') h_{y,\lambda,\eta}^-( \phi_{r+s}g') \neq 0 \), we need to analyze the components of this expression carefully.

1. **Understanding the Notation**:
   - \( h_{x,\lambda,\eta}^+ \) and \( h_{y,\lambda,\eta}^- \) are likely to be some type of functions or operators, possibly related to a specific context such as harmonic analysis, differential equations, or functional analysis.
   - \( g' \) is the derivative of a function \( g \).
   - \( \phi_s \) and \( \phi_{r+s} \) are phase shifts or transformations applied to the function \( g' \).

2. **Interpreting the Condition**:
   The condition \( h_{x,\lambda,\eta}^+(\phi_s g') h_{y,\lambda,\eta}^-( \phi_{r+s}g') \neq 0 \) implies that both \( h_{x,\lambda,\eta}^+(\phi_s g') \) and \( h_{y,\lambda,\eta}^-( \phi_{r+s}g') \) must be non-zero simultaneously for the product to be non-zero.

3. **Analyzing the Functions**:
   - \( h_{x,\lambda,\eta}^+ \) and \( h_{y,\lambda,\eta}^- \) are typically defined over certain intervals or domains where they are non-zero.
   - The phase shifts \( \phi_s \) and \( \phi_{r+s} \) affect the domain of \( g' \) in a way that could potentially overlap with the domains where \( h_{x,\lambda,\eta}^+ \) and \( h_{y,\lambda,\eta}^- \) are non-zero.

4. **Determining the Intervals**:
   - To find the intervals where the condition holds, we need to determine the intervals over which \( g' \) is transformed by \( \phi_s \) and \( \phi_{r+s} \) such that these intervals overlap with the intervals where \( h_{x,\lambda,\eta}^+ \) and \( h_{y,\lambda,\eta}^- \) are non-zero.
   - This requires knowledge of the specific forms of \( h_{x,\lambda,\eta}^+ \), \( h_{y,\lambda,\eta}^- \), \( g' \), \( \phi_s \), and \( \phi_{r+s} \).

5. **Conclusion**:
   Without specific forms of these functions and transformations, it is impossible to provide a general interval. However, if we assume that \( h_{x,\lambda,\eta}^+ \) and \( h_{y,\lambda,\eta}^- \) are defined over intervals \( I_x \) and \( I_y \) respectively, and \( \phi_s \) and \( \phi_{r+s} \) transform \( g' \) into intervals \( J_s \) and \( J_{r+s} \), then the intervals where the condition holds would be those where \( J_s \cap I_x \neq \emptyset \) and \( J_{r+s} \cap I_y \neq \emptyset \).

In summary, the intervals satisfying the condition depend on the specific forms of the functions and transformations involved. If you can provide more details about these functions and transformations, we can derive a more precise answer.