It seems like you're describing a visual representation of two systems, where each system has vertices represented by subfigures with arrows pointing in different directions. Here's a breakdown of what you've described:

1. **Subfigures and Arrows**:
   - The left subfigure in each panel is labeled with red arrows.
   - The right subfigure in each panel is labeled with blue arrows.
   - These arrows do not represent colors in the sense of priority but rather indicate the direction of flow or connections between vertices.

2. **System Representation**:
   - Each system is uncolored, meaning the colors (red and blue) do not denote any specific priority or color coding as mentioned in the main text of the paper.

3. **Incoming and Outgoing Labels**:
   - For each system \( i \in \{1, 2\} \):
     - The label of the horizontally incoming arrow is denoted as \( p^{(i)} \).
     - The label of the vertically incoming arrow is denoted as \( r^{(i)} \).
   - The outgoing labels are assigned as depicted in the figure.

4. **Black Edges**:
   - Black edges represent the absence of an arrow, indicating no connection or flow in that direction.

This description suggests that the figure is used to illustrate the structure and flow within two distinct systems, where the labels \( p^{(i)} \) and \( r^{(i)} \) help identify the incoming connections for each system \( i \). The outgoing labels are then determined based on the structure shown in the figure.

If you have more specific questions about interpreting this figure or need further clarification, feel free to ask!