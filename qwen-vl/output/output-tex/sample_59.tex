Comparing self-supervised and unsupervised video object segmentation (VOS) methods in terms of segmentation accuracy and memory footprint is an important task for evaluating their performance and efficiency. The DAVIS-17 dataset is commonly used for this purpose due to its diverse range of challenging sequences.

### Segmentation Accuracy Metrics:
1. **Intersection over Union (IoU)**: Often referred to as Jaccard Index, it measures the overlap between the predicted segmentation mask and the ground truth mask.
2. **F-measure**: A harmonic mean of precision and recall, which provides a balanced measure of the two metrics.

### Memory Footprint:
The memory footprint refers to the amount of memory required by the model during inference or training. This can be crucial for real-time applications or when deploying models on resource-constrained devices.

### Comparison of Self-Supervised vs. Unsupervised Methods:

#### Self-Supervised Methods:
Self-supervised methods typically rely on pretext tasks that do not require labeled data but still aim to learn representations that are useful for downstream tasks like VOS. These methods often have lower memory footprints because they do not need to store large amounts of labeled data.

- **Examples**:
  - **Mask R-CNN with Pretext Tasks**: Uses bounding box regression and instance segmentation as pretext tasks.
  - **SiamFC++**: Uses a Siamese network with contrastive loss for learning discriminative features.
  - **SiamRPN++**: Extends SiamFC++ with region proposal networks.

#### Unsupervised Methods:
Unsupervised methods aim to segment objects without any labeled data. They often require more complex models and larger memory footprints due to the need to explore the data space more thoroughly.

- **Examples**:
  - **DeepSORT**: Uses deep learning to track objects across frames.
  - **Tracktor**: Uses a combination of deep learning and traditional computer vision techniques.
  - **Tracktor++**: An enhanced version of Tracktor that uses more sophisticated deep learning models.

### Performance Comparison:
- **Segmentation Accuracy**: Self-supervised methods generally achieve comparable or slightly lower accuracy compared to unsupervised methods. This is partly due to the lack of explicit supervision, which can lead to less precise object boundaries.
- **Memory Footprint**: Self-supervised methods typically have a smaller memory footprint since they do not require storing large amounts of labeled data. However, the difference might be minimal if the self-supervised method uses advanced techniques to reduce memory usage.

### Conclusion:
In summary, while self-supervised methods offer advantages in terms of reduced memory footprint and potentially faster training times, they may not match the segmentation accuracy of unsupervised methods, especially those that leverage large amounts of labeled data. The choice between these methods should depend on the specific requirements of the application, such as the availability of labeled data, computational resources, and the desired level of accuracy.