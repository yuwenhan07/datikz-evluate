The expression you've provided seems to be a combination of graph theory concepts and notation, which can be quite complex. Let's break it down step by step:

1. **Understanding the Notation**:
   - $\mathcal{Q}_{R_5}(T(5,q))$: This appears to be a specific type of quotient or transformation applied to a graph $T(5,q)$. The subscript $R_5$ might indicate a particular property or operation related to the graph.
   - $(\overleftrightarrow{K_5}, \hat{5})$: This represents a complete directed graph on 5 vertices, denoted as $\overleftrightarrow{K_5}$, with an additional vertex labeled $\hat{5}$.
   - $\overleftarrow{\nabla}_{\hat{1}}$: This looks like a specific operation or transformation applied to a graph. The subscript $\hat{1}$ suggests that this operation is applied in some way involving the vertex labeled $\hat{1}$.
   - $\left(\overleftrightarrow{K_{20}}, \hat{1}\right)$: This represents a complete directed graph on 20 vertices, denoted as $\overleftrightarrow{K_{20}}$, with an additional vertex labeled $\hat{1}$.

2. **Interpreting the Expression**:
   - The expression $\mathcal{Q}_{R_5}(T(5,q)) = (\overleftrightarrow{K_5}, \hat{5}) \overleftarrow{\nabla}_{\hat{1}} \left(\overleftrightarrow{K_{20}}, \hat{1}\right)$ suggests that the result of applying the transformation $\mathcal{Q}_{R_5}$ to the graph $T(5,q)$ is equivalent to first constructing the graph $(\overleftrightarrow{K_5}, \hat{5})$ and then applying the transformation $\overleftarrow{\nabla}_{\hat{1}}$ to the graph $\left(\overleftrightarrow{K_{20}}, \hat{1}\right)$.

3. **Possible Interpretation**:
   - If we assume that $\mathcal{Q}_{R_5}$ is a specific operation that transforms graphs in a certain way, and if $\overleftarrow{\nabla}_{\hat{1}}$ is a transformation that involves the vertex $\hat{1}$, then the expression could be describing a sequence of operations on graphs.

4. **Conclusion**:
   - Without more context or a clear definition of the operations involved, it's challenging to provide a precise interpretation. However, the expression seems to describe a sequence of transformations applied to graphs, where each step involves specific graphs and operations.

If you have more context or a clearer definition of the operations, please provide it, and I can help further.