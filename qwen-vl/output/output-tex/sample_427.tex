The provided text discusses the orthonormal bases for two different inner product spaces, denoted as \( {\langle \,\cdot\,,\,\cdot\, \rangle}_{s} \) and \( {\langle \,\cdot\,,\,\cdot\, \rangle}_{u} \). These bases are labeled with the moist variable \( M \) and the buoyancy \( b \), which might seem confusing initially, especially in the saturated region where \( q > 0 \).

Let's break down the information given:

### Orthonormal Bases

1. **Basis for \( {\langle \,\cdot\,,\,\cdot\, \rangle}_{s} \):**
   - \(\widehat{M}_s = (0, 1)\)
   - \(\widehat{b}_s = (1, -1)\)

2. **Basis for \( {\langle \,\cdot\,,\,\cdot\, \rangle}_{u} \):**
   - \(\widehat{M}_u = \frac{1}{\sqrt{6}} (1, 1)\)
   - \(\widehat{b}_u = \frac{1}{\sqrt{2}} (1, -1)\)

### Inner Products

The inner products are defined such that the basis vectors are orthonormal with respect to each other.

### Dual Vectors

The dual vectors of the given bases are:
- For \( {\langle \,\cdot\,,\,\cdot\, \rangle}_{s} \):
  - \(\langle \widehat{M}_s, (\theta, q) \rangle_s = \theta + q = M\)
  - \(\langle \widehat{b}_s, (\theta, q) \rangle_s = \theta = b_s\)

- For \( {\langle \,\cdot\,,\,\cdot\, \rangle}_{u} \):
  - \(\langle \widehat{M}_u, (\theta, q) \rangle_u = \sqrt{\frac{3}{2}} (\theta + q) = \sqrt{\frac{3}{2}} M\)
  - \(\langle \widehat{b}_u, (\theta, q) \rangle_u = \frac{1}{\sqrt{2}} \theta - q = \frac{1}{\sqrt{2}} b_u\)

### Explanation

The notation \( \widehat{M}_s \) and \( \widehat{b}_s \) for the \( s \)-inner product space and \( \widehat{M}_u \) and \( \widehat{b}_u \) for the \( u \)-inner product space might seem counterintuitive because they are labeled with the variables \( M \) and \( b \), which are typically associated with the moist variable and buoyancy, respectively. However, the dual vectors of these bases provide a clearer interpretation:

- In the \( s \)-inner product space:
  - \(\langle \widehat{M}_s, (\theta, q) \rangle_s = \theta + q\) corresponds to the moist variable \( M \).
  - \(\langle \widehat{b}_s, (\theta, q) \rangle_s = \theta\) corresponds to the buoyancy \( b_s \).

- In the \( u \)-inner product space:
  - \(\langle \widehat{M}_u, (\theta, q) \rangle_u = \sqrt{\frac{3}{2}} (\theta + q)\) corresponds to the moist variable \( M \).
  - \(\langle \widehat{b}_u, (\theta, q) \rangle_u = \frac{1}{\sqrt{2}} \theta - q\) corresponds to the buoyancy \( b_u \).

This labeling is justified because the dual vectors directly map back to the original variables \( M \) and \( b \), making it clear how the bases relate to the physical quantities they represent.