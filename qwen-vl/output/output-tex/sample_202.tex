To solve the problem of reducing \( T_{20} \) given the conditions on the edges of the graphs \(\Gamma_2\) and \(\Gamma_3\), we need to understand what these conditions imply for the structure of the graphs and how they affect the reduction of \( T_{20} \).

First, let's interpret the conditions:
1. \((1,4) \in E(\Gamma_2)\): This means that there is an edge between vertices 1 and 4 in the graph \(\Gamma_2\).
2. \((1,8) \in E(\Gamma_3)\): This means that there is an edge between vertices 1 and 8 in the graph \(\Gamma_3\).
3. \((2,4) \in E(\Gamma_3)\): This means that there is an edge between vertices 2 and 4 in the graph \(\Gamma_3\).

The term "reduction of \( T_{20} \)" typically refers to a reduction in the number of terms or a simplification of a polynomial expression involving \( T_{20} \). However, without more context about the specific form of \( T_{20} \) or the operations being performed, it's not clear what exactly is meant by "reduction."

Assuming that \( T_{20} \) is some polynomial or expression involving the variables corresponding to the vertices of the graphs, and given the edges provided, we can infer that the vertices 1, 4, and 8 are connected in some way through the edges of the graphs.

If we assume that \( T_{20} \) is a polynomial where each term corresponds to a unique path or combination of paths in the graphs, then the presence of the edges \((1,4)\), \((1,8)\), and \((2,4)\) might simplify certain terms in \( T_{20} \).

For example, if \( T_{20} \) is a polynomial like \( T_{20} = x_1 x_4 + x_1 x_8 + x_2 x_4 + \cdots \), the presence of the edges \((1,4)\), \((1,8)\), and \((2,4)\) might allow us to combine or eliminate some terms. However, without the exact form of \( T_{20} \), it's impossible to provide a precise reduction.

If we assume a simple case where \( T_{20} \) is just the product of all possible paths from vertex 1 to any other vertex, then the presence of the edges \((1,4)\), \((1,8)\), and \((2,4)\) would reduce the number of distinct paths that contribute to \( T_{20} \).

Given the information, the most straightforward reduction we can make is to recognize that the edges \((1,4)\), \((1,8)\), and \((2,4)\) imply that the vertices 1, 4, and 8 are connected in some way, which might simplify the expression for \( T_{20} \).

Therefore, the reduction of \( T_{20} \) could be expressed as:

\[
\boxed{T_{20} \text{ is simplified by the presence of the edges } (1,4), (1,8), \text{ and } (2,4)}
\]

This is a general statement based on the given conditions. The exact form of the reduction would depend on the specific form of \( T_{20} \).