To solve the problem of constructing the 2-circlet \( C(k, m) \), we need to follow the steps outlined in the reference text carefully. Here's a detailed breakdown:

1. **Identify the 1-skeleton of the 2-complex:**
   - The 1-skeleton of the 2-complex is the graph Cartesian product \( K_{1,m} \Box P_{k+1} \).
   - \( K_{1,m} \) is a complete graph on \( m \) vertices, which can be thought of as a cycle with \( m \) edges.
   - \( P_{k+1} \) is a path with \( k+1 \) vertices, which can be thought of as a straight line with \( k \) edges.

2. **Construct the Cartesian product \( K_{1,m} \Box P_{k+1} \):**
   - Each vertex in \( K_{1,m} \) is connected to each vertex in \( P_{k+1} \).
   - This results in a grid-like structure where there are \( m \) rows and \( k+1 \) columns of vertices.

3. **Twist the left \( K_{1,m} \) with the right \( K_{1,m} \):**
   - Twist the left \( K_{1,m} \) by \( 2\pi/m \) relative to the right \( K_{1,m} \).
   - This twisting operation creates a non-trivial 2-complex with a specific boundary.

4. **Cap the boundary with a polygon:**
   - The boundary of the resulting 2-complex is a \( C_{km} \)-shaped boundary.
   - Cap this boundary by adding a polygon with \( km \) sides.

5. **Verify the properties of the 2-circlet:**
   - The \( km \)-gon shares an edge of degree 2 with any square in the complex.
   - Any even \( n \)-subcomplex that contains the \( km \)-gon necessarily contains all the squares too.

The final result is a 2-circlet \( C(k, m) \).

Thus, the construction of the 2-circlet \( C(k, m) \) is complete. The answer is:

\[
\boxed{C(k, m)}
\]