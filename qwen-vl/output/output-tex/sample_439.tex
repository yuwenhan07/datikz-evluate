To understand the critical eigenspace \( V \) at the bifurcation point \((s^*, 0)\) in Example~\ref{exa:lambda13} where \( s^* = 10\pi^2 \), we need to analyze the geometric structure and the dynamics around this point. Here's a step-by-step breakdown:

### 1. **Critical Eigenspace \( V \)**:
The critical eigenspace \( V \) is a subspace of the tangent space at the bifurcation point \((s^*, 0)\). It is characterized by the eigenvalues of the linearization matrix at this point, which are the eigenvalues that change sign as the parameter \( s \) crosses \( s^* \).

### 2. **Diagonal Lines**:
The diagonal lines represent the intersections of two fixed point subspaces with \( V \). These subspaces are associated with the eigenvalues that lead to EBL (Eigenvalue-Based Loss) bifurcations. An EBL bifurcation occurs when an eigenvalue crosses zero, leading to a change in the stability of the fixed points.

### 3. **Horizontal Dashed Line**:
The horizontal dashed line represents the intersection of the defocusing direction \( W_d \) with \( V \). This line corresponds to the intersection of the defocusing subspace with the critical eigenspace. The intersection of \( W_d \) with \( V \) leads to a BLIS (Bifurcation Leading to Invariant Subspace) bifurcation. A BLIS bifurcation occurs when the defocusing direction intersects the critical eigenspace, leading to a change in the invariant subspace of the system.

### 4. **Vertical Dashed Line**:
The vertical dashed line represents another intersection of the critical eigenspace \( V \) with itself, but it is not explicitly clear what type of bifurcation this line corresponds to without more context. However, it is likely another BLIS bifurcation or a similar type of bifurcation, given the symmetry and the nature of the problem.

### 5. **Group Orbits**:
The three different line types correspond to the three group orbits of bifurcating solutions. This means that the solutions bifurcate into three distinct groups, each corresponding to a different orbit under the group action. The group orbits can be visualized as different branches of the bifurcation diagram, each representing a different solution structure.

### Conclusion:
In summary, the critical eigenspace \( V \) at the bifurcation point \((s^*, 0)\) is a subspace where the eigenvalues of the linearization matrix change sign. The diagonal lines represent intersections with subspaces leading to EBL bifurcations, while the horizontal dashed line represents the intersection with the defocusing direction leading to a BLIS bifurcation. The vertical dashed line likely represents another similar BLIS bifurcation. The three different line types correspond to the three group orbits of bifurcating solutions.

If you have any specific questions about the example or need further clarification on any part of the analysis, feel free to ask!