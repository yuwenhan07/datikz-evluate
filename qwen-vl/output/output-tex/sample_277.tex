The Casimir effect is a quantum mechanical phenomenon where two conducting plates in a vacuum can experience a force due to the zero-point energy fluctuations of the electromagnetic field between them. The Casimir force is described by the Hadamard function, which involves boundary terms that arise from the interaction of the electromagnetic field with the boundaries of the plates.

To understand why \( D^{(1)} = 0 \) on both plates requires an infinite set of image sources, let's break down the reasoning step-by-step:

### 1. **Hadamard Function and Boundary Terms:**
The Hadamard function \( H(x) \) for the Casimir effect is given by:
\[
H(x) = -\frac{1}{8\pi^2} \int_{-\infty}^{\infty} \frac{d\omega}{\omega^2} \left[ \ln \left| \frac{x - \omega}{x + \omega} \right| + i \pi \theta(-x) \right]
\]
where \( x \) is the separation between the plates, and \( \theta(-x) \) is the Heaviside step function.

### 2. **Image Sources and Boundary Conditions:**
To satisfy the boundary conditions on the plates, we need to consider the image sources. For a single plate, the image sources are placed at distances \( z = \pm 2an \) for integer \( n \). These image sources effectively cancel out the contributions from the real sources on the other plate, leading to a vanishing Casimir force.

### 3. **Spacing of Image Sources:**
If the image sources are placed at a spacing of \( a \), then there will be a source on each plate at every integer multiple of \( a \). This results in a doubling of the boundary terms because each point along the \( z \)-axis has a corresponding image source on the opposite plate.

### 4. **Effect on the Numerator:**
The doubling of the boundary terms translates into an extra factor of 2 on the numerator of the boundary term for the Hadamard function. Specifically, the integral over the frequency \( \omega \) will now include contributions from both the real sources and their images, leading to a net cancellation of the boundary terms.

### 5. **Conclusion:**
Thus, to ensure that \( D^{(1)} = 0 \) on both plates, we need an infinite set of image sources placed at a distance of \( z = \pm 2an \) for integer \( n \). This ensures that the contributions from the real sources and their images cancel each other out, resulting in a vanishing Casimir force.

In summary, the requirement for \( D^{(1)} = 0 \) on both plates arises from the need to cancel out the boundary terms through the use of image sources, which must be placed at specific distances to achieve this cancellation. The doubling of the boundary terms due to the image sources leads to the extra factor of 2 on the numerator of the Hadamard function, ensuring the desired result.