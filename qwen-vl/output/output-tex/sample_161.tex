To solve the problem, we need to understand the structure of the graph \( \square_{i=1}^4 K_2 \), which is a 4x4 grid where each vertex is connected to its four neighbors (if they exist). The task is to find a feasible set of size two (two discs in red) and then determine all possible extensions to a feasible set of size three (three squares in black).

### Step-by-Step Solution:

1. **Identify the Feasible Set of Size Two:**
   A feasible set of size two means we need to select two vertices such that no two vertices are adjacent. In a 4x4 grid, we can choose any two non-adjacent vertices. For example, let's choose the vertices at positions (1,1) and (3,3).

2. **Determine All Possible Extensions to a Feasible Set of Size Three:**
   To extend this feasible set of size two to a feasible set of size three, we need to add one more vertex that is not adjacent to either of the two already chosen vertices. Let's analyze the possible positions for the third vertex.

   - If the first two vertices are at (1,1) and (3,3), the third vertex cannot be at (1,2), (1,3), (1,4), (2,1), (2,3), (2,4), (3,2), (3,4), or (4,1), (4,2), (4,3), (4,4). The only remaining positions are (2,2) and (4,4).

   Therefore, the possible extensions to a feasible set of size three are:
   - Adding (2,2) to (1,1) and (3,3)
   - Adding (4,4) to (1,1) and (3,3)

So, the possible extensions to a feasible set of size three are:
- (1,1), (3,3), (2,2)
- (1,1), (3,3), (4,4)

The final answer is:
\[
\boxed{(1,1), (3,3), (2,2) \text{ and } (1,1), (3,3), (4,4)}
\]