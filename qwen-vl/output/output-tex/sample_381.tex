It seems like you're referring to a table or section labeled "Table \ref{tab:exp}" that contains details about the different levels of a data challenge. However, since I don't have access to the actual content of this table, I'll provide a general explanation based on the information you've given.

In a typical data challenge setup, different levels might be designed to test various aspects of data processing, analysis, and modeling skills. Here's a breakdown of the levels mentioned:

1. **Filtering Levels**: These could involve different stages where the data is cleaned and preprocessed. For example:
   - Level 1: Basic cleaning (e.g., removing duplicates, handling missing values).
   - Level 2: Advanced cleaning (e.g., dealing with outliers, encoding categorical variables).
   - Level 3: Complex cleaning (e.g., feature engineering, creating new features from existing ones).

2. **Reverb Levels**: These might represent different stages or types of reverberation effects in audio processing, if we're talking about an audio data challenge. For instance:
   - Level 1: Basic reverberation (e.g., adding simple echo effects).
   - Level 2: Intermediate reverberation (e.g., more complex echo patterns, room simulations).
   - Level 3: Advanced reverberation (e.g., realistic room acoustics, multiple reverberation sources).

3. **Combined Levels**: These could be a combination of both filtering and reverb techniques, possibly at different complexity levels:
   - Level 1: Basic combined technique (e.g., applying basic cleaning followed by simple reverberation).
   - Level 2: Intermediate combined technique (e.g., advanced cleaning followed by intermediate reverberation).
   - Level 3: Advanced combined technique (e.g., complex cleaning followed by realistic reverberation).

If you can provide the specific details from Table \ref{tab:exp}, I can give a more precise and detailed explanation.