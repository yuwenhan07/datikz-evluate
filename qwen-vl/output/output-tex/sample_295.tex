To solve the problem, we need to understand the definitions and constraints given:

1. **$(3,0)$-saturated subcubic graph**: A subcubic graph is a graph where each vertex has degree at most 3. A $(3,0)$-saturated subcubic graph means that every vertex has exactly 3 neighbors, and there are no edges between any two vertices that are not directly connected by an edge.

2. **Pairwise distance at most two**: This means that for any two vertices \(x_i\) and \(x_j\), the shortest path between them (in terms of edges) is at most 2.

3. **$(1,2^4)$-packing coloring**: This is a specific type of graph coloring where each color class induces a subgraph with maximum degree 1 (i.e., each color class is an independent set). The notation $(1,2^4)$ indicates that the graph can be colored using up to 4 colors such that each color class is an independent set and the maximum degree of the subgraph induced by the union of these color classes is 1.

Given these definitions, let's construct an example of such a configuration.

### Step-by-Step Construction

1. **Vertices and Edges**:
   - We have 6 vertices \(x_1, x_2, x_3, x_4, x_5, x_6\).
   - Each vertex must have exactly 3 neighbors.
   - The distance between any two vertices should be at most 2.

2. **Graph Configuration**:
   - One possible configuration is a cycle of length 6 with additional edges to ensure each vertex has degree 3. However, this would not satisfy the condition of having distance at most 2 between any two vertices. Instead, we can use a more complex structure like a prism or a specific arrangement of triangles.

3. **Example Configuration**:
   - Consider the following vertices and edges:
     \[
     \begin{aligned}
     &x_1 \sim x_2, x_3, x_4, \\
     &x_2 \sim x_1, x_3, x_5, \\
     &x_3 \sim x_1, x_2, x_6, \\
     &x_4 \sim x_1, x_5, x_6, \\
     &x_5 \sim x_2, x_4, x_6, \\
     &x_6 \sim x_3, x_4, x_5.
     \end{aligned}
     \]
   - Here, each vertex \(x_i\) is connected to three other vertices, forming a structure where the distance between any two vertices is at most 2.

4. **Packing Coloring**:
   - We need to assign colors such that each color class forms an independent set and the maximum degree of the subgraph induced by the union of these color classes is 1.
   - One possible coloring is:
     \[
     \begin{aligned}
     &x_1, x_4 \text{ are colored with color } 1, \\
     &x_2, x_5 \text{ are colored with color } 2, \\
     &x_3, x_6 \text{ are colored with color } 3.
     \end{aligned}
     \]
   - This coloring ensures that each color class is an independent set, and the maximum degree of the subgraph induced by the union of these color classes is 1.

Thus, the final answer is:
\[
\boxed{\text{The vertices and edges form a specific configuration where each vertex has degree 3 and the distance between any two vertices is at most 2, and the graph can be packed with 3 colors such that each color class is an independent set.}}
\]