The energy levels of isotopes like \(^{16}\)N (nitrogen), \(^{16}\)O (oxygen), and \(^{16}\)F (fluorine) near their ground states can be quite complex due to the interactions between nucleons and the effects of isospin. The isospin \(T\) is a quantum number that describes the symmetry properties of the nucleus under the exchange of neutrons and protons.

For \(^{16}\)N, \(^{16}\)O, and \(^{16}\)F, which all have 8 protons and 8 neutrons, they share the same nuclear structure and are isobars. This means they have the same number of nucleons but different atomic numbers, which affects their electronic configurations but not their nuclear structure in terms of the number of protons and neutrons.

### Ground State Energy Levels

The ground state energy levels for these nuclei are primarily determined by the nuclear shell model, which considers the filling of nucleon shells. For \(^{16}\)N, \(^{16}\)O, and \(^{16}\)F, the ground state configuration is:

- **Protons**: \(1s^2 2s^2 2p^6 3s^2 3p^2\)
- **Neutrons**: \(1s^2 2s^2 2p^6 3s^2 3p^4\)

This configuration corresponds to the filling of the \(1s\), \(2s\), \(2p\), and \(3s\) shells, followed by the \(3p\) shell being half-filled.

### Isospin \(T = 1\)

When considering the isospin \(T = 1\), we need to look at the possible isobaric analog states (IAS). These are states where the proton and neutron configurations are interchanged while maintaining the same total angular momentum and parity. For \(^{16}\)N, \(^{16}\)O, and \(^{16}\)F, the IAS will have the same energy as the ground state because they are isobars with the same number of nucleons.

### Energy Levels Near the Ground State

The energy levels near the ground state for \(^{16}\)N, \(^{16}\)O, and \(^{16}\)F can be found in nuclear databases such as the Evaluated Nuclear Structure Data File (ENSDF). However, without direct access to the ENSDF or specific references, I cannot provide exact numerical values here. Typically, these energies are listed in keV (kiloelectronvolts).

### Summary

- **Ground State Configuration**: \(1s^2 2s^2 2p^6 3s^2 3p^2\) for protons and \(1s^2 2s^2 2p^6 3s^2 3p^4\) for neutrons.
- **Isospin \(T = 1\)**: The IAS will have the same energy as the ground state.
- **Energy Levels Near the Ground State**: These can be found in nuclear databases like ENSDF, typically in keV.

For precise values, you would need to consult the ENSDF or a similar nuclear physics database.