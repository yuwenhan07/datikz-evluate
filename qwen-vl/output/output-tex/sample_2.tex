The U-Net architecture is a popular neural network design used for image segmentation tasks, particularly in medical imaging. It is designed to handle the problem of upsampling and downsampling in a way that preserves the spatial information of the input image.

### Key Components of U-Net:

1. **Encoder (Downsampling Path):**
   - The encoder part of the U-Net consists of a series of convolutional layers with decreasing receptive fields.
   - Each layer reduces the spatial dimensions of the input feature maps by a factor of 2 through strided convolutions or pooling operations.
   - The encoder captures the high-level features of the input image.

2. **Decoder (Upsampling Path):**
   - The decoder part of the U-Net consists of a series of transposed convolutional layers with increasing receptive fields.
   - Each layer increases the spatial dimensions of the feature maps by a factor of 2 through transposed convolutions.
   - The decoder reconstructs the spatial details from the high-level features captured by the encoder.

3. **Skip Connections:**
   - Skip connections are used to combine the feature maps from the encoder and decoder paths.
   - These connections allow the network to learn both the fine-grained details and the coarse structure of the input image.

4. **Convolution Layers:**
   - Convolution layers are used to extract features from the input data.
   - In U-Net, there are two types of convolution layers:
     - **Double Lines:** Represent convolutions with all strides equal to 1. These layers are used for smoothing and feature extraction.
     - **Single Line:** Represent convolutions with at least one stride greater than 1. These layers are used for downsampling.
     - **Dashed Line:** Represents transposed convolutions (upsampling).
     - **Curved Line:** Represents skip connections.

### Example of U-Net Architecture with Two Layers:

Let's consider a simplified U-Net with just two layers in each path (encoder and decoder).

#### Encoder Path:
1. Input Image \( I \)
2. Convolution Layer 1 (Single Line) → Feature Map \( F_1 \)
3. Convolution Layer 2 (Single Line) → Feature Map \( F_2 \)

#### Decoder Path:
1. Transposed Convolution Layer 1 (Dashed Line) → Feature Map \( G_1 \)
2. Skip Connection (Curved Line) → Feature Map \( F_2 + G_1 \)
3. Transposed Convolution Layer 2 (Dashed Line) → Feature Map \( G_2 \)

### Explanation:
- **Encoder Path:**
  - The first convolution layer (single line) downsamples the input image \( I \) to produce \( F_1 \).
  - The second convolution layer (single line) further downsamples \( F_1 \) to produce \( F_2 \).

- **Decoder Path:**
  - The first transposed convolution layer (dashed line) upsamples \( F_2 \) to produce \( G_1 \).
  - The skip connection combines \( F_2 \) and \( G_1 \) to preserve the spatial information.
  - The second transposed convolution layer (dashed line) further upsamples the combined feature map to produce \( G_2 \).

This simple example illustrates the basic structure of a U-Net with two layers in each path. In practice, U-Nets often have more layers and additional components like batch normalization, activation functions, and dropout to improve performance.