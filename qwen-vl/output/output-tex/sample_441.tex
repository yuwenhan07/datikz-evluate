To solve the problem, we need to understand the structure of the sequence \(3^\ell 5^\ell 1^+\) and its corresponding 3-line graph with a specific twist change. Let's break it down step by step.

### Step 1: Understanding the Sequence
The sequence \(3^\ell 5^\ell 1^+\) represents a sequence of vertices in a graph where:
- There are \(\ell\) vertices labeled \(3\),
- There are \(\ell\) vertices labeled \(5\),
- There is one vertex labeled \(1\).

### Step 2: Constructing the 3-Line Graph
A 3-line graph is a type of graph where each vertex is represented by three lines (or edges), and the lines are connected in a specific way. For the sequence \(3^\ell 5^\ell 1^+\), the 3-line graph will have \(\ell + \ell + 1 = 2\ell + 1\) vertices, each with three lines.

### Step 3: Twist Change
The twist change involves modifying the connection between the middle edge of the 3-line graph. In the original configuration, the twist angle might be \(0\) or some other value. Here, we are changing this twist angle to \(-6\) degrees.

### Step 4: Visualizing the Graph
To visualize the graph, we can think of it as follows:
- The graph will have \(\ell\) vertices labeled \(3\), each with three lines.
- The graph will have \(\ell\) vertices labeled \(5\), each with three lines.
- The graph will have one vertex labeled \(1\), which also has three lines.

The twist change affects the way these lines are connected at the middle vertex. Specifically, the middle edge of the 3-line graph (which connects the first and second lines of the vertex) will now have a twist angle of \(-6\) degrees instead of the original \(0\) degrees.

### Step 5: Final Answer
The final answer is the description of the 3-line graph with the specified twist change. Since the exact visual representation cannot be provided here due to the limitations of text-based communication, the answer is:

\[
\boxed{\text{The 3-line graph for the sequence } 3^\ell 5^\ell 1^+ \text{ with the middle edge twist changed to } -6 \text{ degrees.}}
\]