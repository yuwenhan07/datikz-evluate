To illustrate the concept of bipartite matching in the context of computing the similarity between sense cluster centroids \( p_1^t \) and \( p_1^{t-1} \), we can follow these steps:

### Step 1: Define the Problem
We have two sets of vectors:
- \( \{e_i^t\}_{i=1}^3 \): The representative embeddings of three semantically nearest neighboring words to \( p_1^t \).
- \( \{e_i^{t-1}\}_{i=1}^3 \): The representative embeddings of three semantically nearest neighboring words to \( p_1^{t-1} \).

We want to find the best way to match each embedding from the set \( \{e_i^t\} \) with an embedding from the set \( \{e_i^{t-1}\} \) such that the overall similarity is maximized.

### Step 2: Formulate the Bipartite Matching Problem
This problem can be formulated as a bipartite matching problem where:
- One set of nodes represents the embeddings from \( \{e_i^t\} \).
- The other set of nodes represents the embeddings from \( \{e_i^{t-1}\} \).
- An edge exists between two nodes if there is a meaningful correspondence (similarity) between the corresponding embeddings.

### Step 3: Compute Pairwise Similarities
For each pair of embeddings \( e_i^t \) and \( e_j^{t-1} \), compute their similarity score. This could be done using any similarity measure such as cosine similarity, Euclidean distance, or any other metric suitable for your data.

### Step 4: Construct the Bipartite Graph
Create a bipartite graph where:
- Each node in the left set corresponds to an embedding in \( \{e_i^t\} \).
- Each node in the right set corresponds to an embedding in \( \{e_i^{t-1}\} \).
- An edge between two nodes has a weight equal to the similarity score computed in Step 3.

### Step 5: Apply a Bipartite Matching Algorithm
Use a bipartite matching algorithm to find the maximum-weight matching in this graph. A common algorithm for this purpose is the Hungarian algorithm or the Kuhn-Munkres algorithm, which finds the optimal assignment of nodes from one set to another such that the total weight of the edges is maximized.

### Step 6: Interpret the Result
The result of the bipartite matching will give you the optimal assignments of embeddings from \( \{e_i^t\} \) to embeddings from \( \{e_i^{t-1}\} \). This means that for each embedding \( e_i^t \), you will have found the most similar embedding \( e_j^{t-1} \) based on the computed similarities.

### Step 7: Visualize the Matching
You can visualize the bipartite matching by drawing arrows from the embeddings in \( \{e_i^t\} \) to the embeddings in \( \{e_i^{t-1}\} \) that were matched. This will help you see how the sense clusters \( p_1^t \) and \( p_1^{t-1} \) are related through their nearest neighbors.

### Final Answer
The bipartite matching process helps us find the optimal correspondences between the nearest neighbors of the sense cluster centroids \( p_1^t \) and \( p_1^{t-1} \). This allows us to understand how the sense clusters have evolved over time and to compute the similarity between them effectively.

\[
\boxed{\text{Bipartite matching is used to find the optimal correspondences between the nearest neighbors of the sense cluster centroids.}}
\]