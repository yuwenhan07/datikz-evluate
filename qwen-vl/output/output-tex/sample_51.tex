To solve the problem of finding a proper \(mn\)-coloring of \((S_{1,m} \square S_{1,n})_\delta\) where the vertices of the same degree are indicated by the same color and are pairwise adjacent, we need to understand the structure of the graph \(S_{1,m} \square S_{1,n}\).

The graph \(S_{1,m} \square S_{1,n}\) is the Cartesian product of two paths \(P_m\) and \(P_n\). This means it has \(mn\) vertices, each corresponding to an ordered pair \((i,j)\) where \(i\) ranges from 1 to \(m\) and \(j\) ranges from 1 to \(n\). The vertices \((i,j)\) and \((i',j')\) are adjacent if and only if either \(i = i'\) and \(|j - j'| = 1\) or \(j = j'\) and \(|i - i'| = 1\).

The degree of a vertex \((i,j)\) in \(S_{1,m} \square S_{1,n}\) is 4 if \(i \neq 1, m\) and \(j \neq 1, n\), 3 if \(i = 1, m\) or \(j = 1, n\), and 2 if \(i = 1, m\) and \(j = 1, n\).

We need to color the vertices such that vertices of the same degree are the same color and are pairwise adjacent. Let's consider the coloring:

1. Color all vertices of degree 4 with one color.
2. Color all vertices of degree 3 with another color.
3. Color all vertices of degree 2 with a third color.

Since the degrees are determined by the positions of the vertices in the paths \(P_m\) and \(P_n\), we can use the following coloring scheme:
- Color all vertices \((i,j)\) where \(i \neq 1, m\) and \(j \neq 1, n\) with color 1.
- Color all vertices \((i,j)\) where \(i = 1, m\) or \(j = 1, n\) but not both with color 2.
- Color all vertices \((i,j)\) where \(i = 1, m\) and \(j = 1, n\) with color 3.

This ensures that all vertices of the same degree are the same color and are pairwise adjacent. Therefore, the number of colors needed is 3.

The final answer is:
\[
\boxed{3}
\]