It seems like you're discussing a 2D map related to a mixing function, where the mixing process is influenced by the cylindrical radius (\(\varpi\)) and the rotational axis (\(z\)). The jet propagates along the \(z\) axis, and \(\varpi\) represents the cylindrical radius in the equatorial plane. The gray mask is used to highlight the mixing region.

### Key Points:
1. **Mixing Function Map**:
   - The map in (a) shows the mixing function transitioning from 1 to 0.
   - \(\varpi\) is the cylindrical radius, and \(z\) is the rotational axis.
   - The jet propagates along the \(z\) axis, and \(\varpi\) is the cylindrical radius in the equatorial plane.
   - The gray mask helps to visually delineate the mixing region.

2. **Speed of Transition**:
   - The speed of transition within the mixing region can be controlled by the parameter \(d_m\).
   - This is illustrated in (b).

3. **ASO Component Extension**:
   - (b) shows the extension of the ASO (Aerosol Single-Scattering Optical Properties) component over the selected region.
   - The functions are of qualitative value, showing the dependence of the parameter \(d\) on the mixing speed.
   - The center of the function is shifted to zero for better presentation of the mixing parameters.

### Addressing Your Questions:

#### 1. **Is the Color Bar Useful Without the Gray Mask?**
The color bar in the map likely indicates the values of the mixing function, which range from 1 to 0. If the gray mask is removed, the color bar would still be useful because it provides a reference for interpreting the intensity or value of the mixing function across the entire region. The color bar would help viewers understand the gradient of the mixing function without the visual cue provided by the gray mask.

#### 2. **Appearance Without the Gray Mask:**
Without the gray mask, the map would show the entire 2D space with the mixing function's values represented by colors. The color bar would remain essential for understanding the intensity of the mixing function at different points. The absence of the gray mask might make it slightly harder to visually identify the mixing region, but the color bar would still provide the necessary information.

### Summary:
- The color bar remains crucial even without the gray mask, as it provides a reference for the values of the mixing function.
- The gray mask helps to visually delineate the mixing region, making it easier to identify areas of high mixing activity.
- Removing the gray mask would require relying more heavily on the color bar for interpretation, but the color bar itself would still be indispensable for understanding the mixing function's intensity across the entire region.