To solve this problem, we need to understand the geometric and algebraic relationships between the parameters \(\lambda_r\), \(\lambda^\perp_r\), and \(\gamma_i\) in the given parameter space. Let's break it down step by step.

1. **Understanding the Projections**:
   - The axion couplings \(\lambda_r\) are given as cyan vectors.
   - The orthogonal couplings \(\lambda^\perp_r\) are the projections of \(\lambda_r\) onto the orthogonal complement of the space spanned by the potential couplings \(\gamma_i\).
   - The projections \(\lambda^\perp_r\) are non-negative for all \(r\).

2. **Potential Convex Hull**:
   - The potential convex hull is the smallest convex set that contains all the potential couplings \(\gamma_i\). It is represented by the orange region in the figure.
   - The minimal-distance vector \(\gamma_\infty\) is the vector from the origin to the boundary of the convex hull that is farthest from the origin. This vector is shown as a purple vector in the figure.

3. **Orthogonal Coordinate System**:
   - The orthogonal coordinate system helps us identify right angles and understand the orientation of the vectors in the space.

4. **Identifying Right Angles**:
   - In an orthogonal coordinate system, right angles are formed where the axes intersect. This can help us understand the orthogonality conditions between different sets of vectors.

Given these points, let's summarize the key aspects:

- The non-negativity of the projections \(\lambda^\perp_r\) implies that the orthogonal couplings are aligned with the potential couplings in a way that they do not point in directions that would make their projections negative.
- The minimal-distance vector \(\gamma_\infty\) indicates the direction of the potential couplings that are farthest from the origin, which is crucial for understanding the boundary of the convex hull.
- The orthogonal coordinate system provides a framework to analyze the relationships between the different sets of couplings.

Since the problem does not ask for a specific numerical answer but rather an understanding of the geometric and algebraic relationships, the final answer is:

\[
\boxed{\text{The projections of the orthogonal couplings \(\lambda^\perp_r\) on all potential couplings \(\gamma_i\) are non-negative, and the minimal-distance vector \(\gamma_\infty\) indicates the direction of the potential couplings that are farthest from the origin.}}
\]