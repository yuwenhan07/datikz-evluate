It seems like you're referring to a specific scenario in the context of millimeter-wave (mmWave) communication networks, particularly focusing on carrier frequency (CF) assignment. Here's a breakdown of what this might mean:

1. **Millimeter-Wave (mmWave) Network**: These are wireless communication systems that operate in the mmWave band (typically 30 GHz to 300 GHz), which offers high data rates due to their short wavelengths and large bandwidths.

2. **Carrier Frequency (CF)**: In the context of mmWave networks, the carrier frequency is the primary frequency used for transmitting data. It's crucial for defining the channel and ensuring that devices can communicate effectively within the same frequency band.

3. **Downlink Scenario**: This refers to the direction of data transmission from the base station (or access point, AP) to the user devices. In the downlink, the base station sends data to the user devices.

4. **Dashed Arrows**: These typically represent assignments or connections between different entities. In the context of your scenario, dashed arrows likely indicate the assignments or connections between Access Points (APs) and User Devices (UDs).

### Interpretation:
In the downlink scenario described, the dashed arrows would show how the APs assign or connect to the user devices. This could be based on various factors such as signal strength, user location, available resources, or other network optimization criteria.

### Example:
- **AP1** might be connected to **User Device A**.
- **AP2** might be connected to **User Device B**.
- And so on...

This kind of assignment ensures efficient use of the mmWave spectrum and optimizes the network performance by distributing the load among the APs and users appropriately.

If you have more specific details about the scenario, such as the number of APs, users, or any particular constraints, it would help provide a more detailed explanation or analysis.