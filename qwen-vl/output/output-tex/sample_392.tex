To understand the situation described in the proof of Lemma~\ref{p-means radii strict inequality lemma}, let's break down the key components and their implications step by step.

1. **Orange Curve and Points**:
   - The orange curve consists of the segment \(\gamma/g_q\) and two rays.
   - None of the points on this orange curve lie inside the interior of \(K\) (\(\inter(K)\)). This means that the orange curve is entirely outside the interior of \(K\).

2. **Supporting Lines**:
   - There are two dashed lines that are parallel and support \(K\) at points \(v\) and \(w\) respectively.
   - These dashed lines are part of the boundary of \(K\) and are tangent to \(K\) at these points.

3. **Dashdotted Line and Segment**:
   - A dashdotted line and a segment are also parallel to each other.
   - The dashdotted line and segment belong to the boundary of the set \(\underline{M}_q(K, L)\), which is a specific construction related to the Minkowski sum of sets \(K\) and \(L\).

4. **Line Containing \(x\)**:
   - The line containing point \(x\) supports the convex hull of the union of \(K\) and \(L\) (\(\conv(K \cup L)\)) at point \(x\).
   - This implies that the line through \(x\) is tangent to the convex hull of \(K \cup L\) at \(x\).

5. **Implications**:
   - Since none of the points on the orange curve lie in \(\inter(K)\), it follows that \(K\) is completely contained within the area bounded by the orange curve.
   - The dashed lines and the dashdotted line/segment provide information about the geometric configuration of \(K\) and its relationship with \(L\) in the context of the Minkowski sum \(\underline{M}_q(K, L)\).

In summary, the situation described in the proof involves a detailed analysis of the geometric properties of the sets \(K\) and \(L\) and their interaction through the Minkowski sum \(\underline{M}_q(K, L)\). The key points are:

- \(K\) is entirely outside the interior of itself but is bounded by the orange curve.
- The dashed lines and the dashdotted line/segment provide specific geometric constraints related to the boundary of \(K\) and the construction of \(\underline{M}_q(K, L)\).
- The line through \(x\) supports the convex hull of \(K \cup L\) at \(x\), indicating a tangency condition.

This setup is crucial for proving the strict inequality in the lemma, as it helps to establish the relative positions and configurations of the sets involved.