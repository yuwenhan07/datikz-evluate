To understand the boundary string energy as a function of \( b \) for \( \frac{\pi}{4} < b < \frac{\pi}{2} \), we need to consider the context in which this expression arises. Typically, such expressions appear in the study of string theory or related fields where the energy of a string is dependent on its shape and parameters.

One common scenario is the study of a string in a two-dimensional space with a specific boundary condition. For example, if we have a string that is constrained to lie on a surface defined by a boundary condition, the energy of the string can be expressed as a function of the parameter \( b \).

Let's assume that the boundary string energy \( E(b) \) is given by some functional form. For simplicity, let's consider a hypothetical form of the energy function:

\[ E(b) = A \sin^2(b - \frac{\pi}{4}) + B \cos^2(b - \frac{\pi}{4}) \]

where \( A \) and \( B \) are constants. This form is chosen because it captures the periodic nature of the trigonometric functions and is symmetric around \( b = \frac{\pi}{4} \), which fits within the interval \( \frac{\pi}{4} < b < \frac{\pi}{2} \).

To find the energy as a function of \( b \), we need to evaluate this expression within the specified range. Let's break it down step-by-step:

1. **Identify the range of \( b \):**
   \[ \frac{\pi}{4} < b < \frac{\pi}{2} \]

2. **Substitute \( b \) into the energy function:**
   \[ E(b) = A \sin^2(b - \frac{\pi}{4}) + B \cos^2(b - \frac{\pi}{4}) \]

3. **Analyze the behavior of the trigonometric functions:**
   - For \( \frac{\pi}{4} < b < \frac{\pi}{2} \), the argument \( b - \frac{\pi}{4} \) ranges from 0 to \(\frac{\pi}{4}\).
   - In this range, both \(\sin(b - \frac{\pi}{4})\) and \(\cos(b - \frac{\pi}{4})\) are positive and decrease from 1 to \(\frac{\sqrt{2}}{2}\).

4. **Evaluate the energy function:**
   Since \(\sin^2(x)\) and \(\cos^2(x)\) are both non-negative and sum up to 1 for any \( x \), the energy function \( E(b) \) will be a combination of these terms. The exact form of \( E(b) \) will depend on the values of \( A \) and \( B \), but generally, it will be a smooth function that varies between certain bounds.

Given the complexity of the exact form without specific values for \( A \) and \( B \), the general form of the energy function within the specified range is:

\[ E(b) = A \sin^2(b - \frac{\pi}{4}) + B \cos^2(b - \frac{\pi}{4}) \]

This expression describes how the energy of the string changes as a function of \( b \) within the interval \( \frac{\pi}{4} < b < \frac{\pi}{2} \). The exact numerical values of \( A \) and \( B \) would need to be determined based on the specific physical or mathematical model being considered.