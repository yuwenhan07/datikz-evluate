In the context of an orthogonal setting where \( (t, s) \) is a tuple with \( t \in \mathbb{S}^{n-1} \) and \( s \in \mathbb{S}^{n-2} \), and \( t \perp s \), we need to understand what these terms mean:

1. **\( t \in \mathbb{S}^{n-1} \)**: This means that \( t \) is a unit vector in \( n \)-dimensional space. The set \( \mathbb{S}^{n-1} \) represents the unit sphere in \( n \)-dimensional Euclidean space.

2. **\( s \in \mathbb{S}^{n-2} \)**: Similarly, \( s \) is a unit vector in \( n-1 \)-dimensional space. The set \( \mathbb{S}^{n-2} \) represents the unit sphere in \( n-1 \)-dimensional Euclidean space.

3. **\( t \perp s \)**: This indicates that the vectors \( t \) and \( s \) are orthogonal to each other. In other words, their dot product is zero:
   \[
   t \cdot s = 0
   \]

### Interpretation

Given this setup, the tuple \( (t, s) \) represents a pair of orthogonal unit vectors where:
- \( t \) lies on the unit sphere in \( n \)-dimensional space.
- \( s \) lies on the unit sphere in \( n-1 \)-dimensional space.
- \( t \) and \( s \) are perpendicular to each other.

This type of configuration often arises in various mathematical and physical contexts, such as:
- In geometric algebra, where \( t \) and \( s \) might represent orthogonal directions or basis vectors.
- In optimization problems, where \( t \) and \( s \) could be part of a constraint involving orthogonality.
- In machine learning, where \( t \) and \( s \) might represent orthogonal features or directions in high-dimensional spaces.

If you have any specific application or further questions about this setup, feel free to ask!