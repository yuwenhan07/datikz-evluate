To understand and visualize the graph of the right-continuous step function \(\alpha(E_{p/q}^{\boxtimes 3})\) for \(p/q \in \mathbb{Q} \cap [2,3)\) with discontinuity points as described in \autoref{th:symm-disc}, we need to break down the problem into several steps.

1. **Understanding the Function \(\alpha\)**:
   - The function \(\alpha\) is a step function that takes values based on the rational number \(p/q\) in the interval \([2, 3)\).
   - Since \(p/q\) is a rational number in this interval, it can be expressed as a fraction where the denominator \(q\) is an integer and the numerator \(p\) is such that \(2 \leq p < q\).

2. **Discontinuity Points**:
   - The discontinuity points of \(\alpha\) are given by \autoref{th:symm-disc}. These points are specific rational numbers in the interval \([2, 3)\) where the value of \(\alpha\) changes abruptly.
   - For simplicity, let's assume that the discontinuity points are known and listed. Let's denote these points as \(x_1, x_2, \ldots, x_n\).

3. **Step Function Behavior**:
   - Between any two consecutive discontinuity points, the function \(\alpha\) is constant. This means that if \(x_i < x < x_{i+1}\), then \(\alpha(x) = c_i\) for some constant \(c_i\).
   - At each discontinuity point \(x_i\), the function \(\alpha\) jumps from one constant value to another. Specifically, \(\alpha(x_i^-) = c_i\) and \(\alpha(x_i^+) = c_{i+1}\), where \(c_i\) and \(c_{i+1}\) are the constants just before and after the jump at \(x_i\).

4. **Constructing the Graph**:
   - To construct the graph, we plot the function \(\alpha\) piecewise constant between the discontinuity points.
   - For each interval \((x_i, x_{i+1})\), draw a horizontal line segment at height \(c_i\).
   - At each discontinuity point \(x_i\), draw a vertical jump from height \(c_i\) to height \(c_{i+1}\).

5. **Example**:
   - Suppose the discontinuity points are \(x_1 = 2.25\), \(x_2 = 2.5\), and \(x_3 = 2.75\). Then the function \(\alpha\) would have the following behavior:
     - For \(2 \leq x < 2.25\), \(\alpha(x) = c_1\).
     - For \(2.25 \leq x < 2.5\), \(\alpha(x) = c_2\).
     - For \(2.5 \leq x < 2.75\), \(\alpha(x) = c_3\).
     - For \(2.75 \leq x < 3\), \(\alpha(x) = c_4\).
   - At \(x = 2.25\), there is a jump from \(c_1\) to \(c_2\).
   - At \(x = 2.5\), there is a jump from \(c_2\) to \(c_3\).
   - At \(x = 2.75\), there is a jump from \(c_3\) to \(c_4\).

The final answer, in terms of a description of the graph, is:

\[
\boxed{\text{The graph consists of horizontal line segments between the discontinuity points, with vertical jumps at each discontinuity point.}}
\]

This description captures the essence of the step function \(\alpha\) with the specified properties.