The given statement appears to be a representation of a process involving sets \(A\) and \(B\), variables \(z_k\), \(\hat{z}_k\), and an index \(i\). Let's break down each part of the statement:

1. **\(f_i(z_{k+1} + 1) \notin A \cap B\)**:
   - This part indicates that the function \(f_i\) applied to the value \(z_{k+1} + 1\) does not belong to the intersection of sets \(A\) and \(B\).
   - In other words, \(f_i(z_{k+1} + 1)\) is not in both sets \(A\) and \(B\) simultaneously.

2. **\(z_k = 0\)**:
   - This part specifies that the variable \(z_k\) has the value 0 at this step of the process.

3. **\(\hat{z}_k = 1\)**:
   - This part indicates that another variable, denoted as \(\hat{z}_k\), has the value 1 at this step of the process.

### Interpretation:
The statement seems to describe a conditional or sequential process where certain conditions must be met for the next step to occur. Specifically, it suggests that if the value of \(z_k\) is 0 and \(\hat{z}_k\) is 1, then the function \(f_i\) evaluated at \(z_{k+1} + 1\) should not lie within the intersection of sets \(A\) and \(B\).

### Example Context:
This could be part of an algorithm or a mathematical proof where these conditions are used to determine the next state or to ensure certain properties hold true. For instance, in a decision-making process, the condition \(f_i(z_{k+1} + 1) \notin A \cap B\) might be a stopping criterion or a condition for transitioning to a different state.

### Summary:
The statement is a logical condition that combines the values of variables \(z_k\), \(\hat{z}_k\), and the result of applying a function \(f_i\) to a transformed value \(z_{k+1} + 1\), with respect to the sets \(A\) and \(B\). It is likely part of a more complex algorithm or proof where these conditions are crucial for the progression of the process.