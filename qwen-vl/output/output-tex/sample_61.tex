The encoding of the combinatorial problem instance described can be broken down into several components: value-vertices, variable-vertices, constraint-vertices, and model-vertices. Let's go through each component step-by-step.

### Value-Vertices
Value-vertices represent the possible values that variables can take. In this case, there are three value-vertices:
- \( v_{1,1} \) representing the value 1.
- \( v_{1,2} \) representing the value 2.
- \( v_{2,1} \) representing the value 1.
- \( v_{2,2} \) representing the value 2.

These vertices are connected to the variable-vertices \( x_1 \) and \( x_2 \) based on their domains:
- \( x_1 \) has the domain \(\{1, 2\}\), so edges connect \( x_1 \) to both \( v_{1,1} \) and \( v_{1,2} \).
- \( x_2 \) also has the domain \(\{1, 2\}\), so edges connect \( x_2 \) to both \( v_{2,1} \) and \( v_{2,2} \).

### Variable-Vertices
Variable-vertices represent the variables themselves. Here, we have two variable-vertices:
- \( x_1 \)
- \( x_2 \)

### Constraint-Vertices
Constraint-vertices represent the constraints in the problem. There are two types of constraints mentioned:
1. **Inequality Constraint (\(\leq\))**
2. **Table Constraint (\(\textsc{table}\))**

#### Inequality Constraint (\(\leq\))
The inequality constraint is represented by a constraint-vertex labeled with the operator \(\leq\):
- \( c_{\leq} \)

This constraint vertex is connected to the model-vertex and the variable-vertices \( x_1 \) and \( x_2 \).

#### Table Constraint (\(\textsc{table}\))
The table constraint is represented by a constraint-vertex labeled with the operator \(\textsc{table}\):
- \( c_{\textsc{table}} \)

This constraint vertex is also connected to the model-vertex and the variable-vertices \( x_1 \) and \( x_2 \).

### Model-Vertex
The model-vertex represents the objective function or the main entity that ties all the constraints together. It is connected to the variable-vertices \( x_1 \) and \( x_2 \), and to the constraint-vertices \( c_{\leq} \) and \( c_{\textsc{table}} \).

### Operators and Features
The operators and features are used to specify the nature of the constraints:
- **Multiplication Operator (\(\times\))**:
  - For the inequality constraint, the operator is \(\times\) with feature 3, indicating the coefficient 3.
  - For the table constraint, the operator is \(\times\) with feature 4, indicating the coefficient 4.

### Summary of Connections
- \( x_1 \) is connected to \( v_{1,1} \) and \( v_{1,2} \).
- \( x_2 \) is connected to \( v_{2,1} \) and \( v_{2,2} \).
- \( c_{\leq} \) is connected to \( x_1 \), \( x_2 \), and the model-vertex.
- \( c_{\textsc{table}} \) is connected to \( x_1 \), \( x_2 \), and the model-vertex.
- The model-vertex is connected to \( x_1 \), \( x_2 \), \( c_{\leq} \), and \( c_{\textsc{table}} \).

### Final Representation
The final representation of the problem can be summarized as follows:

- **Value-Vertices**: \( v_{1,1}, v_{1,2}, v_{2,1}, v_{2,2} \)
- **Variable-Vertices**: \( x_1, x_2 \)
- **Constraint-Vertices**: \( c_{\leq}, c_{\textsc{table}} \)
- **Model-Vertex**: \( m \)

Connections:
- \( x_1 \rightarrow v_{1,1}, v_{1,2} \)
- \( x_2 \rightarrow v_{2,1}, v_{2,2} \)
- \( c_{\leq} \rightarrow x_1, x_2, m \)
- \( c_{\textsc{table}} \rightarrow x_1, x_2, m \)
- \( m \rightarrow x_1, x_2, c_{\leq}, c_{\textsc{table}} \)

This structure captures the combinatorial problem instance and its constraints in a graph-like format, which can be further processed or solved using appropriate algorithms.