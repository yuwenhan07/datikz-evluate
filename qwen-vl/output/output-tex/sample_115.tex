To address the problem, we need to understand the context and the specific elements involved. Let's break it down step by step.

1. **Understanding the Conjecture and Example**: The conjecture in question is likely related to some form of graph theory or combinatorial structure, possibly involving properties like connectivity, independence, or other graph-theoretic concepts. Example \ref{ex:laaksospace} provides a specific instance of this structure, and Lemma \ref{lem:removableedges} describes how to modify this structure by adding edges.

2. **Adding an Edge**: The lemma suggests that by adding an edge to a specific vertex \( v = 3 \) in the structure from Example \ref{ex:laaksospace}, we can obtain a new structure. This new structure is then claimed to be a counterexample to the conjecture.

3. **Smallest Replacement Rule**: The term "smallest replacement rule" implies that we are looking for the simplest or most minimal change that leads to a counterexample. In this case, the change involves adding just one edge to the vertex \( v = 3 \).

4. **Conclusion**: The figure you are referring to would be a visual representation of this new structure, which is obtained by adding the specified edge to the vertex \( v = 3 \). This new structure is designed to demonstrate that the original conjecture does not hold under certain conditions.

Since the exact details of the conjecture, the structure, and the specific modifications are not provided in your question, I cannot provide a detailed figure here. However, the process described above outlines the logical steps to understand and construct such a figure.

If you have access to the original sources (the paper or document containing Examples \ref{ex:laaksospace} and Lemma \ref{lem:removableedges}), you should refer to those for the precise details and the actual figure.

In summary, the figure you are referring to would show the structure resulting from adding the edge to the vertex \( v = 3 \), demonstrating that this simple modification is sufficient to produce a counterexample to the conjecture.