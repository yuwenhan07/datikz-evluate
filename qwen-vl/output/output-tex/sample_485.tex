The process you're describing involves creating a detailed anatomical drawing of the head using X-ray images at three different coronal (horizontal) levels. Here’s a step-by-step guide on how this might be done:

### Step 1: Understanding the X-ray Image
- **X-ray Beam**: The X-ray beam passes through the head, and the resulting shadows are captured on an X-ray film.
- **Coronal Levels**: These are horizontal slices through the head, typically taken at different depths to capture different structures.

### Step 2: Analyzing the X-ray Film
- **Main Drawing**: The X-ray film shows the internal structures of the head as shadows. These shadows can represent various tissues such as bone, soft tissue, and air-filled spaces like sinuses.
- **Color Coding**: Each structure is represented by a specific color or shape in the drawing to differentiate between them.

### Step 3: Creating the Anatomical Drawing
- **Top View**: The drawing is created from a top-down perspective, showing the head as if it were viewed from above.
- **Coronal Levels**: The drawing includes three distinct levels, each representing a different depth within the head:
  - **Level 1**: Closest to the surface of the head.
  - **Level 2**: Mid-depth within the head.
  - **Level 3**: Deepest within the head.

### Step 4: Representing Structures
- **Bone**: Bones are typically drawn in white or light gray, depending on the X-ray density.
- **Soft Tissue**: Soft tissues are often depicted in darker shades, such as black or dark gray.
- **Sinuses**: Air-filled spaces are shown in white or very light gray, indicating high X-ray density.
- **Other Structures**: Other structures like muscles, blood vessels, and nerves may be represented with specific colors or patterns.

### Step 5: Adding Shadows to the Drawing
- **Shadows**: The shadows from the X-ray film are placed next to the corresponding levels in the drawing. These shadows provide a visual representation of the X-ray image and help in understanding the depth and location of the structures.

### Example:
Imagine the following scenario:
- **Level 1**: Shows the outer skull and some superficial soft tissues.
- **Level 2**: Displays the inner skull, sinuses, and deeper soft tissues.
- **Level 3**: Reveals the brain, cerebrospinal fluid spaces, and other deep structures.

### Final Drawing
The final drawing would look something like this:
- **Top View**: A clear top-down view of the head.
- **Three Levels**: Each level is labeled and shows the corresponding structures in white, black, and gray.
- **X-ray Shadows**: Shadows from the X-ray film are placed next to each level, providing a visual reference.

This method helps medical and dental professionals understand the internal anatomy of the head more accurately, aiding in diagnosis and treatment planning.