To solve this problem, we need to understand the context of the graph and the lemma it refers to. Let's break down the problem step by step.

1. **Understanding the Graph and the Lemma**:
   - The graph is likely a directed graph where each node represents a state, and each edge represents a transition between states.
   - The lemma (Lemma \ref{lemma:deg active}) is about the degrees of the active edges in the graph when certain operations (adding \(U\) or \(D\), represented by green or blue arrows) are performed.
   - The green, blue, and purple boxes are visual aids that help prove the lemma.

2. **Analyzing the Green Box**:
   - The green box likely shows a specific configuration of the graph where the degree of the active edge after adding an \(U\) (green arrow) is analyzed.
   - If the green box shows that the degree of the active edge is 0, it means that after adding an \(U\), the degree of the active edge becomes 0. This could be due to the structure of the graph where adding an \(U\) disconnects the edge from the rest of the graph.

3. **Analyzing the Blue Box**:
   - The blue box likely shows a different configuration of the graph where the degree of the active edge after adding a \(D\) (blue arrow) is analyzed.
   - If the blue box shows that the degree of the active edge is 1, it means that after adding a \(D\), the degree of the active edge becomes 1. This could be due to the structure of the graph where adding a \(D\) connects the edge to another node, increasing its degree by 1.

4. **Analyzing the Purple Box**:
   - The purple box likely shows a combination of both configurations, demonstrating how the degrees of the active edges change under both operations.
   - If the purple box shows that the degrees of the active edges can be 0 or 1 depending on whether an \(U\) or \(D\) is added, it confirms the lemma that the degrees of the active edges are 0 or 1.

5. **Conclusion**:
   - Based on the analysis of the green, blue, and purple boxes, we can conclude that the lemma is proven. The lemma states that the degrees of the active edges are 0 or 1, and the boxes provide visual evidence supporting this claim.

Therefore, the answer is:

\[
\boxed{0 \text{ or } 1}
\]