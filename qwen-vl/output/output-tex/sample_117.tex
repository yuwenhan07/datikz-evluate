To solve the problem, we need to understand the context and the relationships between the segments and the arc. However, the problem statement does not provide enough specific details about the geometric configuration or the exact nature of the segments and the arc. Therefore, I will assume a common scenario where these segments and the arc are part of a circle or a similar geometric figure.

Let's assume that the segments \(\widetilde{T}_i\) and \(\widetilde{T}_{i+1}\) are chords of a circle, and the arc \(C_i\) is the arc that subtends these chords. The length of each dotted segment (which appears to be a chord) is given as \(a_{n+1}/100\).

In such a scenario, if we consider the circle with radius \(R\), the length of a chord can be calculated using the formula:
\[ \text{Length of chord} = 2R \sin\left(\frac{\theta}{2}\right) \]
where \(\theta\) is the central angle subtended by the chord at the center of the circle.

Given that the length of each dotted segment is \(a_{n+1}/100\), we can express this as:
\[ 2R \sin\left(\frac{\theta}{2}\right) = \frac{a_{n+1}}{100} \]

However, without more specific information about the radius \(R\) or the central angle \(\theta\), we cannot determine the exact value of \(a_{n+1}\). If we assume that the problem is asking for the relationship or the expression for the length of the chord in terms of \(a_{n+1}\), then the answer would be:

\[ \boxed{\frac{a_{n+1}}{100}} \]

This is the length of each dotted segment, which is given directly in the problem statement. If there is additional context or specific values for \(R\) or \(\theta\), we could use those to find the exact numerical value of \(a_{n+1}\). But based on the information provided, the answer is simply the length of the chord, which is \(\frac{a_{n+1}}{100}\).