The Gibbs property in the context of the $q$-Boson models refers to a specific type of statistical mechanics where the partition function can be expressed as a product over all possible configurations of the system. This property is crucial for understanding the behavior of the system at equilibrium.

### Left: Uncolored $q$-Boson Model

In the uncolored $q$-Boson model, the Gibbs property is defined by conditioning on the number of arrows (representing bosons) that exit the region $\Lambda$ (the blue outlined area). The arrows that enter the region $\Lambda$ are also conditioned on due to consistency, but those that exit the region are not. This means that the probability distribution of the system is determined by the number of arrows leaving the region $\Lambda$, while the number of arrows entering the region is fixed by the conservation of particles within the region.

### Right: Colored $q$-Boson Model

In the colored $q$-Boson model, the Gibbs property is more complex. Here, the system is divided into two regions by a blue dashed line. To the left of this line, the number of horizontally outgoing colored arrows is conditioned on strictly. This means that the probability distribution of the system is determined by the number of colored arrows that exit the region to the left of the blue dashed line. 

To the right of the blue dashed line, only the total number of outgoing arrows (both colored and uncolored) is conditioned on. This includes the total number of arrows that exit the region $\Gamma_k$. This distinction between the two regions allows for a more nuanced description of the system's behavior, particularly in terms of how the colored and uncolored particles interact across the boundary.

### Summary

- **Uncolored $q$-Boson Model**: Conditioned on the number of arrows exiting the region $\Lambda$.
- **Colored $q$-Boson Model**: Conditioned on the number of horizontally outgoing colored arrows to the left of the blue dashed line, and on the total number of outgoing arrows to the right of the blue dashed line.

These conditions ensure that the Gibbs property holds, allowing the partition function to be expressed as a product over all possible configurations of the system, which is essential for calculating thermodynamic properties such as the free energy and entropy.