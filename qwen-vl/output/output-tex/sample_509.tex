It seems like you're referring to a specific type of graph known as a hypergraph, where the parameters \( H(a,b) \) denote a hypergraph with \( a \) vertices and \( b \) hyperedges. However, the terms \(\operatorname{zir}\), \(\operatorname{Z}\), \(\overline{\operatorname{Z}}\), and \(\operatorname{ZIR}\) do not have standard definitions in the context of hypergraphs or graph theory that I'm aware of.

If these terms are specific to a particular context or field (such as combinatorics, algebraic graph theory, or another specialized area), it would be helpful to provide more details about their definitions within that context. Otherwise, if you're looking for information on general properties of hypergraphs, here's some basic information:

1. **Hypergraph Definition**: A hypergraph is a generalization of a graph in which an edge can connect any number of vertices. Formally, a hypergraph \( H \) consists of a set of vertices \( V(H) \) and a set of hyperedges \( E(H) \), where each hyperedge is a subset of \( V(H) \).

2. **Parameters \( H(a,b) \)**: The notation \( H(a,b) \) typically indicates a hypergraph with \( a \) vertices and \( b \) hyperedges. This is a common way to specify the size of a hypergraph.

3. **Properties**:
   - **ZIR (Zero-Index Rank)**: In the context of hypergraphs, the zero-index rank might refer to a specific property related to the rank of certain matrices associated with the hypergraph, but without a specific definition, it's hard to elaborate.
   - **Z (Zero-Index)**: Similarly, the term "Z" could refer to a zero-indexed property, but again, without a specific context, it's unclear what this means.
   - **\(\overline{\operatorname{Z}}\) (Complement of Z)**: If "Z" refers to something, then \(\overline{\operatorname{Z}}\) would likely refer to its complement or negation.
   - **\(\operatorname{ZIR}\) (Zero-Index Rank)**: This could be a combination of the previous terms, possibly referring to a rank property related to the zero-indexed structure of the hypergraph.

If you can provide more context or clarify the definitions of these terms within your specific field or problem, I'd be happy to help further!