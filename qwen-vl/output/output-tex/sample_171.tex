To solve this problem, we need to understand the different patterns described and how they can be constructed with \( k = 3 \) and \( m = 4 \). Let's break down each pattern:

### (a) Sunflower
A sunflower pattern is constructed around one central qubit. Each of the other qubits is connected to the central qubit and to two other qubits that are also connected to the central qubit. This forms a star-like structure where the central qubit is at the center, and the other qubits form a ring around it.

For \( k = 3 \) and \( m = 4 \), the sunflower pattern would look like this:
- One central qubit.
- Four qubits arranged in a ring around the central qubit, each connected to the central qubit and to two adjacent qubits in the ring.

### (b) Loose Chain
A loose chain is a linear sequence of qubits where each qubit is connected to its two neighbors, but there might be some gaps or "loose" connections that do not follow a strict chain pattern. However, for simplicity and given the constraints, we can assume a loose chain as a simple linear chain without any additional connections.

For \( k = 3 \) and \( m = 4 \), a loose chain would look like this:
- Qubits 1, 2, 3, and 4 are connected in a linear sequence: 1 → 2 → 3 → 4.

### (c) Strong Chain
A strong chain is similar to a loose chain but with each qubit connected to \( k = 3 \) clauses, except for the boundary qubits which are only connected to their immediate neighbors. So, each internal qubit is connected to three other qubits, while the boundary qubits are connected to two other qubits.

For \( k = 3 \) and \( m = 4 \), a strong chain would look like this:
- Qubits 1 and 4 are boundary qubits and are connected to two other qubits.
- Qubits 2 and 3 are internal qubits and are connected to three other qubits.

### (d) Loose Cycle
A loose cycle is a circular sequence of qubits where each qubit is connected to its two neighbors, but there might be some gaps or "loose" connections that do not follow a strict cycle pattern. However, for simplicity and given the constraints, we can assume a loose cycle as a simple cycle without any additional connections.

For \( k = 3 \) and \( m = 4 \), a loose cycle would look like this:
- Qubits 1, 2, 3, and 4 are connected in a circular sequence: 1 → 2 → 3 → 4 → 1.

### (e) Strong Cycle
A strong cycle is similar to a loose cycle but with each qubit connected to \( k = 3 \) clauses, except for the boundary qubits which are only connected to their immediate neighbors. So, each internal qubit is connected to three other qubits, while the boundary qubits are connected to two other qubits.

For \( k = 3 \) and \( m = 4 \), a strong cycle would look like this:
- Qubits 1 and 4 are boundary qubits and are connected to two other qubits.
- Qubits 2, 3, and 4 are internal qubits and are connected to three other qubits.

### Final Answer
The patterns for \( k = 3 \) and \( m = 4 \) are:
(a) Sunflower
(b) Loose chain
(c) Strong chain
(d) Loose cycle
(e) Strong cycle

So, the final answer is:
\[
\boxed{(a, b, c, d, e)}
\]