To determine whether the function \( f(x) = |e^x + e^{-x} - 3| \) exhibits exponential growth as \( \|x\| \to +\infty \), we need to analyze the behavior of the expression inside the absolute value as \( x \) becomes very large.

First, let's consider the expression \( e^x + e^{-x} \). As \( x \) approaches \( +\infty \), \( e^x \) grows exponentially and dominates \( e^{-x} \), which approaches 0. Therefore, for large positive \( x \), we can approximate:
\[ e^x + e^{-x} \approx e^x. \]

Now, let's substitute this approximation into the function \( f(x) \):
\[ f(x) = |e^x + e^{-x} - 3| \approx |e^x - 3|. \]
Since \( e^x \) is always positive and grows without bound as \( x \to +\infty \), \( e^x - 3 \) will also grow without bound. Specifically, \( e^x - 3 \) will be positive for all \( x > \ln(3) \), so we have:
\[ f(x) \approx e^x - 3. \]
As \( x \to +\infty \), \( e^x - 3 \) grows exponentially, meaning that \( f(x) \) also grows exponentially.

To confirm this, let's consider the limit:
\[ \lim_{x \to +\infty} \frac{f(x)}{e^x} = \lim_{x \to +\infty} \frac{|e^x + e^{-x} - 3|}{e^x}. \]
For large \( x \), \( e^{-x} \) becomes very small, so we can approximate:
\[ \frac{|e^x + e^{-x} - 3|}{e^x} \approx \frac{|e^x - 3|}{e^x} = \left|1 - \frac{3}{e^x}\right|. \]
As \( x \to +\infty \), \( \frac{3}{e^x} \to 0 \), so:
\[ \left|1 - \frac{3}{e^x}\right| \to 1. \]
This shows that \( f(x) \) grows at least as fast as \( e^x \) as \( x \to +\infty \).

Therefore, the function \( f(x) = |e^x + e^{-x} - 3| \) exhibits exponential growth as \( \|x\| \to +\infty \). The final answer is:
\[ \boxed{\text{exponential growth}}. \]