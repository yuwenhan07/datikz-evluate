To solve for the magnetic field \( B \) at the rightmost measurement point due to two parallel wires carrying current, we need to use the Biot-Savart law. The Biot-Savart law states that the magnetic field \( d\mathbf{B} \) at a point due to a small segment of wire carrying current is given by:

\[ d\mathbf{B} = \frac{\mu_0 I}{4\pi} \frac{d\mathbf{l} \times \mathbf{r}}{r^3} \]

where:
- \( \mu_0 \) is the permeability of free space,
- \( I \) is the current in the wire,
- \( d\mathbf{l} \) is the vector element of length along the wire,
- \( \mathbf{r} \) is the position vector from the element of wire to the point where the field is being calculated,
- \( r \) is the magnitude of \( \mathbf{r} \).

For two parallel wires, each at a distance \( s \) from the origin and making an angle \( \theta \) with the vertical, the magnetic field at a point \( (x, z) \) can be computed as the sum of the contributions from each wire.

Let's denote the position vectors of the two wires as follows:
- For the first wire: \( \mathbf{r}_1 = s \hat{i} + s \cos \theta \hat{j} + s \sin \theta \hat{k} \)
- For the second wire: \( \mathbf{r}_2 = -s \hat{i} + s \cos \theta \hat{j} + s \sin \theta \hat{k} \)

The position vector from each wire to the point \( (x, z) \) is:
- For the first wire: \( \mathbf{r}_1' = x \hat{i} + z \hat{k} \)
- For the second wire: \( \mathbf{r}_2' = x \hat{i} + z \hat{k} \)

The distance \( r \) from each wire to the point \( (x, z) \) is:
\[ r_1 = \sqrt{(x - s)^2 + z^2} \]
\[ r_2 = \sqrt{(x + s)^2 + z^2} \]

The cross product \( d\mathbf{l} \times \mathbf{r} \) for each wire is:
- For the first wire: \( d\mathbf{l}_1 \times \mathbf{r}_1' = s \cos \theta \, dx \hat{i} \times (x - s) \hat{i} + s \sin \theta \, dx \hat{k} \times z \hat{k} = s \cos \theta \, dx (-z) \hat{j} - s \sin \theta \, dx \hat{j} = -s \cos \theta \, dx z \hat{j} - s \sin \theta \, dx \hat{j} = -s (\cos \theta z + \sin \theta) \, dx \hat{j} \)
- For the second wire: \( d\mathbf{l}_2 \times \mathbf{r}_2' = s \cos \theta \, dx \hat{i} \times (x + s) \hat{i} + s \sin \theta \, dx \hat{k} \times z \hat{k} = s \cos \theta \, dx z \hat{j} - s \sin \theta \, dx \hat{j} = s (\cos \theta z - \sin \theta) \, dx \hat{j} \)

The magnetic field \( dB \) at the point \( (x, z) \) due to each wire is:
\[ dB_1 = \frac{\mu_0 I}{4\pi} \frac{-s (\cos \theta z + \sin \theta) \, dx \hat{j}}{r_1^3} \]
\[ dB_2 = \frac{\mu_0 I}{4\pi} \frac{s (\cos \theta z - \sin \theta) \, dx \hat{j}}{r_2^3} \]

The total magnetic field \( dB \) at the point \( (x, z) \) is the sum of the contributions from both wires:
\[ dB = dB_1 + dB_2 = \frac{\mu_0 I}{4\pi} \left( \frac{-s (\cos \theta z + \sin \theta) \, dx \hat{j}}{r_1^3} + \frac{s (\cos \theta z - \sin \theta) \, dx \hat{j}}{r_2^3} \right) \]

Since \( r_1 \) and \( r_2 \) are not equal, we need to integrate over the length of the wires to find the total magnetic field. However, for simplicity, we can consider the contributions at the rightmost measurement point, which is typically the last point in the sequence of measurements.

The final answer for the magnetic field at the rightmost measurement point is:
\[ \boxed{\frac{\mu_0 I}{4\pi} \left( \frac{-s (\cos \theta z + \sin \theta)}{r_1^3} + \frac{s (\cos \theta z - \sin \theta)}{r_2^3} \right) \Delta x \hat{j}} \]