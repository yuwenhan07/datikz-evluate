To understand the graphs \(\operatorname{Graph}(\Sigma_+), \operatorname{Graph}(c + \d T^+[u_c]),\) and \(\operatorname{Graph}(c + \nabla^+ u_c)\) at cohomology class \(c = 0\) in the pendulum case, we need to delve into the context of Morse theory and the geometry of the pendulum system.

### Pendulum System
A simple pendulum is a mass \(m\) attached to a pivot by a rod of length \(L\). The potential energy \(V\) of the pendulum is given by:
\[ V(\theta) = mgL(1 - \cos \theta) \]
where \(\theta\) is the angle from the vertical.

### Morse Theory
Morse theory studies the topology of manifolds using the critical points of smooth functions on these manifolds. In the context of the pendulum, we can consider the energy function \(E(\theta, \dot{\theta}) = \frac{1}{2} m L^2 \dot{\theta}^2 + mgL(1 - \cos \theta)\).

### Critical Points
The critical points of the energy function correspond to the equilibrium positions of the pendulum. For the pendulum, there are two critical points: \(\theta = 0\) (the vertical position) and \(\theta = \pm \pi\) (the horizontal positions).

### Graphs of Interest
1. **\(\operatorname{Graph}(\Sigma_+)\)**:
   - This graph typically refers to the stable manifold of a critical point. In the pendulum case, the stable manifold of the vertical position \(\theta = 0\) would be the set of initial conditions that lead to the pendulum swinging back to the vertical position.
   - At \(c = 0\), this would be the set of initial conditions where the pendulum starts at rest or with a small initial velocity in the vertical direction.

2. **\(\operatorname{Graph}(c + \d T^+[u_c])\)**:
   - Here, \(T^+[u_c]\) likely refers to the positive part of the gradient flow of some function \(u_c\). The gradient flow is a vector field that points in the direction of steepest descent of the function.
   - At \(c = 0\), this would be the set of initial conditions that evolve under the gradient flow starting from the critical point \(\theta = 0\).

3. **\(\operatorname{Graph}(c + \nabla^+ u_c)\)**:
   - \(\nabla^+ u_c\) likely refers to the positive part of the gradient of the function \(u_c\).
   - At \(c = 0\), this would be the set of initial conditions that evolve under the gradient flow starting from the critical point \(\theta = 0\), but only considering the positive part of the gradient.

### Summary
- \(\operatorname{Graph}(\Sigma_+)\) at \(c = 0\) represents the stable manifold of the vertical position \(\theta = 0\).
- \(\operatorname{Graph}(c + \d T^+[u_c])\) at \(c = 0\) represents the set of initial conditions that evolve under the gradient flow starting from the vertical position \(\theta = 0\).
- \(\operatorname{Graph}(c + \nabla^+ u_c)\) at \(c = 0\) represents the set of initial conditions that evolve under the gradient flow starting from the vertical position \(\theta = 0\), but only considering the positive part of the gradient.

These graphs provide insight into the dynamics of the pendulum system around its critical points and help in understanding the topological structure of the phase space.