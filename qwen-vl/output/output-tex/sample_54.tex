It sounds like you're discussing the memory usage of buffers in a graphics or rendering context, specifically focusing on the memory footprint of transparent objects and how your blending algorithm works.

Let's break down the information provided:

1. **Buffer Memory Usage**: 
   - For a \(256 \times 256\) texture, one transparent object buffer uses approximately 4.5 megabytes (MB) of memory.
   
2. **Blending Algorithm**:
   - Your blending algorithm is described as "summation-based," which implies that it likely involves summing up the contributions from different layers or components of the image.
   - Since it is summation-based, it does not require additional space to store each transparency buffer individually. This is a significant advantage because it reduces the memory overhead associated with storing multiple transparency buffers.

### Key Points:
- **Memory Efficiency**: The fact that your blending algorithm is summation-based means that it can process transparency without needing to store intermediate transparency values separately. This can be a substantial memory savings compared to algorithms that need to store each transparency value explicitly.
  
- **Transparent Object Buffer**: The 4.5 MB figure for a single transparent object buffer at \(256 \times 256\) resolution indicates the amount of memory required to store the transparency data for a single object. This is a reasonable amount of memory for a single transparent object, especially considering modern hardware capabilities.

### Considerations:
- **Performance Impact**: While the memory savings are beneficial, it's important to consider the performance implications of your summation-based approach. If the summation operations are computationally expensive, they could impact the overall performance of your application.
  
- **Scalability**: As the size of the textures or the number of transparent objects increases, the memory usage will also increase. It's crucial to monitor and manage this memory usage to ensure that your application remains efficient and performant.

In summary, your summation-based blending algorithm is a smart choice as it leverages the properties of transparency to reduce memory requirements, making it an efficient solution for managing transparent objects in your rendering pipeline.