To compute \( F_{j+1} \mid m_j \) from \( e^{G_{j+1} \tau_j} \) and \( e^{(G_{j+1} - G_j) \tau_j} \), we need to understand the context of these expressions, which typically arise in the context of numerical methods for solving differential equations or in the analysis of Markov chains.

Let's assume that \( G_j \) and \( G_{j+1} \) represent transition matrices or operators at time steps \( j \) and \( j+1 \), respectively, and \( \tau_j \) is the time step size between these two steps. The expressions \( e^{G_{j+1} \tau_j} \) and \( e^{(G_{j+1} - G_j) \tau_j} \) often appear in the context of matrix exponentials used in the analysis of discrete-time systems.

The expression \( F_{j+1} \mid m_j \) suggests a conditional probability or a transformation involving the state at time \( j+1 \) given the state at time \( j \). However, without more specific information about the nature of \( G_j \), \( G_{j+1} \), and \( \tau_j \), it's challenging to provide a precise computation.

If we assume that \( G_j \) and \( G_{j+1} \) are transition matrices in a Markov chain, then \( e^{G_{j+1} \tau_j} \) represents the transition matrix after one time step, and \( e^{(G_{j+1} - G_j) \tau_j} \) can be interpreted as the incremental change in the transition matrix over one time step.

In this context, if we want to compute \( F_{j+1} \mid m_j \), we might be looking for the probability of being in state \( m_{j+1} \) given that we were in state \( m_j \) at the previous time step. This would typically involve using the transition probabilities provided by the matrix \( e^{G_{j+1} \tau_j} \).

However, without more details on the specific form of \( G_j \), \( G_{j+1} \), and the states \( m_j \) and \( m_{j+1} \), it's not possible to provide a concrete formula. If you can provide more context or specify the exact nature of the problem, I can help you with a more detailed solution.