To understand how the projection surface of an exit sign changes with the viewing angle \(\theta\) and the orientation of the sign, we need to consider the geometry of the situation. The viewing angle \(\theta\) is influenced by both the orientation of the sign (which is described by the rotation angle \(\alpha_k\)) and the position of the viewer relative to the sign.

Let's break down the problem step by step:

1. **Orientation of the Sign**: The sign is oriented in the global \(z\)-plane by a rotation angle \(\alpha_k\). This means that the sign is rotated around the \(z\)-axis by \(\alpha_k\).

2. **Viewer's Position**: The viewer is located at a specific point in the agent cell \(i, j\). The position of the viewer affects the viewing angle \(\theta\) because it determines the direction from which the sign is being viewed.

3. **Projection Surface**: The projection surface of the sign is the area on the ground plane that the sign appears to project onto when viewed from the viewer's position. The shape and size of this projection surface depend on the orientation of the sign and the viewing angle \(\theta\).

The viewing angle \(\theta\) can be described mathematically as a function of the rotation angle \(\alpha_k\) and the viewer's position. Specifically, \(\theta\) is the angle between the line of sight from the viewer to the center of the sign and the normal vector to the plane of the sign. This angle \(\theta\) will change as the sign is rotated (\(\alpha_k\) changes) or as the viewer moves to different positions within the agent cell \(i, j\).

Therefore, the viewing angle \(\theta\) can be expressed as:
\[
\theta = f(\alpha_k, i, j)
\]
where \(f\) is a function that depends on the rotation angle \(\alpha_k\) and the coordinates \(i, j\) of the viewer's position.

The final answer is:
\[
\boxed{\theta = f(\alpha_k, i, j)}
\]