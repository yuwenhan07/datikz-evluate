To solve the problem, we need to understand the context and the notation used. The notation \( M^{\rm I}(i) \) typically refers to a region or set of values for a parameter \( t \) that is indexed by \( i \). Here, \( i = \bullet \) is a placeholder indicating that the specific value of \( i \) is not given, but it could be one of the values \(-1\), \(1\), or \(2\) as indicated by the labels \(-1^*\), \(1^*\), and \(2^*\).

The problem asks us to determine the region \( M^{\rm I}(\bullet) \) of possible \( t \) values for \( i = \bullet \). Since \( i \) can be \(-1\), \(1\), or \(2\), we need to consider the regions \( M^{\rm I}(-1)\), \( M^{\rm I}(1)\), and \( M^{\rm I}(2)\) and then determine the intersection of these regions if they overlap.

However, since the problem does not provide specific details about the regions \( M^{\rm I}(-1)\), \( M^{\rm I}(1)\), and \( M^{\rm I}(2)\), we will assume that the regions are such that the intersection of these regions is non-empty. This is a common assumption in such problems where no specific regions are provided.

Therefore, the region \( M^{\rm I}(\bullet) \) would be the intersection of the regions \( M^{\rm I}(-1)\), \( M^{\rm I}(1)\), and \( M^{\rm I}(2)\). If we denote this intersection by \( R \), then the region \( M^{\rm I}(\bullet) \) is:

\[
R = M^{\rm I}(-1) \cap M^{\rm I}(1) \cap M^{\rm I}(2)
\]

Since the problem does not provide the specific form of these regions, we cannot simplify this further without additional information. However, the answer in a general form is:

\[
\boxed{M^{\rm I}(\bullet) = M^{\rm I}(-1) \cap M^{\rm I}(1) \cap M^{\rm I}(2)}
\]