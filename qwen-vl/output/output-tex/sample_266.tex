Pinsky's combinatorial theorem involves a specific type of diagonal transformation on a grid, which is often used in the context of tiling problems or combinatorial games. Let's break down the problem and understand the transformation involved.

### Step-by-Step Explanation

1. **Understanding the Grid and Moves:**
   - Consider a grid where each cell can be represented by coordinates \((x, y)\).
   - The possible moves from a cell \((x, y)\) are \((0, 1)\), \((1, 0)\), \((-1, 0)\), and \((0, -1)\). These represent moving right, up, left, and down respectively.

2. **Diagonal Transformation:**
   - A diagonal transformation involves a specific type of move that combines these basic moves into a more complex pattern.
   - The given transformation is such that the next step is chosen from \(\{(0,1), (1,0), (-1,0), (0,-1)\}\).

3. **Equivalence to Diagonal Encoding:**
   - The problem states that this choice is equivalent to a "diagonal version" of the encoding being in the set \(\{(+1,+1), (+1,-1), (-1,-1), (-1,+1)\}\).
   - This means that the moves \((0,1)\) and \((1,0)\) correspond to \((+1,+1)\), \((+1,-1)\), \((-1,-1)\), and \((-1,+1)\) respectively when viewed in a diagonal perspective.

4. **Interpreting the Diagonal Perspective:**
   - In a diagonal perspective, we consider moves that change both the x-coordinate and y-coordinate simultaneously but in a way that they either increase or decrease together or in opposite directions.
   - For example:
     - \((0,1)\) corresponds to \((+1,+1)\) because it increases both x and y.
     - \((1,0)\) corresponds to \((+1,-1)\) because it increases x and decreases y.
     - \((-1,0)\) corresponds to \((-1,-1)\) because it decreases both x and y.
     - \((0,-1)\) corresponds to \((-1,+1)\) because it decreases y and increases x.

5. **Conclusion:**
   - The diagonal transformation described in Pinsky's combinatorial theorem is essentially a way of encoding the basic moves \((0,1)\), \((1,0)\), \((-1,0)\), and \((0,-1)\) into a set of diagonal moves \(\{(+1,+1), (+1,-1), (-1,-1), (-1,+1)\}\).
   - This equivalence allows us to simplify the problem by working with the diagonal moves instead of the basic moves, which can sometimes make the combinatorial analysis easier.

Thus, the diagonal transformation in Pinsky's combinatorial theorem is a way of encoding the basic moves into a set of diagonal moves, making the problem more manageable in terms of combinatorial analysis. The final answer is:

\[
\boxed{\{(0,1), (1,0), (-1,0), (0,-1)\} \text{ is equivalent to } \{(+1,+1), (+1,-1), (-1,-1), (-1,+1)\}}
\]