It seems like you're describing a set of measurements where \(\hat{\tau}\) is being estimated from different uniform distributions. Let's break down the information provided:

1. **Uniform Distributions**: The notation \(\mathcal{U}^k([a, b])\) represents a uniform distribution over the interval \([a, b]\). Here, \(k\) indicates the dimensionality (e.g., 5 for 5-dimensional and 10 for 10-dimensional), and \([a, b]\) specifies the range of the distribution.

2. **Samples**: You have taken 30 random samples from each of the following distributions:
   - \(\mathcal{U}^5([1, 100])\)
   - \(\mathcal{U}^5([1, 150])\)
   - \(\mathcal{U}^5([1, 200])\)
   - \(\mathcal{U}^{10}([1, 50])\)
   - \(\mathcal{U}^{10}([1, 80])\)
   - \(\mathcal{U}^{10}([1, 100])\)

3. **Estimation of \(\hat{\tau}\)**: For each of these distributions, you have measured or estimated \(\hat{\tau}\), which could be a parameter related to the distribution, such as the mean, median, mode, or some other characteristic.

### Key Points:
- **Dimensionality**: The superscript \(k\) in \(\mathcal{U}^k([a, b])\) indicates that the distribution is \(k\)-dimensional.
- **Range**: The interval \([a, b]\) defines the range over which the uniform distribution is defined. For example, \(\mathcal{U}^5([1, 100])\) means a 5-dimensional uniform distribution over the interval \([1, 100]\).

### Possible Interpretation:
If \(\hat{\tau}\) is a characteristic of the distribution, it could be the mean, median, or another statistical measure. For instance:
- If \(\hat{\tau}\) is the mean, then for \(\mathcal{U}^5([1, 100])\), the mean would be \(\frac{1 + 100}{2} = 50.5\).
- If \(\hat{\tau}\) is the median, then for \(\mathcal{U}^5([1, 100])\), the median would also be 50.5 because the uniform distribution is symmetric around its midpoint.

Given the large number of samples (30 per distribution), the estimates of \(\hat{\tau}\) should be quite accurate if the samples are truly random and representative.

### Summary:
You have collected data from six different uniform distributions, with varying dimensions and ranges, and for each distribution, you have estimated \(\hat{\tau}\). The specific nature of \(\hat{\tau}\) (mean, median, etc.) would depend on the context in which \(\hat{\tau}\) is used. If you need further analysis or interpretation, please specify what \(\hat{\tau}\) represents.