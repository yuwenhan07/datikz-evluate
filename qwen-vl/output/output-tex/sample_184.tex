It seems like you're discussing concepts related to Riemannian geometry or general relativity, specifically focusing on regions of space-time that are cut off by a sphere centered at \( x = x_0 \). Let's break down the key points and clarify the context:

1. **Riemannian Geometry and Space-Time**: In Riemannian geometry, we often deal with manifolds equipped with a metric tensor that defines distances and angles. When considering space-time in the context of general relativity, this metric tensor describes the curvature of spacetime due to mass and energy.

2. **Cutoff Sphere**: A "cutoff sphere" at \( x = x_0 \) refers to a spherical region centered at a specific point \( x_0 \) in the space-time manifold. This sphere serves as a boundary within which certain calculations or integrations might be performed.

3. **RT Surfaces (Regge-Teitelboim Surfaces)**: These are surfaces used in the Regge calculus approach to general relativity, where the geometry of space-time is discretized into a lattice. RT surfaces are typically defined by a set of points that satisfy certain conditions, such as being geodesic or having a specific area.

4. **Exclusion of Region Inside Cutoff Sphere**: When considering RT surfaces anchored at \( x_0 \), the part of the surface that lies inside the cutoff sphere is often excluded from the analysis. This exclusion is done for practical reasons, such as simplifying calculations or focusing on the behavior of the system outside the sphere.

5. **Length of the Excluded Region**: The length of the excluded region is given by \( |\phi(x_0)| \). Here, \( \phi(x_0) \) likely represents a function that characterizes the geometry or the properties of the space-time at the point \( x_0 \). The absolute value \( |\cdot| \) ensures that the length is always positive, regardless of the sign of \( \phi(x_0) \).

In summary, when dealing with RT surfaces anchored at \( x_0 \) and a cutoff sphere at the same point, the part of the surface inside the sphere is excluded, and its length is given by \( |\phi(x_0)| \). This exclusion helps in isolating the behavior of the system outside the sphere, making it easier to analyze or compute relevant quantities.