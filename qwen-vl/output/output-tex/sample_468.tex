To solve this problem, we need to understand the representation of 2-bridge knots and how they differ based on whether the tuple has an even or odd number of elements.

A 2-bridge knot is a knot that can be represented as a rational tangle with two bridges. This representation is given by a continued fraction of the form:
\[ \frac{p}{q} = a_1 + \frac{1}{a_2 + \frac{1}{a_3 + \cdots + \frac{1}{a_{2m}}}} \]
where \(a_1, a_2, \ldots, a_{2m}\) are integers, and \(a_1\) is positive and the other \(a_i\) are either all positive or all negative. The length of the tuple \( (a_1, a_2, \ldots, a_{2m}) \) is \(2m\), which is always even.

On the other hand, a 2-bridge knot can also be represented by a continued fraction of the form:
\[ \frac{p}{q} = a_1 + \frac{1}{a_2 + \frac{1}{a_3 + \cdots + \frac{1}{a_{2m+1}}}} \]
where \(a_1, a_2, \ldots, a_{2m+1}\) are integers, and \(a_1\) is positive and the other \(a_i\) are either all positive or all negative. The length of the tuple \( (a_1, a_2, \ldots, a_{2m+1}) \) is \(2m+1\), which is always odd.

Therefore, the key difference between the two representations is the parity (odd or even nature) of the length of the tuple. If the length of the tuple is even, it represents a 2-bridge knot in the first form. If the length of the tuple is odd, it represents a 2-bridge knot in the second form.

The answer is:
\[
\boxed{\text{The top representation corresponds to a 2-bridge knot with an even-length tuple, and the bottom representation corresponds to a 2-bridge knot with an odd-length tuple.}}
\]