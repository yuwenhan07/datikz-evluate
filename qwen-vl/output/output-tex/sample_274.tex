To understand the statement "the non realizability of the quotient \( P \setminus e \twoheadrightarrow P / e \) depends on the non realizability of the non-Pappus matroid \( P \) in blue," we need to delve into some concepts from matroid theory and geometric representation.

### Matroids and Geometric Representations

A matroid is a combinatorial structure that abstracts the notion of linear independence in vector spaces. A matroid can be represented geometrically if it can be embedded in a vector space such that its independent sets correspond to the linearly independent subsets of vectors.

### Non-Pappus Matroid

The non-Pappus matroid is a specific matroid that cannot be realized geometrically over the real numbers. It is constructed as an extension of the Fano plane, which is a well-known non-realizable matroid. The non-Pappus matroid is a 7-element matroid with 12 elements in its ground set, and it is non-realizable because it does not satisfy the Pappus configuration.

### Quotient Matroids

Given a matroid \( P \), removing an element \( e \) from \( P \) results in the deletion matroid \( P \setminus e \). Similarly, contracting an element \( e \) results in the contraction matroid \( P / e \).

### Quotient Map

The map \( P \setminus e \twoheadrightarrow P / e \) is a quotient map that identifies all the elements of \( P \) except \( e \) with their corresponding elements in \( P / e \). This map is surjective but not injective.

### Dependence on Non-Realizability

The statement suggests that the non-realizability of the quotient \( P \setminus e \twoheadrightarrow P / e \) is related to the non-realizability of the original matroid \( P \). Specifically, if \( P \) is the non-Pappus matroid, then the quotient \( P \setminus e \twoheadrightarrow P / e \) will also be non-realizable for any element \( e \) in \( P \).

This is because the non-Pappus matroid is a fundamental example of a non-realizable matroid, and any quotient or modification of it will inherit this property. The non-realizability of the quotient is a consequence of the non-realizability of the original matroid.

### Conclusion

In summary, the non-realizability of the quotient \( P \setminus e \twoheadrightarrow P / e \) depends on the non-realizability of the non-Pappus matroid \( P \) because the non-Pappus matroid is a basic example of a non-realizable matroid, and any quotient or modification of it will maintain this property. Therefore, the statement is correct, and the non-realizability of the quotient is indeed dependent on the non-realizability of the non-Pappus matroid.