To understand the construction of a 2-cube induced in \(\mathcal{C}_3(P_3)\) by the given generator and colorings, we need to follow the steps outlined in Lemma~\ref{lem:cubeconstruction}. Let's break it down step-by-step.

### Step 1: Understanding the Generator
The generator is given as \((\set{v_0, v_1}, \set{c_{00}, c_{01}, c_{10}, c_{11}})\). This means:
- \(v_0\) and \(v_1\) are vertices.
- \(c_{00}\), \(c_{01}\), \(c_{10}\), and \(c_{11}\) are colors.

### Step 2: Color Assignments
The color assignments are:
- \(k_{0,0} = 1\)
- \(k_{0,1} = 2\)
- \(k_{1,0} = 1\)
- \(k_{1,1} = 2\)

### Step 3: Constructing the 2-Cube
According to Lemma~\ref{lem:cubeconstruction}, the 2-cube is constructed by considering all possible combinations of the colors assigned to the edges incident to each vertex.

#### Vertex \(v_0\):
- Edges incident to \(v_0\) are colored \(c_{00}\) and \(c_{01}\).
- The color combination for \(v_0\) is \((1, 2)\).

#### Vertex \(v_1\):
- Edges incident to \(v_1\) are colored \(c_{10}\) and \(c_{11}\).
- The color combination for \(v_1\) is \((1, 2)\).

### Step 4: Constructing the Edges
We now construct the edges connecting these vertices based on the color combinations.

#### Edge between \(v_0\) and \(v_1\):
- The edge between \(v_0\) and \(v_1\) will be colored such that it matches the color combination of both vertices.
- Since both \(v_0\) and \(v_1\) have the color combination \((1, 2)\), the edge can be colored either \(c_{00}\) or \(c_{10}\) (or any other combination that matches the color pattern).

### Step 5: Final 2-Cube Structure
The 2-cube induced in \(\mathcal{C}_3(P_3)\) by the given generator and colorings will have:
- Two vertices: \(v_0\) and \(v_1\).
- Two edges: one between \(v_0\) and \(v_1\) colored \(c_{00}\) or \(c_{10}\), and another between \(v_0\) and \(v_1\) colored \(c_{01}\) or \(c_{11}\).

### Conclusion
The 2-cube induced in \(\mathcal{C}_3(P_3)\) by the given generator and colorings is a simple 2-dimensional cube with two vertices and two edges, where the edges are colored according to the color combinations of the vertices. The exact coloring of the edges depends on the specific choice of the edge between \(v_0\) and \(v_1\), but it must match the color pattern of both vertices.