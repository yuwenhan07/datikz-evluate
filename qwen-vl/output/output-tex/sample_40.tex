It seems like you're describing a complex geometric or mechanical configuration involving spherical linkages and mesh elements. Let's break down the key points:

1. **Partial and Whole Spherical Linkage**: This likely refers to a mechanism where some parts of the spherical linkages are connected (whole) while others are not (partial). The mesh in Fig.~\ref{sqmesh} might be part of this mechanism.

2. **Complementary Angles**: The angles \((\lambda_i, \gamma_i, \mu_i, \delta_i, \alpha_i, \beta_i)\) and \((\lambda_i', \gamma_i', \mu_i', \delta_i', \alpha_i', \beta_i')\) are complementary to \(\pi\), meaning they sum up to \(\pi\) radians (or 180 degrees). This could imply that these angles are supplementary in some context, such as in a hinge or joint mechanism.

3. **Gap Between \(\beta_1\) and \(\alpha_2\)**: There is a gap between the angles \(\beta_1\) and \(\alpha_2\). This gap is caused by two factors:
   - \(\tau_1\): This could represent a torsional force or torque applied at a certain point.
   - \(\zeta_1\): This could represent another factor, possibly a misalignment or a specific angle difference causing the gap.

Given the complexity, it would be beneficial to have more context about the specific application or system being described. If you can provide more details or clarify the purpose of the mechanism, I can offer more precise assistance.