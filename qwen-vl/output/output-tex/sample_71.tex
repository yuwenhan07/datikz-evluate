It seems like you're describing a scheduling or resource allocation problem, possibly in the context of online algorithms or dynamic programming. Let's break down the key components of your description:

1. **Job Selection**: At each time step \( t \), you select a job \( j \). This selection gives you a value \( v_j \).

2. **System Busy Time**: Selecting job \( j \) makes the system busy for \( S_j \) time units.

3. **Availability of Jobs**: After the system becomes free again, you can select another job \( j' \) from the set of available jobs and obtain an additional value \( v_{j'} \).

4. **Sequential Selection**: The jobs are selected sequentially, meaning once a job is selected and executed, it is no longer available until the system becomes free again.

This type of problem is often encountered in scenarios such as online scheduling, where decisions must be made without knowing future events. The goal is typically to maximize the total value obtained over time.

### Example Scenario

Let's consider a simplified example with three jobs:
- Job 1: \( v_1 = 5 \), \( S_1 = 2 \)
- Job 2: \( v_2 = 7 \), \( S_2 = 3 \)
- Job 3: \( v_3 = 6 \), \( S_3 = 1 \)

At \( t = 1 \):
- You select Job 1, which gives you \( v_1 = 5 \).
- The system is busy for \( S_1 = 2 \) time units, so it will be free at \( t = 3 \).

At \( t = 3 \):
- The system is free, and you have two jobs available: Job 2 and Job 3.
- You select Job 2, which gives you \( v_2 = 7 \).
- The system is busy for \( S_2 = 3 \) time units, so it will be free at \( t = 6 \).

At \( t = 6 \):
- The system is free, and you have one job available: Job 3.
- You select Job 3, which gives you \( v_3 = 6 \).

The total value obtained is \( 5 + 7 + 6 = 18 \).

### General Strategy

To solve this problem optimally, you might use techniques such as:
- **Online Algorithms**: These algorithms make decisions based on the current state and past history but not future information.
- **Dynamic Programming**: If the problem has some structure that allows it to be broken down into smaller subproblems, dynamic programming could be used.
- **Greedy Algorithms**: Sometimes, a greedy approach (selecting the job that maximizes immediate gain) can work well, especially if the problem has certain properties.

In summary, the described problem is a classic example of an online scheduling problem, where the goal is to maximize the total value obtained by making sequential decisions about which job to execute next. The specific strategy would depend on the constraints and the nature of the jobs.