In a Markov Decision Process (MDP), the basic agent-environment relationship is characterized by a series of interactions between the agent and its environment, which are governed by certain rules and dynamics. Here's a detailed breakdown of the components involved:

1. **Agent**: The agent is the entity that makes decisions and takes actions. It has a set of possible actions \( A_t \) available at each time step \( t \).

2. **Environment**: The environment is the system or setting in which the agent operates. It provides feedback to the agent in the form of rewards and transitions.

3. **State Space (\( S \))**: This is the set of all possible states the environment can be in. At each time step \( t \), the environment is in a specific state \( S_t \in S \).

4. **Action Space (\( A \))**: This is the set of all possible actions the agent can take. At each time step \( t \), the agent selects an action \( A_t \in A \).

5. **Reward Function (\( R \))**: This function maps the current state \( S_t \) and the chosen action \( A_t \) to a scalar value \( R_{t+1} \), which represents the immediate reward received by the agent after taking action \( A_t \) in state \( S_t \). The reward function typically depends on the state-action pair and possibly other factors like the next state.

6. **Transition Dynamics**: The environment transitions from one state to another based on the action taken by the agent. The transition probability \( P(S_{t+1} | S_t, A_t) \) describes the probability of moving from state \( S_t \) to state \( S_{t+1} \) when action \( A_t \) is taken. The transition dynamics can be deterministic or stochastic.

7. **Policy (\( \pi \))**: This is a mapping from states to actions, representing the strategy the agent uses to choose actions. A policy determines the probability distribution over actions given a state. For example, a deterministic policy might always choose the same action for a given state, while a stochastic policy might choose different actions with certain probabilities.

The interaction between the agent and the environment in an MDP can be summarized as follows:

- At time step \( t \), the agent observes the current state \( S_t \).
- The agent chooses an action \( A_t \) according to its policy.
- The environment transitions to a new state \( S_{t+1} \) based on the action \( A_t \) and the transition dynamics.
- The environment also provides a reward \( R_{t+1} \) to the agent for taking action \( A_t \) in state \( S_t \).

This sequence of events repeats at each time step until the task is completed or a termination condition is met. The goal of the agent is often to maximize the cumulative reward over time, which involves learning an optimal policy through interaction with the environment.

The dotted line in your figure represents the transition from time step \( t \) to time step \( t+1 \), illustrating the progression of the agent-environment interaction in the MDP framework.