To determine the graph \( G_{\mathcal{F}} \) obtained from the formula \(\mathcal{F} = (X_1 \vee X_3) \wedge (X_2 \vee X_3 \vee X_5) \wedge (X_4 \vee X_5)\), we need to understand how the formula translates into a graph. The formula is a conjunction of three clauses, and each clause represents a disjunction of literals. In a graph, this corresponds to a set of edges where each edge connects two variables that appear together in a clause.

Let's break down the formula:

1. The first clause is \(X_1 \vee X_3\). This means that if either \(X_1\) or \(X_3\) is true, then the clause is satisfied. In terms of a graph, this means there is an edge between \(X_1\) and \(X_3\).
2. The second clause is \(X_2 \vee X_3 \vee X_5\). This means that if any one of \(X_2\), \(X_3\), or \(X_5\) is true, then the clause is satisfied. In terms of a graph, this means there is an edge between \(X_2\) and \(X_3\), an edge between \(X_2\) and \(X_5\), and an edge between \(X_3\) and \(X_5\).
3. The third clause is \(X_4 \vee X_5\). This means that if either \(X_4\) or \(X_5\) is true, then the clause is satisfied. In terms of a graph, this means there is an edge between \(X_4\) and \(X_5\).

Combining all these edges, the graph \(G_{\mathcal{F}}\) will have the following edges:
- An edge between \(X_1\) and \(X_3\)
- An edge between \(X_2\) and \(X_3\)
- An edge between \(X_2\) and \(X_5\)
- An edge between \(X_3\) and \(X_5\)
- An edge between \(X_4\) and \(X_5\)

Thus, the graph \(G_{\mathcal{F}}\) is a complete bipartite graph \(K_{2,3}\) with vertices partitioned into two sets: \(\{X_1, X_2, X_4\}\) and \(\{X_3, X_5\}\). The edges are as follows:
- Between \(\{X_1, X_2, X_4\}\) and \(\{X_3, X_5\}\)

So, the graph \(G_{\mathcal{F}}\) can be represented as:
\[ G_{\mathcal{F}} = K_{2,3} \]

The final answer is:
\[
\boxed{K_{2,3}}
\]