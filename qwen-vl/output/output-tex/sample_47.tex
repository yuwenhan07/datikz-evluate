To understand the plot and the information it conveys, let's break down the components step by step.

1. **Understanding the Axes:**
   - The \( x \)-axis represents the approximation ratio. This is a measure of how close the solution produced by the algorithm is to the optimal solution. For example, if the approximation ratio is 2, then the solution found by the algorithm is at most twice as large as the optimal solution.
   - The \( y \)-axis represents the base of the exponent in the running time. This means that the running time of the algorithm can be expressed in the form \( d^k \cdot n^{\Oh(1)} \), where \( d \) is the value on the \( y \)-axis, \( k \) is some constant (which may depend on the approximation ratio), and \( n \) is the size of the input.

2. **Interpreting the Points:**
   - Each point \((\beta, d)\) on the plot corresponds to a specific approximation ratio \(\beta\) and a specific base \( d \). This means that for an approximation ratio of \(\beta\), the running time of the algorithm is of the form \( d^k \cdot n^{\Oh(1)} \).
   - The value of \( k \) (the exponent in the polynomial part of the running time) is not explicitly given on the plot but is implied by the context of the problem. It typically depends on the details of the algorithm and the approximation ratio.

3. **Example Interpretation:**
   - Suppose we have a point \((2, 3)\) on the plot. This means that for an approximation ratio of 2, the running time of the algorithm is of the form \( 3^k \cdot n^{\Oh(1)} \). The exact value of \( k \) would need to be determined from the specific details of the algorithm or the problem being solved.

In summary, the plot provides a visual representation of how the running time of the algorithm changes with different approximation ratios. Each point \((\beta, d)\) indicates that for an approximation ratio of \(\beta\), the running time is of the form \( d^k \cdot n^{\Oh(1)} \), where \( k \) is a constant that depends on the specifics of the algorithm and the problem.

The final answer is:

\[
\boxed{\text{Each point } (\beta, d) \text{ on the plot describes a running time of the form } d^k \cdot n^{\Oh(1)} \text{ for a } \beta\text{-approximation, where } k \text{ is a constant.}}
\]