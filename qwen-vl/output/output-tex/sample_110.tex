The description you've provided refers to a specific type of stochastic process known as a "sticky Brownian motion" or "sticky diffusion," which is a variant of the standard Brownian motion. This process is particularly interesting because it exhibits behavior where particles can get "stuck" at certain points, leading to jumps to other points after a period of time spent at the sticking point.

Let's break down the key components of this process:

1. **Feller Process**: A Feller process is a continuous-time Markov process with the property that the transition probabilities from any state to any other state are continuous in the initial and final states. In simpler terms, it's a process where the probability of moving from one state to another depends only on the current state and not on the history of the process.

2. **Sticky Brownian Motion**: This is a type of diffusion process where particles can get "stuck" at certain points (the sticking points). Once stuck, they remain there for a random amount of time before jumping to another point. The sticking points are typically absorbing states, meaning once the process reaches one of these points, it stays there indefinitely.

3. **Copies of \([0, \infty]\)**: The process is defined on multiple copies of the interval \([0, \infty]\). Here, \(\ka = 3\) indicates that there are three such intervals, each representing a different copy of the process.

4. **Stickiness Coefficient \(a_i / b_i\)**: For each copy \(i\), the stickiness coefficient is given by \(a_i / b_i\). This coefficient determines how likely the process is to stay at the sticking point \((0, i)\) compared to jumping to another copy. If \(a_i / b_i\) is large, the process is more likely to stay at \((0, i)\).

5. **Jumps**: After spending enough time at \((0, i)\), the process jumps to one of the points \((0, j)\) where \(j \neq i\). The choice of the next copy to jump to is determined by the parameters \(c_i\), which govern the times between jumps.

6. **Times Between Jumps**: The times between jumps are governed by the parameters \(c_i\). These parameters determine the distribution of the waiting times before the next jump occurs. Typically, these times are exponentially distributed with mean \(1/c_i\).

In summary, the process described is a Feller process on three copies of the interval \([0, \infty]\), where each copy behaves like a one-dimensional sticky Brownian motion with a specific stickiness coefficient. The process can jump to another copy after spending enough time at the sticking point \((0, i)\), and the times between these jumps are governed by the parameters \(c_i\). This type of process is often used in modeling phenomena where particles can get "stuck" at certain points and then move to other points after a period of time.