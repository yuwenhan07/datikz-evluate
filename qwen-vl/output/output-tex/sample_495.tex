To solve the problem of finding the invariant probability measure on the Fibonacci shift with respect to the language \( L = \{w \in \{a, b\}^* \mid |w|_a \equiv 0 \mod 2\} \), we need to understand the structure of the Fibonacci shift and the properties of the language \( L \).

### Step-by-Step Solution:

1. **Understanding the Fibonacci Shift:**
   The Fibonacci shift is a one-sided shift space over the alphabet \(\{a, b\}\) where the transition rule is defined by the Fibonacci sequence. Specifically, if \( w \) is a word in the Fibonacci shift, then the next word in the sequence is obtained by appending the last letter of \( w \) to the word formed by removing the first letter of \( w \).

2. **Language \( L \):**
   The language \( L \) consists of all words in \(\{a, b\}^*\) such that the number of \( a \)'s in the word is even. This means that any word in \( L \) can be written as \( a^{2k}b^n \) or \( b^{2k}a^n \) for some non-negative integers \( k \) and \( n \).

3. **Invariant Measure:**
   An invariant measure on a shift space is a probability measure that is preserved under the shift operator. For the Fibonacci shift, the invariant measure is often related to the golden ratio \(\lambda = \frac{1 + \sqrt{5}}{2}\), which is the largest eigenvalue of the adjacency matrix of the Fibonacci shift.

4. **Circled Nodes:**
   The circled nodes in the diagram represent elements from the language \( L \). These nodes correspond to words in the Fibonacci shift where the number of \( a \)'s is even.

5. **Constructing the Invariant Measure:**
   The invariant measure on the Fibonacci shift can be constructed using the Perron-Frobenius theorem for the adjacency matrix of the shift. The measure is typically given by the eigenvector corresponding to the eigenvalue \(\lambda\) (the golden ratio).

6. **Measure on \( L \):**
   Since the circled nodes represent words in \( L \), the invariant measure on these nodes will be proportional to the measure of the entire shift space restricted to \( L \). The measure of \( L \) in the context of the Fibonacci shift is known to be \(\frac{\lambda - 1}{\lambda}\).

Therefore, the invariant probability measure on the Fibonacci shift with respect to the language \( L \) is given by:

\[
\boxed{\frac{\lambda - 1}{\lambda}}
\]

where \(\lambda = \frac{1 + \sqrt{5}}{2}\) is the golden ratio.