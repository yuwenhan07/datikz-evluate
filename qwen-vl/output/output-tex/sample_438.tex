To understand the Plücker coordinate \(\Delta_{I(r)}\) for a rectangle \(r\) of size \(k \times (n-k)\) within a larger rectangle \(R\) of size \(k \times n\), we need to follow these steps:

1. **Identify the Rectangles**: The larger rectangle \(R\) has dimensions \(k \times n\). The smaller rectangle \(r\) has dimensions \(k \times (n-k)\) and is positioned in the upper left corner of \(R\).

2. **Define the Path**: The Plücker coordinate \(\Delta_{I(r)}\) is defined as the vertical steps in the path from the upper right corner to the lower left corner of the rectangle \(R\). This path can be visualized as moving from the top-right corner of \(R\) to the bottom-left corner of \(R\).

3. **Count the Vertical Steps**: To find the number of vertical steps, we need to count how many times we move down from one row to the next as we move from the top-right corner to the bottom-left corner of \(R\). Since the smaller rectangle \(r\) is of size \(k \times (n-k)\), it occupies \(k\) rows and \(n-k\) columns starting from the top-left corner of \(R\). Therefore, when we move from the top-right corner of \(R\) to the bottom-left corner of \(R\), we will pass through \(k-1\) rows below the top row of \(r\). This means there are \(k-1\) vertical steps.

4. **Conclusion**: The number of vertical steps in the path from the upper right corner to the lower left corner of the rectangle \(R\) is \(k-1\). Therefore, the Plücker coordinate \(\Delta_{I(r)}\) is \(k-1\).

The final answer is:
\[
\boxed{k-1}
\]