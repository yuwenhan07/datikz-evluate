The online phase of the neural viscosity approach involves using a neural network to estimate the viscosity value for each element of the mesh independently. This process is typically part of a larger finite element method (FEM) simulation where the goal is to solve partial differential equations (PDEs). Below is a detailed explanation of the scheme:

### Scheme Overview

1. **Mesh and Degrees of Freedom**: Consider a mesh divided into elements \( K \), and let \( U_i^n \) represent the value of the unknown function \( u_h \) at the \( i \)-th degree of freedom within element \( K \) at time step \( t^n \).

2. **Neural Network Input**: The neural network takes as input the nodal values \( U_i^n \) for all degrees of freedom within element \( K \). These values are used to compute relevant statistics.

3. **Statistics Extraction**: The neural network extracts four key statistics from these nodal values:
   - Minimum (\( \min(U_i^n) \))
   - Maximum (\( \max(U_i^n) \))
   - Mean (\( \mu_U \))
   - Standard Deviation (\( \sigma_U \))

4. **Viscosity Estimation**: Based on these statistics, the neural network outputs a viscosity value \( \mu_K^{\text{bnd}} \) for element \( K \). This value is specific to that element and does not depend on the viscosities of other elements.

5. **Application in FEM**: In the context of FEM, this viscosity value is then used to modify the stiffness matrix or the right-hand side of the system of equations, effectively incorporating the estimated viscosity into the solution process.

### Mathematical Representation

Let's denote the nodal values within element \( K \) at time \( t^n \) as \( U = [U_1^n, U_2^n, \ldots, U_m^n] \), where \( m \) is the number of degrees of freedom in element \( K \).

The statistics extracted by the neural network can be represented as:
\[
\text{stats}(U) = [\min(U), \max(U), \mu_U, \sigma_U]
\]

The neural network then maps these statistics to the viscosity value:
\[
\mu_K^{\text{bnd}} = f(\text{stats}(U))
\]
where \( f \) is the mapping function learned by the neural network.

### Example Calculation

For simplicity, consider a linear element with two degrees of freedom \( U_1^n \) and \( U_2^n \). The statistics would be:
\[
\text{stats}(U) = [\min(U_1^n, U_2^n), \max(U_1^n, U_2^n), \frac{U_1^n + U_2^n}{2}, \sqrt{\frac{(U_1^n - U_2^n)^2}{2}}]
\]

The neural network would then map these statistics to a single viscosity value \( \mu_K^{\text{bnd}} \).

### Summary

In summary, the online phase of the neural viscosity approach involves:
1. Extracting relevant statistics from the nodal values within each element.
2. Using these statistics as input to a neural network to estimate an element-wise viscosity value.
3. Incorporating this viscosity value into the FEM solution process to account for local variations in material properties.

This approach allows for more accurate modeling of complex materials with varying properties across different regions of the domain.