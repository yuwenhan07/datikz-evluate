To understand the visualization and the properties of \( B^*_k \) with center 2 and \( n = 5 \), let's break it down step by step.

### Step 1: Understanding the Visualization
The visualization shows a grid where each cell represents a position in the matrix \( B^*_k \). The light red cells are shadowed from above, meaning they are not visible because they are covered by other cells. The numbers in these shadowed cells indicate which layer (or period) has a \(\mathcal{P}\) position at that location.

### Step 2: Identifying Shadowed Cells
The second row and second column are also shadowed, as indicated by the red segments. This means that any cell in the second row or second column will have its value obscured by the cells above or to the left of it.

### Step 3: Determining the Number of Rows and Columns
Since we are dealing with \( n = 5 \), the matrix \( B^*_k \) will have 5 rows and 5 columns. However, due to the shadowing, only the first 5 rows or columns can be shadowed.

### Step 4: Periods and LCM
The problem states that the least common multiple (LCM) of the periods of 0 to 4 with center 2 is 12. This means that every 12th row or column of the matrix \( B^*_k \) will start over with the same pattern but shifted by one position.

### Step 5: Starting Point and Shift
Given that \( B^*_{k+1} \) begins 12 rows down from \( B^*_k \), this indicates a periodicity of 12 in the matrix. The shift by 12 rows means that the pattern repeats every 12 rows, but starting from a different position.

### Conclusion
The visualization and the properties described provide a clear understanding of how the matrix \( B^*_k \) is structured with respect to the center 2 and the period 5. The shadowing and the periodicity of 12 help in identifying the repeating patterns within the matrix.

Thus, the final answer is:

\[
\boxed{12}
\]