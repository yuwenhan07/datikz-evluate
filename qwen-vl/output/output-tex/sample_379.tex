To understand the induced \( P_3 \) in \( \mathcal{C}_3(P_3) \) generated by \( (\{v_1, v_3\}, C) \), where \( C = \{c_1, c_2, c_3\} \), we need to follow these steps:

1. **Define the Graphs and Notations:**
   - \( P_3 \) is a path graph with 3 vertices, typically denoted as \( P_3 = (V, E) \) where \( V = \{v_1, v_2, v_3\} \) and \( E = \{(v_1, v_2), (v_2, v_3)\} \).
   - \( \mathcal{C}_3(P_3) \) is the set of all 3-cycles in \( P_3 \). Since \( P_3 \) does not contain any cycles of length 3, \( \mathcal{C}_3(P_3) = \emptyset \).

2. **Induced Subgraph Concept:**
   - An induced subgraph of a graph \( G \) is a subgraph whose vertex set is a subset of the vertex set of \( G \) and whose edge set consists of all edges of \( G \) that have both endpoints in this subset.
   - Given a set of vertices \( S \subseteq V \) and a set of edges \( T \subseteq E \), the induced subgraph on \( S \) is the subgraph with vertex set \( S \) and edge set \( \{e \in E : e \subseteq S\} \).

3. **Identify the Induced \( P_3 \):**
   - The problem asks for the induced \( P_3 \) in \( \mathcal{C}_3(P_3) \) generated by \( (\{v_1, v_3\}, C) \).
   - Since \( \mathcal{C}_3(P_3) = \emptyset \), there are no 3-cycles in \( P_3 \). Therefore, the induced \( P_3 \) cannot be formed from any 3-cycle.

4. **Conclusion:**
   - Because \( \mathcal{C}_3(P_3) \) is empty, there is no induced \( P_3 \) in \( \mathcal{C}_3(P_3) \) that can be generated by any set of vertices and edges.

Thus, the induced \( P_3 \) in \( \mathcal{C}_3(P_3) \) generated by \( (\{v_1, v_3\}, C) \) does not exist because there are no 3-cycles in \( P_3 \).

\[
\boxed{\text{No induced } P_3 \text{ exists}}
\]