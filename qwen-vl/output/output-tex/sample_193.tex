The image you've described seems to be discussing the concept of a flexible quadrilateral mesh, often referred to as a Kokotsakis polyhedron, particularly when it has a quadrangular base. This type of structure is interesting because it can be deformed while maintaining certain geometric properties.

### Left: Sketch of a \(3 \times 3\) Quadrilateral Mesh

This sketch represents a \(3 \times 3\) quadrilateral mesh, which is essentially a grid of quadrilaterals. In this context, the flexibility of the mesh is limited to a smaller area marked in dashed lines. This suggests that the flexibility of the mesh is constrained to a specific region within the larger grid, implying that the rest of the mesh remains rigid under deformation.

### Right: Flexible Mesh with Fixed Angles

The right part of the image describes a flexible mesh where the angles \(\lambda_i'\), \(\gamma_i'\), \(\mu_i'\), and \(\delta_i'\) are fixed. These angles are determined by vector products and the arccosine function, indicating that they are precisely defined. The flexible angles \(\alpha_i'\) and their complements \(\alpha_i = \pi - \alpha_i'\) represent the dihedral angles between the central planar panel and the surrounding planar panels. These angles are rigorously defined in Appendix \ref{dfnang}, suggesting that there is a detailed mathematical framework for understanding these angles.

### Summary

- **Left**: A \(3 \times 3\) quadrilateral mesh with flexibility constrained to a smaller area.
- **Right**: A flexible mesh with fixed angles \(\lambda_i'\), \(\gamma_i'\), \(\mu_i'\), and \(\delta_i'\), and flexible angles \(\alpha_i'\) and \(\alpha_i\).

The flexibility of the mesh is thus tied to the ability to change the dihedral angles \(\alpha_i'\) while keeping the other angles fixed, allowing the mesh to deform in a controlled manner. The mathematical definitions provided in Appendix \ref{dfnang} ensure that these angles are well-defined and can be used to analyze the behavior of the mesh under different conditions.