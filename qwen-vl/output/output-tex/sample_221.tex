To understand the problem, we need to break it down into several components:

1. **Understanding the Set \(\partial \mathcal{K}\)**: This typically refers to the boundary of a set \(\mathcal{K}\). In this context, \(\mathcal{K}\) is likely a specific subset of the complex plane or a region defined by some geometric or algebraic constraints.

2. **Intersection with \(\Delta_{1,0,-1/5}\)**: The notation \(\Delta_{1,0,-1/5}\) suggests that we are considering a specific triangle in the complex plane. The vertices of this triangle are at \(1\), \(0\), and \(-\frac{1}{5}\).

3. **Imaginary Parts**: We are interested in the imaginary parts of the points on the boundary of \(\mathcal{K}\) that lie within the triangle \(\Delta_{1,0,-1/5}\).

4. **Base \(-4\)**: This indicates that we are working in a base \(-4\) numeral system. In base \(-4\), each digit can be one of \(\{0, 1, 2, 3\}\), and the place values alternate between positive and negative powers of \(-4\).

Given these components, let's construct an automaton that recognizes the imaginary parts of points in \(\partial \mathcal{K} \cap \Delta_{1,0,-1/5}\) in base \(-4\).

### Step-by-Step Construction

1. **Define the Set \(\Delta_{1,0,-1/5}\)**:
   - The vertices of the triangle are \(1\), \(0\), and \(-\frac{1}{5}\).
   - The sides of the triangle are the line segments connecting these points.

2. **Identify Points on the Boundary \(\partial \mathcal{K}\)**:
   - We need to determine which points on the boundary of \(\mathcal{K}\) lie within the triangle \(\Delta_{1,0,-1/5}\).

3. **Extract Imaginary Parts**:
   - For each point on the boundary of \(\mathcal{K}\) that lies within the triangle, extract its imaginary part.

4. **Convert Imaginary Parts to Base \(-4\)**:
   - Convert the imaginary parts from their decimal representation to their base \(-4\) representation.
   - Each digit in the base \(-4\) representation will be one of \(\{0, 1, 2, 3\}\).

### Automaton Design

An automaton for this task would have the following components:

- **States**: Represent the different states of the automaton as it processes the imaginary part of a point.
- **Alphabet**: The input alphabet consists of the digits \(\{0, 1, 2, 3\}\) (the possible digits in base \(-4\)).
- **Transition Function**: Define how the automaton transitions from one state to another based on the current state and the input digit.
- **Start State**: The initial state where the automaton begins processing.
- **Accepting States**: The final states where the automaton accepts the input sequence.

### Example Transition Table

Let's assume we have a simple automaton with the following states and transitions:

- **State 0**: Initial state.
- **State 1**: Represents the first digit of the base \(-4\) representation.
- **State 2**: Represents the second digit of the base \(-4\) representation.
- **State 3**: Represents the third digit of the base \(-4\) representation.

The transition table might look like this:

| Current State | Input Digit | Next State |
|---------------|-------------|------------|
| 0             | 0           | 1          |
| 0             | 1           | 1          |
| 0             | 2           | 1          |
| 0             | 3           | 1          |
| 1             | 0           | 2          |
| 1             | 1           | 2          |
| 1             | 2           | 2          |
| 1             | 3           | 2          |
| 2             | 0           | 3          |
| 2             | 1           | 3          |
| 2             | 2           | 3          |
| 2             | 3           | 3          |

This table assumes that the automaton processes the digits sequentially and moves to the next state upon reading each digit.

### Conclusion

The automaton described above would recognize the base \(-4\) representations of the imaginary parts of points on the boundary of \(\mathcal{K}\) that lie within the triangle \(\Delta_{1,0,-1/5}\). The exact implementation would depend on the specific details of the set \(\mathcal{K}\) and the boundary conditions, but the general structure of the automaton follows the steps outlined above.

Thus, the final answer is:

\[
\boxed{\text{An automaton that processes the imaginary parts of points on the boundary of } \mathcal{K} \text{ within the triangle } \Delta_{1,0,-1/5} \text{ in base } -4.}
\]