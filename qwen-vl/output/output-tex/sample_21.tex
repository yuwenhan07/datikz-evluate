In the context of fluid dynamics and partial differential equations, particularly when dealing with the Navier-Stokes equations or similar models, cutoff functions (also known as mollifiers) are often used to regularize solutions and facilitate the analysis of their properties. These cutoff functions are smooth and compactly supported, meaning they are infinitely differentiable and vanish outside a bounded region.

In the proof of Proposition \(\ref{prop:L2-Linfty-A}\), which likely pertains to some aspect of the linearized Navier-Stokes equations or a related system, the cutoff functions \(\eta\) and \(\tilde{\eta}\) play a crucial role in the following ways:

1. **Regularization**: The cutoff functions help in smoothing out the solution, making it easier to work with in the context of Sobolev spaces or other function spaces where the original solution might not be smooth enough.

2. **Compact Support**: Since these functions have compact support, they can be used to localize the analysis to specific regions of interest without affecting the global behavior of the solution too much.

3. **Smoothness**: Being infinitely differentiable, these functions allow for the application of various calculus operations, such as differentiation, which is essential in proving estimates and inequalities.

4. **Integration by Parts**: The smoothness and compact support of these functions make it possible to use integration by parts in a controlled manner, which is necessary for deriving energy estimates and other integral inequalities.

5. **Approximation**: In many proofs, these cutoff functions are used to approximate the original solution in a way that preserves key properties while making the analysis more tractable.

Given that \(\eta, \tilde{\eta} \in C_c^\infty(\mathbb{R}^d)\), we know that:
- \(C_c^\infty(\mathbb{R}^d)\) denotes the space of infinitely differentiable functions on \(\mathbb{R}^d\) that are zero outside a compact set.
- These functions are typically constructed using convolution with a Gaussian kernel or other smooth, rapidly decaying functions.

In summary, the cutoff functions \(\eta\) and \(\tilde{\eta}\) are smooth, compactly supported, and infinitely differentiable, and they are used to regularize and analyze the solution of the system described in Proposition \(\ref{prop:L2-Linfty-A}\). Their properties allow for the application of various mathematical techniques to derive the desired results.