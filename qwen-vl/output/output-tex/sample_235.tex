To address your question about the Penrose diagram and the implications for particle production in the context of the CPT-conjugated TCC-violating expanding solution, let's break down the key components:

1. **Penrose Diagram**: A Penrose diagram is a spacetime diagram that visualizes the causal structure of a spacetime manifold. It is particularly useful for understanding the causal relationships between events in curved spacetime, such as black holes and cosmological singularities.

2. **CPT Conjugate**: The CPT (Charge, Parity, Time) conjugate of a physical system is obtained by applying the CPT transformation to the system. This transformation involves changing the sign of all spatial coordinates, the sign of time, and the charge of any particles involved. In the context of cosmology, this can help us understand the behavior of the universe under different symmetry transformations.

3. **TCC-Violating Expanding Solution**: TCC stands for Time, Charge, and Conjugation. A TCC-violating solution refers to a spacetime geometry where these symmetries are not preserved. Such solutions often describe scenarios where the universe deviates from standard cosmological models, potentially leading to new physics or exotic phenomena.

4. **Phase Transitions**: In cosmology, phase transitions refer to sudden changes in the properties of the universe, such as the transition from one phase of matter to another. These transitions can have significant effects on the evolution of the universe and the distribution of energy and matter.

5. **Particle Production**: Particle production occurs when quantum fluctuations in the vacuum lead to the creation of particle-antiparticle pairs. This process is particularly important in the early universe, where it can play a role in the formation of cosmic structures and the distribution of dark matter.

6. **Blue-Shifting**: Blue-shifting refers to the increase in frequency of light or other electromagnetic radiation as it moves away from a source. In the context of cosmology, blue-shifting can occur due to the expansion of the universe, which stretches the wavelength of light and increases its frequency.

Given these points, let's analyze the scenario you described:

- **Starting with the Past Vacuum State**: If we start with the past vacuum state, we are considering the initial conditions of the universe before any significant cosmological processes have occurred.
  
- **Transition from Phase II to Phase I**: The transition from Phase II to Phase I could represent a significant change in the properties of the universe, such as a phase transition in the early universe. During this transition, quantum fluctuations might become more pronounced, leading to particle production.

- **Particle Production**: At the moment of transition from Phase II to Phase I, particle production would occur. This is consistent with the idea that quantum fluctuations in the vacuum can lead to the creation of particle-antiparticle pairs.

- **Blue-Shifting and Trans-Planckian Energies**: Under certain assumptions, Phase I might last long enough to cause a significant blue-shifting of the particles produced during the transition. This blue-shifting could lead to the particles reaching trans-Planckian energies, which are energies beyond the Planck scale. At these energies, the effective field theory (EFT) breaks down, and new physics must be considered.

In summary, the Penrose diagram of the CPT-conjugated TCC-violating expanding solution would show the causal structure of the spacetime, including the transition from Phase II to Phase I. During this transition, particle production would occur, leading to the creation of particles that could be blue-shifted to trans-Planckian energies. This scenario highlights the importance of considering phase transitions and particle production in the context of cosmological models, especially when deviations from standard cosmological models are present.