The scenario you described involves a process that is common in belief propagation algorithms, particularly in graphical models like Bayesian networks or factor graphs. Let's break down the steps and clarify the process:

### Step-by-Step Explanation

1. **Graph Representation**:
   - We have a directed graph where nodes represent variables, and edges represent dependencies.
   - The nodes \( v_1 \), \( v_2 \), and \( v_3 \) are connected such that \( v_3 \) has parents \( v_1 \) and \( v_2 \).

2. **Sign Assignment**:
   - Each variable can be assigned a sign, either positive (\( + \)) or negative (\( - \)).
   - In this case, both \( v_1 \) and \( v_2 \) are assigned the sign \( + \).

3. **Message Passing**:
   - Messages are passed between nodes based on their signs.
   - For each parent of \( v_3 \), a message is sent to \( v_3 \). Here, we have two messages:
     - A message from \( v_1 \) with sign \( + \).
     - A message from \( v_2 \) with sign \( + \).

4. **Combining Messages**:
   - The messages are combined using a sign addition operator, denoted by \( \oplus \).
   - The sign addition operator for two signs \( + \) and \( - \) is defined as follows:
     - \( + \oplus + = + \)
     - \( + \oplus - = - \)
     - \( - \oplus + = - \)
     - \( - \oplus - = + \)

5. **Combining the Messages**:
   - The messages from \( v_1 \) and \( v_2 \) are combined:
     - \( + \oplus + = + \)
     - \( + \oplus - = - \)

6. **Final Sign for \( v_3 \)**:
   - The final sign for \( v_3 \) is determined by combining the results of the two messages:
     - \( + \oplus - = ? \)

In this specific case, the combination of \( + \) and \( - \) results in an ambiguous or undefined sign, often represented as \( ? \). This could indicate that there is no clear consensus among the messages, or it might require additional information or constraints to resolve the ambiguity.

### Conclusion

The final sign for \( v_3 \) after combining the messages from \( v_1 \) and \( v_2 \) is \( ? \). This indicates that the sign for \( v_3 \) cannot be definitively determined from the given messages alone, and further information or constraints may be needed to resolve the ambiguity.

This process is a simplified example of how belief propagation works in graphical models, where messages are passed between nodes to update the beliefs about the variables. The sign addition operator helps in combining these messages to determine the updated state of the variables.