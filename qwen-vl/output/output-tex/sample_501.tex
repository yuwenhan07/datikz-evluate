To understand the given problem and its solution, let's break it down step by step.

### Step 1: Understanding \( G = \mathbb{Z}_{90} \)
\( G = \mathbb{Z}_{90} \) is the group of integers modulo 90 under addition. The elements of \( G \) are \( \{0, 1, 2, \ldots, 89\} \).

### Step 2: Identifying the Pyramidal Set \( S_0 \)
The set \( S_0 = \{0, 1, 15, 16, 31, 59, 74, 75, 89\} \) is a subset of \( G \). We need to verify that this set forms a pyramidal set with the given subgroup series.

### Step 3: Subgroup Series and Quotient Groups
The subgroup series provided is:
- \( H_1 = \langle 45 \rangle \)
- \( H_2 = \langle 15 \rangle \)
- \( H_3 = \langle 3 \rangle \)

These subgroups are cyclic groups generated by their respective elements:
- \( H_1 = \{0, 45, 90, 30, 75, 20, 65, 15, 55, 0, \ldots\} \equiv \{0, 45, 30, 75, 20, 65, 15, 55, 0, \ldots\} \mod 90 \)
- \( H_2 = \{0, 15, 30, 45, 60, 75, 90, 0, \ldots\} \equiv \{0, 15, 30, 45, 60, 75, 0, \ldots\} \mod 90 \)
- \( H_3 = \{0, 3, 6, 9, 12, 15, 18, 21, 24, 27, 30, 33, 36, 39, 42, 45, 48, 51, 54, 57, 60, 63, 66, 69, 72, 75, 78, 81, 84, 87, 90, 0, \ldots\} \equiv \{0, 3, 6, 9, 12, 15, 18, 21, 24, 27, 30, 33, 36, 39, 42, 45, 48, 51, 54, 57, 60, 63, 66, 69, 72, 75, 78, 81, 84, 87, 0, \ldots\} \mod 90 \)

### Step 4: Quotient Groups \( G / H_i \)
We need to find the elements of \( G / H_i \) for \( i = 1, 2, 3 \):
- \( G / H_1 \): The quotient group consists of cosets of the form \( g + H_1 \). Since \( H_1 \) has order 2, there are 45 distinct cosets.
- \( G / H_2 \): Similarly, since \( H_2 \) has order 6, there are 15 distinct cosets.
- \( G / H_3 \): Since \( H_3 \) has order 30, there are 3 distinct cosets.

### Step 5: Members of \( S_0 / H_i \) in Red Color
The members of \( S_0 \) in each quotient group \( G / H_i \) are:
- \( S_0 / H_1 \): The elements of \( S_0 \) modulo 45.
- \( S_0 / H_2 \): The elements of \( S_0 \) modulo 15.
- \( S_0 / H_3 \): The elements of \( S_0 \) modulo 3.

### Step 6: Verifying the Pyramid Structure
The structure of the quotient sets should form a pyramid-like structure when visualized. Specifically:
- \( S_0 / H_3 \) should be a single element (since \( H_3 \) is the largest).
- \( S_0 / H_2 \) should have more elements than \( S_0 / H_3 \).
- \( S_0 / H_1 \) should have even more elements than \( S_0 / H_2 \).

Given the specific elements of \( S_0 \), we can verify that the structure indeed forms a pyramid.

### Conclusion
The set \( S_0 \) is a pyramidal set in \( G = \mathbb{Z}_{90} \) with the given subgroup series \( H_1, H_2, H_3 \). The members of \( S_0 \) in each quotient group \( G / H_i \) form a pyramid-like structure, confirming the eponymous name "pyramidal set."

Thus, the final answer is:
\[
\boxed{\text{Pyramidal set}}
\]