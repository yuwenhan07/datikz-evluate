To understand the Penrose diagram of a Friedmann-Robertson-Walker (FRW) solution with negative spatial curvature that transitions from a TCC-violating phase to a non-accelerating phase, let's break down the phases and their characteristics:

### Phase I: TCC-Violating Phase
In this phase, the universe exhibits accelerating epochs where the scale factor \(a(t)\) is increasing at an accelerating rate. This phase violates the cosmic censorship conjecture (TCC), which suggests that singularities should be hidden behind event horizons.

### Phase II: Non-Accelerating Phase
Phase II is divided into two subphases:
1. **Phase II-A**: 
   - The second derivative of the scale factor \(\ddot{a}\) is less than zero (\(\ddot{a} < 0\)).
   - The spatial curvature is negligible.
   - The expansion of the universe is entirely driven by the scalar field.

2. **Phase II-B**:
   - The second derivative of the scale factor \(\ddot{a}\) becomes negligible (\(\ddot{a} \approx 0\)).
   - The spatial curvature becomes important.
   - The spatial curvature parameter \(\Omega_k\) converges to a non-zero value.

### Penrose Diagram
A Penrose diagram is a conformal representation of spacetime that shows the causal structure of the universe. It is particularly useful for visualizing the causal relationships between different regions of spacetime, including the past and future directions.

#### Key Features in the Penrose Diagram:
1. **Conformal Infinity (\(\mathscr{I}^{\pm}\))**: These are the boundaries of the conformal diagram representing the future and past null infinities, respectively.
2. **Event Horizons**: These are the boundaries beyond which light cannot escape the gravitational pull of the singularity or black hole.
3. **Singularity**: This is the point where the curvature of spacetime becomes infinite, typically located at the center of a black hole or at the origin of the FRW metric.
4. **Penrose Diagram for FRW with Negative Spatial Curvature**:
   - The Penrose diagram will show the expansion of the universe as it moves away from the singularity.
   - The transition from Phase I to Phase II will be depicted as a change in the slope of the expansion curve.
   - In Phase II-A, the expansion will be decelerating but still showing some curvature effects due to the scalar field.
   - In Phase II-B, the expansion will be nearly flat, indicating a non-accelerating phase, and the spatial curvature will become significant.

### Visualization in the Penrose Diagram:
- **Phase I**: The diagram will show a region where the expansion is accelerating, leading to a singularity or a black hole.
- **Transition Region**: This is the area where the universe transitions from the accelerating phase to the decelerating phase.
- **Phase II-A**: The expansion will be decelerating, but the spatial curvature will be negligible.
- **Phase II-B**: The expansion will be nearly flat, and the spatial curvature will be significant, leading to a non-zero \(\Omega_k\).

### Conclusion
The Penrose diagram provides a clear visualization of the causal structure of the FRW solution with negative spatial curvature. It shows how the universe evolves from an accelerating phase to a non-accelerating phase, with the transition occurring through a region where the spatial curvature becomes important. This diagram helps in understanding the dynamics of the universe during these phases and the role of the scalar field in driving the expansion.