To form a parallelogram in \(\mathbb{R}^2\) or a parallelepiped in \(\mathbb{R}^3\) using a set of linearly independent vectors, we need to understand the geometric properties and the mathematical definition of these shapes.

### Parallelogram in \(\mathbb{R}^2\)

A parallelogram in \(\mathbb{R}^2\) is defined by two linearly independent vectors \(\mathbf{u}\) and \(\mathbf{v}\). The vertices of the parallelogram can be represented as:
- Origin: \(\mathbf{0}\)
- \(\mathbf{u}\)
- \(\mathbf{v}\)
- \(\mathbf{u} + \mathbf{v}\)

The sides of the parallelogram are parallel to \(\mathbf{u}\) and \(\mathbf{v}\), and the diagonals are given by \(\mathbf{u} + \mathbf{v}\) and \(\mathbf{u} - \mathbf{v}\).

#### Example
Let's choose \(\mathbf{u} = (1, 2)\) and \(\mathbf{v} = (3, 4)\). These vectors are linearly independent because their determinant is non-zero:
\[
\det \begin{pmatrix} 1 & 3 \\ 2 & 4 \end{pmatrix} = 1 \cdot 4 - 2 \cdot 3 = 4 - 6 = -2 \neq 0.
\]
The vertices of the parallelogram are:
- \(\mathbf{0} = (0, 0)\)
- \(\mathbf{u} = (1, 2)\)
- \(\mathbf{v} = (3, 4)\)
- \(\mathbf{u} + \mathbf{v} = (1+3, 2+4) = (4, 6)\)

So, the parallelogram is formed by the points \((0,0)\), \((1,2)\), \((3,4)\), and \((4,6)\).

### Parallelepiped in \(\mathbb{R}^3\)

A parallelepiped in \(\mathbb{R}^3\) is defined by three linearly independent vectors \(\mathbf{u}\), \(\mathbf{v}\), and \(\mathbf{w}\). The vertices of the parallelepiped can be represented as:
- Origin: \(\mathbf{0}\)
- \(\mathbf{u}\)
- \(\mathbf{v}\)
- \(\mathbf{w}\)
- \(\mathbf{u} + \mathbf{v}\)
- \(\mathbf{u} + \mathbf{w}\)
- \(\mathbf{v} + \mathbf{w}\)
- \(\mathbf{u} + \mathbf{v} + \mathbf{w}\)

The faces of the parallelepiped are parallelograms, and the edges are parallel to \(\mathbf{u}\), \(\mathbf{v}\), and \(\mathbf{w}\).

#### Example
Let's choose \(\mathbf{u} = (1, 0, 0)\), \(\mathbf{v} = (0, 1, 0)\), and \(\mathbf{w} = (0, 0, 1)\). These vectors are linearly independent because they form the standard basis for \(\mathbb{R}^3\):
\[
\det \begin{pmatrix} 1 & 0 & 0 \\ 0 & 1 & 0 \\ 0 & 0 & 1 \end{pmatrix} = 1 \cdot (1 \cdot 1 - 0 \cdot 0) - 0 \cdot (0 \cdot 1 - 0 \cdot 0) + 0 \cdot (0 \cdot 0 - 1 \cdot 0) = 1 \neq 0.
\]
The vertices of the parallelepiped are:
- \(\mathbf{0} = (0, 0, 0)\)
- \(\mathbf{u} = (1, 0, 0)\)
- \(\mathbf{v} = (0, 1, 0)\)
- \(\mathbf{w} = (0, 0, 1)\)
- \(\mathbf{u} + \mathbf{v} = (1, 1, 0)\)
- \(\mathbf{u} + \mathbf{w} = (1, 0, 1)\)
- \(\mathbf{v} + \mathbf{w} = (0, 1, 1)\)
- \(\mathbf{u} + \mathbf{v} + \mathbf{w} = (1, 1, 1)\)

So, the parallelepiped is formed by the points \((0,0,0)\), \((1,0,0)\), \((0,1,0)\), \((0,0,1)\), \((1,1,0)\), \((1,0,1)\), \((0,1,1)\), and \((1,1,1)\).

### Final Answer
The example of a parallelogram in \(\mathbb{R}^2\) formed by the vectors \(\mathbf{u} = (1, 2)\) and \(\mathbf{v} = (3, 4)\) is:
\[
\boxed{(0,0), (1,2), (3,4), (4,6)}
\]

The example of a parallelepiped in \(\mathbb{R}^3\) formed by the vectors \(\mathbf{u} = (1, 0, 0)\), \(\mathbf{v} = (0, 1, 0)\), and \(\mathbf{w} = (0, 0, 1)\) is:
\[
\boxed{(0,0,0), (1,0,0), (0,1,0), (0,0,1), (1,1,0), (1,0,1), (0,1,1), (1,1,1)}
\]