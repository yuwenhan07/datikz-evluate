To solve the problem of cutting \( F_{\pi} \) along \( L \), we need to follow a series of steps carefully. Here's a detailed explanation:

1. **Identify Points \( A \) and \( B \)**: The line \( L \) intersects the boundary of the surface at two points, \( A \) and \( B \). These points will be vertices in our new triangulation.

2. **Remove Intersecting Edges and Vertices**: We need to remove all edges that intersect \( L \) and also remove any vertices that lie on \( L \) (other than \( A \) and \( B \)) along with their incident edges. This ensures that the remaining part of the surface does not include any parts that were cut by \( L \).

3. **Form a Polygon**: After removing the intersecting edges and vertices, what remains is a continuous image of a polygon with \( L \) as a secant from one vertex to another. This polygon has \( A \) and \( B \) as its vertices.

4. **Triangulate the Polygon**: To triangulate this polygon, we need to add edges inside the polygon such that every triangle formed has vertices \( A \) and \( B \). This can be done by drawing diagonals from \( A \) and \( B \) to the other vertices of the polygon, ensuring that no additional vertices are introduced except for \( A \) and \( B \).

The final result is a triangulation of the polygon with \( L \) as a secant from one vertex to another, and the triangulation uses only the vertices \( A \) and \( B \).

Thus, the answer is:
\[
\boxed{\text{A triangulation of the polygon with } L \text{ as a secant from one vertex to another, using only vertices } A \text{ and } B.}
\]