To understand the singular dynamics on the graph \(\operatorname{Graph}(c + \d T^+[u_c])\) with different colors representing the corresponding relationships before and after the map \(\Omega_c\), we need to break down the components of this expression and analyze the implications step by step.

1. **Understanding \(T^+[u_c]\)**:
   - \(T^+[u_c]\) typically represents a positive transformation or operator applied to a function \(u_c\). This could be a differential operator, a convolution, or another type of transformation depending on the context.
   - The notation \(T^+[u_c]\) suggests that it is a positive operator, meaning it maps non-negative functions to non-negative functions.

2. **Understanding \(c + \d T^+[u_c]\)**:
   - Here, \(c\) is a constant, and \(\d\) likely denotes differentiation. So, \(c + \d T^+[u_c]\) could be interpreted as a combination of a constant term and the derivative of the transformed function \(T^+[u_c]\).
   - This expression could represent a differential equation or a functional relationship in some context.

3. **Graph \(\operatorname{Graph}(c + \d T^+[u_c])\)**:
   - The graph of a function \(f(x)\) is the set of all points \((x, f(x))\) in the plane. In this case, the graph is defined by the expression \(c + \d T^+[u_c]\).
   - This means we are considering the set of all points \((x, c + \d T^+[u_c](x))\).

4. **Singular Dynamics**:
   - Singular dynamics often refer to the behavior of a system at critical points or points where the system's behavior changes abruptly.
   - In the context of the graph \(\operatorname{Graph}(c + \d T^+[u_c])\), singular dynamics would occur at points where the function \(c + \d T^+[u_c]\) has critical points, such as points where the derivative is zero or undefined, or points where the function exhibits discontinuities.

5. **Different Colors Representing Relationships Before and After \(\Omega_c\)**:
   - The use of different colors to represent the relationships before and after the map \(\Omega_c\) suggests a transformation or mapping from one set of relationships to another.
   - \(\Omega_c\) could be an operator or a function that transforms the original graph into a new graph. The colors help to distinguish the pre-image and post-image of the transformation.

6. **Conclusion**:
   - The singular dynamics on the graph \(\operatorname{Graph}(c + \d T^+[u_c])\) would involve analyzing the critical points of the function \(c + \d T^+[u_c]\) and understanding how these points change under the transformation \(\Omega_c\).
   - The colors would help to visualize and differentiate the pre-image and post-image of the transformation, highlighting any changes in the structure or behavior of the graph.

Therefore, the singular dynamics on the graph \(\operatorname{Graph}(c + \d T^+[u_c])\) with different colors representing the corresponding relationships before and after the map \(\Omega_c\) involves analyzing the critical points of the function \(c + \d T^+[u_c]\) and understanding how these points transform under \(\Omega_c\). The answer can be summarized as:

\[
\boxed{\text{The singular dynamics involve analyzing the critical points of the function } c + \d T^+[u_c] \text{ and understanding how these points transform under the map } \Omega_c.}
\]