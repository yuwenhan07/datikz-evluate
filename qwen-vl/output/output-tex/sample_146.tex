Certainly! Here is an overview of your main results, presented in the format you provided:

1. **Corollary \cref{cor:reflexive}**: This result establishes that under certain conditions, the object in question exhibits reflexive properties.

2. **Theorem \cref{thm:dual}**: This theorem demonstrates the existence of a dual structure or property that complements the original one, providing a deeper understanding of the relationship between the two.

3. **Proposition \cref{prop:NPreflexive}**: This proposition shows that the object is NP-reflexive, which means it has a specific property that is preserved under polynomial-time reductions, indicating its computational complexity.

4. **Theorem \cref{thm:refine}**: This theorem refines the previous results by providing a more precise characterization or a stronger version of the properties discussed earlier, offering a more detailed analysis.

These results collectively contribute to a comprehensive understanding of the subject matter, highlighting various aspects and their interrelations.