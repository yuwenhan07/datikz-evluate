The image you've described seems to be illustrating the process of comparing two sets of data, specifically in the context of Cross-Reference Points (CRPs). Here's a breakdown of what the image might be showing:

1. **Red Generation**: This likely represents the process of generating or creating a set of reference points (CRPs). These could be based on some initial data or a specific algorithm that produces these points.

2. **Blue Verification**: This part shows the process of verifying whether the generated CRPs match with expected or previously known CRPs. The verification is successful if the values \( q_i \) match exactly with the corresponding values \( q'_i \).

3. **Success Condition**: The verification is considered successful only when all the \( q_i \) values match their corresponding \( q'_i \) values. If there is any discrepancy, the verification fails.

### Example:
Suppose we have a set of CRPs represented by \( q_1, q_2, q_3 \) and another set of CRPs represented by \( q'_1, q'_2, q'_3 \). The verification would check if:
\[ q_1 = q'_1, \]
\[ q_2 = q'_2, \]
\[ q_3 = q'_3. \]

If all these conditions are met, the verification is successful. If any one of them does not match, the verification fails.

This kind of process is common in various fields such as data integrity checks, cryptographic protocols, and other applications where precise matching of data points is crucial.