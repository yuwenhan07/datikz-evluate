To understand the composition of the maps and the resultant error in the approximation, let's break down the process step-by-step:

1. **Encoder Map \(\mathcal{E}\)**: This map takes an input \(x\) from the original space and maps it to a lower-dimensional representation \(z = \mathcal{E}(x)\). The goal of the encoder is to capture the essential features of the input data while reducing its dimensionality.

2. **Approximator Map \(\mathcal{A}\)**: This map takes the encoded representation \(z\) and approximates the true map \(\mathcal{G}^{\dagger}_{\text{ext}}\) by some function \(f(z) = \mathcal{A}(z)\). The approximator aims to learn a mapping that closely resembles the true transformation but may introduce some error due to the approximation.

3. **Reconstructor Map \(\mathcal{R}\)**: This map takes the output of the approximator \(f(z)\) and reconstructs it back to a form that is as close as possible to the original input \(x\). The goal of the reconstructor is to minimize the difference between the reconstructed output and the original input.

The overall process can be represented as:
\[ x \rightarrow z = \mathcal{E}(x) \rightarrow f(z) = \mathcal{A}(z) \rightarrow \hat{x} = \mathcal{R}(f(z)) \]

The resultant error in the approximation is the difference between the original input \(x\) and the reconstructed output \(\hat{x}\):
\[ \text{Error} = \| x - \hat{x} \| \]

This error can be decomposed into the contributions from each of the individual maps:
- Encoder Error: The error introduced by the encoder \(\mathcal{E}\), which is the difference between the original input \(x\) and the encoded representation \(z\):
  \[ \text{Encoder Error} = \| x - z \| \]
- Approximator Error: The error introduced by the approximator \(\mathcal{A}\), which is the difference between the encoded representation \(z\) and the output of the approximator \(f(z)\):
  \[ \text{Approximator Error} = \| z - f(z) \| \]
- Reconstructor Error: The error introduced by the reconstructor \(\mathcal{R}\), which is the difference between the output of the approximator \(f(z)\) and the reconstructed output \(\hat{x}\):
  \[ \text{Reconstructor Error} = \| f(z) - \hat{x} \| \]

Thus, the total error in the approximation can be expressed as:
\[ \text{Total Error} = \| x - \hat{x} \| = \| x - z \| + \| z - f(z) \| + \| f(z) - \hat{x} \| \]

In summary, the resultant error in the approximation is the sum of the encoder error, the approximator error, and the reconstructor error. Therefore, the final answer is:
\[
\boxed{\| x - \hat{x} \| = \| x - z \| + \| z - f(z) \| + \| f(z) - \hat{x} \|}
\]