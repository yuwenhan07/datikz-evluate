The image you've described appears to be illustrating concepts related to the theory of local systems and their Fourier transforms, particularly in the context of complex geometry or algebraic topology. Let's break down the key components:

1. **Formal Local System Before Fourier Transform**:
   - This likely refers to a formal local system on a punctured disk or annulus, which is a fundamental object in the study of singularities and monodromy.
   - The formal local system can be thought of as a collection of vector spaces over the punctured disk, together with a monodromy action around the puncture.

2. **Formal Local System After Fourier Transform**:
   - The Fourier transform of a formal local system is another formal local system that encodes information about the original system in a different way.
   - In this context, the Fourier transform often involves a change of basis or a change in the representation of the monodromy, which can be quite intricate.

3. **Legendre Transform and Sectors**:
   - The Legendre transform is a tool used in physics and mathematics to relate two sets of variables that describe the same physical system. Here, it seems to be used to identify sectors of the punctured disk, which are regions separated by the puncture.
   - The identification of sectors is crucial for understanding how the monodromy changes across these sectors.

4. **Strands and Numbering**:
   - Strands refer to the paths around the puncture. The numbering of strands is done in a compatible manner, typically starting at a basepoint \(\widehat{b}\) and continuing in a clockwise direction.
   - This numbering helps in keeping track of the monodromy transformations as one moves along the strands.

5. **Partial Formal Monodromies**:
   - The partial formal monodromies are matrices that describe how the formal local system transforms when moving along the strands. These matrices are derived from the formal monodromies on the left side and may change sign depending on whether one transitions from a positive to a negative strand.

In summary, the image is likely part of a detailed explanation of how to compute and understand the Fourier transform of a formal local system, focusing on the changes in monodromy and the identification of sectors through the Legendre transform. The use of matrices and the careful numbering of strands help in visualizing and tracking these transformations.