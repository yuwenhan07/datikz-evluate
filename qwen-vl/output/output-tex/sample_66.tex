To understand the statement "The multiplication on \( L \) is determined by the complex numbers \( m_{(x,i),(y,j),l}^{(z,k)} \)," we need to break it down and interpret it step by step.

1. **Context of \( L \)**: The symbol \( L \) typically refers to a Lie algebra, which is a vector space equipped with a binary operation called the Lie bracket. However, in this context, it seems that \( L \) might be a more general algebra or structure where multiplication is defined using specific coefficients.

2. **Multiplication in \( L \)**: The multiplication in \( L \) is usually denoted as \( (x,y) \mapsto xy \). In the given statement, the multiplication is described in terms of these complex numbers \( m_{(x,i),(y,j),l}^{(z,k)} \).

3. **Interpretation of the Complex Numbers**: The notation \( m_{(x,i),(y,j),l}^{(z,k)} \) suggests that there is a coefficient associated with each triple of elements from \( L \). Specifically, \( m_{(x,i),(y,j),l}^{(z,k)} \) represents the coefficient for the product of the elements \( x \) and \( y \) when they are combined with other elements \( i, j, l, \) and \( k \).

4. **Formal Definition**: The multiplication in \( L \) can be expressed as:
   \[
   (x,y) \mapsto \sum_{i,j,l,k} m_{(x,i),(y,j),l}^{(z,k)} (ixj + lzk)
   \]
   Here, \( ixj \) and \( lzk \) are the products of the elements involved, and the sum is taken over all possible combinations of \( i, j, l, \) and \( k \).

5. **Conclusion**: The statement "The multiplication on \( L \) is determined by the complex numbers \( m_{(x,i),(y,j),l}^{(z,k)} \)" means that the way elements of \( L \) multiply together is completely specified by these complex numbers. Each coefficient \( m_{(x,i),(y,j),l}^{(z,k)} \) tells us how the product of \( x \) and \( y \) interacts with the other elements \( i, j, l, \) and \( k \).

Therefore, the final answer is:
\[
\boxed{m_{(x,i),(y,j),l}^{(z,k)}}
\]