To solve the problem, we need to understand the concepts of saddle moves and twist moves in the context of link theory, particularly focusing on how these moves can transform a 2-component link into a knot.

### Step-by-Step Solution:

1. **Understanding Saddle Moves**:
   - A saddle move is a local operation that can be performed on a link diagram. It involves cutting the link at two points and then reconnecting it in a different way.
   - Specifically, a saddle move can be used to change the number of components of a link. For example, if you have a 2-component link (a link consisting of two separate loops), a saddle move can potentially reduce the number of components to one, turning the link into a knot.

2. **Link Lemma \ref{lem:link}**:
   - The Link Lemma typically refers to a result in knot theory that states that any link can be transformed into a knot through a series of local moves. These moves include saddle moves, twist moves, and crossing changes.
   - The lemma guarantees that given any link, there exists a sequence of saddle moves, twist moves, and crossing changes that will transform the link into a single component (a knot).

3. **Twist Moves**:
   - Twist moves involve adding or removing twists in a portion of the link diagram. They are indicated by a purple segment in the diagram, which suggests a specific type of local transformation.
   - Twist moves are useful for changing the topology of the link, but they do not directly reduce the number of components unless combined with other moves like saddle moves.

4. **Combining Saddle and Twist Moves**:
   - To transform a 2-component link into a knot, you would typically use a combination of saddle moves and twist moves. The saddle move can reduce the number of components, and then twist moves can further manipulate the remaining components to form a knot.
   - The exact sequence of moves depends on the specific configuration of the link, but the key idea is to use saddle moves to simplify the link and then use twist moves to refine the structure until it becomes a knot.

### Final Answer:
The saddle moves can be used to turn connect summands that are 2-component links into knots by reducing the number of components through saddle moves and then using twist moves to further manipulate the remaining components into a single knot. The process is guaranteed by the Link Lemma, which states that any link can be transformed into a knot through a series of local moves including saddle moves and twist moves.

\boxed{\text{Saddle moves can be used to turn 2-component links into knots by reducing the number of components and then using twist moves to further manipulate the remaining components into a single knot.}}