The process you've described is a common method for predicting bounding boxes in object detection tasks, particularly when using a regression-based approach. Here's a detailed breakdown of the steps involved:

1. **Center Coordinates Calculation**:
   - The center coordinates \((x_c, y_c)\) of the bounding box are calculated using the predicted values \(t_x\) and \(t_y\) (which represent offsets relative to the center of the grid cell). These values are typically passed through a sigmoid function to ensure they lie within the range \([0, 1]\).
   - The actual center coordinates are then obtained by adding these sigmoid-transformed values to the center coordinates of the grid cell \(c_x\) and \(c_y\):
     \[
     x_c = c_x + \sigma(t_x)
     \]
     \[
     y_c = c_y + \sigma(t_y)
     \]
     where \(\sigma(t)\) is the sigmoid function defined as:
     \[
     \sigma(t) = \frac{1}{1 + e^{-t}}
     \]

2. **Width and Height Adjustment**:
   - The width \(w\) and height \(h\) of the bounding box are initially set to the previous width \(p_w\) and height \(p_h\) of the bounding box.
   - The width and height are then adjusted by scaling them with the exponential of the predicted values \(t_w\) and \(t_h\):
     \[
     w = p_w \cdot e^{t_w}
     \]
     \[
     h = p_h \cdot e^{t_h}
     \]

3. **Bounding Box Clipping**:
   - After calculating the center coordinates and adjusting the width and height, it's important to clip the bounding box to ensure that its dimensions do not exceed the image boundaries. This is typically done by checking if the top-left corner \((x_c - w/2, y_c - h/2)\) and bottom-right corner \((x_c + w/2, y_c + h/2)\) of the bounding box fall within the image dimensions.

4. **Final Bounding Box**:
   - The final bounding box is defined by its center coordinates \((x_c, y_c)\), width \(w\), and height \(h\).

This method allows for flexible adjustment of the bounding box size based on the predicted values while ensuring that the center coordinates are accurately positioned relative to the grid cells. The use of exponential scaling for width and height helps in handling cases where the predicted values might be very small or large, which could lead to unrealistic bounding box sizes otherwise.

This approach is commonly used in object detection frameworks like YOLO (You Only Look Once) and SSD (Single Shot MultiBox Detector), where bounding box predictions are made directly from the network output.