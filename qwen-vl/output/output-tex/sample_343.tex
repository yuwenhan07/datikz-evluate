The statement you provided refers to Lemma \ref{lemma:orthog_wave_and_balanced_set}, which asserts that the wave set (often denoted as \( W \)) is \( L^2_H \)-orthogonal to the formal tangent space to the balanced set (denoted as \( T_B \)). Here's a more detailed explanation:

1. **Wave Set (\( W \)):** This typically refers to a set of functions or modes that are associated with waves in a given context, such as in harmonic analysis or partial differential equations.

2. **\( L^2_H \)-Orthogonality:** The notation \( L^2_H \) refers to the Hilbert space of square-integrable functions on some domain \( H \). Two sets of functions are said to be \( L^2_H \)-orthogonal if their inner product in this space is zero. In other words, for any function from one set and any function from the other set, the integral of their product over the domain \( H \) is zero.

3. **Formal Tangent Space (\( T_B \)):** The tangent space at a point \( B \) is a linear subspace that represents the directions in which one can move infinitesimally around \( B \). In the context of a balanced set, the formal tangent space might refer to the directions that are consistent with the properties of the balanced set.

4. **Balanced Set (\( B \)):** A balanced set is often a set that has certain symmetry properties, such as being invariant under some group action or having a specific geometric structure. The tangent space to this set captures the directions in which the set can be deformed while preserving these properties.

In summary, Lemma \ref{lemma:orthog_wave_and_balanced_set} states that the wave set \( W \) is orthogonal to the formal tangent space \( T_B \) in the \( L^2_H \) sense. This orthogonality condition implies that there is no overlap between the wave modes and the directions of deformation allowed by the balanced set, which can have important implications in the study of wave propagation and stability of structures.