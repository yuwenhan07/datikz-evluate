The image you've described seems to be related to a greedy algorithm applied to a tree structure, specifically one that is constructed from a set \( S \). Let's break down the components and the process:

1. **Initial Tree \( I_p = \mathcal{M}(S) \)**:
   - This represents the initial tree constructed from a set \( S \). The tree \( \mathcal{M}(S) \) is likely a minimum spanning tree (MST) or a similar structure derived from \( S \).

2. **Shaded Rectangles**:
   - The two shaded rectangles in the image indicate nodes in the tree that are "arborally satisfied" at a specific time \( i \). This term suggests that these nodes have been processed or selected by some algorithm up to time \( i \).

3. **Point \( z_i \)**:
   - The point \( z_i \) is associated with the time \( i \). It could represent a specific element or value that influences the selection or processing of nodes in the tree.

4. **Greedy Algorithm**:
   - The text `\textsc{Greedy}\xspace` indicates that the algorithm being applied is a greedy algorithm. Greedy algorithms make locally optimal choices at each step with the hope of finding a global optimum.
   - In the context of trees, a common greedy approach might involve selecting the next node based on some criterion (e.g., minimizing the cost, maximizing the benefit, etc.), ensuring that the chosen node satisfies certain conditions (like being part of the MST).

### Execution Process:
- At time \( i \), the algorithm has processed some nodes and is currently considering the point \( z_i \).
- The two shaded rectangles represent nodes that have already been processed and satisfy a condition (arborally satisfied) due to the influence of \( z_i \).
- The algorithm continues to select nodes based on its greedy strategy until all nodes in the tree are processed.

### Example:
If we were dealing with a minimum spanning tree (MST) construction, the algorithm would typically start with an empty tree and iteratively add edges that connect new vertices while maintaining the property of being a tree and having the smallest possible total edge weight.

In this case, the shaded rectangles might represent nodes that have been included in the MST so far, and the point \( z_i \) could be the next vertex to be added to the MST based on the greedy rule (e.g., adding the edge with the smallest weight that connects a new vertex to the existing MST).

### Conclusion:
The image depicts a moment in the execution of a greedy algorithm on a tree \( I_p = \mathcal{M}(S) \). The two shaded rectangles are nodes that have been processed and satisfy a condition due to the influence of the point \( z_i \) at time \( i \). The algorithm continues to process nodes based on its greedy strategy until the entire tree is processed.