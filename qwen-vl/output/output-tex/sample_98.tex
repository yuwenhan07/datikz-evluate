To analyze the impact of an accreted clump on the stellar surface using an idealized mechanical system consisting of two springs and two masses, we can follow these steps:

1. **Define the System:**
   - Let the stellar surface be represented by a mass \(M\) at rest.
   - Let the accreted clump be represented by a small mass \(m \ll M\).
   - The clump is connected to the stellar surface by a spring with spring constant \(K\).
   - The clump is also connected to another point in space by a spring with spring constant \(k\).

2. **Equations of Motion:**
   - For the mass \(M\), the force exerted by the spring \(K\) is given by Hooke's law: \(F = Kx\), where \(x\) is the displacement from equilibrium.
   - For the mass \(m\), the force exerted by the spring \(k\) is given by Hooke's law: \(F = kx_m\), where \(x_m\) is the displacement of the clump from its equilibrium position.

3. **Assumptions:**
   - Assume that the clump is initially at rest and then collides with the stellar surface.
   - Assume that the collision is perfectly elastic, meaning that the kinetic energy is conserved during the collision.

4. **Energy Conservation:**
   - Before the collision, the total potential energy of the system is due to the spring \(K\): \(U_i = \frac{1}{2}Kx_0^2\), where \(x_0\) is the initial displacement of the clump from the stellar surface.
   - After the collision, the total potential energy of the system is due to both springs: \(U_f = \frac{1}{2}Kx_1^2 + \frac{1}{2}kx_m^2\), where \(x_1\) is the new displacement of the clump from the stellar surface after the collision.

5. **Conservation of Momentum:**
   - Since the collision is perfectly elastic, the total momentum before and after the collision is conserved. However, since the stellar surface is much larger than the clump, it will not move significantly, so we can assume that the momentum of the clump is conserved.
   - The initial momentum of the clump is zero because it is at rest. After the collision, the momentum of the clump is \(mv\), where \(v\) is the velocity of the clump after the collision.

6. **Solving for the Displacements:**
   - Using the conservation of energy, we have:
     \[
     \frac{1}{2}Kx_0^2 = \frac{1}{2}Kx_1^2 + \frac{1}{2}kx_m^2
     \]
   - Using the conservation of momentum, we have:
     \[
     mv = kx_m
     \]
     Solving for \(x_m\), we get:
     \[
     x_m = \frac{mv}{k}
     \]
   - Substituting \(x_m\) into the energy equation, we get:
     \[
     \frac{1}{2}Kx_0^2 = \frac{1}{2}Kx_1^2 + \frac{1}{2}k\left(\frac{mv}{k}\right)^2
     \]
     Simplifying, we get:
     \[
     \frac{1}{2}Kx_0^2 = \frac{1}{2}Kx_1^2 + \frac{1}{2}m^2v^2
     \]
     \[
     Kx_0^2 = Kx_1^2 + m^2v^2
     \]
     \[
     x_1^2 = x_0^2 - \frac{m^2v^2}{K}
     \]
     Taking the square root of both sides, we get:
     \[
     x_1 = \pm \sqrt{x_0^2 - \frac{m^2v^2}{K}}
     \]

The final answer is:
\[
\boxed{x_1 = \pm \sqrt{x_0^2 - \frac{m^2v^2}{K}}}
\]