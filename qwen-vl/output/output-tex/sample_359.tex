To understand the cycles \( C_u \) and \( C_v \) and the forbidden pairs \((1,1)\) and \((1,2)\) corresponding to the edge \( uv \), we need to delve into graph theory, specifically the concept of forbidden pairs in the context of graph minors.

### Step-by-Step Explanation:

1. **Graph Minors and Forbidden Pairs:**
   - A graph minor is a graph formed by contracting edges and deleting vertices from another graph.
   - A forbidden pair \((H_1, H_2)\) for a graph \(G\) means that \(G\) does not contain any subgraph that is a minor of both \(H_1\) and \(H_2\).

2. **Forbidden Pair \((1,1)\):**
   - The pair \((1,1)\) typically refers to the forbidden pair where \(H_1\) and \(H_2\) are both the complete graph on 2 vertices, \(K_2\).
   - This means that the graph \(G\) cannot have any subgraph that is a minor of \(K_2\). Since \(K_2\) is just two vertices connected by an edge, this condition is trivially satisfied because no graph can be a minor of itself.
   - Therefore, the pair \((1,1)\) does not provide any useful information about the structure of the graph \(G\).

3. **Forbidden Pair \((1,2)\):**
   - The pair \((1,2)\) refers to the forbidden pair where \(H_1\) is the complete graph on 2 vertices \(K_2\) and \(H_2\) is the complete graph on 3 vertices \(K_3\).
   - This means that the graph \(G\) cannot have any subgraph that is a minor of either \(K_2\) or \(K_3\).
   - Since \(K_2\) is just two vertices connected by an edge, it is always a minor of any graph with at least two vertices. Therefore, the condition that \(G\) does not contain \(K_2\) as a minor is always satisfied.
   - However, \(K_3\) (the complete graph on 3 vertices) is not a minor of any graph with fewer than 3 vertices. So, if \(G\) has at least 3 vertices, it must contain \(K_3\) as a minor. Thus, the only graphs that satisfy the condition of not containing \(K_3\) as a minor are those with fewer than 3 vertices, which are the empty graph and the single vertex graph.

4. **Cycles \(C_u\) and \(C_v\):**
   - The cycles \(C_u\) and \(C_v\) are cycles involving the vertices \(u\) and \(v\).
   - If \(u\) and \(v\) are adjacent, then \(C_u\) and \(C_v\) could be cycles that include the edge \(uv\). For example, if \(u\) and \(v\) are part of a larger cycle, then \(C_u\) and \(C_v\) might be parts of that cycle.
   - If \(u\) and \(v\) are not adjacent but form a forbidden pair with some other structure, the cycles \(C_u\) and \(C_v\) would be more complex and would depend on the specific structure of the graph.

### Conclusion:
The forbidden pair \((1,1)\) does not provide any useful information about the structure of the graph \(G\). The forbidden pair \((1,2)\) implies that \(G\) must be a graph with fewer than 3 vertices, i.e., the empty graph or the single vertex graph. The cycles \(C_u\) and \(C_v\) are cycles involving the vertices \(u\) and \(v\), and their exact form depends on the specific structure of the graph.

Thus, the answer is:
\[
\boxed{(1,1) \text{ does not provide useful information, } (1,2) \text{ implies } G \text{ is the empty graph or a single vertex graph}, \text{ and } C_u \text{ and } C_v \text{ are cycles involving } u \text{ and } v}
\]