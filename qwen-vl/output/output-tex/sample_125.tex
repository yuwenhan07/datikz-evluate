The Vaidya metric is a solution to Einstein's field equations that describes the evolution of a spherically symmetric, time-dependent mass distribution in an anti-de Sitter (AdS) spacetime. This solution is particularly interesting because it interpolates between different geometries depending on the location relative to the mass shell.

### Key Points:

1. **AdS-Vaidya Spacetime**:
   - The Vaidya metric describes a spacetime where the mass distribution changes with time.
   - It is a generalization of the Schwarzschild metric for a time-dependent mass.

2. **Asymptotic Boundary**:
   - The right side boundary of the Vaidya geometry represents the asymptotic boundary of the AdS-Vaidya spacetime.
   - This boundary is typically at infinity in the AdS context and is where the spacetime approaches its asymptotic AdS geometry.

3. **Black Hole Geometry**:
   - Outside the mass shell, the Vaidya metric describes a black hole geometry.
   - The event horizon forms as the mass distribution grows, and the spacetime becomes singular at this point.

4. **Vacuum AdS in Poincaré Patch**:
   - Inside the mass shell, the Vaidya metric reduces to the vacuum anti-de Sitter (AdS) spacetime in the Poincaré patch.
   - The Poincaré patch is a coordinate system that covers a portion of the AdS spacetime, which is useful for studying the asymptotic behavior and the causal structure of the spacetime.

### Summary:
- **Outside the Mass Shell**: The Vaidya metric describes a black hole geometry, with the event horizon forming as the mass grows.
- **Inside the Mass Shell**: The Vaidya metric reduces to the vacuum AdS spacetime in the Poincaré patch, which is a region of negative curvature that extends to infinity in the asymptotic direction.

This description highlights the dynamic nature of the Vaidya metric, where the spacetime evolves from a vacuum AdS state to a black hole geometry as the mass distribution changes over time.