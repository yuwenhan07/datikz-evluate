The process you're describing sounds like it could be related to genetic algorithms, which are used in various fields such as optimization, machine learning, and evolutionary computation. However, there seems to be some confusion in your description. Let's break down the steps more clearly:

### 1. **Order the Current Generation**
   - **Best to Worst**: Evaluate each individual in the current generation based on a fitness function (which measures how well an individual performs a specific task). Rank them from best to worst.

### 2. **Discard the Worst Genes**
   - **Red Part**: Remove the least fit individuals from the population. This step is often referred to as "elitism" if the best individuals are not discarded but rather preserved for the next generation.

### 3. **Copy the Best Genes**
   - **Green Part**: Copy the best individuals directly into the new generation. These are typically the fittest individuals and are expected to contribute positively to the next generation.

### 4. **Create New Individuals via Crossover**
   - **Grey Part**: Use crossover operations to combine the remaining genes from the previous generation. Crossover involves selecting pairs of individuals and combining their genetic material to create offspring. This can be done by:
     - **Single Point Crossover**: Choose a single point along the chromosome and swap the segments before and after this point.
     - **Multi-Point Crossover**: Choose multiple points and swap segments between these points.
     - **Uniform Crossover**: Randomly decide whether to take a gene from one parent or the other at each position.

### 5. **Mutation**
   - After crossover, apply mutation to introduce small random changes in the genes. This helps maintain diversity in the population and prevents premature convergence.

### 6. **Repeat**
   - Repeat the above steps for several generations until a stopping criterion is met (e.g., a maximum number of generations, a satisfactory solution found).

### Example Steps:
1. **Evaluate Fitness**: Calculate the fitness of each individual.
2. **Select Parents**: Choose parents for crossover based on their fitness (e.g., tournament selection).
3. **Crossover**: Perform crossover on selected parents to produce offspring.
4. **Mutate**: Apply mutation to the offspring.
5. **Replace Population**: Replace the old population with the new population (or keep the best individuals from the old population).

This process is iterative and aims to evolve a population towards better solutions over time. The key components are selection, crossover, and mutation, which work together to simulate natural selection and evolution.