To understand the relations between various summation methods as described, let's break down each notation and its meaning:

1. \(a \to b\): This notation means that the \(a\)-summation method is a special case of the \(b\)-summation method. In other words, if a series is summable using the \(a\)-method, it is also summable using the \(b\)-method, but not necessarily vice versa. The \(a\)-method is more restrictive than the \(b\)-method.

2. \(a \twoheadrightarrow b\): This notation means that the \(a\)-summable series imply that they are also \(b\)-summable. In other words, if a series is summable using the \(a\)-method, then it is also summable using the \(b\)-method. The \(a\)-method is less restrictive than the \(b\)-method.

Let's summarize these relationships in a table for clarity:

| \(a \to b\) | \(a \twoheadrightarrow b\) |
|-------------|---------------------------|
| \(a\)-summation is a special case of \(b\)-summation. | \(a\)-summable implies \(b\)-summable. |

So, the final answer is:

\[
\boxed{a \to b \text{ means } a\text{-summation is a special case of } b\text{-summation}; \quad a \twoheadrightarrow b \text{ means } a\text{-summable implies } b\text{-summable.}}
\]