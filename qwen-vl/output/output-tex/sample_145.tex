To solve the problem, we need to understand the structure of the question set \(\mathcal{Q}_{AB}\) in a composite system composed of individual systems \(A\) and \(B\). The question set \(\mathcal{Q}_{AB}\) includes all individual questions from \(\mathcal{Q}_A\) and \(\mathcal{Q}_B\), as well as composite questions that involve logical gates connecting questions from \(A\) and \(B\).

Let's break it down step by step:

1. **Identify Individual Questions**: The individual questions from system \(A\) are denoted as \(\mathcal{Q}_A\) and the individual questions from system \(B\) are denoted as \(\mathcal{Q}_B\). These questions are not connected by any logical gates and can be considered as separate entities.

2. **Composite Questions**: Composite questions are formed by combining questions from \(A\) and \(B\) using logical gates. The logical gates are represented by the different thicknesses of the lines connecting the questions in the figure. For example, if a line connecting a question from \(A\) and a question from \(B\) is thick, it indicates a specific type of logical gate, say AND, OR, etc.

3. **Form of Composite Questions**: Each composite question is in the form \(Q_a *_i Q_b\), where \(Q_a\) is a question from system \(A\), \(Q_b\) is a question from system \(B\), and \(*_i\) is a logical gate. The logical gate \(*_i\) can be any of the standard logical gates such as AND, OR, NOT, etc., depending on the thickness of the line connecting the questions.

4. **Example**: If we have a question \(Q_A\) from system \(A\) and a question \(Q_B\) from system \(B\), and they are connected by a thick line, then the composite question would be \(Q_A \text{ AND } Q_B\). If the line is thin, it might represent an OR gate or another type of logical gate.

Therefore, the structure of the question set \(\mathcal{Q}_{AB}\) is a combination of all individual questions from \(\mathcal{Q}_A\) and \(\mathcal{Q}_B\) along with all possible composite questions formed by connecting questions from \(A\) and \(B\) using logical gates, as indicated by the different thicknesses of the lines in the figure.

The final answer is:
\[
\boxed{\mathcal{Q}_{AB} = \mathcal{Q}_A \cup \mathcal{Q}_B \cup \{Q_a *_i Q_b \mid Q_a \in \mathcal{Q}_A, Q_b \in \mathcal{Q}_B, *_i \text{ is a logical gate}\}}
\]