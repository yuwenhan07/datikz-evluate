It seems like you're referring to a geometric or topological context, possibly related to the study of curves and their intersections in a plane or higher-dimensional space. The notation "the local situation around the \( i \)-th self-intersection of \(\gamma\)" suggests that we are looking at a curve \(\gamma\) and its behavior near one of its self-intersections.

Given the numbers \( k = 1, \ldots, 8 \) in red, it is likely that these numbers represent specific points or features around the \( i \)-th self-intersection of the curve \(\gamma\). Here's a possible interpretation:

1. **Self-intersection**: A self-intersection of a curve occurs when the curve crosses itself.
2. **Local Situation**: This refers to the behavior of the curve near this intersection point.
3. **Numbers \( k = 1, \ldots, 8 \)**: These could be indices or labels for different features or points of interest around the self-intersection. For example:
   - \( k = 1 \) might represent the point where the curve first intersects itself.
   - \( k = 2 \) might represent the point where the curve re-crosses itself after the first intersection.
   - And so on up to \( k = 8 \).

Without more specific details about the curve \(\gamma\) and the nature of these points, it's difficult to provide a more precise interpretation. However, if you have a diagram or further context, I can help you analyze it more thoroughly.

If you have a specific diagram or more information about the curve \(\gamma\) and the points \( k = 1, \ldots, 8 \), please share it, and I'll be happy to help you understand the local situation around the \( i \)-th self-intersection.