The statement you've provided seems to be discussing the relationship between different formulations of a balance law, particularly when dealing with a fixed continuous solution and a non-degenerate assumption on the Lagrangian source. Let's break down the key points:

1. **Balance Law**: The balance law typically describes the conservation of some quantity (like mass, momentum, energy) over time or space. It often takes the form:
   \[
   \frac{\partial u}{\partial t} + \nabla \cdot f(u) = g(u)
   \]
   where \(u\) is the state vector, \(f(u)\) is the flux function, and \(g(u)\) is the source term.

2. **Continuous Solution**: A continuous solution means that the solution \(u(x,t)\) is smooth and does not have any discontinuities within the domain of interest.

3. **Non-Degeneracy Assumption**: This likely refers to an assumption about the properties of the source term \(g(u)\). Non-degeneracy might imply that the source term is well-behaved and does not cause singularities or other issues in the solution.

4. **Lagrangian Source**: In the context of balance laws, the Lagrangian source term can be interpreted as a term that accounts for external influences or interactions that affect the conservation law. If the Lagrangian source is continuous, it suggests that these influences are smooth and do not introduce abrupt changes in the system.

5. **Formulations**: Different formulations of the balance law might arise due to various mathematical or physical interpretations. For example, one might consider the Eulerian formulation (where the variables are functions of space and time), or the Lagrangian formulation (where the variables are functions of the material coordinates).

6. **Source Term**: When the Lagrangian source is continuous, it is often considered the primary source term in the balance law. This is because a continuous source term implies that the external influences are smooth and do not cause sudden changes in the system dynamics.

In summary, when the Lagrangian source is continuous, it is the dominant term in the balance law, and its smoothness ensures that the solution remains continuous. This is particularly important in the context of non-degenerate assumptions, which help in ensuring that the solution is well-defined and physically meaningful.