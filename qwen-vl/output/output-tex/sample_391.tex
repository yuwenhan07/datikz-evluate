The schematic illustration you've described provides a visual representation of how a scientist's research portfolio evolves over time, considering various dynamics such as conversion, displacement, and layering. Here’s a breakdown of what these terms mean in the context of a scientist's research career:

1. **Conversion**: This refers to the process where a scientist applies knowledge gained from one research area to address problems in another area. For example, a scientist who initially studies molecular biology might later apply their understanding of genetic regulation to study evolutionary biology.

2. **Displacement**: This occurs when a scientist shifts focus from one research area to another without transferring much of the prior knowledge. An example could be a scientist moving from studying plant genetics to animal behavior, where they bring minimal knowledge from the previous field but still contribute valuable insights based on their new expertise.

3. **Layering**: This involves adding new research areas to an existing portfolio while maintaining some level of overlap between the new and old areas. For instance, a scientist might start by studying the effects of climate change on plants and then layer in the study of how these effects influence soil microbiomes, thereby expanding their research scope while building upon foundational knowledge.

In the context of the schematic illustration:
- Different colors represent distinct research areas.
- Resources indicate the intellectual and methodological tools used by the scientist.
- The arrows and transitions between colors illustrate how the scientist moves between different research areas, whether through conversion, displacement, or layering.

This type of visualization can help researchers and educators understand the complexity and fluidity of a scientist's career trajectory, highlighting how diverse research interests can coexist and evolve over time. It also underscores the importance of adaptability and continuous learning in scientific careers.