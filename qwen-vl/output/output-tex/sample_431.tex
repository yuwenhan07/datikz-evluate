To analyze the sum rate of the \gls{pcnc} (Partial Coded Network Coding) scheme with respect to the reference distance \( D \), we need to understand the relationship between the sum rate, the parameters \( M \), \( K \), \( N \), and the distance \( d_{kc} \).

Given:
- \( M = 5 \)
- \( K = 5 \)
- \( N = 2 \)
- \( d_{kc} = 2 \quad \forall k, c \)

The sum rate in a network coding scenario is typically influenced by the number of nodes (\( M \)), the number of sources (\( K \)), the number of destinations (\( N \)), and the distances between them (\( d_{kc} \)). In this context, the sum rate can be expressed as a function of these parameters.

For the \gls{pcnc} scheme, the sum rate generally increases with the reference distance \( D \) because more information can be transmitted over longer distances. However, the exact form of the sum rate function depends on the specific details of the network coding scheme and the underlying channel model.

The dashed lines in the plot represent the sum rates of the \gls{fc} (Full Coded) operation, which is a baseline for comparison. The \gls{fc} operation typically has a higher sum rate than the \gls{pcnc} due to its ability to fully exploit the network coding capabilities.

Without the actual plot or further details about the specific sum rate function used in the \gls{pcnc} scheme, it's not possible to provide an explicit mathematical expression for the sum rate. However, based on general principles:

1. **Sum Rate Function**: The sum rate \( R(D) \) of the \gls{pcnc} scheme can be modeled as a function of the reference distance \( D \). It might be something like:
   \[
   R(D) = f(D; M, K, N, d_{kc})
   \]
   where \( f \) is a function that captures the dependency on the parameters \( M \), \( K \), \( N \), and \( d_{kc} \).

2. **Comparison with \gls{fc}**: Since the \gls{fc} operation has a higher sum rate, the \gls{pcnc} sum rate will be lower but still increasing with \( D \). The difference between the two will be more pronounced at smaller values of \( D \) and diminish as \( D \) increases.

3. **Plot Interpretation**: The plot would show the sum rate of the \gls{pcnc} scheme as a curve that starts below the \gls{fc} curve and increases with \( D \). The exact shape of the curve would depend on the specific implementation of the \gls{pcnc} scheme.

In summary, without the actual plot, we can only describe the expected behavior qualitatively. The sum rate of the \gls{pcnc} scheme will increase with the reference distance \( D \) and will be lower than the sum rate of the \gls{fc} operation, but the difference will decrease as \( D \) increases. The exact form of the sum rate function would require the specific details of the \gls{pcnc} scheme and the channel model.