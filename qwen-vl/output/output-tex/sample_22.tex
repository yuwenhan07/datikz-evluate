It seems like you're referring to a mathematical or theoretical computer science context, possibly involving graph theory or algebraic structures. Let's break down the two parts of your statement:

### Left: The Case \( S \nsubseteq N[x] \cup N[y] \) and \( |Z| \geq 5 \)

In this part, you're discussing a scenario where:
- \( S \) is a set or structure.
- \( N[x] \) and \( N[y] \) are neighborhoods (sets of elements adjacent to \( x \) and \( y \), respectively).
- \( S \) is not a subset of the union of these neighborhoods.
- The size of another set \( Z \) is at least 5.

This could be part of a proof by contradiction or an analysis of a specific property of graphs or algebraic structures. For example, if \( S \) represents a set of vertices in a graph, and \( N[x] \cup N[y] \) represents the neighborhood of vertices \( x \) and \( y \), then the condition \( S \nsubseteq N[x] \cup N[y] \) implies that there are elements in \( S \) that are not adjacent to either \( x \) or \( y \). Additionally, \( |Z| \geq 5 \) suggests that there is a significant number of elements in \( Z \).

### Right: The Proof of Claim~\ref{clm3}

Here, you're considering a proof by contradiction for Claim~\ref{clm3}. The claim involves the existence of some element \( x_i \) such that the intersection of its neighborhood with a set \( Y \) has at least 5 elements. This can be interpreted as follows:
- \( N(x_i) \) is the neighborhood of vertex \( x_i \).
- \( Y \) is a subset of the graph or structure being analyzed.
- The condition \( |N(x_i) \cap Y| \geq 5 \) means that there are at least 5 elements in both \( N(x_i) \) and \( Y \).

The proof proceeds by assuming the opposite of what Claim~\ref{clm3} states (i.e., that no such \( x_i \) exists) and showing that this assumption leads to a contradiction. This is a common technique in proofs by contradiction, where one assumes the negation of the desired conclusion and shows that it cannot hold under the given conditions.

### Summary

The left part of your statement is about a specific condition on sets and their relationships within a larger structure, while the right part describes a proof strategy used to establish a claim. Both parts seem to be part of a more comprehensive proof or analysis, likely in a field such as graph theory or abstract algebra. If you need further clarification or have additional details, feel free to ask!