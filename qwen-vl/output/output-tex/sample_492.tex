To solve the problem, we need to understand the structure of the Euler cover \(\varphi: M \to \Delta_5^2\) and how it maps to the 2-skeleton of the 5-dimensional simplex \(\Delta_5\).

First, let's recall some basic properties of the 5-dimensional simplex \(\Delta_5\):
- It has 6 vertices.
- It has 15 edges.
- It has 20 triangular faces.

The Euler cover \(\varphi: M \to \Delta_5^2\) maps the pseudomanifold \(M\) to the 2-skeleton of \(\Delta_5\). The pseudomanifold \(M\) has:
- 6 vertices.
- 30 edges.
- 20 triangular faces (including the unbounded face).

Since vertices with the same labels are identified, \(M\) is a sphere with six pinchpoints. This means that \(M\) is a sphere with six points identified, which is topologically equivalent to a sphere with six pinchpoints.

The problem states that the three shaded areas of \(M\) map to three pairwise face-disjoint tetrahedron boundaries in \(K\). The white triangles (including the unbounded region) map to the boundary of an octahedron in \(K\). This decomposition of \(K\) as a face-disjoint union of four spheres (circlets) can be understood by considering the combinatorial structure of the simplicial complex \(K\).

The key insight here is that the Euler cover \(\varphi\) decomposes the 5-dimensional simplex \(\Delta_5\) into four spheres (circlets), each corresponding to one of the four regions formed by the mapping of the shaded and white areas of \(M\).

Thus, the final answer is:

\[
\boxed{4}
\]