To solve the problem, we need to find the perfect linear realizations \(\mathbf{g_1}\) and \(\mathbf{g_2}\) for the given sets of functions. A perfect linear realization is a way to express each function in the set as a linear combination of the other functions in the set.

### Step 1: Find \(\mathbf{g_1}\) for \(\{ (x-1)^{x-1}, x^x \}\)

We need to express \(x^x\) as a linear combination of \((x-1)^{x-1}\) and some other function. Let's assume:
\[ x^x = a(x-1)^{x-1} + b \]
where \(a\) and \(b\) are constants to be determined.

To find \(a\), we can use the fact that the degree of the polynomial on the right-hand side must match the degree of the polynomial on the left-hand side. The degree of \(x^x\) is \(x\), and the degree of \((x-1)^{x-1}\) is also \(x\). Therefore, the constant term \(b\) must be zero because the degree of \((x-1)^{x-1}\) is not a constant term.

So, we have:
\[ x^x = a(x-1)^{x-1} \]

To find \(a\), we can compare the coefficients of the corresponding terms. However, this is not straightforward algebraically, so we can use another approach by substituting specific values of \(x\).

Let's substitute \(x = 2\):
\[ 2^2 = a(2-1)^{2-1} \]
\[ 4 = a \cdot 1 \]
\[ a = 4 \]

Thus, we have:
\[ x^x = 4(x-1)^{x-1} \]

So, the perfect linear realization \(\mathbf{g_1}\) is:
\[ \boxed{4} \]

### Step 2: Find \(\mathbf{g_2}\) for \(\{ x^{x+2}, (x+1)^{x-1} \}\)

We need to express \(x^{x+2}\) as a linear combination of \((x+1)^{x-1}\) and some other function. Let's assume:
\[ x^{x+2} = c(x+1)^{x-1} + d \]
where \(c\) and \(d\) are constants to be determined.

Again, the degree of the polynomial on the right-hand side must match the degree of the polynomial on the left-hand side. The degree of \(x^{x+2}\) is \(x+2\), and the degree of \((x+1)^{x-1}\) is also \(x-1\). Therefore, the constant term \(d\) must be zero because the degree of \((x+1)^{x-1}\) is not a constant term.

So, we have:
\[ x^{x+2} = c(x+1)^{x-1} \]

To find \(c\), we can use the fact that the leading coefficient of the polynomial on the right-hand side must match the leading coefficient of the polynomial on the left-hand side. The leading coefficient of \(x^{x+2}\) is 1, and the leading coefficient of \((x+1)^{x-1}\) is 1. Therefore, we have:
\[ c = 1 \]

Thus, we have:
\[ x^{x+2} = (x+1)^{x-1} \]

So, the perfect linear realization \(\mathbf{g_2}\) is:
\[ \boxed{1} \]