The image you've described seems to be related to wave interactions in physics, particularly in the context of wave diffraction or wave scattering. Here's a general explanation based on the typical behavior of waves when they encounter boundaries or interact with each other:

1. **Wave Fronts**: Wave fronts represent the locus of points at which the wave is in phase. In the image, we see two wave fronts from different families, labeled as \( j \)-th and \( k \)-th.

2. **Interaction at the Boundary**: When a wave front encounters a boundary, it can undergo reflection, transmission, or both, depending on the properties of the boundary and the wave itself. The reflected wave front will have a different direction but the same frequency and wavelength as the incident wave front.

3. **Interaction Between Wave Fronts**: When two wave fronts from different families interact, they can produce new wave fronts. This interaction can result in constructive interference (where the amplitudes add up), destructive interference (where the amplitudes cancel out), or a combination of both, leading to the formation of new wave patterns.

4. **New Wave Fronts Generated**: The image shows the generation of new wave fronts at the point of interaction. These new wave fronts could be due to the superposition of the original wave fronts or the reflection and refraction of the waves at the boundary.

In summary, the image likely illustrates the complex behavior of wave interactions, including reflections, transmissions, and the creation of new wave patterns at the point of interaction between wave fronts from different families. This kind of analysis is crucial in fields such as acoustics, optics, and seismology where understanding wave behavior is essential.