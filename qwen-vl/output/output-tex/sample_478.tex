To understand the relationship between the rooted network \( N^+ \) and the unrooted network \( N^- \), let's break down the definitions and the process step by step.

1. **Rooted Network \( N^+ \)**:
   - A rooted network is a type of phylogenetic network that includes a designated root node.
   - In this context, the network \( N^+ \) is defined on the set \( X \) and has a root node \( r \).
   - The root node \( r \) is typically chosen to be the least-supported ancestor (LSA) of the set \( X \). This means that \( r \) is the node in the network that is most likely to be the common ancestor of all nodes in \( X \).

2. **Unrooted Network \( N^- \)**:
   - An unrooted network is a phylogenetic network that does not have a designated root node.
   - From the rooted network \( N^+ \), we can obtain an unrooted network \( N^- \) by removing the root node \( r \) and adjusting the structure of the network accordingly.
   - The unrooted network \( N^- \) will still represent the same evolutionary relationships among the nodes in \( X \) but without the specific indication of which node is the root.

3. **Relationship Between \( N^+ \) and \( N^- \)**:
   - The unrooted network \( N^- \) is essentially the same as the rooted network \( N^+ \) except for the absence of the root node \( r \).
   - Removing the root node \( r \) from \( N^+ \) results in \( N^- \), which is a tree or a network that does not have a designated root.
   - The evolutionary relationships among the nodes in \( X \) remain the same in both networks, but the interpretation changes: \( N^+ \) indicates a specific root node, while \( N^- \) does not.

In summary, the unrooted network \( N^- \) is derived from the rooted network \( N^+ \) by removing the root node \( r \). The key difference is that \( N^- \) does not have a designated root, whereas \( N^+ \) does.

The final answer is:

\[
\boxed{N^-}
\]