To solve the problem, we need to understand the concept of the counting measure \(\mu_X\) associated with the set \(X\), which consists of four points on the edges of a square in \(\mathbb{R}^2\).

The counting measure \(\mu_X\) is defined as follows:
- For any subset \(A\) of \(\mathbb{R}^2\), \(\mu_X(A)\) is the number of points in \(X\) that lie in \(A\).
- If no point in \(X\) lies in \(A\), then \(\mu_X(A) = 0\).

Let's denote the four points in \(X\) as \(P_1, P_2, P_3,\) and \(P_4\). Since these points are on the edges of a square, each point can be thought of as lying on one of the four sides of the square.

The counting measure \(\mu_X\) will count the number of times each point appears in any given subset of \(\mathbb{R}^2\). Specifically, if a subset \(A\) intersects the square at exactly one of these points, then \(\mu_X(A) = 1\). If it intersects at two or more points, then \(\mu_X(A)\) will be the number of points in \(A\) that are in \(X\).

For example, if \(A\) is a subset of the square that contains only one of the points \(P_1, P_2, P_3,\) or \(P_4\), then \(\mu_X(A) = 1\). If \(A\) contains two of these points, say \(P_1\) and \(P_2\), then \(\mu_X(A) = 2\). If \(A\) contains all four points, then \(\mu_X(A) = 4\).

In general, for any subset \(A\) of \(\mathbb{R}^2\), \(\mu_X(A)\) is simply the number of points in \(X\) that lie in \(A\).

Therefore, the counting measure \(\mu_X\) is given by:
\[
\boxed{\mu_X(A) = \text{number of points in } X \text{ that lie in } A}
\]