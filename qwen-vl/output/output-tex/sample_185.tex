To address your statement about the convex hull of gonions and a KK tower in circle compactification, let's break down the key concepts and provide some context.

### Circle Compactification
In string theory, circle compactification is a method to reduce the number of spatial dimensions from 6 to 3 by compactifying one of the extra dimensions on a circle. This results in a Kaluza-Klein (KK) tower of states, where each state corresponds to a different radius of the compactified circle.

### Gonions and KK Tower
Gonions are specific configurations of D-branes in string theory that can be used to describe certain types of gauge theories. In the context of circle compactification, the KK tower refers to the spectrum of states that arise from the compactification of the extra dimension.

### Convex Hull
The convex hull of a set of points is the smallest convex set that contains all the points. In this case, the convex hull of gonions and the KK tower would be the smallest convex set that contains both the gonions and the KK states.

### Extremal Circle
An extremal circle in the context of circle compactification is a circle with a specific radius that is determined by the vacuum expectation value (VEV) of the dilaton field. The extremal circle is the boundary of the region where the effective action is extremized, and it plays a crucial role in determining the low-energy physics of the compactified theory.

### Perturbative Region
The perturbative region is defined by the condition \( e^{-\phi_4} = g^{-1} \geq 1 \), where \( \phi_4 \) is the dilaton field and \( g \) is the coupling constant. This condition ensures that the theory remains within the perturbative regime, where quantum corrections are small compared to the classical contributions.

### Statement Analysis
Your statement claims that the convex hull of gonions and the KK tower always contains the extremal circle when we remain in the perturbative region. This is a non-trivial assertion and would depend on the specific details of the theory, such as the form of the potential for the dilaton and the nature of the gonions.

### Mathematical Justification
To prove or disprove this statement, one would need to analyze the geometry of the space and the constraints imposed by the perturbative region. Here are some steps that could be taken:

1. **Define the Geometry**: Understand the geometry of the space after circle compactification, including the location of the extremal circle.
2. **Convex Hull Construction**: Construct the convex hull of the gonions and the KK tower.
3. **Inclusion Check**: Verify whether the extremal circle lies within the convex hull under the given conditions.

### Conclusion
Without more specific details about the theory and the exact forms of the potential and the gonions, it is not possible to definitively prove or disprove the statement. However, the statement suggests a deep connection between the geometry of the space, the properties of the gonions, and the extremal circle, which is an interesting topic for further investigation in string theory and related fields.

If you have more specific information or a reference to a particular theory, it might be possible to provide a more detailed analysis.