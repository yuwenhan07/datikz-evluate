It seems like you've provided two pieces of information about the knot \( o9\_{27767} \):

1. The Alexander polynomial of the knot.
2. A seemingly unrelated equation involving an integral and a constant.

Let's break down each piece:

### 1. Alexander Polynomial
The Alexander polynomial is a knot invariant that can be used to distinguish different knots. For the knot \( o9\_{27767} \), the Alexander polynomial is given by:
\[ t^{130} - t^{129} + t^{121} - t^{120} + t^{114} - t^{113} + t^{111} - t^{110} + t^{105} - t^{104} + t^{102} - t^{101} + t^{98} - t^{97} + t^{95} - t^{94} + t^{92} - t^{91} + t^{89} - t^{88} + t^{86} - t^{85} + t^{83} - t^{82} + t^{81} - t^{80} + t^{79} - t^{78} + t^{76} - t^{75} + t^{73} - t^{71} + t^{70} - t^{69} + t^{67} - t^{66} + t^{65} - t^{64} + t^{63} - t^{61} + t^{60} - t^{59} + t^{57} - t^{55} + t^{54} - t^{52} + t^{51} - t^{50} + t^{49} - t^{48} + t^{47} - t^{45} + t^{44} - t^{42} + t^{41} - t^{39} + t^{38} - t^{36} + t^{35} - t^{33} + t^{32} - t^{29} + t^{28} - t^{26} + t^{25} - t^{20} + t^{19} - t^{17} + t^{16} - t^{10} + t^{9} - t + 1 \]

This polynomial is a Laurent polynomial in \( t \) and provides important information about the knot, such as its genus and whether it is prime or composite.

### 2. Integral Equation
The second part of your statement involves an integral and a constant:
\[ -3\int\Upsilon = \frac{15704}{15} \]

This equation appears to be related to some kind of integral over a function \( \Upsilon \). However, without more context, it's difficult to determine what \( \Upsilon \) represents. If \( \Upsilon \) is a function of some variable, say \( x \), then this could be interpreted as:
\[ -3 \int \Upsilon(x) \, dx = \frac{15704}{15} \]

To solve for \( \Upsilon(x) \), we would need to integrate both sides with respect to \( x \):
\[ \int \Upsilon(x) \, dx = -\frac{15704}{45} \]

If \( \Upsilon(x) \) is a simple function, we might be able to identify it. However, without further information about \( \Upsilon \), it's challenging to proceed.

### Summary
- The Alexander polynomial of the knot \( o9\_{27767} \) is a complex expression that encodes topological properties of the knot.
- The integral equation involving \( \Upsilon \) is not directly related to the knot but rather seems to be a separate mathematical problem.

If you have more context or if there's a specific question about either of these, please provide additional details so I can assist you further.