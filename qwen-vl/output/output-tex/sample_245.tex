Para entender melhor a operação aplicada na sequência \( A \), vamos analisar os detalhes do problema:

- A sequência \( A \) é dada por \( A = (0~\alpha~11~9~7~5~3~\alpha~2~13) \).
- O número de \(\alpha\)s é \( n = 12 \).
- A operação aplicada é a inserção \(\phi(0, (1~4~8))\).

A operação \(\phi(i, B)\) geralmente refere-se à inserção de um conjunto de elementos \( B \) após o elemento \( i \) na sequência \( A \). No entanto, sem mais contexto sobre a função \(\phi\), vamos supor que ela realiza a inserção de \( B \) após o elemento \( i \) na sequência \( A \).

Vamos realizar a inserção passo a passo:

1. **Identificar o índice \( i \):** O índice \( i \) é dado como 0.
2. **Identificar o conjunto \( B \):** O conjunto \( B \) é dado como \( (1~4~8) \).

Agora, vamos inserir os elementos de \( B \) após o elemento \( i \):

- Originalmente, a sequência \( A \) é: \( 0~\alpha~11~9~7~5~3~\alpha~2~13 \).
- Após o elemento \( 0 \), inserimos os elementos \( 1, 4, 8 \).

Portanto, a nova sequência \( A' \) será:
\[ A' = (0~1~4~8~\alpha~11~9~7~5~3~\alpha~2~13) \]

Assim, a operação \(\phi(0, (1~4~8))\) resulta na sequência:
\[ A' = (0~1~4~8~\alpha~11~9~7~5~3~\alpha~2~13) \]

### Observação:
- O rótulo de arestas de destino de diferentes "runs" de um mesmo ciclo não foi explicitamente mencionado no problema. Se você estiver se referindo a algo específico sobre isso, por favor, forneça mais detalhes para que possamos ajudar melhor.

Se a operação \(\phi\) tem algum significado especial ou se há alguma restrição adicional, por favor, forneça mais informações para que possamos interpretar corretamente a operação.