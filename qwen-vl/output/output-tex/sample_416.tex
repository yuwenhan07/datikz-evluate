The image you've described seems to be discussing the mass spectrum of dark particles, specifically dark pions (\(\pi\)) and dark baryons, in a theoretical framework that includes a nonzero mass for the dark quarks. Let's break down the key points:

1. **Left Panel (Limit of \(m = 0\)):**
   - This panel likely shows the mass spectrum of dark pions and dark baryons when the mass of the dark quarks is set to zero.
   - In this scenario, the masses of different dark particles might not be degenerate, meaning they have distinct masses due to the absence of any mass splitting from the dark quarks.

2. **Right Panel (Nonzero Dark Quark Mass):**
   - This panel illustrates the effect of a nonzero mass for the dark quarks on the mass spectrum.
   - When the dark quarks have a nonzero mass, it can lead to mass degeneracy between certain states, such as \(\chi\) and \(\pi\).
   - The presence of a nonzero mass for the dark quarks introduces a splitting in the mass spectrum, leading to the possibility of mass degeneracy between different states.

### Key Concepts:
- **Dark Pions (\(\pi\)) and Dark Baryons:**
  - These are hypothetical particles that could exist in extensions of the Standard Model or other theoretical frameworks.
  - They are often discussed in the context of dark matter, which is believed to make up a significant portion of the universe's mass-energy content but does not interact with light directly.

- **Dark Quarks:**
  - These are hypothetical particles that could be the building blocks of dark matter particles like dark pions and dark baryons.
  - The mass of these dark quarks plays a crucial role in determining the mass spectrum of the resulting dark particles.

- **Mass Degeneracy:**
  - Mass degeneracy occurs when two or more particles have the same mass despite having different quantum numbers.
  - It can arise due to various symmetries or interactions within the theory.

### Possible Theoretical Frameworks:
- **Supersymmetry (SUSY):**
  - SUSY theories often predict the existence of dark matter candidates, including dark pions and dark baryons.
  - The mass spectrum of these particles can be influenced by the mass of the SUSY partners of ordinary quarks and leptons.

- **Beyond the Standard Model (BSM) Theories:**
  - Other BSM theories might introduce new particles and interactions that affect the mass spectrum of dark matter particles.
  - The nonzero mass of dark quarks can be a key parameter in these models.

### Conclusion:
The image highlights how the introduction of a nonzero mass for dark quarks can lead to mass degeneracy in the mass spectrum of dark pions and dark baryons. This phenomenon is important for understanding the properties of dark matter and its potential interactions within the universe. Further experimental and theoretical work is needed to confirm the existence and properties of these dark particles.