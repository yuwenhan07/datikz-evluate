It seems like you're describing a simplified Markov Decision Process (MDP) where the environment is deterministic and has a very short horizon of \( H = 1 \). Let's break down the details:

### Environment Setup:
- **Initial State**: The environment starts in state \( s_0 \).
- **Horizon**: The horizon \( H \) is set to 1, meaning the process ends after one step.
- **Actions**: There are two binary actions: \( a_0 \) and \( a_1 \).

### Transition Probabilities:
For each action, the transition probabilities are given as follows:
- If the agent takes action \( a_0 \), it moves to state \( s_1 \) with probability 1.
- If the agent takes action \( a_1 \), it moves to state \( s_1 \) with probability 1.

### Tabular Representation:
Since the horizon is only 1, the tabular representation of this MDP would be straightforward. We can represent the transition dynamics in a table that shows how the state changes based on the actions taken.

Here’s a simple tabular representation:

| Current State | Action \( a_0 \) | Action \( a_1 \) |
|---------------|------------------|------------------|
| \( s_0 \)     | \( s_1 \)         | \( s_1 \)         |

This table indicates that no matter which action is chosen, the agent will always move to state \( s_1 \) after one step.

### Summary:
- **Initial State**: \( s_0 \)
- **Actions**: \( a_0 \) and \( a_1 \)
- **Transition Probabilities**:
  - \( P(s_1 | s_0, a_0) = 1 \)
  - \( P(s_1 | s_0, a_1) = 1 \)

This setup is quite simple and deterministic, making it easy to analyze and solve using standard techniques for MDPs.