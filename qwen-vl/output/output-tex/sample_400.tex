To provide a detailed response, I need to refer to the specific example labeled as Example~\ref{ex:facefan2}. However, since you have not provided the reference or the content of that example, I will describe what such an example might typically contain and how it would be interpreted.

### Example~\ref{ex:facefan2}

In many scientific and engineering contexts, particularly in fields like fluid dynamics, heat transfer, or computational fluid dynamics (CFD), examples often involve the analysis of flow around a specific geometry, such as a face fan or a similar object. The example might present results for a particular scenario where the flow is analyzed using different boundary conditions or numerical methods.

#### Key Components of Example~\ref{ex:facefan2}

1. **Flow Geometry**: The example likely describes the geometry of the face fan, including its dimensions, shape, and any relevant features like blades or vanes.
   
2. **Boundary Conditions**: It may specify the boundary conditions applied at the inlet, outlet, and walls of the domain. For instance, the inflow might be specified as a uniform velocity profile, while the outflow could be set to zero pressure gradient.

3. **Numerical Method**: The example might detail the numerical method used to solve the governing equations, such as finite volume, finite element, or spectral methods.

4. **Results**: The results section would typically include plots or tables showing the solution variables of interest, such as velocity, pressure, temperature, or other quantities.

5. **Validation**: The example might also include a validation step where the results are compared against experimental data or analytical solutions to ensure accuracy.

6. **Analysis**: Finally, the example would analyze the results, discussing the implications of the findings and how they relate to the physical behavior of the flow.

### Results with \( H_{-v}^{max} \) and \( H_{-v}^{min} \)

The terms \( H_{-v}^{max} \) and \( H_{-v}^{min} \) suggest that the example is analyzing some quantity \( H \) related to the velocity field (\( v \)) of the flow. Here's what these terms might represent:

- \( H_{-v}^{max} \): This could denote the maximum value of the quantity \( H \) when considering the negative component of the velocity vector \( v \).
- \( H_{-v}^{min} \): This could denote the minimum value of the quantity \( H \) when considering the negative component of the velocity vector \( v \).

In the context of a face fan, this might mean analyzing the maximum and minimum values of a quantity like the negative velocity component near the fan blades, which could be important for understanding the flow separation or reattachment points.

### Solid Line vs Dashed Line

The solid line and dashed line in the plot would typically represent two different scenarios or boundary conditions:
- **Solid Line**: This might represent the result obtained under one set of boundary conditions or a specific numerical configuration.
- **Dashed Line**: This might represent the result obtained under another set of boundary conditions or a different numerical configuration.

### Conclusion

Without the actual content of Example~\ref{ex:facefan2}, it's challenging to provide a precise interpretation. However, based on typical practices in such examples, the solid line and dashed line in the plot would represent two different configurations or boundary conditions for the analysis of the flow around the face fan. The maximum and minimum values of the quantity \( H \) would help in understanding the flow behavior, particularly focusing on the negative velocity component near the fan blades.