The image you've described appears to be a graph comparing Intersection over Union (IoU) values for two sets of models over time, where one set includes an extension called "TempCoBEV" and the other uses the default model configuration. Here's a breakdown of what the graph might look like:

1. **X-axis**: Represents time or some other metric that is being measured.
2. **Y-axis**: Represents IoU values, which are typically normalized between 0 and 1, where higher values indicate better performance.

- **Solid Lines**: These represent the IoU values for the models that have been extended with TempCoBEV. The solid lines show how the IoU changes over the specified metric on the X-axis.
  
- **Dashed Lines**: These represent the IoU values for the default models without the TempCoBEV extension. The dashed lines also show how the IoU changes over the same metric as the solid lines.

- **Green and Orange Dashed Lines Overlap Heavily**: This suggests that the IoU values for the default models (dashed green and orange lines) are very similar to each other, indicating that there isn't much difference in performance between these two default configurations. The overlapping nature implies that they perform almost identically across the range of the X-axis.

### Key Observations:
- If the solid lines (with TempCoBEV) are consistently above the dashed lines, it would suggest that the extension improves the performance of the models.
- If the solid lines are below the dashed lines, it would suggest that the extension does not improve the performance or even degrades it compared to the default models.
- If the solid lines and dashed lines are close but not overlapping, it indicates that while the extension may not significantly improve performance, it still provides a noticeable difference.

Without the actual graph, it's hard to provide more specific insights, but this general description should help you understand the context and implications of the graph.