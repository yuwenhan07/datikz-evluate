The illustration you've described seems to be explaining a process related to numerical methods for solving partial differential equations, particularly focusing on the treatment of interfaces between different cells in a computational domain. Here's a breakdown of the key points:

1. **Interface and Cells**: The dashed line represents an interface between two cells \( K_i \) and \( K_j \). The interface is assumed to lie on the edge between these two cells.

2. **Species Coupling**: The species \(\mathfrak{A}\) and \(\mathfrak{B}\) are defined within the cells. The edge \(\partial K_i \cap \partial K_j\) belongs to the species \(\mathfrak{A}\), which means it is responsible for the interaction or coupling between the two cells at this interface.

3. **Empty and Full Cells**: 
   - \( K_{j,\mathfrak{A}} = \emptyset \): This indicates that the cell \( K_j \) contains no species \(\mathfrak{A}\).
   - \( K_{j,\mathfrak{B}} = K_j \): This indicates that the cell \( K_j \) is fully occupied by species \(\mathfrak{B}\).

4. **Agglomeration and Modification**:
   - The process involves modifying the discrete system algebraically.
   - The modification eliminates the contributions on the edge \(\partial K_{j,\mathfrak{A}}\) associated with the empty cell \( K_{j,\mathfrak{A}} \).
   - The cell and species coupling are combined, establishing a connection between \( K_{i,\mathfrak{A}} \) and \( K_{j,\mathfrak{B}} \).

5. **Exploded View**: The lower part of the figure provides a detailed view of the connectivity, showing how the species \(\mathfrak{A}\) interacts across the interface and how the empty cell \( K_{j,\mathfrak{A}} \) is treated during the agglomeration process.

This type of illustration is commonly used in finite volume or finite element methods where the goal is to accurately model the behavior of species across interfaces in a computational domain. The process described helps ensure that the numerical solution respects the physical properties of the system being modeled.