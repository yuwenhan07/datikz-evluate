The problem you've described involves understanding the relationship between the Bipartite Crossing Number (BCN) and the panel crossing minimization (\(\textsc{PanelCrossMin}\)) problem, as well as how different orderings can affect the number of crossings in a bipartite graph.

### Bipartite Crossing Number (BCN)
The Bipartite Crossing Number problem asks for the minimum number of edge crossings in any drawing of a bipartite graph \(G = (A, B, E)\), where \(A\) and \(B\) are the two disjoint sets of vertices, and \(E\) is the set of edges connecting vertices from \(A\) to \(B\).

### Panel Crossing Minimization (\(\textsc{PanelCrossMin}\))
The \(\textsc{PanelCrossMin}\) problem is a specific variant of the BCN problem where the vertices are placed on two parallel lines (one for each part of the bipartite graph), and the goal is to minimize the number of edge crossings while ensuring that no edge crosses itself or another edge more than once.

### Orderings and Crossings
In your description, you mention two orderings \(\tau_1\) and \(\tau_2\) leading to 3 crossings. This means that when these orderings are applied to the bipartite graph, there are 3 edge crossings in the resulting drawing. The ordering \(\sigma\) obtained from \(\tau_1\) and \(\tau_2\) with 3 forced crossings indicates that this particular ordering forces exactly 3 crossings, which might be the minimum possible under the constraints of the \(\textsc{PanelCrossMin}\) problem.

### Relationship Between BCN and \(\textsc{PanelCrossMin}\)
The \(\textsc{PanelCrossMin}\) problem is a special case of the BCN problem where the vertices are constrained to lie on two parallel lines. Therefore, the solution to the \(\textsc{PanelCrossMin}\) problem is always a valid solution to the BCN problem, but not necessarily the optimal one. The \(\textsc{PanelCrossMin}\) problem aims to find the minimum number of crossings under the given constraints, whereas the BCN problem does not have such constraints.

### Conclusion
Given the information provided:
- The bipartite graph has 3 crossings when using the orderings \(\tau_1\) and \(\tau_2\).
- The ordering \(\sigma\) obtained from \(\tau_1\) and \(\tau_2\) results in exactly 3 forced crossings, indicating that this is a valid solution to the \(\textsc{PanelCrossMin}\) problem.
- The \(\textsc{PanelCrossMin}\) problem is a subset of the BCN problem, so the 3 crossings found in \(\sigma\) are also a valid solution to the BCN problem.

Thus, the ordering \(\sigma\) provides a specific way to achieve the minimum number of crossings (3) under the constraints of the \(\textsc{PanelCrossMin}\) problem, which is also a valid solution to the BCN problem.