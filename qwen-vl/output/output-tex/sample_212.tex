It appears that you're referring to a mathematical or computational context, possibly related to fluid dynamics or gas dynamics, where Riemann solvers are used to solve hyperbolic partial differential equations. The text seems to be discussing the behavior of solutions near a "turning curve" for specific initial conditions.

Here's a breakdown of what the text might be referring to:

1. **Riemann Solvers**: These are numerical methods used to solve the Riemann problem, which is a fundamental problem in the theory of hyperbolic conservation laws. The Riemann problem involves finding the solution to the conservation law when the initial state is a discontinuity between two constant states.

2. **Turning Curve**: This likely refers to a specific point or region in the phase space where the characteristics of the solution change direction, leading to complex behavior such as shock waves or rarefaction waves.

3. **Initial Data**: The reference to initial data (\eqref{eq:riemann_data}) suggests that the solver is being applied to a specific set of initial conditions. The parameters \(\alpha\) and \(v(\rho)\) are likely related to the speed of sound or some other characteristic speed in the medium.

4. **Conditions on \(\alpha\)**:
   - When \(\alpha > v(\rho_R)\), it implies that the wave speed \(\alpha\) is greater than the speed of sound at the right state \(\rho_R\). This could lead to a rarefaction wave.
   - When \(\alpha < -v(\rho_L)\), it implies that the wave speed \(\alpha\) is less than the negative speed of sound at the left state \(\rho_L\). This could lead to a shock wave.

5. **Condition on \(\rho_L\) and \(\rho_R\)**: The condition \(\rho_L < \rho_R\) indicates that the density at the left state is less than the density at the right state. This is typical for a shock wave scenario where the density increases across the discontinuity.

Given this context, the text is likely describing how the Riemann solver behaves differently depending on the relative values of \(\alpha\) and the speed of sound at the left and right states. The solver would need to handle these different cases appropriately to accurately model the physical phenomena.

If you have more details about the equations or the specific context, please provide them so I can offer a more detailed explanation or further assistance.