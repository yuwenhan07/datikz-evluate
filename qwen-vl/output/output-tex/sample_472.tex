To apply Theorem \ref{thm:DowkerDualityTotalWeight} to the complexes from Example \ref{example:DowkerDuality}, we need to understand the relationship between the filtration of the Dowker complex \(\mathcal{D}(X, Y, R)\) by total weight and Sheehy's subdivision filtration applied to the dual Dowker complex \(\mathcal{D}(Y, X, R^\top)\).

### Step-by-Step Explanation:

1. **Dowker Complex \(\mathcal{D}(X, Y, R)\):**
   - This complex is constructed based on a matrix \(R\) where rows correspond to elements of \(X\), columns correspond to elements of \(Y\), and entries indicate some relationship (e.g., intersection or other combinatorial structure).
   - The filtration by total weight means that we consider subcomplexes of \(\mathcal{D}(X, Y, R)\) with a certain total weight threshold.

2. **Dual Dowker Complex \(\mathcal{D}(Y, X, R^\top)\):**
   - The dual Dowker complex is obtained by transposing the matrix \(R\) and considering it as a relation between elements of \(Y\) and \(X\).
   - The dual complex \(\mathcal{D}(Y, X, R^\top)\) will have its own filtration by total weight, which corresponds to a different set of subcomplexes.

3. **Sheehy's Subdivision Filtration:**
   - Sheehy's subdivision filtration is a method to refine the simplicial complex in a way that allows for efficient computation of persistent homology.
   - When applied to the dual Dowker complex \(\mathcal{D}(Y, X, R^\top)\), it results in a filtration of the subdivided complex.

4. **Theorem \ref{thm:DowkerDualityTotalWeight}:**
   - This theorem establishes a duality between the filtration of the Dowker complex \(\mathcal{D}(X, Y, R)\) by total weight and the filtration of the dual Dowker complex \(\mathcal{D}(Y, X, R^\top)\) by total weight.
   - Specifically, it states that the \(k\)-th level of the filtration of \(\mathcal{D}(X, Y, R)\) by total weight corresponds to the \(k\)-th level of the filtration of \(\mathcal{D}(Y, X, R^\top)\) by total weight.

### Application to Example \ref{example:DowkerDuality}:

Given the example, let's assume we have the following filtrations:
- Top row: Filtration of \(\mathcal{D}(X, Y, R)\) by total weight.
- Bottom row: Sheehy's subdivision filtration applied to \(\mathcal{D}(Y, X, R^\top)\).

By Theorem \ref{thm:DowkerDualityTotalWeight}, each level of the top row corresponds to a level of the bottom row. Therefore, if the top row shows the filtration of \(\mathcal{D}(X, Y, R)\) at various total weight thresholds, the bottom row will show the corresponding filtration of \(\mathcal{D}(Y, X, R^\top)\) at those same thresholds.

### Conclusion:

The application of Theorem \ref{thm:DowkerDualityTotalWeight} to the given example would reveal that the filtration of \(\mathcal{D}(X, Y, R)\) by total weight is dually related to the filtration of \(\mathcal{D}(Y, X, R^\top)\) by total weight through Sheehy's subdivision filtration. Each level of the top row corresponds to a level of the bottom row, reflecting the duality between the two filtrations.

Thus, the theorem provides a powerful tool for understanding the relationship between these filtrations and can be used to analyze and compute persistent homology more efficiently.