To provide a clear and concise answer, let's break down the problem step by step.

1. **Understanding the Context**: The problem is referring to an illustration of "X-gadgets" for all sets \( X \) that are relevant by Proposition~\ref{chemin2}. This suggests that we need to understand what these gadgets look like and how they are represented in the illustration.

2. **Vertices Representation**:
   - The vertices in set \( Z \) are marked by squares.
   - The remaining vertices (which are not in \( Z \)) are marked by disks.

3. **Illustration Description**:
   - Since the exact illustration is not provided here, we can infer that the illustration would show a graph where the vertices in \( Z \) are distinguished by being marked with squares, and the other vertices are marked with disks.
   - The structure of the gadget will depend on the specific properties of the sets \( X \) and the conditions given by Proposition~\ref{chemin2}.

4. **General Form of the Gadget**:
   - Typically, such gadgets might be used in graph theory or combinatorial problems to represent certain configurations or structures. For example, they could be used to model paths, cycles, or other specific graph patterns.
   - Without the actual illustration, we can't provide a detailed description of the gadget, but we can say that it will likely be a graph where the vertices in \( Z \) are connected in a specific way, and the remaining vertices are connected in a complementary manner.

5. **Conclusion**:
   - The final answer, based on the information provided, is that the illustration of the \( X \)-gadgets for all sets \( X \) relevant by Proposition~\ref{chemin2} would show a graph where the vertices in \( Z \) are marked by squares and the remaining vertices are marked by disks. The exact structure of the gadget will depend on the specific details of the proposition and the sets involved.

Therefore, the answer is:

\[
\boxed{\text{The illustration shows a graph where the vertices in } Z \text{ are marked by squares and the remaining vertices are marked by disks.}}
\]