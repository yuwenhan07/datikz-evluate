To compare the performance of a data stream allocation algorithm in a \gls{pcnc} (Packetized Coded Network Coding) scenario, we need to consider various metrics such as throughput, delay, packet loss rate, and energy consumption. The parameters you've provided (\( L = 10 \), \( M = 3 \), \( K = 10 \), \( N = 4 \), and \( D = 200 \) m) can be used to simulate different scenarios and evaluate the performance.

Here's a step-by-step approach to comparing the performance:

### Step 1: Define the Simulation Environment
- **\( L \)**: Number of sources.
- **\( M \)**: Number of sinks.
- **\( K \)**: Number of packets per source.
- **\( N \)**: Number of nodes in the network.
- **\( D \)**: Distance between nodes.

### Step 2: Implement the Data Stream Allocation Algorithm
- Choose an appropriate data stream allocation algorithm. This could be a greedy algorithm, a heuristic-based method, or a more complex optimization algorithm like genetic algorithms or simulated annealing.
- Ensure that the algorithm is designed to handle the \gls{pcnc} scenario effectively, considering the constraints and objectives.

### Step 3: Implement Other Comparison Methods
- Implement other relevant algorithms or methods for comparison. These could include:
  - A baseline random allocation method.
  - A fixed allocation method where each sink receives a fixed number of packets from each source.
  - An optimal allocation method based on linear programming or other mathematical optimization techniques.
  - A heuristic-based method similar to the one being evaluated but with different parameters or heuristics.

### Step 4: Simulate Different Scenarios
- Vary the parameters \( L \), \( M \), \( K \), \( N \), and \( D \) to see how the performance changes under different conditions.
- Run simulations for each method under the same conditions to ensure fair comparisons.

### Step 5: Evaluate Performance Metrics
- **Throughput**: Measure the total amount of data successfully delivered to all sinks.
- **Delay**: Calculate the average time taken for data to reach the sinks.
- **Packet Loss Rate**: Determine the percentage of packets lost during transmission.
- **Energy Consumption**: Measure the energy required by the nodes to transmit and receive data.

### Step 6: Plot the Results
- Use the results from the simulations to create plots comparing the performance of the data stream allocation algorithm against the other methods.
- For example, you might plot throughput versus delay, or packet loss rate versus energy consumption.

### Example Plot
Assume we have the following hypothetical data for simplicity:

| Method | Throughput (Mbps) | Delay (ms) | Packet Loss (%) | Energy Consumption (Joules) |
|--------|------------------|------------|----------------|-----------------------------|
| Baseline Random | 10 | 500 | 10 | 1000 |
| Fixed Allocation | 8 | 400 | 5 | 900 |
| Optimal Allocation | 12 | 350 | 2 | 1100 |
| Proposed Algorithm | 11 | 375 | 3 | 1050 |

### Plotting the Results
- **Throughput vs. Delay**: Plot throughput on the y-axis and delay on the x-axis. The proposed algorithm should ideally lie within the Pareto front, meaning it should offer a good balance between high throughput and low delay.
- **Packet Loss Rate vs. Energy Consumption**: Plot packet loss rate on the y-axis and energy consumption on the x-axis. The proposed algorithm should also aim to minimize both packet loss and energy consumption simultaneously.

### Conclusion
By comparing the performance of the proposed data stream allocation algorithm with other methods using these plots, you can determine its effectiveness in a \gls{pcnc} scenario. The goal is to identify whether the proposed algorithm outperforms others in terms of key performance indicators while meeting the design objectives of the \gls{pcnc} system.