To understand the \( E^3 \)-page of the spectral sequence for computing \( H_*(\mathcal{B}G; \mathbb{L}_\bullet\langle 1 \rangle) \), we need to break down the components and the process step by step.

### Step 1: Understanding the Spectral Sequence
A spectral sequence is a tool used in algebraic topology to compute homology or cohomology groups. The \( E^2 \)-page of the spectral sequence provides a first approximation, and subsequent pages (like the \( E^3 \)-page) refine this approximation until the final answer is reached.

### Step 2: The \( E^3 \)-Page
The \( E^3 \)-page is the third page of the spectral sequence. It is constructed from the \( E^2 \)-page using the differential \( d^2 \). The \( E^3 \)-page is given by:
\[ E^3_{p,q} = \frac{\ker(d^2_{p,q})}{\operatorname{im}(d^2_{p+2,q-1})}. \]

### Step 3: Relevant Terms and Differentials
For computing \( H_5(\mathcal{B}G; \mathbb{L}_\bullet\langle 1 \rangle) \), we need to focus on the terms that contribute to the \( p+q = 5 \) dimension. This means we look at the \( E^3 \)-page entries where \( p + q = 5 \).

### Step 4: Constructing the \( E^3 \)-Page
Let's assume the \( E^2 \)-page has been computed and we have the following non-zero terms:
\[ E^2_{0,5}, \quad E^2_{1,4}, \quad E^2_{2,3}, \quad E^2_{3,2}, \quad E^2_{4,1}, \quad E^2_{5,0}. \]

We need to determine the kernel and image of the differential \( d^2 \):
- \( d^2_{p,q} : E^2_{p,q} \to E^2_{p+2,q-1} \).

### Step 5: Computing the \( E^3 \)-Page Entries
For each \( E^2_{p,q} \) where \( p + q = 5 \), we compute:
\[ E^3_{p,q} = \frac{\ker(d^2_{p,q})}{\operatorname{im}(d^2_{p+2,q-1})}. \]

#### Example Calculation:
- For \( E^3_{0,5} \):
  - \( \ker(d^2_{0,5}) \): The kernel of the differential from \( E^2_{0,5} \).
  - \( \operatorname{im}(d^2_{2,4}) \): The image of the differential from \( E^2_{2,4} \).
  - \( E^3_{0,5} = \frac{\ker(d^2_{0,5})}{\operatorname{im}(d^2_{2,4})} \).

- For \( E^3_{1,4} \):
  - \( \ker(d^2_{1,4}) \): The kernel of the differential from \( E^2_{1,4} \).
  - \( \operatorname{im}(d^2_{3,3}) \): The image of the differential from \( E^2_{3,3} \).
  - \( E^3_{1,4} = \frac{\ker(d^2_{1,4})}{\operatorname{im}(d^2_{3,3})} \).

- And so on for all other entries.

### Step 6: Final Answer
The \( E^3 \)-page will give us the refined information about the homology groups. For \( H_5(\mathcal{B}G; \mathbb{L}_\bullet\langle 1 \rangle) \), we are particularly interested in the non-zero entries of the \( E^3 \)-page where \( p + q = 5 \).

If the \( E^3 \)-page stabilizes, then the homology group \( H_5(\mathcal{B}G; \mathbb{L}_\bullet\langle 1 \rangle) \) will be the non-zero entry in the \( E^3 \)-page corresponding to \( p + q = 5 \).

Thus, the final answer would be:
\[ H_5(\mathcal{B}G; \mathbb{L}_\bullet\langle 1 \rangle) = E^3_{p,q} \text{ where } p + q = 5 \text{ and } E^3_{p,q} \neq 0. \]

This process involves detailed calculations based on the specific differentials and terms of the spectral sequence, which can be quite complex depending on the specific algebraic structure of \( \mathcal{B}G \) and the coefficients \( \mathbb{L}_\bullet\langle 1 \rangle \).