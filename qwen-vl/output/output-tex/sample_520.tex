It seems like you're describing a scenario involving a process attempting to cross a boundary or threshold represented by a logarithmic function, which is a common concept in various fields such as physics, computer science, and mathematics.

Let's break down the problem:

1. **Layer Representation**: The layers are described using the form \((1 + ka)x \ln x\), where \(k\) and \(a\) are constants, and \(x\) is a variable representing some parameter (like time, distance, etc.). This form suggests that the layers are increasing with \(x\) and are influenced by both linear and logarithmic terms.

2. **Boundary/Threshold**: The process is trying to cross the layer \((1 + a)n \ln n\). Here, \(n\) is another variable, possibly representing a different parameter than \(x\), and the boundary is also defined by a linear term multiplied by a logarithmic term.

3. **Black Bound**: The "black bound" likely refers to a constraint or barrier that prevents the process from crossing this boundary. This could be due to physical limitations, computational constraints, or other limiting factors.

### Analysis

To understand why the process might be blocked, let's consider the behavior of these functions:

- **Layer Function**: \((1 + ka)x \ln x\)
  - As \(x\) increases, the term \(\ln x\) grows slowly, but the product \((1 + ka)x \ln x\) will grow faster if \(x\) is large enough.
  
- **Boundary Function**: \((1 + a)n \ln n\)
  - Similarly, as \(n\) increases, the term \(\ln n\) grows slowly, but the product \((1 + a)n \ln n\) will grow faster if \(n\) is large enough.

### Possible Scenarios for Blocking

1. **Relative Growth Rates**:
   - If the growth rate of \((1 + ka)x \ln x\) is slower than \((1 + a)n \ln n\) for sufficiently large values of \(x\) and \(n\), then the process will not be able to cross the boundary.
   - For example, if \(x\) and \(n\) are very large, the term \((1 + ka)x \ln x\) might not surpass \((1 + a)n \ln n\) because the linear term \((1 + a)n\) might dominate the logarithmic term \(\ln n\).

2. **Specific Values**:
   - There might be specific values of \(x\) and \(n\) at which the two functions intersect. If the process reaches a point where \((1 + ka)x \ln x = (1 + a)n \ln n\), the process would be at the boundary, and further increase in either \(x\) or \(n\) would cause one side to exceed the other.

3. **Constraints**:
   - The "black bound" could represent a hard limit imposed by external conditions. For instance, if there is a maximum value of \(x\) or \(n\) that can be practically achieved, the process might be blocked once it reaches this limit.

### Conclusion

The process is being blocked by the black bound because the boundary \((1 + a)n \ln n\) grows faster than the layer \((1 + ka)x \ln x\) for sufficiently large values of \(x\) and \(n\). This could be due to the relative dominance of the linear terms over the logarithmic terms when \(x\) and \(n\) are large. To cross this boundary, the process would need to find a way to overcome or bypass this constraint, which might require adjusting parameters, finding alternative paths, or overcoming the underlying limitations represented by the black bound.