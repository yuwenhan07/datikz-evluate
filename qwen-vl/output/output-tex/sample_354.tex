To understand the structure of the \((2,2)_3\)-ring graph, let's break it down step by step.

1. **Definition of the Graph**: The \((2,2)_3\)-ring graph is a specific type of graph where each vertex has exactly 2 edges connected to it, and there are 3 such vertices. This means we have a cycle of length 3, but with some additional properties related to the labels on the edges.

2. **Vertices and Edges**: Let's denote the vertices as \(A\), \(B\), and \(C\). Since each vertex has exactly 2 edges, and there are only 3 vertices, the graph must form a cycle. So, we can represent the graph as \(A \rightarrow B \rightarrow C \rightarrow A\).

3. **Edge Labels**: Each edge in this graph has a label. According to the problem, if no label is given at an edge, the label is equal to the target node label. Also, a bidirectional edge represents two edges in opposing directions (with not necessarily equal labels).

4. **Possible Labeling**: Since the graph is a cycle, we need to assign labels to the edges such that each vertex has exactly 2 edges connected to it. Let's denote the edges as follows:
   - Edge from \(A\) to \(B\) with label \(x\)
   - Edge from \(B\) to \(C\) with label \(y\)
   - Edge from \(C\) to \(A\) with label \(z\)

Since each vertex has exactly 2 edges, and the graph is a cycle, the labels must be consistent with the cycle. One possible labeling is:
   - Edge from \(A\) to \(B\) with label \(A\)
   - Edge from \(B\) to \(C\) with label \(B\)
   - Edge from \(C\) to \(A\) with label \(C\)

This is because each vertex \(A\), \(B\), and \(C\) has exactly one edge coming into it and one edge going out of it, and the labels follow the rule that if no label is given, it is the target node label.

Therefore, the \((2,2)_3\)-ring graph can be represented as:
\[ A \rightarrow^A B \rightarrow^B C \rightarrow^C A \]

So, the final answer is:
\[
\boxed{A \rightarrow^A B \rightarrow^B C \rightarrow^C A}
\]