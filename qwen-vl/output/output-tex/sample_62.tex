It seems like you're referring to two different examples involving numerical methods for solving partial differential equations (PDEs), specifically focusing on the evolution of observables over time and the convergence of the $\ell_2$ error of the solution $u$ at a specific time point.

### Left: Change of Observables Over Time

For Example \ref{ex: matrix, nonlinear, time-independent}, the left figure likely shows how certain observables evolve over time. This could be the evolution of the solution $u(x,t)$ or some other quantity derived from it, such as its norm, energy, or other physical quantities that are relevant to the problem being solved. The plot would typically show the observable as a function of time, illustrating how it changes over the course of the simulation.

### Right: Convergence of the $\ell_2$ Error

The right figure is related to Example \ref{ex: matrix, nonlinear, time-dependent} and \ref{ex: matrix, nonlinear, time-independent}, where the focus is on the convergence of the $\ell_2$ error of the solution $u$ at a specific time point, say $t=1$. The $\ell_2$ error is a measure of the difference between the exact solution and the approximate solution obtained using a numerical method.

The plot would show the $\ell_2$ error as a function of the number of grid points or the order of the numerical method used. As the number of grid points increases or the order of the method improves, the $\ell_2$ error should decrease, indicating better convergence of the numerical solution to the true solution.

### Summary

- **Left Figure**: Shows the evolution of observables over time for Example \ref{ex: matrix, nonlinear, time-independent}.
- **Right Figure**: Displays the convergence of the $\ell_2$ error of the solution $u$ at $t=1$ for Examples \ref{ex: matrix, nonlinear, time-dependent} and \ref{ex: matrix, nonlinear, time-independent}.

These figures are crucial for understanding the behavior of the numerical solutions and verifying the accuracy and stability of the methods used in solving the PDEs.