To understand the posets (partially ordered sets) of \(\mathcal{L}\)-classes, \(\mathcal{R}\)-classes, \(\mathcal{J}\)-classes, and \(\mathcal{H}\)-classes for the semigroup \(U_2\), we need to first define these classes and then analyze their structure.

### Definitions:
1. **\(\mathcal{L}\)-class**: The \(\mathcal{L}\)-class of an element \(a\) in a semigroup is the set of all elements that can be reached from \(a\) by a sequence of left multiplication.
2. **\(\mathcal{R}\)-class**: The \(\mathcal{R}\)-class of an element \(a\) in a semigroup is the set of all elements that can be reached from \(a\) by a sequence of right multiplication.
3. **\(\mathcal{J}\)-class**: The \(\mathcal{J}\)-class of an element \(a\) in a semigroup is the smallest equivalence relation containing both the \(\mathcal{L}\)-class and the \(\mathcal{R}\)-class of \(a\).
4. **\(\mathcal{H}\)-class**: The \(\mathcal{H}\)-class of an element \(a\) in a semigroup is the smallest equivalence relation containing both the \(\mathcal{L}\)-class and the \(\mathcal{R}\)-class of \(a\).

### Structure of \(U_2\):
The semigroup \(U_2\) consists of all upper triangular 2x2 matrices with non-negative integer entries. That is,
\[ U_2 = \left\{ \begin{pmatrix} a & b \\ c & d \end{pmatrix} \mid a, b, c, d \in \mathbb{N} \cup \{0\} \right\}. \]

### Analysis of Posets:
1. **\(\mathcal{L}\)-classes**:
   - Each \(\mathcal{L}\)-class corresponds to a specific column of the matrix.
   - For example, the \(\mathcal{L}\)-class of \(\begin{pmatrix} a & b \\ c & d \end{pmatrix}\) includes all matrices where the second column is the same as the second column of the given matrix.
   - The poset of \(\mathcal{L}\)-classes is a chain because each \(\mathcal{L}\)-class is completely ordered by inclusion.

2. **\(\mathcal{R}\)-classes**:
   - Each \(\mathcal{R}\)-class corresponds to a specific row of the matrix.
   - For example, the \(\mathcal{R}\)-class of \(\begin{pmatrix} a & b \\ c & d \end{pmatrix}\) includes all matrices where the first row is the same as the first row of the given matrix.
   - The poset of \(\mathcal{R}\)-classes is also a chain because each \(\mathcal{R}\)-class is completely ordered by inclusion.

3. **\(\mathcal{J}\)-classes**:
   - The \(\mathcal{J}\)-class of an element is the union of its \(\mathcal{L}\)-class and \(\mathcal{R}\)-class.
   - The poset of \(\mathcal{J}\)-classes is more complex but still a chain because each \(\mathcal{J}\)-class is completely ordered by inclusion.

4. **\(\mathcal{H}\)-classes**:
   - The \(\mathcal{H}\)-class of an element is the intersection of its \(\mathcal{L}\)-class and \(\mathcal{R}\)-class.
   - The poset of \(\mathcal{H}\)-classes is a chain because each \(\mathcal{H}\)-class is completely ordered by inclusion.

### Conclusion:
The posets of \(\mathcal{L}\)-classes, \(\mathcal{R}\)-classes, \(\mathcal{J}\)-classes, and \(\mathcal{H}\)-classes for the semigroup \(U_2\) are all chains. This means that each of these posets is totally ordered by inclusion.

Thus, the final answer is:
\[
\boxed{\text{The posets of } \mathcal{L}\text{-classes, } \mathcal{R}\text{-classes, } \mathcal{J}\text{-classes, and } \mathcal{H}\text{-classes are all chains.}}
\]