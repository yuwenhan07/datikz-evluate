The problem you're referring to involves the use of supersymmetric quantum mechanics (SUSY QM) and the Supersymmetric Ladder of Hamiltonians to calculate the eigenstates of the anharmonic oscillator Hamiltonian \( H_0 \). Let's break down the key components:

### 1. **Supersymmetric Quantum Mechanics (SUSY QM)**:
SUSY QM is a framework that links two Hamiltonians, one being the original Hamiltonian \( H_0 \) and the other being its partner Hamiltonian \( H_r \), where \( r \) can be 1, 2, 3, or 4 for the anharmonic oscillator case.

### 2. **Factorization Operators**:
In SUSY QM, the Hamiltonians \( H_0 \) and \( H_r \) are related through factorization operators \( A^\dagger \) and \( A \). These operators are constructed such that they satisfy the following commutation relations:
\[ [A, H_0] = -H_r \]
\[ [A^\dagger, H_r] = H_0 \]

### 3. **Anharmonic Oscillator Hamiltonian**:
The anharmonic oscillator Hamiltonian \( H_0 \) is given by:
\[ H_0 = \frac{p^2}{2m} + \frac{1}{2}m\omega^2 x^2 + \lambda x^4 \]
where \( p \) is the momentum operator, \( m \) is the mass, \( \omega \) is the angular frequency, and \( \lambda \) is the anharmonicity parameter.

### 4. **Eigenstates Calculation Using the Supersymmetric Algebraic Approach (SEA)**:
To find the eigenstates of \( H_0 \) up to \( |4\rangle_0 \), we use the SEA. This approach leverages the supersymmetric ladder operators \( A \) and \( A^\dagger \) to generate the eigenstates.

#### Step-by-Step Process:
1. **Initial State**: Start with the ground state \( |0\rangle_0 \).
2. **First Excited State**: Apply \( A^\dagger \) to \( |0\rangle_0 \):
   \[ |1\rangle_0 = A^\dagger |0\rangle_0 \]
3. **Second Excited State**: Apply \( A^\dagger \) again to \( |1\rangle_0 \):
   \[ |2\rangle_0 = A^\dagger |1\rangle_0 \]
4. **Third Excited State**: Apply \( A^\dagger \) once more to \( |2\rangle_0 \):
   \[ |3\rangle_0 = A^\dagger |2\rangle_0 \]
5. **Fourth Excited State**: Apply \( A^\dagger \) one final time to \( |3\rangle_0 \):
   \[ |4\rangle_0 = A^\dagger |3\rangle_0 \]

### 5. **Connections Between States**:
The connections between these states are established through the action of the factorization operators \( A \) and \( A^\dagger \). Specifically, the action of \( A^\dagger \) on a state \( |n\rangle_0 \) generates the next excited state \( |n+1\rangle_0 \).

### 6. **Edge State**:
The "edge state" in this context refers to the highest excited state generated by the factorization process, which in this case is \( |4\rangle_0 \). This state is the last state in the sequence of excited states generated by the factorization operators.

### Summary:
The eigenstates of the anharmonic oscillator Hamiltonian \( H_0 \) up to \( |4\rangle_0 \) are obtained by applying the factorization operator \( A^\dagger \) successively starting from the ground state \( |0\rangle_0 \). The edge state \( |4\rangle_0 \) is the highest excited state generated by this process.