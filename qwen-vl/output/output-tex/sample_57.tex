The statement you've provided refers to a significant development in theoretical physics, particularly in the context of string theory and holography. Let's break down the key components:

1. **Relativistic AdS$_5$/CFT$_4$ Correspondence**: This is a well-established duality in theoretical physics that connects a five-dimensional anti-de Sitter space (AdS$_5$) with a four-dimensional conformal field theory (CFT$_4$). It is a cornerstone of the AdS/CFT correspondence, which has profound implications for understanding quantum gravity and the nature of spacetime.

2. **Non-Relativistic Theories**: These are theories that do not exhibit the full symmetry of special relativity. They are often used to describe systems at low energies or in the presence of strong external fields where relativistic effects can be neglected.

3. **Near-Horizon/Decoupling Limit ($\alpha' \to 0$)**: In string theory, the parameter $\alpha'$ is related to the string tension and plays a crucial role in the transition from a perturbative description to a non-perturbative one. The limit $\alpha' \to 0$ corresponds to the decoupling of the string modes, effectively reducing the system to a lower-dimensional effective theory.

4. **Non-Relativistic Limit ($c \to \infty$)**: This limit is typically associated with the non-relativistic regime where the speed of light $c$ becomes very large compared to other relevant scales in the problem. This limit is often used to simplify the equations of motion and to study the behavior of systems at low velocities.

The statement "the near-horizon/decoupling limit $\alpha' \to 0$ commutes with the non-relativistic limit $c \to \infty$" means that these two limits can be applied independently without affecting each other. This is a powerful property because it allows researchers to use the insights gained from the relativistic AdS/CFT correspondence to understand non-relativistic systems as well.

In summary, the diagram you mentioned likely represents a new framework or extension of the AdS/CFT correspondence that applies to non-relativistic theories. By allowing the near-horizon/decoupling limit to commute with the non-relativistic limit, this work provides a more versatile tool for studying a broader range of physical systems, including those that are not strictly relativistic.