To understand the statement about the exponent \( r \cdot h(y, \alpha_i) \) in Theorem~\ref{thm:polygon}, let's break down the geometric configuration described:

1. **Circle and Tangent Line**: Consider a circle of radius \( r \) centered at a vertex of a polygon. A tangent line to this circle will touch the circle at exactly one point.

2. **Intersection Points**: The tangent line intersects two edges of the polygon at points that are at distances \( ry \) and \( r h(y, \alpha_i) \) from the vertex, respectively. Here, \( y \) is a parameter related to the geometry of the polygon, and \( h(y, \alpha_i) \) is a function that depends on the angle \( \alpha_i \) and the parameter \( y \).

3. **Angle Condition**: The condition \( \alpha_i < \pi/2 \) implies that the angle between the tangent line and the edge it intersects is less than 90 degrees. This is important because it affects how the tangent line cuts through the polygon.

4. **Distance Condition**: The condition \( y > \cos(\alpha_i)^{-1} \) ensures that the distance \( ry \) is greater than the distance from the vertex to the point where the tangent line intersects the first edge. This is necessary for the tangent line to intersect both edges as described.

5. **Figure Reference**: The reference to Figure~\ref{fig:small-alpha} suggests that when \( \alpha_i < \pi/2 \) and \( y > \cos(\alpha_i)^{-1} \), the geometric configuration changes, and the tangent line intersects the polygon in a specific way that is illustrated in the figure.

In summary, the exponent \( r \cdot h(y, \alpha_i) \) represents the length of the segment of the boundary of the polygon that is cut off by the tangent line, which is determined by the radius \( r \) of the circle, the parameter \( y \), and the angle \( \alpha_i \). The conditions \( \alpha_i < \pi/2 \) and \( y > \cos(\alpha_i)^{-1} \) ensure that the tangent line intersects the polygon in a specific manner, as depicted in the figure.

Thus, the final answer is:
\[
\boxed{r \cdot h(y, \alpha_i)}
\]