To address the problem, we need to understand the context and the notation used. The notation \( i, j \in I \) with \( i < j \) suggests that we are dealing with a set \( I \) and we are considering pairs \((i, j)\) where \( i \) and \( j \) are elements of \( I \) and \( i \) is less than \( j \). This implies that we are looking at all possible ordered pairs \((i, j)\) such that \( i < j \).

Let's assume \( I = \{1, 2, 3, \ldots, n\} \) for some positive integer \( n \). The number of such pairs \((i, j)\) is given by the combination formula \( \binom{n}{2} = \frac{n(n-1)}{2} \), which counts the number of ways to choose 2 elements out of \( n \) without regard to order.

For each pair \((i, j)\), we have the following configurations based on the arcs:
1. Arcs in \( A(C_i) \cup A(C_j) \) are solid.
2. Arcs from \( V(C_j) \) to \( V(C_i) \) are solid.
3. Arcs from \( V(C_i) \) to \( V(C_j) \) are dashed.

Since the problem asks for a listing of all possible configurations, we can represent each configuration as a pair \((i, j)\) along with the specified arc types. However, since the problem does not specify any particular graph or set \( C_i \) and \( C_j \), we will simply list the pairs \((i, j)\) for \( i < j \).

Here is the listing of all possible configurations:

\[
(1, 2), (1, 3), (1, 4), \ldots, (1, n), (2, 3), (2, 4), \ldots, (2, n), \ldots, (n-1, n)
\]

This list contains all pairs \((i, j)\) where \( i < j \) and \( i, j \in \{1, 2, \ldots, n\} \).

The final answer is:

\[
\boxed{(1, 2), (1, 3), (1, 4), \ldots, (1, n), (2, 3), (2, 4), \ldots, (2, n), \ldots, (n-1, n)}
\]