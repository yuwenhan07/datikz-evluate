The provided text discusses an accuracy comparison between different pretrained models, specifically focusing on their performance on English and Turkish question-answering tasks. It also highlights the phenomenon of catastrophic forgetting, which refers to the loss of previously learned knowledge when a model is trained on a new task.

### Key Points:

1. **Models Used**:
   - Hamza\(_{\scriptsize \text{Mistral}}\) and Hamza\(_{\scriptsize \text{GPT2-xl}}\)
   - Original models: Mistral 7B and GPT2-xl

2. **Ablation Study**:
   - The performance of the adapted models was evaluated by progressively increasing the pretraining corpus size from 0.1 GB to 5 GB.
   - Zero-shot and few-shot accuracies were measured on the original ARC and TruthfulQA datasets.

3. **Performance on Turkish and English Tasks**:
   - The table on the left shows the performance of the adapted models on Turkish and English tasks, alongside the original models.
   - The figure on the right demonstrates the Mistral model's performance on both Turkish and English versions of the ARC dataset.
   - The figure illustrates that the Mistral model performs better on the Turkish version of the ARC dataset but shows a decrease in performance on the English version as it continues to be pretrained.

### Summary:
The text emphasizes how the adaptation of pretrained models on Turkish improves their performance on the Turkish version of the ARC dataset but leads to a decline in performance on the English version. This phenomenon is indicative of catastrophic forgetting, where the model loses its ability to perform well on the original language after being trained on a new one.