The image you've described appears to be discussing various tunneling processes involving anyons in a non-Abelian state, specifically focusing on the transfer of an anyon charge \( e^* = e/4 \) between the top and bottom edges of a system. These processes are characterized by how they handle the associated entropy during the transfer.

Let's break down the key points:

1. **Anyon Charge Transfer**: In all four processes, the charge \( e^* \) is transferred from the top edge to the bottom edge. This indicates that these processes are designed to move a specific type of quasiparticle with a fractional charge.

2. **Entropy Transfer**: The entropy transfer varies among the processes:
   - **\( p \) Process**: This process transfers a quasiparticle with charge \( e^* \) along with its internal entropy \( s_\sigma = \log(\sqrt{2}) \) from the top to the bottom.
   - **\( ph \) Process**: Similar to the \( p \) process, this one also transfers a quasiparticle with charge \( e^* \), but it takes entropy from both the top and the bottom edges and heats the bulk. This means that the total entropy of the system increases as a result of this process.
   - **\( hp \) Process**: This process is likely symmetric to the \( ph \) process, meaning it also involves the transfer of a quasiparticle with charge \( e^* \) and takes entropy from both edges, heating the bulk.
   - **\( h \) Process**: This process might involve a different mechanism for transferring the charge \( e^* \) while managing the entropy differently, possibly without heating the bulk or taking entropy from the edges.

3. **Non-Abelian State**: The context suggests that these processes are occurring within a non-Abelian state, such as the \(\nu = 5/2\) state mentioned. Non-Abelian states are known for their rich entanglement properties and the ability to support exotic quasiparticles like anyons.

In summary, the image highlights the complexity of tunneling processes in non-Abelian systems, where the transfer of charge is accompanied by the transfer of entropy. Each process has distinct characteristics regarding how it manages the internal entropy of the quasiparticle and the overall entropy of the system. Understanding these processes is crucial for studying the dynamics of anyonic systems and their potential applications in quantum computing and condensed matter physics.