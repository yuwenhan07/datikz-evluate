The concept you're referring to seems to be related to a technique used in the analysis of random walks or Markov chains, particularly in the context of bounding the length of a walk after a certain point (splitting point \( u \)). Let's break down the key components:

1. **Reflection-Extension Technique**: This is a method used to analyze the behavior of a random walk or a Markov chain by reflecting the walk across a certain point and extending it. The idea is to use symmetry properties to simplify the analysis.

2. **Splitting Point \( u \)**: In this context, \( u \) is a specific point on the path of the walk where the walk is split into two parts. One part is reflected and extended, while the other part remains unchanged.

3. **Gray Part**: The gray part represents the last segment of the original walk before the splitting point \( u \). This segment is not modified in the reflection-extension process.

4. **Black Part**: The black part represents the modified walk after the reflection and extension. This includes the original segment up to \( u \), the reflected segment, and any additional steps added during the extension.

5. **Distance Bound**: The text mentions that the distance between \( u' \) (a point in the reflected segment) and \( g_e(u') \) (the image of \( u' \) under the reflection) is bounded by an absolute constant. This means that the reflection does not introduce a significant deviation from the original path.

6. **Negligible Increase in Length**: As the length of the original walk increases, the increase in length due to the reflection and extension becomes negligible. This is because the distance between points in the reflected segment is bounded, and thus the additional steps added during the extension do not significantly affect the overall length of the walk for large lengths.

In summary, the reflection-extension technique allows us to analyze the behavior of a random walk or Markov chain by breaking it into segments and using symmetry to simplify the analysis. The bounded distance between points in the reflected segment ensures that the increase in length due to the reflection and extension becomes negligible as the length of the original walk increases.