It seems like you're referring to a Markov chain related to a process described in Lemma \ref{lem:lower_mutant}. However, the notation and structure provided in your message are not entirely clear. Let's break down what you might be trying to convey:

1. **Markov Chain States**: You mentioned states \( S_2 = \text{Config} \), \( S_3 = \text{Config}^\star \), and \( S_4 = \text{Config}^+ \). These appear to be different configurations or states within the context of the Markov chain.

2. **Process Context**: The reference to Lemma \ref{lem:lower_mutant} suggests that this Markov chain is part of a larger discussion or proof involving some form of mutation or change in configuration.

Given these points, let's construct a simplified representation of the Markov chain based on the information provided:

### Simplified Representation of the Markov Chain

Let's denote the states as follows:
- \( S_2 = \text{Config} \)
- \( S_3 = \text{Config}^\star \)
- \( S_4 = \text{Config}^+ \)

A Markov chain can be represented by a transition matrix where each entry \( P_{ij} \) represents the probability of transitioning from state \( i \) to state \( j \).

For simplicity, let's assume the following transitions (these would need to be defined based on the specific details of the process):

\[
P = \begin{pmatrix}
p_{22} & p_{23} & p_{24} \\
p_{32} & p_{33} & p_{34} \\
p_{42} & p_{43} & p_{44}
\end{pmatrix}
\]

Where:
- \( p_{22} \): Probability of staying in state \( S_2 = \text{Config} \).
- \( p_{23} \): Probability of transitioning from \( S_2 \) to \( S_3 = \text{Config}^\star \).
- \( p_{24} \): Probability of transitioning from \( S_2 \) to \( S_4 = \text{Config}^+ \).
- \( p_{32} \): Probability of transitioning from \( S_3 \) to \( S_2 = \text{Config} \).
- \( p_{33} \): Probability of staying in state \( S_3 = \text{Config}^\star \).
- \( p_{34} \): Probability of transitioning from \( S_3 \) to \( S_4 = \text{Config}^+ \).
- \( p_{42} \): Probability of transitioning from \( S_4 \) to \( S_2 = \text{Config} \).
- \( p_{43} \): Probability of transitioning from \( S_4 \) to \( S_3 = \text{Config}^\star \).
- \( p_{44} \): Probability of staying in state \( S_4 = \text{Config}^+ \).

### Example Transition Probabilities

For illustration purposes, let's assume the following transition probabilities:

\[
P = \begin{pmatrix}
0.5 & 0.3 & 0.2 \\
0.2 & 0.5 & 0.3 \\
0.1 & 0.2 & 0.7
\end{pmatrix}
\]

This means:
- From \( S_2 \) to \( S_2 \): 50%
- From \( S_2 \) to \( S_3 \): 30%
- From \( S_2 \) to \( S_4 \): 20%
- From \( S_3 \) to \( S_2 \): 20%
- From \( S_3 \) to \( S_3 \): 50%
- From \( S_3 \) to \( S_4 \): 30%
- From \( S_4 \) to \( S_2 \): 10%
- From \( S_4 \) to \( S_3 \): 20%
- From \( S_4 \) to \( S_4 \): 70%

### Conclusion

The above transition matrix provides a basic framework for the Markov chain described. The actual values of the transition probabilities would depend on the specific details of the process being modeled in Lemma \ref{lem:lower_mutant}. If you have more specific details about the process, those could be used to refine the transition probabilities accordingly.