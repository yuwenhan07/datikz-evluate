It appears you're discussing a comparison between two methods for modeling a common rail pressure system, specifically focusing on the root mean square error (RMSE) of the model predictions. The left side of your description refers to a common rail pressure system with controllable inputs \( v_k \) and \( n_k \), and a measured output \( \psi_k \). This is likely based on research by Zimmer et al. (2018) and Tietze et al. (2014).

The right side of your description compares the performance of two approaches: entropy (red line) and your proposed approach, T-IMSPE (blue line). The diagram illustrates how the RMSE decreases over 1000 steps. The entropy method shows a slow decline in RMSE, while your T-IMSPE approach consistently and more rapidly reduces the RMSE, ultimately achieving much lower values.

Here's a structured summary:

### Left Side:
- **System**: Common rail pressure system.
- **Inputs**: \( v_k \) and \( n_k \).
- **Output**: \( \psi_k \).
- **References**: \cite{zimmer2018safe, tietze2014model}.

### Right Side:
- **Comparison**: Decline in RMSE over 1000 steps.
- **Methods**:
  - **Entropy**: Shows a slow decline in RMSE.
  - **T-IMSPE**: Consistently and more rapidly reduces RMSE, ending in much smaller values.

This comparison highlights the effectiveness of your T-IMSPE approach in improving the accuracy of the common rail pressure model compared to the entropy method.