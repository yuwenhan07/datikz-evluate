It seems like you're describing a method called Forecasting Model Search (FMS) that builds upon DyHPO's multifidelity method. Here’s a structured summary of the key points:

### Overview of Forecasting Model Search (FMS)

1. **Foundation**: FMS is built on DyHPO's multifidelity method, which is detailed in Algorithm~\ref{alg:dyhpo}.
   
2. **Novel Components**:
   - **Permutation-Invariant Graph Metanetwork (PIGMN)**: FMS uses PIGMN to featureize the model's checkpointed weights (\(\mathbf{W}\)) for input into a deep kernel Gaussian Process (GP). This step is detailed in Section~\ref{pigms}.

3. **Features from DyHPO**:
   - FMS incorporates DyHPO's features related to hyperparameter configuration, budget, and learning curve.
   - These features are described in more detail in \cite{wistuba2023supervising}.

4. **Key Benefits**:
   - **Improved Predictions**: FMS provides better predictions about hyperparameter performance across different compute budgets (refer to Table~\ref{ktau-values}).
   - **Improved Quality of Final Configuration**: The method improves the quality of the final selected configuration across various compute budgets (refer to Figure~\ref{fig:regret_over_time}).
   - **Generalization Potential**: FMS has the potential to generalize beyond the data seen during training (refer to Figure~\ref{fig:fms-transfer}).

5. **Design Choices**:
   - Specific design choices for the surrogate model are detailed in Appendix Section~\ref{meta-hyperparams}.

### References

- \cite{wistuba2023supervising}: For details on DyHPO's features used in FMS.
- Section~\ref{pigms}: For details on the use of PIGMN.
- Appendix Section~\ref{meta-hyperparams}: For specific design choices of the surrogate model.

This summary should help clarify the structure and benefits of the Forecasting Model Search (FMS) method. If you need further clarification or additional details, feel free to ask!