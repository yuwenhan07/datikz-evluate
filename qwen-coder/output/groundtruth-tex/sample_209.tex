\documentclass[10pt, regno]{amsart}
\usepackage[utf8]{inputenc}
\usepackage[T1]{fontenc}
\usepackage{amsmath,amssymb}
\usepackage{amsmath}
\usepackage{tikz}
\usetikzlibrary{positioning,matrix,arrows,decorations.pathmorphing, patterns, math, intersections, calc}

\begin{document}

\begin{tikzpicture}
            \draw[-latex] (-.5,0) -- (7,0);
            \draw (7,0) node[right] {$t$};
            \draw [-latex] (0,-.5) -- (0,5);
            
            
            \draw [domain=0:7, red ,line width=0.4mm] plot(\x,{.09*(\x)^2});
            
            \draw [domain=0:6.1, green] plot(\x,{sqrt(2.7^2-\x)-.03*\x^2});
            
            \draw [domain=0:6.1] plot(\x,{1.2*(sqrt(2.7^2-\x)-.03*\x^2)+.75});
            
            % path 1 the red one
            \path[name path=line1, domain=0:7, variable = \x] plot(\x,{.09*(\x)^2});
            % path 2 the green one
          \path[name path=line2, domain=0:6.1, variable = \x] plot(\x,{sqrt(2.7^2-\x)-.03*\x^2});
          % intersection red green 
        \path[name intersections={of = line1 and line2, by = P1}];
        \filldraw[black] (P1) circle(1.5pt);
        
        % path 3 the black one
          \path[name path=line3, domain=0:6.1, variable = \x] plot(\x,{1.2*(sqrt(2.7^2-\x)-.03*\x^2)+.75});
          % intersection black red
        \path[name intersections={of = line1 and line3, by = P2}];
        \filldraw[black] (P2) circle(1.5pt);
        
            % path 4 sarebber una linea verticale per l'intersezione tra rosso e verde
            \path[name path = line4] (P1) --++ (-90:3);
            % path 5 sarebbe l'asse delle ascisse
            \path[name path = line5] (0,0) --++ (0:11);
            % proiezione di P1 sull'asse delle ascisse
            \path[name intersections={of = line4 and line5, by = P3}];
          \draw[black, dashed] (P1) -- (P3);
          
          \draw (P1 |- 0,0) --++ (-90:0.1);
          \draw[black] (P1 |- 0,0) node[below] {$\frac{1}{h(\Omega)}$};

            % path 5 sarebbe la linea verticale per P2
            \path[name path = line6] (P2) --++ (-90:5);
            % proiezione di P4 sull'asse delle ascisse
            \path[name intersections={of = line5 and line6, by = P4}];
          \draw[black, dashed] (P2) -- (P4);
          
          \draw[black] (P4 |- 0,0) --++ (-90:0.1);
          \draw[black] (P4 |- 0,0) node[below] {$t_{g^\Omega}$};
          
          % path 7 sarebbe l'asse delle ordinate
          \path[name path = line7] (0,0) --++ (90:7);
          % intersezione tra asse verticale e linea verde
          \path[name intersections={of = line2 and line7, by = P5}];
          \draw[black] (P5) --++ (0:0.1);
          \draw[black] (P5) --++ (180:0.1);
          \draw (P5) node[left] {$|\Omega|$};
            
            % intersezione tra asse verticale e linea nera
          \path[name intersections={of = line3 and line7, by = P6}];
          \draw[black] (P6) --++ (0:0.1);
          \draw[black] (P6) --++ (180:0.1);
          \draw (P6) node[left] {$g^{\Omega}(0)$};
          
          % intersezione tra linea verde e asse orizzontale
          \path[name intersections={of = line2 and line5, by = P7}];
          \draw[black] (P7) --++ (90:0.1);
          \draw[black] (P7) --++ (-90:0.1);
          \draw (P7) node[below] {$r$};
          
            % Legend
            
            \draw[red] (8.5,3.5)--  (9,3.5);
            \draw(9,3.5) node[right] {Graph of the function $t\longmapsto \pi t^2$.};
            
            \draw (8.5,3) --  (9,3);
           \draw(9,3) node[right] {Graph of the function $t\longmapsto g^\Omega(t)$.};
            
            \draw [green] (8.5,2.5) --  (9,2.5);
            \draw(9,2.5)  node[right] {Graph of the function $t\longmapsto |\Omega_{-t}|$.};
            
            \draw (8.2,2.2) --  (8.2,3.8);
            \draw (8.2,2.2) --  (15,2.2);
            \draw (15,2.2) --  (15,3.8);
            \draw (8.2,3.8) --  (15,3.8);
        \end{tikzpicture}

\end{document}