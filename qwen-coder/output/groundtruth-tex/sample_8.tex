\documentclass{article}
\usepackage[T1]{fontenc}
\usepackage{tikz}
\usepackage[showframe]{geometry}
\begin{document}
\begin{figure}
    \centering %better than {center} environment, no vertical spurious space
        \fbox{\begin{tikzpicture}[scale = 0.4]
            \def\nx{12}
            \def\ny{12}
            \foreach \i in {0, ..., \nx}{
                \path (0:13+\i) -- ++ (-60:\i) -- ++ (-120:\i) coordinate (f\i) circle (2pt) node[right] {\small$f_{\i}$}; %node[right] {\i};
            }
            \foreach \j in {0, ..., \ny}{
                \fill (0:11-\j) ++ (-60:\j) ++ (-120:\j) coordinate (e\j) circle (2pt) node[left]{\small$e_{\j}$}; % node[left] (e\j) {\j};
            }
            \draw (f0) -- (f12);
            \draw (e0) -- (e12);
            %% take responsibility for the bounding box
            \pgfresetboundingbox
            \draw [use as bounding box]
                ([shift={(-2,2)}] e12 |- f0) rectangle ([shift={(2,-2)}] f12);
            %% Notice the use of the perpendicular coordinate system here
            %% and a shift added to avoid clipping labels.
    \end{tikzpicture}}
\end{figure}
\end{document}