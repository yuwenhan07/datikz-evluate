\documentclass[11pt]{book}
\usepackage[paperwidth=20cm, paperheight=29cm]{geometry}
\usepackage[T1]{fontenc}
\usepackage[french]{babel}
\usepackage{amsmath, amsfonts, amsthm, amssymb}
\usepackage{multicol}
\usepackage{qtree}
\usepackage{xcolor}
\usepackage{caption}
\usepackage{parskip}
\usepackage{hyperref}
\usepackage{listings}
\usepackage{tikz}

\usetikzlibrary{arrows.meta,arrows,chains,automata,shapes,matrix,positioning,scopes,calc,babel}
\usepackage{tikz-qtree}

\definecolor{light-gray}{gray}{0.9}
\newcommand{\code}[1]{\colorbox{light-gray}{\texttt{#1}}}
\newcommand{\phase}[1]{\par\medskip\textbf{#1}\par\nopagebreak}

\RequirePackage[framemethod=default]{mdframed} % Required for creating the theorem, definition, exercise and corollary boxes

\definecolor{ocre}{RGB}{70,50,212} % Define the orange color used for highlighting throughout the book

\newmdenv[skipabove=7pt,
skipbelow=7pt,
rightline=false,
leftline=true,
topline=false,
bottomline=false,
backgroundcolor=ocre!10,
linecolor=ocre,
innerleftmargin=5pt,
innerrightmargin=5pt,
innertopmargin=5pt,
innerbottommargin=5pt,
leftmargin=0cm,
rightmargin=0cm,
linewidth=4pt]{eBox}

\renewcommand{\qedsymbol}{$\blacksquare$}

\newtheorem{exempleT}{Exemple}[chapter]
\newenvironment{exemple}% environment name
{
\par
    \begin{exempleT}
        %\par\vspace{\baselineskip}\noindent
        %\hfill{\tiny\ensuremath{\blacksquare}}
        \par\vspace{\baselineskip}\noindent
        \tiny\ensuremath{\blacksquare}
        \par\vspace{\baselineskip}\noindent\ignorespaces
        %\ignorespacesafterend
    \end{exempleT}
}

\newtheorem{exerciceT}{Exercice}[chapter]
\newenvironment{exercice}% environment name
{
\par
    \begin{eBox}\begin{exerciceT}
        %\par\vspace{\baselineskip}\noindent
        %\hfill{\tiny\ensuremath{\blacksquare}}
        %\par\vspace{\baselineskip}\noindent
        \hfill{\color{ocre}\tiny\ensuremath{\blacksquare}}
        \par\vspace{\baselineskip}\noindent\ignorespaces
        %\ignorespacesafterend
    \end{exerciceT}\end{eBox}
}

\begin{document}
            \begin{exercice}\label{exercice:deterministeVersNonDeterministe}
                Transformez l'automate non déterministe \code{A = (Q,$\Sigma,\delta$,s,F)} de la figure~\ref{exemple:afnd} en un automate fini déterministe \code{A' = (Q',$\Sigma',\delta'$,s',F')}.\\

                Construction de l'automate fini déterministe \code{A'}:\\

                \begin{enumerate}
                    \item Si \code{Q} $= \{1, 2, 3\}$ est l'ensemble des états de l'automate \code{A} alors:\\
                                        \code{Q'} est l'ensemble des sous ensembles de \code{Q}.\\
                                        \code{Q'} $= \{\varnothing, \{1\}, \{2\}, \{3\}, \{1,2\}, \{1,3\}, \{2,3\}, \{1,2,3\} \}$.\\
                    \item $\Sigma$ ne change pas. Donc $\Sigma' = \Sigma$.
                    \item Si $\delta$ est la fonction de transition de \code{A}, alors $\delta'$ est la fonction de transition de \code{A'}. $\delta'$ est la fonction de transition suivante, et est un sous ensemble de \code{Q'} $\times \Sigma \times$ \code{Q'}:
                        \begin{multicols}{2}
                            $\delta = \{\\
                            \delta(\varnothing, a)=\varnothing$\\           
                            $\delta(\{1\}, a)=\varnothing$\\
                            $\delta(\{2\}, a)=\varnothing$\\
                            $\delta(\{3\}, a)=\{1,2\}$\\
                            $\delta(\{1,2\}, a)=\varnothing$\\
                            $\delta(\{1,3\},a)=\{1,2\}$\\
                            $\delta(\{2,3\},a)=\{1,2\}$\\
                            $\delta(\{1,2,3\}, a)=\{1,2\}$

                        \columnbreak

                        \vline \hspace{1cm}

                            $\delta(\{\varnothing,b\})=\varnothing$\\
                            $\delta(\{1\},b)=\{2,3\}$\\
                            $\delta(\{2\},b)=\varnothing$\\
                            $\delta(\{3\},b)=\varnothing$\\
                            $\delta(\{1,2\},b)=\{2,3\}$\\
                            $\delta(\{1,3\},b)=\{2,3\}$\\
                            $\delta(\{2,3\},b)=\varnothing$\\
                            $\delta(\{1,2,3\},b)=\{2,3\}\}$
                        \end{multicols}

                    \item \code{s} ne change pas. \code{s} étant l'état initial de \code{A}, il demeure l'état initial de \code{A'}. Si \code{s'} est l'état initial de \code{A'}, alors \code{s' = s = 1}.
                    \item \code{F} étant l'ensemble des états finaux de \code{A} alors \code{F'} est l'ensemble des états finaux de \code{A'}.
                        Les états finaux du nouvel automate \code{A'} sont tous des sous ensembles des états contenant l'état $3$.

                        Donc \code{F'} $= \{\{3\},\{1,3\},\{2,3\},\{1,2,3\}\}$\\
                        D'où le diagramme des transitions suivant:
                        \begin{center}
                            \begin{tikzpicture}[>=Stealth, shorten >=1pt, auto, node distance=2.5cm, initial text = Début]
                                \node[initial, state](4){$1$};
                                \node[state](1)[above left=of 4]{$2$};
                                \node[state](2)[right=of 1]{$\varnothing$};
                                \node[state, accepting](3)[right=of 2]{$3$};
                                \node[state, accepting](5)[right=of 4]{${\{2, 3\}}$};
                                \node[state](6)[right of=5]{${\{1, 2\}}$};
                                \node[state, accepting](7)[below=of 5]{${\{1, 3\}}$};
                                \node[state, accepting](8)[below=of 6]{${\{1, 2, 3\}}$};

                                \path[->] (1) edge node {a}   (2);
                                \path[->] (1) edge [swap] node {b}   (2);

                                \path[->] (2) edge [loop above] node {a, b}   (2);

                                \path[->] (3) edge node {b}   (2);
                                \path[->] (3) edge node {a}   (6);

                                \path[->] (4) edge node {a}   (2);
                                \path[->] (4) edge node {b}   (5);
                                \path[->] (5) edge node {b}   (2);
                                \path[->] ({5}) edge node {a}   (6);
                                \path[->] ({6}) edge node {b}   (5);
                                \path[->] ({6}) edge node {b}   (2);

                                \path[->] (7) edge node {b}   (5);
                                \path[->] (7) edge node {a}   (6);

                                \path[->] (8) edge node {a}   (6);
                                \path[->] (8) edge node {b}   (5);
                            \end{tikzpicture}
                        \end{center}

                        Si on supprime les états qui ne peuvent être atteints, on obtient l'AFD suivant:

                    \begin{tikzpicture}[>=Stealth, shorten >=1pt, auto, node distance=3cm, initial text = Début]
                        \node[initial, state](A){$\{1\}$};
                        \node[state](B)[above right=of A]{$\varnothing$};
                        \node[state, accepting](C)[below of=B]{$\{2, 3\}$};
                        \node[state](D)[right of=C]{$\{1, 2\}$};

                        \path[->] (A) edge node {a}   (B);
                        \path[->] (B) edge [loop above] node {a, b}   (B);
                        \path[->] (A) edge node {b}  (C);
                        \path[->] (C) edge [bend right] node {a}   (D);
                        \path[->] (C) edge node {b}   (B);
                        \path[->] (D) edge [bend right] node {b}   (C);
                        \path[->] (D) edge node {a}   (B);
                    \end{tikzpicture}

                    Le langage que cet automate accepte est le suivant:\\

                    \begin{center}
                        $\{b(ab)^n$ : $n \in \mathbb{N} \, \}$
                    \end{center}
                \end{enumerate}
            \end{exercice}
\end{document}