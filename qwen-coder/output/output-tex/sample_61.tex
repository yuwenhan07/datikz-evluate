To represent the given combinatorial problem instance using TikZ in LaTeX, we can create a diagram that includes value-vertices, variable-vertices, constraint-vertices, and model-vertex. Below is the TikZ code to generate this diagram:

```latex
\documentclass[tikz,border=5mm]{standalone}
\usetikzlibrary{shapes.geometric, arrows.meta}

\tikzset{
    value/.style={draw, fill=green!30, circle, minimum size=1em},
    variable/.style={draw, fill=red!30, rectangle, minimum size=1em},
    constraint/.style={draw, fill=blue!30, ellipse, minimum size=1.5em},
    model/.style={draw, fill=yellow!30, diamond, minimum size=1.5em},
    operator/.style={draw, fill=orange!30, circle, inner sep=1pt},
    edge/.style={->, thick}
}

\begin{document}
\begin{tikzpicture}[node distance=2cm]

    % Value vertices
    \node[value] (v1) at (0,0) {1};
    \node[value] (v2) at (2,0) {2};

    % Variable vertices
    \node[variable] (x1) at (-1,-1) {$x_1$};
    \node[variable] (x2) at (1,-1) {$x_2$};

    % Constraint vertices
    \node[constraint] (c1) at (0,-2) {$\leq$};
    \node[constraint] (c2) at (2,-2) {\texttt{ext}};

    % Model vertex
    \node[model] (m) at (0,-3) {};

    % Operators
    \node[operator] (op1) at (-1,-2.5) {$\times$};
    \node[operator] (op2) at (1,-2.5) {$\times$};

    % Edges
    \draw[edge] (v1) -- node[left] {1} (x1);
    \draw[edge] (v2) -- node[right] {2} (x1);
    \draw[edge] (x2) -- node[right] {2} (op2);
    \draw[edge] (op1