Certainly! Below is a TikZ LaTeX code that generates a diagram illustrating the relationship between a bipartite graph for the \textsc{Bipartite Crossing Number} problem and its corresponding instance of \panelcrossmin.

```latex
\documentclass[tikz,border=5mm]{standalone}
\usetikzlibrary{shapes,arrows.meta}

\tikzset{
    node distance=2cm,
    vertex/.style={circle, draw, fill=blue!20},
    edge/.style={thick,->,>=Stealth},
    panel/.style={rectangle, draw, fill=green!20, minimum width=4cm, minimum height=3cm},
}

\begin{document}

\begin{tikzpicture}[scale=0.8]

% Bipartite Graph for Bipartite Crossing Number
\node (v1) at (-3,2) [vertex] {v1};
\node (v2) at (-3,0) [vertex] {v2};
\node (v3) at (-1,2) [vertex] {v3};
\node (v4) at (-1,0) [vertex] {v4};
\node (v5) at (1,2) [vertex] {v5};
\node (v6) at (1,0) [vertex] {v6};

\draw[edge] (v1) -- (v3);
\draw[edge] (v1) -- (v5);
\draw[edge] (v2) -- (v4);
\draw[edge] (v2) -- (v6);
\draw[edge] (v3) -- (v5);
\draw[edge] (v4) -- (v6);

\node[above=of v1] {$\tau_1$:};
\node[below=of v2] {$\tau_2$:};
\node[left=of v3] {$v_3$};
\node[right=of v5] {$v_5$};
\node[left=of v4] {$v_4$};
\node[right=of v6] {$v_6$};

% Panel for \panelcrossmin
\node (panel) at (4,-0.5) [panel] {};
\node[above=of panel] {\panelcrossmin{}:};
\node[below=of panel] {$\sigma$:};

%