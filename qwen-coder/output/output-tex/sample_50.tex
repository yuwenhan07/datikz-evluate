To illustrate multidimensional consistency in a TikZ LaTeX diagram, we can create a visual representation of the three different routes to calculate \( u_{ijk} \) from initial values \( u, u_i, u_j, u_k \). Each route will be represented by an arrow leading to the final value \( u_{ijk} \).

Here is the TikZ LaTeX code to generate this diagram:

```latex
\documentclass{standalone}
\usepackage{tikz}

\begin{document}

\begin{tikzpicture}[node distance=2cm]
    % Nodes
    \node (initial) [circle, draw, fill=blue!30, label=left:$u$] {};
    \node (i) [circle, draw, fill=red!30, right of=initial, xshift=1cm, label=right:$u_i$] {};
    \node (j) [circle, draw, fill=green!30, below of=i, yshift=-1cm, label=below:$u_j$] {};
    \node (k) [circle, draw, fill=purple!30, above of=j, yshift=1cm, label=above:$u_k$] {};
    \node (ijk) [circle, draw, fill=orange!30, right of=j, xshift=2cm, label=right:$u_{ijk}$] {};

    % Arrows for the first route: u -> i -> j -> k -> u_ijk
    \draw[->, thick] (initial) -- node[midway, above] {Route 1} (i);
    \draw[->, thick] (i) -- node[midway, above] {} (j);
    \draw[->, thick] (j) -- node[midway, above] {} (k);
    \draw[->, thick] (k) -- node[midway, above] {} (ijk);

    % Arrows for the second route: u -> j -> k -> u_ijk
    \draw[->, thick, dashed] (initial) -- node[midway, below] {Route 2} (j);
    \draw[->, thick, dashed] (j) -- node[midway, below] {} (k);
    \draw[->, thick, dashed] (k) -- node[midway, below] {} (ijk);

    % Arrows for the third