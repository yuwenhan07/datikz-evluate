To create a TikZ diagram that represents the described scenario, we need to visualize two $b$-cycles on the left and their product on the right. Below is the TikZ code to achieve this:

```latex
\documentclass{standalone}
\usepackage{tikz}

\begin{document}

\begin{tikzpicture}[node distance=2cm]

% Left side
\node (start1) [circle, draw, fill=white] {0};
\node (a1) [circle, draw, fill=white, below of=start1] {$\alpha$};
\node (b1) [circle, draw, fill=green!30, right of=a1] {1};
\node (c1) [circle, draw, fill=green!30, above of=b1] {1};

\draw[->] (start1) -- node[left] {0} (a1);
\draw[->] (a1) -- node[right] {0} (b1);
\draw[->] (b1) -- node[above] {1} (c1);
\draw[->] (c1) -- node[below] {1} (start1);

% Right side
\node (start2) [circle, draw, fill=white, xshift=5cm] {0};
\node (a2) [circle, draw, fill=white, below of=start2] {$\alpha$};
\node (b2) [circle, draw, fill=green!30, right of=a2] {1};
\node (c2) [circle, draw, fill=green!30, above of=b2] {1};
\node (d2) [circle, draw, fill=green!30, right of=c2] {1};
\node (e2) [circle, draw, fill=green!30, above of=d2] {1};
\node (f2) [circle, draw, fill=green!30, right of=e2] {1};
\node (g2) [circle, draw, fill=green!30, below of=f2] {1};

\draw[->] (start2) -- node[left] {0} (a2);
\draw[->] (a2) -- node[right] {0} (b2);
\draw[