Certainly! Below is a TikZ LaTeX code that creates a conceptual map demonstrating the context of the electron-mirror duality as the lower leg of a thermal-particle-creation triangle. This map includes nodes representing the electron, mirror, black hole, and various concepts related to their interactions.

```latex
\documentclass{standalone}
\usepackage{tikz}
\usetikzlibrary{shapes.geometric, arrows.meta}

\tikzset{
    node distance=2cm,
    start chain/.style={start chain=#1 going below},
    box/.style={draw, rectangle, minimum width=4cm, minimum height=1cm, align=center, font=\small},
    arrow/.style={->, thick, >=Stealth[length=5mm]},
}

\begin{document}
\begin{tikzpicture}[node distance=2cm]
    % Nodes
    \node[box, on chain=A] (electron) {Electron};
    \node[box, on chain=A] (mirror) {Mirror};
    \node[box, on chain=B, join=by {arrow}] (black_hole) {Black Hole};
    \node[box, on chain=C, join=by {arrow}] (thermodynamics) {Thermodynamics};

    % Connections
    \draw[arrow] (electron.east) -- ++(1.5,0) |- (mirror.west);
    \draw[arrow] (mirror.east) -- ++(1.5,0) |- (black_hole.west);
    \draw[arrow] (black_hole.east) -- ++(1.5,0) |- (thermodynamics.west);

    % Labels
    \node[above=0.5cm of electron] {Classical/Quantum Model};
    \node[above=0.5cm of mirror] {Analog for Black Hole Radiation};
    \node[above=0.5cm of black_hole] {Link Between Experiments};
    \node[above=0.5cm of thermodynamics] {Relativistic Thermodynamics};

    % Additional notes
    \node[below right=0.5cm and 2cm of electron] {Electron Results};
    \node[below right=0.5cm and 2cm of mirror] {Half-Century History};
    \node[below right=0.5cm and 2cm of