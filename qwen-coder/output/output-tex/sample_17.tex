Certainly! Below is a TikZ LaTeX code that constructs a short odd cycle containing \( v \) in the case \( d = 2 \), where both \( v \) and its blue neighbor \( v' \) lie outside \( Q_{2r} \). The highlighted cells are used to construct paths from \( v \) and \( v' \) to \( q'_{t'} \), and an application of Claim \ref{claim:paths} allows closing these paths into a cycle.

```latex
\documentclass[tikz,border=10pt]{standalone}
\usetikzlibrary{shapes, arrows.meta}

\begin{document}

\tikzset{
    vertex/.style={circle, draw, fill=blue!30, inner sep=5pt},
    edge/.style={->, thick},
    highlight/.style={thick, dashed, red}
}

\begin{tikzpicture}[scale=1.5]

% Define the vertices
\node[vertex] (v) at (0,0) {\( v \)};
\node[vertex] (v') at (2,0) {\( v' \)};
\node[vertex] (q') at (1,-1) {\( q'_{t'} \)};
\node[vertex] (p1) at (-1,-1) {\( p_1 \)};
\node[vertex] (p2) at (3,-1) {\( p_2 \)};
\node[vertex] (q1) at (0,-2) {\( q_1 \)};
\node[vertex] (q2) at (2,-2) {\( q_2 \)};

% Draw the edges
\draw[edge] (v) -- (v');
\draw[edge] (v) -- (q1);
\draw[edge] (v) -- (q2);
\draw[edge] (v') -- (q1);
\draw[edge] (v') -- (q2);
\draw[edge] (q1) -- (q2);
\draw[edge] (q1) -- (q');
\draw[edge] (q2) -- (q');

% Highlight the path from v to q'_{t'}
\draw[highlight] (v) -- (q1) -- (q2) -- (q');

% Highlight the path from v' to q'_{t'}
\draw[highlight] (v') -- (q1) --