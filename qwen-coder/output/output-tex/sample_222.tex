Certainly! Below is a TikZ LaTeX code that represents two permutations \(\sigma_0\) and \(\sigma_1\) on twelve elements, with \(\sigma_0\) shown in cycle notation and \(\sigma_1\) shown using arrows.

```latex
\documentclass[tikz]{standalone}
\usetikzlibrary{arrows.meta}

\tikzset{
    node distance=2cm,
    every node/.style={circle, draw, fill=blue!30, minimum size=1cm},
    arrow style/.style={->, thick, >=Stealth[length=5pt]}
}

\begin{document}
\begin{tikzpicture}[x=1cm, y=1cm]

% Nodes for permutation sigma_0
\node (1) at (0,0) {1};
\node (2) at (1,0) {2};
\node (3) at (2,0) {3};
\node (4) at (0,-1) {4};
\node (5) at (1,-1) {5};
\node (6) at (2,-1) {6};
\node (7) at (0,-2) {7};
\node (8) at (1,-2) {8};
\node (9) at (2,-2) {9};
\node (10) at (0,-3) {10};
\node (11) at (1,-3) {11};
\node (12) at (2,-3) {12};

% Cycle notation for sigma_0
\draw[thick] (1) edge[bend right] (2);
\draw[thick] (2) edge[bend right] (3);
\draw[thick] (3) edge[bend right] (1);

\draw[thick] (4) edge[bend left] (5);
\draw[thick] (5) edge[bend left] (6);
\draw[thick] (6) edge[bend left] (4);

\draw[thick] (7) edge[bend right] (8);
\draw[thick] (8) edge[bend right] (9);
\draw[thick] (9) edge[bend right] (7);

\draw[thick] (10) edge[bend left] (1