To create a TikZ diagram representing a cluster \(\mathcal{G}(P)\) with the given properties, we need to carefully arrange nodes such that the total number of nodes is 11 and the perimeter is 19. The perimeter of a cluster is calculated by counting the external boundary edges.

Here's a step-by-step approach to constructing this cluster:

1. **Determine the Shape**: Given the perimeter of 19, we can infer that the shape is likely complex but relatively compact. A common approach is to use a combination of squares and rectangles.

2. **Calculate Node Positions**: We need to place nodes in such a way that they form a connected cluster with exactly 11 nodes and a perimeter of 19.

3. **Draw the Diagram**: Using TikZ, we will draw the nodes and edges to represent the cluster.

Below is the TikZ code for the described cluster:

```latex
\documentclass[tikz,border=10pt]{standalone}
\begin{document}
\begin{tikzpicture}[node distance=1cm]
    % Define nodes
    \node (a1) at (0,0) {};
    \node (a2) at (1,0) {};
    \node (a3) at (2,0) {};
    \node (a4) at (3,0) {};
    \node (a5) at (4,0) {};
    \node (b1) at (0,1) {};
    \node (b2) at (1,1) {};
    \node (b3) at (2,1) {};
    \node (c1) at (0,2) {};
    \node (c2) at (1,2) {};

    % Draw edges
    \draw (a1) -- (a2);
    \draw (a2) -- (a3);
    \draw (a3) -- (a4);
    \draw (a4) -- (a5);
    \draw (a5) -- (b5);
    \draw (b5) -- (c5);
    \draw (c5) -- (c4);
    \draw (c4) -- (c3);
    \draw (c3) -- (c2);
    \draw (c2) -- (b2);
    \draw (b2) -- (a2);
    \draw (a