Certainly! Below is a TikZ LaTeX code that represents the process of extracting the tuple \((P, X, s, Y)\) and the injective word \(a\) from a given blob diagram. The blocks in \(P\), and the labelings of \(Y\) are indicated at their lowermost nodes.

```latex
\documentclass[tikz,border=10pt]{standalone}
\usetikzlibrary{shapes.geometric, arrows.meta}

\tikzset{
    block/.style={rectangle, draw=black, fill=white!30, text width=4em, text centered, rounded corners, minimum height=2em},
    line/.style={->, thick, >=stealth'}
}

\begin{document}
\begin{tikzpicture}[node distance=2cm]

    % Nodes for P
    \node [block] (P1) {Block 1};
    \node [block, below=of P1] (P2) {Block 2};
    \node [block, below=of P2] (P3) {Block 3};

    % Nodes for X
    \node [block, right=of P1, xshift=3cm] (X1) {Block A};
    \node [block, below=of X1] (X2) {Block B};
    \node [block, below=of X2] (X3) {Block C};

    % Node for s
    \node [block, above=of P1, yshift=3cm] (s) {State S};

    % Nodes for Y
    \node [block, left=of P1, xshift=-3cm] (Y1) {Label 1};
    \node [block, below=of Y1] (Y2) {Label 2};
    \node [block, below=of Y2] (Y3) {Label 3};

    % Arrows
    \draw [line] (P1) -- (P2);
    \draw [line] (P2) -- (P3);
    \draw [line] (X1) -- (X2);
    \draw [line] (X2) -- (X3);
    \draw [line] (s) -- (P1);
    \draw [line] (Y1) -- (Y2);
    \draw [line]