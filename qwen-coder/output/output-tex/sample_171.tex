To create a TikZ diagram that represents the construction step you described, we need to carefully depict the blocks as cycles, label them appropriately, and color them according to the given rules. Below is the TikZ code that accomplishes this:

```latex
\documentclass[tikz,border=2mm]{standalone}
\usetikzlibrary{shapes.geometric, positioning}

\tikzset{
    block/.style={draw,circle,inner sep=3pt},
    thickblock/.style={draw,circle,inner sep=3pt,thick},
    blueblock/.style={draw,circle,inner sep=3pt,blue},
    orangeblock/.style={draw,circle,inner sep=3pt,orange},
    candidate/.style={draw,dashed,circle,inner sep=3pt,red}
}

\begin{document}
\begin{tikzpicture}[node distance=2cm]
    % Define nodes
    \node[block] (A1) {};
    \node[block] (A2) [right of=A1] {};
    \node[block] (A3) [right of=A2] {};
    \node[thickblock, blueblock] (B1) [below right of=A3] {};
    \node[thickblock, blueblock] (B2) [below left of=B1] {};
    \node[thickblock, blueblock] (B3) [below left of=A1] {};
    \node[thickblock, blueblock] (B4) [below left of=A2] {};
    \node[thickblock, orangeblock] (B5) [below right of=B4] {};
    
    % Draw edges (connections)
    \draw (A1) -- (A2);
    \draw (A2) -- (A3);
    \draw (A3) -- (B1);
    \draw (B1) -- (B2);
    \draw (B2) -- (B3);
    \draw (B3) -- (B4);
    \draw (B4) -- (B5);
    \draw (B5) -- (B1);
    
    % Label nodes
    \node[left of=A1] {A};
    \node[right of=B1] {B};
    \node[above right of=B5] {P};
    
    % Candidates for A4
    \node[candidate