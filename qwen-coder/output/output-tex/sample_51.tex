To create a TikZ diagram representing an \( mn \)-coloring of \((S_{1,m} \square S_{1,n})_\delta\), we need to understand the structure of the grid graph \( S_{1,m} \square S_{1,n} \) and its dual graph \((S_{1,m} \square S_{1,n})_\delta\). Here, \( S_{1,m} \) is a path graph with \( m \) vertices and \( S_{1,n} \) is a path graph with \( n \) vertices.

The dual graph \((S_{1,m} \square S_{1,n})_\delta\) consists of faces (which are squares for each edge in the original grid) and their dual vertices (which are the centers of these squares).

Let's assume \( m = 3 \) and \( n = 2 \) for simplicity. This means we have a grid graph \( S_{1,3} \square S_{1,2} \) which is a \( 3 \times 2 \) grid. The dual graph \((S_{1,3} \square S_{1,2})_\delta\) will be a \( 2 \times 3 \) grid of faces.

Here is the TikZ code to draw this:

```latex
\documentclass{standalone}
\usepackage{tikz}

\begin{document}
\begin{tikzpicture}[scale=1]
    % Draw the grid
    \foreach \x in {0,1,2} {
        \draw (\x,0) -- (\x,2);
    }
    \foreach \y in {0,1,2} {
        \draw (0,\y) -- (2,\y);
    }

    % Label the vertices
    \node at (0.5,0.5) [circle, fill, inner sep=2pt] {};
    \node at (0.5,1.5) [circle, fill, inner sep=2pt] {};
    \node at (1.5,0.5) [circle, fill, inner sep=2pt] {};
    \node at (1.5,1.5) [circle, fill, inner sep=2pt] {};
    \node at (2.5,0.5) [circle, fill, inner sep=2pt] {};
    \node at (