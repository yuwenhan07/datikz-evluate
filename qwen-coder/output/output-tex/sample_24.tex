Certainly! Below is the TikZ LaTeX code to illustrate the online NDWT (Non-Decimated Wavelet Transform) pyramid algorithm for \( J = 3 \) and \( W = 2 \). The input time series is shown in the top row, and the coefficients with dotted borders indicate those obtained through constant-end extension. Dashed arrows represent filtering operations with \( \mathcal{G} \), and solid arrows represent filtering operations with \( \mathcal{H} \).

```latex
\documentclass[tikz,border=10pt]{standalone}
\usetikzlibrary{matrix,shapes.geometric,arrows.meta}

\tikzset{
    mybox/.style={draw, rectangle, minimum width=2cm, minimum height=1cm},
    myarrow/.style={->, thick},
    dottedarrow/.style={->, dashed, thick},
    constantendextension/.style={dotted, draw}
}

\begin{document}
\begin{tikzpicture}[node distance=1.5cm]

% Time series input
\node[mybox] (input) {Input Time Series};
\node[mybox, below=of input] (level1) {};
\node[mybox, below=of level1] (level2) {};
\node[mybox, below=of level2] (level3) {};

% Level 1
\node[mybox, right=of input] (h1) {H};
\node[mybox, above=of h1, yshift=-0.5cm] (g1) {G};
\node[dottedarrow, from=g1.west, to=h1.west] {};
\node[dottedarrow, from=g1.east, to=h1.east] {};
\node[dottedarrow, from=g1.south, to=h1.south] {};
\node[dottedarrow, from=g1.north, to=h1.north] {};
\node[dottedarrow, from=h1.west, to=input.west] {};
\node[dottedarrow, from=h1.east, to=input.east] {};

% Level 2
\node[mybox, right=of h1] (h2) {H};
\node[mybox, above=of h2, yshift=-0.5cm] (g2) {G};
\node[dottedarrow, from=g2.west, to=h