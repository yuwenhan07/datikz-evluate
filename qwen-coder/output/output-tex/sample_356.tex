To create a TikZ LaTeX diagram for the described graph \(G_D\) and its square \(G_D^2\), we need to carefully place the vertices and edges according to the given conditions. Here's a step-by-step approach to generate the TikZ code:

1. **Define the Black Vertices**: The black vertices form a clique of order \( \frac{5D}{2} \).
2. **Place White Vertices**: Each black vertex should have at most one white neighbor.
3. **Connect Gray Vertices**: Each pair of black vertices without a common white neighbor should be connected by a gray vertex.

Let's assume \(D = 4\) for simplicity, so the clique size will be \( \frac{5 \times 4}{2} = 10 \).

Here's the TikZ code to achieve this:

```latex
\documentclass[tikz]{standalone}
\usetikzlibrary{shapes.geometric, positioning}

\begin{document}
\begin{tikzpicture}[node distance=2cm]

% Define the black vertices
\foreach \i in {1,...,10} {
    \node[draw, circle, fill=black] (B-\i) at (\i*1.5-7.5,0) {};
}

% Connect black vertices to form a clique
\foreach \i in {1,...,9} {
    \foreach \j in {\i+1,...,10} {
        \draw (B-\i) -- (B-\j);
    }
}

% Place white vertices
\node[draw, circle, fill=white] (W1) at (-2,-1) {};
\node[draw, circle, fill=white] (W2) at (2,-1) {};

% Connect black vertices to white vertices
\foreach \i in {1,...,10} {
    \draw (B-\i) -- (W1);
    \draw (B-\i) -- (W2);
}

% Place gray vertices
\node[draw, circle, fill=gray] (G1) at (-2,-2) {};
\node[draw, circle, fill=gray] (G2) at (2,-2) {};

% Connect black pairs without common white neighbors to gray vertices
\draw (B-1) -- (G1);
\draw (B-2) -- (G1);
\draw (B-3)