Certainly! Below is a TikZ LaTeX code that illustrates the \( X \)-gadgets for all sets \( X \) which are relevant by Proposition~\ref{chemin2}. In this example, the vertices in \( Z \) are marked by squares and the remaining vertices by disks.

```latex
\documentclass{standalone}
\usepackage{tikz}

\tikzset{
    vertex/.style={circle, draw=black, fill=white},
    squarevertex/.style={rectangle, draw=black, fill=white}
}

\begin{document}

\begin{tikzpicture}[scale=1.5]
    % Vertices in Z (squares)
    \node[squarevertex] (Z1) at (-3, 0) {};
    \node[squarevertex] (Z2) at (-1, 0) {};
    \node[squarevertex] (Z3) at (1, 0) {};
    \node[squarevertex] (Z4) at (3, 0) {};

    % Remaining vertices (disks)
    \node[vertex] (V1) at (-4, -2) {};
    \node[vertex] (V2) at (-2, -2) {};
    \node[vertex] (V3) at (0, -2) {};
    \node[vertex] (V4) at (2, -2) {};
    \node[vertex] (V5) at (4, -2) {};

    % Edges between vertices
    \draw (Z1) -- (V1);
    \draw (Z1) -- (V2);
    \draw (Z2) -- (V2);
    \draw (Z2) -- (V3);
    \draw (Z3) -- (V3);
    \draw (Z3) -- (V4);
    \draw (Z4) -- (V4);
    \draw (Z4) -- (V5);

    \draw (V1) -- (V2);
    \draw (V2) -- (V3);
    \draw (V3) -- (V4);
    \draw (V4) -- (V5);
    \draw (V5) -- (V1);

    % Labels
    \node at (-3, -2.5) {$X = \{Z_1, Z_2, Z_3, Z_