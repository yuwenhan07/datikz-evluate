Certainly! Below is a TikZ LaTeX code that visualizes the partial and whole spherical linkage described in your prompt. The code includes labels for the angles and gaps as specified.

```latex
\documentclass[tikz,border=3mm]{standalone}
\usepackage{tikz-3dplot}

\begin{document}

\tikzset{
    dot/.style={circle,fill,inner sep=1pt},
    line/.style={thick}
}

\tdplotsetmaincoords{70}{45} % Set the main coordinates

\begin{tikzpicture}[tdplot_main_coords]
    
    % Define points
    \coordinate (O) at (0,0,0);
    \coordinate (A) at (2,0,0);
    \coordinate (B) at (2,2,0);
    \coordinate (C) at (0,2,0);
    \coordinate (D) at (0,0,-2);
    \coordinate (E) at (2,0,-2);
    \coordinate (F) at (2,2,-2);
    \coordinate (G) at (0,2,-2);

    % Draw lines
    \draw[line] (O) -- (A) -- (B) -- (C) -- cycle;
    \draw[line] (O) -- (D) -- (E) -- (F) -- cycle;
    \draw[line] (A) -- (E);
    \draw[line] (B) -- (F);
    \draw[line] (C) -- (G);
    \draw[line] (D) -- (A);
    \draw[line] (D) -- (B);
    \draw[line] (D) -- (C);
    \draw[line] (E) -- (F);
    \draw[line] (E) -- (G);
    \draw[line] (F) -- (G);

    % Draw angles and labels
    \node[dot,label=below left:$\lambda_i$] at (1,0,0) {};
    \node[dot,label=above right:$\gamma_i$] at (2,1,0) {};
    \node[dot,label=below right:$\mu_i$] at (1,2,0) {};
    \node[dot,label=above left:$\delta_i$] at (0,1,0) {};

   