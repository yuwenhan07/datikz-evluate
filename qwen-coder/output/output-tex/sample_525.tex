To create a TikZ diagram representing the described directed acyclic graph (DAG), we can use the `tikz` package along with the `graphdrawing` library for better control over the layout. Below is an example of how you might represent this DAG:

```latex
\documentclass[tikz,border=2mm]{standalone}
\usetikzlibrary{graphs,graphdrawing,shapes.geometric}
\usegdlibrary{layered}

\tikzset{
  node distance=2cm,
  start chain=going below,
  mynode/.style={draw, circle, inner sep=3pt},
  myedge/.style={->, thick},
}

\begin{document}
\begin{tikzpicture}[layered layout]
    \node[myedge] (t) {t};
    \node[myedge] (t1) [below=of t] {t+1};

    % Nodes at time t
    \node[mynode] (C) [left=of t] {$C$};
    \node[mynode] (Y) [right=of t] {$Y$};
    
    % Nodes at time t+1
    \node[mynode] (X) [below=of C, left=of t1] {$X$};
    \node[mynode] (Z) [below=of Y, right=of t1] {$Z$};
    
    % Edges
    \path[myedge] (C) edge (X);
    \path[myedge] (Y) edge (Z);
    
    % Semi-deterministic paths
    \path[red, dashed] (C) edge (X);
    \path[red, dashed] (Y) edge (Z);
    
    % Selection bias edges
    \path[myedge] (C) edge[bend left] (t1);
    \path[myedge] (Y) edge[bend right] (t1);
    
    % Time nodes
    \node[myedge] (t2) [above=of t1] {t+2};
    \node[mynode] (W) [below=of X, left=of t2] {$W$};
    \node[mynode] (V) [below=of Z, right=of t2] {$V$};
    
    % Additional edges
    \path[my