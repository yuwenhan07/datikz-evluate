Certainly! Below is a TikZ LaTeX code snippet that generates a diagram illustrating the definition of \( x_p^{\pm} \) and \( y_p^{\pm} \) values for a pixel \( p \) in a poorly textured region, with orange points indicating potential errors in estimating \( \Delta x_p \) and \( \Delta y_p \).

```latex
\documentclass{standalone}
\usepackage{tikz}

\begin{document}

\begin{tikzpicture}[scale=2]
    % Define the center point
    \coordinate (p) at (0, 0);
    
    % Draw the central pixel
    \filldraw[black] (p) circle (0.1) node[below right] {$p$};
    
    % Define the ± values
    \coordinate (xp_plus) at (1, 0);
    \coordinate (xp_minus) at (-1, 0);
    \coordinate (yp_plus) at (0, 1);
    \coordinate (yp_minus) at (0, -1);
    
    % Draw the ± lines
    \draw[->] (p) -- (xp_plus) node[midway, above] {$x_p^+$};
    \draw[->] (p) -- (xp_minus) node[midway, below] {$x_p^-$};
    \draw[->] (p) -- (yp_plus) node[midway, left] {$y_p^+$};
    \draw[->] (p) -- (yp_minus) node[midway, right] {$y_p^-$};
    
    % Orange points indicating potential errors
    \filldraw[orange] (0.8, 0.1) circle (0.05) node[right] {Error};
    \filldraw[orange] (0.3, -0.1) circle (0.05) node[right] {Error};
    \filldraw[orange] (-0.8, -0.1) circle (0.05) node[left] {Error};
    \filldraw[orange] (-0.3, 0.1) circle (0.05) node[left] {Error};
    
    % Labels
    \node[above right] at (1.2, 0) {\(\Delta x_p\)};
    \node[below right] at (-1.2, 0) {\