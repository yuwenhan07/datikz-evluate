Certainly! Below is a TikZ LaTeX code that visualizes the duality transformation in a replica space with the specified elements:

```latex
\documentclass[tikz,border=3mm]{standalone}
\usetikzlibrary{arrows.meta,shapes.geometric}

\begin{document}
\begin{tikzpicture}[scale=1.5]
    % Draw the lattice
    \draw[thick] (0,0) grid (4,2);
    
    % Label the nodes
    \node at (0,0) [circle,fill=white,inner sep=2pt,label={left:$i$}] {};
    \node at (1,0) [circle,fill=white,inner sep=2pt,label={below:$i+1$}] {};
    \node at (2,0) [circle,fill=white,inner sep=2pt,label={right:$i+2$}] {};
    \node at (3,0) [circle,fill=white,inner sep=2pt,label={above:$i+3$}] {};
    \node at (0,1) [circle,fill=white,inner sep=2pt,label={left:$i+\hat{0}$}] {};
    \node at (1,1) [circle,fill=white,inner sep=2pt,label={below:$i+1+\hat{0}$}] {};
    \node at (2,1) [circle,fill=white,inner sep=2pt,label={right:$i+2+\hat{0}$}] {};
    \node at (3,1) [circle,fill=white,inner sep=2pt,label={above:$i+3+\hat{0}$}] {};
    
    % Draw the entangling surface
    \draw[dashed,thick] (0,0) -- (4,0) -- (4,1) -- (0,1) -- cycle;
    
    % Draw the red cut
    \fill[red] (2,0) rectangle (3,1);
    
    % Draw the red dot
    \fill[red] (2.5,0.5) circle (2pt);
    
    % Add labels for clarity
    \node at (2,-0.5) [below] {Cut};
    \node at (2.5,0.75) [above right] {$i$-$i