Certainly! Below is an example of a TikZ LaTeX code that generates a simple argumentation graph for the purpose of demonstrating how to create a structured network or graph in TikZ.

```latex
\documentclass{article}
\usepackage{tikz}

\begin{document}

\begin{figure}[h]
    \centering
    \begin{tikzpicture}[node distance=2cm, auto]
        % Define styles for nodes
        \tikzset{
            decision/.style={diamond, draw, fill=blue!20, text width=4.5em, text badly centered, node distance=3cm, inner sep=0pt},
            block/.style={rectangle, draw, fill=blue!20, text width=5em, text centered, rounded corners, minimum height=4em},
            line/.style={draw, thick, ->, >=stealth'},
            cloud/.style={draw, ellipse, fill=red!20, node distance=3cm, minimum height=2em}
        }

        % Nodes
        \node (start) [block] {Début};
        \node (dec1) [decision, below of=start] {Est-ce que la personne est une IRM?};
        \node (dec2) [decision, right of=dec1, xshift=3cm] {Est-ce qu'il y a un signal radio?};
        \node (radio) [cloud, above of=dec2] {Signal Radio};
        \node (irm) [cloud, left of=dec1] {IRM};
        \node (end) [block, below of=dec2, yshift=-2cm] {Fin};

        % Edges
        \path [line] (start) -- (dec1);
        \path [line] (dec1) -| node[anchor=north east] {Non} (end);
        \path [line] (dec1) -| node[anchor=south west] {Oui} (dec2);
        \path [line] (dec2) |- node[anchor=north] {Non} (end);
        \path [line] (dec2) |- node[anchor=south] {Oui} (radio);
        \path [line] (radio) -- (end);

    \end{tikzpicture}
    \caption{Graphique d'Argumentation pour IRM ou