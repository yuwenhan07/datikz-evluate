Certainly! Below is a TikZ LaTeX code that represents the relaxation system in the 1-to-1 coupling case on the discretized real line. This code includes various lines and mathematical symbols, similar to what you might see on a whiteboard.

```latex
\documentclass[tikz,border=2mm]{standalone}
\usetikzlibrary{arrows.meta,calc}

\tikzset{
    mynode/.style={draw, rectangle, minimum width=3em, minimum height=2em, align=center},
    myarrow/.style={->, thick, >=Stealth[length=5pt]},
    eqnode/.style={mynode, fill=yellow!30}
}

\begin{document}
\begin{tikzpicture}[x=2cm,y=1cm]

% Nodes
\node[eqnode] (eq1) at (0,0) {$m_1 \ddot{x}_1 = -k(x_1 - x_2)$};
\node[eqnode] (eq2) at (0,-2) {$m_2 \ddot{x}_2 = k(x_1 - x_2)$};

% Equations
\node[mynode, right=of eq1] (eq3) {Equation 1: $m_1 \ddot{x}_1 = -k(x_1 - x_2)$};
\node[mynode, right=of eq2] (eq4) {Equation 2: $m_2 \ddot{x}_2 = k(x_1 - x_2)$};

% Arrows between nodes
\draw[myarrow] (eq1.east) -- node[above] {$+$} (eq3.west);
\draw[myarrow] (eq2.east) -- node[below] {$-$} (eq4.west);

% Lines connecting nodes
\draw[dashed, thick] ($(eq1.south) + (-1,0)$) -- ($(eq2.north) + (-1,0)$);
\draw[dashed, thick] ($(eq1.south) + (1,0)$) -- ($(eq2.north) + (1,0)$);

% Labels
\node[left=of eq1] (label1) {Object 1};
\node[left=of eq2] (label2) {Object 2};

\end{tikzpicture