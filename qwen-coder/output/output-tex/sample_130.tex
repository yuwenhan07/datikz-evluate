Certainly! Below is a TikZ LaTeX code that depicts a Bernoulli path \( h_0 \) and the associated particle configuration \( \eta_{h_0} \). The dashed portions of the path illustrate the definition of the new height function, under the corresponding displayed particle movements.

```latex
\documentclass{standalone}
\usepackage{tikz}

\begin{document}

\begin{tikzpicture}[scale=2]
    % Define the grid
    \draw[step=1cm,gray!30] (-4,-2) grid (4,2);
    
    % Draw the Bernoulli path h_0
    \draw[-stealth, thick] (-4,0) -- (4,0) node[right] {$x$};
    \draw[-stealth, thick] (0,-2) -- (0,2) node[above] {$h_0(x)$};
    \foreach \x in {-3,-2,-1,0,1,2,3} {
        \draw[dashed] (\x,-2) -- (\x,2);
    }
    
    % Particle configuration η_h_0
    \filldraw[blue] (-3,-2) circle (2pt) node[left] {$\eta_{h_0}(-3) = 0$};
    \filldraw[red] (-2,-1) circle (2pt) node[left] {$\eta_{h_0}(-2) = -1$};
    \filldraw[green] (-1,0) circle (2pt) node[left] {$\eta_{h_0}(-1) = 0$};
    \filldraw[yellow] (0,1) circle (2pt) node[left] {$\eta_{h_0}(0) = 1$};
    \filldraw[cyan] (1,0) circle (2pt) node[left] {$\eta_{h_0}(1) = 0$};
    \filldraw[magenta] (2,-1) circle (2pt) node[left] {$\eta_{h_0}(2) = -1$};
    \filldraw[brown] (3,-2) circle (2pt) node[left] {$\eta_{h_0}(3) = 0$};
    
    % Highlight the dashed portions
    \draw[dashed, blue]