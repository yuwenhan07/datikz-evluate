To create a TikZ diagram comparing the PSF (Point Spread Function) of a single sparse array and a joint setup of a fixed communication array and a varying sensing array, we can follow these steps:

1. **Define the arrays**: We will define the positions of the elements in both arrays.
2. **Draw the arrays**: We will draw the arrays using TikZ nodes.
3. **Highlight the differences**: We will highlight the key differences between the two setups.

Here is the TikZ code to generate this diagram:

```latex
\documentclass[tikz,border=5mm]{standalone}
\usetikzlibrary{positioning}

\begin{document}
\begin{tikzpicture}[node distance=0.5cm]

% Sparse Array
\node (sparse_array) at (0,0) {Sparse Array};
\foreach \i in {0,...,8} {
    \foreach \j in {0,...,8} {
        \ifnum\i=1 \&\& \j=1 \else
            \ifnum\i=2 \&\& \j=2 \else
                \node[draw, circle, inner sep=1pt] (s-\i-\j) at (\i,-\j) {};
            \fi
        \fi
    }
}

% Communication Array
\node (comm_array) at (4,0) {Communication Array};
\foreach \i in {0,...,10} {
    \foreach \j in {0,...,10} {
        \node[draw, circle, inner sep=1pt] (c-\i-\j) at (\i,-\j) {};
    }
}

% Sensing Array
\node (sense_array) at (8,0) {Sensing Array};
\foreach \i in {0,...,10} {
    \foreach \j in {0,...,10} {
        \node[draw, circle, inner sep=1pt] (s-\i-\j) at (\i,-\j) {};
    }
}

% Highlight Differences
\draw[dashed, thick] (s-0-0.south west) rectangle (s-8-8.north east);
\draw[dashed, thick] (c-0-0.south west) rectangle (c-10-10.north east);
\draw[dashed, thick] (s-0-0