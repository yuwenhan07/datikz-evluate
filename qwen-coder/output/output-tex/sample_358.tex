Certainly! Below is a TikZ LaTeX code to create a flowchart that outlines the steps in a process, including decision points and logical flow. This example assumes that "Hat," "Task," and "Representation" are nodes in the flowchart, and "Ellie" and "Pi" are labels for certain actions or decisions.

```latex
\documentclass{article}
\usepackage{tikz}
\usetikzlibrary{shapes.geometric, arrows.meta}

\tikzstyle{startstop} = [rectangle, rounded corners, minimum width=3cm, minimum height=1cm,text centered, draw=black, fill=red!30]
\tikzstyle{io} = [trapezium, trapezium left angle=70, trapezium right angle=110, minimum width=3cm, minimum height=1cm, text centered, draw=black, fill=blue!30]
\tikzstyle{process} = [rectangle, minimum width=3cm, minimum height=1cm, text centered, draw=black, fill=orange!30]
\tikzstyle{decision} = [diamond, minimum width=3cm, minimum height=1cm, text centered, draw=black, fill=green!30]
\tikzstyle{arrow} = [thick,->,>=stealth]

\begin{document}

\begin{figure}[h]
    \centering
    \begin{tikzpicture}[node distance=2cm]
        % Nodes
        \node (start) [startstop] {Start};
        \node (hat) [io, below of=start] {Hat};
        \node (task) [process, below of=hat] {Task};
        \node (representation) [io, below of=task] {Representation};
        \node (turn) [arrow, sloped, above of=representation, yshift=-0.5cm] {};
        \node (ellie) [decision, left of=turn, xshift=-4cm] {Ellie};
        \node (pi) [decision, right of=turn, xshift=4cm] {Pi};

        % Arrows
        \draw [arrow] (start) -- (hat);
        \draw [arrow] (hat) -- (task);
        \draw [arrow] (task) -- (representation);
        \draw [arrow