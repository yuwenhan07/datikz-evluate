Certainly! Below is a TikZ LaTeX code that generates a visually appealing tree representing the computation of the packaged surface Tutte polynomial using deletion and contraction. Each level of the tree is colored differently to make it easy to follow.

```latex
\documentclass{standalone}
\usepackage{tikz}
\usetikzlibrary{shapes.geometric, arrows.meta}

\tikzset{
  node distance=2cm,
  start chain/.style={start chain=going below},
  box/.style={draw, rectangle, rounded corners, fill=#1, text width=4em, align=center, join=by {-Stealth}},
  level 1/.style={fill=red!30},
  level 2/.style={fill=blue!30},
  level 3/.style={fill=green!30},
  level 4/.style={fill=orange!30},
  level 5/.style={fill=purple!30}
}

\begin{document}
\begin{tikzpicture}[auto]

% Level 1
\node (n1) [box=level 1] {Node 1};
\node (n2) [box=level 1, right=of n1] {Node 2};

% Level 2
\node (n3) [box=level 2, below=of n1] {Node 3};
\node (n4) [box=level 2, below=of n2] {Node 4};

% Level 3
\node (n5) [box=level 3, below=of n3] {Node 5};
\node (n6) [box=level 3, below=of n4] {Node 6};

% Level 4
\node (n7) [box=level 4, below=of n5] {Node 7};
\node (n8) [box=level 4, below=of n6] {Node 8};

% Level 5
\node (n9) [box=level 5, below=of n7] {Node 9};
\node (n10) [box=level 5, below=of n8] {Node 10};

% Connections
\draw[->] (n1) -- (n3);
\draw[->] (n1) -- (n4);
\draw