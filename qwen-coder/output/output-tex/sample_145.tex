To visualize the structure of a composite system composed of individual systems \( \mathcal{Q}_A \) and \( \mathcal{Q}_B \), we can use TikZ to create a diagram that illustrates how individual questions from each system combine into composite questions through different logical gates.

Here's a TikZ code snippet that creates such a diagram:

```latex
\documentclass{standalone}
\usepackage{tikz}

\begin{document}

\begin{tikzpicture}[node distance=2cm, auto]

% Define styles for nodes and edges
\tikzstyle{question} = [rectangle, draw=black, fill=white!30, text width=5em, text centered]
\tikzstyle{gate} = [circle, draw=black, fill=blue!30, text width=3em, text centered]

% Nodes representing questions in system A
\node (qa1) [question] {Qa1};
\node (qa2) [question, below of=qa1] {Qa2};

% Nodes representing questions in system B
\node (qb1) [question, right of=qa1, xshift=4cm] {Qb1};
\node (qb2) [question, below of=qb1] {Qb2};

% Nodes representing composite questions
\node (qc1) [gate, above of=qa1, yshift=-2cm] {$Qa1 *_{AND} Qb1$};
\node (qc2) [gate, above of=qa2, yshift=-2cm] {$Qa2 *_{AND} Qb2$};
\node (qc3) [gate, left of=qc1, xshift=-2cm] {$Qa1 *_{OR} Qb1$};
\node (qc4) [gate, left of=qc2, xshift=-2cm] {$Qa2 *_{OR} Qb2$};

% Draw edges between questions and composite questions
\draw[->] (qa1) -- (qc1);
\draw[->] (qa2) -- (qc2);
\draw[->] (qb1) -- (qc1);
\draw[->] (qb2) -- (qc2);

\draw[->] (qa1) -- (qc3);
\draw[->] (qa2) -- (qc3);
