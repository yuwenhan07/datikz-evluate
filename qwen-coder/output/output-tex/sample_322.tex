To create a TikZ LaTeX diagram for the remaining vertices of the six hyperedges in which Breaker has not yet played in Case 1 of Theorem~\ref{TD(4,4)}, we need to first understand the structure of these hyperedges and then represent them visually.

Here's a step-by-step approach:

1. Identify the vertices involved in each hyperedge.
2. Create a graph where each vertex is represented by a node and each hyperedge is represented by an edge connecting multiple nodes.
3. Use TikZ to draw this graph.

Let's write the TikZ code to visualize this:

```latex
\documentclass{standalone}
\usepackage{tikz}

\begin{document}
\begin{tikzpicture}[node distance=2cm]
    % Define the nodes
    \node (a1) at (0,0) {$\alpha_1$};
    \node (a2) at (2,0) {$\alpha_2$};
    \node (a3) at (4,0) {$\alpha_3$};
    \node (a4) at (6,0) {$\alpha_4$};
    \node (b1) at (0,-2) {$\beta_1$};
    \node (b2) at (2,-2) {$\beta_2$};
    \node (b3) at (4,-2) {$\beta_3$};
    \node (b4) at (6,-2) {$\beta_4$};
    \node (c2) at (1,-1) {$c_2$};
    \node (c3) at (3,-1) {$c_3$};
    \node (c4) at (5,-1) {$c_4$};

    % Draw the edges representing the hyperedges
    \draw (a3) -- (b2);
    \draw (c2) -- (a2) -- (b2);
    \draw (c2) -- (a4) -- (b1);
    \draw (c3) -- (a3) -- (b3);
    \draw (c4) -- (a2) -- (b3);
    \draw (c4) -- (a4) -- (b4);

    % Label the nodes
    \foreach \node in {a1, a2, a3