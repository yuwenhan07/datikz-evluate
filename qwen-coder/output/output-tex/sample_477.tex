To generate the TikZ LaTeX code for the described multiline diagram \( D \), we need to create a visual representation of the Young tableau with the given content and properties. Here is the step-by-step process:

1. **Define the dimensions and content**: The diagram has 6 columns and 5 rows, with the content \(\lambda = (5, 4, 3, 1, 0, 0)\). The bottom row \(\rho^{(1)}(D) = (4, 0, 1, 5, 3, 0)\).

2. **Calculate the weights**: The weight \(\wt(D) = \wt_x(D) \wt_t(D) = x_1^3 x_3^2 x_4^4 x_5^2 x_6^2 \, t^2\).

3. **Construct the TikZ code**:
   - Use `\node` to place each element in the correct position.
   - Use `matrix` environment to organize the nodes into rows and columns.

Here is the TikZ LaTeX code for the diagram:

```latex
\documentclass{standalone}
\usepackage{tikz}

\begin{document}
\begin{tikzpicture}[scale=1.5]
    % Define the matrix
    \matrix (m) [matrix of nodes,
                 column sep=-\pgflinewidth,
                 row sep=-\pgflinewidth,
                 nodes={draw, minimum size=1em},
                 anchor=north west] {
        & 5 & 4 & 3 & 1 & 0 & 0 \\
        & & 4 & 0 & 1 & 5 & 3 \\
        & & & 0 & 1 & 5 & 3 \\
        & & & & 1 & 5 & 3 \\
        & & & & & 5 & 3 \\
    };

    % Draw the lines between cells
    \foreach \i in {1,...,5} {
        \foreach \j in {1,...,6} {
            \ifnum\j<6
                \draw[dashed] ([yshift=-0.5ex,xshift=0.5ex]m-\i-\j.north east) -- ([yshift=-0.5ex,xshift=0.5ex]m-\i+\j.west);
            \fi
            \ifnum