To construct the intersection poset \( L(\mathcal{A}) \) for an arrangement of \( N \) families \( A_i \) of \( D \) parallel hyperplanes in general position, we need to understand the structure of the poset. The intersection poset is formed by considering all intersections of these hyperplanes, ordered by inclusion.

Here's how you can generate the TikZ LaTeX code to visualize this:

```latex
\documentclass{article}
\usepackage{tikz}

\begin{document}

% Define the number of families and dimensions
\def\N{3} % Number of families
\def\D{2} % Number of dimensions

% Define the colors for different levels of intersections
\colorlet{level0}{white}
\colorlet{level1}{yellow}
\colorlet{level2}{green!50!black}
\colorlet{level3}{blue!70!black}

% Draw the hyperplanes
\foreach \i in {0,...,\N-1} {
    \draw[thick] (0,-\i*\D/(\N+1)) -- (\textwidth,-\i*\D/(\N+1));
}

% Draw the intersection points
\foreach \x in {0,...,\D-1} {
    \foreach \y in {0,...,\N-1} {
        \fill[level=\y] (\x*\textwidth/\D,-\y*\D/(\N+1)) circle (2pt);
    }
}

% Add labels
\node at (-0.1,-\N*\D/(\N+1)+0.1) {\(H_0\)};
\node at (-0.1,-\N*\D/(\N+1)-0.1) {\(H_1\)};
\node at (-0.1,-\N*\D/(\N+1)-\D/(2*(\N+1))) {\(H_2\)};
\node at (0.5*\textwidth,-\N*\D/(\N+1)-0.1) {\(H_{\N-1}\)};
\node at (0.5*\textwidth,-\N*\D/(\N+1)-\D/(2*(\N+1))) {\(H_{\N}\)};
\node at (0.5*\textwidth,-\N*\D/(\N+1