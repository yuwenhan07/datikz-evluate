To create a semantically accurate and structurally clear TikZ LaTeX code for illustrating bipartite matching in the context of computing the similarity between sense clusters' centroids \( p_1^t \) and \( p_1^{t-1} \), we can follow these steps:

1. **Define the centroids and their embeddings**: We will represent the centroids as nodes and their corresponding embeddings as smaller nodes.
2. **Draw the edges representing bipartite matching**: These edges will connect the centroids with their respective embeddings.
3. **Add labels and annotations**: To make the diagram more informative, we will add labels and annotations.

Here is the TikZ LaTeX code to achieve this:

```latex
\documentclass{standalone}
\usepackage{tikz}

\begin{document}

\begin{tikzpicture}[node distance=2cm, auto]

% Define styles for nodes
\tikzset{
    centroid/.style={circle, draw, fill=blue!20, minimum size=15pt},
    embedding/.style={rectangle, draw, fill=green!20, minimum size=10pt},
    edge/.style={->, >=stealth', shorten >=1pt}
}

% Draw centroids
\node (centroid_t) [centroid] at (0,0) {$p_1^t$};
\node (centroid_tp1) [centroid, below left of=centroid_t] {$p_1^{t-1}$};

% Draw embeddings
\node (embedding_t1) [embedding, right of=centroid_t, xshift=1.5cm] {$e_1^t$};
\node (embedding_t2) [embedding, above right of=centroid_t, yshift=-1.5cm] {$e_2^t$};
\node (embedding_t3) [embedding, below right of=centroid_t, yshift=1.5cm] {$e_3^t$};

\node (embedding_tp11) [embedding, right of=centroid_tp1, xshift=1.5cm] {$e_1^{t-1}$};
\node (embedding_tp12) [embedding, above right of=centroid_tp1, yshift=-1.5cm] {$e_2^{t-1}$};
\node (embedding_tp13) [embedding, below right of=centroid_tp1, yshift=1.5cm] {$e_3