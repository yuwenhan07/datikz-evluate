To create a TikZ diagram illustrating the species coupling in case of a coinciding interface, we can use the following LaTeX code:

```latex
\documentclass[tikz,border=3mm]{standalone}
\usetikzlibrary{shapes.geometric, arrows.meta}

\tikzset{
    cell/.style={draw, rectangle, minimum size=1cm},
    interface/.style={dashed, thick},
    species/.style={fill=blue!20},
    empty/.style={species, fill=white},
    full/.style={species, fill=blue!60},
    arrow/.style={thick, ->, >=Stealth[]}
}

\begin{document}
\begin{tikzpicture}[node distance=1cm]

    % Nodes for cells
    \node[cell] (K1) {Cell $K_1$};
    \node[cell] (K2) [below right=of K1] {Cell $K_2$};

    % Interface
    \draw[interface] (K1.east) -- node[midway, above] {$\partial K_1 \cap \partial K_2$} (K2.west);

    % Species nodes
    \node[species, left=of K1.south, yshift=-1cm] (SA1) {Species $\mathfrak{A}$};
    \node[species, right=of K2.south, yshift=-1cm] (SB2) {Species $\mathfrak{B}$};

    % Empty and Full species nodes within K2
    \node[empty, below=of SA1, yshift=-1cm] (SA1_empty) {};
    \node[full, below=of SB2, yshift=-1cm] (SB2_full) {};

    % Arrows representing coupling
    \draw[arrow] (SA1_empty) -- node[midway, above] {Coupling} (SB2_full);
    \draw[arrow] (SA1_empty) -- node[midway, below] {Agglomeration} (SA1_empty.west |- K1.west);
    \draw[arrow] (SB2_full) -- node[midway, above] {Agglomeration} (SB2_full.east |- K2.east);

    % Exploded view
    \draw[dotted] (K1.north west) -- (K1