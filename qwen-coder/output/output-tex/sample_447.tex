Certainly! Below is a TikZ LaTeX code that illustrates Riemannian optimization on the 2-torus \(\mathbb{T}_2\) (represented as a torus embedded in \(\mathbb{R}^3\)). The iterate \(\mathbf{w}_{t+1}\) is obtained from the retraction \(\mathcal{R}_{\mathbf{w}_t}\) applied to the direction descent \(\boldsymbol{\xi}_t \in T_{\mathbf{w}_t} \mathbb{T}_2\).

```latex
\documentclass{standalone}
\usepackage{tikz}
\usetikzlibrary{calc, arrows.meta}

\begin{document}

\tikzset{
    arrow style/.style={->, thick},
    vector style/.style={thick, ->, >=Stealth[length=5pt]},
    tangent line style/.style={dashed, thin}
}

\begin{tikzpicture}[scale=2]

% Draw the torus
\def\R{2}
\def\r{0.5}
\pgfdeclarelayer{background}
\pgfsetlayers{background,main}

\begin{scope}[on layer=background]
    \draw[torus] (\R+\r,0,0) arc (0:360:\R+\r);
    \foreach \x in {0,...,7} {
        \draw[torus] (\R+\r,\x*\r/\pi,0) arc (0:360:\R+\r);
    }
    \foreach \y in {-\r,-\r/2,0,\r/2,\r} {
        \draw[torus] (\R+\r,0,\y) arc (0:360:\R+\r);
    }
    \fill[torus] (\R+\r,0,0) circle (0.05);
    \fill[torus] (\R+\r,0,0) circle (0.05);
\end{scope}

% Draw points and vectors
\coordinate (w_t) at (2,0,0);
\coordinate (xi_t) at (4,0,0);

% Draw the retraction
\draw[vector style] (w_t) -- node[midway, above] {$\boldsymbol{\xi}_t$} (xi_t);

% Draw