To visualize the concept of sticky snapping out Brownian motion as described, we can create a TikZ diagram that illustrates the behavior of the process on multiple copies of \([0, \infty]\). Each copy will represent a different dimension where the process can jump from one point to another.

Here's a sample TikZ code to create this visualization:

```latex
\documentclass[tikz]{standalone}
\usepackage{tikz}

\tikzset{
    state/.style={circle, draw=black, fill=white, inner sep=2pt},
    jump/.style={->, thick},
    line/.style={-stealth, thin}
}

\begin{document}
\begin{tikzpicture}[scale=1.5]
    % Define states for each dimension
    \node[state] (S1) at (0, 0) {};
    \node[state] (S2) at (4, 0) {};
    \node[state] (S3) at (8, 0) {};

    % Draw lines between states within each dimension
    \draw[line] (S1.east) -- node[midway, above] {$c_1$} ++(4, 0);
    \draw[line] (S2.east) -- node[midway, above] {$c_2$} ++(4, 0);
    \draw[line] (S3.east) -- node[midway, above] {$c_3$} ++(4, 0);

    % Draw arrows between states in different dimensions
    \draw[jump] (S1.south) -- node[midway, below] {$a_1/b_1$} ++(0, -2);
    \draw[jump] (S2.south) -- node[midway, below] {$a_2/b_2$} ++(0, -2);
    \draw[jump] (S3.south) -- node[midway, below] {$a_3/b_3$} ++(0, -2);

    \node[state] (T1) at (0, -2) {};
    \node[state] (T2) at (4, -2) {};
    \node[state] (T3) at (8, -2) {};

    % Draw lines between states within each dimension after jumping
    \draw[line] (T1.east) -- node[mid