To create a TikZ diagram that illustrates the bounding box calculation process as described, we can break down the steps into visual components. We will use nodes to represent the various elements (predicted values, grid cell location, previous dimensions, and final bounding box), and arrows to show how these elements interact.

Here is a sample TikZ code that demonstrates this:

```latex
\documentclass[tikz,border=2cm]{standalone}
\usetikzlibrary{shapes.geometric, arrows.meta}

\tikzset{
    block/.style = {rectangle, draw, fill=blue!20, text width=5em, text centered, rounded corners, minimum height=4em},
    line/.style = {draw, thick, -Stealth}
}

\begin{document}
\begin{tikzpicture}[node distance=3cm]

\node [block] (gridCell) {Grid Cell};
\node [block, below left of=gridCell] (prevWidth) {Previous Width};
\node [block, below right of=gridCell] (prevHeight) {Previous Height};
\node [block, below left of=prevWidth] (tx) {Predicted $t_x$};
\node [block, below right of=prevHeight] (ty) {Predicted $t_y$};
\node [block, below of=tx, yshift=-1cm] (sigmoidTx) {Sigmoid($t_x$)};
\node [block, below of=ty, yshift=-1cm] (sigmoidTy) {Sigmoid($t_y$)};
\node [block, below of=sigmoidTx, yshift=-1cm] (cx) {Location $c_x$};
\node [block, below of=sigmoidTy, yshift=-1cm] (cy) {Location $c_y$};
\node [block, below of=cx, yshift=-1cm] (finalX) {$x_{\text{final}} = \sigma(t_x) + c_x$};
\node [block, below of=cy, yshift=-1cm] (finalY) {$y_{\text{final}} = \sigma(t_y) + c_y$};
\node [block, below of=finalX, yshift=-1cm] (expTw) {Exp($t_w$)};
\node [block, below of=finalY, yshift=-1cm] (expTh)