Certainly! Below is an example of how you can generate a TikZ LaTeX diagram to represent the dual graph for a pseudo-periodic map with one genus \( g_1 \) component connected by a genus zero component, along with its corresponding 3D mirror.

```latex
\documentclass{standalone}
\usepackage{tikz}
\usetikzlibrary{graphs,graphdrawing,shapes.geometric}

\tikzset{
    vertex/.style={circle, draw, fill=blue!20},
    edge/.style={-stealth, thick}
}

\begin{document}

\begin{figure}[h]
    \centering
    \begin{tikzpicture}[scale=1.5]
        % Top: Dual Graph
        \node[vertex] (v1) at (0,0) {G};
        \node[vertex] (v2) at (2,0) {H};
        
        \draw[edge] (v1) -- node[above] {$e_1$} (v2);
        
        % Bottom: 3D Mirror
        \begin{scope}[yshift=-4cm]
            \begin{scope}[xshift=-2cm]
                \filldraw[cyan!20, opacity=0.7] (-1,-1) rectangle (1,1);
                \draw[-stealth, thick] (-1,-1) -- (1,-1);
                \draw[-stealth, thick] (-1,-1) -- (-1,1);
                \node at (0,0) [circle, draw, fill=green!20] {};
            \end{scope}
            
            \begin{scope}[xshift=2cm]
                \filldraw[magenta!20, opacity=0.7] (-1,-1) rectangle (1,1);
                \draw[-stealth, thick] (-1,-1) -- (1,-1);
                \draw[-stealth, thick] (-1,-1) -- (-1,1);
                \node at (0,0) [circle, draw, fill=red!20] {};
            \end{scope}
            
            \draw[-stealth, thick] (-2,-1) -- (2,-1);
        \end{scope}
    \end{tikzpicture}
    \caption{Top: Dual Graph for a Pseudo-Periodic Map with One Genus