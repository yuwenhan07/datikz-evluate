Certainly! Below is a TikZ LaTeX code to illustrate the scenario you described:

```latex
\documentclass[tikz,border=5mm]{standalone}
\usepackage{amsmath}

\tikzset{
    myellipse/.style={draw, ellipse, minimum width=4cm, minimum height=2cm},
    mydashedline/.style={dashed},
    myarrow/.style={->, >=stealth', thick},
    mylabel/.style={above right, font=\small},
}

\begin{document}

\begin{tikzpicture}[scale=0.8]

% Unit Sphere S
\node (S) at (0,0) [circle, radius=2cm, draw] {};

% Ellipse h(S)
\node (hS) at (0,-2) [myellipse, rotate around={-30:(0,-2)}];
\draw[mydashedline] (-2,-2) -- (2,-2);
\node at (0,-2.5) [mylabel] {$h(\mathbf{S})$};

% Angle epsilon_2
\draw[dotted] (0,-2) -- node[midway, above, sloped] {$\varepsilon_2$} (1,-2);

% Ellipse g(S)
\node (gS) at (3,-2) [myellipse, rotate around={-60:(3,-2)}];
\draw[mydashedline] (1,-2) -- (5,-2);
\node at (3,-2.5) [mylabel] {$g(\mathbf{S})$};

% Angle epsilon_3
\draw[dotted] (3,-2) -- node[midway, above, sloped] {$\varepsilon_3$} (4,-2);

% Plain Ellipse gh(S)
\node (ghS) at (6,-2) [myellipse, rotate around={-90:(6,-2)}];
\draw[mydashedline] (4,-2) -- (8,-2);
\node at (6,-2.5) [mylabel] {$gh(\mathbf{S})$};

% Arrows and labels
\draw[myarrow] (S.east) -- (hS.west);
\node at (1.5,-1.5) [mylabel] {$\varepsilon_