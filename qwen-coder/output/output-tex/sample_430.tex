Certainly! Below is an example of a TikZ LaTeX diagram that illustrates how the denoising performance of Noise2Clean scales with the size of the training set. The diagram includes both Noise2Clean performance and dataset-free method performance for comparison.

```latex
\documentclass[tikz,border=10pt]{standalone}
\usepackage{pgfplots}
\pgfplotsset{compat=1.18}

\begin{document}

\begin{tikzpicture}
    \begin{axis}[
        title={Denoising Performance of Noise2Clean},
        xlabel={Training Set Size (in millions)},
        ylabel={Denoising Performance (PSNR)},
        legend entries={Noise2Clean, Dataset-Free Methods},
        ymin=25,
        ymax=40,
        xmin=0.1,
        xmax=10,
        xtick={0.1, 0.5, 1, 2, 3, 4, 5, 6, 7, 8, 9, 10},
        ytick={25, 30, 35, 40},
        grid=major,
        width=10cm,
        height=6cm,
        ]
        
        % Plotting Noise2Clean performance
        \addplot[blue, thick] coordinates {
            (0.1, 28) (0.5, 32) (1, 34) (2, 36) (3, 37) (4, 38) (5, 39) (6, 39.5) (7, 39.8) (8, 39.9) (9, 40) (10, 40)
        };
        \addlegendentry{Noise2Clean};
        
        % Plotting Dataset-Free Methods performance
        \addplot[red, dashed, thick] coordinates {
            (0.1, 28) (0.5, 28) (1, 28) (2, 28) (3, 28) (4, 28) (5, 28) (6, 28) (7, 28) (8, 28) (9, 28) (10, 28)
        };
        \addlegendentry{