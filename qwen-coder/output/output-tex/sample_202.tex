To generate a TikZ diagram for the reduction of \( T_{20} \) when \((1,4) \in E(\Gamma_2)\), \((1,8)\), and \((2,4) \in E(\Gamma_3)\), we need to understand the structure of these graphs and how they reduce.

First, let's define the initial graph \( T_{20} \). The graph \( T_{n} \) is a tree with \( n \) nodes labeled from 1 to \( n \). For \( T_{20} \), the nodes are labeled from 1 to 20, and it is a tree.

Next, we need to add the edges specified in \(\Gamma_2\) and \(\Gamma_3\):
- \((1,4) \in E(\Gamma_2)\)
- \((1,8) \in E(\Gamma_2)\)
- \((2,4) \in E(\Gamma_3)\)

These edges will be added to the tree \( T_{20} \).

Finally, we need to describe the reduction process. However, since the problem does not specify what kind of reduction we are performing, I'll assume a simple reduction where we remove the redundant edges that form cycles or simplify the graph structure.

Here is the TikZ code to draw the graph \( T_{20} \) with the specified edges:

```latex
\documentclass{standalone}
\usepackage{tikz}

\begin{document}
\begin{tikzpicture}[level distance=1.5cm,
                    level 1/.style={sibling distance=3cm},
                    level 2/.style={sibling distance=2cm},
                    level 3/.style={sibling distance=1.5cm}]
    % Nodes
    \node [circle,draw] (1) {1};
    \node [circle,draw] (2) [below left of=1] {2};
    \node [circle,draw] (3) [below right of=1] {3};
    \node [circle,draw] (4) [below left of=2] {4};
    \node [circle,draw] (5) [below right of=2] {5};
    \node [circle,draw] (6) [below left of=3] {6};
    \node [circle,draw