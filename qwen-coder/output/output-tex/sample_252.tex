Certainly! Below is a TikZ LaTeX code that visually represents the structure of the UPINN method as described:

```latex
\documentclass[border=10pt]{standalone}
\usepackage{tikz}
\usetikzlibrary{shapes.geometric, arrows.meta}

\tikzset{
    block/.style = {rectangle, draw, text width=5em, text centered, rounded corners, minimum height=4em},
    line/.style = {draw, -Stealth, thick},
    cloud/.style = {draw, ellipse, node distance=3cm, minimum height=2em}
}

\begin{document}
\begin{tikzpicture}[node distance=2cm]

    % Nodes
    \node [block] (input_time) {Input Time $t$};
    \node [block, below of=input_time] (surrogate_solution) {Surrogate Solution $U$};
    \node [cloud, right of=surrogate_solution, xshift=4cm] (known_component) {Known Component $F$};
    \node [cloud, above of=known_component, yshift=-2cm] (unknown_component) {Unknown Component $G$};
    \node [block, below of=unknown_component, yshift=-1cm] (prediction_network) {Prediction Network};
    \node [cloud, left of=prediction_network, xshift=-4cm] (data) {Data};
    \node [block, right of=prediction_network, xshift=4cm] (pinn_loss) {PINN Loss};
    \node [block, below of=pinn_loss, yshift=-1cm] (mse_loss) {MSE Loss};

    % Lines
    \path [line] (input_time) -- (surrogate_solution);
    \path [line] (surrogate_solution) -- (known_component);
    \path [line] (surrogate_solution) -- (unknown_component);
    \path [line] (known_component) -- (pinn_loss);
    \path [line] (unknown_component) -- (pinn_loss);
    \path [line] (prediction_network) -- (pinn_loss);
    \path [line] (prediction_network) -- (mse_loss);
    \path [line] (data) -- (mse_loss);

    % Arrows for derivatives
    \draw [-Stealth, thick, dashed] ([yshift=-1