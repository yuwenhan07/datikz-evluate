Certainly! Below are two TikZ LaTeX diagrams representing the two cases you described for Operation `op01`. The first case shows a scenario where `v` is a 0-anchor, and the green edge is the edge `{v, v'}`. The second case shows a scenario where `v` is a 1-anchor, and the green edge is still the edge `{v, v'}`.

```latex
\documentclass[tikz,border=5mm]{standalone}
\usetikzlibrary{shapes.geometric, arrows.meta}

\tikzset{
    node/.style={circle, draw, minimum size=1cm},
    anchor/.style={rectangle, draw, minimum width=0.8cm, minimum height=0.4cm},
    edge/.style={->, thick},
    greenedge/.style={->, thick, green}
}

\begin{document}

% Case 1: v is a 0-anchor
\begin{tikzpicture}[x=2cm, y=2cm]
    % Nodes
    \node[node] (v) at (0,0) {v};
    \node[node] (v') at (2,0) {v'};
    
    % Anchor
    \node[anchor] (anchor) at (1,-1) {};
    
    % Edges
    \draw[edge] (v) -- (anchor);
    \draw[edge] (v') -- (anchor);
    \draw[greenedge] (v) -- (v');
    
    % Labels
    \node[left] at (-0.3,-1) {0-anchor};
\end{tikzpicture}

% Case 2: v is a 1-anchor
\begin{tikzpicture}[x=2cm, y=2cm]
    % Nodes
    \node[node] (v) at (0,0) {v};
    \node[node] (v') at (2,0) {v'};
    
    % Anchor
    \node[anchor] (anchor) at (1,-1) {};
    
    % Edges
    \draw[edge] (v) -- (anchor);
    \draw[edge] (v') -- (anchor);
    \draw[greenedge] (v) -- (v');
    
    % Labels
    \node[right] at (2.3,-1) {1-anchor};
\end{tikzpicture}

