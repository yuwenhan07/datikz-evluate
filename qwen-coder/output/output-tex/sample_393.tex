\documentclass{standalone}
\usepackage{tikz}
\usetikzlibrary{shapes.geometric, arrows.meta}

\tikzset{
  node distance=2cm,
  start chain=going right,
  every join/.style={->, thick},
  state/.style={circle, draw, minimum size=10mm, inner sep=0pt, outer sep=0pt, on chain},
  equation/.style={rectangle, draw, minimum width=3cm, minimum height=1cm, align=center, anchor=north west}
}

\begin{document}

\begin{tikzpicture}
  % Nodes
  \node[state] (A) {A};
  \node[state] (B) [below=of A] {B};
  \node[state] (C) [right=of B] {C};
  \node[state] (D) [left=of C] {D};
  
  % Equations
  \node[equation] (eq1) at ([yshift=-3cm]current bounding box.west|-A) {$A^2$};
  \node[equation] (eq2) at ([xshift=3cm,yshift=-3cm]current bounding box.east|-B) {$A^2$};
  
  % Connections
  \draw[join] (A) -- (B);
  \draw[join] (B) -- (C);
  \draw[join] (C) -- (D);
  \draw[join] (D) -- (A);
  
  % Equation connections
  \draw[->, thick] (eq1.south) -- ++(0,-1) |- (A.north);
  \draw[->, thick] (eq2.south) -- ++(0,-1) |- (B.north);
\end{tikzpicture}

\end{document}