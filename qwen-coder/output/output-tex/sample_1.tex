\documentclass[tikz,border=10pt]{standalone}
\usepackage{tikz}

\begin{document}

\begin{tikzpicture}[scale=1.5]

% Define coordinates
(0,0) node[below left] {0} -- (6,0) node[right] {$Q$};
(0,0) -- (0,6) node[above left] {$C$};

% Initial production function curve
\draw[thick, blue] plot coordinates {(1,1) (2,2) (3,3) (4,4) (5,5)};
node at (2.5,2.5) [circle, fill=blue, inner sep=1pt];

% New production function curve after agglomeration
\draw[thick, red] plot coordinates {(1,1.5) (2,3) (3,4.5) (4,6) (5,7.5)};
node at (2.5,3.5) [circle, fill=red, inner sep=1pt];

% Labeling
\node at (3,-0.5) [below] {Initial Production Function};
\node at (3,6.5) [above] {New Production Function};
\node at (8,3) [right] {City Size Increase ($N \to N'$)};

\end{tikzpicture}

\end{document}