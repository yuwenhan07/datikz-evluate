To create a TikZ diagram that represents the sets \(X_0\), \(X_1\), and the relationships between them as described, along with the vertices in \(M'\) and \(M''\), we can use the following TikZ code:

```latex
\documentclass{standalone}
\usepackage{tikz}

\begin{document}
\begin{tikzpicture}[scale=1.5]

% Define nodes for the sets X0 and X1
\node[draw, circle, fill=blue!30] (x0) at (0,0) {X0};
\node[draw, circle, fill=pattern=crosshatch dots, inner sep=6pt] (x1) at (2,0) {X1};

% Draw edges to represent the intersection of N(x) with A0 and A1 \ A0
\draw[-stealth] (-1,0) -- node[left] {$N(x) \cap A_0$} (x0);
\draw[-stealth] (1,0) -- node[right] {$N(x) \cap (A_1 \setminus A_0)$} (x1);

% Define nodes for the pairs of vertices
\node[circle, draw, fill=red, inner sep=2pt] (v1) at (-2,-2) {};
\node[circle, draw, fill=red, inner sep=2pt] (v2) at (-1,-2) {};
\node[circle, draw, fill=yellow, inner sep=2pt] (v3) at (1,-2) {};
\node[circle, draw, fill=yellow, inner sep=2pt] (v4) at (2,-2) {};

% Connect the vertices to form the pairs
\draw[dashed] (v1) -- (v2);
\draw[dashed] (v3) -- (v4);

% Label M' and M''
\node[below] at (-2,-3) {M'};
\node[below] at (1,-3) {M''};

% Draw edges to show common neighbors in S(x, 2)
\draw[-stealth] (-2,-3) -- node[below left] {2 common neighbors} (-1,-3);
\draw[-stealth] (1,-3) -- node[below right] {1 common neighbor}