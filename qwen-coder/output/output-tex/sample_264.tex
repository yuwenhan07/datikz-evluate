To create a TikZ LaTeX diagram that visually represents the mean (normalized) R10 ONT raw signal value conditioned on individual nucleotide base at a given (relative) position, we can follow these steps:

1. **Define the Axes**: We will have a horizontal axis for the relative positions and a vertical axis for the normalized signal values.
2. **Plot Data Points**: We will plot data points for each nucleotide base at different relative positions.
3. **Highlight the 10-mers**: We will highlight the region corresponding to the 10-mers.

Here's a sample TikZ LaTeX code to achieve this:

```latex
\documentclass{standalone}
\usepackage{tikz}
\usetikzlibrary{shapes.geometric, arrows.meta}

\begin{document}

\begin{tikzpicture}[scale=0.8]

% Define the axes
\draw[->] (-5,0) -- (5,0) node[right] {Relative Position};
\draw[->] (0,-1) -- (0,4) node[above] {Normalized Signal Value};

% Plot data points
\foreach \x in {-4,-3,-2,-1,0,1,2,3,4} {
    \foreach \y in {0.2,0.4,0.6,0.8,1.0} {
        \fill (\x,\y) circle (2pt);
    }
}

% Highlight the 10-mers
\draw[thick, fill=gray!20] (-4,0) rectangle (1,1);

% Legend
\node at (-5,4.5) [anchor=east] {Legend};
\node at (-5,4) [anchor=east] {A};
\node at (-5,3.5) [anchor=east] {C};
\node at (-5,3) [anchor=east] {G};
\node at (-5,2.5) [anchor=east] {T};

% Labels for nucleotides
\node at (-4,-0.5) {A};
\node at (-3,-0.5) {C};
\node at (-2,-0.5) {G};
\node at (-1,-0.5) {T};
\node at (0,-0.5) {A};
\node at (1,-0.5) {C