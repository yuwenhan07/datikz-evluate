To create a TikZ LaTeX diagram that accurately represents the described geometric objects and their relationships, we can use the following code:

```latex
\documentclass[tikz,border=2mm]{standalone}
\usepackage{amsmath}

\tikzset{
    mycircle/.style={draw,circle},
    myrectangle/.style={draw,rectangle},
    mypoint/.style={fill,circle,inner sep=1pt}
}

\begin{document}
\begin{tikzpicture}[scale=2]
    % Draw K and -K
    \draw[thick,black] (-1,0) circle (1);
    \draw[dashed,thick,black] (1,0) circle (1);

    % Draw M_0(K, -K)
    \draw[orange,thick] plot[smooth cycle,tension=0.75] coordinates {(-1,0) (-0.5,1) (0.5,1) (1,0)};
    
    % Draw M_1(K, -K)
    \draw[blue,thick] plot[smooth cycle,tension=0.75] coordinates {(-1,0) (-0.5,1) (0.5,1) (1,0)};
    
    % Draw M_3(K, -K)
    \draw[red,thick] plot[smooth cycle,tension=0.75] coordinates {(-1,0) (-0.5,1) (0.5,1) (1,0)};
    
    % Draw M_infinity(K, -K)
    \draw[dashdotted,thick,blue] plot[smooth cycle,tension=0.75] coordinates {(-1,0) (-0.5,1) (0.5,1) (1,0)};
    
    % Mark common boundary points
    \node[mycircle,fill=white] at (0,0) {};
    
    % Mark vertices of M_1(K, -K)
    \node[mypoint,fill=white] at (-0.5,1) {};
    \node[mypoint,fill=white] at (0.5,1) {};
    
    % Label the objects
    \node[above left] at (-1,0) {$K$};
    \node[below right] at (1,0)