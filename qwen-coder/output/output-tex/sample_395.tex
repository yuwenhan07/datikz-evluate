To create a TikZ diagram that visually represents a spanning tree matching decoder for a $[[85,1,7]]$ surface code with three $\M{Z}$ channel errors and two alternative minimum spanning trees (MSTs), you can use the following LaTeX code:

```latex
\documentclass[tikz,border=3mm]{standalone}
\usetikzlibrary{shapes.geometric, positioning, arrows.meta}

\tikzset{
    node distance=2cm,
    block/.style={rectangle, draw, fill=white!20, text width=6em, text centered, rounded corners, minimum height=4em},
    line/.style={draw, -Stealth, thick}
}

\begin{document}
\begin{tikzpicture}[node distance=2cm]
    % Lattice nodes
    \foreach \x in {0,...,8} {
        \foreach \y in {0,...,8} {
            \node[fill=blue!20] (\x,\y) at (\x,\y) {};
        }
    }

    % Error nodes (red)
    \node[fill=red!50] (e1) at (3,3) {};
    \node[fill=red!50] (e2) at (5,5) {};
    \node[fill=red!50] (e3) at (7,7) {};

    % Ancilla nodes (green)
    \node[fill=green!50] (a1) at (1,9) {};
    \node[fill=green!50] (a2) at (9,1) {};

    % MST edges
    \draw[line] (e1) -- (a1);
    \draw[line] (e2) -- (a2);
    \draw[line] (e3) -- (a1);

    % Alternative MST edges
    \draw[dashed, line] (e1) -- (a2);
    \draw[dashed, line] (e2) -- (a1);
    \draw[dashed, line] (e3) -- (a2);

    % Decoder output
    \node[block] (decoder) [right=of e3, xshift=3cm] {Decoder};
    \node[block] (error) [right=of decoder, xshift=3cm] {$\mathcal{E}$