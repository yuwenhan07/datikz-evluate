Certainly! Below is a TikZ LaTeX code snippet that visually represents the relationship between the non-realizability of the quotient \( P \setminus e \) and the non-realizability of the non-Pappus matroid \( P \). The non-Pappus matroid \( P \) is depicted in blue.

```latex
\documentclass{standalone}
\usepackage{tikz}

\begin{document}

\begin{tikzpicture}[node distance=2cm, auto]

    % Define nodes for the elements of the matroid P
    \node (e1) [circle, draw] {1};
    \node (e2) [circle, draw] at (3,0) {2};
    \node (e3) [circle, draw] at (6,0) {3};
    \node (e4) [circle, draw] at (9,0) {4};

    % Draw edges to represent the independence structure of the matroid P
    \draw[blue] (e1) -- (e2);
    \draw[blue] (e1) -- (e3);
    \draw[blue] (e1) -- (e4);
    \draw[blue] (e2) -- (e3);
    \draw[blue] (e2) -- (e4);
    \draw[blue] (e3) -- (e4);

    % Node for the element e to be removed
    \node (e) [circle, draw, fill=red] at (5,-2) {$e$};

    % Nodes for the quotient matroid P \ e
    \node (e1_prime) [circle, draw, right=of e1] {1'};
    \node (e2_prime) [circle, draw, right=of e2] {2'};
    \node (e3_prime) [circle, draw, right=of e3] {3'};
    \node (e4_prime) [circle, draw, right=of e4] {4'};

    % Draw edges to represent the independence structure of the quotient matroid P \ e
    \draw (e1_prime) -- (e2_prime);
    \draw (e1_prime) -- (e3_prime);
    \draw (e1_prime) -- (e4_prime);
    \draw (e2_prime) -- (e3_prime);
