Certainly! Below is a LaTeX code using TikZ to create a visual representation of the described k-supertile pattern. This pattern consists of equilateral triangles of varying sizes, starting from the smallest at the center and increasing in size towards the edges.

```latex
\documentclass{standalone}
\usepackage{tikz}

\begin{document}

\begin{tikzpicture}[scale=1.5]
    % Define colors
    \definecolor{background}{RGB}{255,255,255} % White background

    % Set up the coordinate system
    \newcommand{\tileSize}[1]{#1}
    \newcommand{\centerTileSize}{2cm}

    % Draw the central tile (k=0)
    \fill[background] (-1, -1) rectangle (1, 1);
    
    % Function to draw a triangle at given coordinates
    \newcommand{\drawTriangle}[4]{
        \fill[#2] (#1-\tileSize{#3}, #1+\tileSize{#3}/sqrt(3)) -- 
                  (#1, #1-\tileSize{#3}) -- 
                  (#1+\tileSize{#3}, #1+\tileSize{#3}/sqrt(3));
    }

    % Draw the outermost tiles (k=3)
    \foreach \x in {-3,-2,-1,0,1,2,3}{
        \foreach \y in {-3,-2,-1,0,1,2,3}{
            \ifnum\x<2 \ifnum\x>-2 \ifnum\y<2 \ifnum\y>-2
                \drawTriangle{\x}{background}{#3}
            \fi\fi\fi\fi
        }
    }

    % Draw the inner tiles (k=2)
    \foreach \x in {-2,-1,0,1,2}{
        \foreach \y in {-2,-1,0,1,2}{
            \ifnum\x<1 \ifnum\x>-1 \ifnum\y<1 \ifnum\y>-1
                \drawTriangle{\x}{background}{#3}
            \fi\fi\fi\fi
        }
    }

    % Draw the middle tiles (k=1)
    \foreach \x in {-1,0,1}{
        \foreach \y in {-1