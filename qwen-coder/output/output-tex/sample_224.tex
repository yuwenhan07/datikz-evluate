To create a TikZ LaTeX diagram that visualizes the posterior mean of total \( R^2 \) over time for different priors, we can use the following code. This example assumes you have defined the colors `arr2flatcolour`, `arr2minncolour`, `sparsenessinducingcolour`, `minnesotacolour`, and `rhscolour` in your preamble.

```latex
\documentclass[tikz,border=10pt]{standalone}
\usepackage{xcolor}

% Define custom colors (if not already defined)
\definecolor{arr2flatcolour}{RGB}{255, 0, 0} % Red
\definecolor{arr2minncolour}{RGB}{0, 255, 0} % Green
\definecolor{sparsenessinducingcolour}{RGB}{0, 0, 255} % Blue
\definecolor{minnesotacolour}{RGB}{128, 128, 0} % Olive
\definecolor{rhscolour}{RGB}{160, 32, 240} % Purple

\begin{document}
\begin{tikzpicture}[scale=0.7]

% Axes
\draw[->] (-1,0) -- (10,0) node[right] {Time};
\draw[->] (0,-1) -- (0,5) node[above] {$R^2$};

% Labels for priors
\node at (0.5, -1.5) [below] {\textcolor{arr2flatcolour}{ARR2 flat}};
\node at (2.5, -1.5) [below] {\textcolor{arr2minncolour}{ARR2 Minnesota}};
\node at (4.5, -1.5) [below] {\textcolor{sparsenessinducingcolour}{Sparsity-inducing}};
\node at (6.5, -1.5) [below] {\textcolor{minnesotacolour}{Minnesota-type}};
\node at (8.5, -1.5) [below] {\textcolor{rhscolour}{Regularised Horseshoe}};

% Data points (example data)
\foreach \x/\y in {
    1/0.3,
    2/0.4,
    3/0