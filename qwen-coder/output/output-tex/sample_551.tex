Certainly! Below is a LaTeX code using TikZ that attempts to recreate the described structure. This example includes a line with various symbols and numbers along its length, which seems to represent a loop game or some kind of structured diagram.

```latex
\documentclass[tikz,border=3mm]{standalone}
\usetikzlibrary{arrows.meta, positioning}

\tikzset{
    node distance=2cm,
    every node/.style={draw, circle, fill=white},
    label distance=-1pt
}

\begin{document}
\begin{tikzpicture}[thick]
    % Nodes along the line
    \node (start) at (0,0) {Start};
    \node (l0) [right=of start] {$l_0$};
    \node (l1) [right=of l0] {$l_1$};
    \node (l2) [right=of l1] {$l_2$};
    \node (l3) [right=of l2] {$l_3$};
    \node (l4) [right=of l3] {$l_4$};
    \node (lEnd) [right=of l4] {End};

    % Arrows connecting nodes
    \draw[-Stealth] (start) -- (l0);
    \draw[-Stealth] (l0) -- (l1);
    \draw[-Stealth] (l1) -- (l2);
    \draw[-Stealth] (l2) -- (l3);
    \draw[-Stealth] (l3) -- (l4);
    \draw[-Stealth] (l4) -- (lEnd);

    % Adding labels along the arrows
    \foreach \i/\j in {0/$l_0$, 1/$l_1$, 2/$l_2$, 3/$l_3$, 4/$l_4$} {
        \path (l\i.west) -- node[anchor=east] {\j} (l\i.east);
    }

    % Adding additional symbols and numbers as needed
    \node[below=of start] {Symbol A};
    \node[above=of lEnd] {Symbol B};
    \node[right=of l2] {Number 7};
    \node[left=of l3] {Number 5