To create a TikZ LaTeX diagram that represents the multiplication on \( L \) determined by the complex numbers \( m_{(x,i),(y,j),l}^{(z,k)} \), we need to visualize this concept in a structured manner. Here's a step-by-step approach to creating such a diagram:

1. **Define the sets involved**: We have four sets: \( X, Y, Z, \) and \( L \).
2. **Represent elements from these sets**: Each element will be represented as a node.
3. **Connect nodes with arrows** to represent the multiplication operation.

Here's a sample TikZ code to achieve this:

```latex
\documentclass{standalone}
\usepackage{tikz}

\begin{document}

\begin{tikzpicture}[node distance=2cm]
    % Define styles for nodes
    \tikzset{
        set/.style={circle, draw, fill=blue!20, minimum size=1cm},
        element/.style={rectangle, draw, fill=red!20, minimum size=1cm}
    }

    % Define sets
    \node[set] (X) {\( X \)};
    \node[set] (Y) [below left of=X] {\( Y \)};
    \node[set] (Z) [below right of=X] {\( Z \)};
    \node[set] (L) [below of=Y] {\( L \)};

    % Define elements
    \node[element] (x1) [left of=X] {\( x_1 \)};
    \node[element] (x2) [right of=X] {\( x_2 \)};
    \node[element] (y1) [left of=Y] {\( y_1 \)};
    \node[element] (y2) [right of=Y] {\( y_2 \)};
    \node[element] (z1) [left of=Z] {\( z_1 \)};
    \node[element] (z2) [right of=Z] {\( z_2 \)};
    \node[element] (l1) [left of=L] {\( l_1 \)};
    \node[element] (l2) [right of=L] {\( l_2 \)};

    % Draw connections
    \draw[-stealth] (x1) -- (y1) node[midway, above