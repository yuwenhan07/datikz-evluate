To create a TikZ diagram that visually represents the described configuration, we can use the following LaTeX code. This code will generate a grid-like structure with specified numbers of colored lines in each direction.

```latex
\documentclass[tikz,border=5mm]{standalone}
\usepackage{xcolor}

% Define colors
\definecolor{red}{rgb}{1,0,0}
\definecolor{blue}{rgb}{0,0,1}
\definecolor{orange}{rgb}{1,0.647,0}

\begin{document}
\begin{tikzpicture}[scale=2]

% Draw the grid
\draw[thick] (0,0) rectangle (4,4);

% Left panel description
\node[left] at (-0.5,2) {Fix $n = 3$};
\node[left] at (-0.5,1) {$i = (1, 0, 0)$};
\node[left] at (-0.5,0) {$j = (0, 1, 1)$};
\node[left] at (-0.5,-1) {$k = (1, 1, 1)$};
\node[left] at (-0.5,-2) {$l = (0, 0, 0)$};

% Right panel illustration
\foreach \x in {0,...,3}{
    \foreach \y in {0,...,3}{
        % Bottom
        \ifnum\x>0
            \draw[red, thick] (\x-0.5,\y+0.5) -- (\x+0.5,\y+0.5);
        \fi
        % Left
        \ifnum\y>0
            \draw[blue, thick] (\x+0.5,\y-0.5) -- (\x+0.5,\y+0.5);
        \fi
        % Top
        \ifnum\y<3
            \draw[orange, thick] (\x+0.5,\y+0.5) -- (\x+0.5,\y+1.5);
        \fi
        % Right
        \ifnum\x<3
            \draw[green, thick] (\x+1.5,\y+0.5) -- (\x+0.5,\y+0.5);
        \fi
