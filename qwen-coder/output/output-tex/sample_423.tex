Certainly! Below is a TikZ LaTeX code that visualizes the $\sqrt{n}$-decomposition of a group with processes labeled as \(a, b, c, d, e\) and includes the described communication arrows.

```latex
\documentclass{article}
\usepackage{tikz}

\begin{document}

\begin{figure}[h]
    \centering
    \begin{tikzpicture}[level distance=2cm,
        level 1/.style={sibling distance=4cm},
        level 2/.style={sibling distance=2cm}]
        
        % Nodes for the group decomposition
        \node[draw, circle] (root) {Root};
        \node[draw, circle, below left of=root] (child1) {Child 1};
        \node[draw, circle, below right of=root] (child2) {Child 2};
        \node[draw, circle, below left of=child1] (grandchild1) {Grandchild 1};
        \node[draw, circle, below right of=child1] (grandchild2) {Grandchild 2};
        \node[draw, circle, below left of=child2] (grandchild3) {Grandchild 3};
        \node[draw, circle, below right of=child2] (grandchild4) {Grandchild 4};
        
        % Pink arrows for the three-round process
        \draw[-stealth, thick, pink] (root) ++(0,-0.5) node[left] {\#1} |- (child1);
        \draw[-stealth, thick, pink] (root) ++(0,-0.5) node[right] {\#1} |- (child2);
        \draw[-stealth, thick, pink] (child1) ++(0,-0.5) node[left] {\#1} |- (grandchild1);
        \draw[-stealth, thick, pink] (child1) ++(0,-0.5) node[right] {\#1} |- (grandchild2);
        \draw[-stealth, thick, pink] (child2) ++(0,-0.5) node[left] {\#1} |- (grandchild3);
        \draw[-stealth, thick, pink] (child2) ++(0,-0.5) node[right] {\#1} |- (