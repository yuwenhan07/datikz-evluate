Certainly! Below is the TikZ LaTeX code to generate two octaves of the Temperament $\mathcal{T}_5(100)$ together with the interval $\Delta_5$.

```latex
\documentclass[tikz,border=2mm]{standalone}
\usetikzlibrary{calc,arrows.meta}

\tikzset{
    mynode/.style={circle, draw, inner sep=2pt},
    myarrow/.style={->, thick, >=Stealth[]}
}

\begin{document}
\begin{tikzpicture}[scale=1.5]
    % Define the intervals for one octave
    \def\noteA{0}
    \def\noteB{\noteA + 100/12}
    \def\noteC{\noteB + 100/12}
    \def\noteD{\noteC + 100/12}
    \def\noteE{\noteD + 100/12}
    \def\noteF{\noteE + 100/12}
    \def\noteG{\noteF + 100/12}
    \def\noteAup{\noteG + 100/12}

    % Draw notes for one octave
    \foreach \i in {0,...,7} {
        \node[mynode] (note\i) at (\i,0) {};
    }
    
    % Draw arrows between notes for one octave
    \foreach \i in {0,...,6} {
        \draw[myarrow] (note\i) -- (note\the\numexpr\i+1);
    }

    % Define the second octave starting from the first note
    \foreach \i in {8,...,15} {
        \node[mynode] (note\i) at (\i,1) {};
    }
    
    % Draw arrows between notes for the second octave
    \foreach \i in {8,...,14} {
        \draw[myarrow] (note\i) -- (note\the\numexpr\i+1);
    }

    % Draw Delta_5 as an arrow from the last note of the first octave to the first note of the second octave
    \draw[myarrow, dashed, thick] (note7.east) -- (note8.west