Certainly! Below is an example of a TikZ LaTeX diagram representing a stochastic control model (SCM) with a comonotonic copula, where the noise nodes \( U_t \) and \( V_t \) are scalars rather than vectors.

```latex
\documentclass{standalone}
\usepackage{tikz}

\tikzset{
    node distance=2cm,
    every node/.style={circle, draw, align=center},
    process/.style={rectangle, draw, rounded corners, align=center},
    arrow/.style={thick,->,>=stealth}
}

\begin{document}
\begin{tikzpicture}[scale=1.5]

% Nodes
\node (X0) at (0, 0) {Initial State};
\node (X1) at (-3, -2) {Control Policy};
\node (U1) at (-6, -4) {$U_1$};
\node (V1) at (-3, -4) {$V_1$};
\node (Y1) at (0, -4) {$Y_1$};
\node (X2) at (-3, -6) {Control Policy};
\node (U2) at (-6, -8) {$U_2$};
\node (V2) at (-3, -8) {$V_2$};
\node (Y2) at (0, -8) {$Y_2$};

% Arrows
\draw[arrow] (X0) -- node[above]{$x_0$} (X1);
\draw[arrow] (X1) -- node[below]{$u_1$} (U1);
\draw[arrow] (X1) -- node[below]{$v_1$} (V1);
\draw[arrow] (U1) -- node[right]{$\epsilon_1^U$} (Y1);
\draw[arrow] (V1) -- node[right]{$\epsilon_1^V$} (Y1);
\draw[arrow] (Y1) -- node[above]{$y_1$} (X2);
\draw[arrow] (X2) -- node[below]{$u_2$} (U2);
\draw[arrow] (X2) -- node[below]{$v_2$} (V2);
