Certainly! Below is a TikZ LaTeX code that represents the described protocol for two iterations:

```latex
\documentclass[tikz,border=2cm]{standalone}
\usetikzlibrary{quantikz}

\tikzset{
    mybox/.style={draw=black, thick, rectangle, minimum width=4cm, minimum height=1cm},
    mycircle/.style={draw=black, thick, circle, radius=0.5cm},
    myarrow/.style={->,thick},
    mynode/.style={font=\small, text=black}
}

\begin{document}
\begin{quantikz}[row sep=1em]
    & \gate{U_c} & \qw \\
    \ctrl{1} & \qw & \qw \\
    \qw & \qw & \qw \\
    \qw & \qw & \qw \\
    \qw & \qw & \qw \\
    \qw & \qw & \qw \\
    \qw & \qw & \qw \\
    \qw & \qw & \qw \\
    \qw & \qw & \qw \\
    \qw & \qw & \qw \\
    \qw & \qw & \qw \\
    \qw & \qw & \qw \\
    \qw & \qw & \qw \\
    \qw & \qw & \qw \\
    \qw & \qw & \qw \\
    \qw & \qw & \qw \\
    \qw & \qw & \qw \\
    \qw & \qw & \qw \\
    \qw & \qw & \qw \\
    \qw & \qw & \qw \\
    \qw & \qw & \qw \\
    \qw & \qw & \qw \\
    \qw & \qw & \qw \\
    \qw & \qw & \qw \\
    \qw & \qw & \qw \\
    \qw & \qw & \qw \\
    \qw & \qw & \qw \\
    \qw & \qw & \qw \\
    \qw & \qw & \qw \\
    \qw & \qw & \qw \\
    \qw & \qw & \qw \\
    \qw & \qw & \qw \\
    \qw & \qw & \qw \\
    \qw & \qw & \qw \\
    \qw & \qw & \qw \\
    \qw & \qw & \qw \\
    \qw & \