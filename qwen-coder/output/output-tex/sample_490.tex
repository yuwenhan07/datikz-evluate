Certainly! Below is an example of how you can create a TikZ diagram to illustrate estimated partitions over time based on both the independent Chinese Restaurant Process (CRP) model and your temporally-dependent random partitions model. This example will include two plots side by side, with different color schemes representing the cluster labels.

```latex
\documentclass{standalone}
\usepackage{tikz}
\usetikzlibrary{plotmarks}

\begin{document}

\begin{tikzpicture}[scale=0.8]
    % Define colors for clusters
    \definecolor{cluster1}{RGB}{255, 99, 71} % Red
    \definecolor{cluster2}{RGB}{54, 162, 235} % Blue
    \definecolor{cluster3}{RGB}{255, 206, 86} % Yellow
    \definecolor{cluster4}{RGB}{75, 192, 192} % Teal

    % Time steps
    \def\timesteps{5}

    % Independent CRP Model Plot
    \begin{scope}[xshift=-3cm]
        \draw[thick,->] (0,0) -- (\timesteps,0);
        \foreach \i in {1,...,\timesteps}{
            \fill[cluster\i] (0.5*\i-0.25,-0.1) rectangle (0.5*\i+0.25,0.1);
        }
        \node at (-1,-0.2) [below] {Independent CRP};
        \foreach \i in {1,...,\timesteps}{
            \node at (0.5*\i, -0.3) [below] {\i};
        }
    \end{scope}

    % Temporally-Dependent Random Partitions Model Plot
    \begin{scope}[xshift=3cm]
        \draw[thick,->] (0,0) -- (\timesteps,0);
        \foreach \i in {1,...,\timesteps}{
            \ifnum\i=1
                \fill[cluster1] (0.5*\i-0.25,-0.1) rectangle (0.5*\i+0.25,0.1);
            \else
                \pgfmathsetmacro{\