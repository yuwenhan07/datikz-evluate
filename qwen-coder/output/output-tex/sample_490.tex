Certainly! To create lines connecting values from the first column to values from the second column in a LaTeX table using TikZ, you can use the `TikZ` package along with the `matrix` library for handling tables. Below is an example of how you might achieve this:

```latex
\documentclass{article}
\usepackage{tikz}
\usetikzlibrary{matrix}

\begin{document}

\begin{table}[h]
    \centering
    \begin{tabular}{|c|c|}
        \hline
        \textbf{Year} & \textbf{Values} \\
        \hline
        2018 & A \\
        2019 & B \\
        2020 & C \\
        2021 & D \\
        \hline
    \end{tabular}
    \caption{Example Table}
    \label{tab:example}
\end{table}

% Add TikZ code to connect values from the first column to the second column
\begin{tikzpicture}[overlay, remember picture]
    % Define nodes for each cell in the first column
    \node (n1) at ([yshift=-3ex]pic cs:n1) {};
    \node (n2) at ([yshift=-3ex]pic cs:n2) {};
    \node (n3) at ([yshift=-3ex]pic cs:n3) {};
    \node (n4) at ([yshift=-3ex]pic cs:n4) {};

    % Define nodes for each cell in the second column
    \node (v1) at ([xshift=3cm,yshift=-3ex]pic cs:v1) {};
    \node (v2) at ([xshift=3cm,yshift=-3ex]pic cs:v2) {};
    \node (v3) at ([xshift=3cm,yshift=-3ex]pic cs:v3) {};
    \node (v4) at ([xshift=3cm,yshift=-3ex]pic cs:v4) {};

    % Draw lines connecting corresponding cells
    \draw[-stealth] (n1) -- (v1);
    \draw[-stealth] (n2) -- (v2);
    \draw[-stealth] (n3) -- (v3);
    \draw[-stealth] (n4