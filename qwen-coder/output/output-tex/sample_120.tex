To create a TikZ LaTeX diagram that visualizes the number of steps taken by an RBQL (Randomized Backtracking with Learning) agent in a \(50 \times 50\) grid world maze, we can use the following code. This example will include a histogram to represent the distribution of steps, with the light green area representing the range of the highest and lowest step counts (excluding the top and bottom outliers), and a green line indicating the average performance.

```latex
\documentclass{standalone}
\usepackage{tikz}
\usetikzlibrary{calc, decorations.pathmorphing}

\begin{document}
\begin{tikzpicture}[scale=1]
    % Define colors
    \definecolor{lightgreen}{rgb}{0.6, 1, 0.6}
    \definecolor{darkgreen}{rgb}{0, 0.75, 0}

    % Data points for the histogram
    \def\data{{300, 320, 340, 360, 380, 400, 420, 440, 460, 480}}
    
    % Calculate minimum, maximum, and average
    \pgfmathsetmacro{\minVal}{min(\data)}
    \pgfmathsetmacro{\maxVal}{max(\data)}
    \pgfmathsetmacro{\avgVal}{mean(\data)}

    % Define the width of each bar
    \def\barWidth{15pt}
    \def\barHeightFactor{0.005} % Adjust this factor to control the height of the bars

    % Draw the histogram
    \foreach \step [count=\i] in \data {
        \draw[fill=darkgreen!50!white] (\i*\barWidth-7.5pt,0) rectangle ++(\barWidth,-\step*\barHeightFactor);
    }

    % Highlight the range of the highest and lowest step counts (excluding the top and bottom outliers)
    \draw[fill=lightgreen!50!white] (1*\barWidth-7.5pt,0) rectangle ++(9*\barWidth,-\minVal*\barHeightFactor);
    \draw[fill=lightgreen!50!white] (1*\barWidth-7.5pt,0) rectangle ++(9*\