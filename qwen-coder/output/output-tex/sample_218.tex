Certainly! Below is a TikZ LaTeX code that visualizes the example you provided:

```latex
\documentclass{standalone}
\usepackage{tikz}

\begin{document}

\begin{tikzpicture}[level distance=1.5cm,
                    level 1/.style={sibling distance=3cm},
                    level 2/.style={sibling distance=2cm},
                    level 3/.style={sibling distance=1cm}]
    % Nodes
    \node (root) [circle, draw] {T};
    child { node (n1) [circle, draw] {7} 
        child { node (n2) [circle, draw] {2} }
        child { node (n3) [circle, draw] {1}
            child { node (n4) [circle, draw] {3} }
            child { node (n5) [circle, draw] {4} }
        }
    };
    
    % Edges
    \draw[red] (root) -- (n1);
    \draw[red] (n1) -- (n2);
    \draw[red] (n1) -- (n3);
    \draw[red] (n3) -- (n4);
    \draw[red] (n3) -- (n5);

    % Labels
    \node at ([yshift=-1cm]n1.east) {7};
    \node at ([yshift=-1cm]n2.west) {2};
    \node at ([yshift=-1cm]n3.west) {1};
    \node at ([yshift=-1cm]n4.east) {3};
    \node at ([yshift=-1cm]n5.east) {4};

    % Binomial numbers and k indicators
    \node at ([xshift=1.5cm,yshift=-1.5cm]root.south) {B_1: 1};
    \node at ([xshift=2.5cm,yshift=-1.5cm]n1.south) {B_2: 1};
    \node at ([xshift=3.5cm,yshift=-1.5cm]n2.south) {B_3: 1};
    \node at ([xshift=4.5cm,yshift=-1.5cm]n3