To generate a TikZ diagram for the \( E^3 \)-page of the spectral sequence for computing \( H_*(\mathcal{B}G; \mathbb{L}_\bullet\langle1\rangle) \), we need to understand the structure of the spectral sequence and the relevant terms and differentials up to \( E^3 \). Here, we will assume that the spectral sequence is related to the homology of the classifying space \( \mathcal{B}G \) with coefficients in the completed group ring \( \mathbb{L}_\bullet\langle1\rangle \).

For simplicity, let's consider a hypothetical example where the spectral sequence has a specific form. We'll use a 2-dimensional grid to represent the \( E^3 \)-page.

```latex
\documentclass[tikz,border=10pt]{standalone}
\usepackage{tikz-cd}

\tikzset{
    mycell/.style={draw, rectangle, minimum size=1cm, text centered},
    myarrow/.style={->, thick}
}

\begin{document}
\begin{tikzpicture}[node distance=1.5cm]
    % Nodes for E^3-page
    \node (E300) at (0,0) [mycell] {$E^{3,0}_3$};
    \node (E310) at (1.5,0) [mycell] {$E^{3,1}_3$};
    \node (E320) at (3,0) [mycell] {$E^{3,2}_3$};
    
    \node (E301) at (0,-1.5) [mycell] {$E^{3,0}_3$};
    \node (E311) at (1.5,-1.5) [mycell] {$E^{3,1}_3$};
    \node (E321) at (3,-1.5) [mycell] {$E^{3,2}_3$};
    
    % Arrows for E^3-page
    \draw[myarrow] (E300) -- node[above]{$d_{3,0}$} (E310);
    \draw[myarrow] (E310) -- node[above]{$d_{3,1}$}