To create a TikZ diagram illustrating the stencil for the diffusion and advection cases, we can use the following code:

```latex
\documentclass[tikz]{standalone}
\usetikzlibrary{matrix}

\tikzset{
    node style/.style={draw=black, fill=white, circle, inner sep=2pt},
    line style/.style={thick, ->, >=stealth'}
}

\begin{document}
\begin{tikzpicture}[node distance=1cm]
    \matrix (diffusion) [matrix of nodes, row sep=1cm, column sep=1cm] {
        & & V_{i-1,j} & V_i & V_{i+1,j} & \\
        & & V_{i-1,j-1} & V_{i,j-1} & V_{i+1,j-1} & \\
        & & V_{i-1,j+1} & V_{i,j+1} & V_{i+1,j+1} & \\
    };
    \foreach \i in {0,...,4}{
        \foreach \j in {0,...,2}{
            \node[below left=0.5cm and 0.5cm of diffusion-\i-\j] {$\partial^{(h)}_{x_j,x_j}V$};
        }
    }

    \matrix (advection) [matrix of nodes, below=2cm of diffusion, row sep=1cm, column sep=1cm] {
        & & V_{i-1,j} & V_i & V_{i+1,j} & \\
        & & V_{i-1,j-1} & V_{i,j-1} & V_{i+1,j-1} & \\
        & & V_{i-1,j+1} & V_{i,j+1} & V_{i+1,j+1} & \\
    };
    \foreach \i in {0,...,4}{
        \foreach \j in {0,...,2}{
            \node[below left=0.5cm and 0.5cm of advection-\i-\j] {$\partial^{(h)}_{x_j}V$};
        }
    }

    % Draw lines for diffusion stencils
    \draw[line style] (diffusion-2-3) -- (diffusion-2-4);
    \