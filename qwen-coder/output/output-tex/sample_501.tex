To create a TikZ diagram that visualizes the given structure, we need to represent the group \( G = \mathbb{Z}_{90} \), the subsets \( S_0 \) and its cosets modulo the subgroups \( H_1, H_2, \) and \( H_3 \). Here’s how you can do it:

```latex
\documentclass[tikz,border=10pt]{standalone}
\usetikzlibrary{shapes.geometric, positioning}

\begin{document}

\tikzset{
    node distance=2cm,
    circle/.style={circle, draw, inner sep=2pt},
    rectangle/.style={rectangle, draw, inner sep=2pt},
    every node/.style={font=\small}
}

\begin{tikzpicture}[level distance=2cm]

% Level 0: G = Z_90
\node (G) [circle] {$G = \mathbb{Z}_{90}$};

% Level 1: H_1 = <45>
\node (H1) [below left=of G, circle] {$H_1 = \langle 45 \rangle$};
\node (coset_H1_1) [right=of H1, rectangle, red] {0};
\node (coset_H1_2) [right=of coset_H1_1, rectangle, red] {45};

% Level 2: H_2 = <15>
\node (H2) [below left=of H1, circle] {$H_2 = \langle 15 \rangle$};
\node (coset_H2_1) [right=of H2, rectangle, red] {0};
\node (coset_H2_2) [right=of coset_H2_1, rectangle, red] {15};
\node (coset_H2_3) [right=of coset_H2_2, rectangle, red] {30};
\node (coset_H2_4) [right=of coset_H2_3, rectangle, red] {45};
\node (coset_H2_5) [right=of coset_H2_4, rectangle, red] {60};
\node (coset_H2_6) [right=of coset_H2_5, rectangle,