\documentclass[tikz,border=10pt]{standalone}
\usetikzlibrary{shapes.geometric, arrows.meta}

\tikzset{
    ring/.style={draw, rectangle, rounded corners, minimum size=2cm},
    arrow/.style={thick, ->, >=Stealth[length=5mm]}
}

\begin{document}
\begin{tikzpicture}[node distance=3cm]

    % Define nodes for different classes of rings
    \node[ring] (CommutativeRings) {Commutative Rings};
    \node[ring, below left of=CommutativeRings] (IntegralDomains) {Integral Domains};
    \node[ring, below right of=CommutativeRings] (PID) {Principal Ideal Domains};
    \node[ring, below of=PID] (EuclideanDomains) {Euclidean Domains};
    \node[ring, below of=EuclideanDomains] (UniqueFactorizationDomains) {Unique Factorization Domains};
    \node[ring, below of=UniqueFactorizationDomains] (BézoutDomains) {Bézout Domains};
    \node[ring, below of=BézoutDomains, yshift=-1cm] (ElementaryDivisorDomains) {Elementary Divisor Domains};

    % Draw arrows indicating inclusion relations
    \draw[arrow] (CommutativeRings) -- (IntegralDomains);
    \draw[arrow] (IntegralDomains) -- (PID);
    \draw[arrow] (PID) -- (EuclideanDomains);
    \draw[arrow] (EuclideanDomains) -- (UniqueFactorizationDomains);
    \draw[arrow] (UniqueFactorizationDomains) -- (BézoutDomains);
    \draw[arrow, dashed] (BézoutDomains) -- (ElementaryDivisorDomains);

\end{tikzpicture}
\end{document}