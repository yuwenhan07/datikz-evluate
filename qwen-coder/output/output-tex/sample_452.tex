Certainly! Below is a TikZ LaTeX code that visually represents the described scenario:

```latex
\documentclass[tikz,border=3mm]{standalone}
\usetikzlibrary{arrows.meta,backgrounds,calc,decorations.pathreplacing,fit,matrix,patterns,positioning,shapes.geometric,shadings,trees}

\begin{document}
\begin{tikzpicture}[scale=2]
    % Time axis
    \draw[->] (-2,0) -- (4,0) node[right] {$t$};
    
    % Cauchy surfaces
    \foreach \t in {0, 1, 2, 3} {
        \ifnum\t=0
            \draw[line width=2pt, purple] plot[domain=-2:4,samples=100] (\x,{sin(\x r)*2});
            \node at (0,-1.5) {$\Sigma_{t=0} = \Sigma_P \cup \partial \Sigma \cup \Sigma_{P'}$};
        \else
            \draw[dashed, purple] plot[domain=-2:4,samples=100] (\x,{sin((\x-1)r)*2});
            \node at (\t,-1.5) {$\Sigma_t$};
        \fi
    }
    
    % Entangling surface
    \draw[thick, dashed, blue] (-2,0) to[out=90,in=270] ++(0,-1);
    \draw[thick, dashed, blue] (4,0) to[out=90,in=270] ++(0,-1);
    \node at (1,-2) {$\partial \Sigma$};
    
    % Time band
    \draw[thick, red, ->] (1,-3) -- (3,-3) node[midway, below] {\boldsymbol{$B$}};
    
    % Timelike envelope
    \draw[fill opacity=0.2, gray] (1,-3) rectangle (3,-4);
    \node at (2,-4) [below] {\textit{\large${\cal E}(\boldsymbol{B})$}};
    
    % Entangling surface for the wedge
    \draw[thick, dashed, green] (1,-4) to[out=