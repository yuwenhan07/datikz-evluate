To generate a TikZ LaTeX diagram for the graph \( E_{(m_1, m_2, m_3, m_4, m_5)} \) of the Cuntz pentagon \( \mathcal{P}_{(m_1, m_2, m_3, m_4, m_5)} \), we need to understand the structure of the Cuntz pentagon and how its edges are labeled.

The Cuntz pentagon \( \mathcal{P}_{(m_1, m_2, m_3, m_4, m_5)} \) is a directed graph with five vertices labeled \( v_0, v_1, v_2, v_3, v_4 \). Each vertex \( v_i \) has an outgoing edge labeled by each integer from 1 to \( m_i \).

Here's the TikZ LaTeX code to draw this graph:

```latex
\documentclass{standalone}
\usepackage{tikz}

\begin{document}
\begin{tikzpicture}[scale=1.5]
    % Define the vertices
    \node (v0) at (0,0) {};
    \node (v1) at (1,0) {};
    \node (v2) at (1,-1) {};
    \node (v3) at (0,-1) {};
    \node (v4) at (-0.5,-0.5) {};

    % Draw the edges
    \foreach \i in {1,...,3} {
        \draw[->] (v0) -- node[above] {$\i$} (v1);
    }
    \foreach \i in {1,...,2} {
        \draw[->] (v1) -- node[right] {$\i$} (v2);
    }
    \foreach \i in {1,...,3} {
        \draw[->] (v2) -- node[below] {$\i$} (v3);
    }
    \foreach \i in {1,...,2} {
        \draw[->] (v3) -- node[left] {$\i$} (v4);
    }
    \foreach \i in {1,...,3} {
        \draw[->] (v4) -- node[above left] {$\i$} (v0);
    }

