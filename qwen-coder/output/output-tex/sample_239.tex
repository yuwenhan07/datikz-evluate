To create a TikZ diagram that visualizes \( B^*_k \) for the specific case where \( n = 5 \) and \( k = 2 \), we need to consider the structure and properties described. Here's a step-by-step guide to generating the TikZ code:

1. **Define the grid size**: Since \( n = 5 \), the grid will have 5 rows and 5 columns.
2. **Shadowing rules**: 
   - Light red cells indicate shadowed positions.
   - The second row and second column are shadowed.
   - Numbers in these cells indicate the layer with a \(\mathcal{P}\) position.
3. **Periodic structure**: The lcm of the periods of 0 to 4 with center 2 is 12, meaning \( B^*_{k+1} \) starts 12 rows down from \( B^*_k \).

Given these points, let's write the TikZ code:

```latex
\documentclass[tikz]{standalone}
\usetikzlibrary{matrix}

\begin{document}
\begin{tikzpicture}
    \matrix (grid) [matrix of nodes,
                    nodes in empty cells,
                    nodes={draw, minimum width=2em, minimum height=2em},
                    row sep=-\pgflinewidth,
                    column sep=-\pgflinewidth] {
        & & & & \\
        & 1 & 2 & 3 & 4 \\
        & 5 & 6 & 7 & 8 \\
        & 9 & 10 & 11 & 12 \\
        & 13 & 14 & 15 & 16 \\
    };

    % Shadow the second row and second column
    \foreach \i in {1,...,5} {
        \fill[light red] (grid-2-\i) rectangle ([yshift=-\pgflinewidth,xshift=\pgflinewidth]grid-2-\i.south west);
        \fill[light red] (grid-\i-2) rectangle ([xshift=-\pgflinewidth,yshift=\pgflinewidth]grid.south west-\i-2.east);
    }

    % Fill in the numbers
    \node at (grid-2-2) {1};
    \node at (grid-2-3) {2};
    \node at