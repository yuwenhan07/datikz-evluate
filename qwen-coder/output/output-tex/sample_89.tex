Certainly! Below is a TikZ LaTeX code snippet that illustrates the concept of saddle moves and twist moves for converting 2-component links into knots. The code includes comments to explain each part of the diagram.

```latex
\documentclass[tikz,border=3mm]{standalone}
\usetikzlibrary{topaths}

\tikzset{
    link/.style={thick, draw},
    twist/.style={purple, thick, ->},
    cross/.style={red, thick, ->},
    knot/.style={blue, thick, <->}
}

\begin{document}

\begin{tikzpicture}[scale=1]
    % Draw the initial 2-component link
    \draw[link] (0,0) -- (2,2);
    \draw[link] (2,-2) -- (0,0);

    % Highlight the crossings
    \fill[red] (1,1) circle (2pt);
    \fill[red] (1,-1) circle (2pt);

    % Indicate the twist move with a purple arrow
    \draw[twist] (1,1) .. controls (1.5,1.5) and (1.5,-1.5) .. (1,-1);

    % Perform the twist move
    \draw[link] (1,1) -- (1,-1);
    \draw[link] (1,1) -- (1,-1);

    % Highlight the new crossings after the twist move
    \fill[red] (1,1) circle (2pt);
    \fill[red] (1,-1) circle (2pt);

    % Indicate the crossing change with a red arrow
    \draw[cross] (1,1) .. controls (1.5,1.5) and (1.5,-1.5) .. (1,-1);

    % Perform the crossing change
    \draw[knot] (1,1) -- (1,-1);
    \draw[knot] (1,1) -- (1,-1);

    % Finalize the knot
    \draw[link] (1,1) -- (1,-1);
    \draw[link] (1,1) -- (1,-1);

    % Labeling the steps
    \node at (4,2) {Initial Link};
    \node at (4,-2)