Certainly! Below is a TikZ LaTeX code that represents the described scenario:

```latex
\documentclass[tikz,border=10pt]{standalone}
\usetikzlibrary{arrows.meta,calc}

\begin{document}

\begin{tikzpicture}[scale=1.5]
    % Axes
    \draw[->] (-1,0) -- (12,0) node[right] {$s$};
    \draw[->] (0,-1) -- (0,4) node[above] {$\lambda$};

    % Critical point
    \coordinate (critical_point) at (10,0);
    \fill (critical_point) circle[radius=2pt] node[below right] {$s^* = 10\pi^2$};

    % Diagonal lines for EBL bifurcations
    \draw[dashed, thick] ($(critical_point) + (-1,-1)$) -- ($(critical_point) + (1,1)$);
    \draw[dashed, thick] ($(critical_point) + (-1,1)$) -- ($(critical_point) + (1,-1)$);

    % Horizontal dashed line for BLIS bifurcation
    \draw[dashed, thick] ($(critical_point) - (1,-0.5)$) -- ($(critical_point) + (1,-0.5)$);

    % Vertical dashed line for another BLIS bifurcation
    \draw[dashed, thick] ($(critical_point) - (0.5,-0.5)$) -- ($(critical_point) - (0.5,0.5)$);

    % Labels for line types
    \node[anchor=north east] at (8,-0.75) {EBL};
    \node[anchor=south west] at (11,-0.75) {BLIS};
    \node[anchor=north west] at (6.5,-0.75) {BLIS};

    % Group orbits
    \draw[thick, ->, >=stealth] (0,0) -- (4,1) node[midway, above left] {Group Orbit 1};
    \draw[thick, ->, >=stealth] (4,1) -- (8,3) node[midway, above right] {Group Orbit 2};
    \draw[thick, ->, >=ste