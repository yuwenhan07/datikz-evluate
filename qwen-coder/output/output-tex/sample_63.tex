Certainly! Below is a TikZ LaTeX code that generates a diagram illustrating how policies may vary with respect to changes in their reward or transition functions.

```latex
\documentclass{standalone}
\usepackage{tikz}
\usetikzlibrary{shapes.geometric, arrows.meta}

\tikzset{
    block/.style = {rectangle, draw=black, fill=white, text width=5em, text centered, minimum height=4em},
    line/.style = {draw, thick, -Stealth, >=Stealth}
}

\begin{document}

\begin{figure}[h]
    \centering
    \begin{tikzpicture}[
        node distance=2cm,
        x=1cm,
        y=1cm
    ]

    % Left: Reward Variation
    \node (reward_left) [block] {Reward};
    \node (policy_left) [block, below of=reward_left, yshift=-0.5cm] {Policy};
    \node (state_left) [block, below of=policy_left, yshift=-0.5cm] {State};

    \draw [line] (reward_left.south) -- (policy_left.north);
    \draw [line] (policy_left.south) -- (state_left.north);

    % Right: Transition Function Variation
    \node (transition_right) [block, right of=reward_left, xshift=3cm] {Transition};
    \node (policy_right) [block, below of=transition_right, yshift=-0.5cm] {Policy};
    \node (state_right) [block, below of=policy_right, yshift=-0.5cm] {State};

    \draw [line] (transition_right.south) -- (policy_right.north);
    \draw [line] (policy_right.south) -- (state_right.north);

    % Treadmill Box
    \node (treadmill_box) [rectangle, draw=black, fill=gray!20, text width=5em, text centered, minimum height=2em, right of=transition_right, xshift=1cm] {Treadmill};
    \node (agent_left) [circle, draw=black, fill=blue!20, radius=0.5cm, above of=treadmill_box, yshift=-1cm] {Agent};
    \node (agent_right