Certainly! Below is a TikZ LaTeX code that visually represents the influence of the standard deviation of Gaussian utilities on the average convergence time. The x-axis represents the standard deviation, and the y-axis represents the number of games where the dynamics did not converge within the given time limit.

```latex
\documentclass{standalone}
\usepackage{tikz}
\usepackage{pgfplots}
\pgfplotsset{compat=1.16}

\begin{document}

\begin{tikzpicture}
    \begin{axis}[
        title={Influence of Standard Deviation on Convergence Time},
        xlabel={Standard Deviation},
        ylabel={Number of Games Not Converged},
        ymin=0,
        ymax=10,
        xmin=60,
        xmax=100,
        xtick={60, 70, 80, 90, 100},
        ytick={0, 1, 2, 3, 4, 5, 6, 7, 8, 9, 10},
        grid=major,
        width=10cm,
        height=6cm,
        legend style={at={(0.5,-0.1)}, anchor=north,legend columns=-1}
    ]
        
        % Data points for resentful agents
        \addplot[mark=square*, color=blue] coordinates {(60, 7) (70, 23) (80, 48) (90, 63) (100, 75)};
        \addlegendentry{Resentful Agents}
        
        % Data points for appreciative agents
        \addplot[mark=triangle*, color=red] coordinates {(60, 0) (70, 0) (80, 0) (90, 0) (100, 0)};
        \addlegendentry{Appreciative Agents}
        
        % Data points for resentful-appreciative agents
        \addplot[mark=diamond*, color=green] coordinates {(60, 0) (70, 0) (80, 0) (90, 0) (100, 0)};
        \addlegendentry{Resentful-Appreciative Agents}
    \end{axis}
\end{tikzpicture