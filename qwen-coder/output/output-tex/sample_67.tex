Certainly! Below is a TikZ LaTeX code that creates a diagram illustrating the region \( D(\mathbf{s}) \) in the image domain \( \Omega \), showing the gradients \(\nabla I_i(\mathbf{s})\) for \( i = 0, 1, 2 \). The dotted line represents where the gradient is assumed to be constant.

```latex
\documentclass{standalone}
\usepackage{tikz}

\begin{document}

\begin{tikzpicture}[scale=2]
    % Define the coordinates for the region D(s)
    \coordinate (s0) at (0,0);
    \coordinate (s1) at (2,0);
    \coordinate (s2) at (4,0);
    \coordinate (s3) at (6,0);

    % Draw the region D(s)
    \draw[thick] (s0) rectangle (s3) node[midway, below] {$D(\mathbf{s})$};

    % Define the coordinates for the image domain Omega
    \coordinate (o0) at (-1,-2);
    \coordinate (o1) at (7,-2);

    % Draw the image domain Omega
    \draw[thick] (o0) -- (o1) node[midway, above] {$\Omega$};

    % Draw the dotted line representing where the gradient is constant
    \draw[dotted] (2,-1) -- (4,-1) node[midway, below] {Constant Gradient};

    % Draw the gradients as arrows
    \draw[-stealth, thick] (s0) -- ++(0,-1) node[midway, left] {$\nabla I_0(\mathbf{s})$};
    \draw[-stealth, thick] (s1) -- ++(0,-1) node[midway, left] {$\nabla I_1(\mathbf{s})$};
    \draw[-stealth, thick] (s2) -- ++(0,-1) node[midway, left] {$\nabla I_2(\mathbf{s})$};

    % Add labels for clarity
    \node[left] at (0,-2.5) {$\mathbf{s}_0$};
    \node[left] at (2,-2.5) {$\mathbf{s}_1$};
    \node[left