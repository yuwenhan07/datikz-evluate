Certainly! Below is a TikZ LaTeX code that illustrates the quasi-interpolation operator \( Q \). The code includes nodes representing triangles, points, and lines to represent the relationships described.

```latex
\documentclass[tikz,border=10pt]{standalone}
\usetikzlibrary{shapes.geometric, arrows.meta}

\tikzset{
    triangle/.style={regular polygon, regular polygon sides=3, draw},
    point/.style={circle, fill, inner sep=2pt},
    line/.style={->, thick}
}

\begin{document}
\begin{tikzpicture}[scale=2]

% Define points
\node (A) at (0, 0) [point] {};
\node (B) at (2, 0) [point] {};
\node (C) at (1, 1.732) [point] {};

% Define triangles
\node (T1) at (1, 0) [triangle, label=above:$\tau_\xi$] {};
\node (T2) at (1, 1.5) [triangle, label=right:$\tau_\eta$] {};

% Draw lines
\draw[thick] (A) -- (B);
\draw[thick] (B) -- (C);
\draw[thick] (C) -- (A);

% Connect triangles
\draw[line] (T1) -- (T2);

% Add labels
\node at (0.5, -0.5) {$\xi$};
\node at (1.5, -0.5) {$\tau_\xi$};
\node at (1.5, 1.9) {$\eta$};
\node at (1.5, 1.4) {$\tau_\eta$};

% Add text for c_xi(f)
\node at (0.8, -1.5) {\( c_\xi(f) = \gamma_\xi(F_{d,\tau_\xi} f) \)};
\node at (1.2, -1.5) {(\( F_{d,\tau_\xi} f \) is the averaged Taylor polynomial of degree \( d \))};

% Add text for c_eta(f)
\node at (1.2, 2.0) {\( c_\eta(f) = \sum_{\xi \in \mathcal