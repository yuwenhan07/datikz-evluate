To illustrate this concept using TikZ LaTeX, we can create two graphs \( G'_1 \) and \( G'_2 \) with the specified properties, and then show that their complements \( G_1 \) and \( G_2 \) become isomorphic after removing the respective edges.

Here's the TikZ LaTeX code to generate the diagrams:

```latex
\documentclass{standalone}
\usepackage{tikz}

\begin{document}

% Define styles for nodes and edges
\tikzstyle{node} = [circle, draw, fill=blue!20, inner sep=5pt]
\tikzstyle{edge} = [-stealth, thick]

% Create graph G'_1
\begin{figure}[h]
    \centering
    \begin{tikzpicture}[scale=1.5]
        % Nodes
        \node[node] (u1) at (0,0) {\( u_1 \)};
        \node[node] (v1) at (2,0) {\( v_1 \)};
        \node[node] (u2) at (0,-2) {\( u_2 \)};
        \node[node] (v2) at (2,-2) {\( v_2 \)};
        
        % Edges
        \draw[edge] (u1) -- (v1);
        \draw[edge] (u2) -- (v2);
        \draw[edge] (u1) -- (u2);
        \draw[edge] (v1) -- (v2);
        
        % Highlight added edge
        \draw[dashed, red] (u1) -- (v2);
        
        % Labels
        \node[above left] at (u1.west) {\( G'_1 \)};
    \end{tikzpicture}
\end{figure}

% Create graph G'_2
\begin{figure}[h]
    \centering
    \begin{tikzpicture}[scale=1.5]
        % Nodes
        \node[node] (w1) at (0,0) {\( w_1 \)};
        \node[node] (x1) at (2,0) {\( x_1 \)};
        \node[node] (w2) at (0,-2) {\( w_2 \)};
        \node[node] (x2) at (2