To align and rotate text in a TikZ state diagram for better readability and visual appeal, you can use the `node` command with options like `align`, `rotate`, and `anchor`. Below is an example of how you might modify your existing code to achieve this:

```latex
\documentclass{article}
\usepackage{tikz}
\usetikzlibrary{shapes.geometric, arrows}

\tikzstyle{startstop} = [rectangle, rounded corners, minimum width=3cm, minimum height=1cm,text centered, draw=black, fill=red!30]
\tikzstyle{process} = [rectangle, minimum width=3cm, minimum height=1cm, text centered, draw=black, fill=orange!30]
\tikzstyle{decision} = [diamond, minimum width=3cm, minimum height=1cm, text centered, draw=black, fill=green!30]
\tikzstyle{arrow} = [thick,->,>=stealth]

\begin{document}

\begin{figure}[h]
    \centering
    \begin{tikzpicture}[node distance=2cm]
        % Nodes
        \node (init) [startstop] {Init};
        \node (lc_state_machine) [process, below of=init, yshift=-1cm] {LC State Machine};
        \node (lc_counter) [process, below of=lc_state_machine, yshift=-1cm] {LC Counter};
        \node (bit_counter) [process, below of=lc_counter, yshift=-1cm] {Bit Counter};
        \node (end) [startstop, below of=bit_counter, yshift=-1cm] {End};

        % Arrows
        \draw [arrow] (init) -- (lc_state_machine);
        \draw [arrow] (lc_state_machine) -- (lc_counter);
        \draw [arrow] (lc_counter) -- (bit_counter);
        \draw [arrow] (bit_counter) -- (end);

        % Adding rotated and aligned labels
        \node at ([yshift=-1.5cm,xshift=-1cm]lc_state_machine.south) [below, align=center, rotate=90] {LC State\\Machine};
        \node at ([yshift=-1.5cm,xshift=1cm]lc_counter.south) [below, align=center