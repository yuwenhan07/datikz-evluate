Certainly! Below is a TikZ LaTeX code that illustrates the Average Confidence Score (ACS) for unlabeled data across tasks. This code will create a plot showing the decaying trend of the ACS, highlighting the issue with using a fixed high threshold.

```latex
\documentclass{standalone}
\usepackage{tikz}
\usetikzlibrary{plotmarks}

\begin{document}
\begin{tikzpicture}[scale=1]
    \draw[thick,->] (0,0) -- (6,0) node[right] {Number of Tasks};
    \draw[thick,->] (0,0) -- (0,4) node[above] {Average Confidence Score (ACS)};
    
    % Data points
    \coordinate (task1) at (1,3);
    \coordinate (task2) at (2,2.5);
    \coordinate (task3) at (3,2);
    \coordinate (task4) at (4,1.5);
    \coordinate (task5) at (5,1);
    
    % Plotting the data points
    \filldraw[blue, mark=*] (task1) circle (2pt) node[anchor=north east] {Task 1};
    \filldraw[blue, mark=*] (task2) circle (2pt) node[anchor=north east] {Task 2};
    \filldraw[blue, mark=*] (task3) circle (2pt) node[anchor=north east] {Task 3};
    \filldraw[blue, mark=*] (task4) circle (2pt) node[anchor=north east] {Task 4};
    \filldraw[blue, mark=*] (task5) circle (2pt) node[anchor=north east] {Task 5};
    
    % Connecting the data points with lines
    \draw[blue, thick] (task1) -- (task2) -- (task3) -- (task4) -- (task5);
    
    % Adding labels and annotations
    \node[below left] at (0,0) {(0,0)};
    \node[below right] at (6,0) {(5,0)};
    \node[left] at (0,4) {(0,4)};
    \node[right] at (6,-0.5) {Decaying Trend};
