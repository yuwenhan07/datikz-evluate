Certainly! Below is a TikZ LaTeX code that represents the Snyk Code Codesearch system as described:

```latex
\documentclass[tikz,border=10pt]{standalone}
\usetikzlibrary{shapes.geometric, arrows.meta}

\tikzset{
    block/.style = {rectangle, draw, fill=blue!20, 
        text width=5em, text centered, rounded corners, minimum height=4em},
    ellipsis/.style = {ellipse, draw, fill=red!20, 
        text width=3em, text centered, minimum height=3em},
    line/.style = {draw, thick, ->, >=stealth}
}

\begin{document}
\begin{tikzpicture}[node distance=2cm]
    % Inputs
    \node [block] (input1) {Input 1};
    \node [block, below=of input1] (input2) {Input 2};

    % System Components
    \node [block, left=of input1, xshift=-3cm] (codesearch_stl) {Codesearch STL};
    \node [block, right=of input2, xshift=3cm] (starlang_api) {StarLang API};
    \node [block, below=of codesearch_stl, yshift=-2cm] (computation1) {Computation 1};
    \node [block, below=of computation1] (computation2) {Computation 2};
    \node [ellipsis, below=of computation2, yshift=-2cm] (output_data) {Output Data};

    % Outputs
    \node [block, above=of output_data, yshift=2cm] (output1) {Output 1};
    \node [block, above=of output1, yshift=2cm] (output2) {Output 2};

    % Arrows
    \draw [line] (input1) -- (codesearch_stl);
    \draw [line] (input2) -- (codesearch_stl);
    \draw [line] (codesearch_stl) -- (computation1);
    \draw [line] (computation1) -- (computation2);
    \draw [line] (computation2) -- (output_data);
    \draw [line] (output_data) |- (output1);
