To create a TikZ LaTeX diagram that represents the graphs \( P_n \) and \( C_n \) with specific vertices labeled as "black" to indicate an FTD-code, you can use the following code. This example assumes \( n = 5 \) for simplicity.

```latex
\documentclass{standalone}
\usepackage{tikz}

\begin{document}
\begin{tikzpicture}[scale=1.5]
    % Define the number of vertices
    \def\n{5}
    
    % Draw the path graph P_n
    \foreach \i in {1,...,\n} {
        \node[draw, circle, fill=white] (P-\i) at (\i,0) {};
        \ifnum\i<\n
            \draw[-stealth] (P-\i) -- (P-\the\numexpr\i+1\relax);
        \fi
    }
    \foreach \i in {1,...,\n} {
        \ifnum\i=2
            \node[draw, circle, fill=black] at (\i,-0.8) {$v_1$};
        \else ifnum\i=\n
            \node[draw, circle, fill=black] at (\i,-0.8) {$v_n$};
        \else
            \node[draw, circle, fill=white] at (\i,-0.8) {};
        \fi
    }
    
    % Draw the cycle graph C_n
    \foreach \i in {1,...,\n} {
        \node[draw, circle, fill=white] (C-\i) at ({cos(72*\i)}, {sin(72*\i)}) {};
        \ifnum\i<\n
            \draw[-stealth] (C-\i) -- (C-\the\numexpr\i+1\relax);
        \fi
    }
    \draw[-stealth] (C-\n) -- (C-1);
    \foreach \i in {1,...,\n} {
        \ifnum\i=3
            \node[draw, circle, fill=black] at ({cos(72*\i)}, {sin(72*\i)} - 0.8) {$v_1$};
        \else ifnum\i=4
            \node[draw