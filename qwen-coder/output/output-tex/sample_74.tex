To create a TikZ diagram that represents the geometric transformation you described, we need to illustrate the following steps:

1. **Original Geometry**: A torus (representing \(\frac{\mathbb{R} \times [-1,1]}{\mathbb{Z}_2}\)).
2. **Orbifold Singularity**: A point where the torus has a singularity.
3. **Smoothing**: The singularity being smoothed out to form a smooth surface.
4. **Mapping**: The singular point being mapped to an interior point in the smoothed surface.

Here's a TikZ code that attempts to visualize this process:

```latex
\documentclass[tikz,border=5mm]{standalone}
\usepackage{tikz-3dplot}

\begin{document}
\tdplotsetmaincoords{70}{110}

\begin{tikzpicture}[scale=1]
    % Original Torus
    \draw[thick] (0,0) circle (1);
    \foreach \x in {0,...,360} {
        \pgfmathsetmacro{\y}{sin(\x)*cos(4*\x)}
        \pgfmathsetmacro{\z}{sin(\x)*sin(4*\x)}
        \ifnum\x=0
            \coordinate (start) at (\y,\z);
        \else
            \draw[thick] plot[smooth] coordinates {(start) (\y,\z)};
            \coordinate (start) at (\y,\z);
        \fi
    }

    % Orbifold Singularity
    \fill[red] (0,0,0) circle (0.05);

    % Smoothed Surface
    \begin{scope}[shift={(2,0,0)}]
        \draw[thick] (0,0) circle (1);
        \foreach \x in {0,...,360} {
            \pgfmathsetmacro{\y}{sin(\x)*cos(4*\x)}
            \pgfmathsetmacro{\z}{sin(\x)*sin(4*\x)}
            \ifnum\x=0
                \coordinate (start) at (\y,\z);
            \else
                \draw[thick] plot[smooth] coordinates {(start) (\y,\z)};
                \coordinate (start) at (\y,\z);
            \fi
        }
    \