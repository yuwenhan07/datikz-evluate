Certainly! Below is a TikZ LaTeX code to illustrate several symmetric padding modes for a 1D discrete signal:

```latex
\documentclass{standalone}
\usepackage{tikz}

\begin{document}
\begin{tikzpicture}[scale=0.8]
    % Original Signal
    \draw[thick, ->] (0,0) -- (5,0);
    \foreach \i in {1,...,4} {
        \fill (\i,0.2) circle (2pt) node[above] {$x_{\i}$};
    }
    
    % Symmetric Padding Modes
    \def\n{4}
    \def\pad{3}
    \def\shift{\n+\pad}
    \def\width{8}
    
    % Left Symmetric Padding
    \draw[thick, ->] (\shift,0) -- (\width,0);
    \foreach \i in {1,...,\pad} {
        \fill (\shift+\i,0.2) circle (2pt) node[above] {$x_{\i}$};
    }
    \foreach \i in {1,...,\n} {
        \fill (\shift+\n+\i,0.2) circle (2pt) node[above] {$x_{\n-\i+1}$};
    }
    
    % Right Symmetric Padding
    \draw[thick, ->] (\shift+\width,0) -- (\shift+2*\width,0);
    \foreach \i in {1,...,\pad} {
        \fill (\shift+\width+\i,0.2) circle (2pt) node[above] {$x_{\n-\i+1}$};
    }
    \foreach \i in {1,...,\n} {
        \fill (\shift+\width+\n+\i,0.2) circle (2pt) node[above] {$x_{\i}$};
    }
    
    % Full Symmetric Padding
    \draw[thick, ->] (\shift+2*\width,0) -- (\shift+3*\width,0);
    \foreach \i in {1,...,\pad} {
        \fill (\shift+2*\width+\i,0.2) circle (2pt) node[above] {$x_{\n-\i+1}$};
    }
    \foreach \i in {1