Certainly! Below is a TikZ LaTeX code to generate a plot of the described smooth convex function \( f \):

```latex
\documentclass[tikz]{standalone}
\usepackage{pgfplots}
\pgfplotsset{compat=1.18}

\begin{document}
\begin{tikzpicture}
    \begin{axis}[
        axis lines=middle,
        xlabel=$r$,
        ylabel=$f(r)$,
        ymin=-0.5,
        ymax=1.5,
        xmin=-1,
        xmax=2,
        domain=-0.7:1.5,
        samples=400,
        grid=major,
        tick style={thick},
        every axis label/.style={font=\small},
        legend style={at={(0.5,-0.1)}, anchor=north,legend columns=-1}
    ]
        % Define the function segments
        \addplot[blue, thick] {ifthenelse(x > -0.35, ifthenelse(x < 0.35, 0.5 - 0.5*x/0.35, 1 - x), 0)};
        \addlegendentry{$f(r) = \frac{1}{2} - \frac{e}{2}r$ for $-0.35 < r < 0.35$}
        
        \addplot[red, thick] {ifthenelse(x > 0.65, a-x, 0)};
        \addlegendentry{$f(r) = a - r$ for $0.65 < r < a$}
        
        % Plot points at critical values
        \filldraw (0,0.5) circle (2pt);
        \filldraw (1,0) circle (2pt);
        
        % Draw vertical lines at critical points
        \draw[dashed] (0,-0.5) -- (0,1.5);
        \draw[dashed] (1,-0.5) -- (1,1.5);
        
        % Annotate critical points
        \node at (-0.4,0.5) [below] {$-\frac{\eta}{2}$};
        \node at (0.4,0.5) [below] {$\frac{\eta}{2}$};
        \node at (1.1,0) [below] {$