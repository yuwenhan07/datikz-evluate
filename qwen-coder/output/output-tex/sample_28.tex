Certainly! Below is a TikZ LaTeX code that illustrates the reflection-extension process with a splitting point \( u \). The last part of the original walk is drawn in gray, and the modified walk is drawn in black.

```latex
\documentclass{standalone}
\usepackage{tikz}

\begin{document}

\begin{tikzpicture}[scale=2]
    % Define points
    \coordinate (A) at (0,0);
    \coordinate (B) at (1,1);
    \coordinate (C) at (2,0);
    \coordinate (D) at (3,1);
    \coordinate (E) at (4,0);
    \coordinate (F) at (5,1);
    \coordinate (G) at (6,0);
    \coordinate (H) at (7,1);
    \coordinate (I) at (8,0);
    \coordinate (J) at (9,1);

    % Draw original walk
    \draw[gray] (A) -- (B) -- (C) -- (D) -- (E) -- (F) -- (G) -- (H) -- (I) -- (J);

    % Splitting point u
    \node[circle, fill=black, inner sep=2pt] (u) at (4,0) {};

    % Reflection extension
    \coordinate (u_prime) at (6,-1);
    \coordinate (g_u_prime) at (6,1);
    \draw (u) to[bend right=45] node[midway, above] {$g_u(u)$} (u_prime);
    \draw (u_prime) -- (g_u_prime);
    \draw (g_u_prime) to[bend left=45] node[midway, below] {$g_u(u')$} (u);

    % Modified walk
    \draw[thick, black] (A) -- (B) -- (C) -- (D) -- (E) -- (F) -- (G) -- (H) -- (I) -- (u_prime) -- (g_u_prime) -- (u) -- (A);

    % Labels
    \node[above] at (A) {A};
    \node[above] at (B) {B};
    \node[above] at (C) {C};
    \node[above]