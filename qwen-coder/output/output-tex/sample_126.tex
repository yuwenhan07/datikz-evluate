To create the perfect linear realizations \(\mathbf{g_1}\) and \(\mathbf{g_2}\) for the given sets, we need to understand that these sets represent functions in terms of \(x\). The perfect linear realization of a set of polynomials is a way to map them into a vector space such that they form a basis.

For the first set \(\{(x-1)^{x-1}, x^x\}\), let's consider the following:
1. For \(x = 0\):
   - \((x-1)^{x-1} = (-1)^{-1}\) which is undefined.
   - \(x^x = 0^0\) which is also undefined.
   
Since both functions are undefined at \(x = 0\), let's consider another value, say \(x = 1\):
1. For \(x = 1\):
   - \((x-1)^{x-1} = 0^0\) which is undefined.
   - \(x^x = 1^1 = 1\).

Let's consider \(x = 2\):
1. For \(x = 2\):
   - \((x-1)^{x-1} = 1^1 = 1\).
   - \(x^x = 2^2 = 4\).

So, we can represent the function values at \(x = 2\) as:
\[ \mathbf{g_1} = \begin{pmatrix} 1 \\ 4 \end{pmatrix} \]

For the second set \(\{x^{x+2}, (x+1)^{x-1}\}\), let's consider the same points:
1. For \(x = 2\):
   - \(x^{x+2} = 2^{2+2} = 2^4 = 16\).
   - \((x+1)^{x-1} = 3^{2-1} = 3^1 = 3\).

So, we can represent the function values at \(x = 2\) as:
\[ \mathbf{g_2} = \begin{pmatrix} 16 \\ 3 \end{pmatrix} \]

Now, let's use TikZ to visualize these vectors:

```latex
\documentclass{standalone}
\usepackage{