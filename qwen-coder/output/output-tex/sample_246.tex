To create a TikZ diagram that represents the network described, we need to follow these steps:

1. Define the nodes and their positions.
2. Draw the edges between the nodes.
3. Label the nodes and edges with the appropriate values.

Here is the TikZ code to generate the diagram:

```latex
\documentclass{standalone}
\usepackage{tikz}

\begin{document}

\begin{tikzpicture}[node distance=2cm, every node/.style={circle, draw, fill=blue!20, text width=1em, align=center}]

    % Nodes
    \node (P1) {0};
    \node (P2) [right of=P1] {1};
    \node (P3) [below right of=P2] {2};
    \node (P4) [below right of=P1] {3};
    \node (P5) [below left of=P3] {4};
    \node (P6) [below left of=P2] {5};

    % Edges
    \draw (P1) -- (P2);
    \draw (P2) -- (P3);
    \draw (P3) -- (P5);
    \draw (P1) -- (P4);
    \draw (P4) -- (P5);
    \draw (P2) -- (P6);

    % Labels
    \node [above left of=P1] {0};
    \node [above right of=P2] {1};
    \node [below right of=P3] {2};
    \node [below right of=P1] {3};
    \node [below left of=P3] {4};
    \node [below left of=P2] {5};

    % Throughput values
    \node [below of=P1, yshift=-1cm] {$\frac{29}{55}$};
    \node [below of=P2, yshift=-1cm] {$\frac{29}{55}$};
    \node [below of=P3, yshift=-1cm] {$\frac{29}{55}$};
    \node [below of=P4, yshift=-1cm] {$\frac{29}{55}$};
    \node [below of=P5, yshift=-1cm] {$\frac{29}{