Certainly! Below is a TikZ LaTeX code that visualizes the scenario you described:

```latex
\documentclass{standalone}
\usepackage{tikz}

\begin{document}

\begin{tikzpicture}[scale=2]
    % Axes
    \draw[->] (-3,0) -- (4,0) node[right] {$\theta_0$};
    \draw[->] (0,-1) -- (0,3) node[above] {$\mu$};

    % True value line
    \draw[dashed, thick, red] (0,1) -- (3,1) node[midway, above] {\(\mu^*\)};
    
    % Optimistic estimate line
    \draw[thick, blue] (0,0) -- (2,1) node[midway, above right] {\(\bar{\mu}\)};
    
    % True value range
    \draw[dashed, thick, green] (0,1) -- (1,1);
    \node at (0.5, 1.1) {True \(\theta_0^*\)};
    
    % Confidence interval bounds
    \draw[thick, dashed, black] (0,0) -- (2,0) node[midway, below left] {\(\hat{\mu}\)};
    \draw[thick, dashed, black] (2,1) -- (4,1) node[midway, above right] {\(\bar{\mu}\)};
    
    % Worst-case scenarios
    \draw[thick, orange] (0,0) -- (2,0) node[midway, below left] {\(\hat{\mu}_A'\)};
    \draw[thick, orange] (2,1) -- (4,1) node[midway, above right] {\(\bar{\mu}_B'\)};
    
    % Delta label
    \draw[thick, ->, orange] (2,0) -- (4,1) node[midway, sloped, above right] {\(\Delta_\mu = \bar{\mu}_B' - \hat{\mu}_A'\)};
    
    % Labels
    \node at (-1, 1.5) {Optimistic Estimate};
    \node at (-1, 0.5) {Confidence Interval Bounds};
    \node at (-1, -0.5