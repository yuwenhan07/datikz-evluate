Certainly! Below is a TikZ LaTeX code that illustrates the proof of Proposition \ref{pro: LWconv}. The code includes two parts: one showing balls in \( G_{k,n} \) for \( r < \lfloor n/2 \rfloor \), and another showing balls in \( G_{k,n} \) for \( r \geq 1 \).

```latex
\documentclass{standalone}
\usepackage{tikz}
\usetikzlibrary{shapes.geometric, arrows.meta}

\tikzset{
    vertex/.style={circle, draw=black, fill=white, inner sep=2pt},
    edge/.style={->, thick}
}

\begin{document}

\begin{tikzpicture}[scale=0.8]

% Part 1: Balls in G_{k,n} for r < floor(n/2)
\draw[dotted] (-3,-3) rectangle (3,3);
\node[vertex] at (0,0) {0};
\foreach \i in {-2,-1,1,2}{
    \node[vertex] at (\i,0) {};
    \node[vertex] at (0,\i) {};
    \node[vertex] at (0,-\i) {};
    \node[vertex] at (\i,\i) {};
    \node[vertex] at (\i,-\i) {};
    \node[vertex] at (-\i,\i) {};
    \node[vertex] at (-\i,-\i) {};
}

% Part 2: Balls in G_{k,n} for r >= 1
\draw[dashed] (5,-3) rectangle (9,3);
\node[vertex] at (6,0) {0};
\foreach \i in {1,...,4}{
    \node[vertex] at (6+\i,0) {};
    \node[vertex] at (6-\i,0) {};
    \node[vertex] at (6,0+\i) {};
    \node[vertex] at (6,0-\i) {};
    \node[vertex] at (6+\i,\i) {};
    \node[vertex] at (6+\i,-\i) {};
    \node[vertex] at (6-\i,\i) {};
    \node[vertex] at (6-\i,-\i) {};
    \node[vertex] at (6+\i,0+\i) {};
    \node[vertex] at