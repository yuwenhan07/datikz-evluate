To create a TikZ LaTeX diagram that visualizes the concept you described, we can represent the convex hull \( C \) as a polyhedron, the support vectors \( (p_n)_{n \in \mathbb{N}} \) approaching a limit \( p \), and the point \( x \) with the largest \( w' \)-component on the corresponding face of \( C \). Here's a step-by-step guide to generating this diagram:

1. **Define the Convex Hull \( C \)**: We will use a simple 3D polyhedron for demonstration.
2. **Represent Support Vectors**: Show a few support vectors approaching the limit \( p \).
3. **Highlight Point \( x \)**: Mark the point \( x \) on the corresponding face of \( C \).

Below is the TikZ code to achieve this:

```latex
\documentclass[tikz,border=3mm]{standalone}
\usepackage{tikz-3dplot}

\begin{document}
\tikzset{
    axis/.style={->,thick},
    vector/.style={->,thick,color=red},
    point/.style={circle,fill=blue,inner sep=2pt}
}

\tdplotsetmaincoords{70}{45} % Set the main coordinate system

\begin{tikzpicture}[tdplot_main_coords]
    % Define the vertices of the convex hull
    \coordinate (A) at (-1,-1,-1);
    \coordinate (B) at (1,-1,-1);
    \coordinate (C) at (1,1,-1);
    \coordinate (D) at (-1,1,-1);
    \coordinate (E) at (-1,-1,1);
    \coordinate (F) at (1,-1,1);
    \coordinate (G) at (1,1,1);
    \coordinate (H) at (-1,1,1);

    % Draw the edges of the convex hull
    \draw[thick] (A) -- (B) -- (C) -- (D) -- cycle;
    \draw[thick] (A) -- (B) -- (F) -- (E) -- cycle;
    \draw[thick] (B) -- (C) -- (G) -- (F) -- cycle;
    \draw[thick] (C) -- (D