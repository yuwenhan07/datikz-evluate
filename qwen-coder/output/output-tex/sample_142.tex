Certainly! Below is a TikZ LaTeX code that generates a diagram resembling the description you provided. This diagram includes a chain of interconnected circles, each labeled with a mathematical expression related to a variable \(x_a\) in the context of a 2-attractor reduction in a 2-substring graph.

```latex
\documentclass[tikz,border=3mm]{standalone}
\usetikzlibrary{shapes.geometric, arrows.meta}

\tikzset{
    node style/.style={circle, draw, fill=black!30, minimum size=15pt},
    edge style/.style={thick, ->, >=Stealth[round]},
    label style/.style={font=\small}
}

\begin{document}
\begin{tikzpicture}[node distance=2cm]
    % Nodes
    \node[node style] (n1) at (0,0) {$x_{a_1}$};
    \node[node style] (n2) at (2,0) {$x_{a_2}$};
    \node[node style] (n3) at (4,0) {$x_{a_3}$};
    \node[node style] (n4) at (6,0) {$x_{a_4}$};

    % Edges
    \draw[edge style] (n1) -- node[label style, above] {Edge 1} (n2);
    \draw[edge style] (n2) -- node[label style, above] {Edge 2} (n3);
    \draw[edge style] (n3) -- node[label style, above] {Edge 3} (n4);

    % Additional nodes and edges for more complexity
    \node[node style] (n5) at (8,0) {$x_{a_5}$};
    \draw[edge style] (n4) -- node[label style, above] {Edge 4} (n5);

    \node[node style] (n6) at (0,-2) {$x_{a_6}$};
    \node[node style] (n7) at (2,-2) {$x_{a_7}$};
    \node[node style] (n8) at (4,-2) {$x_{a_8}$};
    \node[node style] (n9) at (6,-2) {$x_{a_9}$};
    \node[node style]