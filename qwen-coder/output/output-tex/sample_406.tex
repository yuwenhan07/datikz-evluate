To generate the Hasse diagram for the poset \((2^{\{1,2,3,4\}}, \subseteq)\), we can use TikZ in LaTeX. Below is the complete TikZ code to create this Hasse diagram:

```latex
\documentclass{standalone}
\usepackage{tikz}

\begin{document}

\begin{tikzpicture}[level distance=1.5cm,
  level 1/.style={sibling distance=3cm},
  level 2/.style={sibling distance=2cm},
  level 3/.style={sibling distance=1.5cm},
  node distance=0.5cm,
  every node/.style={circle, draw, inner sep=1pt}]
  
  % Nodes at level 0 (maximal elements)
  \node (violet1) [violet] {};
  \node (violet2) [violet] {};
  \node (violet3) [violet] {};
  
  % Nodes at level 1
  \node (blue1) [blue, above left of=violet1] {};
  \node (blue2) [blue, below left of=violet1] {};
  \node (blue3) [blue, above right of=violet2] {};
  \node (blue4) [blue, below right of=violet2] {};
  
  % Nodes at level 2
  \node (green1) [green, above left of=blue1] {};
  \node (green2) [green, below left of=blue1] {};
  \node (green3) [green, above right of=blue2] {};
  \node (green4) [green, below right of=blue2];
  
  % Nodes at level 3
  \node (orange1) [orange, above left of=green1] {};
  \node (orange2) [orange, below left of=green1] {};
  \node (orange3) [orange, above right of=green2] {};
  \node (orange4) [orange, below right of=green2];
  
  % Edges
  \draw[->] (orange1) -- (green1);
  \draw[->] (orange2) -- (green1);
  \draw[->] (orange3) -- (green2);
  \draw[->] (orange4