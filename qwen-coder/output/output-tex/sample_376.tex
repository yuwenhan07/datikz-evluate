Certainly! To create a TikZ diagram that illustrates the construction described in Theorem~\ref{thm:const}, we need to represent both the tournament graph and the resulting vertex coloring. Below is an example of how you might construct such a diagram:

```latex
\documentclass{article}
\usepackage{tikz}

\begin{document}

% Define styles for nodes and edges
\tikzset{
  node style/.style={circle, draw=black, fill=white, inner sep=5pt},
  edge style/.style={->, thick, ->>
}

% Create the tournament graph
\begin{figure}[h]
\centering
\begin{tikzpicture}
  % Nodes for the tournament
  \node[node style] (A) at (0,2) {A};
  \node[node style] (B) at (-1,-1) {B};
  \node[node style] (C) at (1,-1) {C};
  
  % Edges representing the tournament
  \draw[edge style] (A) -- (B);
  \draw[edge style] (A) -- (C);
  \draw[edge style] (B) -- (C);
\end{tikzpicture}
\caption{Tournament Graph}
\label{fig:tournament}
\end{figure}

% Create the colored graph
\begin{figure}[h]
\centering
\begin{tikzpicture}
  % Nodes for the colored graph
  \node[node style, fill=red] (A) at (0,2) {A};
  \node[node style, fill=blue] (B) at (-1,-1) {B};
  \node[node style, fill=green] (C) at (1,-1) {C};
  
  % Edges representing the colored graph
  \draw[edge style] (A) -- (B);
  \draw[edge style] (A) -- (C);
  \draw[edge style] (B) -- (C);
\end{tikzpicture}
\caption{Colored Graph}
\label{fig:colored}
\end{figure}

% Explanation
\begin{figure}[h]
\centering
\begin{tikzpicture}
  % Nodes for the explanation
  \node[node style, fill=red] (A) at (0,2) {A};
  \node