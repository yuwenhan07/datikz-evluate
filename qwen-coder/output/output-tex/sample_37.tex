Certainly! Below is a sample TikZ LaTeX code that generates a graph similar to the one you described. This code includes a vertical axis representing accuracy values and horizontal bars for different voice recognition models or settings.

```latex
\documentclass{standalone}
\usepackage{tikz}
\usetikzlibrary{calc}

\begin{document}

\begin{tikzpicture}[scale=1]
    % Define colors
    \colorlet{blueModel}{blue!70!black}
    \colorlet{whiteModel}{white!70!black}

    % Define dimensions
    \def\graphWidth{8cm}
    \def\graphHeight{6cm}
    \def\barWidth{0.5cm}
    \def\axisXStart{-1}
    \def\axisYStart{0}
    \def\axisXEnd{4}
    \def\axisYEnd{100}

    % Draw background grid
    \draw[step=20, thin, gray] (\axisXStart,\axisYStart) grid (\axisXEnd,\axisYEnd);

    % Draw axes
    \draw[-stealth] (\axisXStart,\axisYStart) -- (\axisXEnd,\axisYStart) node[right] {Voice Recognition Models};
    \draw[-stealth] (\axisXStart,\axisYStart) -- (\axisXStart,\axisYEnd) node[above] {Accuracy};

    % Draw data bars
    \node at (0,30) [rectangle, fill=blueModel, minimum width=\barWidth, minimum height=30pt] {};
    \node at (1,60) [rectangle, fill=whiteModel, minimum width=\barWidth, minimum height=60pt] {};

    % Add labels for data bars
    \node at (-1.5, 30) [anchor=east] {Model A};
    \node at (-1.5, 60) [anchor=east] {Model B};

    % Draw tick marks and labels on the Y-axis
    \foreach \y in {0,20,...,100} {
        \draw (\axisXStart, \y) -- ++(-0.1, 0);
        \node at (\axisXStart - 0.3, \y) {\y\%};
    }

    % Draw tick marks and labels