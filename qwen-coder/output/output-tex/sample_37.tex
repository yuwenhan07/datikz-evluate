Certainly! To create a TikZ diagram representing the curves \( K_t = K^+_t \cup L_t \cup K^-_t \), we need to define three parts: \( K^+_t \), \( L_t \), and \( K^-_t \). These can be visualized as three interconnected curves.

Here is a sample TikZ code that you can use:

```latex
\documentclass{standalone}
\usepackage{tikz}

\begin{document}

\begin{tikzpicture}[scale=2]

% Define points for K^+_t
\coordinate (A) at (-1, 0);
\coordinate (B) at (0, 1);
\coordinate (C) at (1, 0);

% Draw K^+_t
\draw[thick, red] (A) .. controls (0, -1) and (2, 1) .. (B);
\draw[thick, red] (B) .. controls (2, -1) and (0, 3) .. (C);
\draw[thick, red] (C) .. controls (-2, -1) and (0, -3) .. (A);

% Define points for L_t
\coordinate (D) at (-1, -1);
\coordinate (E) at (0, -2);
\coordinate (F) at (1, -1);

% Draw L_t
\draw[thick, blue] (D) .. controls (0, 1) and (2, -1) .. (E);
\draw[thick, blue] (E) .. controls (2, 1) and (0, -3) .. (F);
\draw[thick, blue] (F) .. controls (-2, 1) and (0, 3) .. (D);

% Define points for K^-_t
\coordinate (G) at (-1, 1);
\coordinate (H) at (0, 2);
\coordinate (I) at (1, 1);

% Draw K^-_t
\draw[thick, green] (G) .. controls (0, -1) and (-2, 1) .. (H);
\draw[thick, green] (H) .. controls (-2, -1) and (0, -3) .. (I);
\draw[thick