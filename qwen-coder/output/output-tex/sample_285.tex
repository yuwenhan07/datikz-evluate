To create the described TikZ diagram, you can use the following LaTeX code:

```latex
\documentclass[tikz,border=3mm]{standalone}
\usetikzlibrary{arrows.meta,backgrounds}

\tikzset{
    myinterval/.style={draw,fill=blue!20},
    myexpired/.style={draw,dashed,fill=red!20},
    myspecial/.style={draw,dotted,fill=green!20},
    mylabel/.style={below=-2pt}
}

\begin{document}
\begin{tikzpicture}[scale=0.8]
    % Draw the timeline
    \draw[thick,-Stealth] (0,0) -- (10,0);
    \foreach \x in {1,...,10} {
        \node at (\x,0) [mylabel] {\x};
    }

    % Define intervals
    \coordinate (I1start) at (2,0);
    \coordinate (I1end) at (4,0);
    \coordinate (I2start) at (6,0);
    \coordinate (I2end) at (8,0);
    \coordinate (IJstart) at (7,0);
    \coordinate (IJend) at (9,0);
    \coordinate (ILminus1start) at (5,0);
    \coordinate (ILminus1end) at (7,0);

    % Draw intervals
    \draw[myinterval] (I1start) rectangle (I1end);
    \draw[myinterval] (I2start) rectangle (I2end);
    \draw[myspecial] (IJstart) rectangle (IJend);
    \draw[myexpired] (ILminus1start) rectangle (ILminus1end);

    % Label intervals
    \node at ($(I1start)!0.5!(I1end)$) [mylabel] {$I_1$};
    \node at ($(I2start)!0.5!(I2end)$) [mylabel] {$I_2$};
    \node at ($(IJstart)!0.5!(IJend)$) [mylabel] {$I_J$};
    \node at ($(ILminus1start)!0.5!(ILminus1end)$) [mylabel] {$I_{L-1}$};

    % Highlight the special interval
   