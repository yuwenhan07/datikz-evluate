To create a TikZ diagram that illustrates the sequence of \( |\lambda_f| \) for all \( f \in \mathcal{I}_2 \) and \( m = 7 \), we need to understand the structure of \( \mathcal{I}_2 \). For \( m = 7 \), \( \mathcal{I}_2 \) consists of all subsets of size 2 from a set of 7 elements. There are \( \binom{7}{2} = 21 \) such subsets.

The sequence \( S_j \) is given as \( 1, 1, 2, 2, 3, 3, 3, 2, 2, 1, 1 \). This sequence represents the values of \( |\lambda_f| \) for each subset \( f \).

Here's the TikZ code to generate the diagram:

```latex
\documentclass{standalone}
\usepackage{tikz}

\begin{document}

\begin{tikzpicture}[scale=0.8]
    % Define the positions for the nodes
    \foreach \i in {1,...,21}{
        \node (n\i) at (\i*0.5,0) {};
    }

    % Draw the horizontal lines
    \draw[thick] (1,-0.1) -- (21,-0.1);
    \draw[dashed] (1,-0.2) -- (21,-0.2);

    % Label the nodes with the values of S_j
    \foreach \j/\v in {1/1, 2/1, 3/2, 4/2, 5/3, 6/3, 7/3, 8/2, 9/2, 10/1, 11/1}{
        \node at (\j*0.5, -0.3) {\v};
    }

    % Highlight the maximum value
    \node[fill=yellow!30, inner sep=1pt] at (5.5, 0) {};

    % Add labels for clarity
    \node at (0.5, 0.5) [above] {Index};
    \node at (10.5, 0.5) [above] {Value};

    % Draw arrows to indicate the sequence
    \foreach