Certainly! Below is the TikZ LaTeX code to generate the graphs \( K_{1,n}^* \), \( K_n^* \), \( C\!K_n \), \( B\!S_n^p \), and \( F_n \).

```latex
\documentclass{article}
\usepackage{tikz}

\begin{document}

% Define styles for nodes and edges
\tikzset{
    node/.style={circle, draw=black, fill=white, inner sep=2pt},
    edge/.style={->, thick}
}

% Graph K_{1,n}^*
\begin{figure}[h]
    \centering
    \begin{tikzpicture}[scale=0.8]
        % Nodes
        \node[node] (v) at (0,0) {};
        \foreach \i in {1,...,4} {
            \node[node] (u\i) at (\i,0) {};
        }
        % Edges
        \draw[edge] (v) -- (u1);
        \draw[edge] (v) -- (u2);
        \draw[edge] (v) -- (u3);
        \draw[edge] (v) -- (u4);
    \end{tikzpicture}
    \caption{Graph \( K_{1,n}^* \)}
\end{figure}

% Graph K_n^*
\begin{figure}[h]
    \centering
    \begin{tikzpicture}[scale=0.8]
        % Nodes
        \foreach \i in {1,...,4} {
            \node[node] (u\i) at (\i,0) {};
        }
        % Edges
        \foreach \i in {1,...,3} {
            \foreach \j in {\i+1,...,4} {
                \draw[edge] (u\i) -- (u\j);
            }
        }
    \end{tikzpicture}
    \caption{Graph \( K_n^* \)}
\end{figure}

% Graph C\!K_n
\begin{figure}[h]
    \centering
    \begin{tikzpicture}[scale=0.8]
        % Nodes
        \foreach \i in {1,...,4} {
            \node[node] (u\i) at ({cos(90*\i)},