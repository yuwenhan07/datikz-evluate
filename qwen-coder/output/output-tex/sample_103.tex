Certainly! Below is a TikZ LaTeX code to visualize the Newsvendor Pricing Game scenario you described. This diagram will help illustrate the interaction between the supplier (leader) and the retailer (follower).

```latex
\documentclass[tikz,border=2cm]{standalone}
\usetikzlibrary{shapes.geometric, arrows.meta}

\tikzset{
  decision/.style={diamond, draw, fill=blue!20, text width=4.5em, text badly centered, node distance=3cm, inner sep=0pt},
  block/.style={rectangle, draw, fill=orange!30, text width=10em, text centered, rounded corners, minimum height=4em},
  line/.style={draw, thick, -Stealth},
  cloud/.style={cloud, draw, fill=red!20, text width=6em, text centered, aspect=2},
}

\begin{document}
\begin{tikzpicture}[node distance = 2cm]
    % Nodes
    \node [block] (supplier) {Supplier};
    \node [decision, below of=supplier] (retailer) {Retailer};
    
    % Lines
    \path [line] (supplier) -- node[anchor=east] {Wholesale Price: $a$} (retailer);
    \path [line] (retailer) -- node[anchor=south] {Purchase Quantity: $b$} ++(0,-1);
    \path [line] (retailer) -- node[anchor=north] {Retail Price: $p$} ++(0,1);
    
    % Arrows for interactions
    \draw [<->, dashed, thick] (supplier.east) -- node[midway, above] {Utility Function} ++(2,0);
    \draw [<->, dashed, thick] (retailer.west) -- node[midway, above] {Utility Function} ++(-2,0);
    
    % Clouds to represent uncertainty
    \node [cloud, left of=supplier, xshift=-3cm] (demand) {Demand};
    \node [cloud, right of=retailer, xshift=3cm] (market) {Market};
    
    % Labels
    \node [below of=demand, yshift=-1cm] {Uncertain Demand};
    \node [above of=market, yshift=1