Certainly! Below is an example of how you can create a TikZ diagram that represents the transformation \( G' \) from \( G \) with \( c = 3 \). This example includes two depictions of a triangle with a red dot in the middle, where the dots are labeled with numbers.

```latex
\documentclass{standalone}
\usepackage{tikz}

\begin{document}
\begin{tikzpicture}[scale=2]

% Define coordinates for the original triangle G
\coordinate (A) at (-1, -1);
\coordinate (B) at (1, -1);
\coordinate (C) at (0, 1);

% Draw the original triangle G
\draw[fill=white] (A) -- (B) -- (C) -- cycle;
\node[circle, fill=red, inner sep=1pt] (G_dot) at ($(A)!0.5!(B)$) {};
\node[circle, fill=red, inner sep=1pt] (G_midpoint) at ($(A)!0.5!(C)$) {};
\node[circle, fill=red, inner sep=1pt] (G_third_point) at ($(B)!0.5!(C)$) {};

% Label the points
\node[below left] at (A) {1};
\node[below right] at (B) {2};
\node[left] at (C) {3};

% Define coordinates for the transformed triangle G'
\coordinate (A_prime) at (-1.5, -1.5);
\coordinate (B_prime) at (1.5, -1.5);
\coordinate (C_prime) at (0, 1.5);

% Draw the transformed triangle G'
\draw[fill=white] (A_prime) -- (B_prime) -- (C_prime) -- cycle;
\node[circle, fill=red, inner sep=1pt] (G'_dot) at ($(A_prime)!0.5!(B_prime)$) {};
\node[circle, fill=red, inner sep=1pt] (G'_midpoint) at ($(A_prime)!0.5!(C_prime)$) {};
\node[circle, fill=red, inner sep=1pt] (G'_third_point) at ($(B_prime)!0.5!(C_prime)$) {};

% Label the points
\node[below left] at