\documentclass[tikz,border=3mm]{standalone}
\usetikzlibrary{calc,shapes.geometric}

\begin{document}

\tikzset{
    mydot/.style={circle,fill,inner sep=1pt},
    myline/.style={thick}
}

\begin{tikzpicture}[scale=2]
    % Define the axes
    \draw[->] (-1,0) -- (1.5,0) node[right] {$\frac{1}{p}$};
    \draw[->] (0,-1) -- (0,1.5) node[above] {$\frac{1}{q}$};

    % Draw the boundary lines
    \draw[dashed] (0,0) -- (1,0);
    \draw[dashed] (0,0) -- (0,1);

    % Draw the region for case (1)
    \draw[myline] (0,0) -- ($(0,0)!0.6!(1,0)$);
    \draw[myline] ($(0,0)!0.6!(1,0)$) -- ($(0,0)!0.8!(0,1)$);
    \draw[myline] ($(0,0)!0.8!(0,1)$) -- (0,0);
    \node at (0.3,0.1) {Case (1)};

    % Draw the line segment for case (2)
    \draw[myline] (0.4,0) -- ($(0.4,0)!0.4!(0,1)$);
    \node at (0.7,0.2) {Case (2)};
\end{tikzpicture}

\end{document}