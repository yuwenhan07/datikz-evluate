Certainly! Below is a TikZ LaTeX code that visualizes the deformation gradients \( F_+ \) and \( F_- \) for oscillating solutions:

```latex
\documentclass{standalone}
\usepackage{tikz}
\usetikzlibrary{arrows.meta, decorations.markings}

\begin{document}
\begin{tikzpicture}[scale=2]
    % Define coordinates for the identity matrix
    \coordinate (I1) at (0,0);
    \coordinate (I2) at (1,0);
    \coordinate (I3) at (0,1);
    \coordinate (I4) at (1,1);

    % Draw the identity matrix
    \draw[thick] (I1) rectangle (I2);
    \draw[thick] (I3) rectangle (I4);

    % Label the identity matrix
    \node at (0.5,-0.1) {Identity Matrix};
    \node at (-0.1,0.5) {Identity Matrix};

    % Define coordinates for F+
    \coordinate (Fp1) at (3,0);
    \coordinate (Fp2) at (4,0);
    \coordinate (Fp3) at (3,1);
    \coordinate (Fp4) at (4,1);

    % Draw F+
    \draw[thick, ->, >=Stealth] (Fp1) -- (Fp2);
    \draw[thick, ->, >=Stealth] (Fp3) -- (Fp4);
    \draw[thick, ->, >=Stealth] (Fp1) -- (Fp3);
    \draw[thick, ->, >=Stealth] (Fp2) -- (Fp4);

    % Label F+
    \node at (3.5,-0.1) {$F_+$};
    \node at (3,-0.6) {$F_0 + a \otimes \nu$};

    % Define coordinates for F-
    \coordinate (Fn1) at (6,0);
    \coordinate (Fn2) at (7,0);
    \coordinate (Fn3) at (6,1);
    \coordinate (Fn4) at (7,1);

    % Draw F-
    \draw[thick, ->, >=Stealth] (Fn