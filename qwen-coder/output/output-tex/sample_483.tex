To create an illustration of the reduction from the densest \( k \)-subgraph problem to the unweighted rectangle packing problem (u-rcp), we can use TikZ in LaTeX. Below is a sample code that generates the required diagram:

```latex
\documentclass{standalone}
\usepackage{tikz}

\begin{document}

\begin{tikzpicture}[node distance=1cm]

    % Left side: Simple undirected graph G
    \node (v1) at (0,0) [circle, draw] {v};
    \node (e) at (1,0) [circle, draw] {e};

    % Draw edges between v and e
    \draw (v1) -- (e);

    % Right side: 2-reduced directed graph of G
    \node (v1_1) at (-3,-2) [circle, draw] {v1};
    \node (v1_2) at (-3,-1) [circle, draw] {v1};
    \node (v1_3) at (-2,-2) [circle, draw] {v1};
    \node (v1_4) at (-2,-1) [circle, draw] {v1};
    
    \node (e_1) at (-1,-2) [circle, draw] {e};
    \node (e_2) at (-1,-1) [circle, draw] {e};
    
    % Draw edges within the copies of v
    \draw (v1_1) -- (v1_2);
    \draw (v1_3) -- (v1_4);
    \draw (v1_1) -- (v1_3);
    \draw (v1_2) -- (v1_4);
    
    % Draw edges from copies of v to e
    \draw (v1_1) -- (e_1);
    \draw (v1_2) -- (e_1);
    \draw (v1_3) -- (e_2);
    \draw (v1_4) -- (e_2);
    
    % Connect e1 and e2
    \draw (e_1) -- (e_2);

    % Labels
    \node[below] at (v1.south) {Left: Simple Undirected Graph \( G \