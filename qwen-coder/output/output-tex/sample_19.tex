To create a TikZ LaTeX diagram for the Hackenbush game described, we need to represent the structure of the game using nodes and edges. The game is given by the literal form:

\[
\{ 1-1, -\frac{1}{2} \mid \{ 0, 1 \mid \}
\]

This represents a tree-like structure where:
- The root has two children: one with value \(1-1\) (which is equivalent to \(0\)), and another with value \(-\frac{1}{2}\).
- The second child (with value \(-\frac{1}{2}\)) has two children: one with value \(0\) and another with value \(1\).

Here's how you can draw this using TikZ:

```latex
\documentclass{standalone}
\usepackage{tikz}

\begin{document}

\begin{tikzpicture}[level distance=3cm,
    level 1/.style={sibling distance=4cm},
    level 2/.style={sibling distance=2cm},
    level 3/.style={sibling distance=1cm}]
    
    % Root node
    \node (root) [circle,draw] {}
        % Left child (value 0)
        child {node (left) [circle,draw] {} 
            child {node (left_left) [circle,draw] {}}
            child {node (left_right) [circle,draw] {}}
        }
        % Right child (-1/2)
        child {node (right) [circle,draw] {} 
            child {node (right_left) [circle,draw] {}}
            child {node (right_right) [circle,draw] {}}
        };
        
    % Labeling the nodes
    \node at (left_left.west) {$0$};
    \node at (left_right.west) {$0$};
    \node at (right_left.west) {$0$};
    \node at (right_right.west) {$1$};
    
    % Drawing the edges
    \draw[-stealth] (root.south west) -- (left.north);
    \draw[-stealth] (root.south east) -- (right.north);
    
    \draw[-stealth] (left.south west) -- (left_left.north);
    \draw[-stealth] (left.south east