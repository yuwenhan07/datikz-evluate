Certainly! Below is an example of how you can create a TikZ diagram representing a genus one degeneration of curves using the dual graph approach described. This example assumes you have a specific dual graph in mind.

```latex
\documentclass{standalone}
\usepackage{tikz}

\begin{document}

\begin{tikzpicture}[scale=2]
    % Define nodes for the components
    \node[circle, draw, fill=blue!20] (C1) at (0,0) {C_1};
    \node[circle, draw, fill=red!20] (C2) at (2,0) {C_2};
    \node[circle, draw, fill=green!20] (C3) at (1,-1) {C_3};

    % Draw edges between nodes to represent intersections
    \draw[-stealth] (C1) -- node[above] {$C_{12}$} (C2);
    \draw[-stealth] (C1) -- node[left] {$C_{13}$} (C3);
    \draw[-stealth] (C2) -- node[right] {$C_{23}$} (C3);

    % Labeling the nodes with n_i values if necessary
    % For simplicity, we assume n_i = 1 for all components
    % If n_i != 1, uncomment the following lines:
    % \node[above left] at (0.5,0.5) {$n_1$};
    % \node[above right] at (2.5,0.5) {$n_2$};
    % \node[below] at (1.5,-1.5) {$n_3$};

    % Labeling the intersection numbers C_ij
    \node[above right] at (1,0.5) {$C_{12} = 1$};
    \node[left] at (0.5,-0.5) {$C_{13} = 1$};
    \node[right] at (2.5,-0.5) {$C_{23} = 1$};

    % Adding labels for k_i (all zero in this case)
    \node[below left] at (-0.2,-0.1) {$k_1 = 0$};
    \node[below right