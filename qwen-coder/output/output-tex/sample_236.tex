To create a visual representation of the configuration \(\mathcal{C}\) and its associated Möbius function \(\hat{\mu}_{\mathbbm{B}_S,\mathcal{C}}\) beside each node using TikZ in LaTeX, we need to define the structure of \(\mathcal{C}\), calculate the Möbius function for each node, and then draw the graph with the Möbius values displayed.

Here's an example of how you might set up this visualization:

```latex
\documentclass[tikz,border=5mm]{standalone}
\usetikzlibrary{shapes.geometric, positioning}

% Define colors
\definecolor{red}{RGB}{255,0,0}
\definecolor{green}{RGB}{0,255,0}
\definecolor{blue}{RGB}{0,0,255}

% Function to calculate Möbius function (simplified version)
\newcommand{\mobius}[1]{
  \ifnum#1=1 1
  \else ifnum#1=2 -1
  \else 0
  \fi
}

\begin{document}
\begin{tikzpicture}[node distance=3cm]

  % Nodes
  \node[draw,circle,fill=red,minimum size=8mm,label=left:$v_1$] (v1) {};
  \node[draw,circle,fill=green,minimum size=8mm,right of=v1,label=right:$v_2$] (v2) {};
  \node[draw,circle,fill=blue,minimum size=8mm,below of=v1,label=below:$v_3$] (v3) {};
  
  % Edges
  \draw[-stealth] (v1) -- (v2);
  \draw[-stealth] (v1) -- (v3);
  \draw[-stealth] (v2) -- (v3);

  % Möbius function values
  \node[right=of v1,font=\tiny] {$\hat{\mu}(v_1) = \mobius{1}$};
  \node[right=of v2,font=\tiny] {$\hat{\mu}(v_2) = \mobius{2}$};
  \node[right=of v3,font=\tiny] {$\hat{\mu}(v