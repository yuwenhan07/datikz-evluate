To create the described TikZ diagram in LaTeX, you can use the following code:

```latex
\documentclass[tikz,border=5mm]{standalone}
\usetikzlibrary{arrows.meta}

\begin{document}
\begin{tikzpicture}[scale=2]
    % Define nodes for vertices of the triangle
    \node[circle, draw, fill=white] (v1) at (0,0) {};
    \node[circle, draw, fill=white] (v2) at (1,0) {};
    \node[circle, draw, fill=white] (v3) at (0.5,-sqrt(3)/2) {};

    % Draw edges with labels
    \draw[-Stealth, thick] (v1) -- node[midway, above] {$\psi_{(v_1,v_2)}$} (v2);
    \draw[-Stealth, thick] (v2) -- node[midway, above] {$\psi_{(v_2,v_3)}$} (v3);
    \draw[-Stealth, thick] (v3) -- node[midway, above] {$\psi_{(v_3,v_1)}$} (v1);

    % Draw the cyclic path C
    \draw[dashed, blue] (v3) to[bend left=60] (v1) to[bend left=60] (v2) to[bend left=60] (v3);

    % Label the angles
    \node at ($(v1)!0.2!(v2)$) [anchor=south west] {\(\theta_{(v_1,v_2)} = \frac{\pi}{4}\)};
    \node at ($(v2)!0.2!(v3)$) [anchor=south east] {\(\theta_{(v_2,v_3)} = \frac{\pi}{4}\)};
    \node at ($(v3)!0.2!(v1)$) [anchor=north] {\(\theta_{(v_3,v_1)} = \frac{\pi}{4}\)};

    % Points z1, z1', z2, z3
    \node[below right] at (v1) {\(\mathbf{C}_{v_1}\)};
    \node[above right] at (v2) {\(\