Certainly! Below is a TikZ LaTeX code to generate a Hasse diagram for the poset of discontinuity points, assuming you have defined the elements \( u \) and \( v \) as required.

```latex
\documentclass{article}
\usepackage{tikz}

% Define custom arrows for Hasse diagram
\tikzset{
    >=stealth,
    level 1/.style={sibling distance=3cm},
    level 2/.style={sibling distance=2cm},
    level 3/.style={sibling distance=1cm},
}

\begin{document}

\begin{figure}[h]
    \centering
    \begin{tikzpicture}[level distance=2cm]
        % Nodes representing elements of the poset
        \node (root) {u};
        
        % Level 1 nodes
        \node [below left of=root] (child1) {u_1};
        \node [below right of=root] (child2) {u_2};

        % Level 2 nodes
        \node [below left of=child1] (grandchild1) {u_{11}};
        \node [below right of=child1] (grandchild2) {u_{12}};
        \node [below left of=child2] (grandchild3) {u_{21}};
        \node [below right of=child2] (grandchild4) {u_{22}};

        % Draw edges between nodes
        \draw[->] (root) -- (child1);
        \draw[->] (root) -- (child2);
        \draw[->] (child1) -- (grandchild1);
        \draw[->] (child1) -- (grandchild2);
        \draw[->] (child2) -- (grandchild3);
        \draw[->] (child2) -- (grandchild4);

        % Label edges with ≤ relation
        \path[->] node[midway, above] {$\leq$} (root) edge (child1);
        \path[->] node[midway, below] {$\leq$} (root) edge (child2);
        \path[->] node[midway, above] {$\leq$} (child1) edge (grandchild1);
        \path[->] node[midway, below] {$