Certainly! Below is a TikZ LaTeX code that depicts the transformation of the domain of the colored S6V model. This code includes nodes labeled with \( z_i \) to represent the spectral parameters for the colored S6V weights on each row.

```latex
\documentclass{standalone}
\usepackage{tikz}

\begin{document}

\begin{tikzpicture}[node distance=2cm, auto]
    % Nodes representing rows and columns
    \node (row1) at (0, 3) {Row 1};
    \node (row2) at (0, 1) {Row 2};
    \node (row3) at (0, -1) {Row 3};

    \node (col1) at (-4, 2) {Column 1};
    \node (col2) at (-4, 0) {Column 2};
    \node (col3) at (-4, -2) {Column 3};
    \node (col4) at (-4, -4) {Column 4};
    \node (col5) at (-4, -6) {Column 5};
    \node (col6) at (-4, -8) {Column 6};

    % Drawing arrows between nodes
    \draw[->] (row1.south) -- (col1.north);
    \draw[->] (row1.south) -- (col2.north);
    \draw[->] (row1.south) -- (col3.north);
    \draw[->] (row1.south) -- (col4.north);
    \draw[->] (row1.south) -- (col5.north);
    \draw[->] (row1.south) -- (col6.north);

    \draw[->] (row2.south) -- (col1.north);
    \draw[->] (row2.south) -- (col2.north);
    \draw[->] (row2.south) -- (col3.north);
    \draw[->] (row2.south) -- (col4.north);
    \draw[->] (row2.south) -- (col5.north);
    \draw[->] (row2.south) -- (col6.north);

    \draw[->] (row