To create a TikZ diagram that represents both a \(3 \times 3\) quadrilateral mesh and a flexible mesh with fixed angles, we can use the following LaTeX code:

```latex
\documentclass[tikz,border=5mm]{standalone}
\usetikzlibrary{calc,angles,quotes}

\begin{document}
\begin{tikzpicture}[scale=2]

% Draw the 3x3 quadrilateral mesh
\foreach \i in {0,...,2} {
    \foreach \j in {0,...,2} {
        \node[draw,circle,inner sep=1pt] (q\i\j) at (\i,\j) {};
    }
}

\foreach \i in {0,...,2} {
    \foreach \j in {0,...,1} {
        \draw (q\i\j) -- (q\i\j+1);
    }
    \foreach \j in {0,...,2} {
        \draw (q\j\i) -- (q\j+1\i);
    }
}

% Highlight the smaller area for flexibility
\draw[dashed] (q00) rectangle (q11);

% Define points for the flexible mesh
\coordinate (A) at (0,0);
\coordinate (B) at (1,0);
\coordinate (C) at (1,1);
\coordinate (D) at (0,1);
\coordinate (E) at (0.5,0.5);
\coordinate (F) at (1.5,0.5);
\coordinate (G) at (1.5,1.5);
\coordinate (H) at (0.5,1.5);

% Draw the flexible mesh
\draw (A) -- (B) -- (C) -- (D) -- cycle;
\draw (A) -- (E);
\draw (B) -- (F);
\draw (C) -- (G);
\draw (D) -- (H);

% Label the angles
\pic [angle radius=1cm,"$\alpha_1$",draw] {angle = F--B--A};
\pic [angle radius=1cm,"$\alpha_2$",draw] {angle = G--C--B};
\pic [angle radius=1cm,"$\alpha_3$",draw] {angle = H--D--C};
\