Certainly! Below is a TikZ LaTeX code snippet that generates a diagram representing the profile \( P \) as described in your prompt. This diagram will include horizontal bars representing each agent's ranking, with the top-ranked alternative highlighted, and specific alternatives marked for symmetry.

```latex
\documentclass[tikz,border=5pt]{standalone}
\usepackage{tikz}
\usetikzlibrary{arrows.meta}

\tikzset{
    alt/.style={rectangle, minimum width=2cm, minimum height=1cm, draw=black},
    top-alt/.style={alt, fill=blue!30},
    mark-alt/.style={alt, fill=red!30},
    agent-bar/.style={rectangle, minimum width=2cm, minimum height=0.5cm, draw=gray}
}

\begin{document}
\begin{tikzpicture}[scale=0.8]

% Agents
\node[agent-bar] (A1) at (0,0) {Agent 1};
\node[agent-bar] (A2) at (0,-1) {Agent 2};
\node[agent-bar] (A3) at (0,-2) {Agent 3};
\node[agent-bar] (A4) at (0,-3) {Agent 4};

% Alternatives
\node[alt] (A) at (-3,-0.25) {1};
\node[alt] (B) at (-2,-0.25) {2};
\node[alt] (C) at (-1,-0.25) {3};
\node[alt] (D) at (0,-0.25) {4};
\node[alt] (E) at (1,-0.25) {5};
\node[alt] (F) at (2,-0.25) {6};
\node[alt] (G) at (3,-0.25) {7};
\node[alt] (H) at (-3,-1.25) {8};
\node[alt] (I) at (-2,-1.25) {9};
\node[alt] (J) at (-1,-1.25) {10};
\node[alt] (K) at (0,-1.25) {11};
\node[alt] (