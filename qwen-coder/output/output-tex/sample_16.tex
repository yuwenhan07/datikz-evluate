Certainly! Below is a TikZ LaTeX code that visually represents the described scenario. This code will create a graph with nodes representing vertices and edges representing paths.

```latex
\documentclass{standalone}
\usepackage{tikz}

\begin{document}
\begin{tikzpicture}[node distance=2cm, auto]
    % Nodes
    \node (Q) at (0,0) {Q};
    \node (Qp) at (3,0) {Q'};
    \node (y) at (6,0) {y};
    \node (x1) at (9,0) {x1};
    \node (V1) at (1.5,-2) {V1};
    \node (Vu) at (4.5,-2) {V+1};

    % Edges for Q
    \draw (Q) -- node[above] {$P_1$} (V1);
    \draw (Q) -- node[above] {$P_2$} (Vu);

    % Edges for Q'
    \draw (Qp) -- node[above] {$P_1$} (V1);
    \draw (Qp) -- node[above] {$P_2$} (Vu);

    % Edges for y
    \draw (y) -- node[right] {$P_u$} (V1);
    \draw (y) -- node[right] {$x_1$} (x1);

    % Edges for x1
    \draw (x1) -- node[left] {$P_y$} (Vu);
    \draw (x1) -- node[left] {$v$} (V1);

    % Edges for S(V1)
    \draw (V1) -- node[below] {$S(V_1)$} (Vu);

    % Labels for sets
    \node[below right=of V1] (V1_set) {$V(Q') - V_1$};
    \node[below left=of Vu] (Vu_set) {$V(Q') - V_+$};

    % Arrows indicating collections
    \draw[-stealth] (V1) -- node[midway, above] {not covered by $P_1$, $P_u$, $x_1$, and $S(V_1)$} (y);
    \draw[-stealth]