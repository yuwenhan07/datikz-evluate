To create a TikZ diagram representing a tree \( T \) with the properties described, we need to follow these steps:

1. **Define the Tree Structure**: The tree \( T \) should have \( n \) leaves such that \( \beta(T^3) = n \). This implies that \( T \) should be structured in a way that its cube \( T^3 \) has exactly \( n \) leaves.
2. **Identify Metric Basis Vertices**: The red vertices should form a metric basis for \( T \), meaning they should be enough to uniquely determine any other vertex in the tree by their distances.
3. **Insert Blue Vertices**: The blue vertices should be added to \( T \) to form a metric basis for \( T^3 \).

Given that \( \beta(T^3) = n \), we can infer that \( T \) must be a path or a structure that, when cubed, results in \( n \) leaves. For simplicity, let's assume \( T \) is a path.

Here is the TikZ code to generate this diagram:

```latex
\documentclass{standalone}
\usepackage{tikz}

\begin{document}
\begin{tikzpicture}[level distance=1.5cm,
  level 1/.style={sibling distance=3cm},
  level 2/.style={sibling distance=2cm},
  level 3/.style={sibling distance=1.5cm}]
  
  % Define the tree T
  \node (root) [circle, draw, fill=blue] {}
    child {node (a) [circle, draw, fill=red] {}}
    child {node (b) [circle, draw, fill=red] {}
      child {node (c) [circle, draw, fill=red] {}}
      child {node (d) [circle, draw, fill=red] {}}
    }
    child {node (e) [circle, draw, fill=red] {}};
  
  % Draw the edges
  \draw (root) -- (a);
  \draw (root) -- (b);
  \draw (b) -- (c);
  \draw (b) -- (d);
  \draw (root) -- (e);
  
  % Add blue vertices to form T^3
  \node (f) [circle, draw, fill=blue] at ($(