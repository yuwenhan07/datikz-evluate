Certainly! Below is a TikZ LaTeX code that illustrates the described passageway environment:

```latex
\documentclass[tikz]{standalone}
\usetikzlibrary{shapes.geometric}

\tikzset{
    agent/.style={circle, fill=blue},
    wall/.style={rectangle, fill=black},
    switch/.style={circle, draw=red, thick},
    door/.style={rectangle, draw=gray, thick},
    goal/.style={circle, fill=green}
}

\begin{document}
\begin{tikzpicture}[node distance=2cm]

    % Walls
    \node[wall] (W1) at (0, 4);
    \node[wall] (W2) at (0, 3);
    \node[wall] (W3) at (0, 2);
    \node[wall] (W4) at (0, 1);
    \node[wall] (W5) at (0, 0);
    
    \node[wall] (W6) at (1, 4);
    \node[wall] (W7) at (2, 4);
    \node[wall] (W8) at (3, 4);
    \node[wall] (W9) at (4, 4);
    \node[wall] (W10) at (5, 4);
    \node[wall] (W11) at (6, 4);
    \node[wall] (W12) at (7, 4);
    \node[wall] (W13) at (8, 4);
    \node[wall] (W14) at (9, 4);
    \node[wall] (W15) at (10, 4);
    
    \node[wall] (W16) at (1, 3);
    \node[wall] (W17) at (2, 3);
    \node[wall] (W18) at (3, 3);
    \node[wall] (W19) at (4, 3);
    \node[wall] (W20) at (5, 3);
    \node[wall] (W21) at (6, 3);
    \node[wall] (W22)