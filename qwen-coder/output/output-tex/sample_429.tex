\documentclass[tikz,border=10pt]{standalone}
\usetikzlibrary{shapes.geometric,positioning}

\begin{document}

\tikzset{
    point/.style={circle, fill=black, inner sep=2pt},
    line/.style={thick, -stealth}
}

\begin{tikzpicture}[node distance=2cm]

% Define the points
\node (A) [point] {};
\node (B) [point, right of=A] {};
\node (C) [point, above of=B] {};
\node (D) [point, left of=C] {};
\node (E) [point, below of=D] {};
\node (F) [point, left of=E] {};

% Connect the points
\draw[line] (A) -- (B);
\draw[line] (B) -- (C);
\draw[line] (C) -- (D);
\draw[line] (D) -- (E);
\draw[line] (E) -- (F);
\draw[line] (F) -- (A);

% Additional connections to make it more complex
\draw[line] (A) -- (C);
\draw[line] (B) -- (D);
\draw[line] (C) -- (E);
\draw[line] (D) -- (F);
\draw[line] (E) -- (A);
\draw[line] (F) -- (B);

% Label the points
\node[above left] at (A.west) {A};
\node[above right] at (B.east) {B};
\node[right] at (C.north) {C};
\node[left] at (D.south) {D};
\node[below left] at (E.west) {E};
\node[below right] at (F.east) {F};

\end{tikzpicture}

\end{document}