Certainly! Below is the TikZ LaTeX code to generate a caterpillar graph with the specified degree sequence \((5, 4^3, 2, 1^{11})\). The spine vertices will be gray, and the leaves will be yellow.

```latex
\documentclass[tikz,border=10pt]{standalone}
\usetikzlibrary{graphs,graphs.standard}

\begin{document}
\begin{tikzpicture}[scale=0.8]
    % Define colors
    \colorlet{spine}{gray}
    \colorlet{leaf}{yellow}

    % Draw the spine
    \node[draw, circle, fill=spine] (v1) at (0,0) {};
    \foreach \i in {1,...,4} {
        \node[draw, circle, fill=spine] (v\i) at (\i*2,0) {};
    }
    \foreach \i in {5} {
        \node[draw, circle, fill=spine] (v\i) at (10,0) {};
    }

    % Connect spine nodes
    \foreach \i [count=\j from 2] in {1,...,4} {
        \draw (v\j) -- (v\i);
    }
    \draw (v5) -- (v4);

    % Draw the leaves
    \foreach \i in {6,...,17} {
        \node[draw, circle, fill=leaf] (l\i) at (\i/2,1) {};
        \draw (l\i) -- (v4);
    }
    \foreach \i in {18,...,29} {
        \node[draw, circle, fill=leaf] (l\i) at (\i/2,-1) {};
        \draw (l\i) -- (v4);
    }

    % Labeling the spine vertices
    \node at (0,-1.5) {\textcolor{spine}{Spine}};
    \node at (2,-1.5) {\textcolor{spine}{Spine}};
    \node at (4,-1.5) {\textcolor{spine}{Spine}};
    \node at (10,-1.5) {\textcolor{spine}{Spine}};

    % Labeling the leaves
    \