To create a TikZ LaTeX diagram that visually represents the behavior of the function \(\varphi_\Delta(x)\), we need to break down the components of the function and plot them separately before combining them. The function is given by:

\[
\varphi_\Delta(x) = B_1 - \left(\frac{\diffc^2}{2} + B_2\right)x^{2(\alpha-1)} + \frac{b(0)}{x} + b(0)\left(\diffc^2 + B_2\right)x^{2\alpha-3}\Delta
\]

Here, \(\Delta\) is a small parameter that scales the last term. Let's assume some values for \(B_1\), \(B_2\), \(\diffc\), \(\alpha\), \(b(0)\), and \(\Delta\) for simplicity. For example, let:
- \(B_1 = 1\)
- \(B_2 = 0.5\)
- \(\diffc = 1\)
- \(\alpha = 1.5\)
- \(b(0) = 1\)
- \(\Delta = 0.1\)

Now, we can write the TikZ code to plot this function.

```latex
\documentclass{standalone}
\usepackage{tikz}
\usetikzlibrary{calc}

\begin{document}

\begin{tikzpicture}[scale=1]
    % Define constants
    \def\BOne{1}
    \def\BBTwo{0.5}
    \def\deltac{1}
    \def\alphaVal{1.5}
    \def\bZero{1}
    \def\delta{0.1}

    % Function definition
    \newcommand{\phiDelta}[1]{
        \pgfmathsetmacro{\termOne}{\BOne}
        \pgfmathsetmacro{\termTwo}{-\left(\deltac^2/2 + \BBTwo\right)*#1^(2*({\alphaVal}-1))}
        \pgfmathsetmacro{\termThree}{\bZero/#1}
        \pgfmathsetmacro{\termFour}{\bZero*\left(\deltac^2 + \BBTwo\right)*#1^(2*\alphaVal-3)*\delta}
        \pgfmath