\documentclass[tikz,border=2mm]{standalone}
\usetikzlibrary{shapes.geometric, arrows.meta}

\tikzset{
    state/.style={circle, thick, draw=black, fill=white!30},
    edge/.style={->, >=stealth', shorten >=1pt},
    color1/.style={red},
    color2/.style={blue},
    color3/.style={green},
    color4/.style={orange},
    color5/.style={purple},
    color6/.style={cyan},
}

\begin{document}

\begin{tikzpicture}[node distance=2cm]

% States
\node[state] (q0) {$q_0$};
\node[state] (q1) [below left of=q0] {$q_1$};
\node[state] (q2) [above right of=q0] {$q_2$};

% Edges
\draw[edge, color1] (q0) -- node[above] {$(\chi_+, \chi_-)$} (q1);
\draw[edge, color2] (q0) -- node[below] {$(\chi_-, \chi_+)$} (q2);
\draw[edge, color3] (q1) -- node[left] {$(\chi_-, \chi_+)$} (q0);
\draw[edge, color4] (q1) -- node[right] {$(\chi_+, \chi_-)$} (q2);
\draw[edge, color5] (q2) -- node[left] {$(\chi_+, \chi_-)$} (q0);
\draw[edge, color6] (q2) -- node[right] {$(\chi_-, \chi_+)$} (q1);

\end{tikzpicture}

\end{document}