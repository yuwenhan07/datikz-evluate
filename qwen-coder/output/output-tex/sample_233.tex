To generate a TikZ LaTeX diagram that represents the supports of the polynomial \( f(X) = 1 + x^{-1}y^{-1} + x^{-1}y^{1} + x^{1}y^{-1} + x^{1}y^{1} \) and its dilations \( f(X^2) \) and \( f(X^3) \), we need to understand the structure of these polynomials in terms of their support sets.

### Step-by-Step Explanation:

1. **Original Polynomial \( f(X) \)**:
   - The polynomial \( f(X) \) is given by:
     \[
     f(X) = 1 + x^{-1}y^{-1} + x^{-1}y^{1} + x^{1}y^{-1} + x^{1}y^{1}
     \]
   - The support set of \( f(X) \) consists of the exponents of \( X \) and \( Y \):
     \[
     \{(0,0), (-1,-1), (-1,1), (1,-1), (1,1)\}
     \]

2. **Dilation \( f(X^2) \)**:
   - When we dilate \( f(X) \) by \( X^2 \), each term in the polynomial will have its exponent increased by 2 for \( X \).
   - The new polynomial \( f(X^2) \) will be:
     \[
     f(X^2) = 1 + x^{-2}y^{-2} + x^{-2}y^{2} + x^{2}y^{-2} + x^{2}y^{2}
     \]
   - The support set of \( f(X^2) \) consists of the exponents of \( X \) and \( Y \):
     \[
     \{(0,0), (-2,-2), (-2,2), (2,-2), (2,2)\}
     \]

3. **Dilation \( f(X^3) \)**:
   - When we dilate \( f(X) \) by \( X^3 \), each term in the polynomial will have its exponent increased by 3 for \( X \).
   - The new polynomial \( f(X^3) \) will be:
     \[
     f(X^3) = 1 + x^{-3}y^{-3} + x^{-