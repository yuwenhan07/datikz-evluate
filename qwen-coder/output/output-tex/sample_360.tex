To create a TikZ LaTeX diagram representing the frame classes as rectangles with their respective buffers and transitions, you can use the following code:

```latex
\documentclass{article}
\usepackage{tikz}

\begin{document}

\begin{tikzpicture}[node distance=2cm]

% Define styles
\tikzset{
    rectangle/.style={draw, rectangle, minimum width=3cm, minimum height=1cm},
    buffer/.style={fill=gray!50, draw=black},
    arrow/.style={->, >=stealth', shorten >=1pt},
    color/green/.style={green},
    color/blue/.style={blue},
    color/orange/.style={orange},
    color/red/.style={red},
    color/black/.style={black}
}

% Nodes
\node (0) [rectangle, label={[yshift=-1.5em]below:$0$}] {};
\node (1) [rectangle, right of=0, label={[yshift=-1.5em]below:$1$}] {};
\node (2) [rectangle, below of=1, label={[yshift=-1.5em]below:$2$}] {};
\node (3) [rectangle, left of=2, label={[yshift=-1.5em]below:$3$}] {};
\node (4) [rectangle, above of=3, fill=none, draw=none, label={[yshift=-1.5em]above:$4$}] {};

% Buffers
\node (buffer0) [buffer, below left of=0, xshift=-0.5cm, yshift=-0.5cm] {};
\node (buffer1) [buffer, below right of=0, xshift=0.5cm, yshift=-0.5cm] {};
\node (buffer2) [buffer, below left of=1, xshift=-0.5cm, yshift=-0.5cm] {};
\node (buffer3) [buffer, below right of=1, xshift=0.5cm, yshift=-0.5cm] {};
\node (buffer4) [buffer, below left of=2, xshift=-0.5cm, yshift=-0.5cm] {};
\node (buffer5) [buffer, below right of=2, xshift=0.5cm, yshift=-0.5