\documentclass{article}
\usepackage{tikz-cd}

\begin{document}

\[
\begin{tikzcd}
H^*(Q_4) \arrow[r, "M_1"] & H^*(Q_4) \\
1 \arrow[u, "\text{identity}" description] \arrow[ru, "0" description]
\end{tikzcd}
\]

Explanation:
- The diagram has two objects: \( H^*(Q_4) \) on top and \( H^*(Q_4) \) on the bottom.
- There is an arrow labeled \( M_1 \) from the top \( H^*(Q_4) \) to the bottom \( H^*(Q_4) \).
- There is an identity map from the bottom \( H^*(Q_4) \) to itself, labeled as \( 1 \).
- There is a diagonal arrow labeled \( 0 \) from the bottom \( H^*(Q_4) \) to the top \( H^*(Q_4) \).

This diagram visually represents the relation \( M_1 \cdot 1 = 0 \) in \( H^*(Q_4) \).

\end{document}