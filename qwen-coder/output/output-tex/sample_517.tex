Certainly! Below is a TikZ LaTeX code snippet that illustrates the quantum circuits described:

```latex
\documentclass[tikz,border=5mm]{standalone}
\usepackage{tikz-feynman}

\tikzfeynmanset{
    dot/.style={circle, fill=black, inner sep=2pt},
    small dot/.style={circle, fill=black, inner sep=1pt},
    double line/.style={thick},
}

\begin{document}
\begin{tikzpicture}[scale=0.8]
    \begin{feynman}
        \vertex (A);
        \vertex[right=of A] (B);
        \vertex[right=of B] (C);
        \vertex[right=of C] (D);
        \vertex[right=of D] (E);
        \vertex[right=of E] (F);
        \vertex[right=of F] (G);

        \diagram* {
            (A) -- [double line] node[dot] {} -- (B),
            (B) -- [double line] node[dot] {} -- (C),
            (C) -- [double line] node[dot] {} -- (D),
            (D) -- [double line] node[dot] {} -- (E),
            (E) -- [double line] node[dot] {} -- (F),
            (F) -- [double line] node[dot] {} -- (G)
        };

        % Measurement nodes
        \node[below=of A] (M1) {$\sigma_{\alpha}$};
        \node[below=of C] (M2) {$\sigma_{\beta}$};
        \node[below=of E] (M3) {$\sigma_{\alpha}$};

        % Amplitude damping channel
        \node[left=of B] (E1) {};
        \node[right=of G] (E2) {};

        % Bayesian inverse channel
        \node[left=of D] (F1) {};
        \node[right=of F] (F2) {};

        % Gates for sigma_y and R_z
        \node[above=of B, yshift=0.5cm] (H) {H};
        \node[above=of C, yshift=0.5cm] (Ryz) {R_y(\pi/2)};
       