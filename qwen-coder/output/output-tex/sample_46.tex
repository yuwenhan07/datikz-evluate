Certainly! Below is a TikZ LaTeX code that represents an infinite decision tree with branches corresponding to the elements \( \{a, b, c\} \). The tree is colored in blue to indicate its role as part of a balancer.

```latex
\documentclass[tikz]{standalone}
\usetikzlibrary{shapes.geometric, arrows.meta}

\tikzset{
    decision/.style={isosceles triangle, isosceles triangle apex angle=60, draw=blue, fill=blue!20},
    line/.style={draw, thick, ->, >=stealth'},
    level distance=1.5cm,
    level 1/.style={sibling distance=4cm},
    level 2/.style={sibling distance=2cm},
    level 3/.style={sibling distance=1cm}
}

\begin{document}
\begin{tikzpicture}[scale=0.8]
    % Root node
    \node[decision] (root) {Root};
    
    % Level 1 nodes
    \node[decision, below left of=root] (left) {a};
    \node[decision, below right of=root] (right) {b};
    \node[decision, below of=root] (middle) {c};
    
    % Level 2 nodes
    \node[decision, below left of=left] (left-left) {a};
    \node[decision, below right of=left] (left-right) {b};
    \node[decision, below of=left] (left-middle) {c};
    
    \node[decision, below left of=right] (right-left) {a};
    \node[decision, below right of=right] (right-right) {b};
    \node[decision, below of=right] (right-middle) {c};
    
    \node[decision, below left of=middle] (middle-left) {a};
    \node[decision, below right of=middle] (middle-right) {b};
    \node[decision, below of=middle] (middle-middle) {c};
    
    % Draw lines between nodes
    \path [line] (root) -- (left);
    \path [line] (root) -- (right);
    \path [line] (root) -- (middle);
    
    \path [line] (left) -- (left