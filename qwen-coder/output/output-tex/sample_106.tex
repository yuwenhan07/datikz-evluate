To create a TikZ diagram that represents an \(\mathsf{MFL}[3]\)-frame \(\frk{F}\) where the relativization to each layer is a locally finite \(\mathsf{S5}_2\)-algebra, but \(\frk{F}^*\) (the dual frame) is not locally finite, we need to carefully represent the layers and their properties.

Here's a step-by-step approach to creating this diagram:

1. **Layers Representation**: We will represent the three layers as horizontal lines.
2. **Locally Finite Property**: We will indicate that each layer is locally finite by showing a finite number of elements within each layer.
3. **Non-Local Finiteness in Dual Frame**: We will show that the dual frame \(\frk{F}^*\) has infinitely many elements.

Let's write the TikZ code for this:

```latex
\documentclass[tikz,border=10pt]{standalone}
\usepackage{amsmath}

\tikzset{
    layer/.style={draw, thick, fill=gray!20},
    element/.style={circle, draw, fill=blue!50, inner sep=2pt},
    non-finite/.style={fill=red!50}
}

\begin{document}
\begin{tikzpicture}[scale=1]

% Layers
\node[layer] at (0,4) (layer1) {Layer 1};
\node[layer] at (0,2) (layer2) {Layer 2};
\node[layer] at (0,0) (layer3) {Layer 3};

% Elements in each layer
\node[element] at (-1,4) {};
\node[element] at (0,4) {};
\node[element] at (1,4) {};
\node[element] at (-1,2) {};
\node[element] at (0,2) {};
\node[element] at (1,2) {};
\node[element] at (-1,0) {};
\node[element] at (0,0) {};
\node[element] at (1,0) {};

% Non-finite element in the dual frame
\node[non-finite] at (0,-2) {Non-finite Element};

% Arrows between layers
\draw[-stealth] (layer1.south) -- node[midway, below] {$\alpha_