To create an illustration of a \((k,3)\)-decomposition \((F_k, F_3)\) of a long \((k,3)\)-out-forcer for some \(k \geq 4\), we need to understand the structure of these decompositions. A \((k,3)\)-out-forcer is a directed graph where each vertex has out-degree at most 3, and it can be decomposed into two subgraphs \(F_k\) and \(F_3\) such that:

1. \(F_k\) is a directed path of length \(k-1\) (i.e., \(k\) vertices connected by \(k-1\) edges).
2. \(F_3\) is a directed cycle of length 3 (i.e., 3 vertices forming a triangle).

The decomposition involves assigning arcs from \(F_k\) and \(F_3\) to the original graph.

Here's the TikZ LaTeX code to illustrate this:

```latex
\documentclass{standalone}
\usepackage{tikz}

\begin{document}
\begin{tikzpicture}[scale=0.8]
    % Draw the path F_k
    \foreach \x in {0,...,3} {
        \node[circle, draw] (V\x) at (\x,0) {};
        \ifnum\x<3
            \draw[thick, green!50!black] (V\x) -- (V\numexpr\x+1);
        \fi
    }

    % Draw the cycle F_3
    \foreach \x in {0,1,2} {
        \node[circle, draw] (W\x) at (\x,-2) {};
        \ifnum\x<2
            \draw[thick, red!50!black] (W\x) -- (W\numexpr\x+1);
        \fi
    }
    \draw[thick, red!50!black] (W2) -- (W0);

    % Connect F_k and F_3
    \draw[dashed, thick, red!50!black] (V0) -- (W0);
    \draw[dashed, thick, red!50!black] (V1) -- (W1);
    \draw[dashed, thick, red!50!black] (V2) -- (W2);

    % Label