Certainly! Below is a TikZ LaTeX code that creates a diagram illustrating the theoretical approximations for \(E^N_2\) on the left, the speed of selection \(\nu_N\) on the right, along with simulated values (black circles). The parameters include \(N = 10^6\), \(c = \frac{1}{2}\), and different regimes of selection strength.

```latex
\documentclass[tikz,border=5mm]{standalone}
\usetikzlibrary{arrows.meta, decorations.markings}

\begin{document}
\begin{tikzpicture}[scale=0.8]

% Left Panel: E^N_2 vs Parameter
\draw[thick,->] (-3,0) -- (4,0) node[right] {$\text{Parameter}$};
\draw[thick,->] (-3,0) -- (-3,4) node[above] {$E^N_2$};

% Right Panel: Speed of Selection vs Parameter
\draw[thick,->] (4,0) -- (10,0) node[right] {$\text{Parameter}$};
\draw[thick,->] (4,0) -- (4,4) node[above] {$\nu_N$};

% Simulated Values (Black Circles)
\foreach \x/\y in {0/2, 1/3, 2/4, 3/5, 4/6} {
    \fill (\x-3, \y) circle[radius=2pt];
}
\foreach \x/\y in {5/1, 6/2, 7/3, 8/4, 9/5} {
    \fill (\x-3, \y) circle[radius=2pt];
}

% Labels
\node at (-3,-0.5) [below] {\textbf{Left Panel}};
\node at (4,-0.5) [below] {\textbf{Right Panel}};

% Legends
\node at (-3,4.5) [left] {\textcolor{black}{$E^N_2$}};
\node at (4,4.5) [right] {\textcolor{black}{$\nu_N$}};
\node at (-3,4.2) [left] {\textcolor{black}{Simulated Values}};

\end{t