Certainly! Below is a TikZ LaTeX code that generates a diagram based on your description. This diagram includes a cycle labeled `_0^-1_1` with encircled elements, a white background, and three columns of numbers and variables colored red, white, and black.

```latex
\documentclass[tikz,border=3mm]{standalone}
\usetikzlibrary{shapes.geometric,backgrounds}

\begin{document}
\begin{tikzpicture}[node distance=2cm]
    % Background rectangle
    \fill[white] (0,-4) rectangle (6,4);

    % Cycle
    \draw[thick] (0,0) circle (2);
    \node[circle,draw,inner sep=1pt,label={above:$0$}] at (0,0) {};
    \node[circle,draw,inner sep=1pt,label={below:$-1$}] at (0,-2) {};
    \node[circle,draw,inner sep=1pt,label={right:$1$}] at (2,0) {};

    % Encircled elements
    \foreach \i/\j in {0/1, 1/-1, -1/0} {
        \node[circle,draw,inner sep=1pt] at (\i,\j) {};
    }

    % Columns
    \foreach \x/\y/\color in {1/0/red, 2/0/white, 3/0/black} {
        \node[rectangle,draw,fill=\color,minimum width=1cm,minimum height=2cm] at (\x,\y) {};
        \node[text=black,anchor=north west,xshift=5pt,yshift=-5pt] at (\x,\y) {\textbf{\x}};
    }

    % Equations
    \node[anchor=east,text=red] at (1,2) {$a = b + c$};
    \node[anchor=east,text=red] at (1,0) {$d = e - f$};
    \node[anchor=east,text=red] at (1,-2) {$g = h \times i$};

    \node[anchor=east,text=white] at (2,2) {$j = k / l$};
    \node[anchor=east,text=white] at (2,0) {$m = n