Certainly! Below is a TikZ LaTeX code snippet that generates a figure illustrating the scenario described. This figure will show a circle with radius \( r \) centered at a vertex of a polygon, along with the tangent lines and the lengths involved.

```latex
\documentclass[tikz,border=3mm]{standalone}
\usetikzlibrary{calc}

\begin{document}

\begin{tikzpicture}[scale=2]
    % Define the center of the circle (vertex)
    \coordinate (O) at (0,0);
    
    % Define the radius of the circle
    \def\radius{1};
    
    % Draw the circle
    \draw (O) circle (\radius);
    
    % Draw the tangent lines
    \draw[dashed] (O) -- ($(O)+(\radius*cos(45),\radius*sin(45))$); % Tangent line 1
    \draw[dashed] (O) -- ($(O)+(-\radius*cos(45),-\radius*sin(45))$); % Tangent line 2
    
    % Mark the points where the tangent lines intersect the edges
    \coordinate (A) at ($(O)+(\radius*cos(45),\radius*sin(45))$);
    \coordinate (B) at ($(O)+(-\radius*cos(45),-\radius*sin(45))$);
    
    % Draw the edges of the polygon
    \draw[->] (O) -- (A) node[midway,right] {Distance $ry$};
    \draw[->] (O) -- (B) node[midway,left] {Distance $rh(y,\alpha_i)$};
    
    % Label the radius
    \node at (0,-1.2) {$r$};
    
    % Add annotations
    \node at (0.7,0.7) {$\text{Tangent Line 1}$};
    \node at (-0.7,-0.7) {$\text{Tangent Line 2}$};
    
    % Add labels for distances
    \node at (A) [above right] {$ry$};
    \node at (B) [below left] {$rh(y,\alpha_i)$};
    
    % Add alpha_i angle label
    \node at (0.2,-0.2) {\footnotesize $\alpha_i$};
\end{tik