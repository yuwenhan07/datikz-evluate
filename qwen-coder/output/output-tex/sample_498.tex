To generate a TikZ LaTeX diagram for $\mathcal{C}_3(P_3)$, the 3-coloring graph for $P_3$ (the path graph on 3 vertices), we need to represent each vertex as a coloring of the edges of $P_3$. The path graph $P_3$ has three vertices connected in a line, so it has two edges.

Each edge can be colored using one of three colors: let's call them red ($R$), green ($G$), and blue ($B$). Therefore, there are $3^2 = 9$ possible colorings of the edges of $P_3$. We will label each vertex with its corresponding coloring.

Here is the TikZ LaTeX code to create this diagram:

```latex
\documentclass{standalone}
\usepackage{tikz}

\begin{document}
\begin{tikzpicture}[scale=1.5]
    % Define nodes
    \node (v1) at (0,0) [circle, draw] {};
    \node (v2) at (1,0) [circle, draw] {};
    \node (v3) at (2,0) [circle, draw] {};

    % Draw edges
    \draw[->] (v1) -- node[midway, above] {$R$} (v2);
    \draw[->] (v2) -- node[midway, below] {$G$} (v3);

    % Label nodes with colorings
    \node at (-0.5, -0.7) [below left] {$(R,R)$};
    \node at (0.5, -0.7) [below right] {$(R,G)$};
    \node at (1.5, -0.7) [below left] {$(R,B)$};
    \node at (2.5, -0.7) [below right] {$(G,R)$};
    \node at (-0.5, -1.4) [above left] {$(G,G)$};
    \node at (0.5, -1.4) [above right] {$(G,B)$};
    \node at (1.5, -1.4) [above left] {$(B,R)$};
    \node at (2.5, -1.4) [above right] {$(B,G)$};
