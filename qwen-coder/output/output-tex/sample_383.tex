Certainly! Below is a TikZ LaTeX code to illustrate Newton's method for finding the root of the nonlinear equation \( f(x) = 0.5x^2 - 3x + 4 \). This example includes two initial guesses, \( x_0 \) and \( x_m \), to show how they lead to different roots.

```latex
\documentclass{standalone}
\usepackage{tikz}
\usetikzlibrary{calc}

\begin{document}

\begin{tikzpicture}[scale=1.5]

% Define the function
\def\f(#1){0.5*#1^2 - 3*#1 + 4}

% Draw the axes
\draw[->] (-2,0) -- (6,0) node[right] {$x$};
\draw[->] (0,-2) -- (0,8) node[above] {$y$};

% Draw the function
\draw[domain=-2:6,samples=100,blue] plot (\x,\f{\x});

% Initial guess x0
\coordinate (x0) at (2,0);
\fill (x0) circle[radius=2pt];
\node[below] at (x0) {$(x_0, f(x_0))$};

% Tangent line at x0
\pgfmathsetmacro\slopeA{0.5*2 - 3} % Derivative of f(x) at x0
\pgfmathsetmacro\interceptA{-\slopeA*2 + \f{2}} % y-intercept of tangent line
\draw[dashed,red] (x0) -- ($(x0)+(1,-\slopeA)$) node[midway,right] {Tangent Line};

% Intersection of tangent line with x-axis
\pgfmathsetmacro\x1{-(\interceptA/\slopeA)}
\coordinate (x1) at (\x1,0);
\fill (x1) circle[radius=2pt];
\node[below] at (x1) {$(x_1, f(x_1))$};

% Initial guess xm
\coordinate (xm) at (4,0);
\fill (xm) circle[radius=2pt];
\node[below] at (xm) {$(x_m, f(x_m))$};

% Tangent line at xm