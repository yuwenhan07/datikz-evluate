To create a TikZ diagram that visualizes the execution of the \(\textsc{Greedy}\xspace\) algorithm, we need to represent the rectangles, the red points (searched keys), and the blue points (keys touched by \(\textsc{Greedy}\xspace\)). Below is a sample TikZ code that demonstrates this:

```latex
\documentclass[tikz]{standalone}
\usetikzlibrary{shapes.geometric, arrows.meta}

\tikzset{
    rectangle/.style={draw=black, fill=none},
    point/.style={circle, draw=red, fill=red},
    key/.style={circle, draw=blue, fill=blue}
}

\begin{document}
\begin{tikzpicture}[scale=1.5]
    % Define the rectangles
    \node[rectangle, minimum width=2cm, minimum height=2cm] (R1) at (0,0) {};
    \node[rectangle, minimum width=2cm, minimum height=3cm] (R2) at (4,0) {};
    \node[rectangle, minimum width=2cm, minimum height=2cm] (R3) at (8,0) {};

    % Draw the points
    \node[point] (P1) at (1,1) {};
    \node[point] (P2) at (5,1.5) {};
    \node[point] (P3) at (9,1) {};

    % Draw the keys
    \node[key] (K1) at (2,2) {};
    \node[key] (K2) at (6,2.5) {};
    \node[key] (K3) at (10,2) {};

    % Connect the points to the keys
    \draw[-Stealth] (P1) -- (K1);
    \draw[-Stealth] (P2) -- (K2);
    \draw[-Stealth] (P3) -- (K3);

    % Add labels
    \node at (1,-0.5) {Red Points};
    \node at (5,-0.5) {Blue Points};
    \node at (9,-0.5) {Keys Touched};

    % Add text explanation
    \node at (-2,2) {\(\textsc{Greedy}\xspace\) Execution};
   