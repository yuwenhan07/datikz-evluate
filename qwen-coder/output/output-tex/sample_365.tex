\documentclass[border=5mm]{standalone}
\usepackage{tikz}

\begin{document}

\begin{tikzpicture}[scale=2]

% Define points
\coordinate (A) at (-3,0);
\coordinate (B) at (0,3);
\coordinate (C) at (3,0);

% Draw lines
\draw[->] (A) -- (B) node[midway, above] {$f_1$};
\draw[->] (B) -- (C) node[midway, right] {$f_2$};

% Draw fixed points
\node[circle, fill=black, inner sep=1pt] (F1) at (0,-1) {};
\node[circle, fill=black, inner sep=1pt] (F2) at (0,4) {};

% Draw repelling and attracting intervals
\draw[dashed] (F1) -- ($(F1)+(0,-2)$);
\draw[dashed] (F2) -- ($(F2)+(0,2)$);

% Draw arrows indicating directions
\draw[->] ($(F1)+(0,-2)$) -- ($(F1)+(0,-3)$);
\draw[<-] ($(F2)+(0,2)$) -- ($(F2)+(0,3)$);

% Label points
\node at (A) [left] {$(a,b)$};
\node at (B) [above] {$(c,d)$};
\node at (C) [right] {$(e,f)$};
\node at (F1) [below left] {$b_i$};
\node at (F2) [above right] {$b_{i+1}$};

\end{tikzpicture}

\end{document}