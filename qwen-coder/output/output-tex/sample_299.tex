To create a TikZ diagram that represents the sets \(\mathcal{A}, \mathcal{B}, \mathcal{C}, \mathcal{D}, \mathcal{E}, \mathcal{F}, \mathcal{G}, \mathcal{H}\) and their relationships, we can use circles to represent each set and lines or arrows to show their intersections or other relationships.

Here is an example of how you might generate such a diagram:

```latex
\documentclass[border=10pt]{standalone}
\usepackage{tikz}

\begin{document}
\begin{tikzpicture}[scale=2]
    % Define the positions for the circles
    \coordinate (A) at (-3, 2);
    \coordinate (B) at (0, 2);
    \coordinate (C) at (3, 2);
    \coordinate (D) at (-3, -2);
    \coordinate (E) at (0, -2);
    \coordinate (F) at (3, -2);
    \coordinate (G) at (-1.5, 0);
    \coordinate (H) at (1.5, 0);

    % Draw the circles
    \draw (A) circle (1cm) node[above] {\(\mathcal{A}\)};
    \draw (B) circle (1cm) node[above] {\(\mathcal{B}\)};
    \draw (C) circle (1cm) node[above] {\(\mathcal{C}\)};
    \draw (D) circle (1cm) node[below] {\(\mathcal{D}\)};
    \draw (E) circle (1cm) node[below] {\(\mathcal{E}\)};
    \draw (F) circle (1cm) node[below] {\(\mathcal{F}\)};
    \draw (G) circle (1cm) node[left] {\(\mathcal{G}\)};
    \draw (H) circle (1cm) node[right] {\(\mathcal{H}\)};

    % Draw lines to show relationships (e.g., intersections)
    \draw[dashed] (A) -- (B);
    \draw[dashed] (B) -- (C);
    \draw[dashed] (C) -- (A);
    \draw[dashed] (D) -- (E);
    \draw