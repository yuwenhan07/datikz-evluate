To create a TikZ diagram for a proper 10-coloring of \((P_6 \square P_7)_\delta\) with specific colorings for vertices in \(V_2\), \(V_3\), and \(V_4\), we need to understand the structure of the graph \((P_6 \square P_7)_\delta\). This is the square of the Cartesian product of two paths, \(P_6\) and \(P_7\).

Here's how you can represent this in TikZ:

```latex
\documentclass{standalone}
\usepackage{tikz}

\begin{document}
\begin{tikzpicture}[node distance=2cm]

% Define nodes for P6
\node (v1) at (0,0) {1};
\node (v2) at (1,0) {2};
\node (v3) at (2,0) {3};
\node (v4) at (3,0) {4};
\node (v5) at (4,0) {5};
\node (v6) at (5,0) {6};

% Define nodes for P7
\node (w1) at (0,-1) {A};
\node (w2) at (1,-1) {B};
\node (w3) at (2,-1) {C};
\node (w4) at (3,-1) {D};
\node (w5) at (4,-1) {E};
\node (w6) at (5,-1) {F};
\node (w7) at (6,-1) {G};

% Connect nodes to form P6
\draw[thick] (v1) -- (v2) -- (v3) -- (v4) -- (v5) -- (v6);

% Connect nodes to form P7
\draw[thick] (w1) -- (w2) -- (w3) -- (w4) -- (w5) -- (w6) -- (w7);

% Connect corresponding nodes in P6 and P7 to form the square of the Cartesian product
\foreach \i in {1,...,5} {
    \draw[thick] (v\i) -- (w\i);
}
\foreach \i in {2,...,6}