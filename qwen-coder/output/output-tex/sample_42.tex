Certainly! Below is a TikZ LaTeX code to create a figure that visually represents the element \( A = M_1 + M_4 - X \) as described. The code includes nodes labeled with the given values and conditions.

```latex
\documentclass[tikz,border=10pt]{standalone}
\usetikzlibrary{positioning}

\begin{document}
\begin{tikzpicture}[node distance=2cm]
    % Nodes for the elements
    \node (M1) {\( M_1 \)};
    \node (M4) [right=of M1] {\( M_4 \)};
    \node (X) [below=of M1] {\( X \)};
    
    % Node for A
    \node (A) [above right=of M4, yshift=1cm] {\( A \)};
    
    % Connections between nodes
    \draw[->] (M1) -- node[midway, above] {+} (M4);
    \draw[->] (M4) -- node[midway, below] {-} (X);
    \draw[->] (A) -- (M1);
    \draw[->] (A) -- (M4);
    \draw[->] (A) -- (X);
    
    % Labels for A at specific points
    \node at ([yshift=1cm]M1) {\( A(3) = A(6) = -x_2 \)};
    \node at ([yshift=1cm]M4) {};
    \node at ([yshift=1cm]X) {};
    
    % Additional labels for specific points
    \node at ([xshift=1cm]M1) {\( A(1) \)};
    \node at ([xshift=1cm]M4) {\( A(2) \)};
    \node at ([xshift=1cm]X) {\( A(4) \)};
    \node at ([xshift=1cm]A) {\( A(5) \)};
    
    % Conditions for specific points
    \node at ([yshift=0.5cm]M1) {\(_{\{3,6\}}(1) = _{\{3,6\}}(2) = _{\{3,6\}}(4) = _{\{3,