To create a TikZ diagram that represents both the grid diagram obtained from the permutation \(68517324\) and the Seifert circles obtained from a link diagram, we need to follow several steps:

1. **Grid Diagram**: This is typically represented as a directed graph where nodes represent positions in the permutation and edges represent transitions between these positions.
2. **Seifert Circles**: These are closed curves in the plane that enclose crossings in a link diagram.

Since you didn't provide a specific link diagram, I'll focus on creating the grid diagram for the permutation \(68517324\). Here's how you can do it using TikZ:

```latex
\documentclass{standalone}
\usepackage{tikz}

\begin{document}

\begin{tikzpicture}[scale=1.5]
    % Nodes
    \foreach \i in {1,...,8} {
        \node (n\i) at (\i-0.5,-\i+0.5) [circle, draw, fill=white] {$\i$};
    }

    % Edges
    \draw[->] (n1) -- (n2);
    \draw[->] (n2) -- (n3);
    \draw[->] (n3) -- (n4);
    \draw[->] (n4) -- (n5);
    \draw[->] (n5) -- (n6);
    \draw[->] (n6) -- (n7);
    \draw[->] (n7) -- (n8);
    \draw[->] (n8) -- (n1);

    % Permutation arrows
    \draw[->, dashed] (n1) -- node[above] {6} (n6);
    \draw[->, dashed] (n2) -- node[above] {8} (n8);
    \draw[->, dashed] (n3) -- node[above] {5} (n5);
    \draw[->, dashed] (n4) -- node[above] {1} (n1);
    \draw[->, dashed] (n5) -- node[above] {7} (n7);
    \draw[->, dashed] (n6) -- node[above] {3} (n3);
    \