Certainly! Below is a TikZ LaTeX code snippet that generates a dependency graph based on your description:

```latex
\documentclass{standalone}
\usepackage{tikz}

\tikzset{
    definition/.style={rounded rectangle, draw=black, fill=white},
    lemma/.style={rectangle, draw=black, fill=white},
    theorem/.style={rectangle, draw=blue, fill=white},
    application/.style={rectangle, draw=orange, fill=white},
    file/.style={draw=gray, dashed},
    node distance=2cm,
    every node/.style={align=center}
}

\begin{document}

\begin{tikzpicture}

% Nodes for Definitions, Abbreviations, and Intermediate Results
\node (def1) [definition] {Definition 1};
\node (def2) [definition] at (3,0) {Definition 2};
\node (def3) [definition] at (6,0) {Definition 3};

\node (abbr1) [definition] at (1,-2) {Abbreviation 1};
\node (abbr2) [definition] at (4,-2) {Abbreviation 2};

\node (intres1) [lemma] at (7,-2) {Intermediate Result 1};
\node (intres2) [lemma] at (10,-2) {Intermediate Result 2};

% Nodes for Lemmas and Theorems
\node (thm1) [theorem] at (1,-4) {Theorem 1};
\node (thm2) [theorem] at (4,-4) {Theorem 2};

% Nodes for Applications
\node (app1) [application] at (7,-4) {Application 1};
\node (app2) [application] at (10,-4) {Application 2};

% Nodes for Files/Sections
\node (file1) [file] at (2,-5) {File 1};
\node (file2) [file] at (5,-5) {File 2};
\node (file3) [file] at (8,-5) {File 3};

% Edges
\draw[-stealth] (def1) -- (def2);
\draw[-stealth] (def2) -- (def3);
\draw[-stealth] (def1)