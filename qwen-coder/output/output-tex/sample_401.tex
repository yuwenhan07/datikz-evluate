Certainly! Below is a TikZ LaTeX code to illustrate the construction of \(\tilde{\kappa}\) as described:

```latex
\documentclass[tikz,border=3mm]{standalone}
\usetikzlibrary{shapes.geometric, arrows.meta}

\tikzset{
    node distance=2cm,
    every node/.style={circle, draw, minimum size=10pt},
    edge/.style={->, thick},
    highlight/.style={draw=red, line width=1pt}
}

\begin{document}
\begin{tikzpicture}[scale=1.5]
    % Nodes representing k values
    \node (kpq) at (0,0) {$\mathtt{k}_{p,q}$};
    \node (kpm1q) at (-1,-1) {$\mathtt{k}_{p-1,q}$};
    \node (kp1q) at (1,-1) {$\mathtt{k}_{p+1,q}$};
    \node (kpq1) at (0,-2) {$\mathtt{k}_{p,q-1}$};
    \node (kpq1) at (0,-2) {$\mathtt{k}_{p,q+1}$};

    % Nodes representing v values
    \node (vpq) at (0,0) {$\mathtt{v}_{p,q}$};
    \node (vp1q) at (1,-1) {$\mathtt{v}_{p+1,q}$};
    \node (vpq1) at (0,-2) {$\mathtt{v}_{p,q-1}$};

    % Highlighted fine element
    \node[highlight] (fine_elem) at (0,0) {};

    % Edges surrounding the highlighted fine element
    \draw[edge] (fine_elem.north) -- (fine_elem.north west);
    \draw[edge] (fine_elem.north) -- (fine_elem.north east);
    \draw[edge] (fine_elem.south) -- (fine_elem.south west);
    \draw[edge] (fine_elem.south) -- (fine_elem.south east);

    % Labels for edges
    \node[left] at (fine_elem.west) {$e_1$};
    \node[right] at (fine_elem.east) {$e_2$};
    \node[below] at (fine_elem