Certainly! Below is a TikZ LaTeX code that creates a vertical cross-section through a simple city environment illustrating Line-of-Sight (LoS) encoding and elevation angles.

```latex
\documentclass{standalone}
\usepackage{tikz}
\usetikzlibrary{arrows.meta}

\tikzset{
    building/.style={fill=red},
    tx/.style={fill=green},
    locint/.style={fill=yellow},
    minheight/.style={fill=blue},
    elevangle/.style={fill=purple}
}

\begin{document}
\begin{tikzpicture}[scale=2]
    % Draw the ground plane
    \draw[thick] (-1,0) -- (4,0);
    
    % Draw the buildings
    \draw[building] (0,-1) rectangle (1,3);
    \draw[building] (2,-2) rectangle (3,2);
    \draw[building] (3,-3) rectangle (4,1);
    
    % Location of Interest (LOI)
    \node[locint] at (2,0) {};
    
    % Minimum Height for LoS
    \draw[minheight] (2,-1) rectangle (3,0);
    
    % Transmitter (Tx)
    \node[tx] at (1,2) {};
    \draw[-Stealth, thick] (1,2) -- (1,5); % Antenna Pattern
    
    % Main Direction
    \draw[orange, thick] (1,2) -- (4,0);
    
    % Elevation Angles
    \draw[elevangle] (1,2) -- (2,0); % Angle corresponding to LOI
    \draw[elevangle] (1,2) -- (2,-2); % Angle corresponding to left corner of the building
    
    % Labels
    \node[above right] at (1,5) {Antenna};
    \node[below right] at (1,-1) {Building};
    \node[right] at (2,0) {Location of Interest};
    \node[left] at (2,-1) {Minimum Height};
    \node[above] at (1,2) {Transmitter};
    \node[above right] at (4,0) {Main Direction};
    \node[above right] at (2,0) {E