To visually represent the different types of sparsity you mentioned using TikZ, we can create diagrams that illustrate how data is distributed across rows and columns. Below is a LaTeX code snippet that generates a visual representation of these sparsity types:

```latex
\documentclass[tikz,border=10pt]{standalone}
\usetikzlibrary{matrix}

\tikzset{
    matrix/.style={
        matrix of nodes,
        nodes in empty cells,
        row sep=-\pgflinewidth,
        column sep=-\pgflinewidth,
        nodes={draw, minimum size=1em, inner sep=0pt},
        align=center
    },
    sparse/.style={fill=gray!50},
    dense/.style={fill=white}
}

\begin{document}
\begin{tikzpicture}[scale=1.5]

% Random Sparsity
\node (random) [matrix] {
|[sparse]| & |[dense]| & |[sparse]| \\
|[dense]| & |[dense]| & |[sparse]| \\
|[sparse]| & |[dense]| & |[dense]| \\
};
\node at (random.north west) {Random};

% Block-Random Sparsity
\node (block_random) [matrix, xshift=4cm] {
|[sparse]| & |[dense]| & |[dense]| & |[dense]| \\
|[dense]| & |[sparse]| & |[sparse]| & |[dense]| \\
|[sparse]| & |[dense]| & |[dense]| & |[dense]| \\
};
\node at (block_random.north west) {Block-Random};

% Random Column Sparsity
\node (random_column) [matrix, xshift=8cm] {
|[sparse]| & |[dense]| & |[sparse]| \\
|[]| & |[dense]| & |[dense]| \\
|[]| & |[sparse]| & |[dense]| \\
};
\node at (random_column.north west) {Random Column};

% Random Block Column Sparsity
\node (random_block_column) [matrix, xshift=12cm] {
|[sparse]| & |[dense]| & |[dense]| & |[dense]| \\
|[]| & |[sparse]| & |[sparse]| & |[dense]| \\
|[]| & |[dense]| & |[dense]| & |[dense]| \\
};
\n