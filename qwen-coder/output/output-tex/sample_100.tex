To create a TikZ diagram that summarizes the variation of constraints as described, we can use nodes to represent different levels and arrows to show the flow of constraints and the effects of boosts. Below is a sample TikZ code that represents this information:

```latex
\documentclass[tikz]{standalone}
\usetikzlibrary{shapes.geometric, arrows.meta}

\tikzset{
    startstop/.style={rectangle, rounded corners, minimum width=3cm, minimum height=1cm,text centered, draw=black, fill=red!30},
    process/.style={rectangle, minimum width=3cm, minimum height=1cm, text centered, draw=black, fill=orange!30},
    decision/.style={diamond, minimum width=3cm, minimum height=1cm, text centered, draw=black, fill=green!30},
    arrow/.style={thick,->,>=stealth}
}

\begin{document}
\begin{tikzpicture}[node distance=2cm]

\node (start) [startstop] {Initial Constraints};
\node (process1) [process, below of=start] {Level 1: $\hat{T}_{ab}{}^A = 0$};
\node (process2) [process, below of=process1] {Level 2: $\hat{T}_a{}^{\{AB\}} = 0$};
\node (process3) [process, below of=process2] {Level 3: $\hat{f}_{abcd} = 0$};
\node (process4) [process, below of=process3] {Level 4: $\hat{f}_{Aabc} = 0$};
\node (process5) [process, below of=process4] {Level 5: Additional Constraints};
\node (process6) [process, below of=process5] {Level 6: Tower Terminates};

\node (boost1) [decision, right of=process1, xshift=3cm] {Boosts Act Vertically};
\node (boost2) [decision, right of=process2, xshift=3cm] {};
\node (boost3) [decision, right of=process3, xshift=3cm] {};
\node (boost4) [decision, right of=process4, xshift=3cm] {};
\node (boost5) [decision