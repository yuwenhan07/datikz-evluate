Certainly! Below is a TikZ LaTeX code that generates a diagram representing the partition of an enlarged annulus \(PQu\) into \(P\), \(Q_1\), \(A(PQ_1)\), \(B\), and \(C\). The diagram includes interconnected circles and arrows labeled accordingly.

```latex
\documentclass{standalone}
\usepackage{tikz}

\begin{document}

\begin{tikzpicture}[scale=2]
    % Define coordinates for the circles
    \coordinate (P) at (-3, 0);
    \coordinate (Q1) at (0, 0);
    \coordinate (APQ1) at (3, 0);
    \coordinate (B) at (0, -2);
    \coordinate (C) at (0, 2);

    % Draw the circles
    \draw[thick, fill=white] (P) circle (1cm);
    \draw[thick, fill=white] (Q1) circle (1cm);
    \draw[thick, fill=white] (APQ1) circle (1cm);
    \draw[thick, fill=white] (B) circle (1cm);
    \draw[thick, fill=white] (C) circle (1cm);

    % Label the circles
    \node at (P) [above left] {P};
    \node at (Q1) [above] {Q};
    \node at (APQ1) [above right] {A(PQ)};
    \node at (B) [below] {B};
    \node at (C) [above] {C};

    % Draw arrows between the circles
    \draw[->, thick] (P) -- node[midway, above] {$\rightarrow$} (Q1);
    \draw[->, thick] (Q1) -- node[midway, above] {$\rightarrow$} (APQ1);
    \draw[->, thick] (APQ1) -- node[midway, above] {$\rightarrow$} (B);
    \draw[->, thick] (APQ1) -- node[midway, below] {$\rightarrow$} (C);
    \draw[->, thick] (B) -- node[midway, left] {$\rightarrow$} (P);
    \draw[->, thick] (C