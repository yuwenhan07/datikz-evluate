To illustrate how the words of an equivalence class are related by swaps, we can use a directed graph in TikZ where each node represents a word and each directed edge represents a swap operation. Let's consider a simple example with the set of all permutations of the word "abc". We will show how these permutations are connected through swap operations.

Here's a TikZ code to create this illustration:

```latex
\documentclass{standalone}
\usepackage{tikz}

\begin{document}

\begin{tikzpicture}[node distance=2cm, auto]
    % Define nodes for each permutation
    \node (a) {abc};
    \node (b) at (4,0) {acb};
    \node (c) at (2,-2) {bac};
    \node (d) at (6,-2) {bca};
    \node (e) at (1,-4) {cab};
    \node (f) at (5,-4) {cba};

    % Draw edges representing swap operations
    \draw[->] (a) -- node[above] {swap b and c} (b);
    \draw[->] (a) -- node[left] {swap a and b} (c);
    \draw[->] (b) -- node[right] {swap a and c} (d);
    \draw[->] (b) -- node[below] {swap b and c} (f);
    \draw[->] (c) -- node[above] {swap a and c} (e);
    \draw[->] (c) -- node[left] {swap a and b} (d);
    \draw[->] (d) -- node[below] {swap a and c} (f);
    \draw[->] (d) -- node[right] {swap b and c} (b);
    \draw[->] (e) -- node[above] {swap a and c} (b);
    \draw[->] (e) -- node[left] {swap a and b} (f);
    \draw[->] (f) -- node[below] {swap a and c} (d);
    \draw[->] (f) -- node[right] {swap b and c} (b);

    % Add labels to nodes
    \foreach \node in {a,b,c,d,e,f} {
       