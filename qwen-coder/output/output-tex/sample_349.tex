To create a TikZ diagram for an automaton that recognizes the addition of two binary numbers represented as strings over \(\{0,1\}^3\), we need to design an automaton that processes three-bit binary inputs \( (m)_2 \), \( (n)_2 \), and \( (m+n)_2 \) and accepts them if they satisfy the condition \( (m)_2 \otimes (n)_2 \otimes (m+n)_2 \).

Here's a step-by-step guide to creating this automaton:

1. **Define the States**: We will have states for each bit position and a final accept state.
2. **Transitions**: Define transitions based on the binary digits and the addition operation.

Let's write the TikZ code for this automaton:

```latex
\documentclass[tikz,border=5]{standalone}
\usetikzlibrary{automata, positioning}

\begin{document}
\begin{tikzpicture}[node distance=2cm, auto]
    % States
    \node[state, initial] (q0) {q_0};
    \node[state] (q1) [below left of=q0] {q_1};
    \node[state] (q2) [below right of=q0] {q_2};
    \node[state, accepting] (q3) [below of=q1] {q_3};

    % Transitions
    \path[->] 
        (q0) edge node[above] {$0$} (q1)
             edge node[below] {$1$} (q2);
    \path[->] 
        (q1) edge node[above] {$0$} (q3)
             edge node[below] {$1$} (q3);
    \path[->] 
        (q2) edge node[above] {$0$} (q3)
             edge node[below] {$1$} (q3);

    % Labels for clarity
    \node at ($(q0.north)!0.5!(q1.west)$) {$(m)_2$};
    \node at ($(q0.north)!0.5!(q2.east)$) {$(n)_2$};
    \node at ($(q1.south)!0.5!(q2.west)$) {$(m+n)_2$};
\end{tik