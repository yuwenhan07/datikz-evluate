\documentclass[tikz,border=10pt]{standalone}
\usetikzlibrary{shapes.geometric, arrows.meta}

\tikzset{
  node distance=2cm,
  start chain=going below,
  rectangle/.style={draw, rectangle, rounded corners, minimum width=3cm, minimum height=1cm},
  shaded/.style={rectangle, fill=blue!30},
  arrow/.style={thick, ->, >=stealth'}
}

\begin{document}
\begin{tikzpicture}[start chain]

  % Initial Tree Nodes
  \node (root) [rectangle] {Root};
  \node (child1) [rectangle, right=of root] {Child 1};
  \node (child2) [rectangle, right=of child1] {Child 2};
  
  % Connect Nodes
  \draw[arrow] (root) -- (child1);
  \draw[arrow] (child1) -- (child2);

  % Shaded Rectangles for Arborally Satisfied Nodes
  \node (shaded1) [shaded, below=of child1, xshift=-1cm] {Shaded Node 1};
  \node (shaded2) [shaded, below=of child2, xshift=1cm] {Shaded Node 2};

  % Point z_i
  \node (zi) [circle, draw, inner sep=2pt, below=of root, yshift=-1.5cm] {$z_i$};

  % Arrows from zi to Shaded Nodes
  \draw[arrow] (zi) -- (shaded1);
  \draw[arrow] (zi) -- (shaded2);

\end{tikzpicture}
\end{document}