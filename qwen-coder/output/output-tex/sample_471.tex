To create a TikZ LaTeX diagram that represents the 6 SLOCC (Stochastic Local Operations and Classical Communication) classes in \( \mathbb{C}^2 \otimes \mathbb{C}^2 \otimes \mathbb{C}^2 \), we need to structure the diagram hierarchically, starting from simpler concepts and progressing to more complex ones. Below is a sample TikZ code that demonstrates this structure:

```latex
\documentclass[tikz,border=10pt]{standalone}
\usetikzlibrary{shapes.geometric, arrows.meta}

\tikzset{
    node distance=2cm,
    start chain/.style={start chain=going below, node distance=1cm},
    box/.style={draw, rectangle, minimum width=3cm, minimum height=1cm, align=center, on chain},
    arrow/.style={thick, ->, >=stealth'},
}

\begin{document}

\begin{tikzpicture}[node distance=2cm]

% Level 1: Vectors
\node[box] (v1) {Vectors};
\node[box, right=of v1] (v2) {Complex Numbers};

% Level 2: Matrices
\node[box, below=of v1] (m1) {Matrices};
\node[box, below=of v2] (m2) {Hermitian Matrices};
\node[box, right=of m1] (m3) {Pauli Matrices};
\node[box, right=of m2] (m4) {Density Matrices};

% Level 3: Quantum States
\node[box, below=of m1] (qs1) {Quantum States};
\node[box, below=of m2] (qs2) {Pure States};
\node[box, below=of m3] (qs3) {Basis States};
\node[box, below=of m4] (qs4) {Mixed States};

% Level 4: Entanglement
\node[box, below=of qs1] (e1) {Entanglement};
\node[box, below=of qs2] (e2) {Bell States};
\node[box, below=of qs3] (e3) {GHZ State};
\node[box, below=of qs4] (e4