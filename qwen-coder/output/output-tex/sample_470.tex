Certainly! Below is a TikZ LaTeX code snippet that illustrates the possible structures for a critical component \( K \) of \( H + C \) as described:

```latex
\documentclass[tikz,border=10pt]{standalone}
\usetikzlibrary{shapes.geometric, arrows.meta}

\tikzset{
    vertex/.style={circle, draw=black, fill=white, inner sep=3pt},
    anchor/.style={rectangle, draw=gray, fill=gray!50, inner sep=3pt},
    critical/.style={rectangle, draw=red, fill=red!50, inner sep=3pt},
    edge/.style={draw=black, thick},
    dashed/.style={draw=black, dashed},
    thin/.style={draw=black, thin},
    filled/.style={fill=black},
    blank/.style={}
}

\begin{document}
\begin{tikzpicture}[scale=1.5]

% Structure 1: Matching M
\node[vertex] (v1) at (0,0) {};
\node[vertex] (v2) at (2,0) {};
\node[vertex] (v3) at (4,0) {};
\node[vertex] (v4) at (6,0) {};

\draw[edge] (v1) -- (v2);
\draw[dashed] (v3) -- (v4);

% Structure 2: Path-Cycle Cover C
\node[vertex] (w1) at (-2,-2) {};
\node[vertex] (w2) at (0,-2) {};
\node[vertex] (w3) at (2,-2) {};
\node[vertex] (w4) at (4,-2) {};
\node[vertex] (w5) at (2,-4) {};
\node[vertex] (w6) at (4,-4) {};

\draw[edge] (w1) -- (w2);
\draw[edge] (w2) -- (w3);
\draw[edge] (w3) -- (w4);
\draw[dashed] (w4) -- (w5);
\draw[dashed] (w5) -- (w6);
\draw[dashed] (w6) -- (w1);

% Structure 3: Critical Component K
\node[critical] (k1) at (8,0) {};
\node[critical