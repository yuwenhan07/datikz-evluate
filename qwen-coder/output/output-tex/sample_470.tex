To create a TikZ diagram representing the described geometry of the wormhole, we need to carefully construct the hyperbolic quadrilaterals and identify their edges. Below is a LaTeX code snippet that generates this diagram:

```latex
\documentclass{standalone}
\usepackage{tikz}
\usetikzlibrary{calc}

\begin{document}
\begin{tikzpicture}[scale=3]

% Define points for the first hyperbolic quadrilateral
\coordinate (R') at (0, 0);
\coordinate (R) at (1, 0);
\coordinate (P'Q') at (-1, -1);
\coordinate (P''Q'') at (1, -1);

% Draw the first hyperbolic quadrilateral
\draw[thick] (R') -- (P'Q') -- (P''Q'') -- (R) -- cycle;

% Define points for the second hyperbolic quadrilateral
\coordinate (S') at (0, 0);
\coordinate (S) at (1, 0);
\coordinate (Q'S') at (-1, 1);
\coordinate (Q''S'') at (1, 1);

% Draw the second hyperbolic quadrilateral
\draw[thick] (S') -- (Q'S') -- (Q''S'') -- (S) -- cycle;

% Identify edges P'R' and P''R'' with R' and R''
\draw[dashed, thick] (P'Q') -- (P''Q'');
\draw[dashed, thick] (Q'S') -- (Q''S'');

% Labeling
\node at (0.5, 0.5) {$R$};
\node at (0.5, -0.5) {$S$};

% Angles at R' and R''
\node at (-1.5, -1.5) [anchor=east] {\footnotesize $\alpha$};
\node at (1.5, -1.5) [anchor=west] {\footnotesize $\alpha$};

% Angles at S' and S''
\node at (-1.5, 1.5) [anchor=east] {\footnotesize $\beta$};
\node at (1.5, 1.5) [anchor=west] {\footnotesize $\beta$};

\end{tikz