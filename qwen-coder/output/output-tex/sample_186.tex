Certainly! Below is a TikZ LaTeX code that creates an illustration of the EIT (Electrical Impedance Tomography) measurement tank, including the electrodes and different injection patterns as described.

```latex
\documentclass[tikz]{standalone}
\usepackage{amsmath}

\tikzset{
    electrode/.style={circle, fill=black, inner sep=0pt, minimum size=3pt},
    injection/.style={draw, thick},
    adjacent/.style={injection, draw=black},
    all_against/.style={injection, dashed, draw=#1},
    dashed_in_2nd/.style={injection, dashed, draw=#1, dash pattern=on 4pt off 2pt},
    dotted_in_4th/.style={injection, dashed, draw=#1, dash pattern=on 2pt off 2pt},
    dash_dotted_in_6th/.style={injection, dashed, draw=#1, dash pattern=on 2pt off 2pt on 2pt off 2pt}
}

\begin{document}
\begin{tikzpicture}[scale=2]

% Draw the tank
\node[draw, rectangle, minimum width=8cm, minimum height=4cm] (tank) at (0,0) {EIT Measurement Tank};

% Draw electrodes
\node[electrode] (el1) at (-3,1) {};
\node[electrode] (el2) at (-2,1) {};
\node[electrode] (el3) at (-1,1) {};
\node[electrode] (el4) at (0,1) {};
\node[electrode] (el5) at (1,1) {};
\node[electrode] (el6) at (2,1) {};
\node[electrode] (el7) at (3,1) {};

\node[electrode] (el8) at (-3,-1) {};
\node[electrode] (el9) at (-2,-1) {};
\node[electrode] (el10) at (-1,-1) {};
\node[electrode] (el11) at (0,-1) {};
\node[electrode] (el12) at (1,-1) {};
\node[electrode]