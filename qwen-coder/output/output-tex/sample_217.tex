Certainly! Below is a TikZ LaTeX code snippet that generates the described figure. This code includes the parameter space configuration, the projections of the orthogonal couplings, the minimal-distance vector, and an orthogonal coordinate system.

```latex
\documentclass{standalone}
\usepackage{tikz}
\usetikzlibrary{arrows.meta, decorations.pathreplacing}

\begin{document}
\begin{tikzpicture}[scale=1.5]
    % Define colors
    \colorlet{axion_coupling}{cyan}
    \colorlet{orthogonal_coupling}{fadedcyan}
    \colorlet{potential_coupling}{teal}
    \colorlet{minimal_distance_vector}{purple}
    \colorlet{convex_hull}{orange}

    % Draw axes
    \draw[->, thick] (-2,0) -- (2,0) node[right] {$\lambda_r$};
    \draw[->, thick] (0,-2) -- (0,2) node[above] {$\lambda^\perp_r$};

    % Draw minimal-distance vector
    \draw[-Stealth, thick, minimal_distance_vector] (0,0) -- (1.5,1.5) node[midway, above right] {$\gamma_\infty$};

    % Draw convex hull
    \fill[convex_hull, opacity=0.3] (0,0) -- (1,1) -- (1,-1) -- cycle;

    % Draw potential couplings
    \foreach \i in {1,...,4} {
        \draw[dashed, potential_coupling] (0,0) -- (\i/4,\i/4);
    }

    % Draw orthogonal couplings
    \foreach \i in {1,...,4} {
        \draw[orthogonal_coupling] (0,0) -- (0,-\i/4);
    }

    % Draw axion couplings
    \foreach \i in {1,...,4} {
        \draw[axion_coupling] (0,0) -- (\i/4,-\i/4);
    }

    % Add labels
    \node at (1.75,1.75) [below right] {\small $\gamma_1$};
    \node at (1.75,-1.75) [above right] {\small $\gamma