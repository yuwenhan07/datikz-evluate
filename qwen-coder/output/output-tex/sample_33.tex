Certainly! Below is a TikZ LaTeX code that visualizes the Weitzman indices, minimum indices, and the capped values as described:

```latex
\documentclass{standalone}
\usepackage{tikz}

\begin{document}

\begin{tikzpicture}[scale=0.8]

% Define colors
\colorlet{yellow}{yellow!50}
\colorlet{blue}{blue!50}
\colorlet{red}{red!50}

% Draw the main frame
\draw[thick] (0,0) rectangle (4,2);
\node at (2,1) {Pandora Basket};

% Define nodes for indices
\node at (1,1.5) {$\sigma_i^{(1)}$};
\node at (2,1.5) {$\sigma_i^{(2)}$};
\node at (3,1.5) {$\sigma_i^{(3)}$};
\node at (1,1) {$\gamma_i^{(1)}$};
\node at (2,1) {$\gamma_i^{(2)}$};
\node at (3,1) {$\gamma_i^{(3)}$};
\node at (1,0.5) {$\kappa_i^{(1)}$};
\node at (2,0.5) {$\kappa_i^{(2)}$};
\node at (3,0.5) {$\kappa_i^{(3)}$};

% Highlight yellow regions
\fill[yellow] (1,1.5) rectangle (2,0.5);
\fill[yellow] (2,1.5) rectangle (3,0.5);

% Draw arrows indicating changes
\draw[->, thick] (1,1.5) -- node[midway, above] {$\sigma_i^{(2)} > \gamma_i^{(1)}$} (2,1.5);
\draw[->, thick] (2,1.5) -- node[midway, above] {$\sigma_i^{(3)} > \gamma_i^{(2)}$} (3,1.5);

% Draw lines for minimum values
\draw[dashed, blue] (1,1) -- (1,0.5);
\draw[dashed, blue] (2,1) -- (2,0.5);
