To create a minimal poset with the given multiset of maximal chain cardinalities using TikZ in LaTeX, we can construct a series of chains where each chain has a length corresponding to one of the numbers in the set \{3, 4, 5, 6, 7, 8, 9, 10\}. We will then combine these chains into an ordinal sum, ensuring that each color represents a different poset.

Here's the TikZ code to generate this poset:

```latex
\documentclass[tikz,border=1cm]{standalone}
\usetikzlibrary{positioning}

\tikzset{
    node style/.style={circle, draw, fill=blue!20, minimum size=10pt},
    label style/.style={font=\footnotesize}
}

\begin{document}
\begin{tikzpicture}[node distance=1cm]

% First poset (length 3)
\node[node style] (a) at (0,0) [label=label style:1];
\node[node style] (b) at (1,0) [label=label style:2];
\node[node style] (c) at (2,0) [label=label style:3];

% Second poset (length 4)
\node[node style] (d) at (3,-1) [label=label style:4];
\node[node style] (e) at (4,-1) [label=label style:5];
\node[node style] (f) at (5,-1) [label=label style:6];
\node[node style] (g) at (6,-1) [label=label style:7];

% Third poset (length 5)
\node[node style] (h) at (7,-2) [label=label style:8];
\node[node style] (i) at (8,-2) [label=label style:9];
\node[node style] (j) at (9,-2) [label=label style:10];
\node[node style] (k) at (10,-2);
\node[node style] (l) at (11,-2);

% Fourth poset (length 6)
\node[node style] (m) at (12,-3) [label=label style:11];
\node[node style] (n) at (13,-3) [label=label style: