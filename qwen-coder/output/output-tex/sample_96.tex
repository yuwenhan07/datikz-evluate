Certainly! Below is a TikZ LaTeX code that illustrates a call to the `SkewTranspose` procedure, showing how pointers `col[i]` and `buc[j]` advance during the process.

```latex
\documentclass[tikz,border=10pt]{standalone}
\usetikzlibrary{shapes.geometric, arrows.meta}

\tikzset{
    block/.style={rectangle, draw=black, fill=white, text width=5em, text centered, rounded corners, minimum height=3em},
    line/.style={draw=black, thick, ->, >=stealth'}
}

\begin{document}

\begin{tikzpicture}[node distance=2cm]

\node (start) [block] {Start};
\node (initialize) [block, below of=start] {Initialize};
\node (whileLoop) [block, below of=initialize] {While $i < n$};
\node (assignCol) [block, below of=whileLoop, yshift=-1cm] {$col[i] \gets \text{some value}$};
\node (incrementI) [block, right of=assignCol, xshift=4cm] {$i \gets i + 1$};
\node (advanceBuc) [block, below of=assignCol, yshift=-2cm] {$buc[j] \gets \text{some value}$};
\node (incrementJ) [block, right of=advanceBuc, xshift=4cm] {$j \gets j + 1$};
\node (endWhile) [block, below of=incrementI, yshift=-1cm] {End While};
\node (end) [block, below of=endWhile] {End};

% Drawing lines
\draw [line] (start) -- (initialize);
\draw [line] (initialize) -- (whileLoop);
\draw [line] (whileLoop) -- node[anchor=east] {True} (assignCol);
\draw [line] (assignCol.east) |- node[anchor=north] {} (incrementI.west);
\draw [line] (incrementI) |- node[anchor=north] {} (advanceBuc.west);
\draw [line] (advanceBuc.east) |- node[anchor=north] {} (incrementJ.west);
\draw [line] (incrementJ) |- node[anchor=n