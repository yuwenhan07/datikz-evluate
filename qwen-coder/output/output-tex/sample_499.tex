To create a TikZ diagram comparing the performance of a data stream allocation algorithm with other methods in a Peer-to-Peer Content Caching (PCNC) scenario, we need to define the parameters and plot the results accordingly. Here's an example of how you can achieve this using TikZ:

```latex
\documentclass[tikz,border=5mm]{standalone}
\usepackage{tikzscale}
\usetikzlibrary{arrows.meta, decorations.markings}

% Define colors for different algorithms
\definecolor{algorithmA}{RGB}{0, 102, 204} % Blue
\definecolor{algorithmB}{RGB}{255, 165, 0}   % Orange
\definecolor{algorithmC}{RGB}{153, 0, 153}   % Purple

% Define node styles
\tikzset{
    box/.style={draw, rectangle, minimum width=2cm, minimum height=1cm},
    legend/.style={draw=none, rectangle, fill=white, inner sep=2pt}
}

\begin{document}
\begin{tikzpicture}[x=1cm, y=1cm]

% Plot axis
\draw[->] (0,0) -- (8,0) node[right] {$Time$};
\draw[->] (0,0) -- (0,5) node[above] {$Performance$};

% Data points for Algorithm A
\draw[algorithmA, thick, -Stealth] (1,1) -- (2,2) -- (3,3) -- (4,4) -- (5,4.5) -- (6,5);
\node at (5.5,4.75) [legend] {Algorithm A};

% Data points for Algorithm B
\draw[algorithmB, thick, -Stealth] (1,2) -- (2,2.5) -- (3,3) -- (4,3.5) -- (5,4) -- (6,4.5);
\node at (5.5,3.75) [legend] {Algorithm B};

% Data points for Algorithm C
\draw[algorithmC, thick, -Stealth] (1,1.5) -- (2,2) -- (3,2.5) -- (4,3) -- (