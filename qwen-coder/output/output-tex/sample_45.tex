To achieve the described operation in TikZ, we need to follow these steps:

1. Define the surface \( F_\pi \) and the line \( L \).
2. Find the intersection points \( A \) and \( B \) of \( L \) with the boundary of \( F_\pi \).
3. Subdivide the triangulation such that \( A \) and \( B \) become vertices.
4. Remove all edges and vertices that intersect \( L \), except for \( A \) and \( B \).
5. Triangulate the remaining polygon with edges emanating only from \( A \) and \( B \).

Here's an example TikZ code to illustrate this process. For simplicity, let's assume \( F_\pi \) is a simple polygon and \( L \) is a straight line.

```latex
\documentclass{standalone}
\usepackage{tikz}

\begin{document}
\begin{tikzpicture}[scale=0.8]
    % Define the polygon F_pi
    \coordinate (P1) at (0, 0);
    \coordinate (P2) at (2, 0);
    \coordinate (P3) at (3, 2);
    \coordinate (P4) at (1, 2);

    % Draw the polygon
    \draw (P1) -- (P2) -- (P3) -- (P4) -- cycle;

    % Define the line L
    \draw[dashed] (0, -1) -- (3, 3);

    % Find the intersection points A and B
    \path[name path=A] (P1) -- (P2);
    \path[name path=B] (P2) -- (P3);
    \path[name path=C] (P3) -- (P4);
    \path[name path=D] (P4) -- (P1);
    \path[name intersections={of=A and L, by=A}];
    \path[name intersections={of=B and L, by=B}];

    % Mark the intersection points
    \fill[red] (A) circle (2pt);
    \fill[blue] (B) circle (2pt);

    % Remove edges and vertices that intersect L, except A and B
    % This step requires manual inspection and editing of the paths

    % Triangulate the remaining polygon with edges emanating only from A and B
