Certainly! Below is a TikZ LaTeX code snippet that illustrates the schematic you described:

```latex
\documentclass{standalone}
\usepackage{tikz}

\begin{document}

\begin{tikzpicture}[scale=0.8]
    % Define colors
    \definecolor{face}{RGB}{0, 0, 255} % Blue
    \definecolor{edge}{RGB}{0, 255, 0} % Green
    \definecolor{vertex}{RGB}{255, 0, 0} % Red

    % Draw block 1
    \draw[fill=gray!30] (-1, -1) rectangle (1, 1);
    \node at (0, 0) {Block 1};

    % Highlight face, edge, and vertex halo cells in block 1
    \draw[fill=face] (1, -1) rectangle (1.5, -0.5); % East face
    \draw[fill=edge] (1, -0.5) rectangle (1.5, 0.5); % East edge
    \draw[fill=vertex] (1, 0.5) rectangle (1.5, 1); % East vertex

    % Draw block 2
    \draw[fill=gray!30] (2, -1) rectangle (4, 1);
    \node at (3, 0) {Block 2};

    % Highlight corresponding cells in block 2
    \draw[fill=face] (2, -1) rectangle (2.5, -0.5); % West face
    \draw[fill=edge] (2, -0.5) rectangle (2.5, 0.5); % West edge
    \draw[fill=vertex] (2, 0.5) rectangle (2.5, 1); % West vertex

    % Arrows showing the west permutation
    \draw[-stealth, thick] (1.5, -1) -- node[midway, below] {West Permutation} (2.5, -1);
    \draw[-stealth, thick] (1.5, 0) -- node[midway, above] {West Permutation} (2.5, 0);

    % Labels
    \node at (0