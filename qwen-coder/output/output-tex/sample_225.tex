To create a TikZ LaTeX diagram that accurately represents the description provided, we need to visualize the fundamental domain \( F_i \), its fixed points, the stabilized value \( T_{i1.23i} \), the reduced fundamental domain, and the solid green contour.

Here's a step-by-step breakdown of how to achieve this:

1. **Define the complex plane**: We will use the real part on the x-axis and the imaginary part on the y-axis.
2. **Mark the fixed points**: These are typically points where the transformation has no effect.
3. **Indicate the stabilized value**: This is a specific point within the fundamental domain.
4. **Show the reduced fundamental domain**: This involves identifying certain parts of the domain.
5. **Draw the solid green contour**: This includes the boundary of the group action and a vertical line.

Below is the TikZ code to generate the described diagram:

```latex
\documentclass[tikz,border=3mm]{standalone}
\usetikzlibrary{shapes.geometric,backgrounds}

\begin{document}
\begin{tikzpicture}[scale=1.5]
    % Define the axes
    \draw[->] (-2,0) -- (2,0) node[right] {$\text{Re}(T_i)$};
    \draw[->] (0,-2) -- (0,2) node[above] {$\text{Im}(T_i)$};

    % Draw the boundary of the fundamental domain
    \draw[dashed,thick] plot[domain=-1.5:1.5,samples=100] (\x,{sqrt(1-\x*\x)});
    \draw[dashed,thick] plot[domain=-1.5:1.5,samples=100] (\x,{-sqrt(1-\x*\x)});

    % Mark the fixed points
    \fill[red] (-1,0) circle (2pt) node[left] {$F_1$};
    \fill[red] (1,0) circle (2pt) node[right] {$F_2$};

    % Indicate the stabilized value
    \fill[blue] (0,1.23) circle (2pt) node[above left] {$T_{i1.23i}$};

    % Show the reduced fundamental domain
    \shade[green!20] plot[domain=-