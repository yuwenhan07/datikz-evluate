Certainly! Below is an example of how you might create a TikZ diagram to visualize the speedup of PolyTOPS compared to Pluto for the Jacobi-1D problem, using two different configurations and multiple data set sizes.

```latex
\documentclass{standalone}
\usepackage{tikz}

\begin{document}

\begin{tikzpicture}[x=1cm,y=0.5cm]
    % Define styles
    \tikzset{
        bar/.style={rectangle, draw=black, fill=blue},
        bar2/.style={rectangle, draw=black, fill=red},
        axis line style={draw=black, thick},
        tick label style={font=\small},
        xlabel style={at={(current axis.right)}, anchor=north west},
        ylabel style={at={(current axis.above origin)}, rotate=-90, anchor=south east}
    }

    % Axis setup
    \begin{scope}[local bounding box=axis]
        \draw[axis line style] (-0.5,0) -- (6,0);
        \draw[axis line style] (0,-1) -- (0,4);
        \foreach \x in {1,...,5} {
            \draw[thick] (\x,0) -- ++(0,-0.1);
            \node at (\x,-0.3) {\x};
        }
        \foreach \y in {1,...,3} {
            \draw[thick] (0,\y) -- ++(-0.1,0);
            \node at (-0.3,\y) {\y};
        }
        \node[xlabel] at (6,0) {Data Set Size};
        \node[ylabel] at (0,4) {Speedup};
    \end{scope}

    % Data points
    \node[bar, minimum width=1.5cm, minimum height=2.5cm] (polytops-large) at (1,2.5) {};
    \node[bar2, minimum width=1.5cm, minimum height=1.8cm] (pluto-large) at (1,1.8) {};

    \node[bar, minimum width=1.5cm, minimum height=3.0cm] (polytops-medium) at (3,3.0) {};
    \node[bar2, minimum width=1.