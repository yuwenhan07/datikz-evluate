Certainly! Below is a TikZ LaTeX code to generate a diagram illustrating the mass spectrum of dark pions and dark baryons, with two panels showing the situation at \( m = 0 \) and for nonzero dark quark mass.

```latex
\documentclass{standalone}
\usepackage{tikz}

\begin{document}

\begin{figure}[h]
    \centering
    \begin{tikzpicture}[scale=1.5]

        % Left Panel: Mass Spectrum at m = 0
        \begin{scope}[xshift=-4cm]
            \draw[->] (0,0) -- (0,3) node[above] {$M$};
            \draw[->] (-2,-0.5) -- (2,-0.5);
            \node at (-2,-0.75) {$-\Delta/2$};
            \node at (2,-0.75) {$\Delta/2$};

            \filldraw[black] (0,2) circle[radius=2pt];
            \filldraw[blue] (0,1) circle[radius=2pt];

            \node at (0,2.2) {Dark Pion (\(\pi\))};
            \node at (0,1.2) {Dark Baryon (\(\chi\))};
        \end{scope}

        % Right Panel: Mass Spectrum with Nonzero Dark Quark Mass
        \begin{scope}[xshift=4cm]
            \draw[->] (0,0) -- (0,3) node[above] {$M$};
            \draw[->] (-2,-0.5) -- (2,-0.5);
            \node at (-2,-0.75) {$-\Delta/2$};
            \node at (2,-0.75) {$\Delta/2$};

            \filldraw[black] (0,2) circle[radius=2pt];
            \filldraw[blue] (0,1) circle[radius=2pt];
            \filldraw[red] (0,1.5) circle[radius=2pt];

            \node at (0,2.2) {Dark Pion (\(\pi\))};
            \node at (0,1.7) {Dark Baryon (\(\chi\))};
            \node at