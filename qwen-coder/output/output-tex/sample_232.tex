Certainly! Below is a TikZ LaTeX code snippet that represents the described scenario. This code includes a simple representation of a cluster graph and the two types of regularizations applied.

```latex
\documentclass[tikz,border=10pt]{standalone}
\usetikzlibrary{shapes.geometric, arrows.meta, positioning}

\tikzstyle{nodeStyle} = [draw, circle, minimum size=10pt]
\tikzstyle{clusterStyle} = [draw, rectangle, rounded corners, minimum width=2cm, minimum height=1cm]

\begin{document}

\begin{tikzpicture}[auto, node distance=2cm]

    % Nodes
    \node[nodeStyle] (x1) at (0,0) {$x_1$};
    \node[nodeStyle] (x2) at (-2,-2) {$x_2$};
    \node[nodeStyle] (x3) at (2,-2) {$x_3$};
    \node[nodeStyle] (x4) at (0,-4) {$x_4$};
    \node[nodeStyle] (x5) at (-2,-6) {$x_5$};
    \node[nodeStyle] (x6) at (2,-6) {$x_6$};
    \node[nodeStyle] (x7) at (0,-8) {$x_7$};
    \node[nodeStyle] (x8) at (-2,-10) {$x_8$};
    \node[nodeStyle] (x9) at (2,-10) {$x_9$};
    \node[nodeStyle] (x10) at (0,-12) {$x_{10}$};

    % Edges
    \draw[-Stealth] (x1) -- (x2);
    \draw[-Stealth] (x1) -- (x3);
    \draw[-Stealth] (x2) -- (x4);
    \draw[-Stealth] (x3) -- (x4);
    \draw[-Stealth] (x4) -- (x5);
    \draw[-Stealth] (x4) -- (x6);
    \draw[-Stealth] (x5) -- (x7);
    \draw[-Stealth] (x6) -- (x7);
    \draw[-Stealth] (x7