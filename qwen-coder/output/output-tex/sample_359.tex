Certainly! Below is a TikZ LaTeX code that generates a graph with two cycles \( C_u \) and \( C_v \), and it highlights the forbidden pairs \((1,1)\) and \((1,2)\) corresponding to the edge \( uv \).

```latex
\documentclass{standalone}
\usepackage{tikz}

\begin{document}

\begin{tikzpicture}[scale=1.5]
    % Nodes for cycle C_u
    \node (u1) at (0,0) [circle, draw] {};
    \node (u2) at (1,1) [circle, draw] {};
    \node (u3) at (-1,1) [circle, draw] {};

    % Nodes for cycle C_v
    \node (v1) at (2,0) [circle, draw] {};
    \node (v2) at (3,1) [circle, draw] {};
    \node (v3) at (1,-1) [circle, draw] {};
    \node (v4) at (3,-1) [circle, draw] {};

    % Edges for cycle C_u
    \draw (u1) -- (u2);
    \draw (u2) -- (u3);
    \draw (u3) -- (u1);

    % Edges for cycle C_v
    \draw (v1) -- (v2);
    \draw (v2) -- (v3);
    \draw (v3) -- (v4);
    \draw (v4) -- (v1);

    % Edge connecting u1 and v1
    \draw[dashed] (u1) -- (v1);

    % Highlighting forbidden pairs (1,1) and (1,2)
    \fill[red] (u1) circle (2pt);
    \fill[blue] (v1) circle (2pt);
    \fill[green] (u2) circle (2pt);
    \fill[orange] (v2) circle (2pt);

    % Labels for nodes
    \node at (u1.south) {$u_1$};
    \node at (u2.north east) {$u_2$};
    \node at (u3.north west) {$u_3$};

    \node at (v1.south)