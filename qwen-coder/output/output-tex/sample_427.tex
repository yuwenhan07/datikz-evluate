To visualize the given orthonormal bases and their dual vectors using TikZ, we can create a simple diagram that shows the vectors and their relationships. Below is the TikZ code to generate this diagram:

```latex
\documentclass{standalone}
\usepackage{tikz}

\begin{document}

\begin{tikzpicture}[scale=2]
    % Define coordinates for vectors
    \coordinate (Ms) at (0,1);
    \coordinate (bs) at (1,-1);
    \coordinate (Mu) at (0.577,0.577);
    \coordinate (bu) at (0.707,-0.707);

    % Draw vectors Ms and bs
    \draw[-stealth] (0,0) -- node[midway, below right] {$\widehat{M}_s$} (Ms);
    \draw[-stealth] (0,0) -- node[midway, above left] {$\widehat{b}_s$} (bs);

    % Draw vectors Mu and bu
    \draw[-stealth] (0,0) -- node[midway, above right] {$\widehat{M}_u$} (Mu);
    \draw[-stealth] (0,0) -- node[midway, below right] {$\widehat{b}_u$} (bu);

    % Draw labels for vectors
    \node at (Ms) [above right] {$(0,1)$};
    \node at (bs) [below left] {$(1,-1)$};
    \node at (Mu) [above right] {$\left(\frac{1}{\sqrt{6}}, \frac{1}{\sqrt{6}}\right)$};
    \node at (bu) [below right] {$\left(\frac{1}{\sqrt{2}}, -\frac{1}{\sqrt{2}}\right)$};

    % Draw unit circle
    \draw[dashed] (0,0) circle (1);

    % Draw lines showing the projections
    \draw[dotted] (0,0) -- (Ms |- Mu);
    \draw[dotted] (0,0) -- (bs |- Mu);
    \draw[dotted] (0,0) -- (Ms |- bu);
    \draw[dotted] (0,0) -- (bs |- bu);

