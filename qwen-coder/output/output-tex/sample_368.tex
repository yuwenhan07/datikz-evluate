To visualize the Singular Value Decomposition (SVD) of a matrix \( \mathbf{J} \) in the context provided, we will create a TikZ diagram that represents the decomposition \( \mathbf{J} = \mathbf{U}_J \mathbf{\Sigma}_J \mathbf{V}_J^\top \) for \( r_J = 2 \), \( n = 3 \), and \( p = 4 \).

Here's how you can generate the TikZ code:

```latex
\documentclass{standalone}
\usepackage{tikz}
\usetikzlibrary{matrix, arrows.meta}

\begin{document}

\begin{tikzpicture}[scale=0.8]

% Define styles for matrices
\tikzset{
    matrix/.style={
        matrix of nodes,
        nodes in empty cells,
        row sep=-0.5em,
        column sep=-0.5em,
        text height=1.5ex,
        text depth=0.25ex,
        inner xsep=0pt,
        outer xsep=0pt,
        outer ysep=0pt
    },
    node/.style={draw, rectangle, minimum size=1cm, align=center},
    arrow/.style={->, thick, >=stealth'}
}

% Create matrices
\node[matrix] (UJ) {
    U_{J11} & U_{J12} \\
    U_{J21} & U_{J22} \\
    U_{J31} & U_{J32} \\
};
\node[matrix, right=of UJ.east, anchor=west] (SJ) {
    \sigma_1 & 0 \\
    0 & \sigma_2 \\
};
\node[matrix, below=of SJ.south, anchor=north] (VJT) {
    V_{J11} & V_{J12} & V_{J13} & V_{J14} \\
    V_{J21} & V_{J22} & V_{J23} & V_{J24} \\
};

% Draw arrows
\draw[arrow] ([yshift=0.75em]UJ.south east) -- ([yshift=0.75em]SJ.north west);
\draw[arrow] ([xshift=0.75em