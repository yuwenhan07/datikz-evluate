To create a TikZ LaTeX diagram representing a \( k \)-strong semicomplete multipartite digraph \( D' \) with partite sets \( V_1, V_2, \ldots, V_c \) where \( |V_i| = k+1 \) for \( i \in [c-1] \) and \( |V_c| = k \), all arcs between different \( V_i \)'s are directed from left to right, we can use the following code:

```latex
\documentclass{standalone}
\usepackage{tikz}

\begin{document}

\begin{tikzpicture}[node distance=2cm]

% Define the number of parts (c) and the size of each part (k)
\def\c{4} % Number of parts
\def\k{3} % Size of each part except the last one

% Draw nodes
\node[circle, draw, fill=blue!20] (v11) at (0,0) {};
\node[circle, draw, fill=blue!20] (v12) at (-1,-1) {};
\node[circle, draw, fill=blue!20] (v13) at (1,-1) {};

\node[circle, draw, fill=red!20] (v21) at (0,-2) {};
\node[circle, draw, fill=red!20] (v22) at (-1,-3) {};
\node[circle, draw, fill=red!20] (v23) at (1,-3) {};

\node[circle, draw, fill=green!20] (v31) at (0,-4) {};
\node[circle, draw, fill=green!20] (v32) at (-1,-5) {};

\node[circle, draw, fill=yellow!20] (v41) at (0,-6) {};

% Draw edges
\foreach \i in {1,...,\c}{
    \foreach \j in {1,...,\k}{
        \foreach \l in {1,...,\k}{
            \ifnum\i>\j\relax
                \draw[->] (\i\j) -- (\i\l);
            \fi
        }
    }
}

\end{tikzpicture}

