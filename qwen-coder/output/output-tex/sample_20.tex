Certainly! Below is a TikZ LaTeX code that visualizes the scenario you've described:

```latex
\documentclass[tikz]{standalone}
\usetikzlibrary{shapes.geometric, arrows.meta}

\begin{document}
\begin{tikzpicture}[scale=2]
    % Draw the y-axis
    \draw[->] (0,-2) -- (0,2) node[above right] {$y$};
    
    % Draw the x-axis
    \draw[->] (-4,0) -- (4,0) node[below right] {$x$};
    
    % Shaded red area
    \fill[red!30] (-2,-2) rectangle (2,2);
    
    % Blue rectangles
    \draw[blue, thick] (-1.5, 1.5) rectangle (-0.5, -1.5); % First layer
    \draw[blue, thick] (0.5, 1.5) rectangle (1.5, -1.5);   % Second layer
    
    % Activation regions
    \draw[dashed, blue] (-1.5, 1.5) -- (1.5, 1.5); % Horizontal line for first layer
    \draw[dashed, blue] (-1.5, -1.5) -- (1.5, -1.5); % Horizontal line for second layer
    
    % Vertices of cones
    \draw[green, fill=green] (-1, 0) circle (0.05); % Vertex of cone for first layer
    \draw[green, fill=green] (1, 0) circle (0.05);  % Vertex of cone for second layer
    
    % Labels
    \node at (-1.75, 1.5) [left] {Blue Site};
    \node at (1.75, 1.5) [right] {Blue Site};
    \node at (-1.75, -1.5) [left] {Blue Line};
    \node at (1.75, -1.5) [right] {Blue Line};
    \node at (-1, 0) [below left] {Vertex of Cone};
    \node at (1, 0) [below right] {Vertex of Cone};
    
    % Distance indicators
