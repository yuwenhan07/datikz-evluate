Certainly! Below is an example of how you can create two side-by-side diagrams using TikZ in LaTeX to represent the change of observables over time and the convergence of the $\ell_2$ error of $u$ at $t = 1$ computed by different methods.

```latex
\documentclass[tikz,border=5mm]{standalone}
\usetikzlibrary{arrows.meta, decorations.pathmorphing, backgrounds, fit, petri}

\begin{document}

\begin{figure}[h]
    \centering
    \begin{minipage}{0.48\textwidth}
        \centering
        \begin{tikzpicture}[scale=0.7]
            \draw[->] (0,0) -- (6,0) node[right] {Time};
            \draw[->] (0,-1) -- (0,3) node[above] {Observable Value};
            \foreach \x in {0,1,...,6} {
                \draw (\x,-0.1) -- (\x,0.1);
                \node at (\x,-0.3) {\x};
            }
            \foreach \y in {-1,-0.5,0,0.5,1} {
                \draw (-0.1,\y) -- (0.1,\y);
                \node at (-0.3,\y) {\y};
            }
            \draw[thick,blue] (0,1) to[out=-45,in=135] (1,2) to[out=-45,in=135] (2,1) to[out=-45,in=135] (3,2) to[out=-45,in=135] (4,1) to[out=-45,in=135] (5,2) to[out=-45,in=135] (6,1);
            \draw[dashed,red] (0,-0.5) to[out=-45,in=135] (1,-0.25) to[out=-45,in=135] (2,-0.5) to[out=-45,in=135] (3,-0.25) to[out=-45,in=135] (4,-0.5) to[out=-45,in=135