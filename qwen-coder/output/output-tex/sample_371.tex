Certainly! Below is a TikZ LaTeX code to construct the digraph \( D' \) from a given digraph \( D \) with \( k = 2 \), along with an \((\infty, k)\)-decomposition \((F_\infty, F_k)\) of \( D' \). The Hamiltonian cycle of \( D \) is colored in blue, dashed red arcs are in \( A(F_k) \), and solid green arcs are in \( A(F_\infty) \).

```latex
\documentclass{standalone}
\usepackage{tikz}

\begin{document}

\begin{tikzpicture}[scale=1.5]

% Define nodes for the original digraph D
\node (v1) at (0,0) {1};
\node (v2) at (2,0) {2};
\node (v3) at (4,0) {3};
\node (v4) at (6,0) {4};

% Draw edges for the original digraph D
\draw[->] (v1) -- node[above] {$a_1$} (v2);
\draw[->] (v2) -- node[above] {$a_2$} (v3);
\draw[->] (v3) -- node[above] {$a_3$} (v4);
\draw[->] (v4) -- node[above] {$a_4$} (v1);

% Add extra vertices for D'
\node (v1') at (0,-2) {1'};
\node (v2') at (2,-2) {2'};
\node (v3') at (4,-2) {3'};
\node (v4') at (6,-2) {4'};

% Draw edges for D'
\draw[->] (v1) -- node[below] {$b_1$} (v1');
\draw[->] (v2) -- node[below] {$b_2$} (v2');
\draw[->] (v3) -- node[below] {$b_3$} (v3');
\draw[->] (v4) -- node[below] {$b_4$} (v4');

% Hamiltonian cycle in blue
\draw[blue, thick] (v1) -- (v