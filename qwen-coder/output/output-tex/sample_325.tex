Certainly! Below is an example of how you might create a TikZ diagram to represent a Partially Directed Acyclic Graph (PDAG) or Probability Assessment Framework (PAF) decomposition, along with partial tables for computing $P^{\mathit{ext}}_{com}(\{a, c, e\})$.

```latex
\documentclass[tikz]{standalone}
\usetikzlibrary{shapes.geometric, arrows.meta, positioning}

\tikzset{
  decision/.style={diamond, draw, fill=blue!20, text width=4em, text badly centered, node distance=3cm, inner sep=0pt},
  block/.style={rectangle, draw, fill=blue!20, text width=5em, text centered, rounded corners, minimum height=4em},
  line/.style={draw, thick, ->, >=stealth},
  cloud/.style={cloud, draw, aspect=2, fill=red!20, text width=6em, text centered, rounded corners, minimum height=7em}
}

\begin{document}

\begin{tikzpicture}[node distance=2cm]
  \node [decision] (A) {A};
  \node [block, below of=A] (B) {B};
  \node [block, right of=B] (C) {C};
  \node [block, left of=B] (D) {D};
  \node [block, below of=C] (E) {E};

  \path [line] (A) -- (B);
  \path [line] (B) -- (C);
  \path [line] (B) -- (D);
  \path [line] (D) -- (E);
  \path [line] (C) -- (E);

  % Tables
  \node [cloud, below of=D, xshift=-3cm] (Table1) {
    \begin{tabular}{|c|c|}
      \hline
      A & B \\
      \hline
      T & F \\
      \hline
      F & T \\
      \hline
    \end{tabular}
  };
  \node [cloud, below of=E, xshift=3cm] (Table2) {
    \begin{tabular}{|c|c|c|}
      \hline
      C & D &