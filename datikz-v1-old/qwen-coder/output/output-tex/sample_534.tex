\documentclass[tikz]{standalone}
\usetikzlibrary{shapes.geometric, arrows.meta}

\begin{document}

\tikzset{
  loop/.style={draw=black, thick, ->},
  node distance=2cm,
  start chain=going below,
  every join/.style=-stealth,
}

\begin{tikzpicture}[start chain]

% Define the nodes for the loops
\node[on chain] (1) {Loop 1};
\node[on chain] (2) {Loop 2};
\node[on chain] (3) {Loop 3};

% Draw the loops
\foreach \i [count=\j from 1] in {1,...,3} {
  \node[on chain, draw=white, circle, inner sep=0pt, minimum size=1cm, label=left:Loop \i] (loop-\i) {};
  \draw[loop] (loop-\i.west) -- ++(-1,-1) |- (loop-\i.east);
}

% Reflect the loops vertically
\foreach \i [count=\j from 1] in {1,...,3} {
  \node[on chain, draw=black, circle, inner sep=0pt, minimum size=1cm, label=right:Loop \i] (loop-reflected-\i) {};
  \draw[loop] (loop-reflected-\i.east) -- ++(1,-1) |- (loop-reflected-\i.west);
}

\end{tikzpicture}

\end{document}