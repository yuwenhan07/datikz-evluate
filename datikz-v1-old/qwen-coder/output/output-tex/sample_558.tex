Certainly! Below is a TikZ LaTeX code that creates a graph illustrating the differences between two payments with points of intersection labeled. The graph shows a downward trend in the monetary value paid over time.

```latex
\documentclass{standalone}
\usepackage{tikz}
\usetikzlibrary{datavisualization.formats.functions}

\begin{document}
\begin{tikzpicture}
    \datavisualization [
        style sheet=strong colors,
        visualize as line/.list={payment1,payment2},
        payment1={
            label in legend={label=\textbf{Payment 1}},
            x=read from table[col sep=comma]{data1.csv},
            y=read from table[col sep=comma]{data1.csv}
        },
        payment2={
            label in legend={label=\textbf{Payment 2}},
            x=read from table[col sep=comma]{data2.csv},
            y=read from table[col sep=comma]{data2.csv}
        }
    ]
    data [format=function] {
        var x : read = {value};
        var y : read = {value};
    };
    
    % Sample data for Payment 1 (downward trend)
    \path [only marks, mark=*] plot coordinates {(0,500) (10,400) (20,300) (30,250)};
    
    % Sample data for Payment 2 (upward trend)
    \path [only marks, mark=x] plot coordinates {(0,600) (10,700) (20,800) (30,900)};
    
    % Intersection point
    \draw [dashed] (0,250) -- (30,250);
    \node [below right] at (0,250) {$y = 250$};
    \filldraw [black] (0,250) circle (2pt);
    
    % Axes and labels
    \draw[->] (-1,0) -- (32,0) node[right] {$Number\ of\ Years$};
    \draw[->] (0,-1) -- (0,1000) node[above] {\$Amount\ of\ Money};
    \foreach \x in {0,10,20,30} \draw (\x