To generate a TikZ LaTeX diagram that accurately represents the two different ways of writing the mathematical expression for a simple tree with \( N = 7 \), we can create a visual representation of both expressions side by side. Here's how you can do it:

```latex
\documentclass{standalone}
\usepackage{tikz}

\begin{document}

\begin{tikzpicture}[node distance=1cm]
    % First expression
    \node (expr1) at (0,0) {
        \begin{tabular}{l}
            $x_1 = x_{11} + x_{12}$ \\
            $x_2 = x_{21} + x_{22}$ \\
            $x_3 = x_{31} + x_{32}$ \\
            $x_4 = x_{41} + x_{42}$ \\
            $x_5 = x_{51} + x_{52}$ \\
            $x_6 = x_{61} + x_{62}$ \\
            $x_7 = x_{71} + x_{72}$ \\
            $y_1 = y_{11} + y_{12}$ \\
            $y_2 = y_{21} + y_{22}$ \\
            $y_3 = y_{31} + y_{32}$ \\
            $y_4 = y_{41} + y_{42}$ \\
            $y_5 = y_{51} + y_{52}$ \\
            $y_6 = y_{61} + y_{62}$ \\
            $y_7 = y_{71} + y_{72}$ \\
            $z_1 = z_{11} + z_{12}$ \\
            $z_2 = z_{21} + z_{22}$ \\
        \end{tabular}
    };

    % Second expression
    \node (expr2) at (5,0) {
        \begin{tabular}{l}
            $a_1 = a_{11} + a_{12} + a_{13}$ \\
            $a_2 = a_{21} + a_{22} + a_{23}$ \\
            $a_3 = a_{31} + a_{32} + a_{33