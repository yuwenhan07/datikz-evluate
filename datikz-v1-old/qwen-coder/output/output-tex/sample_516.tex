Certainly! Below is a TikZ LaTeX code that represents the described photovoltaic power processing architecture:

```latex
\documentclass{standalone}
\usepackage{tikz}

\begin{document}
\begin{tikzpicture}[node distance=2cm]
    % Nodes
    \node[draw, rectangle, fill=green!30] (solar1) {Solar Panel 1};
    \node[draw, rectangle, fill=green!30] (solar2) [right of=solar1] {Solar Panel 2};
    \node[draw, rectangle, fill=green!30] (solar3) [below of=solar1, xshift=-2cm] {Solar Panel 3};
    
    \node[draw, rectangle, fill=green!60, minimum height=2cm] (inverter) [below of=solar3, xshift=-1cm] {Inverter};
    
    \node[draw, rectangle, fill=blue!50] (dc1) [above right of=solar1, yshift=-1cm] {DC Converter 1};
    \node[draw, rectangle, fill=blue!50] (dc2) [below right of=solar2, yshift=-1cm] {DC Converter 2};
    \node[draw, rectangle, fill=blue!50] (dc3) [below of=dc2, yshift=-1cm] {DC Converter 3};

    % Arrows
    \draw[->] (solar1.east) -- node[midway, above] {Current} (dc1.west);
    \draw[->] (solar2.east) -- node[midway, above] {Current} (dc2.west);
    \draw[->] (solar3.east) -- node[midway, above] {Current} (dc3.west);

    \draw[->] (dc1.east) -- node[midway, above] {Voltage} (inverter.west);
    \draw[->] (dc2.east) -- node[midway, above] {Voltage} (inverter.west);
    \draw[->] (dc3.east) -- node[midway, above] {Voltage} (inverter.west);

    % Additional connections
    \draw[dashed, ->] (solar3