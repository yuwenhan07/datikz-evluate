To create a TikZ LaTeX diagram that matches your description, we need to consider several components: a square grid, various symbols and equations, and connections between these elements. Below is a sample code that attempts to capture this complexity:

```latex
\documentclass{standalone}
\usepackage{tikz}
\usetikzlibrary{matrix, positioning}

\begin{document}
\begin{tikzpicture}[node distance=1cm]
    % Define styles for nodes
    \tikzset{
        nodeStyle/.style={draw, minimum size=2em, align=center},
        equationNode/.style={nodeStyle, fill=white!80},
        symbolNode/.style={nodeStyle, fill=gray!50}
    }

    % Create a matrix of nodes
    \matrix (grid) [matrix of nodes, nodes in empty cells, row sep=1cm, column sep=1cm] {
        & F = ma & E = mc^2 & W = FL \\
        & P = \frac{F}{A} & K = \frac{1}{2}mv^2 & A = \pi r^2 \\
        & \vec{v} = \frac{\Delta x}{\Delta t} & U = mgh & V = pV \\
        & \vec{a} = \frac{\Delta v}{\Delta t} & Q = IV & I = \sqrt{R^2 + X_L^2} \\
    };

    % Add connections between nodes
    \foreach \i in {1,...,4}{
        \foreach \j in {1,...,4}{
            \ifnum\i<4
                \draw[->] (grid-\i-\j.south) -- (grid-\the\numexpr\i+1\relax-\j.north);
            \fi
            \ifnum\j<4
                \draw[->] (grid-\i-\j.east) -- (grid-\i-\the\numexpr\j+1\relax.west);
            \fi
        }
    }

    % Place additional symbols and equations outside the grid
    \node[symbolNode, above right=of grid-1-1] (force) {$\vec{F}$};
    \node[symbolNode, below left=of grid-4-4] (energy) {$E$};
    \node[equation