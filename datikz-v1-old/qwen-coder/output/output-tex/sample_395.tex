To illustrate the local complementation of a graph \( G \) with respect to a vertex \( u \), we need to understand that local complementation involves flipping the edges between the neighbors of \( u \). In this case, let's assume \( G \) is a simple undirected graph where vertices \( v_1 \), \( v_2 \), and \( v_3 \) are connected as follows:

- \( v_1 \) is connected to \( u \)
- \( v_2 \) is connected to \( u \)
- \( v_3 \) is connected to \( u \)

Additionally, there are direct edges:
- \( v_1 \) is connected to \( v_3 \)
- \( v_1 \) is connected to \( v_2 \)
- \( v_2 \) is connected to \( v_3 \)

The local complementation of \( G \) with respect to \( u \) will flip the edges between \( v_1 \), \( v_2 \), and \( v_3 \).

Here’s how you can represent this using TikZ in LaTeX:

```latex
\documentclass{standalone}
\usepackage{tikz}

\begin{document}
\begin{tikzpicture}[scale=1.5]
    % Nodes
    \node (u) at (0, 0) [circle, draw] {$u$};
    \node (v1) at (-1.5, -1.5) [circle, draw] {$v_1$};
    \node (v2) at (0, -1.5) [circle, draw] {$v_2$};
    \node (v3) at (1.5, -1.5) [circle, draw] {$v_3$};

    % Edges before local complementation
    \draw[thick] (u) -- (v1);
    \draw[thick] (u) -- (v2);
    \draw[thick] (u) -- (v3);
    \draw[thick] (v1) -- (v2);
    \draw[thick] (v2) -- (v3);
    \draw[thick] (v1) -- (v3);

    % Labeling edges before local complementation
    \path (u) -- (v1) node[midway, below left