Certainly! Below is a TikZ LaTeX code that generates an illustration of two gray rectangles (representing pages of a book) with randomly drawn lines representing magnetic fields. The lines will be filled with opposite polarity to illustrate the concept that even if a magnet is broken, the pieces retain their polarity.

```latex
\documentclass[tikz,border=10pt]{standalone}
\usetikzlibrary{decorations.pathmorphing}

\begin{document}
\begin{tikzpicture}[scale=2]

% Define the gray rectangles
\fill[gray!30] (-2,-2) rectangle (2,2);
\fill[gray!30] (2,-2) rectangle (4,2);

% Draw random lines with opposite polarity
\draw[dashed, thick, decorate, decoration={zigzag, amplitude=0.5cm, segment length=1cm}] (-2,1.5) -- (2,1.5);
\draw[dotted, thick, decorate, decoration={snake, amplitude=0.5mm, segment length=2mm}] (2,1.5) -- (4,1.5);

\draw[dashed, thick, decorate, decoration={zigzag, amplitude=0.5cm, segment length=1cm}] (-2,0.5) -- (2,0.5);
\draw[dotted, thick, decorate, decoration={snake, amplitude=0.5mm, segment length=2mm}] (2,0.5) -- (4,0.5);

\draw[dashed, thick, decorate, decoration={zigzag, amplitude=0.5cm, segment length=1cm}] (-2,-0.5) -- (2,-0.5);
\draw[dotted, thick, decorate, decoration={snake, amplitude=0.5mm, segment length=2mm}] (2,-0.5) -- (4,-0.5);

\draw[dashed, thick, decorate, decoration={zigzag, amplitude=0.5cm, segment length=1cm}] (-2,-1.5) -- (2,-1.5);
\draw[dotted, thick, decorate, decoration={snake, amplitude=0.5mm, segment length=2mm}] (2,-1.5) -- (4,-1.5);

% Add text annotations
\node at (-2,2.5) {Page 1};
\n