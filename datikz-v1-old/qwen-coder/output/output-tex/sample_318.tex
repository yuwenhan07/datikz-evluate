To create a TikZ LaTeX diagram that represents a tree with multiple branches and circles, each labeled, you can use the following code. This example will create a simple flowchart-like structure with nodes connected by edges.

```latex
\documentclass{standalone}
\usepackage{tikz}

\begin{document}

\begin{tikzpicture}[level distance=3cm,
                    level 1/.style={sibling distance=4cm},
                    level 2/.style={sibling distance=2cm},
                    level 3/.style={sibling distance=1cm},
                    node distance=3cm,
                    every node/.style={circle, draw, fill=white, text centered, minimum width=2cm}]
    
    % Root node
    \node (root) {Start};
    
    % Level 1
    \node [below left of=root] (A) {Step A};
    \node [below right of=root] (B) {Step B};
    
    % Level 2
    \node [below left of=A] (A1) {Substep A1};
    \node [below right of=A] (A2) {Substep A2};
    \node [below left of=B] (B1) {Substep B1};
    \node [below right of=B] (B2) {Substep B2};
    
    % Level 3
    \node [below left of=A1] (A1a) {Action A1a};
    \node [below right of=A1] (A1b) {Action A1b};
    \node [below left of=A2] (A2a) {Action A2a};
    \node [below right of=A2] (A2b) {Action A2b};
    \node [below left of=B1] (B1a) {Action B1a};
    \node [below right of=B1] (B1b) {Action B1b};
    \node [below left of=B2] (B2a) {Action B2a};
    \node [below right of=B2] (B2b) {Action B2b};
    
    % Edges
    \draw (root) -- (A);
    \draw (root) -- (B);
    \draw (A) -- (A1);
    \draw (A) -- (A2);
