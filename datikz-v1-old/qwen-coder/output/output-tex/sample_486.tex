Certainly! Below is a TikZ LaTeX code that represents the described configuration:

```latex
\documentclass{article}
\usepackage{tikz}

\begin{document}

\begin{figure}[h]
    \centering
    \begin{tikzpicture}[scale=0.8]

        % Upper part: Weighted Graph Corresponding to Rank l Version of H_n and E_n Theories

        % Genus Zero Component
        \node[circle, draw, fill=blue!20] (G0_1) at (0, 0) {};
        \node[circle, draw, fill=blue!20] (G0_2) at (2, 0) {};
        \node[circle, draw, fill=red!20] (G0_3) at (4, 0) {};
        
        % Edges for Genus Zero Component
        \draw[->] (G0_1) -- node[midway, above] {$1 + \frac{1}{l} + \frac{l - 1}{l}$} (G0_2);
        \draw[->] (G0_2) -- node[midway, above] {$\frac{3}{4} + \frac{3}{4} + \frac{1}{2}$} (G0_3);

        % Genus One Component
        \node[circle, draw, fill=green!20] (G1_1) at (-2, -2) {};
        \node[circle, draw, fill=green!20] (G1_2) at (0, -2) {};
        \node[circle, draw, fill=green!20] (G1_3) at (2, -2) {};

        % Edges for Genus One Component
        \draw[->] (G1_1) -- node[midway, below] {$3$} (G1_2);
        \draw[->] (G1_2) -- node[midway, below] {$3$} (G1_3);
        \draw[->] (G1_3) -- node[midway, below] {$\frac{1}{2}$} (G1_1);

        % Automorphism Permutation
        \draw[dashed, ->] (G0_1) -- (G1_1);
