To create a TikZ LaTeX diagram that captures both the black and white and colorful representations of a pentagon with vertices labeled 0 and 1, you can use the following code:

```latex
\documentclass{standalone}
\usepackage{tikz}

\begin{document}

% Define the vertices of the pentagon
\coordinate (A) at (0,0);
\coordinate (B) at (2,0);
\coordinate (C) at (3,1.732);
\coordinate (D) at (1,2.464);
\coordinate (E) at (-1,1.732);

% Black and White Pentagon
\begin{scope}[xshift=3cm]
    \draw[black] (A) -- (B) -- (C) -- (D) -- (E) -- cycle;
    \fill[white] (A) circle[radius=0.05];
    \node at (A) [below left] {0};
    \fill[black] (B) circle[radius=0.05];
    \node at (B) [below right] {1};
    \fill[white] (C) circle[radius=0.05];
    \node at (C) [above right] {0};
    \fill[black] (D) circle[radius=0.05];
    \node at (D) [above left] {1};
    \fill[white] (E) circle[radius=0.05];
    \node at (E) [below] {0};
\end{scope}

% Colorful Pentagon
\begin{scope}[xshift=6cm]
    \draw[red] (A) -- (B) -- (C) -- (D) -- (E) -- cycle;
    \fill[yellow] (A) circle[radius=0.05];
    \node at (A) [below left] {0};
    \fill[green] (B) circle[radius=0.05];
    \node at (B) [below right] {1};
    \fill[blue] (C) circle[radius=0.05];
    \node at (C) [above right] {0};
    \fill[magenta] (D) circle[radius=0.05];
    \node at (D) [above