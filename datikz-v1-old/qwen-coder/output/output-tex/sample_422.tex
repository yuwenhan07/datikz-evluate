To create a TikZ diagram based on your description, we need to plot the function \( f(x) \) over the specified range and indicate the behavior of the function at specific points. Here's a step-by-step guide to creating the diagram:

1. **Define the range and steps**: We will use steps of 0.1 for both the parameter and \( x \).
2. **Plot the function values**: We will calculate \( f(x) \) for each \( x \) in the range and determine whether it is positive or negative.
3. **Indicate the sign changes**: We will use blue for positive values and red for negative values.

Let's assume the function \( f(x) = x^2 - 4x + 3 \) as an example. This function has roots at \( x = 1 \) and \( x = 3 \), and it opens upwards (since the coefficient of \( x^2 \) is positive).

Here is the TikZ code to generate the diagram:

```latex
\documentclass{standalone}
\usepackage{tikz}

\begin{document}

\begin{tikzpicture}[scale=1]
    % Define the domain and step size
    \def\xmin{-5}
    \def\xmax{5}
    \def\ymin{-10}
    \def\ymax{10}
    \def\step{0.1}

    % Draw the axes
    \draw[->] (\xmin,0) -- (\xmax,0) node[right] {$x$};
    \draw[->] (0,\ymin) -- (0,\ymax) node[above] {$f(x)$};

    % Plot the function
    \foreach \x in \xmin\step...\xmax {
        \pgfmathsetmacro{\y}{(\x)^2 - 4*\x + 3}
        \ifdim\y pt>0pt
            \filldraw[blue] (\x,\y) circle (0.05);
        \else
            \filldraw[red] (\x,\y) circle (0.05);
        \fi
    }

    % Indicate the critical points
    \filldraw[black] (1,-1) circle (0.05) node[below right] {$(1, -1)$};
    \filldraw[black] (3,-6) circle (0.