To create a TikZ diagram that represents a modified instance of the Hansen graph with a large number of equations arranged in a tree-like structure, we can use nested `node` commands within a `tikzpicture`. Each node will represent an equation of the form \( C_{n+1} = C_n + 1 \), and the branches of the tree will represent different values of \( n \).

Here's a LaTeX code snippet that generates such a diagram:

```latex
\documentclass{standalone}
\usepackage{tikz}

\begin{document}

\begin{tikzpicture}[level distance=3cm,
                    level 1/.style={sibling distance=6cm},
                    level 2/.style={sibling distance=4cm},
                    level 3/.style={sibling distance=2cm},
                    every node/.style={circle, draw, align=center}]
    
    % Root node
    \node (root) {C1};
    
    % Level 1
    \foreach \i in {1,...,5} {
        node [below left of=root] (L1-\i) {$C_2$};
    }
    
    % Level 2
    \foreach \i in {1,...,5} {
        node [below left of=L1-\i] (L2-\i-1) {$C_3$};
        node [below right of=L1-\i] (L2-\i-2) {$C_3$};
    }
    
    % Level 3
    \foreach \i in {1,...,5} {
        foreach \j in {1,...,2} {
            node [below left of=L2-\i-\j] (L3-\i-\j-1) {$C_4$};
            node [below right of=L2-\i-\j] (L3-\i-\j-2) {$C_4$};
        }
    }
    
    % Connect nodes
    \foreach \i in {1,...,5} {
        \draw (root) -- (L1-\i);
        \foreach \j in {1,...,2} {
            \draw (L1-\i) -- (L2-\i-\j);
        }
        \foreach \k in {1,...,2} {
            \foreach \l in {1,...,2} {
                \draw (L2-\i-\k