To translate the 4-regular graph \( L \) on six vertices into the disjoint union of three paths of length 1, which corresponds to the complement graph \( L^c \), we need to follow these steps:

1. **Identify the structure of \( L \)**:
   - A 4-regular graph on six vertices means each vertex is connected to exactly four other vertices.
   - The total number of edges in \( L \) can be calculated using the degree sum formula: 
     \[
     \text{Total number of edges} = \frac{\sum_{i=1}^{n} \deg(v_i)}{2}
     \]
     where \( n \) is the number of vertices and \( \deg(v_i) \) is the degree of vertex \( v_i \).
     For \( L \):
     \[
     \text{Total number of edges} = \frac{6 \times 4}{2} = 12
     \]

2. **Construct the complement graph \( L^c \)**:
   - The complement graph \( L^c \) will have the same set of vertices as \( L \), but it will include an edge between any two vertices that are not adjacent in \( L \).
   - Since \( L \) has 12 edges, \( L^c \) will have:
     \[
     \text{Total number of possible edges} = \binom{6}{2} = 15
     \]
     Therefore, the number of edges in \( L^c \) is:
     \[
     15 - 12 = 3
     \]

3. **Determine the structure of \( L^c \)**:
   - \( L^c \) must consist of three edges connecting pairs of non-adjacent vertices.
   - To achieve this, we need to ensure that no three vertices form a triangle (since if they did, their complements would share an edge).

4. **Visualize \( L^c \)**:
   - One way to construct \( L^c \) is to start with a complete graph \( K_6 \) (which has all possible edges) and remove the edges that correspond to \( L \).
   - The remaining edges will form the paths of length 1.

Here is the TikZ code to draw \( L^c \):

```latex
\documentclass[tikz,border=