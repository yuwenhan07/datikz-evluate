\documentclass{standalone}
\usepackage{tikz}

\begin{document}
\begin{tikzpicture}[scale=2]

% Define the vertices for the incoming particles
\node (in1) at (-2,0) [circle, draw, fill=blue!30] {$C_i$};
\node (in2) at (2,0) [circle, draw, fill=red!30] {$C_j$};

% Draw arrows from the vertices to indicate incoming particles
\draw[->] (in1) -- node[midway, above] {Particle 1} ++(0,-1);
\draw[->] (in2) -- node[midway, above] {Particle 2} ++(0,-1);

% Define the vertex for the outgoing particle
\node (out) at (0,-4) [circle, draw, fill=green!30] {};

% Draw arrows from the outgoing vertex to indicate the outgoing particle
\draw[<-] (out) -- node[midway, below] {Outgoing Particle} ++(0,1);

% Draw lines connecting the incoming and outgoing vertices
\draw[dashed] (in1) -- (out);
\draw[dashed] (in2) -- (out);

% Label the outgoing particle as a linear combination of 1-particle states
\node at (0,-5) [below] {
    \(\sum_{k} c_k | \psi_k \rangle\)
};

% Add labels for the conjugacy classes
\node at (-2,-6) [left] {Conjugacy Class 1};
\node at (2,-6) [right] {Conjugacy Class 2};

\end{tikzpicture}
\end{document}