Certainly! Below is an example of how you can create a TikZ diagram to represent an ear decomposition \( ED = \{P_1, P_2, P_3, P_4\} \) of a graph. In this example, \( P_1 \) is a cycle with 10 edges, \( P_2 \) is a path with 4 edges, \( P_3 \) is a path with 3 edges, and \( P_4 \) is a path with 2 edges.

```latex
\documentclass[tikz]{standalone}
\usetikzlibrary{arrows.meta, positioning}

\tikzset{
    node distance=2cm,
    every node/.style={circle, draw, fill=white},
    edge/.style={thick, -Stealth},
    thinedge/.style={thin, -Stealth}
}

\begin{document}

\begin{tikzpicture}[scale=0.8]

% Cycle P1
\node (P1_1) at (0,0) {};
\node (P1_2) at (2,0) {};
\node (P1_3) at (4,-2) {};
\node (P1_4) at (6,-2) {};
\node (P1_5) at (8,-4) {};
\node (P1_6) at (8,-6) {};
\node (P1_7) at (6,-6) {};
\node (P1_8) at (4,-4) {};
\node (P1_9) at (2,-4) {};
\node (P1_10) at (0,-2) {};

\draw[edge] (P1_1) -- (P1_2);
\draw[edge] (P1_2) -- (P1_3);
\draw[edge] (P1_3) -- (P1_4);
\draw[edge] (P1_4) -- (P1_5);
\draw[edge] (P1_5) -- (P1_6);
\draw[edge] (P1_6) -- (P1_7);
\draw[edge] (P1_7) -- (P1_8);
\draw[edge] (P1_8) -- (P1_9);
\draw[edge] (P1_9) -- (P1_