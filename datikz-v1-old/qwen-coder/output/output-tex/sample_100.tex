Certainly! Below is a TikZ LaTeX code that generates a diagram illustrating the differences between two forests, `tw S_N` and `tw S_M`, as described in your prompt. The key aspects include:

1. The radius doubling with each decrementing layer, except for the specific step.
2. The left circles representing `tw S_N`.
3. The right circles representing `tw S_M`.
4. The grey set representing the estimated forest `w_N` (lower layers of `tw S_M`) being subtracted from `tw S_N`.

```latex
\documentclass{standalone}
\usepackage{tikz}

\begin{document}
\begin{tikzpicture}[scale=0.8]
    % Define the positions and sizes of the circles
    \def\R{2} % Radius of the first circle
    \def\n{5} % Number of layers

    % Draw the left circles (new tower tw S_N)
    \foreach \k in {0,...,\n}{
        \draw[fill=blue] (0, -\R*\k) circle (\R);
    }

    % Draw the right circles (old tower tw S_M)
    \foreach \k in {0,...,\n}{
        \ifnum\k=\n-1
            \draw[fill=green] (5, -\R*2*\k) circle (2*\R); % Quadruple radius for the special case
        \else
            \draw[fill=green] (5, -\R*\k) circle (\R);
        \fi
    }

    % Draw the grey set (estimated forest w_N)
    \foreach \k in {0,...,\n-2}{
        \draw[fill=gray] (2.5, -\R*\k) circle (\R);
    }
    
    % Draw labels
    \node at (-2, 0) [left] {tw S_N};
    \node at (7, 0) [right] {tw S_M};
    \node at (2.5, -\R*(\n)) [below] {$w_N$};

    % Draw lines connecting corresponding points
    \foreach \k in {0,...,\n}{
        \draw[-stealth] (0, -\R*\k) -- (5, -\R*\k);
    }
\end{tikzpicture}
\end{document