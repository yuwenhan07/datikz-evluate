\documentclass{article}
\usepackage{tikz}
\usetikzlibrary{matrix}

\begin{document}

\begin{figure}[h]
    \centering
    \begin{tikzpicture}
        % Define styles for matrices
        \tikzset{
            matrix/.style={matrix of nodes, nodes in empty cells, row sep=0.5cm, column sep=0.5cm},
            cell/.style={draw, minimum size=1cm}
        }

        % First matrix H_i^j
        \node[matrix] (H) at (0,0) {
            \node[cell]{1} & \node[cell]{2} & \node[cell]{3} \\
            \node[cell]{4} & \node[cell]{5} & \node[cell]{6} \\
            \node[cell]{7} & \node[cell]{8} & \node[cell]{9} \\
        };
        \node[below=0.5cm of H] {\(H_i^j\)};
        
        % Second matrix K_i^j
        \node[matrix] (K) at (3,0) {
            \node[cell]{a} & \node[cell]{b} & \node[cell]{c} \\
            \node[cell]{d} & \node[cell]{e} & \node[cell]{f} \\
            \node[cell]{g} & \node[cell]{h} & \node[cell]{i} \\
        };
        \node[below=0.5cm of K] {\(K_i^j\)};
    \end{tikzpicture}
    \caption{Matrices \(H_i^j\) and \(K_i^j\) used in the C-step after each iteration step.}
    \label{fig:matrices}
\end{figure}

\end{document}