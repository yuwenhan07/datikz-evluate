Certainly! Below is a TikZ LaTeX code that generates a Venn diagram for optimality and stability notions as described. The diagram includes the specified sections labeled and colored appropriately.

```latex
\documentclass{standalone}
\usepackage{tikz}
\usetikzlibrary{shapes.geometric, positioning}

\begin{document}

\begin{tikzpicture}[node distance=2cm]
    % Define colors
    \colorlet{stable}{blue!50!black}
    \colorlet{optimal}{red!50!black}
    \colorlet{swc}{green!50!black}
    \colorlet{ls}{yellow!50!black}
    
    % Draw the main circle (white)
    \draw[thick] (0,0) circle (3cm);
    
    % Draw the inner circles for stability and optimality
    \draw[thick, stable] (0,0) circle (2.4cm);
    \draw[thick, optimal] (0,0) circle (2cm);
    
    % Label the sections
    \node at (0,-1.8) [text width=6em, align=center, fill=white, rounded corners] {Stable CTC Optimal CEO};
    \node at (-1.2,-0.5) [text width=4em, align=center, fill=white, rounded corners] {Stable C P O};
    \node at (1.2,-0.5) [text width=4em, align=center, fill=white, rounded corners] {SWC};
    \node at (0,1.2) [text width=4em, align=center, fill=white, rounded corners] {LS};
    
    % Add shadows to the sections
    \fill[gray!30] (0,-1.8) rectangle (6,-3);
    \fill[gray!30] (-2,-1) rectangle (2,-2);
    \fill[gray!30] (2,-1) rectangle (6,-2);
    \fill[gray!30] (0,1.2) rectangle (4,1.8);
    
    % Add labels for the sections
    \node at (3,-2.5) [below right] {IC};
    \node at (3,-3.5) [below right] {IR