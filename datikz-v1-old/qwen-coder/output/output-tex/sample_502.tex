Certainly! Below is a TikZ LaTeX code to generate the Petri nets for the processes \( \text{CB} \), \( (\text{CB})^h \), and \( (\text{BB})^h \).

```latex
\documentclass{article}
\usepackage{tikz}

\begin{document}

% Define the style for places and transitions
\tikzset{
    place/.style={circle, draw=black, fill=white, inner sep=6pt},
    transition/.style={rectangle, draw=black, fill=gray!30, rounded corners, inner sep=8pt},
    arrow/.style={->, thick, >=stealth}
}

\begin{figure}[ht]
    \centering
    \begin{tikzpicture}[node distance=2cm]

        % Process CB
        \node (place1_cb) [place] {P1};
        \node (transition1_cb) [transition, right of=place1_cb] {$T_1$};
        \node (place2_cb) [place, right of=transition1_cb] {P2};

        \draw[arrow] (place1_cb) -- node[above] {a} (transition1_cb);
        \draw[arrow] (transition1_cb) -- node[above] {b} (place2_cb);

        \node at (4, 0) {\textbf{(a) Process CB}};

        % Process (CB)^h
        \node (place1_cbh) [place, below left of=place1_cb, yshift=-3cm] {P1'};
        \node (transition1_cbh) [transition, right of=place1_cbh] {$T_1'$};
        \node (place2_cbh) [place, right of=transition1_cbh] {P2'};

        \draw[arrow] (place1_cbh) -- node[below] {a'} (transition1_cbh);
        \draw[arrow] (transition1_cbh) -- node[below] {b'} (place2_cbh);

        \node at (2, -3) {\textbf{(b) $(\text{CB})^h$}};

        % Process (BB)^h
        \node (place1_bbh) [place, below left of=place1_cbh, xshift=-3cm] {P1