Certainly! Below is a TikZ LaTeX code that generates a visual representation of a white circle with various mathematical symbols and lines inside it, resembling a complex mathematical concept such as a differential equation or an integral.

```latex
\documentclass{standalone}
\usepackage{tikz}

\begin{document}

\begin{tikzpicture}[scale=2]
    % Background circle
    \fill[black] (0,0) circle (1);

    % White circle
    \draw[white, thick] (0,0) circle (0.8);

    % Mathematical symbols and lines
    \node at (0,0.5) {$f(x)$};
    \node at (-0.3,0.3) {$\frac{d}{dx}$};
    \node at (0.3,-0.3) {$\int$};
    \node at (0,-0.5) {$g(x)$};

    \draw[white, thick] (0,0.4) -- (0,0.6);
    \draw[white, thick] (0,-0.4) -- (0,-0.6);
    \draw[white, thick] (-0.4,0) -- (-0.6,0);
    \draw[white, thick] (0.4,0) -- (0.6,0);

    % Additional lines and symbols
    \draw[white, thick] (-0.2,0.2) arc (0:90:0.2);
    \draw[white, thick] (0.2,0.2) arc (90:180:0.2);
    \draw[white, thick] (0.2,-0.2) arc (180:270:0.2);
    \draw[white, thick] (-0.2,-0.2) arc (270:360:0.2);

    \node at (-0.3,0.6) {$x_1$};
    \node at (0.3,0.6) {$x_2$};
    \node at (-0.3,-0.6) {$x_3$};
    \node at (0.3,-0.6) {$x_4$};

    \node at (-0.5,0.5) {$y_1$