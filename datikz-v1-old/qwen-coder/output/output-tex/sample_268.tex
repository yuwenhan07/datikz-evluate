Certainly! Below is a TikZ LaTeX code that generates an image resembling the description you provided. This includes a flowchart with arrows indicating different paths and several mathematical equations, including a Taylor series expansion.

```latex
\documentclass{standalone}
\usepackage{tikz}
\usetikzlibrary{shapes.geometric, arrows.meta}

\tikzstyle{startstop} = [rectangle, rounded corners, minimum width=3cm, minimum height=1cm,text centered, draw=black, fill=red!30]
\tikzstyle{process} = [rectangle, minimum width=3cm, minimum height=1cm, text centered, draw=black, fill=orange!30]
\tikzstyle{decision} = [diamond, minimum width=3cm, minimum height=1cm, text centered, draw=black, fill=green!30]
\tikzstyle{arrow} = [thick,->,>=stealth]

\begin{document}
\begin{tikzpicture}[node distance=2cm]

\node (start) [startstop] {Start};
\node (input) [process, below of=start] {Input Data};
\node (flowchart) [decision, below of=input] {Flowchart Decision};
\node (output1) [process, below left of=flowchart, xshift=-2cm] {Output 1};
\node (output2) [process, below right of=flowchart, xshift=2cm] {Output 2};
\node (equation1) [process, below of=output1, yshift=-2cm] {$f(x) = \sum_{n=0}^{\infty} \frac{f^{(n)}(a)}{n!}(x-a)^n$};
\node (equation2) [process, below of=output2, yshift=-2cm] {$g(x) = \sum_{n=0}^{\infty} \frac{g^{(n)}(b)}{n!}(x-b)^n$};
\node (end) [startstop, below of=equation2, yshift=-2cm] {End};

\draw [arrow] (start) -- (input);
\draw [arrow] (input) -- (flowchart);
\draw [arrow] (flowchart) -| node[anchor=south] {Yes} (output