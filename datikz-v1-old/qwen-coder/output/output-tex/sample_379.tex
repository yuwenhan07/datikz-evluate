Certainly! Below is a TikZ LaTeX code that visualizes the described process step by step:

```latex
\documentclass{standalone}
\usepackage{tikz}

\begin{document}
\begin{tikzpicture}[scale=1.5, every node/.style={inner sep=0pt}]

% Step (a)
\node (A) at (0,0) {A};
\node (U) at (-2,-2) {U};

% Draw boundary of A
\draw[thick] (A.north east) -- (A.south east) -- (A.south west) -- (A.north west);

% Draw hole in A for U
\filldraw[gray!30] (-1.5,-1.5) rectangle (0.5,0.5);

% Step (b)
\node (A_prime) at (4,0) {A'};
\node (B) at (5,-1) {};
\node (C) at (3,-1) {};

% Draw boundary of A'
\draw[thick] (A_prime.north east) -- (A_prime.south east) -- (A_prime.south west) -- (A_prime.north west);
\draw[thick] (B.east) -- (B.west);
\draw[thick] (C.east) -- (C.west);

% Partially cover boundary of A'BC
\clip (A_prime.north east) rectangle (C.south west);
\draw[thick] (A_prime.north east) -- (A_prime.south east) -- (A_prime.south west) -- (A_prime.north west);
\draw[thick] (B.east) -- (B.west);
\draw[thick] (C.east) -- (C.west);

% Step (c)
\node (U_double_prime) at (8,0) {U''};
\node (gates) at (7,-1) {};

% Draw U''
\draw[thick] (U_double_prime.north east) -- (U_double_prime.south east) -- (U_double_prime.south west) -- (U_double_prime.north west);
\draw[thick] (gates.east) -- (gates.west);

% Remove some gates around A'
\clip (A_prime.north east) rectangle (C