To create a visual representation of the graph \( C_7 + v \) using TikZ in LaTeX, we need to understand what this notation means:

- \( C_7 \) represents a cycle graph with 7 vertices.
- \( v \) represents an isolated vertex (a single node that is not connected to any other nodes).
- \( C_7 + v \) represents the union of the cycle graph \( C_7 \) and the isolated vertex \( v \).

Here's how you can draw this graph using TikZ in LaTeX:

```latex
\documentclass{standalone}
\usepackage{tikz}

\begin{document}
\begin{tikzpicture}[scale=1.5]

% Draw the cycle graph C_7
\node[draw, circle] (v1) at (0,2) {};
\node[draw, circle] (v2) at (2,3) {};
\node[draw, circle] (v3) at (4,2) {};
\node[draw, circle] (v4) at (3,-1) {};
\node[draw, circle] (v5) at (-1,-1) {};
\node[draw, circle] (v6) at (-3,2) {};
\node[draw, circle] (v7) at (-1,3) {};

% Connect the nodes to form a cycle
\foreach \i/\j in {v1/v2,v2/v3,v3/v4,v4/v5,v5/v6,v6/v7,v7/v1} {
    \draw (\i) -- (\j);
}

% Draw the isolated vertex v
\node[draw, circle, fill=white, inner sep=2pt] (v8) at (0,0) {};

% Optionally, add labels if needed
\node at (0,2.5) [label=above:$v_1$] {};
\node at (2,3.5) [label=right:$v_2$] {};
\node at (4,2.5) [label=below right:$v_3$] {};
\node at (3,-1.5) [label=below left:$v_4$] {};
\node at (-1,-1.5) [label=left:$v_5$] {};
\node at (-3,2.5) [label=above left:$v_6