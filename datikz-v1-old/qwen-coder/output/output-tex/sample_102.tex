Certainly! Below is a TikZ LaTeX code that generates a visual representation of the English score spectrum of Vietnamese students in 2019. The image includes a blue and green color scheme with a white background. There are two distinct sections represented by blue and green curves.

```latex
\documentclass{standalone}
\usepackage{tikz}
\usepackage{xcolor}

\begin{document}

\begin{tikzpicture}[scale=1.5]
    % Set up the background
    \fill[white] (0,0) rectangle (8,4);

    % Define colors
    \definecolor{blue_curve}{RGB}{0,0,255}
    \definecolor{green_curve}{RGB}{0,255,0}

    % Draw the x-axis
    \draw[thick,->] (0,0) -- (8,0) node[right] {Number of Students};

    % Draw the y-axis
    \draw[thick,->] (0,0) -- (0,4) node[above] {English Score};

    % Label the axes
    \foreach \x in {1,2,...,7} {
        \node at (\x,-0.2) {\x};
    }
    \foreach \y in {1,2,...,3} {
        \node at (-0.2,\y) {\y};
    }

    % Draw the blue curve
    \draw[blue_curve, thick] plot coordinates {(1,1) (2,1.5) (3,2) (4,2.5) (5,3) (6,3.5) (7,4)};
    \node at (3.5,3.5) [anchor=north west] {Blue Curve: Group A};

    % Draw the green curve
    \draw[green_curve, thick] plot coordinates {(1,3) (2,2.5) (3,2) (4,1.5) (5,1) (6,0.5) (7,0)};
    \node at (3.5,1.5) [anchor=south east] {Green Curve: Group B};

    % Add titles and labels
    \node at (4,4.2) [anchor=north] {English Score Spectrum of Vietnamese Students in 2019};
    \node at