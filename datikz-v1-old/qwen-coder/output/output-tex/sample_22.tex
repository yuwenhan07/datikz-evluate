Certainly! Below is an example of how you can create a visually appealing diagram using TikZ and TikZ-cd that includes overlapping circles in various sizes and colors.

```latex
\documentclass{standalone}
\usepackage{tikz-cd}

\begin{document}

\begin{tikzpicture}[overlay, remember picture]
    % Define colors
    \definecolor{red}{RGB}{255,0,0}
    \definecolor{blue}{RGB}{0,0,255}
    \definecolor{green}{RGB}{0,255,0}

    % Draw circles
    \draw[fill=red] (0,0) circle (1cm);
    \draw[fill=blue] (-2,-2) circle (1.5cm);
    \draw[fill=green] (3,-1) circle (1cm);
    \draw[fill=red] (1,2) circle (0.7cm);
    \draw[fill=blue] (-1,1) circle (1cm);
    \draw[fill=green] (2,-4) circle (1.2cm);

    % Add labels if needed
    \node at (0,0) [above right] {A};
    \node at (-2,-2) [below left] {B};
    \node at (3,-1) [above left] {C};
    \node at (1,2) [below right] {D};
    \node at (-1,1) [above] {E};
    \node at (2,-4) [below] {F};

    % Connect circles with arrows for a sequence
    \draw[-stealth] (0,0) -- node[midway, above] {$f_1$} (-2,-2);
    \draw[-stealth] (-2,-2) -- node[midway, below] {$f_2$} (3,-1);
    \draw[-stealth] (3,-1) -- node[midway, above left] {$f_3$} (1,2);
    \draw[-stealth] (1,2) -- node[midway, below right] {$f_4$} (-1,1);
    \draw[-stealth] (-1,1) -- node[midway, above] {$f_5$} (2,-4);
    \draw