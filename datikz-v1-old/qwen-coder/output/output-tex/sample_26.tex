Certainly! Below is a TikZ LaTeX code that illustrates the trajectory \( a(t) = (0, 0, 2t) \) for \( t_1, t_2 \) in the \( xOz \) plane. The green arc represents the observable points, while the dotted arc represents the non-observable points.

```latex
\documentclass{standalone}
\usepackage{tikz}

\begin{document}
\begin{tikzpicture}[scale=1.5]
    % Draw the axes
    \draw[->] (-3,0,0) -- (3,0,0) node[right] {$x$};
    \draw[->] (0,-3,0) -- (0,3,0) node[above] {$y$};
    \draw[->] (0,0,-3) -- (0,0,3) node[below left] {$z$};

    % Draw the trajectory
    \draw[domain=-3:3,samples=100,blue] plot (\x,0,\x*\x);
    
    % Define points t1 and t2
    \def\t1{-2}
    \def\t2{2}
    
    % Observable points (green)
    \fill[green] (\t1,0,\t1*\t1) circle (2pt) node[above right] {Observable Point at $t_1$};
    \fill[green] (\t2,0,\t2*\t2) circle (2pt) node[above right] {Observable Point at $t_2$};
    
    % Non-observable points (dotted)
    \fill[dotted,red] (-(\t1+0.5),0,\t1*\t1+1) circle (2pt) node[above right] {Non-Observable Point at $t_1 + 0.5$};
    \fill[dotted,red] ((\t2-0.5),0,\t2*\t2-1) circle (2pt) node[above right] {Non-Observable Point at $t_2 - 0.5$};
    
    % Labels for the trajectory
    \node at (0,0,4) [above] {Trajectory $a(t) = (0, 0, 2t)$};
\end{tikz