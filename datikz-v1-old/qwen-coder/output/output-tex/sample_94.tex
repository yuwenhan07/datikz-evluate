Certainly! Below is an example of a TikZ LaTeX code that generates a diagram showing the function \( f_b \) in the regions \([v_0-2, v_0+2]\) for both bit values \( b_k = 0 \) (left) and \( b_k = 1 \) (right).

```latex
\documentclass{standalone}
\usepackage{tikz}

\begin{document}

\begin{tikzpicture}[scale=0.8]
    % Define the axes
    \draw[->] (-3,0) -- (5,0) node[right] {$v$};
    \draw[->] (0,-1) -- (0,4) node[above] {$f_b$};

    % Draw the regions and functions for b_k = 0
    \fill[blue!30] (-3,0) rectangle (-1,2);
    \fill[red!30] (-1,2) rectangle (1,4);
    \fill[green!30] (1,4) rectangle (3,6);
    \fill[yellow!30] (3,6) rectangle (5,8);

    \node at (-2.5,1) {0};
    \node at (0,3) {0};
    \node at (2,5) {0};
    \node at (4,7) {0};

    % Draw the regions and functions for b_k = 1
    \fill[blue!30] (-3,0) rectangle (-1,2);
    \fill[red!30] (-1,2) rectangle (1,4);
    \fill[green!30] (1,4) rectangle (3,6);
    \fill[yellow!30] (3,6) rectangle (5,8);

    \node at (-2.5,1) {1};
    \node at (0,3) {1};
    \node at (2,5) {1};
    \node at (4,7) {1};

    % Draw the lines indicating the regions
    \draw[dashed] (-3,0) -- (-3,8);
    \draw[dashed] (-1,0) -- (-1,8);
    \draw[dashed] (1,0) -- (1,8);
    \