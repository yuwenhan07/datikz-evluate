To create a TikZ LaTeX diagram that accurately represents the described network, we can use circles for the interconnected parts and nodes with labels for the numbers. Here's a sample code that you can modify and customize further:

```latex
\documentclass[tikz]{standalone}
\usetikzlibrary{shapes.geometric, positioning}

\tikzset{
    node style/.style={
        circle,
        draw=black,
        fill=white,
        minimum size=1cm,
        font=\sffamily\bfseries,
        inner sep=0pt,
        outer sep=0pt
    },
    label style/.style={
        above left,
        font=\sffamily\small
    }
}

\begin{document}
\begin{tikzpicture}[node distance=2cm]
    % Nodes for the interconnected parts
    \node[node style] (circle1) {1};
    \node[node style, right=of circle1] (circle2) {2};

    % Nodes for the group of interconnected nodes
    \node[node style, below=of circle1] (group1-1) {3};
    \node[node style, right=of group1-1] (group1-2) {4};
    \node[node style, below=of group1-2] (group1-3) {5};

    % Nodes for the smaller nodes
    \node[node style, below=of group1-3] (small1) {0.3};
    \node[node style, right=of small1] (small2) {0.5};
    \node[node style, below=of small2] (small3) {0.7};

    % Connections between nodes
    \draw[-stealth] (circle1) -- (circle2);
    \draw[-stealth] (circle1) -- (group1-1);
    \draw[-stealth] (group1-1) -- (group1-2);
    \draw[-stealth] (group1-2) -- (group1-3);
    \draw[-stealth] (group1-3) -- (small1);
    \draw[-stealth] (small1) -- (small2);
    \draw[-stealth] (small2) -- (small3);

    % Labels for the connections
    \node[label style, above=of circle1.east] {Connection 1};
