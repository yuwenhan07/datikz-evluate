Certainly! Below is a TikZ LaTeX code snippet that captures the described quiver diagram with the specified features:

```latex
\documentclass[tikz,border=3mm]{standalone}
\usetikzlibrary{arrows.meta,shapes.geometric}

\tikzset{
    gauge/.style={draw,fill=cyan!50,circle,inner sep=1pt},
    hyper/.style={draw,->,thick},
    wavy/.style={draw,wave,dashed,thick},
    fermi/.style={draw,dashed,thick}
}

\begin{document}
\begin{tikzpicture}[node distance=2cm]
    % Nodes representing gauge nodes
    \node[gauge] (G1) at (0,0) {};
    \node[gauge] (G2) at (4,0) {};
    
    % Nodes representing vector multiplets
    \node[rectangle,draw,inner sep=2pt] (V1) at (-1,-1) {Vector};
    \node[rectangle,draw,inner sep=2pt] (V2) at (5,-1) {Vector};
    
    % Straight solid lines representing (4,4) hypermultiplets
    \draw[hyper] (G1) -- node[above,midway] {(4,4)} (V1);
    \draw[hyper] (G2) -- node[below,midway] {(4,4)} (V2);
    
    % Wavy lines representing (4,0) hypermultiplets
    \draw[wavy] (G1) -- node[left,midway] {(4,0)} (V1.west);
    \draw[wavy] (G2) -- node[right,midway] {(4,0)} (V2.east);
    
    % Dashed lines representing Fermi multiplets
    \draw[fermi] (G1) -- node[left,midway] {Fermi} (V1.west);
    \draw[fermi] (G2) -- node[right,midway] {Fermi} (V2.east);
    
    % Labels for the nodes
    \node[above left] at (G1.west) {AdS$_2$};
    \node[above right] at (G2.east) {NS-NS};
    \node[below left] at (G1.south