Certainly! Below is a TikZ LaTeX code snippet that generates a diagram representing the uniform grid you described. The boundaries, known values, and unknown values are clearly marked with different colors.

```latex
\documentclass{standalone}
\usepackage{tikz}

\begin{document}
\begin{tikzpicture}[scale=1.5]
    % Define constants
    \def\T{2} % Total time T
    \def\Smax{4} % Maximum value of S
    \def\nsteps{8} % Number of steps in the grid

    % Draw the horizontal axis (time)
    \draw[->] (-0.5, 0) -- (\nsteps+0.5, 0) node[right] {$q$};
    
    % Draw the vertical axis (space)
    \draw[->] (0, -0.5) -- (0, \Smax+0.5) node[above] {$S$};
    
    % Draw the boundaries
    \draw[dashed, thick, color=brown] (0, 0) -- (\nsteps+0.5, 0) node[midway, below] {S=0};
    \draw[dashed, thick, color=yellow] (0, \Smax) -- (\nsteps+0.5, \Smax) node[midway, above] {S=\Smax};
    \draw[dashed, thick, color=green] (0, 0) -- (0, \Smax+0.5) node[left, midway] {q=0};
    
    % Draw the grid points
    \foreach \i in {0,...,\nsteps}{
        \foreach \j in {0,...,\Smax}{
            \ifnum\i=0 \or \i=\nsteps \or \j=0 \or \j=\Smax
                \node[circle, draw, fill=red, inner sep=1pt] at (\i, \j) {};
            \else
                \node[circle, draw, fill=blue, inner sep=1pt] at (\i, \j) {};
            \fi
        }
    }
    
    % Add labels for clarity
    \node[below left] at (0, 0) {Start};
    \node[above right] at (\nsteps, \Smax) {End};
\