Certainly! Below is an example of a TikZ LaTeX code that generates a sequential decision-making framework with the specified conditions:

```latex
\documentclass{standalone}
\usepackage{tikz}
\usetikzlibrary{shapes.geometric, arrows.meta}

\tikzstyle{startstop} = [rectangle, rounded corners, minimum width=3cm, minimum height=1cm,text centered, draw=black, fill=red!30]
\tikzstyle{io} = [trapezium, trapezium left angle=70, trapezium right angle=110, minimum width=3cm, minimum height=1cm, text centered, draw=black, fill=blue!30]
\tikzstyle{process} = [rectangle, minimum width=3cm, minimum height=1cm, text centered, draw=black, fill=orange!30]
\tikzstyle{decision} = [diamond, minimum width=3cm, minimum height=1cm, text centered, draw=black, fill=green!30]
\tikzstyle{arrow} = [thick,->,>=stealth]

\begin{document}

\begin{tikzpicture}[node distance=2cm]

\node (start) [startstop] {Start};
\node (step1) [process, below of=start] {Step 1: Decision A};
\node (step2) [process, below of=step1] {Step 2: Decision B};

% Dashed line for "Does it rank well?" decision node
\node (rankDecision) [decision, dashed, below of=step2, yshift=-1cm] {Does it rank well?};

% Nodes for paths after "Does it rank well?"
\node (yesRank) [process, below of=rankDecision, yshift=-1cm] {Yes};
\node (noRank) [process, below of=yesRank, yshift=-1cm] {No};

% Nodes for paths after Step 2
\node (success) [process, below of=step2, yshift=-2cm] {Success};
\node (failure) [process, below of=success, yshift=-1cm] {Failure};

% Arrows between nodes
\draw [arrow] (start) -- (step1);
\draw [arrow] (step1) -- (step2);
