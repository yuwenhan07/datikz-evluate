To create a stream plot in TikZ with singularities, you can use the `stream plot` command from the `pgfplots` package. Here's an example code that produces a stream plot with three different equations, each with variables in different colors:
```
\documentclass{article}
\usepackage{tikz}
\usepackage{pgfplots}
\begin{document}
\begin{tikzpicture}
    \begin{axis}[
        axis equal,
        xmin=-2, xmax=2,
        ymin=-2, ymax=2,
        xlabel=$x$,
        ylabel=$y$,
        domain=-2:2,
        samples=50,
        ]
        % Red equation
        \addplot[red] {x^3 - y};
        % Tanh equation
        \addplot[tanh] {tanh(x)};
        % White equation
        \addplot[white] {1/(x^2 + 1)};
    \end{axis}
\end{tikzpicture}
\end{document}
```
In this code, we first load the necessary packages (`tikz` and `pgfplots`). Then, we create a `tikzpicture` environment and add an `axis` environment inside it. We set some options for the axis, such as the range of x and y values, labels, and sample density.
Next, we add three `addplot` commands to the `axis` environment, each representing one of the equations. We use the `red`, `tanh`, and `white` options to specify the color of each plot. The first equation is `x^3 - y`, which has a singularity at `(0,0)`. The second equation is `tanh(x)`, which is periodic and has no singularities. The third equation is `1/(x^2 + 1)`, which also has a singularity at `(0,0)`.
Finally, we close the `tikzpicture` environment and compile the document using a LaTeX editor like TeXShop or Overleaf. This will produce a stream plot with three different equations, each with variables in different colors and showing the behavior of the tanh function and its application with different variables.