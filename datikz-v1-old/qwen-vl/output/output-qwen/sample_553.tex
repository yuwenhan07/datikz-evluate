It sounds like you're describing a type of fractal or geometric pattern that could be referred to as a "max-mdim" graph, though this term isn't commonly used in standard mathematical or graphical contexts. Typically, such patterns might be described using terms like "Sierpinski triangle," "Menger sponge," or "Koch curve," depending on the specific arrangement and iteration.

Given your description of a white square divided into smaller squares by black lines, it seems you might be referring to a Sierpinski carpet, which is a fractal pattern created by recursively removing the central square from each remaining square in an infinite process. However, since you mentioned "max-mdim," it's possible you're referring to a more complex or higher-dimensional concept, but without further details, it's hard to specify exactly what you mean.

If you're looking for a description or analysis of the pattern you've described, here’s a general approach:

1. **Identify the Pattern**: The pattern you've described appears to be a grid-like structure where a larger square is divided into smaller squares, with some squares possibly being removed or shaded differently.

2. **Mathematical Description**: If this is indeed a Sierpinski carpet, the fractal dimension can be calculated using the formula \( D = \log(N) / \log(1/s) \), where \( N \) is the number of self-similar pieces, and \( s \) is the scaling factor (in this case, 3 because each large square is divided into 9 smaller squares).

3. **Visual Analysis**: You can analyze the pattern by counting the number of squares at different levels of recursion. For example, if you start with one large square and remove the center square, you have 8 smaller squares left. At the next level, each of these 8 squares will be divided into 9 smaller squares, and so on.

4. **Software Tools**: If you want to create or manipulate such patterns, you can use software tools like MATLAB, Python with libraries like NumPy and Matplotlib, or specialized fractal generation software.

If you have any specific questions about the pattern or need further assistance, feel free to ask!