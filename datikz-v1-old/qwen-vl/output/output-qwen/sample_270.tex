The NA62ResNet architecture you're referring to seems to be a specific implementation or variation of a ResNet (Residual Network) architecture, which is commonly used in deep learning for various tasks such as image classification and object detection. Here's a breakdown of the key components mentioned:

1. **Residual Blocks (ResBlock)**: These are the building blocks of the ResNet architecture. A ResBlock typically consists of two convolutional layers with batch normalization and ReLU activation functions. The output of the second convolutional layer is added to the input of the first convolutional layer, allowing the network to learn more complex features without vanishing gradients.

2. **Stacked in Pairs**: In this context, it suggests that the ResBlocks are arranged in pairs, meaning that after one ResBlock, another identical ResBlock follows. This stacking helps in increasing the depth of the network while maintaining the simplicity of the ResBlock structure.

3. **Stride S**: The stride \( S \) is a parameter of the filter in the convolutional layers. It determines how much the filter moves across the input feature map. A stride of 1 means the filter moves one pixel at a time, while a larger stride (e.g., 2) skips every other pixel, effectively reducing the spatial dimensions of the feature maps.

4. **Output Tensor Dimensions**: The output tensor dimensions at each step indicate the size of the feature maps produced by the network at different stages. For example, if the input image has dimensions \( H \times W \times C \), where \( H \) and \( W \) are the height and width of the image, and \( C \) is the number of channels, the output tensor dimensions would show how these dimensions change as the network processes the data through its layers.

### Example of Output Tensor Dimensions

Let's say we have an input image with dimensions \( 256 \times 256 \times 3 \) (height, width, and three color channels). If we use a ResNet-like architecture with a stride of 2 in the convolutional layers, the output dimensions might look like this:

- **Input**: \( 256 \times 256 \times 3 \)
- **After First Convolutional Layer (Stride = 2)**: \( 128 \times 128 \times 64 \)
- **After Second Convolutional Layer (Stride = 2)**: \( 64 \times 64 \times 128 \)
- **After Third Convolutional Layer (Stride = 2)**: \( 32 \times 32 \times 256 \)
- **After Fourth Convolutional Layer (Stride = 2)**: \( 16 \times 16 \times 512 \)
- **After Fifth Convolutional Layer (Stride = 2)**: \( 8 \times 8 \times 1024 \)
- **After Sixth Convolutional Layer (Stride = 2)**: \( 4 \times 4 \times 2048 \)

This pattern continues until the final output layer, which might have a smaller dimension depending on the task and the architecture design.

### Conclusion

The NA62ResNet architecture, as described, likely involves a series of ResBlocks with specific strides and output tensor dimensions. The stacking of ResBlocks in pairs and the use of strides help in capturing hierarchical features from the input data, making the network more powerful and efficient. The exact details of the architecture would depend on the specific implementation and the task it is designed to solve.