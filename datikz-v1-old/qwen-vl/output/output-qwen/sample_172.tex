The statement you've provided is related to the concept of "cheating" or "hiding" information in a game or task setting where Alice and Bob are interacting. Let's break down the key points:

1. **Sets \( S_j \)**: These represent the set of optimal first messages that Alice can send for each task \( j \).

2. **Disjoint Sets**: If the sets \( S_j \) and \( S_k \) are disjoint, it means there is no overlap between the optimal first messages for these two tasks. In other words, the optimal message for one task cannot be used as an optimal message for the other task.

3. **Hedging Cheating Attempts**: If Bob switches between two tasks with disjoint sets of optimal messages, Alice must change her strategy to avoid being detected as cheating. This is because she cannot use the same optimal message for both tasks without being caught.

4. **Nonempty Intersection**: If the sets \( S_j \) and \( S_k \) have a nonempty intersection, it means there exists at least one message that is optimal for both tasks. In this case, Alice can use the same message for both tasks without any issues, as the message is valid for both.

5. **Cheating Without Hedging**: If Bob switches between two tasks with a nonempty intersection, Alice does not need to change her strategy. She can continue using the same optimal message for both tasks, thus avoiding the need to hedge her cheating attempts.

To summarize:
- If the sets \( S_j \) and \( S_k \) are disjoint, Alice must hedge her cheating attempts when Bob switches between tasks \( j \) and \( k \).
- If the sets \( S_j \) and \( S_k \) have a nonempty intersection, Alice can cheat without hedging when Bob switches between tasks \( j \) and \( k \).

This analysis helps in understanding how the structure of the optimal messages for different tasks affects Alice's ability to cheat and how Bob's switching between tasks impacts Alice's strategy.