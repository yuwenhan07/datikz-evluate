It seems you're describing an illustration that uses arrows to visually represent the equivalence between two equations (Eqn.~(eqn:p1) and Eqn.~(eqn:p2)). Here's a textual description of how such an illustration might look:

---

**Illustration Description:**

The illustration features a white background with a variety of green and black arrows. These arrows vary in length and thickness, adding depth and complexity to the design. The arrows are scattered across the canvas, creating a sense of randomness and dynamism.

Key elements:
- **Arrows:** Green and black arrows are used to symbolize the relationship between Eqn.~(eqn:p1) and Eqn.~(eqn:p2).
  - Some arrows point from Eqn.~(eqn:p1) to Eqn.~(eqn:p2), indicating that Eqn.~(eqn:p1) implies or leads to Eqn.~(eqn:p2).
  - Other arrows point from Eqn.~(eqn:p2) to Eqn.~(eqn:p1), suggesting that Eqn.~(eqn:p2) also implies or leads to Eqn.~(eqn:p1).
- **Network-like Pattern:** The arrangement of the arrows forms a network-like structure, emphasizing the interconnectedness and equivalence between the two equations.
- **Visual Complexity:** The varying sizes and thicknesses of the arrows add visual interest and help guide the viewer’s eye through the illustration, highlighting the relationships between the equations.

---

This description should help in visualizing the illustration and understanding its purpose in demonstrating the equivalence between the two equations. If you need further customization or specific details, feel free to ask!