To solve this problem, we need to understand the geometric configuration described and then determine how to draw a larger angle outside two smaller angles.

First, let's identify the key elements:
- We have a white triangle on a black background.
- The triangle has a right angle at one of its vertices.
- The coordinates of the center of the triangle are (-1, 1).
- The triangle is quite large, covering most of the area within the black background.

Since the triangle is a right triangle, it has an angle of 90 degrees at the right angle vertex. Let's denote the vertices of the triangle as \(A\), \(B\), and \(C\) such that \(\angle BAC = 90^\circ\). Without loss of generality, assume \(A\) is at the right angle vertex, \(B\) is at the bottom-left corner, and \(C\) is at the top-right corner.

The task is to draw a larger angle outside two smaller angles. Since the triangle is a right triangle, the two smaller angles are complementary (i.e., they add up to 90 degrees). To draw a larger angle outside these two smaller angles, we can consider drawing an angle that is greater than 90 degrees but less than 180 degrees.

For simplicity, let's choose an angle of 120 degrees. This angle is larger than the two smaller angles (each being 45 degrees) and can be drawn outside the two smaller angles formed by the right angle vertex and the other two vertices of the triangle.

Here’s a step-by-step process to draw the larger angle:

1. Identify the right angle vertex of the triangle, which is point \(A\).
2. Draw a line from point \(A\) through a point that will form the larger angle. For example, if we choose a point 120 degrees counterclockwise from the line extending from \(A\) to \(B\), we can draw a line from \(A\) through this new point.
3. Repeat the process for the other smaller angle, ensuring the new angle is larger than the original two smaller angles.

The exact placement and size of the angles depend on the specific coordinates and the scale of the drawing. However, the general approach is to ensure that the new angle is larger than the two smaller angles and is drawn outside them.

Thus, the final answer is:

\[
\boxed{120^\circ}
\]