Based on your description, it seems you're referring to a visual representation of a mathematical concept involving geometry and topology, specifically the relationship between a vertex state and its corresponding tetrahedron within a dual simplicial complex.

In this context:

1. **Vertex State**: This could refer to a specific point or node in a graph or a simplicial complex. In the context of a dual simplicial complex, a vertex state might represent a particular configuration or assignment of values to vertices (nodes) that could be relevant for a proof or analysis.

2. **Tetrahedron in the Dual Simplicial Complex**: The dual simplicial complex is a concept where each face of the original simplicial complex becomes a vertex in the dual, and vice versa. A tetrahedron in the dual complex would correspond to a vertex in the original complex. The tetrahedron here likely represents a higher-dimensional structure that is derived from the vertex state in the original complex.

3. **Geometric Shapes**: The shapes in the image are probably visual representations of these concepts. The three different geometric shapes could symbolize different vertices or faces in the original simplicial complex, and their relative sizes might indicate the importance or significance of those elements in the proof or analysis.

4. **Coordinates**: The unique set of coordinates mentioned are likely used to specify the exact location or properties of the vertices and faces in the simplicial complex. These coordinates are crucial for defining the relationships and transformations involved in the mathematical proof.

Given the complexity of the concepts involved, it's important to have a solid understanding of simplicial complexes, dual complexes, and how they relate to vertex states and higher-dimensional structures like tetrahedra. If you need further clarification or assistance with a specific part of the proof or analysis, feel free to provide more details!