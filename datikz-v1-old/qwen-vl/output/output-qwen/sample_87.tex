It seems like you're describing a visual representation of a graph or a tree structure, possibly related to the study of isomorphism problems in graph theory. However, there's no specific mention of an "anti-isomorphism" problem. In graph theory, an isomorphism is a one-to-one correspondence between the vertices of two graphs that preserves adjacency, meaning that if two vertices are connected by an edge in one graph, their corresponding vertices must be connected by an edge in the other graph.

An anti-isomorphism, on the other hand, is a mapping between two graphs that reverses the direction of edges. If two vertices are connected by an edge in one graph, their corresponding vertices will not be connected by an edge in the other graph under an anti-isomorphism.

Given your description of a tree with red and blue dots, it sounds like you might be dealing with a specific type of graph where certain nodes have special properties (e.g., being at the end of branches). This could be relevant for studying isomorphisms or anti-isomorphisms, but without more context about the specific rules governing the placement of the red and blue dots, it's hard to provide a precise interpretation.

If you're looking to decode the (anti)isomorphism problem based on this image, you would need to define what constitutes an isomorphism or anti-isomorphism in the context of this particular tree. For example, you might need to specify whether the red and blue dots represent certain types of nodes (like leaves or internal nodes), or if they indicate some other property that affects the isomorphism status of the graph.

If you can provide more details about the rules or criteria used to determine whether two instances of this tree structure are isomorphic or anti-isomorphic, I'd be happy to help further!