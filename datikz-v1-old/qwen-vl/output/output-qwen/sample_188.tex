To partially color a node with a gradient of specific colors in an infinite sum of \(\frac{1}{2^n}\) sub parts, we can use a technique similar to the one used in creating a fractal or a color gradient in digital art software like GIMP. Here's a step-by-step explanation:

1. **Define the Node and Colors**: Let's say we have a node (or a point) that we want to partially color with two colors, say \(C_1\) and \(C_2\). We will use a gradient that smoothly transitions from \(C_1\) to \(C_2\).

2. **Determine the Gradient Function**: A simple way to create a gradient is to use a linear interpolation between the two colors. For a value \(x\) in the range \([0, 1]\), the interpolated color \(C(x)\) can be defined as:
   \[
   C(x) = (1-x)C_1 + xC_2
   \]
   where \(C_1\) and \(C_2\) are the RGB values of the two colors.

3. **Discretize the Gradient**: Since we are dealing with an infinite sum of \(\frac{1}{2^n}\) sub parts, we can discretize the gradient into a finite number of steps. For example, if we want to use 8 steps, then \(x\) will take on the values \(0, \frac{1}{8}, \frac{2}{8}, \ldots, \frac{7}{8}, 1\).

4. **Color the Node**: To color the node, we can assign each of these discrete values to a small region around the node. The size of these regions can be determined by the resolution of the image or the desired smoothness of the gradient. For simplicity, let's assume we are using a very fine resolution, so the regions are infinitesimally small.

5. **Sum the Contributions**: The contribution of each region to the final color of the node can be approximated by the average color of that region. If the node is located at position \(p\) and the regions are infinitesimally small, the color at \(p\) will be the weighted average of the colors at the boundaries of these regions, where the weights are the areas of the regions.

6. **Example Calculation**: Suppose we have a node at position \(p\) and we want to approximate its color using 8 steps of the gradient. The positions of the boundaries of these regions would be \(0, \frac{1}{8}, \frac{2}{8}, \ldots, \frac{7}{8}, 1\). The color at each boundary would be calculated using the gradient function. The final color at \(p\) would be the weighted average of these colors, where the weights are the areas of the regions.

Since the exact calculation depends on the specific positions and the resolution, the final answer for the color of the node would be a combination of the colors at the boundaries of the regions, weighted by their areas. However, without specific numerical values, we can't provide a precise numerical answer. The general approach is to use the gradient function to interpolate the colors and then sum the contributions of the regions to get the final color.

The final answer, in a general form, is:
\[
\boxed{\text{The color of the node is a weighted average of the colors at the boundaries of the regions, where the weights are the areas of the regions.}}
\]