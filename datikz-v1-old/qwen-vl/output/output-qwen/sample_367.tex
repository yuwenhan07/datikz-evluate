The statement you've made pertains to the moduli space of type IIB string theory, which is a fundamental concept in theoretical physics, particularly in string theory. Let's break down the key points:

1. **Moduli Space of Type IIB Theory**:
   - In string theory, the moduli space refers to the space of all possible configurations of the extra dimensions (beyond the usual three spatial and one time dimension) that can be compactified.
   - For type IIB string theory, the moduli space is given by \( \mathcal{M}_{IIB} = H / SL(2, \mathbb{Z}) \), where \( H \) is the Siegel upper half-space, and \( SL(2, \mathbb{Z}) \) is the special linear group over the integers.

2. **Siegel Upper Half-Space (\( H \))**:
   - The Siegel upper half-space \( H \) is a complex manifold that parametrizes the possible shapes of the compactified dimensions in type IIB string theory.
   - It is defined as the set of symmetric matrices with positive definite imaginary part.

3. **\( SL(2, \mathbb{Z}) \)**:
   - \( SL(2, \mathbb{Z}) \) is the group of 2x2 matrices with integer entries and determinant 1.
   - This group acts on the Siegel upper half-space \( H \) through fractional linear transformations, which are known as modular transformations.

4. **Moduli Space as a Quotient**:
   - The quotient \( H / SL(2, \mathbb{Z}) \) represents the moduli space of type IIB string theory. This quotient space captures the different ways the compactified dimensions can be shaped while respecting the symmetries encoded by \( SL(2, \mathbb{Z}) \).

5. **Infinite Distance Limit**:
   - The moduli space \( H / SL(2, \mathbb{Z}) \) has an infinite distance limit, which corresponds to the boundary of the Siegel upper half-space. This boundary is known as the "boundary at infinity" or "modular curve."
   - This boundary is significant because it describes the limits of the possible shapes of the compactified dimensions.

6. **Cusp Singularities**:
   - The moduli space \( H / SL(2, \mathbb{Z}) \) also has two cusp singularities. These singularities arise from the action of certain elements in \( SL(2, \mathbb{Z}) \) on the boundary of the Siegel upper half-space.
   - The cusp singularities are important because they correspond to specific types of compactifications that are not smooth but have a well-defined structure.

In summary, the moduli space of type IIB string theory is a quotient space \( H / SL(2, \mathbb{Z}) \), which has an infinite distance limit and two cusp singularities. These features are crucial for understanding the possible shapes and configurations of the compactified dimensions in this theory.