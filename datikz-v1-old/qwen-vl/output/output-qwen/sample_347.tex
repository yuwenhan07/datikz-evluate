It seems like you're referring to a visual representation of geometric shapes, but without the actual image or more specific details about the shapes, it's difficult to provide an accurate analysis. However, I can explain what "monotonic" and "non-monotonic" might mean in this context.

In mathematics and computer science, particularly in the study of algorithms and computational complexity, the terms "monotonic" and "non-monotonic" are often used to describe certain types of functions or processes:

- **Monotonic**: A function is monotonic if it is either entirely non-increasing or non-decreasing. In simpler terms, for a function \( f \), if \( x_1 \leq x_2 \), then \( f(x_1) \leq f(x_2) \). This means that as the input increases, the output either stays the same or increases, or decreases but never increases again.
  
- **Non-Monotonic**: A function is non-monotonic if it does not satisfy the conditions of being monotonic. This means that the function may increase and decrease at various points, or it may have regions where it is increasing and decreasing simultaneously.

If we were to apply these concepts to geometric shapes, we would need to define what "monotonic" and "non-monotonic" mean in the context of the shapes' properties. For example, if we consider the area of the shapes as a function of some parameter (like the side length of a square or the radius of a circle), we could say that a shape's area is monotonic if its area always increases or always decreases as the parameter changes, and non-monotonic if the area sometimes increases and sometimes decreases.

Without the actual image, it's hard to determine which shapes are monotonic and which are non-monotonic based on their properties. If you could provide more details about the shapes or the context in which they are presented, I'd be happy to help further!