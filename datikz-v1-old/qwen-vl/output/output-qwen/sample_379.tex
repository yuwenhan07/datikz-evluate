It seems like you're describing a process related to circuit design or topology, possibly in the context of quantum computing or electronic circuits. Let's break down each step and provide some context:

### Step (a): Removing Gates from U on a "Hole" within A

1. **Identify the Circuit and the Hole**: You start with a circuit \( U \) which is embedded within an annulus \( A \). The annulus \( A \) has two boundaries: an inner boundary and an outer boundary.
2. **Locate the Hole**: Within the annulus \( A \), there is a "hole." This hole could be a region where certain components or gates are removed, creating an open space.
3. **Remove Gates**: You remove gates from the region within this hole. After removing these gates, you obtain a new circuit \( U' \).

### Step (b): Deforming the Annulus A to A'

1. **Deformation Process**: You deform the annulus \( A \) into a new annulus \( A' \). This deformation is such that the boundary of the new annulus \( A' \) is only partially covered by the original circuit \( U' \).
2. **Partial Coverage**: The deformation ensures that the boundary of the new annulus \( A' \) does not completely enclose the entire circuit \( U' \). Instead, it leaves some part of the boundary uncovered.

### Step (c): Further Removing Gates in the Vicinity of A'

1. **Identify the Vicinity**: You now focus on the vicinity of the new annulus \( A' \). This vicinity includes the area around the boundary of \( A' \) and any additional regions that might be affected by the deformation.
2. **Remove Additional Gates**: You remove some of the gates in this vicinity. After this removal, you obtain a final circuit \( U'' \).

### Summary

- **Step (a)**: You start with a circuit \( U \) inside an annulus \( A \) and remove gates from a specific region (the "hole") within \( A \), resulting in a new circuit \( U' \).
- **Step (b)**: You deform the annulus \( A \) into a new annulus \( A' \) such that the boundary of \( A' \) is only partially covered by the circuit \( U' \).
- **Step (c)**: You further remove gates in the vicinity of the new annulus \( A' \), resulting in the final circuit \( U'' \).

This process could be used in various contexts, such as simplifying a circuit, optimizing its layout, or preparing it for further analysis or implementation. If you have more specific details about the context (e.g., quantum computing, electronic circuits, etc.), please provide them so I can give a more detailed explanation or answer your question more accurately.