The image you've described appears to be a visual representation of a mathematical discussion or exercise involving Riemann solvers, which are numerical methods used in computational fluid dynamics (CFD) to solve hyperbolic partial differential equations. The Riemann solver is particularly useful for handling discontinuities like shocks in the flow field.

Here's a breakdown of what the elements might represent:

1. **Large Triangle**: This could symbolize a domain or region of interest in the problem, often representing a computational cell in a grid-based simulation.
2. **Smaller Triangles**: These might represent subdomains within the larger triangle, each potentially having its own set of conditions or variables.
3. **Letters and Numbers**: These likely denote specific parameters, variables, or constants that are part of the equations being discussed. For example:
   - "a", "b", "c" could represent coefficients in the equations.
   - "d", "e", "f" might be related to initial or boundary conditions.
   - "g", "h", "i" could be state variables at different points in space or time.
   - "j", "k", "l" might be fluxes or other quantities derived from the equations.
   - "m", "n", "o" could be related to the solution process or specific steps in the algorithm.
   - "p", "q", "r" might be intermediate results or approximations.
   - "s", "t", "u" could be additional parameters or variables.
   - "v", "w", "x" might be final results or outputs from the solver.

### Riemann Solvers in Context

Riemann solvers are typically used to resolve the discontinuities that arise when solving hyperbolic conservation laws numerically. They are essential for accurately capturing shock waves, contact discontinuities, and other features in the flow field. The choice of Riemann solver can significantly affect the accuracy and stability of the numerical solution.

### Example Equations

For instance, consider a simple Riemann problem where the flow is governed by the Euler equations of gas dynamics. The Riemann problem involves finding the solution to the system of conservation laws at a discontinuity between two states \( \mathbf{U}_L \) and \( \mathbf{U}_R \). The Riemann solver computes the solution \( \mathbf{U} \) at the interface based on these states.

The equations might look something like this:
\[
\begin{cases}
\frac{\partial U}{\partial t} + \frac{\partial F(U)}{\partial x} = 0 \\
U(x,t) = 
\begin{cases}
\mathbf{U}_L & \text{if } x < 0 \\
\mathbf{U}_R & \text{if } x > 0
\end{cases}
\end{cases}
\]
where \( U \) represents the vector of conserved variables (mass, momentum, energy), \( F(U) \) is the flux function, and \( \mathbf{U}_L \) and \( \mathbf{U}_R \) are the left and right states at the discontinuity.

### Summary

The whiteboard seems to be part of a detailed explanation or demonstration of how to apply Riemann solvers in a computational setting. The large triangle could represent a computational domain, and the smaller triangles and letters/numbers could be part of the equations and variables involved in the numerical method. The goal is likely to illustrate how to handle discontinuities in the flow field using Riemann solvers, ensuring accurate and stable solutions.