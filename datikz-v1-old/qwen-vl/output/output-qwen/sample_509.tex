The notation \((u, v)\) typically represents an ordered pair in mathematics, often used in contexts such as vectors, functions, or coordinate systems. On the other hand, \(u + v\) is the sum of two quantities \(u\) and \(v\), which could be scalars, vectors, or functions depending on the context.

When comparing \((u, v)\) with \(u + v\):

1. **Ordered Pair vs Sum**: 
   - \((u, v)\) is an ordered pair, meaning the order of elements matters. For example, \((3, 4)\) is different from \((4, 3)\).
   - \(u + v\) is the sum of \(u\) and \(v\), which is a scalar or vector depending on the nature of \(u\) and \(v\). The order does not matter here; \(u + v\) is the same as \(v + u\).

2. **Context**:
   - In vector spaces, \((u, v)\) might represent a vector \(\vec{v} = u\hat{i} + v\hat{j}\) in a 2D plane.
   - In function composition, \((u, v)\) might represent a function that takes two arguments, like \(f(u, v)\).
   - In calculus, \(u + v\) could be part of a derivative or integral expression, such as \(\frac{d}{dx}(u + v)\) or \(\int (u + v) dx\).

3. **Notation**:
   - The notation \((u, v)\) is more general and can encompass a variety of mathematical objects, while \(u + v\) is specific to addition.

Given the image you described, if the equations include expressions like \((u, v)\), they likely involve ordered pairs or vectors. If the equations include \(u + v\), they are likely involving sums of quantities. The column format suggests that each equation might be part of a sequence or a set of related problems being worked out or studied.