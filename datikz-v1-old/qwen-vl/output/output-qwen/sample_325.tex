To solve this problem, we need to understand the relationship between the angles formed by the sides of the triangle and how they can be used to draw a larger angle outside these smaller angles.

Given:
- A white triangle on a black background.
- The triangle's sides are numbered from 0 to 2.
- The numbers are positioned at the intersection of the triangle's sides, forming a coordinate plane.

The task is to draw a larger angle outside two smaller angles formed by the sides of the triangle. However, since the problem does not provide specific numerical values for the angles or any specific geometric configuration, we will assume a general approach to solving such a problem.

Let's consider a simple example where the triangle is an equilateral triangle (since it is the simplest type of triangle). In an equilateral triangle, all angles are equal to 60 degrees. If we draw a line extending from one vertex of the triangle to a point outside the triangle, the angle formed will be greater than 60 degrees.

For instance, if we draw a line from vertex 0 to a point outside the triangle, the angle formed will be greater than 60 degrees. Similarly, if we draw a line from vertex 1 to a point outside the triangle, the angle formed will also be greater than 60 degrees. The exact measure of the larger angle will depend on the specific point chosen outside the triangle.

Since the problem does not provide specific coordinates or a diagram, we cannot determine the exact measure of the larger angle. However, the general approach would be to extend lines from the vertices of the triangle to points outside the triangle and measure the angles formed.

Therefore, the answer to the problem, assuming a general approach, is:

\boxed{\text{Draw lines from the vertices of the triangle to points outside the triangle and measure the angles formed.}}