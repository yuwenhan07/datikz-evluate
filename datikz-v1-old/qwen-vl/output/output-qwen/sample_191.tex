It looks like you're describing a cycle decomposition of a permutation or a graph, where cycles are represented as sequences of vertices connected by edges. Let's break down the information you've provided:

- \( c_1 \) is the first cycle, which is colored red.
- \( c_2 \) is the second cycle, which is colored blue.
- The sequence \( v_0 \, w_0 \, v_1 \, w_6 \, v_5 \) represents part of the cycle \( c_2 \).

To understand this better, let's assume we have a graph with vertices labeled \( v_0, v_1, v_5, w_0, w_6 \), and edges connecting these vertices in a specific order to form cycles.

### Example Cycle Decomposition

Let's consider a simple example where the graph has the following edges:
- \( v_0 \) is connected to \( w_0 \)
- \( w_0 \) is connected to \( v_1 \)
- \( v_1 \) is connected to \( w_6 \)
- \( w_6 \) is connected to \( v_5 \)
- \( v_5 \) is connected back to \( v_0 \)

This forms a single cycle that includes all the vertices mentioned: \( v_0, w_0, v_1, w_6, v_5, v_0 \). This cycle can be written as:
\[ (v_0 \, w_0 \, v_1 \, w_6 \, v_5) \]

In this case, there is only one cycle, so \( c_1 \) and \( c_2 \) would not be applicable since they are not distinct cycles. However, if we were to add another cycle, say:
- \( v_2 \) is connected to \( w_2 \)
- \( w_2 \) is connected to \( v_3 \)
- \( v_3 \) is connected to \( w_4 \)
- \( w_4 \) is connected to \( v_2 \)

This would form another cycle: \( (v_2 \, w_2 \, v_3 \, w_4 \, v_2) \).

So, the complete cycle decomposition would be:
\[ c_1 = (v_0 \, w_0 \, v_1 \, w_6 \, v_5) \]
\[ c_2 = (v_2 \, w_2 \, v_3 \, w_4 \, v_2) \]

Here, \( c_1 \) is the red cycle, and \( c_2 \) is the blue cycle.

If you meant something different or need further clarification, please provide more details!