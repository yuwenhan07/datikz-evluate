The scenario you've described seems to be an artistic or conceptual representation rather than a precise mathematical or physical model. In the context of 11-dimensional supergravity and its compactifications, the concept of "exotic internal geometries" often involves highly complex structures that are difficult to visualize directly.

In string theory and M-theory, compactification refers to the process of reducing the number of dimensions from 11 down to 4 (the dimensions we observe in our universe). This reduction is typically achieved by compactifying some of the extra dimensions into small, curled-up spaces known as "internal manifolds." These manifolds can have intricate topologies and geometries, which can significantly affect the physics of the resulting lower-dimensional theory.

The consistency of such compactifications indeed relies on the presence of certain types of branes, particularly non-perturbative branes, which are thought to be localized at specific points within these internal manifolds. These branes play crucial roles in stabilizing the geometry and ensuring that the resulting theory is consistent with the observed physics.

The whiteboard animation you mentioned, with its continuous line passing through a series of dots, could be a metaphorical way of representing the path of a brane or the flow of information through a network of branes. However, this is not a standard method used in theoretical physics to describe these concepts. Instead, mathematicians and physicists use advanced tools like differential geometry, algebraic topology, and string theory to analyze and construct these exotic geometries.

If you're interested in exploring this topic further, you might want to look into the following areas:

1. **Compactification Manifolds**: These are the internal spaces where the extra dimensions are curled up. Examples include Calabi-Yau manifolds, which are often used in string theory due to their rich geometric properties.
   
2. **Brane Dynamics**: Non-perturbative branes are often studied in the context of M-theory and its various dualities. Understanding how these branes interact and influence the compactified geometry is crucial for constructing consistent theories.

3. **String Theory and M-Theory**: These frameworks provide the theoretical framework for understanding the compactification of extra dimensions and the role of branes in stabilizing the geometry.

4. **Mathematical Tools**: Advanced mathematical techniques such as homotopy theory, cohomology, and K-theory are often employed to study the topological and geometric properties of these manifolds.

If you have any specific questions about these topics or need further clarification, feel free to ask!