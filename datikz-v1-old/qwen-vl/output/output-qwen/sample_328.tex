To accurately represent an arc with angles given in trigonometric format (in radians), we need to understand the context of the problem. Since you've described a sphere and a cylinder, it seems like you might be dealing with spherical geometry or a projection onto a plane.

Let's break down the steps:

1. **Identify the Center and Radius**: The sphere is centered at the origin \((0, 0, 0)\) with a radius of \(0.75\). This means any point on the surface of the sphere can be represented as:
   \[
   (\cos(\theta) \sin(\phi), \sin(\theta) \sin(\phi), \cos(\phi))
   \]
   where \(\theta\) is the azimuthal angle (longitude) and \(\phi\) is the polar angle (latitude).

2. **Angles in Trigonometric Format**: If you have angles in radians, they will directly correspond to the values of \(\theta\) and \(\phi\). For example, if you have an angle \(\alpha = \frac{\pi}{4}\) radians, this would be the value for \(\theta\) or \(\phi\).

3. **Projection onto a Plane**: If you want to project the arc onto a plane, you'll need to specify the plane. Common planes include the xy-plane, xz-plane, or yz-plane. For simplicity, let's assume you want to project onto the xy-plane.

4. **Arc Representation**: To represent an arc, you need to specify the start and end points of the arc. If you have an angle \(\alpha\) that defines the arc, you can use the parametric equations of the sphere to find the coordinates of the endpoints.

For example, if you want to draw an arc from \(\theta = 0\) to \(\theta = \frac{\pi}{4}\) with \(\phi = \frac{\pi}{4}\):

- At \(\theta = 0\):
  \[
  (\cos(0) \sin(\frac{\pi}{4}), \sin(0) \sin(\frac{\pi}{4}), \cos(\frac{\pi}{4})) = (0, 0, \frac{1}{\sqrt{2}})
  \]

- At \(\theta = \frac{\pi}{4}\):
  \[
  (\cos(\frac{\pi}{4}) \sin(\frac{\pi}{4}), \sin(\frac{\pi}{4}) \sin(\frac{\pi}{4}), \cos(\frac{\pi}{4})) = (\frac{1}{\sqrt{2}} \cdot \frac{1}{\sqrt{2}}, \frac{1}{\sqrt{2}} \cdot \frac{1}{\sqrt{2}}, \frac{1}{\sqrt{2}}) = (\frac{1}{2}, \frac{1}{2}, \frac{1}{\sqrt{2}})
  \]

So, the arc starts at \((0, 0, \frac{1}{\sqrt{2}})\) and ends at \((\frac{1}{2}, \frac{1}{2}, \frac{1}{\sqrt{2}})\).

If you provide specific angles or a more detailed description, I can help you with a more precise representation.