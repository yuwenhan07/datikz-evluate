The problem you're referring to, "Weighted-Leafless-Partial-Cover," seems to be a specific type of optimization or covering problem in graph theory or combinatorial geometry. However, the description provided does not align directly with this problem as it is typically defined. Instead, let's break down the given tetrahedron and discuss some possible interpretations.

### Tetrahedron Description:
- **Vertices**: There are 4 vertices.
- **Edges**: There are 6 edges connecting these vertices.
- **Faces**: There are 4 triangular faces (each face is either red or blue).

### Possible Interpretations:

1. **Coloring Problem**:
   - The tetrahedron has 4 faces colored either red or blue.
   - We need to determine if there is a way to cover all the faces using a minimum number of colors such that each face is covered by at least one color.

2. **Covering Problem**:
   - Each face can be considered a "leaf" in a tree-like structure where each leaf must be covered by a "branch" (color).
   - The goal is to find the minimum number of colors needed to cover all the faces.

3. **Weighted Covering Problem**:
   - If we assign weights to the faces based on their importance or cost, we might want to minimize the total weight while ensuring all faces are covered.

### Example Solution:

Let's assume the weights of the faces are as follows:
- Red face: 5 units
- Blue face at the top: 3 units
- Blue face on the side: 4 units
- Blue face at the bottom: 2 units

We need to cover all faces with the minimum total weight.

#### Step-by-Step Solution:

1. **Identify Faces**:
   - Red face: 5 units
   - Blue face at the top: 3 units
   - Blue face on the side: 4 units
   - Blue face at the bottom: 2 units

2. **Covering Strategy**:
   - Since the red face is the most expensive, we should try to cover it first.
   - The blue face at the top costs 3 units.
   - The blue face on the side costs 4 units.
   - The blue face at the bottom costs 2 units.

3. **Calculate Minimum Weight**:
   - If we use only one color (say red), the total weight would be 5 units.
   - If we use two colors (red and blue), we can cover the red face with red and the other three faces with blue.
     - Total weight = 5 (red) + 3 (blue at top) + 4 (blue on side) + 2 (blue at bottom) = 14 units.
   - If we use three colors, we can distribute the faces more efficiently but the minimum will still be 5 units (using just the red face).

Thus, the minimum total weight to cover all faces is **5 units**, achieved by using only the red face.

### Conclusion:
In this specific case, the minimum weighted partial cover involves using the red face alone, achieving a total weight of 5 units. This is a simplified interpretation of the problem, and the actual problem definition might require more complex constraints or objectives.