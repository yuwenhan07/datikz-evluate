To understand the figure and the statement provided, let's break down the information step-by-step:

### Figure~fig_rel5:
This figure likely represents a relationship or a diagram involving elements \( M_1 \), \( M_4 \), and \( X \). The expression \( A = M_1 + M_4 - X \) suggests that \( A \) is defined as the sum of \( M_1 \) and \( M_4 \) minus \( X \).

### Figure~fig_0-cochain_presentation:
This figure provides a cochain presentation for some space or complex. It seems to give us information about the cohomology groups \( H^2(BT^3) \) and possibly the values of certain elements at specific points.

### Key Points from the Statement:
1. **Equality at Specific Points:**
   - \( A(3) = A(6) = -x_2 \): This means that when we evaluate \( A \) at points 3 and 6, the result is \(-x_2\).
   
2. **Non-zero Values:**
   - \( A(1) \), \( A(2) \), \( A(4) \), \( A(5) \) might not be zero in \( H^2(BT^3) \): This indicates that these evaluations could potentially yield non-trivial elements in the second cohomology group of \( BT^3 \).

3. **Zero Values:**
   - \( _{3,6}(1) = _{3,6}(2) = _{3,6}(4) = _{3,6}(5) = 0 \): This suggests that there is a specific operation or evaluation (denoted by the subscript notation) that results in zero when applied to points 1, 2, 4, and 5 at both points 3 and 6.

### Interpretation:
The statement is essentially describing how the function \( A \) behaves at different points in the context of the cochain presentation. Specifically, it highlights that while \( A \) can take non-zero values at some points (like 1, 2, 4, and 5), it evaluates to \(-x_2\) at points 3 and 6. Additionally, there is a specific operation or evaluation that consistently yields zero at points 1, 2, 4, and 5 when evaluated at points 3 and 6.

### Conclusion:
The figure and the statement together provide a detailed description of the behavior of the function \( A \) across different points in the space \( BT^3 \). The key takeaway is that \( A \) has specific values at certain points, and there are operations that result in zero under certain conditions. This kind of information is crucial for understanding the topological properties of the space involved.