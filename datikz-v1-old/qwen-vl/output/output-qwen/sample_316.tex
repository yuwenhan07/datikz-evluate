It seems like you're describing a specific type of graph transformation or geometric operation, particularly related to rotating vertices in a graph where each vertex has a certain state and edges can be oriented in different ways. Let's break down the key points:

1. **Rotation Operation**: The RV (Rotation Vertex) move involves rotating a vertex around one of its incident edges. In this case, the vertex is rotated by \( \frac{2\pi}{3} \) radians (or 120 degrees) around the left edge in the direction indicated by the arrow.

2. **Vertex State Change**: After the rotation, the vertex changes to the opposite state. This implies that there might be a binary state associated with each vertex, such as "on" or "off," "active" or "inactive," etc., which flips from one to the other after the rotation.

3. **Edge Crossing**: The rotation also creates a crossing between two of the edges incident to the vertex. This means that before the rotation, these two edges did not cross, but after the rotation, they do.

4. **State Transition Diagrams**: The top row likely shows the forward transitions (from one state to another), and the bottom row shows the reverse transitions (from the opposite state back to the original state).

### Example Scenario

Let's consider a simple example to illustrate this:

- Suppose we have a vertex \( v \) with two incident edges \( e_1 \) and \( e_2 \). Initially, let's say \( v \) is in state "A".
- When we rotate \( v \) by \( \frac{2\pi}{3} \) around \( e_1 \), the vertex state changes to "B", and an edge crossing occurs between \( e_1 \) and \( e_2 \).
- If we rotate \( v \) by \( -\frac{2\pi}{3} \) (or \( \frac{4\pi}{3} \)) around \( e_1 \), the vertex state changes back to "A", and the crossing is resolved.

### Mathematical Representation

If we represent the states of the vertex as \( S = \{ A, B \} \), then:
- Before rotation: \( v \in S_A \)
- After rotation: \( v \in S_B \)

The rotation operation can be described as:
\[ v \rightarrow v' \]
where \( v' \) is the new state of \( v \) after the rotation, and it is the opposite of the original state.

### Conclusion

The RV move is a specific operation that rotates a vertex around one of its incident edges, changing its state and creating a crossing between two edges. The exact nature of the state transition and the crossing creation depends on the specific rules defined for your graph or system. The diagrams provided help visualize these transitions clearly.