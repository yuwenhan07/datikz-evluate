Archimedes, a renowned ancient Greek mathematician, physicist, engineer, inventor, and astronomer, made significant contributions to various fields including mathematics and physics. One of his notable achievements was the use of the method of exhaustion to find the area under curves and the volumes of solids of revolution. However, the specific trigonometric relation you're referring to might be related to his work on the quadrature of the parabola or his approximation of pi.

If we consider the context of a right-angled triangle, one of the fundamental relations Archimedes might have used is the Pythagorean theorem. This theorem states that in a right-angled triangle, the square of the length of the hypotenuse (the side opposite the right angle) is equal to the sum of the squares of the lengths of the other two sides. Mathematically, it can be expressed as:

\[ c^2 = a^2 + b^2 \]

where \( c \) is the length of the hypotenuse, and \( a \) and \( b \) are the lengths of the other two sides.

Another important trigonometric relation that Archimedes might have used is the tangent function. For a right-angled triangle with an angle \( \theta \), the tangent of \( \theta \) is defined as the ratio of the length of the opposite side to the length of the adjacent side:

\[ \tan(\theta) = \frac{\text{opposite}}{\text{adjacent}} \]

These relationships are foundational in trigonometry and were crucial for many of Archimedes' mathematical explorations.

Given the image description, if the white triangle represents a right-angled triangle, the equations surrounding it would likely involve these basic trigonometric identities or the Pythagorean theorem. If there are specific equations provided in the image, they could be used to solve for unknown sides or angles of the triangle based on the given information.

For example, if the lengths of two sides of the triangle are known, the third side can be calculated using the Pythagorean theorem. Similarly, if the measure of one of the non-right angles is known, the tangent function can be used to find the ratio of the opposite side to the adjacent side.

Without the actual equations from the image, it's difficult to provide more specific details. However, the general principles of right-angled triangles and their trigonometric properties align well with Archimedes' contributions to mathematics.