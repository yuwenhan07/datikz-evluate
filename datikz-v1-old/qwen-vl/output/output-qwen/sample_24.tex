It seems like you're describing a visual case involving mathematical conditions and a geometric figure. Let's break down the components:

1. **Mathematical Conditions**:
   - \( k > 0 \): This indicates that the variable \( k \) is positive.
   - \( y_1 < U_B \): This suggests that the value of \( y_1 \) is less than some upper bound \( U_B \).

2. **Geometric Figure**:
   - A white triangle with a blue line on it: This could be a diagram or a representation in a mathematical context, possibly indicating a specific relationship or inequality.

Given these elements, let's interpret them together:

- The condition \( k > 0 \) implies that \( k \) is a positive number, which might have implications for the behavior of functions or systems where \( k \) is involved.
- The condition \( y_1 < U_B \) suggests that \( y_1 \) is constrained to be less than an upper bound \( U_B \), which could be relevant in contexts such as optimization problems or constraints in a system.

The white triangle with a blue line on it could represent a specific region or set in a coordinate plane, where the blue line might indicate a boundary or a constraint. If this is part of a larger problem, the blue line could be a level set, a contour, or a threshold that \( y_1 \) must satisfy relative to other variables or parameters.

If you need further analysis or interpretation, please provide more details about the context or the specific problem you are trying to solve.