The illustration you've described seems to depict a numerical method commonly used in computational fluid dynamics (CFD) or other numerical simulations. Here’s a breakdown of what the elements might represent:

1. **Cell Structure**: The central part of the image likely represents a computational cell, which is a fundamental unit in numerical simulations. These cells are typically used to discretize the domain into smaller volumes.

2. **Blue Lines**: These lines could represent the boundaries of the cell. In CFD, these boundaries are crucial as they define the interface between adjacent cells and where fluxes need to be calculated. Numerical fluxes are used at these interfaces to update the values of variables like velocity, pressure, or temperature.

3. **Black Dots**: These dots could indicate specific points within the cell that are used for calculations. For example:
   - **Centroid**: The center of the cell, often used for averaging properties.
   - **Nodes**: Points on the boundary of the cell where boundary conditions are applied.
   - **Marker Points**: Points used for tracking the movement of fluid particles or for defining the location of certain physical phenomena.

4. **Numerical Fluxes**: The process of updating the values within an intermediate cell involves calculating numerical fluxes at the interfaces. These fluxes are computed based on the values at the neighboring cells and are used to ensure conservation of mass, momentum, and energy across the interfaces.

### Example of Numerical Flux Calculation

Consider a simple one-dimensional case where we have two cells \(i\) and \(i+1\). The numerical flux \(F_{i,i+1}\) at the interface between these two cells can be calculated using various methods such as:

- **Lax-Friedrichs Flux**: 
  \[
  F_{i,i+1} = \frac{1}{2} \left( u_i + u_{i+1} \right) \left( \rho_i + \rho_{i+1} \right)
  \]
  where \(u_i\) and \(u_{i+1}\) are the velocities, and \(\rho_i\) and \(\rho_{i+1}\) are the densities in cells \(i\) and \(i+1\).

- **Godunov Flux**:
  \[
  F_{i,i+1} = \begin{cases}
  F^L & \text{if } \lambda_L > 0 \\
  F^R & \text{if } \lambda_R < 0 \\
  \min(F^L, F^R) & \text{otherwise}
  \end{cases}
  \]
  where \(F^L\) and \(F^R\) are the left and right states, respectively, and \(\lambda_L\) and \(\lambda_R\) are the eigenvalues corresponding to the left and right states.

These fluxes are then used to update the values within the intermediate cell, ensuring that the simulation remains stable and accurate.

In summary, the blue lines represent the boundaries of the cell, and the black dots indicate specific points within the cell that are used for calculations. The numerical fluxes at these boundaries are crucial for updating the values within the cell and maintaining the conservation laws in the simulation.