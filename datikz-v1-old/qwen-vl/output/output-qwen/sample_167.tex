Your statement touches on an important aspect of theoretical physics, particularly at the interface between string theory and black hole physics. Let's break down the key points:

1. **Non-Perturbative Regime and Black Hole Formation/Evaporation:**
   - In the non-perturbative regime, the scattering processes are dominated by the formation and evaporation of black holes. This is because the gravitational interactions become strong enough that they cannot be described by perturbation theory.
   - The scattering amplitudes in this regime are exponentially suppressed due to the large number of degrees of freedom involved in the process.

2. **Perturbative Regime and String Theory:**
   - In the perturbative regime, where gravitational interactions are weak, scattering amplitudes can be calculated using string theory. String theory provides a framework for describing quantum gravity and the behavior of particles at very high energies or short distances.
   - The scattering amplitudes in string theory are often computed using techniques like Feynman diagrams, but these diagrams must be interpreted within the context of string theory.

3. **Matching Across the Boundary:**
   - The boundary you mentioned (I) likely refers to the boundary between the perturbative and non-perturbative regimes. This is a crucial point where the predictions from different theories must match.
   - Checking the consistency of the results from string theory in the perturbative regime with those from black hole physics in the non-perturbative regime is a non-trivial test of the consistency of these theories. This is known as the "black hole information paradox" and has been a significant area of research in theoretical physics.

4. **Significance of the Matching:**
   - The matching of the stringy result and the black hole result across the boundary is not just a technical exercise; it is a fundamental test of our understanding of quantum gravity and the nature of spacetime.
   - If the two descriptions agree, it suggests that we have a consistent picture of the universe at all scales, from the smallest to the largest. If they do not agree, it could indicate new physics beyond what we currently understand.

In summary, your statement highlights the importance of verifying the consistency of different theoretical frameworks at the boundary between perturbative and non-perturbative regimes. This is a critical step towards a unified theory of quantum gravity, which would reconcile general relativity with quantum mechanics.