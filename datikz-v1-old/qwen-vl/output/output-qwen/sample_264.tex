The description you've provided seems to be a visual representation rather than a mathematical problem. However, I can provide some general information about convex points on capillary hypersurfaces.

A capillary hypersurface is a surface that minimizes its area under the influence of surface tension, often resembling a soap film or a bubble. In the context of your image, if we consider the inner and outer spheres as part of a capillary system, the points where these spheres meet the plane (the "capillary line") could potentially be convex points.

In a capillary system, the surface tension causes the interface between two fluids to minimize its area. This results in a shape that is locally minimal, meaning it is a critical point for the area functional. Convexity in this context would refer to the curvature of the surface at those points.

For a capillary hypersurface, the points where the inner and outer spheres meet the plane (the capillary line) will generally have zero mean curvature, making them saddle points rather than convex points. However, if we consider the boundary conditions and the specific geometry of the setup, there might be points where the surface is locally convex.

To determine whether such a point exists, one would typically need to solve the equations governing the capillary surface, which involve the Laplace-Beltrami equation with appropriate boundary conditions. These equations are nonlinear and usually solved numerically.

If you're looking for a more specific analysis or a solution to a particular problem involving these spheres, please provide more details or clarify the specific question you have.