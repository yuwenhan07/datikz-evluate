It sounds like you're describing a visual representation of a graph, which is a fundamental structure used in mathematics and computer science to model pairwise relations between objects. In this context, the "fourth step of encryption" might refer to a specific method or process within an encryption algorithm that involves the manipulation of such a graph.

Encryption algorithms often use graphs to represent data structures or relationships between elements. For example:

1. **Key Generation**: This could involve creating a key based on the properties of the graph.
2. **Data Transformation**: The data being encrypted could be mapped onto the nodes and edges of the graph.
3. **Encryption Process**: This step might involve applying transformations to the graph's structure or its associated data.
4. **Final Encryption Step**: This could be the final transformation or encoding of the data into a form that is secure against unauthorized access.

Given your description of a complex network with nodes and arrows, it seems like the fourth step could involve some form of encoding or transformation of the graph itself. Here’s how it might work:

- **Graph Encoding**: Convert the graph into a format that can be easily encrypted. This could involve assigning numerical values to each node and edge, forming a sequence or matrix.
- **Transformation**: Apply a transformation function to the encoded graph. This could be a simple shift, a substitution cipher, or a more complex algorithm like a permutation or a hash function.
- **Final Encoding**: Encode the transformed graph into a secure format using a standard encryption algorithm (e.g., AES, RSA).

If you have a specific encryption algorithm or a particular type of graph in mind, providing more details would help in giving a more precise explanation of the fourth step.