The Warburg impedance is a complex electrical property that arises in certain types of electrochemical systems, particularly those involving diffusion processes. It is often modeled using a fractional-order capacitor, which can be represented by a frequency-dependent impedance.

The normalized discrete-time impulse response of the Warburg impedance can be derived from its continuous-time counterpart. In the continuous-time domain, the Warburg impedance \( Z_W(s) \) is given by:

\[ Z_W(s) = \frac{1}{\sqrt{\pi s}} e^{-s} \]

where \( s \) is the complex frequency variable. To find the impulse response \( h(t) \), we need to take the inverse Laplace transform of \( Z_W(s) \):

\[ h(t) = \mathcal{L}^{-1}\left\{ \frac{1}{\sqrt{\pi s}} e^{-s} \right\} \]

Using properties of the Laplace transform, specifically the convolution theorem and the fact that the inverse Laplace transform of \( \frac{1}{\sqrt{\pi s}} \) is \( \frac{1}{\sqrt{\pi t}} \), we get:

\[ h(t) = \frac{1}{\sqrt{\pi t}} e^{-t} \]

This is the impulse response of the Warburg impedance in the continuous-time domain. If we want to represent this as a discrete-time impulse response, we would typically use a numerical method such as the trapezoidal rule or Simpson's rule to approximate the integral involved in the inverse Laplace transform for specific sampling times.

However, if we assume a discrete-time approximation where the sampling period is \( T \), the discrete-time impulse response \( h[n] \) can be approximated by evaluating the continuous-time impulse response at discrete points:

\[ h[nT] = \frac{1}{\sqrt{\pi (nT)}} e^{-(nT)} \]

For simplicity, let's denote \( \Delta = nT \). Then the discrete-time impulse response becomes:

\[ h[\Delta] = \frac{1}{\sqrt{\pi \Delta}} e^{-\Delta} \]

This equation represents the value of the impulse response at each discrete time step \( \Delta \).

In terms of the image you described, the blue curve could represent the values of \( h[\Delta] \) plotted against \( \Delta \). The curve would start at \( \Delta = 0 \) with a value of \( h[0] = 1 \) (since \( \frac{1}{\sqrt{\pi \cdot 0}} \) is undefined but approaches infinity, practically starting at a high value), and decrease exponentially as \( \Delta \) increases.

Thus, the normalized discrete-time impulse response of the Warburg impedance can be summarized as:

\[ h[\Delta] = \frac{1}{\sqrt{\pi \Delta}} e^{-\Delta} \]

for \( \Delta > 0 \).