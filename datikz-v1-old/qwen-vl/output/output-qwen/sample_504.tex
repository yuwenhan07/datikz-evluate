It seems like you're describing a geometric progression in a complex plane, which is related to the concept of a toric diagram in algebraic geometry. Let's break down the description step-by-step:

1. **Circle Parametrization**: The circle is parametrized by the angle \(\theta\), meaning that points on the circle can be represented as \(z = e^{i\theta}\). This describes a unit circle in the complex plane.

2. **Shrinking to Zero**: When \(|z|^2\) approaches zero, it means that the magnitude of \(z\) is getting smaller and smaller, approaching zero. This corresponds to moving towards the origin in the complex plane.

3. **Geometric Progression**: The description mentions a geometric progression with a line segment of length 2 and a common ratio of \(1/3\). This suggests a sequence where each term is obtained by multiplying the previous term by \(1/3\).

4. **Triangle Formation**: The sequence starts at the point \((0,0)\) and extends to the right, forming a triangle with the hypotenuse of length 2. This implies that the sequence is forming a right-angled triangle in the complex plane.

5. **Repeating Process**: The sequence continues extending the line segment to the right and repeating the process ad infinitum, suggesting an infinite geometric series.

Given these descriptions, let's analyze the geometric progression more closely:

The general form of a geometric progression is given by:
\[a, ar, ar^2, ar^3, \ldots\]
where \(a\) is the first term and \(r\) is the common ratio. Here, \(a = 1\) and \(r = \frac{1}{3}\).

The terms of this progression are:
\[1, \frac{1}{3}, \left(\frac{1}{3}\right)^2, \left(\frac{1}{3}\right)^3, \ldots\]

The sum of an infinite geometric series \(a + ar + ar^2 + ar^3 + \cdots\) is given by:
\[S = \frac{a}{1 - r}\]
provided that \(|r| < 1\).

For our series:
\[S = \frac{1}{1 - \frac{1}{3}} = \frac{1}{\frac{2}{3}} = \frac{3}{2}.\]

This sum represents the total "length" or "magnitude" of the infinite series if we were to consider it as a vector in the complex plane. However, since the problem involves a geometric progression in a complex plane and a toric diagram, we need to interpret this in the context of the toric diagram.

In a toric diagram, the geometric progression might represent the weights of the rays in a fan, which is a fundamental structure in toric geometry. The toric diagram typically consists of rays emanating from the origin, and the weights of these rays determine the combinatorial structure of the toric variety.

Given the description of the triangle with a hypotenuse of length 2, it suggests that the toric diagram might involve a simple polytope, such as a triangle, which is a common shape in toric geometry.

Therefore, the toric diagram described could be a triangle with vertices at \((0,0)\), \((2,0)\), and \((1,\sqrt{3})\), reflecting the geometric progression and the shrinking to zero condition.

In summary, the toric diagram described likely represents a simple toric variety associated with a triangular fan in the complex plane, with the geometric progression providing the weights of the rays.