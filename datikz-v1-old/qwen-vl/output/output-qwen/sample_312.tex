It seems like you're describing a geometric progression involving shapes (triangles and squares) and their side lengths. Let's break down the information provided:

1. **Geometric Progression Formula**: This refers to a sequence where each term after the first is found by multiplying the previous one by a fixed, non-zero number called the common ratio.
   
2. **Shapes Involved**: Triangles and squares are mentioned.

3. **Side Lengths**:
   - The first triangle has sides of 1, 1, and 2.
   - The squares have side lengths of 2, 3, 4, and 5.

Given these details, let's analyze the progression further:

### Step-by-Step Analysis

#### Triangle Progression
The side lengths of the triangle are given as 1, 1, and 2. However, for a triangle, all three sides must be positive numbers, and they must satisfy the triangle inequality theorem (the sum of any two sides must be greater than the third side). In this case, the triangle inequality theorem is not satisfied because \(1 + 1 = 2\), which means the sides do not form a valid triangle. Therefore, we cannot proceed with a geometric progression based on this triangle.

#### Square Progression
The side lengths of the squares are given as 2, 3, 4, and 5. These values can be used to form a geometric progression if we consider the side lengths themselves or their squares.

If we consider the side lengths directly:
- The progression is: 2, 3, 4, 5.

This does not form a geometric progression since there is no common ratio between consecutive terms. For example, the ratio between the second and the first term is \(\frac{3}{2}\), but the ratio between the third and the second term is \(\frac{4}{3}\), and so on.

If we consider the squares of the side lengths:
- The progression is: \(2^2 = 4\), \(3^2 = 9\), \(4^2 = 16\), \(5^2 = 25\).

This forms a geometric progression with a common ratio of \(\frac{9}{4} = \frac{16}{9}\).

### Conclusion
The progression described involves squares of side lengths 2, 3, 4, and 5, forming a geometric progression with a common ratio of \(\frac{9}{4} = \frac{16}{9}\).

If you need further clarification or have additional questions about geometric progressions or specific shapes, feel free to ask!