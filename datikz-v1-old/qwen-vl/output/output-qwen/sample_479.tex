It sounds like you're describing a Mealy machine that models a prioritized arbiter system, where different requests (r_0 and r_1) can be pending at various states, and the order of granting these requests matters based on their priority.

Let's break down the states and transitions:

### States:
- **q0**: No pending requests.
- **q1**: Request \( r_0 \) is pending.
- **q2**: Request \( r_1 \) is pending.
- **q3**: Both requests \( r_0 \) and \( r_1 \) are pending, and \( r_0 \) has higher priority; thus, \( r_0 \) is granted first.
- **q4**: Both requests \( r_0 \) and \( r_1 \) are pending, and \( r_1 \) has higher priority; thus, \( r_1 \) is granted first.

### Transitions:
1. **From q0 to q1**: Granting \( r_0 \) results in moving from q0 to q1.
2. **From q0 to q2**: Granting \( r_1 \) results in moving from q0 to q2.
3. **From q1 to q3**: Granting \( r_1 \) results in moving from q1 to q3.
4. **From q2 to q4**: Granting \( r_0 \) results in moving from q2 to q4.
5. **From q3 to q4**: Granting \( r_1 \) results in moving from q3 to q4.
6. **From q4 to q3**: Granting \( r_0 \) results in moving from q4 to q3.

### Example Transitions:
- If the system starts in q0 and receives request \( r_0 \), it moves to q1.
- If the system is in q1 and receives request \( r_1 \), it moves to q3.
- If the system is in q2 and receives request \( r_0 \), it moves to q4.
- If the system is in q4 and receives request \( r_0 \), it moves back to q3.

This Mealy machine ensures that the requests are granted according to their priority, with \( r_0 \) having higher priority than \( r_1 \) when both are pending. The machine's behavior reflects the prioritization logic required for a prioritized arbiter system.