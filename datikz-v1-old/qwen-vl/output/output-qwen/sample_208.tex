The image you've described appears to be a Penrose diagram, which is a graphical representation used in general relativity to depict the causal structure of spacetime. Here's a breakdown of the elements you mentioned:

1. **Penrose Diagram**: This is a two-dimensional representation of a four-dimensional spacetime. It shows the causal relationships between events in spacetime, where the vertical axis represents time and the horizontal axis represents spatial coordinates.

2. **Dotted Lines**: These likely represent the Penrose diagram for a de Sitter space at zero temperature. De Sitter space is a solution to Einstein's field equations that describes an expanding universe with positive cosmological constant. At zero temperature, the Penrose diagram would show the causal structure of this spacetime without any singularities or horizons.

3. **Black Dot (Apparent Horizon)**: In the context of a black hole, the apparent horizon is the boundary beyond which light cannot escape to infinity. For a black hole in de Sitter space, the apparent horizon is typically depicted as a closed curve in the Penrose diagram.

4. **Red Dots (Cosmological Horizons)**: These red dots represent the cosmological horizons in the de Sitter space. In de Sitter space, there are two cosmological horizons: one on each side of the central region. These horizons mark the boundaries of the observable universe in the context of de Sitter space.

In summary, the image seems to compare the causal structure of de Sitter space at zero temperature (dotted lines) with the causal structure of de Sitter space at finite temperature (solid lines). The solid lines include the apparent horizon of the black hole and the cosmological horizons, which are not present in the zero-temperature case. This comparison helps illustrate how the presence of a black hole and its thermodynamic properties affect the global structure of spacetime in de Sitter space.