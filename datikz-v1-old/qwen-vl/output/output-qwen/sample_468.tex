The description you've provided seems to be referring to a specific type of mathematical structure or graph known as a "Hansen graph," but with a twist that makes it particularly interesting. Typically, a Hansen graph is a type of graph used in combinatorial optimization problems, often related to shortest paths or network flows.

However, your mention of the set of non-dominated paths being exponential in \( n \) suggests a more complex scenario, possibly involving a modified version of the Hansen graph or another type of graph where the complexity of finding such paths increases exponentially with the size of the graph (denoted by \( n \)).

Given the structure described—where the equations are of the form \( C_{n+1} = C_n + 1 \), starting from some initial constant \( C_1 \)—this could be interpreted as a sequence or a series of steps in a process. If we visualize this as a tree, each node represents a step in the sequence, and the branches represent different values of \( n \).

Here's how we can interpret this:

1. **Initial Condition**: \( C_1 \) is a constant.
2. **Recursive Relation**: Each subsequent term in the sequence is obtained by adding 1 to the previous term: \( C_{n+1} = C_n + 1 \).

This recursive relation implies that the value of \( C_n \) increases linearly with \( n \). For example:
- \( C_1 = C_1 \)
- \( C_2 = C_1 + 1 \)
- \( C_3 = C_2 + 1 = C_1 + 2 \)
- \( C_4 = C_3 + 1 = C_1 + 3 \)

And so on. This sequence grows linearly with \( n \), not exponentially.

If the structure you're describing is indeed a tree with these equations, then the number of paths from the root (initial condition) to any leaf (final term) would simply be \( n \), since each step adds one to the previous term. This is a polynomial growth pattern, not exponential.

Therefore, if the structure is indeed a tree with the given equations, the number of non-dominated paths between \( C_1 \) and \( C_n \) would be \( n \), not exponential in \( n \).

If there's something specific about the problem or the structure that I'm missing, please provide more details, and I'll be happy to help further!