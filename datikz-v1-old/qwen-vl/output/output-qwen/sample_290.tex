F(T) gravity, also known as teleparallel gravity, is an alternative theory of gravity that modifies general relativity by replacing the Riemann curvature tensor with the torsion tensor. This theory is based on the idea that the gravitational field can be described using a connection that has non-zero torsion but no curvature.

In F(T) gravity, the action is constructed from a function \( f(T) \), where \( T \) is the torsion scalar. The torsion scalar is defined as the trace of the torsion tensor, which measures the difference between the Levi-Civita connection and the affine connection used in the theory. The action for F(T) gravity is given by:

\[
S = \int d^4x \sqrt{-g} \left( f(T) + L_m \right)
\]

where \( g \) is the determinant of the metric tensor, \( f(T) \) is the function of the torsion scalar, and \( L_m \) is the matter Lagrangian.

The choice of the function \( f(T) \) can lead to different cosmological models and sequences. Different choices of \( f(T) \) can result in various cosmological behaviors such as accelerated expansion, decelerated expansion, or even oscillatory behavior. Here are some examples of different cosmological sequences that can be realized under different choices of \( f(T) \):

1. **Accelerated Expansion**: For certain choices of \( f(T) \), the theory can predict an accelerated expansion of the universe, similar to what is observed in the late-time universe. This is often achieved when \( f(T) \) is a function that grows faster than linearly with the torsion scalar.

2. **Decelerated Expansion**: On the other hand, there exist choices of \( f(T) \) that can lead to a decelerated expansion phase, where the expansion rate of the universe decreases over time. This can be relevant for early-time cosmology.

3. **Oscillatory Behavior**: Some models of F(T) gravity can exhibit oscillatory behavior in the expansion history of the universe. This can occur if the function \( f(T) \) is chosen in a way that introduces periodic terms in the equations of motion.

### Model Examples

1. **F(R) Gravity**:
   - **Model**: \( f(T) = R \)
   - **Cosmological Sequence**: This model reduces to general relativity when the torsion scalar \( T \) vanishes. It can describe both accelerating and decelerating phases depending on the initial conditions and the specific form of the metric.

2. **F(T) Gravity with a Linear Term**:
   - **Model**: \( f(T) = T + \alpha \)
   - **Cosmological Sequence**: This model can lead to an accelerated expansion phase if \( \alpha > 0 \). The parameter \( \alpha \) controls the strength of the acceleration.

3. **F(T) Gravity with a Polynomial Term**:
   - **Model**: \( f(T) = T^n \)
   - **Cosmological Sequence**: For \( n > 1 \), this model can exhibit accelerated expansion. The value of \( n \) determines the rate of acceleration.

4. **F(T) Gravity with a Logarithmic Term**:
   - **Model**: \( f(T) = \ln(T) \)
   - **Cosmological Sequence**: This model can lead to a deceleration phase followed by an acceleration phase, mimicking the cosmic evolution seen in the standard model of cosmology.

5. **F(T) Gravity with a Nonlinear Term**:
   - **Model**: \( f(T) = T + \beta T^2 \)
   - **Cosmological Sequence**: This model can exhibit a transition from deceleration to acceleration, depending on the values of \( \beta \).

The choice of the Lagrangian \( f(T) \) thus plays a crucial role in determining the cosmological behavior of the model. By varying the functional form of \( f(T) \), one can explore different cosmological sequences and understand the potential of F(T) gravity to describe the evolution of the universe.