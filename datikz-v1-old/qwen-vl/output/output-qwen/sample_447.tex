The image you've described seems to depict a visual representation of a compact capillary horosphere, which is a concept from differential geometry. A capillary horosphere is a surface that is tangent to a given surface (in this case, the sphere) along its boundary and has a constant mean curvature. In simpler terms, it's a surface that looks like it's "hugging" the boundary of another surface while maintaining a uniform curvature.

In your description:
- The white sphere represents the original surface.
- The green lines represent the intersections or tangents where the capillary horosphere touches the sphere. These lines could be thought of as the "contact points" or "tangency points" between the two surfaces.

The term "compact" refers to the fact that the capillary horosphere is a closed surface, meaning it doesn't extend infinitely but rather forms a bounded shape.

This kind of visualization is often used in mathematical physics and geometry to study the behavior of surfaces under certain constraints, such as minimal surfaces or surfaces with constant mean curvature. It can also be related to concepts in fluid dynamics, where the capillary horosphere might represent the interface between two fluids under specific conditions.

If you're looking for a more precise mathematical description or further exploration of these concepts, you might want to delve into differential geometry textbooks or research papers on minimal surfaces and capillary surfaces.