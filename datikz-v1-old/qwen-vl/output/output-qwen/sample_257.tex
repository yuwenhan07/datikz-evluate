It seems like you're describing an image that includes a white arrow pointing to the right, surrounded by blue and red dots of varying sizes and distances. However, without the actual image or further context, I can't provide specific details about the sequences of Vietoris-Rips complexes mentioned in your question.

The Vietoris-Rips complex is a concept used in topological data analysis (TDA) to construct a simplicial complex from a set of points in a metric space. It's defined as follows:

1. **Vertices**: The points in the dataset.
2. **Edges**: Formed by pairs of points that are within a certain distance \( \epsilon \).
3. **Higher-dimensional simplices**: Formed by sets of points where all pairwise distances between them are less than or equal to \( \epsilon \).

Given this definition, if you have two sequences of Vietoris-Rips complexes, they would typically be constructed for different values of \( \epsilon \), starting from a small value and increasing step-by-step. Each sequence would represent a filtration of the complex, where the complexes grow as \( \epsilon \) increases.

If you could provide more details about the sequences or the specific context in which these complexes are being discussed, I might be able to offer more precise information or insights. For now, here’s a general idea of how such sequences might look:

- **Sequence 1**: Constructed for \( \epsilon_1 < \epsilon_2 < \epsilon_3 < \ldots \)
- **Sequence 2**: Constructed for \( \epsilon'_1 < \epsilon'_2 < \epsilon'_3 < \ldots \)

Each sequence would show how the Vietoris-Rips complex evolves as the threshold \( \epsilon \) increases, potentially revealing different topological features of the underlying point cloud.

If you need help with a specific aspect of constructing or analyzing these complexes, feel free to ask!