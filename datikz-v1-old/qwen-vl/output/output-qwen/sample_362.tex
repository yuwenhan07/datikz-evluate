The scenario you described involves a waveform that undergoes a change in frequency, followed by a return to its original frequency but with a synchronization call applied. This situation can lead to a discontinuous phase jump at the point where the frequency returns to its original value.

Let's break down the process step-by-step:

1. **Initial Frequency and Phase**: The waveform starts at a specific frequency \( f_1 \) and has an initial phase \( \phi_1 \).

2. **Frequency Change**: The frequency is changed to \( f_2 \). During this period, the waveform will have a new phase \( \phi_2 \), which is determined by the frequency \( f_2 \) and the time elapsed since the start of the waveform.

3. **Return to Original Frequency**: After some time, the frequency is returned to \( f_1 \). However, due to the change in frequency, the phase of the waveform at this point will be different from the initial phase \( \phi_1 \). Let's denote the phase at the moment the frequency changes back to \( f_1 \) as \( \phi_{\text{change}} \).

4. **Synchronization Call**: A synchronization call is applied to bring the waveform back into alignment with the original phase \( \phi_1 \). This synchronization call effectively resets or adjusts the phase of the waveform so that it aligns with the original phase \( \phi_1 \).

5. **Discontinuous Phase Jump**: When the frequency returns to \( f_1 \) and the synchronization call is applied, the phase of the waveform will not immediately match the original phase \( \phi_1 \). Instead, there will be a discontinuity in the phase, often referred to as a phase jump. This phase jump is the difference between the phase at the moment the frequency changes back to \( f_1 \) (\( \phi_{\text{change}} \)) and the original phase \( \phi_1 \).

Mathematically, if we denote the phase at the moment the frequency changes back to \( f_1 \) as \( \phi_{\text{change}} \), then the phase jump \( \Delta \phi \) can be expressed as:
\[ \Delta \phi = \phi_{\text{change}} - \phi_1 \]

This phase jump is what causes the discontinuity in the waveform when the frequency returns to the original value and the synchronization call is applied. It is important to note that this phase jump is a result of the non-linear behavior of the waveform during the frequency change and the subsequent synchronization.

In summary, the waveform experiences a discontinuous phase jump when the frequency returns to the original value and a synchronization call is applied, due to the difference between the phase at the moment the frequency changes back to the original value and the original phase.