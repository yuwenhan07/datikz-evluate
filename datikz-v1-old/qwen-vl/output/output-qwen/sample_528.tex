The description you've provided sounds like it could be referring to a visual representation of a dependency graph, where the central circle represents a primary entity or concept, and the surrounding circles represent dependent entities or concepts that rely on the central one. In a dependency graph, the size of the circles often correlates with the importance or complexity of the entities they represent.

Here's how this might look:

1. **Central Circle**: This represents the primary entity or concept (e.g., "Project A").
2. **Surrounding Circles**: These represent dependent entities or concepts (e.g., "Task 1", "Task 2", etc.) that depend on the central entity.

In your case:
- The central circle is smaller, indicating it might be less complex or less important compared to the surrounding circles.
- The six surrounding circles are of varying sizes, suggesting different levels of importance or complexity for each task or dependency.

This kind of diagram is commonly used in project management, software development, and other fields to visualize relationships and dependencies between components. It helps in understanding the structure and flow of work, identifying critical paths, and managing resources effectively.

If you have any specific questions about interpreting or creating such a dependency graph, feel free to ask!