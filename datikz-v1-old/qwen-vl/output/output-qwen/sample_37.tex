The graph you've described appears to be a scatter plot where the vertical axis represents accuracy values, and the horizontal axis could represent some other variable such as model type, setting parameter, or another metric that influences the accuracy of the voice recognition models.

To compare the accuracy of different voice recognition models or settings based on this graph:

1. **Identify the Data Points**: Each point on the graph represents a specific combination of the horizontal variable (e.g., model type or setting) and its corresponding accuracy value on the vertical axis.

2. **Compare Vertical Positions**: The vertical position of each data point indicates the accuracy level. Higher points on the vertical axis generally indicate higher accuracy.

3. **Horizontal Comparison**: If the horizontal axis represents different models or settings, you can compare the accuracy across these categories. For example, if there are two models labeled "Model A" and "Model B," you would look at their respective data points to see which one has higher accuracy.

4. **Trend Analysis**: Look for any trends in the data points. Are there any patterns or clusters that suggest certain models or settings perform better under specific conditions?

5. **Outliers**: Check for any outliers—data points that stand out from the general trend. These might indicate unusual performance or errors in the data.

6. **Legend and Labels**: Ensure you understand what the blue and white colors represent. If they are used to differentiate between different models or settings, make sure to refer to the legend or labels provided in the graph.

7. **Statistical Measures**: Depending on the context, you might want to calculate additional statistical measures like mean accuracy, median accuracy, standard deviation, etc., to get a more comprehensive view of the data.

By following these steps, you can effectively compare the accuracy of different voice recognition models or settings based on the information presented in the graph.