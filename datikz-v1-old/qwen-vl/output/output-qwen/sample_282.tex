To eliminate the blue dot at the 0,0 coordinate in your clustered graph with a separated line graph above the individual bars, you can follow these steps:

### Step-by-Step Guide

1. **Identify the Graph**: Ensure that you have identified which graph contains the blue dot at the 0,0 coordinate. This could be one of the four graphs mentioned.

2. **Access the Data**: Open the graph in a software tool like Excel, Google Sheets, or any other data visualization tool where you can edit the graph.

3. **Locate the Blue Dot**: In the clustered bar graph, locate the blue dot at the 0,0 coordinate. This point might represent an outlier or a specific data point that you want to remove.

4. **Edit the Data**:
   - **In Excel or Google Sheets**: 
     - Select the data range that includes the blue dot.
     - Right-click on the selected data and choose "Delete" or "Remove Data Point."
   - **In Other Tools**:
     - Identify the data series that contains the blue dot.
     - Edit the data series to exclude the unwanted point.

5. **Update the Graph**:
   - After removing the data point, the graph should automatically update to reflect the changes.
   - If the graph does not update automatically, you may need to refresh the graph or reapply the chart settings.

6. **Verify the Changes**:
   - Check if the blue dot has been successfully removed from the graph.
   - Ensure that the remaining data points are correctly displayed without the blue dot.

7. **Save Your Work**: Save the changes to ensure they are not lost.

### Example in Excel

If you're using Excel, here's how you can remove the blue dot:

1. **Select the Data**:
   - Click on the clustered bar graph to select it.
   - Click on the blue dot at the 0,0 coordinate to select it specifically.

2. **Delete the Data Point**:
   - Press `Delete` on your keyboard to remove the data point.

3. **Check the Graph**:
   - The blue dot should now be gone, and the graph should adjust accordingly.

By following these steps, you should be able to remove the blue dot at the 0,0 coordinate in your clustered graph.