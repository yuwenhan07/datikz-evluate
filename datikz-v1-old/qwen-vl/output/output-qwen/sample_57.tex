The description you've provided seems to be referring to a fractal tiling pattern known as the Sierpinski carpet or a similar type of fractal tiling that involves equilateral triangles. However, it's important to clarify that the term "k-supertile" is not standard in the context of fractal geometry or tiling patterns.

In traditional fractal geometry, a supertile refers to a larger tile that is made up of smaller tiles, which is a common feature in fractal constructions like the Sierpinski triangle or the Menger sponge. For example, in the Sierpinski triangle, each large triangle (the supertile) is divided into three smaller equilateral triangles.

If we consider the Sierpinski triangle as an example, here’s how the positions within the supertiles might look for \( k = 0 \) through \( k = 3 \):

1. **\( k = 0 \)**: This would represent the initial stage where there is only one large equilateral triangle.
2. **\( k = 1 \)**: This would show the first iteration where the large triangle is divided into four smaller triangles, but the central one is removed, leaving three smaller triangles.
3. **\( k = 2 \)**: In this stage, each of the remaining three triangles from the previous step is further divided into four smaller triangles, and again, the central one is removed from each, resulting in nine even smaller triangles.
4. **\( k = 3 \)**: Continuing the process, each of the nine triangles from the previous step is divided into four smaller triangles, and the central one is removed, leading to 27 even smaller triangles.

Each of these stages can be visualized by recursively dividing the triangles into smaller ones and removing the central one at each level. The pattern continues infinitely, creating a fractal structure.

If you have a specific image or pattern in mind that you want to analyze, please provide more details so I can assist you better!