The equations you mentioned seem to be related to quantum mechanics or algebraic structures, as they involve complex conjugates and braiding operations. However, without seeing the actual equations, I can provide some general information about these concepts.

1. **Complex Conjugate**: In mathematics, the complex conjugate of a complex number \( z = a + bi \) (where \( a \) and \( b \) are real numbers, and \( i \) is the imaginary unit) is given by \( \overline{z} = a - bi \). The complex conjugate has several important properties, such as:
   - \( \overline{\overline{z}} = z \)
   - \( |z|^2 = z \cdot \overline{z} \)

2. **Opposite Braiding**: In the context of braided monoidal categories, which are used in advanced areas of mathematics like knot theory and quantum groups, "braiding" refers to a specific operation that exchanges objects in a tensor product. The "opposite braiding" would be the braiding operation where the order of the objects is reversed. For example, if \( x \otimes y \) is the usual tensor product, then the opposite braiding might be denoted as \( y \otimes x \).

If you could provide the specific equations from the whiteboard, I could offer more detailed explanations or help solve them based on the context provided.