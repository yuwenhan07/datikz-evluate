It seems like you're describing two different visualizations of a pentagon, each with a unique approach to representing it. Let's break down what you've mentioned:

1. **Black and White Picture with Binary Numbers:**
   - This depiction uses a monochromatic (black and white) scheme.
   - Two vertices of the pentagon are labeled with the numbers 0 and 1.
   - This artistic choice might be intended to highlight certain symmetries or properties of the pentagon in a non-traditional way.

2. **Colorful Version with Binary Numbers:**
   - This is a more conventional and colorful representation of the pentagon.
   - Again, two vertices are labeled with the numbers 0 and 1.
   - This version is likely meant to provide a more familiar and visually appealing view of the pentagon.

### Key Points:
- Both images are essentially the same pentagon but presented differently.
- The use of binary numbers (0 and 1) on the vertices could be highlighting specific vertices for analysis or demonstration purposes.
- The black and white image might be used to emphasize certain mathematical or geometric properties, whereas the colorful image is more straightforward and easier to recognize as a pentagon.

### General Position Set:
The term "general position set" typically refers to a set of points in space where no three points are collinear, and no four points are coplanar. In the context of a pentagon, this would mean that all five vertices are in general position, meaning they do not lie on the same plane and no three of them are collinear.

Given the descriptions provided, both images seem to depict a pentagon in general position since there's no indication that any of the vertices are collinear or coplanar.

### Summary:
- The black and white image with binary numbers is an artistic and unique visualization.
- The colorful image with binary numbers is a more traditional and recognizable view.
- Both images represent the same pentagon in general position, with vertices labeled 0 and 1 for specific purposes.