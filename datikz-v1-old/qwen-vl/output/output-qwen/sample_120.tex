It seems like you're describing a geometric problem involving an isosceles triangle divided into smaller triangles, each labeled with a number from 1 to 5. Let's break down the problem step by step.

### Problem Description:
- You have an isosceles triangle divided into smaller triangles.
- Each side of the larger triangle is labeled with a number from 1 to 5.
- The labels indicate the divisions within the triangle.
- The term "est" suggests that we need to estimate or calculate something related to the triangle.

### Key Points:
1. **Isosceles Triangle**: An isosceles triangle has two equal sides.
2. **Division**: The triangle is divided into smaller triangles, and these smaller triangles are labeled with numbers from 1 to 5.
3. **Estimation**: The term "est" implies that we need to make an estimation or calculation based on the given information.

### Possible Interpretations:
Given the context, it seems like the problem might be asking for an estimation or calculation related to the areas of the smaller triangles or some other geometric property.

### Steps to Solve:
1. **Identify the Structure**: Since the triangle is isosceles and divided into smaller triangles, we can infer that the division might be symmetric along the axis of the isosceles triangle.
2. **Labeling**: The numbers 1 through 5 likely represent the areas of the smaller triangles or some other property.
3. **Estimation**: If "est" refers to estimating, we might need to use the properties of similar triangles or the area ratios of the smaller triangles to the whole triangle.

### Example Calculation:
If the triangle is divided into smaller triangles in a way that the areas of the smaller triangles form a pattern, we can use the fact that the sum of the areas of the smaller triangles equals the area of the larger triangle.

For instance, if the triangle is divided into 5 smaller triangles and the areas are proportional to the numbers 1 through 5, we can assume the areas are in the ratio \(1:2:3:4:5\).

The total area of the larger triangle would then be the sum of the areas of the smaller triangles:
\[ \text{Total Area} = 1 + 2 + 3 + 4 + 5 = 15 \]

If we need to find the area of one of the smaller triangles, say the one labeled with the number 3, we can use the ratio:
\[ \text{Area of triangle labeled 3} = \frac{3}{15} \times \text{Total Area} = \frac{3}{15} \times 15 = 3 \]

So, the area of the triangle labeled with the number 3 is 3.

### Conclusion:
Without more specific details about how the triangle is divided and what exactly "est" refers to, the above example provides a general approach to solving such problems. If you provide more details or clarify the exact nature of the estimation needed, I can help further.