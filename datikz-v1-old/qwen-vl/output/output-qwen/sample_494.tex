To address your question, let's break it down into two parts: 

### Part (a): An NFA Not Representable in \( \text{BPA}^* \), Up to Bisimilarity

1. **Understanding NFA and BPA**: 
   - An NFA (Nondeterministic Finite Automaton) is a model of computation that can be used to recognize regular languages.
   - \( \text{BPA}^* \) refers to the star of the BPA (Basic Process Algebra) which includes all possible finite sequences of basic processes.

2. **Bisimilarity**:
   - Bisimilarity is an equivalence relation on automata that captures the notion of observational equivalence. Two automata are bisimilar if they behave identically from any initial state under any sequence of inputs.

3. **Example of an NFA Not Representable in \( \text{BPA}^* \)**:
   - Consider an NFA with a single state \( q_0 \) that has a self-loop on input \( a \) and transitions to another state \( q_1 \) on input \( b \). This NFA recognizes the language \( L = \{a^n b^m \mid n, m \geq 0\} \).
   - The language \( L \) cannot be represented by any finite sequence of basic processes in \( \text{BPA}^* \) because \( \text{BPA}^* \) does not have the capability to capture the infinite repetition of symbols as seen in \( L \).

### Part (b): One Expressible in \( \text{BPA}^* \) That Is Language Equivalent to the NFA in (a)

1. **Constructing a BPA Expression for the NFA in (a)**:
   - We need to find a way to express the behavior of the NFA using \( \text{BPA}^* \).
   - Let's define the following basic processes:
     - \( A \) represents the transition on input \( a \) from \( q_0 \) to itself.
     - \( B \) represents the transition on input \( b \) from \( q_0 \) to \( q_1 \).
     - \( C \) represents the transition on input \( b \) from \( q_1 \) back to \( q_0 \).

2. **Formulating the BPA Expression**:
   - The NFA recognizes \( L = \{a^n b^m \mid n, m \geq 0\} \).
   - We can express this using the following BPA expression:
     \[
     \text{BPA}^* = A^* B^*
     \]
   - Here, \( A^* \) represents the repetition of \( a \)'s, and \( B^* \) represents the repetition of \( b \)'s. Together, they form the language \( L \).

3. **Verification**:
   - Any string in \( L \) can be generated by concatenating zero or more \( a \)'s followed by zero or more \( b \)'s.
   - The BPA expression \( A^* B^* \) correctly captures this behavior, ensuring that the language recognized by the BPA expression is exactly the same as the language recognized by the NFA in part (a).

Thus, we have shown that there exists a BPA expression \( A^* B^* \) that is language equivalent to the NFA in part (a), up to bisimilarity.