It seems like you're referring to the RIV graph equivalence move, which is a concept from knot theory or graph theory. However, the image you've described doesn't seem to directly relate to this concept. Instead, it appears to be a simple diagram showing four white lines arranged in a cross-like pattern, each labeled with an abbreviation: BIV, RIV, T, and X.

If we interpret "RIV" as a label for one of these lines, let's break down what might be meant by the RIV graph equivalence move:

1. **BIV**: This could stand for "Bivalent," meaning the line connects two points.
2. **RIV**: This could stand for "Relevant," suggesting that this line is significant in some context.
3. **T**: This could represent "Terminal," indicating the end point of the line.
4. **X**: This could be a placeholder or a specific identifier.

In the context of graph theory or knot theory, an "equivalence move" typically refers to a transformation that preserves the fundamental properties of the structure being studied. For example, in knot theory, an equivalence move might involve changing how a knot is tied without altering its essential properties (like crossing number).

Given the labels and the cross-like arrangement, if we were to consider a hypothetical "RIV graph equivalence move," it might involve a transformation where the "RIV" line undergoes a change while maintaining the overall structure of the graph or the knot. However, without more context, it's difficult to provide a precise definition or explanation.

If you have a specific problem or scenario in mind related to the RIV graph equivalence move, please provide more details so I can offer a more accurate and helpful response.