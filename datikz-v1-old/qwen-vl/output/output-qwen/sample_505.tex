The Atiyah-Hirzebruch spectral sequence is a powerful tool in algebraic topology used to compute the cohomology of spaces. For an Enriques surface \(X\) over a finite field \(\mathbb{F}_q\), the second page \(E_2\) of this spectral sequence provides important information about the cohomology groups of \(X\).

The \(E_2\)-page of the Atiyah-Hirzebruch spectral sequence for a smooth projective variety \(X\) over a field \(\mathbb{F}_q\) is given by:
\[ E_2^{p,q} = H^p(X, \mathbb{Z}/(p)) \otimes H^q(X, \mathbb{Z}/(p)) \]

For an Enriques surface \(X\), the cohomology groups \(H^i(X, \mathbb{Z}/(p))\) can be computed using the properties of Enriques surfaces. Specifically:

- \(H^0(X, \mathbb{Z}/(p)) = \mathbb{Z}/(p)\)
- \(H^1(X, \mathbb{Z}/(p)) = 0\)
- \(H^2(X, \mathbb{Z}/(p)) = \mathbb{Z}/(p)\)
- \(H^3(X, \mathbb{Z}/(p)) = 0\)

Thus, the \(E_2\)-page of the spectral sequence for an Enriques surface \(X\) over \(\mathbb{F}_q\) will look like:
\[ E_2^{p,q} = 
\begin{cases}
\mathbb{Z}/(p) & \text{if } (p,q) = (0,0) \text{ or } (2,0) \\
0 & \text{otherwise}
\end{cases}
\]

This means that the only non-zero terms on the \(E_2\)-page are at positions \((0,0)\) and \((2,0)\). The value at these positions is \(\mathbb{Z}/(p)\), indicating that there is a \(\mathbb{Z}/(p)\)-module in degree 0 and degree 2.

In the context of the whiteboard, you would see expressions like:
- \(E_2^{0,0} = \mathbb{Z}/(p)\)
- \(E_2^{2,0} = \mathbb{Z}/(p)\)
- All other \(E_2^{p,q}\) are zero

The presence of fractions and factors on the whiteboard might represent specific computations or intermediate steps in the construction of the spectral sequence, but they should ultimately lead to the above structure. The whiteboard could also contain notes on how to interpret these terms, such as the fact that the \(\mathbb{Z}/(p)\) modules correspond to the cohomology groups of the Enriques surface modulo \(p\).