The Casimir cut equation in the context of quantum field theory often involves the calculation of effective amplitudes, which can be represented by diagrams like the one you've described. This diagram appears to represent a specific configuration of vertices and lines, which could correspond to a particular Feynman diagram.

In this case, the circle is divided into five equal parts, and the arcs are labeled with numbers from 1 to 5, starting from the bottom and proceeding clockwise. Each arc represents a different vertex or interaction point in the diagram. The black dots on the arcs likely indicate the positions of the effective vertices (or propagators) that contribute to the amplitude \(M_{1234s,s56}^{eff}\).

To understand how these vertices contribute to the effective amplitude, we need to consider the following:

1. **Vertices**: The black dots on the arcs represent vertices where particles interact. In quantum field theory, these vertices are associated with the coupling constants of the theory.
2. **Lines**: The lines connecting the vertices represent propagators, which describe the propagation of particles between the vertices.
3. **Angles**: The angles between the arcs help determine the relative positions of the vertices and thus the structure of the diagram. These angles are crucial for calculating the amplitude using the Feynman rules.

### Steps to Calculate the Effective Amplitude:
1. **Identify the Vertices**: The black dots on the arcs represent the vertices where the interactions occur. For example, if there are two black dots on an arc, it means there is an interaction at that point.
2. **Apply Feynman Rules**: Use the Feynman rules to calculate the contribution of each vertex. This typically involves multiplying the coupling constant associated with the vertex by the propagator contributions.
3. **Sum Over All Contributions**: Sum over all possible ways the vertices can be connected by propagators to form the full diagram contributing to the amplitude.

### Example Calculation:
If the diagram represents a specific interaction involving particles \(a\), \(b\), \(c\), \(d\), \(e\), and \(f\) with the vertices labeled as follows:
- Vertex 1: Interaction involving \(a\) and \(b\)
- Vertex 2: Interaction involving \(c\) and \(d\)
- Vertex 3: Interaction involving \(e\) and \(f\)

The amplitude \(M_{1234s,s56}^{eff}\) would be calculated by summing over all possible ways to connect these vertices with propagators, taking into account the coupling constants at each vertex and the propagator contributions.

### Conclusion:
The Casimir cut equation, as depicted in your diagram, is a visual representation of a specific Feynman diagram contributing to the effective amplitude \(M_{1234s,s56}^{eff}\). By applying the Feynman rules and summing over all possible configurations, you can calculate the total contribution of this diagram to the overall amplitude.

If you have more specific details about the coupling constants or the nature of the particles involved, please provide them so I can give a more detailed calculation.