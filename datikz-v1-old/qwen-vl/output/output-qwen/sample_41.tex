It sounds like you're describing a visually engaging representation of a Teetering Towers game, which is indeed a type of Hackenbush game. In Teetering Towers, players take turns removing blocks from a structure built on a grid, aiming to make the structure collapse. Each block has a value associated with it, and the goal is to make the opponent's side of the structure heavier, causing it to tip over.

In the context of Hackenbush, each block can be thought of as a node in a graph where edges represent connections between nodes (blocks). The value of a block is determined by the number of edges connected to it. The game ends when one player makes the structure unstable, causing it to collapse.

The colors of the blocks (red, blue, and black) could represent different values or types of blocks in the game. For example, red might represent a block with a positive value, blue might represent a negative value, and black might represent a neutral or zero-value block. The arrangement of these blocks would determine the overall value of the structure and influence the game's outcome.

If you have any specific questions about how to play Teetering Towers or how to analyze the game using Hackenbush principles, feel free to ask!