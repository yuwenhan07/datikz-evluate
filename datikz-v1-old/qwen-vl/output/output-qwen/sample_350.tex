It sounds like you're describing a simple diagram or illustration that could be part of a mathematical or geometric explanation. The "lasso graph" term isn't commonly used in standard mathematical or scientific contexts, but it might refer to a type of diagram where a label is used to highlight a specific point within a circle.

Here's how such an illustration might be interpreted:

1. **Circle**: The white circle represents a geometric shape, often used in geometry to denote a region or a set of points equidistant from a central point (the center of the circle).

2. **Label**: The label pointing to the circle likely indicates a specific point or feature within the circle. This could be a vertex, a midpoint, a radius, a tangent, or any other significant point depending on the context of the diagram.

3. **Focal Point**: The circle being the focal point suggests that it is the main subject of the diagram, possibly highlighting its properties or significance in a particular problem or concept.

If this is part of a larger explanation or a specific field of study (like geometry, physics, or engineering), the label would provide additional information about the circle's role or characteristics. For example:
- If the label says "center," it points to the center of the circle.
- If it says "radius," it might indicate the length of a line segment from the center to the edge of the circle.
- If it says "tangent," it might refer to a line touching the circle at exactly one point.

Without more context, it’s hard to determine the exact meaning, but the general idea is that the label is meant to draw attention to a specific aspect of the circle. If you have more details or a specific question about the diagram, feel free to share!