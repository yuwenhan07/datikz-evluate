It seems like you're describing a geometric figure involving a circle and its division. Let's break down the information provided:

1. **Circle Description**: You mentioned a white circle with a radius of 5.12 inches (or 13 cm), centered at the origin (0,0) on a black and white background.

2. **Division of the Circle**: The circle is divided into nine equal parts. This division is based on the relationship between the circle's radius and the square root of 2, which is approximately 1.414. This suggests that each part of the circle is an angle of \( \frac{360^\circ}{9} = 40^\circ \).

3. **Coordinates of Divided Points**: The coordinates of these points are displayed within the circle. Since the circle is divided into nine equal parts, the points would be located at angles of \( 0^\circ, 40^\circ, 80^\circ, \ldots, 320^\circ \) from the positive x-axis.

Given this information, let's summarize the key points:
- The circle has a radius of 5.12 inches.
- It is divided into nine equal parts, each subtending an angle of 40 degrees.
- The coordinates of the points can be calculated using the formula for polar to Cartesian coordinates: \( x = r \cos(\theta) \) and \( y = r \sin(\theta) \), where \( r = 5.12 \) inches and \( \theta \) is the angle in degrees.

For example, the coordinates of the point at \( 0^\circ \) would be:
\[ x = 5.12 \cos(0^\circ) = 5.12 \]
\[ y = 5.12 \sin(0^\circ) = 0 \]

The coordinates of the point at \( 40^\circ \) would be:
\[ x = 5.12 \cos(40^\circ) \approx 5.12 \times 0.766 = 3.90 \]
\[ y = 5.12 \sin(40^\circ) \approx 5.12 \times 0.643 = 3.29 \]

This process can be repeated for all other angles to find the coordinates of the other points.

If you need further assistance or have specific questions about the coordinates or the division of the circle, feel free to ask!