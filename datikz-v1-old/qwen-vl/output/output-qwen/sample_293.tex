To understand the statement and its implications, let's break it down step by step.

### 1. **Understanding \(K_{1,3}\)**
\(K_{1,3}\) is a complete bipartite graph with one vertex in one part and three vertices in the other part. It can be visualized as follows:
\[
K_{1,3} = \{a, b_1, b_2, b_3\}
\]
where \(a\) is connected to each of \(b_1, b_2, b_3\).

### 2. **Replacing an Edge of \(K_{1,3}\) by a \(P_4\)**
A \(P_4\) (path on 4 vertices) is a path consisting of four vertices where each pair of consecutive vertices is connected by an edge. We need to replace one of the edges of \(K_{1,3}\) with a \(P_4\). There are three possible edges to replace: \(ab_1\), \(ab_2\), or \(ab_3\).

Let's consider each case:

#### Case 1: Replace \(ab_1\) with a \(P_4\)
The new graph becomes:
\[
G' = \{a, b_1', b_2, b_3\}
\]
where \(b_1'\) is connected to \(b_2\) and \(b_3\), but not to \(a\).

#### Case 2: Replace \(ab_2\) with a \(P_4\)
The new graph becomes:
\[
G'' = \{a, b_1, b_2', b_3\}
\]
where \(b_2'\) is connected to \(b_1\) and \(b_3\), but not to \(a\).

#### Case 3: Replace \(ab_3\) with a \(P_4\)
The new graph becomes:
\[
G''' = \{a, b_1, b_2, b_3'\}
\]
where \(b_3'\) is connected to \(b_1\) and \(b_2\), but not to \(a\).

### 3. **Resulting Tree \(T\)**
Each of these graphs \(G'\), \(G''\), and \(G'''\) can be represented as nodes in a tree \(T\). The edges of \(T\) represent the transformations between these graphs. Specifically, each edge in \(T\) corresponds to replacing one specific edge of \(K_{1,3}\) with a \(P_4\).

### 4. **TS_2-Reconfiguration Graph**
A TS_2-reconfiguration graph is a graph that represents all possible configurations of a given problem, where each configuration is a solution to the problem, and two configurations are connected if they differ by a single operation (in this context, replacing an edge with a \(P_4\)).

### 5. **Nodes Representing Size-2 Stable Sets**
In the context of the graph \(G\), a size-2 stable set is a set of two vertices such that no two vertices in the set are adjacent. For \(K_{1,3}\), the size-2 stable sets are:
\[
\{(b_1, b_2), (b_1, b_3), (b_2, b_3)\}
\]

When we replace an edge of \(K_{1,3}\) with a \(P_4\), the resulting graph \(G'\) will have different size-2 stable sets. For example, if we replace \(ab_1\) with a \(P_4\), the new stable sets will be:
\[
\{(b_1', b_2), (b_1', b_3), (b_2, b_3)\}
\]

Thus, each node \(ab\) in the tree \(T\) represents a size-2 stable set of the original graph \(G\).

### Conclusion
The statement that replacing an edge of \(K_{1,3}\) by a \(P_4\) results in a tree \(T\) which is also a TS_2-reconfiguration graph, and each node \(ab\) in \(T\) represents a size-2 stable set of \(G\), encapsulates the idea that the tree \(T\) captures all possible configurations of the graph \(G\) under the specified transformation, and each node in \(T\) corresponds to a specific stable set in \(G\).

\[
\boxed{T \text{ is a TS}_2\text{-reconfiguration graph where each node } ab \text{ represents a size-2 stable set of } G.}
\]