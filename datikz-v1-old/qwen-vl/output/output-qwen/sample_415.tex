To understand why each of the four situations in a given row will appear the same number of times in the collection of Eulerian circuits for a given graph, we need to delve into the properties of Eulerian circuits and the structure of the graph.

An Eulerian circuit is a closed walk that visits every edge exactly once and returns to the starting vertex. For a graph to have an Eulerian circuit, all vertices must have even degrees (i.e., the number of edges incident to each vertex must be even).

Let's denote the four situations as follows:
1. Situation A: The graph has 4 vertices with degrees 2, 2, 2, and 2.
2. Situation B: The graph has 4 vertices with degrees 2, 2, 2, and 4.
3. Situation C: The graph has 4 vertices with degrees 2, 2, 4, and 4.
4. Situation D: The graph has 4 vertices with degrees 2, 4, 4, and 4.

In each of these situations, the total number of edges in the graph is half the sum of the degrees of the vertices. Since the sum of the degrees of all vertices is \(2 + 2 + 2 + 2 = 8\) or \(2 + 2 + 2 + 4 = 10\) or \(2 + 2 + 4 + 4 = 12\) or \(2 + 4 + 4 + 4 = 14\), the number of edges is 4, 5, 6, or 7 respectively. However, the key point is that the structure of the graph and the distribution of the degrees among the vertices determine the number of Eulerian circuits.

The number of Eulerian circuits in a graph depends on the structure of the graph and the specific degrees of the vertices. In each of the four situations described, the graph is structured such that it can be traversed in a way that every edge is used exactly once and the path ends at the starting vertex. The number of distinct Eulerian circuits in a graph with a given degree sequence is determined by the symmetries and the connectivity of the graph, but the problem statement implies that the number of times each situation appears is the same due to some underlying symmetry or uniformity in the construction of the graph.

Therefore, the answer is:

\[
\boxed{4}
\]