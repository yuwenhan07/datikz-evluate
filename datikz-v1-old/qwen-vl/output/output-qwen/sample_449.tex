To solve this problem, we need to interpret the pictorial representation and the mathematical symbols correctly. Let's break it down step by step.

1. **Understanding the Pictorial Representation:**
   - There is a whiteboard.
   - On the whiteboard, there is a black arrow pointing to the right.
   - The black arrow is labeled with mathematical symbols, which indicate a relationship between the variables \(x\), \(y\), and \(z\).

2. **Interpreting the Mathematical Symbols:**
   - The symbols on the arrow suggest a conditional relationship. Specifically, they indicate that if \(x\) and \(y\) are both odd, then \(z\) is also odd. This can be written as:
     \[
     (x \text{ is odd}) \land (y \text{ is odd}) \implies z \text{ is odd}
     \]
   - Here, "\(\land\)" represents "and," and "\(\implies\)" represents "implies."

3. **Conditional Steps:**
   - The dashed arrows in the image represent conditional steps. However, since the image only shows one arrow, we assume that the condition for \(z\) being odd is directly derived from the conditions on \(x\) and \(y\). There are no other conditional steps shown in the image.

4. **Final Answer:**
   - The pictorial representation indicates that if \(x\) and \(y\) are both odd, then \(z\) must also be odd. This is a direct implication without any additional conditions or steps.

Thus, the final answer is:
\[
\boxed{(x \text{ is odd}) \land (y \text{ is odd}) \implies z \text{ is odd}}
\]