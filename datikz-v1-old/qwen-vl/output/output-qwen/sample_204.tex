It seems like you're describing a visual representation rather than a mathematical or computational problem. However, if we interpret this as a request to understand an interpolation strategy for points \((1/p, s)_q\) within a context involving a white triangle with red and green squares, it might be related to a specific field such as computer graphics, data visualization, or numerical analysis.

In general, interpolation strategies are used to estimate values between known data points. For instance, in the context of a 3D object represented by a white triangle with red and green squares, one might use interpolation to determine the color at any point within the triangle based on the colors of the vertices (red and green).

Here's a step-by-step approach to understanding how interpolation could be applied:

1. **Identify the Known Points**: The vertices of the triangle are the known points where the colors are defined. Let's say the vertices are \(A\), \(B\), and \(C\), with colors \(C_A\), \(C_B\), and \(C_C\) respectively.

2. **Define the Interpolation Function**: A common method for interpolating within a triangle is linear interpolation. For a point \(P\) inside the triangle, the color at \(P\) can be calculated using barycentric coordinates. Barycentric coordinates represent the position of a point relative to the vertices of the triangle.

   If \(P\) has barycentric coordinates \((u, v, w)\) where \(u + v + w = 1\), then the color at \(P\) is given by:
   \[
   C_P = u \cdot C_A + v \cdot C_B + w \cdot C_C
   \]

3. **Determine Barycentric Coordinates**: To find the barycentric coordinates of a point \(P\) inside the triangle, you would typically solve a system of linear equations derived from the geometry of the triangle. This involves projecting the point onto the plane of the triangle and solving for the ratios of the distances from \(P\) to the sides of the triangle.

4. **Apply the Interpolation Formula**: Once you have the barycentric coordinates, plug them into the interpolation formula to get the interpolated color at \(P\).

If the points \((1/p, s)_q\) refer to specific coordinates within the triangle, you would need to convert these coordinates into barycentric coordinates and then apply the interpolation formula.

This is a simplified explanation, and depending on the complexity of the problem, more sophisticated methods such as bilinear or bicubic interpolation might be necessary, especially if the triangle is part of a larger grid or mesh.

If you have specific coordinates or a more detailed description of the problem, please provide additional details so I can offer a more tailored solution.