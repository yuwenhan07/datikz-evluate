To solve this problem, we need to understand the geometric configuration described in the problem statement. We have a circle with a radius of 4, and a line segment of length 2 that is cut out from the circle. The vertices \(v_i\) and \(v_j\) belong to three consecutive cliques, which implies they are part of a sequence of connected points forming a path or cycle around the circle.

The key observation here is that the line segment of length 2 is cut out from the circle, meaning there is a gap of 2 units along the circumference of the circle. This gap can be visualized as an arc of the circle with a length of 2 units.

Since the circle has a radius of 4, the circumference of the circle is given by:
\[ C = 2\pi \times 4 = 8\pi \]

The length of the arc corresponding to the gap is 2 units. To find the central angle \(\theta\) subtended by this arc, we use the formula for the arc length:
\[ \text{Arc length} = r \theta \]
where \(r\) is the radius and \(\theta\) is the central angle in radians. Plugging in the values, we get:
\[ 2 = 4 \theta \]
\[ \theta = \frac{2}{4} = \frac{\pi}{4} \]

This means that the central angle subtended by the gap is \(\frac{\pi}{4}\) radians. Since the vertices \(v_i\) and \(v_j\) belong to three consecutive cliques, they must be separated by exactly one vertex (or one gap) on the circle. Therefore, the distance between \(v_i\) and \(v_j\) along the circumference of the circle is equal to the length of the arc corresponding to the gap, which is 2 units.

Thus, the distance between the vertices \(v_i\) and \(v_j\) is:
\[ \boxed{2} \]