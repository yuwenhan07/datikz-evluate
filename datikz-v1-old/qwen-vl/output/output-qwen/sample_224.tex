The image you've described seems to be an artistic representation of the concept of "Games of Blue-Red Hackenbush," which is a mathematical game theory problem. In this game, players take turns removing edges from a graph (which can be thought of as a network of nodes connected by lines) that is initially formed by a collection of blue and red squares. The goal is to make the last move that results in a disconnected graph.

In the context of your description, the blue and red squares could represent different types of nodes or vertices in the graph, and the lines connecting them would be the edges. The game's rules dictate that on each turn, a player must remove one edge, and the player who cannot make a move loses the game.

The visual effect created by the arrangement of the squares might be inspired by the idea of a fractal, where patterns repeat at different scales, or by natural phenomena like the branching structure of trees or the formation of crystals. The randomness and organization in the placement of the squares could symbolize the unpredictable yet structured nature of the game.

If you're interested in exploring more about Hackenbush or its variations, there are many resources available online, including academic papers and interactive games that simulate the game. These resources can provide deeper insights into the mathematical strategies involved and the fascinating properties of the game.