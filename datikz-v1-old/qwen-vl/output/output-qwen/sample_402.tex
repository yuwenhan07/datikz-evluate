To address the problem, let's break it down step by step.

### Definitions and Setup

1. **Status of Each Player:**
   - **Favorite While the Election is Secured (FS):** A player is considered an FS if they are the favorite in the election and the election outcome is certain.
   - **Still Open (FO):** A player is considered FO if the election outcome is not yet certain.
   - **Contestable Favorite (CF):** A player is considered CF if they are the favorite but the election outcome is still open.
   - **Uncontestable Favorite (UCF):** A player is considered UCF if they are the favorite and the election outcome is certain.

2. **Position of the Median Voter \( m \):**
   - The median voter is the voter whose position on the scale is exactly at the midpoint between the positions of all voters. If there are an odd number of voters, the median voter is the one whose position is the middle value when all positions are ordered.

3. **Without Loss of Generality (WLOG):**
   - We assume \( x_1 < x_2 \), meaning the positions of the players are ordered such that \( x_1 \) is less than \( x_2 \).

### Analysis

Given the definitions and the assumption \( x_1 < x_2 \), we need to determine the status of each player (FS, FO, CF, UCF) based on their position relative to the median voter \( m \).

#### Case 1: \( m \leq x_1 \)
- If the median voter \( m \) is less than or equal to \( x_1 \), then:
  - All players with positions less than or equal to \( m \) are UCF because they are the favorite and the election outcome is certain.
  - Players with positions greater than \( m \) but less than \( x_1 \) are CF because they are the favorite but the election outcome is still open.
  - Players with positions greater than \( x_1 \) are FO because the election outcome is still uncertain.

#### Case 2: \( x_1 < m < x_2 \)
- If the median voter \( m \) is between \( x_1 \) and \( x_2 \), then:
  - All players with positions less than or equal to \( m \) are UCF because they are the favorite and the election outcome is certain.
  - Players with positions greater than \( m \) but less than \( x_2 \) are CF because they are the favorite but the election outcome is still open.
  - Players with positions greater than \( x_2 \) are FO because the election outcome is still uncertain.

#### Case 3: \( m \geq x_2 \)
- If the median voter \( m \) is greater than or equal to \( x_2 \), then:
  - All players with positions less than or equal to \( x_2 \) are UCF because they are the favorite and the election outcome is certain.
  - Players with positions greater than \( x_2 \) are FO because the election outcome is still uncertain.

### Conclusion

The status of each player depends on the position of the median voter \( m \) relative to the positions \( x_1 \) and \( x_2 \). Here is a summary:

- **UCF:** Players with positions less than or equal to \( m \).
- **CF:** Players with positions greater than \( m \) but less than \( x_1 \) or \( x_2 \).
- **FO:** Players with positions greater than \( x_1 \) but less than \( x_2 \) or greater than \( x_2 \).

This analysis covers all possible scenarios based on the position of the median voter \( m \) relative to \( x_1 \) and \( x_2 \).