It seems you're describing a visual representation of mathematical concepts, specifically related to the concept of ℓ_q and its variations, as well as equations involving b, w, and b2w. This could be part of a lecture or a study session where these mathematical ideas are being explained.

To provide more context or assistance, let's break down what might be happening:

1. **ℓ_q**: This often refers to a sequence or series in mathematics, particularly in the context of functional analysis or signal processing. It could represent a specific type of norm (like the q-norm) applied to a sequence or vector.

2. **Difference Between Two Choices of ℓ_q**: This suggests that there are multiple ways to define or compute an ℓ_q norm, and the discussion is about the differences between these definitions or applications.

3. **Equations Involving b, w, and b2w**: These variables likely represent parameters or coefficients in the equations. For example:
   - \( b \) could be a base or a scaling factor.
   - \( w \) could be a weight or a variable parameter.
   - \( b2w \) might be a product or transformation of \( b \) and \( w \).

4. **Compact, Whiteboard-Style Notation**: This style of writing is common in mathematical discussions, where equations are simplified for clarity and ease of understanding. It often involves using shorthand or specific notational conventions to convey complex ideas succinctly.

If you have specific questions about any of these elements or need further clarification, feel free to ask!