The image you've described seems to be a visual representation of a concept from graph theory, specifically related to the 2-substring graph and its reduction process. In this context, "variable \(x_a\)" likely refers to a specific node or element within the graph.

In the 2-substring graph, each vertex represents a substring of length two (a pair of characters) from some larger string, and edges connect vertices if their corresponding substrings overlap by at least one character. The 2-attractor reduction is a process that simplifies the graph by removing certain nodes under specific conditions.

Here's how we might interpret the "gadgets" for a variable \(x_a\) in this context:

1. **Node Representation**: Each circle in the chain could represent a node in the 2-substring graph. The label on the node, such as \(x_a\), indicates the specific substring it corresponds to.

2. **Interconnected Circles**: The interconnected nature of the circles suggests that there are edges between these nodes, indicating that the corresponding substrings overlap by at least one character.

3. **Mathematical Notation**: The mathematical notation around the circles, such as \(x_a\), is likely used to denote the specific substring represented by that node. For example, if \(x_a = ab\), then the node would represent the substring "ab".

4. **Reduction Process**: The 2-attractor reduction process involves identifying and removing nodes that do not contribute significantly to the structure of the graph. This could involve checking if a node can be removed without changing the connectivity properties of the graph.

Given the chain-like structure, it's possible that the reduction process is being applied sequentially, where each step removes a node based on certain criteria until no further reductions can be made.

If you're looking for a more detailed explanation or need help with a specific part of the reduction process, feel free to ask!