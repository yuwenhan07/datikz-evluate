It seems like you're describing a Directed Acyclic Graph (DAG) that illustrates a scenario involving common parent confounding, where the edge between \( U_z \) and \( z \) has been deleted. This setup aims to show that a counterfactual quantity can be independent of its parents under certain conditions.

Here's a textual representation of what your description might look like:

---

Consider a DAG with nodes representing variables:
- \( U_z \): A latent variable or unobserved confounder.
- \( z \): An observed variable.
- \( X \): Another observed variable.

The DAG initially shows an edge from \( U_z \) to \( z \), indicating that \( U_z \) influences \( z \). Additionally, there may be other edges in the graph, such as \( z \) influencing \( X \).

Now, let's assume we delete the edge between \( U_z \) and \( z \). This modification implies that \( z \) no longer directly depends on \( U_z \). Instead, \( z \) could still depend on other variables not shown here, or it could be independent of \( U_z \) given these other variables.

In this modified DAG, the counterfactual quantity of interest might be represented by the variable \( X \). The equation \( X = Y \) suggests that \( X \) is set to some value \( Y \) in a hypothetical scenario, which is a common way to define counterfactuals in causal inference.

The white sphere with a blue arrow pointing to the right labeled with "U" and the equation "X=Y" likely represents the following:
- The white sphere symbolizes the variable \( U \), which could be a latent variable or a potential outcome.
- The blue arrow pointing to the right labeled with "U" indicates the influence of \( U \) on another variable, possibly \( X \).
- The equation "X=Y" signifies that \( X \) is set to a specific value \( Y \) in a counterfactual scenario.

---

This setup helps illustrate how deleting an edge in a DAG can change the relationships between variables and how counterfactual quantities can be defined and analyzed.