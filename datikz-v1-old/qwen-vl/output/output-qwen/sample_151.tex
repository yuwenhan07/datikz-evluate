It seems you're describing a visual representation or arrangement of characters (F's and T's) on a white background, which could be interpreted as a pattern or sequence. However, your mention of "quotient graph of groups for PGL(2,F_q[t])-action" suggests a more mathematical context.

In mathematics, particularly in group theory and algebraic geometry, the quotient graph of a group action can be a way to visualize the structure of the quotient space formed by the action of a group on a set. For the specific case of \( \text{PGL}(2, F_q[t]) \)-action, where \( F_q[t] \) denotes the field of rational functions over a finite field \( F_q \), this would involve understanding how the group acts on some geometric object (like a curve or surface).

Given that the F's and T's are arranged in a column with specific positions, it might be a simplified or abstract representation of such a quotient space. Here’s a possible interpretation:

- The F's and T's could represent different points or elements in the quotient space.
- The positions (1, 2, 3, 4, 5) could indicate the coordinates or labels of these points within the quotient space.
- The pattern of F's and T's could reflect the symmetry or structure imposed by the group action.

If you have a specific question about this pattern or need further clarification, feel free to ask!