It seems like you're describing an image that might be a stylized or artistic representation rather than a traditional spectral diagram, which typically shows data points on a graph with axes representing frequency and magnitude. However, based on your description, it sounds like the image could be a creative or abstract depiction of a spectral diagram.

Here's a breakdown of what you've described:

1. **Spectral Diagram**: This term usually refers to a graphical representation of the spectrum of a signal, where the x-axis represents frequency and the y-axis represents amplitude or power.
2. **|T|**: This likely refers to the magnitude of a transfer function T, which is common in signal processing and control systems.
3. **AN(H)**: This could refer to a specific type of analysis or transformation, possibly related to the Hilbert transform (H) applied to a signal (AN).
4. **White-on-white drawing of a ruler**: This suggests that the numbers on the ruler are not easily distinguishable due to the color scheme.
5. **Series of numbers along its edge**: These numbers are written in various sizes and include both regular numbers and scientific notation.
6. **Orange and blue color scheme**: This indicates the use of these colors for the ruler and the numbers.

Given this information, if you're looking for a more conventional spectral diagram, you would typically see a plot with frequency on the x-axis and magnitude or power on the y-axis. The numbers would be clearly visible and organized in a way that makes sense for the data being represented.

If you need assistance with creating or interpreting a spectral diagram, feel free to provide more details about the specific data or context, and I can offer more tailored advice!