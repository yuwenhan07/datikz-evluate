To determine the maximum goodput between one publisher and one subscription for different message sizes, we need to analyze the data presented in the image. Here's a step-by-step approach:

1. **Identify the Message Sizes**: The message sizes are given as 3,000 KB, 2,000 KB, 1,000 KB, 500 KB, 100 KB, and 10 KB.

2. **Analyze the Lines**: The lines represent different performance metrics or goodput values for these message sizes. Each color (green, blue, and red) corresponds to a specific message size. We need to identify which line corresponds to which message size based on the legend or the context provided.

3. **Determine Goodput Values**: The goodput value is typically the throughput or the rate at which messages are transferred from the publisher to the subscriber. From the graph, we can see that each line represents a different goodput value for the respective message size.

4. **Find the Maximum Goodput**: By examining the graph, we look for the highest point on any of the lines. This will give us the maximum goodput for the corresponding message size.

Since the exact numerical values are not provided in the description, let's assume the following hypothetical goodput values for illustration purposes:
- Green line: 1000 KB/s for 3000 KB
- Blue line: 800 KB/s for 2000 KB
- Red line: 600 KB/s for 1000 KB

From this hypothetical data:
- The maximum goodput for 3000 KB is 1000 KB/s.
- The maximum goodput for 2000 KB is 800 KB/s.
- The maximum goodput for 1000 KB is 600 KB/s.

If we were to generalize without the exact values, we would need to visually inspect the graph to find the highest point on any of the lines. Assuming the graph shows the same pattern as our hypothetical example, the maximum goodput would be the highest value among the three lines.

Therefore, the maximum goodput between one publisher and one subscription for different message sizes would be the highest goodput value observed on the graph, which in our hypothetical case is 1000 KB/s for the 3000 KB message size.

In conclusion, the maximum goodput depends on the actual data points shown in the graph. If you have the exact values, you should compare them to determine the maximum goodput.