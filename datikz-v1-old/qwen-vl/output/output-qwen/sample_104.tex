It seems like you're describing a visual representation rather than a mathematical graph. However, I can help you understand how to represent the bow tie pattern as a graph.

A bow tie pattern typically consists of two loops that intersect at a central point. If we were to translate this into a graph, we would have:

1. Six vertices (nodes).
2. Nine edges connecting these vertices in a specific way to form the bow tie shape.

Here's a step-by-step breakdown of how to represent it mathematically:

### Vertices:
Let's label the vertices as \( A, B, C, D, E, F \).

### Edges:
The bow tie pattern can be described by the following edges:
- \( AB \)
- \( AC \)
- \( AD \)
- \( BC \)
- \( BD \)
- \( CD \)
- \( EF \)
- \( EG \)
- \( FH \)

This gives us a total of 9 edges.

### Graph Representation:
We can represent this graph using an adjacency matrix or a list of edges. Here’s the adjacency matrix for the graph:

\[
\begin{array}{cccccc}
 & A & B & C & D & E & F \\
A & 0 & 1 & 1 & 1 & 0 & 0 \\
B & 1 & 0 & 1 & 1 & 0 & 0 \\
C & 1 & 1 & 0 & 1 & 0 & 0 \\
D & 1 & 1 & 1 & 0 & 0 & 0 \\
E & 0 & 0 & 0 & 0 & 0 & 1 \\
F & 0 & 0 & 0 & 0 & 1 & 0 \\
\end{array}
\]

Or, we can simply list the edges:
\[ \{AB, AC, AD, BC, BD, CD, EF, EG, FH\} \]

This graph structure captures the bow tie pattern with its six vertices and nine edges. If you need further assistance or have any specific questions about this graph, feel free to ask!