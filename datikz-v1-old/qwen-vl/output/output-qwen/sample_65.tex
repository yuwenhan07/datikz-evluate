The schematic you've described pertains to the study of New Physics (NP) contributions to flavor-changing neutral current (FCNC) semileptonic decays within the context of the Standard Model (SM) in its two-generation limit. Here's a breakdown of the key components:

### 1. **Standard Model (SM) Framework**
- The SM is a theory that describes the fundamental forces of nature except gravity.
- It includes three generations of quarks and leptons.
- In the two-generation limit, only the first two generations of quarks and leptons are considered.

### 2. **Flavor-Changing Neutral Current (FCNC) Semileptonic Decays**
- These are processes where a quark changes flavor through the weak interaction but does not change its electric charge.
- Examples include \( B \to X_s \ell^+ \ell^- \), where \( B \) is a bottom meson, \( X_s \) is a hadronic state containing a strange quark, and \( \ell \) is a lepton (e.g., electron or muon).

### 3. **New Physics (NP) Contributions**
- NP refers to physics beyond the SM, which could introduce new particles or interactions that modify the SM predictions.
- These contributions can be classified into two types based on their CP properties:
  - **CP Conserving (CP)**: These contributions respect the CP symmetry and are characterized by magnitudes \( |z_{C=1}| \) and \( |z_{S=1}| \).
  - **CP Violating (CPV)**: These contributions violate CP symmetry and are characterized by complex phases \( c_I \).

### 4. **Alignment Angle (\(\theta_d\))**
- The alignment angle \( \theta_d \) is a parameter that characterizes the alignment between the NP couplings and the SM couplings.
- It affects the magnitude of the CP-conserving NP contributions but not the CP-violating ones.

### 5. **Dependence on Alignment Angle (\(\theta_d\))**
- The magnitudes \( |z_{C=1}| \) and \( |z_{S=1}| \) depend on the alignment angle \( \theta_d \). This dependence arises because the NP couplings are aligned with the SM couplings at some angle.
- The alignment angle \( \theta_d \) modifies how strongly the NP effects are felt in the decay amplitudes.

### 6. **CP-Violating NP Contributions**
- CP-violating NP contributions are denoted by \( c_I \) and are independent of the alignment angle \( \theta_d \).
- These contributions arise from the interference between NP and SM amplitudes and can lead to observable CP violation in the decay rates.

### 7. **Summary**
- The schematic illustrates how the magnitudes of CP-conserving NP contributions depend on the alignment angle \( \theta_d \), while the CP-violating contributions remain constant.
- Understanding these contributions is crucial for testing the SM and searching for evidence of NP in FCNC semileptonic decays.

This framework is essential for theoretical physicists working on precision measurements in particle physics, particularly in the context of flavor physics and the search for new physics beyond the SM.