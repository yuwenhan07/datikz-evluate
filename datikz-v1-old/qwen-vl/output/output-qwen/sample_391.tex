It seems you're describing an equilateral star graph with 8 vertices (E=8). In graph theory, an equilateral star graph is a type of star graph where all edges have the same length. For E=8, this means there is one central vertex connected to 7 other vertices, forming a star shape.

However, your description of "dots" arranged in a circular pattern within a large black circle sounds more like a visual representation of a geometric figure rather than a graph. If we interpret the dots as vertices of the star graph, they would be placed around the circumference of the circle, with the central dot representing the center of the star graph.

Here's how you might visualize it:

1. **Central Dot**: This represents the central vertex of the star graph.
2. **Circumference Dots**: These represent the 7 outer vertices of the star graph. They are evenly spaced around the circumference of the circle.

If you need further clarification or assistance with a specific aspect of this geometric or graph-theoretic concept, feel free to ask!