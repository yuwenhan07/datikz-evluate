To solve this problem, we need to construct a minimal poset (partially ordered set) whose multiset of maximal chain cardinalities is \(\{3, 4, 5, 6, 7, 8, 9, 10\}\). The notation \(3 + \text{sums}(\{1, 2, 4\})\) suggests that the poset can be constructed by taking an ordinal sum of three posets, where the sizes of the maximal chains in these posets are 1, 2, and 4 respectively.

Let's denote the posets as follows:
- \(P_1\) with maximal chain size 1.
- \(P_2\) with maximal chain size 2.
- \(P_3\) with maximal chain size 4.

The ordinal sum of these posets is denoted as \(P_1 + P_2 + P_3\). The maximal chain cardinalities in the resulting poset will be the sums of the maximal chain cardinalities of the individual posets plus the number of elements in the posets themselves minus 1 (since each element in the poset contributes to the length of at least one maximal chain).

The maximal chain cardinalities in the resulting poset will be:
\[1 + 2 + 4 = 7,\]
\[1 + 2 + 4 + 1 = 8,\]
\[1 + 2 + 4 + 2 = 9,\]
\[1 + 2 + 4 + 3 = 10.\]

However, we need the maximal chain cardinalities to be \(\{3, 4, 5, 6, 7, 8, 9, 10\}\). To achieve this, we need to add additional elements to the posets such that the maximal chain cardinalities include the numbers 3, 4, 5, and 6. We can do this by adding a new element to the poset \(P_1\) or by adjusting the structure of the posets.

One way to achieve this is to consider the poset \(P_1\) as a single-element poset, which has no maximal chains other than itself. Then we can add a new element to the poset \(P_2\) to get a maximal chain of length 3, and adjust the structure of \(P_3\) to get the required maximal chain cardinalities.

Let's construct the poset step-by-step:
1. \(P_1\) is a single-element poset with no maximal chains other than itself.
2. \(P_2\) is a poset with two elements \(a < b\) and a third element \(c\) such that \(a < c\) and \(b < c\). This gives us maximal chains of lengths 2 and 3.
3. \(P_3\) is a poset with four elements \(d < e < f < g\) and an additional element \(h\) such that \(d < h\), \(e < h\), \(f < h\), and \(g < h\). This gives us maximal chains of lengths 4, 5, 6, 7, 8, 9, and 10.

The ordinal sum \(P_1 + P_2 + P_3\) will then have maximal chain cardinalities \(\{3, 4, 5, 6, 7, 8, 9, 10\}\).

Thus, the minimal poset whose multiset of maximal chain cardinalities is \(\{3, 4, 5, 6, 7, 8, 9, 10\}\) is:
\[
\boxed{P_1 + P_2 + P_3}
\]
where \(P_1\) is a single-element poset, \(P_2\) is a poset with three elements \(a < b < c\) and \(P_3\) is a poset with five elements \(d < e < f < g < h\) with \(d < h\), \(e < h\), \(f < h\), and \(g < h\).