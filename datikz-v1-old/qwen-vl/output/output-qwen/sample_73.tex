The notation \( G_{(1^4,2),1} \) and \( G_{M(3^2),\text{PG}_1(3^2)} \) seem to be referencing specific geometric configurations or groups related to the structure of the tetrahedron.

- \( G_{(1^4,2),1} \): This notation might refer to a group action on a set where there are 4 elements of order 1 (which could mean they are fixed points under the group action) and one element of order 2. The subscript "1" could indicate a specific subgroup or configuration within this group.
  
- \( G_{M(3^2),\text{PG}_1(3^2)} \): Here, \( M(3^2) \) likely refers to the matrix group over the field with 9 elements (since \( 3^2 = 9 \)), and \(\text{PG}_1(3^2)\) refers to the projective line over the same field. This suggests a connection to the automorphism group of the projective line over a finite field.

Given the tetrahedron's vertices labeled, it seems you're dealing with a geometric configuration involving symmetries and transformations. If we consider the tetrahedron as a geometric object, its symmetry group is the alternating group \( A_4 \), which has 12 elements. However, the specific notations provided suggest a more specialized context, possibly related to finite fields and their projective geometries.

If the drawing indeed shows a tetrahedron with vertices labeled, the task might involve identifying the symmetries of the tetrahedron, calculating the automorphisms, or understanding how these symmetries relate to the given notations. For instance, if the labels correspond to specific points in a finite projective space, the symmetries of the tetrahedron could be related to the automorphisms of the projective line over the field with 9 elements.

Without further details about the specific labeling and the exact nature of the problem, it's challenging to provide a precise solution. However, if you have more information about the labels or the context of the problem, I can help you analyze it further.