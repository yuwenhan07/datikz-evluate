It seems like you're describing a model of a stream or river system with six patches (or segments) and some parameters that describe the movement of entities within this system. Let's break down the components:

1. **Stream with Six Patches**: This suggests a segmented model of a stream where the stream is divided into six distinct sections or patches.

2. **d (random movement rate)**: This parameter likely represents the rate at which entities (such as fish, particles, or other organisms) move randomly from one patch to another without any specific directionality. It could be influenced by factors such as turbulence in the water, wind, or other environmental conditions.

3. **q_i, i=1,..,6 (directed movement rates)**: These parameters represent the rates at which entities move from one patch to another in a specific direction. For example:
   - \( q_1 \): Rate of movement from patch 1 to patch 2.
   - \( q_2 \): Rate of movement from patch 2 to patch 3.
   - And so on up to \( q_6 \), which would be the rate of movement from patch 6 back to patch 1 (assuming a circular or closed loop).

### Example Model

Let's consider a simple model where the stream is represented as a circular loop with six patches labeled \( P_1, P_2, P_3, P_4, P_5, P_6 \). The movement rates can be described as follows:

- \( d \): Random movement rate.
- \( q_{ij} \): Directed movement rate from patch \( i \) to patch \( j \).

The total movement rate from any patch \( i \) to any other patch \( j \) can be expressed as the sum of the directed movement rates and the random movement rate. For instance, the total movement rate from patch \( i \) to patch \( j \) is given by:

\[ r_{ij} = q_{ij} + d \]

Where:
- \( r_{ij} \) is the total movement rate from patch \( i \) to patch \( j \).
- \( q_{ij} \) is the directed movement rate from patch \( i \) to patch \( j \).
- \( d \) is the random movement rate.

### Example Calculation

If we assume the following values for the directed movement rates and the random movement rate:
- \( d = 0.1 \)
- \( q_{12} = 0.2 \)
- \( q_{23} = 0.3 \)
- \( q_{34} = 0.4 \)
- \( q_{45} = 0.5 \)
- \( q_{56} = 0.6 \)
- \( q_{61} = 0.7 \)

Then the total movement rate from patch 1 to patch 2 would be:
\[ r_{12} = q_{12} + d = 0.2 + 0.1 = 0.3 \]

Similarly, the total movement rate from patch 2 to patch 3 would be:
\[ r_{23} = q_{23} + d = 0.3 + 0.1 = 0.4 \]

And so on for all other patches.

This model can be extended to more complex scenarios, including multiple types of entities moving through the stream, interactions between different types of entities, and additional environmental factors affecting the movement rates.