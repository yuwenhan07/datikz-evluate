The image you've described seems to be related to a mathematical concept, specifically a lemma that might involve the analysis of functions and their properties. Let's break down the idea behind proving such a lemma.

### Context and Notation:
- **Lemma**: This is a statement that needs to be proven.
- **Red Curve**: This likely represents the main function \( f(x) \).
- **Other Curves**: These could represent variations or transformations of \( f(x) \), such as \( f(x) + g(x) \), \( f(x) - h(x) \), or \( f(ax + b) \).

### Steps in Proving the Lemma:

1. **Define the Function**:
   - Start by clearly defining the function \( f(x) \). For example, if \( f(x) = x^2 \), then we need to understand its behavior, domain, and range.

2. **Identify the Curves**:
   - The red curve is \( f(x) \). The other curves could be transformations of \( f(x) \):
     - \( g(x) \): A possible transformation like \( g(x) = f(x) + c \) (a vertical shift).
     - \( h(x) \): Another transformation like \( h(x) = f(x) - d \) (another vertical shift).
     - \( k(x) \): A horizontal shift or scaling, like \( k(x) = f(ax + b) \).

3. **Analyze the Curves**:
   - Examine how these transformations affect the graph of \( f(x) \):
     - Vertical shifts change the y-intercept but not the shape of the curve.
     - Horizontal shifts change the x-intercept but not the shape.
     - Scaling changes the width or height of the curve.

4. **Formulate the Lemma**:
   - Based on the analysis, formulate the lemma. For instance, if the lemma states something about the maximum value of \( f(x) \) under certain conditions, you would need to consider the transformations and their effects on the maximum value.

5. **Proof**:
   - **Step-by-step Proof**:
     1. **Base Case**: Prove the lemma for a simple case, like when \( f(x) = x^2 \).
     2. **Inductive Step**: If applicable, use induction to generalize the result.
     3. **Transformation Analysis**: Analyze how each transformation affects the function. For example, if \( f(x) = x^2 \), then \( f(x) + c \) will have a new maximum at \( x = 0 \) if \( c > 0 \).
     4. **Conclusion**: Combine all the results to prove the lemma for the general case.

6. **Example Lemma**:
   - Suppose the lemma states: "For any function \( f(x) \) and constants \( a \) and \( b \), the maximum value of \( f(ax + b) \) is the same as the maximum value of \( f(x) \)."
   - **Proof**:
     - Consider \( f(x) = x^2 \). The maximum value of \( f(x) = x^2 \) is 0 at \( x = 0 \).
     - For \( f(ax + b) = (ax + b)^2 \), the maximum value occurs at \( ax + b = 0 \), which simplifies to \( x = -\frac{b}{a} \). Substituting back, we get \( f(-\frac{b}{a}) = (-\frac{b}{a})^2 = \frac{b^2}{a^2} \).
     - Since the transformation does not change the maximum value, the lemma holds true.

By following these steps, you can systematically approach the proof of the lemma based on the specific details provided in the problem.