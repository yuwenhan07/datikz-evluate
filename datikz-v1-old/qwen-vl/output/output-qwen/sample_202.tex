To solve the problem, we need to understand the given sequence \( \text{UD} = (\text{RRRLLRLL})(\text{LLRLRLRR}) \) and how it relates to the elevation changes. Let's break it down step by step.

1. **Understanding the Sequence:**
   - The sequence \( \text{RRRLLRLL} \) represents the movement in the "U" direction.
   - The sequence \( \text{LLRLRLRR} \) represents the movement in the "D" direction.
   - Each "R" (right) means moving up one unit, and each "L" (left) means moving down one unit.

2. **Elevation Calculation:**
   - We start at elevation 0.
   - For each "R", we move up by 1 unit.
   - For each "L", we move down by 1 unit.
   - The sequence \( \text{RRRLLRLL} \) can be broken down as follows:
     - R: +1
     - R: +1
     - R: +1
     - L: -1
     - L: -1
     - R: +1
     - L: -1
     - L: -1
   - So, the elevation changes are: +1, +1, +1, -1, -1, +1, -1, -1.
   - Summing these changes: \( 1 + 1 + 1 - 1 - 1 + 1 - 1 - 1 = 0 \).

3. **Combining with the Second Sequence:**
   - Now, we apply the second sequence \( \text{LLRLRLRR} \):
     - L: -1
     - L: -1
     - R: +1
     - L: -1
     - R: +1
     - L: -1
     - R: +1
     - R: +1
   - So, the elevation changes are: -1, -1, +1, -1, +1, -1, +1, +1.
   - Summing these changes: \( -1 - 1 + 1 - 1 + 1 - 1 + 1 + 1 = 0 \).

4. **Final Elevation:**
   - Starting from elevation 0, after applying both sequences, the final elevation is 0.

5. **Marking the Zero Elevation Locations:**
   - The zero elevation locations are where the sum of the elevation changes is 0.
   - From the above calculations, we see that the final elevation is 0, which means there are no additional zero elevation locations beyond the starting point.

Therefore, the final answer is that there are exactly three indices where the elevation is zero, and they are marked with a black dot on the dashed line representing elevation zero.

The answer is \(\boxed{3}\).