It seems like you're describing a concept related to geometric algorithms or data structures, possibly in computational geometry or computer science. Let's break down the key points:

1. **Tower of a Simplex**: This likely refers to a hierarchical structure where each level represents a different size or resolution of the simplex (e.g., a triangle in 2D or a tetrahedron in 3D).

2. **Layers and Radius Growth**: Each layer is represented by a circle, and the radius of these circles grows exponentially. This suggests that as we move up the hierarchy (or "tower"), the size of the elements (vertices, edges, faces) increases.

3. **Estimate (est)**: The term "est" might refer to an estimate function or a threshold value used to determine when a new vertex should be added to a higher-level layer.

4. **New Vertex Placement**: When a new vertex is added to a simplex, it is placed in the next-level layer (dotted circle). If this new vertex lies within the continuous circle representing the current layer, then the entire simplex must be included in that layer.

### Interpretation:
- **Hierarchical Structure**: The tower of a simplex can be thought of as a way to organize and manage geometric objects at different scales.
- **Exponential Growth**: The exponential growth of the radius indicates that the complexity or size of the geometric objects increases rapidly as we move up the hierarchy.
- **Layer Inclusion**: The condition that if a new vertex lies within the continuous circle, the simplex must be included in that layer ensures that the structure remains consistent and coherent across different levels.

### Example:
Imagine a 2D case where the simplex is a triangle. The layers could represent different resolutions of the triangle:
- **Level 0**: A small triangle.
- **Level 1**: A slightly larger triangle.
- **Level 2**: An even larger triangle, etc.

If a new vertex is added to the triangle and it falls within the boundary of Level 1, the entire triangle must be considered part of Level 1. This ensures that the structure remains consistent and that no part of the triangle is left out of the higher-level layer.

### Conclusion:
The concept you're describing is likely part of a more complex algorithm or data structure used in computational geometry, such as a hierarchical clustering method for geometric objects. The exponential growth of the layers helps in efficiently managing and processing large datasets by organizing them into manageable hierarchies.