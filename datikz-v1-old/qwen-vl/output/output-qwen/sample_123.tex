It seems like you're describing a visual representation of a tree structure, possibly related to a specific algorithm or model in database theory, such as Online Yannakakis' Non-redundant PMTD (Partial Materialized Tree Decomposition). However, without a specific diagram or more detailed description, it's challenging to provide precise information.

In the context of Online Yannakakis' Non-redundant PMTD, the tree structure often represents a way to decompose a complex query into simpler subqueries that can be processed independently. Here’s a general explanation:

1. **Tree Structure**: The tree structure is used to represent the dependencies between different parts of a query. Each node in the tree corresponds to a subquery or a part of the query that needs to be evaluated.

2. **Materialization Set (S-views)**: These are the subqueries that are precomputed and stored in the materialization set. They are typically the leaves of the tree because they do not depend on other subqueries.

3. **Head Variables**: These are the variables that appear in the final output of the query. They are usually represented by underlining them in the tree structure.

4. **Hierarchical Structure**: The tree has a hierarchical structure where each node represents a subquery, and the edges represent the dependencies between these subqueries. The numbers on the nodes might indicate the order in which the subqueries are processed or their importance in the query evaluation process.

5. **Non-redundancy**: The term "non-redundant" means that the subqueries in the materialization set are chosen in such a way that no redundant work is done. This ensures that the query is optimized for efficiency.

If you have a specific diagram or more details about the tree structure, please share it, and I can provide a more accurate and detailed explanation.