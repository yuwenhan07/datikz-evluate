To analyze the given information about the function \( f(x) \), let's break down the details step by step:

1. **Parameter Range**: The parameter varies in the interval \((-3, 3)\).
2. **Domain of \( x \)**: The variable \( x \) varies in the interval \((-5, 5)\).
3. **Orientation of Lines**: 
   - Blue lines indicate that the function \( f(x) \) takes a positive value at the points of the grid.
   - Red lines indicate that the function \( f(x) \) takes a negative value at the points of the grid.
4. **Zero of the Function**: The orientation of the lines around the zero (where the function crosses the x-axis) indicates whether the function is increasing or decreasing at that point:
   - If the line changes from blue to red or vice versa as you move across the zero, the function has a local extremum (either a maximum or minimum) at that point.
   - If the line remains blue or red without changing, the function is either strictly increasing or strictly decreasing at that point.

5. **Nonnegativity Condition**: For \( 0 < < 0.8955 \), the function \( f(x) \) is nonnegative on the whole real line. This means that for any \( x \in \mathbb{R} \), \( f(x) \geq 0 \).

### Steps to Analyze the Grid:

1. **Identify Zeros**:
   - Look for points where the color of the lines changes from blue to red or vice versa. These points are likely zeros of the function.

2. **Determine Monotonicity**:
   - Check the orientation of the lines around each zero:
     - If the line changes from blue to red or vice versa, there is a local extremum at that point.
     - If the line remains blue or red, the function is either strictly increasing or strictly decreasing at that point.

3. **Behavior Outside the Interval**:
   - Since no further zeros exist outside the interval \([a, b]\), the behavior of the function outside this interval can be inferred from the behavior within the interval.

4. **Nonnegativity Condition**:
   - For \( 0 < < 0.8955 \), the function is nonnegative on the whole real line. This means that for any \( x \in \mathbb{R} \), \( f(x) \geq 0 \).

### Example Interpretation:

Suppose we have a grid with the following observations:
- There is a zero at \( x = 0 \).
- The line changes from blue to red as you move across \( x = 0 \).
- The function is strictly increasing for \( x < 0 \) and strictly decreasing for \( x > 0 \).

Given the nonnegativity condition for \( 0 < < 0.8955 \), we can infer:
- For \( 0 < < 0.8955 \), the function \( f(x) \) is nonnegative on the whole real line, meaning \( f(x) \geq 0 \) for all \( x \in \mathbb{R} \).

### Conclusion:

The analysis of the grid provides insights into the zeros, monotonicity, and overall behavior of the function \( f(x) \). By identifying the zeros and their nature (local extrema or monotonicity), and considering the nonnegativity condition, we can understand how the function behaves both within and outside the specified intervals.