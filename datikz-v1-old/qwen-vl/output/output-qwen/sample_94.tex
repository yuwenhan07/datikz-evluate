It seems like you're referring to a probability density function (PDF) \( f_b \) that is defined over the interval \([v_0 - 2, v_0 + 2]\), where \( v_0 \) is some central value, and the function varies depending on whether the bit \( b_k \) is 0 or 1.

To provide a more detailed explanation, let's break down the components:

1. **Probability Density Function (\( f_b \))**: This is a function that describes the relative likelihood or probability of a random variable taking on a given value within a specified range. The area under the curve of the PDF between any two points represents the probability of the random variable falling within that interval.

2. **Interval \([v_0 - 2, v_0 + 2]\)**: This is the domain over which the PDF is defined. It means the PDF is only relevant for values of \( v \) between \( v_0 - 2 \) and \( v_0 + 2 \).

3. **Bit \( b_k \)**: This could be a binary variable (0 or 1) that influences the shape of the PDF. In this case, there are two different PDFs, one for \( b_k = 0 \) and another for \( b_k = 1 \).

4. **Left and Right**: These terms likely refer to the two different PDFs corresponding to \( b_k = 0 \) and \( b_k = 1 \), respectively. The left PDF would represent the distribution when \( b_k = 0 \), and the right PDF would represent the distribution when \( b_k = 1 \).

### Example PDFs

Let's assume the PDFs are Gaussian (normal) distributions centered at \( v_0 \):

- For \( b_k = 0 \):
  \[
  f_{b=0}(v) = \frac{1}{\sqrt{2\pi\sigma^2}} e^{-\frac{(v - v_0)^2}{2\sigma^2}}
  \]
  Here, \( \sigma \) is the standard deviation, which determines the width of the distribution.

- For \( b_k = 1 \):
  \[
  f_{b=1}(v) = \frac{1}{\sqrt{2\pi\sigma^2}} e^{-\frac{(v - (v_0 + \Delta))^2}{2\sigma^2}}
  \]
  Here, \( \Delta \) is a shift parameter that moves the peak of the distribution to the right by \( \Delta \).

### Visualization

If we were to visualize these PDFs, we would see two normal distributions:
- The left PDF \( f_{b=0} \) would have its peak at \( v_0 \).
- The right PDF \( f_{b=1} \) would have its peak at \( v_0 + \Delta \).

The width of both distributions would be determined by \( \sigma \).

### Summary

In summary, the PDF \( f_b \) is a function that describes the probability density of a random variable \( v \) within the interval \([v_0 - 2, v_0 + 2]\), with the shape of the PDF depending on the value of the binary variable \( b_k \). The left PDF corresponds to \( b_k = 0 \), and the right PDF corresponds to \( b_k = 1 \). Both PDFs are likely Gaussian distributions, but they may differ in their mean and/or standard deviation.