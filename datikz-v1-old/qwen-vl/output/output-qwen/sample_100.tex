It seems like you're describing a complex process involving the estimation of differences between two sets of forests, specifically focusing on how the radius changes across different layers and how certain elements are included or excluded in the estimation.

Let's break down the key points:

1. **Radius Doubling and Quadrupling**: 
   - Typically, as you move through the layers (denoted as \(k\)), the radius of the circles representing the forests generally doubles with each decrement in the layer index.
   - However, there is an exception: from the step \(k = hS_N - n + 1\) to \(k = hS_N - n\), the radius quadruples instead of doubling.

2. **New and Old Towers**:
   - The "new tower" \(tw_{S_N}\) is represented by the left circles.
   - The "old tower" \(tw_{S_M}\) is represented by the right circles.

3. **Set \(w_N \cup w_{N-1}\)**:
   - This set is added in the \(N\)-th round to the estimated forest.
   - It is estimated using the grey set, which consists of lower layers of \(tw_{S_M}\).

4. **Subtraction and Ignoring**:
   - Lower layers of \(tw_{S_M}\) (the large balls centered at \(p_{S_M}\)) are subtracted from the estimate.
   - Upper layers of \(tw_{S_M}\) (the small non-concentric circles) are ignored in the estimation.

### Summary:
The process involves comparing two sets of forests (\(tw_{S_N}\) and \(tw_{S_M}\)) and estimating the differences based on their radii and layers. The key difference is that the radius changes quadrupled in one specific range compared to the usual doubling pattern. Additionally, certain layers of \(tw_{S_M}\) are either subtracted or ignored depending on their position relative to the new forest \(tw_{S_N}\).

This description suggests a detailed mathematical or computational model, possibly related to graph theory or network analysis, where the radius and layer changes play crucial roles in defining the structure and differences between the two sets of forests.