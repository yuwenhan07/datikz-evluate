It seems like you're discussing concepts from dynamical systems, particularly focusing on the convergence of intervals \(I_n\) to a nonconstant flow line that connects two critical points in a system. Let's break down the statement and provide some context:

1. **Interval Convergence**: The interval \(I_n\) is likely referring to an interval in the phase space of a dynamical system. As \(n\) increases, these intervals converge to a specific trajectory or flow line.

2. **Nonconstant Flow Line**: This flow line is not constant, meaning it represents a dynamic path through the phase space that changes over time. It connects two critical points, which are points where the vector field (the system's dynamics) is zero or undefined.

3. **Repeating Arguments**: You mention repeating the arguments for \((a_n, b_n')\) and \((a_n', b_n)\). This suggests that the process described for \(I_n\) is being applied iteratively to other intervals or regions within the phase space.

### Detailed Explanation

- **Critical Points**: In a dynamical system, critical points are points where the system's behavior changes qualitatively. For example, in a flow, they can be saddles, nodes, or foci.
  
- **Flow Lines**: These are trajectories that describe how points in the phase space evolve over time according to the system's dynamics. A nonconstant flow line means that the path traced by the system is not a straight line but rather a curve that changes direction as it moves through the phase space.

- **Convergence**: The statement implies that as \(n\) increases, the intervals \(I_n\) get closer and closer to a specific flow line. This convergence is uniform, meaning that the rate at which the intervals approach the flow line does not depend on the initial conditions but is consistent across all points in the interval.

- **Iterative Application**: By applying the same argument to \((a_n, b_n')\) and \((a_n', b_n)\), you are suggesting that the process of finding the flow line connecting critical points is repeated for different regions of the phase space. This could be done to ensure that the flow line is robust and consistent across various parts of the system.

### Example Context

Consider a simple example of a 2D dynamical system with a saddle point at \((0,0)\) and a stable node at \((2,2)\). The flow lines between these points would be curves that connect the saddle to the node. If we start with an interval around the saddle, as we apply the process iteratively, the intervals would converge to one of these flow lines.

In more complex systems, this process might involve numerical methods or analytical techniques to find the flow lines and ensure their uniform convergence.

### Conclusion

The statement you provided is part of a broader discussion about the behavior of dynamical systems, specifically focusing on the convergence of intervals to flow lines connecting critical points. The iterative application of the arguments ensures that the analysis is thorough and covers all relevant regions of the phase space.