The problem involves understanding the structure of a 4-regular graph \( L \) on six vertices and its complement \( L^c \). A 4-regular graph means each vertex in the graph has exactly four edges connected to it.

First, let's consider the properties of \( L \):
- \( L \) is a 4-regular graph with 6 vertices.
- Each vertex in \( L \) has degree 4, meaning each vertex is connected to 4 other vertices.

Next, we need to determine the complement graph \( L^c \). The complement graph \( L^c \) is formed by taking the same set of vertices but connecting two vertices if they were not connected in \( L \), and disconnecting them if they were connected in \( L \).

Given that \( L \) is a 4-regular graph on 6 vertices, the total number of edges in \( L \) can be calculated using the formula for the sum of degrees:
\[
\sum_{v \in V(L)} \deg(v) = 2|E(L)|
\]
Since each vertex has degree 4 and there are 6 vertices:
\[
6 \times 4 = 24 \implies |E(L)| = 12
\]

Now, the number of edges in the complement graph \( L^c \) is:
\[
|E(L^c)| = \binom{6}{2} - |E(L)| = \frac{6 \times 5}{2} - 12 = 15 - 12 = 3
\]

We need to find a configuration where the complement graph \( L^c \) consists of three paths of length 1. A path of length 1 means a single edge between two vertices.

To achieve this, we need to ensure that the complement graph \( L^c \) has exactly 3 edges, and these edges form three separate paths of length 1. This implies that the vertices must be arranged such that no two vertices in any pair of paths share an edge in \( L \).

One possible configuration for \( L \) that satisfies these conditions is a specific arrangement of vertices and edges. However, without loss of generality, we can describe the complement graph \( L^c \) as follows:

Consider the vertices labeled as \( v_1, v_2, v_3, v_4, v_5, v_6 \). In \( L^c \), the edges could be:
\[
(v_1, v_2), (v_3, v_4), (v_5, v_6)
\]

This configuration ensures that each pair of vertices in \( L^c \) forms a path of length 1, and there are no additional edges connecting any of these pairs, thus forming three separate paths of length 1.

Therefore, the complement graph \( L^c \) is the disjoint union of three paths of length 1, which is:
\[
L^c = P_1 \sqcup P_1 \sqcup P_1
\]
where \( P_1 \) represents a path of length 1 (a single edge).

Thus, the final answer is:
\[
\boxed{P_1 \sqcup P_1 \sqcup P_1}
\]