The visual impact of blocking JavaScript (JS) scripts can vary depending on the content and functionality that the JS provides. In your example, where the left side shows a normal website and the right side shows a broken version due to JS blocking, here's what you might observe:

### Normal Website (Left Side):
- **Functionality**: The website likely has interactive elements such as buttons, forms, animations, or dynamic content that changes based on user interaction.
- **Visual Impact**: The page may have smooth transitions, animated effects, or responsive design features that enhance the user experience.
- **User Experience**: Users would be able to interact with the site naturally, and the content would load and display correctly.

### Broken Website (Right Side):
- **Functionality**: The website lacks the interactive elements and dynamic content that were present in the normal version.
- **Visual Impact**: The page will appear static, with no animations, transitions, or responsive design features.
- **User Experience**: Users might find it less engaging and potentially more difficult to navigate if the site relies heavily on JS for its core functionality.

### Example Breakdown:
1. **Static Content**: Without JS, the page might only show text and images without any interactivity.
2. **No Animations**: Smooth scrolling, hover effects, or other visual animations will not work.
3. **Responsive Design**: The layout might not adjust responsively to different screen sizes.
4. **Form Interactions**: Submitting forms might not trigger any feedback or validation messages.
5. **Interactive Elements**: Buttons, sliders, or other interactive components will not function.

### Conclusion:
Blocking JS can significantly alter the visual and functional aspects of a website. It can make the site appear less polished and user-friendly, especially if the site heavily depends on JS for its core features. Users might find the site less engaging and potentially more difficult to use compared to the normal, JS-enabled version.