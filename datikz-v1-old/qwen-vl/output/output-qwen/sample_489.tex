The Gauss diagram you're referring to is a tool used in knot theory to represent a knot or link. It's a way to encode the crossing information of a knot or link into a diagram that can be analyzed mathematically.

### Gauss Diagram:
A Gauss diagram is a planar diagram that represents a knot or link by drawing arcs on a plane (or a sphere) and marking the endpoints with arrows to indicate over/under crossings at those points. Each arc represents a segment of the knot or link, and the arrows at the endpoints show whether the strand goes over or under another strand at each crossing point.

### Interlacement Graph:
The interlacement graph associated with a Gauss diagram is a bipartite graph where the vertices correspond to the strands of the knot or link. Two vertices are connected by an edge if the corresponding strands cross each other in the Gauss diagram. This graph is bipartite because it can be colored with two colors such that no two adjacent vertices share the same color.

### Representation of a Five-Point Star:
If we consider a five-point star (pentagram), we can think of it as a simple example of a link. A five-point star can be represented as a link with 5 strands, and the Gauss diagram would show how these strands cross each other. In this case, the interlacement graph would be a bipartite graph with 5 vertices, representing the 5 strands, and edges connecting pairs of vertices that correspond to crossings in the Gauss diagram.

### Geometric Pattern:
In your description, the "black and white drawing of a sphere with a dot in the middle" could be interpreted as a representation of the Gauss diagram on a sphere. The "lines connecting the points" would then be the arcs of the Gauss diagram, and the "connection between the various elements" would be the edges of the interlacement graph, showing the crossings of the strands.

### Summary:
- **Gauss Diagram**: A planar diagram encoding the crossing information of a knot or link.
- **Interlacement Graph**: A bipartite graph derived from the Gauss diagram, representing the crossings of the strands.
- **Five-Point Star Example**: A simple link with 5 strands, whose Gauss diagram and interlacement graph can be visualized on a sphere.

This geometric pattern you described seems to be a creative visualization of a Gauss diagram and its associated interlacement graph, highlighting the connections between the strands of a knot or link.