The scenario you're describing seems to be related to a specific type of lattice model or graph theory problem, possibly involving the study of vertex configurations and their effects on observables in a lattice system.

Let's break down the key points:

1. **Jumps and Vertex Configuration**: The jumps you mentioned involve moving two vertices from different edges onto the same site \(j_r\). This operation can change the structure of the graph, potentially creating cycles where there were none before.

2. **Cycle Formation**: When these jumps are performed, they can create cycles in the new configuration of particles. A cycle in this context means a closed path where each vertex is visited exactly once except for the starting point, which is also the ending point.

3. **An ansatz observable**: This term suggests that we are considering a specific observable or quantity that depends on the configuration of the particles. The behavior of this observable changes based on the number of vertices involved in the jumps.

4. **Behavior with Different Values of \(k\)**:
   - For \(k = 3\) (as shown in the left figure, where \(k = 4\)), the ansatz observable vanishes. This implies that the configuration resulting from the jumps does not contribute to the observable in any meaningful way.
   - For \(k = 2\) (as shown in the right figure), the ansatz observable is nonzero. This indicates that the configuration resulting from the jumps does contribute to the observable, and its value is non-zero.

### Possible Interpretation

This could be related to a model where the observable is sensitive to the presence of cycles in the graph. For example, consider a model where the observable is the number of cycles in the graph. If the jumps create cycles, it might affect the observable differently depending on the number of vertices involved.

- When \(k = 3\), the jumps might not create enough cycles to significantly alter the observable, leading to it vanishing.
- When \(k = 2\), the jumps might create cycles that do have an impact on the observable, making it nonzero.

### Conclusion

The behavior of the ansatz observable being zero or nonzero when \(k = 3\) versus \(k = 2\) suggests that the number of vertices involved in the jumps affects whether cycles are formed and how they influence the observable. This kind of analysis is common in studies of lattice models and graph theory, particularly in contexts like percolation, statistical mechanics, or network analysis.