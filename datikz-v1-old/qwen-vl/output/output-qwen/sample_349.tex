To illustrate the relationship between the words in an equivalence class using swaps, we can represent this as a directed graph where each vertex represents a word in the equivalence class, and there is a directed edge from one vertex to another if the two words are related by a swap.

Here's a step-by-step explanation:

1. **Identify the Equivalence Class**: Start with a set of words that belong to the same equivalence class. For example, let's consider the equivalence class of words formed by the permutation group \( S_3 \) acting on the set \(\{a, b, c\}\). The equivalence class might include words like "abc", "acb", "bac", "bca", "cab", and "cba".

2. **Define the Swap Operation**: A swap operation is a transposition, which is a permutation that exchanges two elements while leaving the others fixed. For example, the swap "ab" -> "ba" is a transposition.

3. **Construct the Graph**:
   - Each word in the equivalence class is a vertex.
   - Draw a directed edge from vertex \( w_1 \) to vertex \( w_2 \) if \( w_2 \) can be obtained from \( w_1 \) by a single swap.

Let's construct the graph for the equivalence class of permutations of \(\{a, b, c\}\):

- Vertices: "abc", "acb", "bac", "bca", "cab", "cba"
- Directed Edges:
  - From "abc" to "bac" (swap 'a' and 'b')
  - From "abc" to "bca" (swap 'a' and 'c')
  - From "abc" to "cab" (swap 'b' and 'c')
  - From "acb" to "abc" (swap 'a' and 'b')
  - From "acb" to "bca" (swap 'a' and 'c')
  - From "acb" to "cab" (swap 'b' and 'c')
  - From "bac" to "abc" (swap 'a' and 'b')
  - From "bac" to "bca" (swap 'a' and 'c')
  - From "bac" to "cab" (swap 'b' and 'c')
  - From "bca" to "abc" (swap 'a' and 'b')
  - From "bca" to "acb" (swap 'a' and 'c')
  - From "bca" to "cab" (swap 'b' and 'c')
  - From "cab" to "abc" (swap 'a' and 'b')
  - From "cab" to "acb" (swap 'a' and 'c')
  - From "cab" to "bca" (swap 'b' and 'c')
  - From "cba" to "abc" (swap 'a' and 'b')
  - From "cba" to "acb" (swap 'a' and 'c')
  - From "cba" to "bac" (swap 'b' and 'c')

The resulting graph will be a directed graph where each vertex is connected to exactly three other vertices, forming a cycle. This cycle represents the fact that any permutation in the equivalence class can be transformed into any other permutation in the class through a series of swaps.

The final answer is the directed graph representing the equivalence class of permutations of \(\{a, b, c\}\):

\[
\boxed{
\begin{array}{ccc}
\text{Vertices} & \text{Directed Edges} \\
\hline
\text{"abc"} & \text{"bac", "bca", "cab"} \\
\text{"acb"} & \text{"abc", "bca", "cab"} \\
\text{"bac"} & \text{"abc", "bca", "cab"} \\
\text{"bca"} & \text{"abc", "acb", "cab"} \\
\text{"cab"} & \text{"abc", "acb", "bca"} \\
\text{"cba"} & \text{"abc", "acb", "bac"} \\
\end{array}
}
\]

This graph shows that each permutation can be transformed into any other permutation in the class through a series of swaps, forming a cycle.