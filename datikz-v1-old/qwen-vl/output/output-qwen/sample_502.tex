To provide a detailed explanation, we need to clarify what you mean by "Petri net" and the context of processes \( C \) and \( B \). Petri nets are a graphical and mathematical modeling language used to describe and analyze concurrent systems. They consist of places (circles), transitions (rectangles), and directed arcs connecting places to transitions and vice versa.

Let's assume that:

- \( C \) and \( B \) are two different processes or activities.
- \( (a) \), \( (b) \), and \( (c) \) refer to different configurations or states of these processes.
- \( (C B) \) represents the parallel execution of \( C \) and \( B \).
- \( (C B)_h \) might represent a specific configuration or state where \( C \) and \( B \) are in some particular relationship or have some specific constraints.
- \( (B B)_h \) might represent another specific configuration or state where both \( B \) and \( B \) (possibly indicating a loop or repetition of \( B \)) are in some specific relationship or have some specific constraints.

Given this context, let's construct the Petri nets step-by-step:

### Petri Net for \( (C B) \)

1. **Places**: 
   - \( P_C \): Place representing process \( C \).
   - \( P_B \): Place representing process \( B \).

2. **Transitions**:
   - \( T_{start} \): Transition starting the execution of \( C \) and \( B \).
   - \( T_{end} \): Transition ending the execution of \( C \) and \( B \).

3. **Arcs**:
   - \( P_{start} \rightarrow T_{start} \)
   - \( T_{start} \rightarrow P_C \)
   - \( T_{start} \rightarrow P_B \)
   - \( P_C \rightarrow T_{end} \)
   - \( P_B \rightarrow T_{end} \)
   - \( T_{end} \rightarrow P_{end} \)

### Petri Net for \( (C B)_h \)

Assuming \( (C B)_h \) has some specific constraints or relationships between \( C \) and \( B \):

1. **Places**:
   - \( P_C \): Place representing process \( C \).
   - \( P_B \): Place representing process \( B \).
   - \( P_h \): Place representing the specific constraint or relationship \( h \).

2. **Transitions**:
   - \( T_{start} \): Transition starting the execution of \( C \) and \( B \).
   - \( T_{end} \): Transition ending the execution of \( C \) and \( B \).

3. **Arcs**:
   - \( P_{start} \rightarrow T_{start} \)
   - \( T_{start} \rightarrow P_C \)
   - \( T_{start} \rightarrow P_B \)
   - \( P_C \rightarrow T_{end} \)
   - \( P_B \rightarrow T_{end} \)
   - \( T_{start} \rightarrow P_h \)
   - \( P_h \rightarrow T_{end} \)
   - \( T_{end} \rightarrow P_{end} \)

### Petri Net for \( (B B)_h \)

Assuming \( (B B)_h \) has some specific constraints or relationships between \( B \) and \( B \):

1. **Places**:
   - \( P_B \): Place representing process \( B \).
   - \( P_h \): Place representing the specific constraint or relationship \( h \).

2. **Transitions**:
   - \( T_{start} \): Transition starting the execution of \( B \).
   - \( T_{end} \): Transition ending the execution of \( B \).

3. **Arcs**:
   - \( P_{start} \rightarrow T_{start} \)
   - \( T_{start} \rightarrow P_B \)
   - \( P_B \rightarrow T_{end} \)
   - \( T_{end} \rightarrow P_{end} \)
   - \( T_{start} \rightarrow P_h \)
   - \( P_h \rightarrow T_{end} \)

These are basic Petri net representations. The actual implementation would depend on the specific constraints and relationships described by \( h \). If there are more complex interactions or specific conditions, additional places and transitions would be necessary to model those accurately.