PICV (Particle Image Velocimetry) is a non-intrusive optical technique used to measure the velocity field of a fluid flow. It has found widespread application across various fields due to its ability to provide detailed information about fluid dynamics without disturbing the flow. Here's a breakdown of the application domains mentioned in your query:

1. **Fluid and Solid Mechanics**: This is indeed the most applied domain for PICV. It is extensively used in aerospace engineering, automotive industry, and environmental studies to analyze airflow around objects, turbulence, boundary layer effects, and other fluid dynamics phenomena.

2. **Imaging and Photonics Applications**: PICV is also widely used in the field of imaging and photonics. It helps in understanding light propagation through materials, studying the behavior of light in different mediums, and analyzing the interaction between light and matter.

3. **Robotics**: While robotics has been less capable in adapting PICV techniques directly for problem-solving, there are some emerging applications. For instance, PICV can be used in the development of advanced robotic systems that require precise control over fluid dynamics, such as underwater robots or microfluidic devices. However, the integration of PICV with robotics is still an area of active research and development.

4. **Other Domains**: Other domains where PICV has shown potential include biomedical engineering, where it can be used to study blood flow in the human body, and geophysics, where it can help in understanding fluid dynamics in geological processes like groundwater flow.

In summary, while fluid and solid mechanics remain the primary domain for PICV applications, the technique is increasingly being explored in imaging and photonics, and there is growing interest in its use within robotics and other interdisciplinary fields.