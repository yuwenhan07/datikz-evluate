The Nautilus spiral you're referring to is indeed a fascinating subject in both mathematics and art. The spiral is often associated with the growth patterns of the Nautilus shell, which is why it's sometimes called the "golden spiral" or "logarithmic spiral." This type of spiral is characterized by its constant angle and increasing radius as it spirals outward.

Mathematically, the equation for a logarithmic spiral can be expressed as:

\[ r = ae^{b\theta} \]

where \( r \) is the distance from the origin, \( \theta \) is the angle from the positive x-axis, and \( a \) and \( b \) are constants that determine the size and shape of the spiral.

In the context of art, the Nautilus spiral has been used extensively due to its aesthetic appeal and the way it represents growth and expansion. Artists and designers often use this spiral to create visually appealing compositions, as it naturally draws the eye towards the center while maintaining a sense of balance and harmony.

The spiral's appearance in nature, such as in the shell of a nautilus, is believed to be an evolutionary adaptation that allows the creature to grow without changing the shape of its shell. This is because the logarithmic spiral maintains a constant proportion between the shell's width and height at any point along its length.

In summary, the Nautilus spiral is a beautiful example of how mathematical concepts can be translated into art, creating a harmonious blend of science and aesthetics.