The image you've described sounds like it could be depicting a fractal known as the Sierpinski triangle, which is indeed a type of anti-tree or a tree-like structure that has been inverted or negated in its traditional form. However, the term "anti-tree" isn't commonly used to describe this specific fractal.

The Sierpinski triangle is a well-known fractal pattern that starts with a single triangle and recursively removes smaller triangles from the remaining ones. This process creates a pattern where each iteration reveals more detail, leading to an infinite complexity at every level.

If you're referring to a different type of anti-tree or a similar concept, please provide more details or clarify your question. If the image is indeed a Sierpinski triangle, here are some key points about it:

1. **Self-Similarity**: The Sierpinski triangle is self-similar, meaning that each part of the triangle looks like a smaller version of the whole.
2. **Recursive Construction**: It can be constructed by starting with a large equilateral triangle and repeatedly removing smaller triangles from the remaining parts.
3. **Mathematical Properties**: The Sierpinski triangle has a fractal dimension of approximately 1.585, indicating that it lies between a one-dimensional line and a two-dimensional plane.
4. **Applications**: Fractals like the Sierpinski triangle have applications in various fields, including computer graphics, signal processing, and even in understanding natural phenomena such as coastlines and clouds.

If you need further clarification or have another specific question related to this fractal or any other topic, feel free to ask!