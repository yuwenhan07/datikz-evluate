The image you described consists of two gray rectangles on a white background. Given the context provided, these rectangles could indeed be interpreted as representing the pages of a book or a pair of open books with their covers closed. However, the mention of filling the gaps between random lines with opposite polarity to illustrate that if a magnet is broken, the pieces remain polar suggests a different interpretation.

In this context, the gray rectangles might symbolize the two poles (north and south) of a magnet. When a magnet breaks into pieces, each piece retains its own north and south poles. This concept can be visualized by drawing lines between the rectangles to represent the magnetic field lines, which would naturally curve from one pole to the other, creating a continuous loop around the rectangles.

Here's a step-by-step breakdown:

1. **Identify the Rectangles**: The two gray rectangles represent the two poles of a magnet.
2. **Draw Magnetic Field Lines**: Draw curved lines between the rectangles to represent the magnetic field lines. These lines should start at the north pole of one rectangle and end at the south pole of the other rectangle, forming a continuous loop.
3. **Opposite Polarity**: Ensure that the lines alternate in direction, indicating the opposite polarity between the rectangles.

This visualization helps to illustrate the concept that even when a magnet is broken, each piece retains its own north and south poles, maintaining the overall magnetic field pattern.

So, the final image would look something like this:
- Two gray rectangles on a white background.
- Curved lines drawn between the rectangles, alternating directions to represent the opposite polarity.

This approach effectively conveys the idea of magnetic poles and their behavior when a magnet is broken.