In the context of simulating abelian exchange statistics using anyons, it is important to understand how the phase factors associated with the exchange of anyons are determined by their string configurations. For semions (which have abelian exchange statistics), the phase factor is given by the product of the phases associated with the individual segments of the anyon's string.

Let's break down the scenario you've described:

1. **Strings and Phases**: Anyons can be represented as strings in a two-dimensional space. These strings can be decomposed into vertical and horizontal segments. The phase associated with each segment is \(\pi/2\).

2. **Vertical Segment**: If an anyon has a vertical segment, the phase associated with this segment is \(\pi/2\). This means that when two such anyons exchange positions, the overall phase change due to the vertical segment is \(e^{i\pi/2} = i\).

3. **Horizontal Segment**: Similarly, if an anyon has a horizontal segment, the phase associated with this segment is also \(\pi/2\). When two such anyons exchange positions, the overall phase change due to the horizontal segment is \(e^{i\pi/2} = i\).

4. **Total Phase Change**: Since the exchange of anyons involves both vertical and horizontal segments, the total phase change is the product of the phases from each segment. Therefore, the total phase change for the exchange of two semions is:
   \[
   e^{i\pi/2} \cdot e^{i\pi/2} = e^{i\pi} = -1
   \]
   This indicates that the exchange of two semions results in a phase factor of \(-1\), which is characteristic of abelian exchange statistics.

### Summary:
- Each vertical or horizontal segment of a semion's string contributes a phase of \(\pi/2\).
- The total phase change for the exchange of two semions is \(-1\), indicating abelian exchange statistics.
- This phase factor is consistent with the requirement for simulating abelian exchange statistics using anyons.

This approach ensures that the anyons exhibit the correct statistical behavior under exchange, which is essential for various quantum computing and topological quantum computation applications.