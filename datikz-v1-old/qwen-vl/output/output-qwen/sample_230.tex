It seems you've described a visual representation of a hexagon with numbered vertices, but there's no mention of a flow diagram or a function \( f \) with a Möbius inverse \( g \). Let's clarify what you're asking about:

1. **Flow Diagram**: Typically, a flow diagram (or flowchart) is a graphical representation of a process or algorithm using standard symbols and arrows to show the sequence of steps. It doesn't usually involve numerical labels on a geometric shape like a hexagon.

2. **Function \( f \) with Möbius Inverse \( g \)**: A Möbius transformation (or Möbius function) is a specific type of function that maps complex numbers to other complex numbers in a way that preserves angles and ratios of distances. If \( f \) has a Möbius inverse \( g \), it means that applying \( f \) followed by \( g \) (or vice versa) results in the identity function.

Given your description of a hexagon with numbered vertices, it sounds more like a geometric or topological problem rather than a functional one. If you have a specific question related to this hexagon or if you need help with a particular aspect of functions and their inverses, please provide more details so I can assist you better.