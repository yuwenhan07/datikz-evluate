The image you've described appears to be related to the study of Calabi-Yau threefolds and their mirror symmetry, which is a fundamental concept in string theory and algebraic geometry.

1. **Left: The Dual Graph for Singularity \( \text{III}_n \)**:
   - The left side of the image shows the dual graph of the singularity \( \text{III}_n \). In algebraic geometry, a singularity is a point where the geometric structure is not smooth. The singularity \( \text{III}_n \) is a type of rational double point that arises in the classification of singularities.
   - The dual graph is a way to represent the resolution of the singularity. Each vertex in the graph corresponds to a exceptional divisor (a curve that is contracted to a point during the resolution process), and each edge represents the intersection between these divisors. For the singularity \( \text{III}_n \), the dual graph typically looks like a tree with \( n \) vertices, where each vertex has degree 3.

2. **Right: The 3d Mirror for the Low Energy Theory Associated with the Singular Fiber \( \text{III}_n \)**:
   - The right side of the image represents the 3-dimensional mirror of the low-energy theory associated with the singular fiber \( \text{III}_n \). In mirror symmetry, two Calabi-Yau manifolds are said to be mirror partners if they have equivalent physical properties at the quantum level, despite having different geometries.
   - The mirror manifold is often simpler or more symmetric than the original manifold, making it easier to study certain aspects of the physics. For the singularity \( \text{III}_n \), its mirror can be another Calabi-Yau threefold with a different but equivalent set of physical properties.

In summary, the left part of the image describes the geometric structure of the singularity \( \text{III}_n \) through its dual graph, while the right part describes the corresponding mirror manifold, which is a key concept in understanding the physics associated with this singularity.