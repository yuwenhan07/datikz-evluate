To analyze the performance of your parallel algorithm in terms of running time (in minutes) as you vary the alphabet size, we need to interpret the information provided by the image. Here's a step-by-step approach:

1. **Identify the Axes**: Typically, in such bar charts, one axis represents the variable being measured (in this case, the alphabet size), and the other axis represents the value being measured (running time in minutes). The bars represent different categories or conditions.

2. **Interpret the Colors**: The colors (red, blue, and green) likely represent different conditions or algorithms being compared. For instance:
   - Red might represent the baseline or standard algorithm.
   - Blue might represent an optimized version of the algorithm.
   - Green might represent another variant or a different approach.

3. **Analyze the Bars**:
   - Each color group of bars corresponds to a specific condition or algorithm.
   - The height of the bars indicates the running time in minutes for that particular condition at a given alphabet size.

4. **Compare Conditions**: By comparing the heights of the bars within each color group, you can determine which condition performs better (i.e., has a shorter running time) for a given alphabet size.

5. **Trend Analysis**: Look for trends across different alphabet sizes. For example, does the running time generally increase or decrease with increasing alphabet size? Are there any anomalies where the running time behaves differently?

6. **Conclusion**: Based on the analysis, you can draw conclusions about the performance of each algorithm under different conditions and how it scales with the alphabet size.

If you provide more details or if the image is available, I can offer a more detailed analysis.