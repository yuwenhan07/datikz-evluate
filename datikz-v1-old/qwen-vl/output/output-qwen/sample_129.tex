To shrink the interval \([5, 499]\) in the x-axis while preserving the positive slope on the left and the negative slope on the right, we need to transform the x-coordinates in such a way that the overall shape of the function is maintained but the interval is compressed.

Let's assume the original function is \(y = f(x)\). We want to transform this function into a new function \(y = g(x)\) where the x-coordinates are scaled down. A common approach is to use a linear transformation that maps the interval \([5, 499]\) to a smaller interval, say \([0, 1]\), while preserving the slopes.

The general form of the transformation can be written as:
\[ x' = \frac{x - a}{b - a} \]
where \(a\) and \(b\) are the endpoints of the original interval. For our interval \([5, 499]\), we have \(a = 5\) and \(b = 499\). Therefore, the transformation becomes:
\[ x' = \frac{x - 5}{499 - 5} = \frac{x - 5}{494} \]

This transformation will map the interval \([5, 499]\) to the interval \([0, 1]\). Now, if the original function is \(y = f(x)\), the new function \(y = g(x')\) will be:
\[ y = f\left(\frac{494x'}{1} + 5\right) = f(494x' + 5) \]

So, the resulting graph of the function \(y = f(x)\) after shrinking the interval \([5, 499]\) in the x-axis while preserving the positive slope on the left and the negative slope on the right is:
\[ \boxed{y = f(494x + 5)} \]

This transformation ensures that the function retains its original shape but is now defined over the interval \([0, 1]\) instead of \([5, 499]\).