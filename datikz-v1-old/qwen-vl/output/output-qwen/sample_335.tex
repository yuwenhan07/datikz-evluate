The SUT-G generator is a tool designed to solve SAT (Boolean satisfiability) problems by breaking them down into smaller, more manageable subproblems. Here's a summary diagram of the process:

1. **Input**: A SAT instance is provided as input.
2. **Identify Clauses**: The SAT instance is analyzed to identify all the clauses (constraints).
3. **Define SUT**: The Set of Unknown Variables (SUT) is defined based on the variables involved in the clauses.
4. **Build SUT Parameters**: Parameters for the SUT are constructed from the variables, which will be used to generate subproblems.
5. **Generate Subproblems**: Using the SUT parameters, the SAT instance is divided into multiple subproblems.
6. **Solve Subproblems**: Each subproblem is solved independently.
7. **Combine Solutions**: Solutions from the subproblems are combined to form a solution for the original SAT instance.

This process ensures that the SAT problem is broken down into smaller, more tractable parts, making it easier to find a satisfiable assignment.