To provide a comprehensive answer, let's break down the notation and concepts involved in your question:

1. **Notation and Definitions**:
   - \(\lambda_2, 3^{(0)}(\alpha)\): This is likely referring to a specific type of exponential sum or a related quantity in analytic number theory.
   - \(\lambda_2, 3^{\alpha}(\alpha)\): This could be another variant or a different function related to the same concept but with a parameter \(\alpha\).
   - \(N^{-}\): This typically denotes a large number or a variable that is inversely proportional to some other quantity.
   - Wooley's conjecture (denoted as Wool3): This refers to a conjecture by Trevor Wooley, a mathematician known for his work in analytic number theory.

2. **Context and Conjectures**:
   - The conjecture Wool3 by Trevor Wooley is often associated with estimates of exponential sums and their applications in number theory. Specifically, it deals with the estimation of certain types of exponential sums over arithmetic progressions.

3. **Upper Bounds and Conjectures**:
   - Upper bounds in this context refer to the best possible estimates or limits that can be placed on the magnitude of these quantities. These bounds are crucial in understanding the behavior of the functions \(\lambda_2, 3^{(0)}(\alpha)\) and \(\lambda_2, 3^{\alpha}(\alpha)\) as \(\alpha\) varies and \(N^{-}\) becomes large.

4. **Comparison and Values of \(\alpha\)**:
   - The comparison of upper bounds and conjectures for various values of \(\alpha\) and \(N^{-}\) involves analyzing how these quantities behave under different conditions. For instance, if \(\alpha = 2/3\), then we are dealing with a specific case where the exponent is fixed at \(2/3\).

5. **Specific Case \(\alpha = 2/3\)**:
   - If \(\alpha = 2/3\), then the conjecture Wool3 would apply directly to the quantity \(\lambda_2, 3^{(0)}(2/3)\). This means that the upper bound for \(\lambda_2, 3^{(0)}(2/3)\) would be estimated using the techniques and results from Wooley's conjecture.

6. **General Analysis**:
   - To compare upper bounds and conjectures for different values of \(\alpha\), one would need to analyze the behavior of the functions \(\lambda_2, 3^{(0)}(\alpha)\) and \(\lambda_2, 3^{\alpha}(\alpha)\) as \(\alpha\) varies. This typically involves advanced techniques from analytic number theory, such as the use of the Hardy-Littlewood circle method, Vinogradov's methods, and other tools developed by mathematicians like Wooley.

In summary, the comparison of upper bounds and conjectures for \(\lambda_2, 3^{(0)}(\alpha)\) and \(\lambda_2, 3^{\alpha}(\alpha)\) for various values of \(\alpha\) and \(N^{-}\) involves detailed analysis using advanced number-theoretic techniques. If \(\alpha = 2/3\), then the conjecture Wool3 provides a specific framework for estimating the upper bounds of \(\lambda_2, 3^{(0)}(2/3)\).