To solve this problem, we need to understand the structure of the graph and the properties of an Eulerian circuit. An Eulerian circuit is a closed walk that visits every edge exactly once. In the context of the given problem, we have a graph with vertices \(A\), \(B\), \(C\), and \(A'\), \(B'\), \(C'\) and edges \(AC'\), \(BC\), and \(A'B'\).

First, let's visualize the graph:
- There are two sets of vertices: \(\{A, B, C\}\) and \(\{A', B', C'\}\).
- The edges are \(AC'\), \(BC\), and \(A'B'\).

We need to determine if there is only one possible way for an Eulerian circuit to pass through the internal 3-cycle formed by the vertices \(A\), \(B\), and \(C\). The internal 3-cycle here refers to the cycle involving the vertices \(A\), \(B\), and \(C\) in some order.

Since the graph is bipartite (the vertices can be divided into two disjoint sets such that no two vertices within the same set are adjacent), it does not contain any odd cycles. Therefore, the graph itself is Eulerian, meaning it has an Eulerian circuit. However, we need to consider the specific path through the 3-cycle involving \(A\), \(B\), and \(C\).

The edges connecting these vertices are \(AC'\), \(BC\), and \(A'B'\). To form an Eulerian circuit, we need to ensure that each vertex is visited exactly twice (once entering and once leaving). Let's analyze the possible paths:

1. Start at \(A\). The next vertex could be \(C'\) or \(B\).
2. If we go to \(C'\), then we need to return to \(A\) via \(A'C'\) and then visit \(B\) via \(BC\). This gives us the sequence \(A \to C' \to A' \to B \to C \to B'\).
3. If we go to \(B\), then we need to return to \(A\) via \(AB\) and then visit \(C\) via \(CA\). This gives us the sequence \(A \to B \to A' \to C \to C'\).

These are the only two possible ways to form an Eulerian circuit that passes through the 3-cycle involving \(A\), \(B\), and \(C\). Therefore, there is only one possible way for the circuit to pass through the internal 3-cycle.

The answer is \(\boxed{1}\).