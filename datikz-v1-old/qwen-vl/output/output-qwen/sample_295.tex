To address the problem of finding a counterexample for the "2 out of 3" property in a graph, we need to understand what this property means. The "2 out of 3" property typically refers to a specific condition in graph theory where every set of three vertices has at least two edges connecting them. If such a property does not hold, then there must be some set of three vertices that do not satisfy this condition.

Let's consider a simple example to illustrate this:

Consider a graph with four vertices labeled \(A\), \(B\), \(C\), and \(D\). Suppose the edges in the graph are as follows:
- \(AB\)
- \(AC\)
- \(AD\)

This graph looks like a star graph with center \(A\) and leaves \(B\), \(C\), and \(D\). Let's check if it satisfies the "2 out of 3" property.

1. Consider the set \(\{A, B, C\}\):
   - There are edges \(AB\) and \(AC\).
   - This set has 2 edges, so it satisfies the "2 out of 3" property.

2. Consider the set \(\{A, B, D\}\):
   - There are edges \(AB\) and \(AD\).
   - This set has 2 edges, so it satisfies the "2 out of 3" property.

3. Consider the set \(\{A, C, D\}\):
   - There are edges \(AC\) and \(AD\).
   - This set has 2 edges, so it satisfies the "2 out of 3" property.

4. Consider the set \(\{B, C, D\}\):
   - There are no edges between \(B\), \(C\), and \(D\).
   - This set has 0 edges, which is less than 2.
   - Therefore, this set does not satisfy the "2 out of 3" property.

Since the set \(\{B, C, D\}\) does not have at least 2 edges, the graph does not satisfy the "2 out of 3" property. Thus, the graph provided as an example is a counterexample for the "2 out of 3" property.

The final answer is:
\[
\boxed{\{B, C, D\}}
\]