It seems you're describing a visual pattern rather than a mathematical problem. However, if we interpret this as a question about the substitution rule in calculus or a similar context, it might be referring to the substitution method used in integration.

The substitution rule (also known as the chain rule in reverse) is a technique used to evaluate integrals by making a substitution that simplifies the integral. Here's a brief overview:

### Substitution Rule in Integration

Given an integral of the form:
\[ \int f(g(x)) g'(x) \, dx \]

We can use the substitution \( u = g(x) \). Then, \( du = g'(x) \, dx \).

The integral becomes:
\[ \int f(u) \, du \]

This new integral is often easier to solve and then we substitute back \( u = g(x) \) to get the final answer in terms of \( x \).

### Example

Consider the integral:
\[ \int 2x \sqrt{x^2 + 1} \, dx \]

Let \( u = x^2 + 1 \). Then, \( du = 2x \, dx \).

Substituting \( u \) and \( du \) into the integral, we get:
\[ \int \sqrt{u} \, du \]

This is now a simpler integral:
\[ \int u^{1/2} \, du \]

Integrating \( u^{1/2} \):
\[ \frac{2}{3} u^{3/2} + C \]

Substitute back \( u = x^2 + 1 \):
\[ \frac{2}{3} (x^2 + 1)^{3/2} + C \]

So, the integral evaluates to:
\[ \frac{2}{3} (x^2 + 1)^{3/2} + C \]

This is a simple example of using the substitution rule to simplify an integral. If your question was about something else, please provide more details so I can assist you better!