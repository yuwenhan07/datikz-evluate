It sounds like you're discussing the performance of six different machine learning algorithms when compared to the MOBIL model's predictions for two scenarios: right-lane and left-lane predictions. The table you mentioned likely shows the correlation coefficients between these predictions, indicating how closely they align with each other.

Here’s a general format you might use to present such data:

| Algorithm | Right Lane Correlation | Left Lane Correlation |
|-----------|------------------------|-----------------------|
| Algorithm 1 | 0.85                   | 0.79                  |
| Algorithm 2 | 0.82                   | 0.81                  |
| Algorithm 3 | 0.87                   | 0.84                  |
| Algorithm 4 | 0.83                   | 0.80                  |
| Algorithm 5 | 0.86                   | 0.83                  |
| Algorithm 6 | 0.84                   | 0.78                  |

### Key Points:
- **Correlation Coefficients**: These values range from -1 to +1. A value close to +1 indicates a strong positive correlation, meaning that as one prediction increases, the other also tends to increase. Conversely, a value close to -1 indicates a strong negative correlation.
- **High Correlation**: In your case, the correlations are generally high, suggesting that the predictions from the machine learning algorithms are quite consistent with those from the MOBIL model.
- **Right vs. Left Lane**: The correlation coefficients for the right and left lanes can differ, which might indicate some variability in the performance of the algorithms across different lanes.

If you have specific questions or need further analysis, feel free to ask!