The figure you're describing appears to illustrate a concept from control theory or dynamical systems, particularly focusing on stability analysis using a Lyapunov function. Here's a breakdown of the elements:

1. **Set \(X\)**:
   - Defined in equation (eq:X).
   - Illustrated as the strip between two horizontal lines.
   - This represents a region in the state space where the system dynamics are of interest.

2. **Set \(O_{n-1}\)**:
   - Defined in equation (eq:On-1).
   - Illustrated as the hatched box.
   - This is likely a region that has been identified as a stable invariant set or an attractor for the system at some previous time step \(n-1\).

3. **Lyapunov Function \(V(x)\)**:
   - The Lyapunov function \(V(x)\) is a scalar function of the state vector \(x\) that is used to assess the stability of the system.
   - The smaller ellipse represents the largest level set of \(V(x)\) that is entirely contained within \(X\). This means that any trajectory starting within this ellipse will remain within \(X\), indicating that it is a region of attraction.
   - The larger ellipse represents the smallest level set of \(V(x)\) that encloses \(O_{n-1}\). This indicates that \(O_{n-1}\) is a stable set because trajectories starting within this larger ellipse will converge to \(O_{n-1}\).

### Key Idea Behind the Second Method:
The second method likely refers to a technique for proving the stability of a system using a Lyapunov function. Here’s how it works:

1. **Positive Definiteness**: The Lyapunov function \(V(x)\) must be positive definite, meaning \(V(x) > 0\) for all \(x \neq 0\) and \(V(0) = 0\). This ensures that the function is well-behaved and can be used to assess stability.

2. **Negative Definite Derivative**: The derivative of the Lyapunov function along the trajectories of the system must be negative definite. That is, \(\dot{V}(x) < 0\) for all \(x\) in the region of interest. This condition implies that the energy of the system decreases over time, leading to convergence to a stable equilibrium point.

3. **Level Sets**: The level sets of the Lyapunov function provide a way to visualize the regions of attraction and stability. The smaller ellipse (largest level set within \(X\)) indicates the region from which trajectories will not leave \(X\), while the larger ellipse (smallest level set enclosing \(O_{n-1}\)) shows that \(O_{n-1}\) is a stable set because trajectories starting within this ellipse will converge to \(O_{n-1}\).

In summary, the figure illustrates how the Lyapunov function helps to identify regions of attraction and stability by defining level sets that bound the system's behavior. The smaller ellipse within \(X\) shows the region of attraction, while the larger ellipse enclosing \(O_{n-1}\) confirms that \(O_{n-1}\) is a stable set.