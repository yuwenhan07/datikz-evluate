The comparison between RIS-enabled cooperative localization and beacon-aided cooperative localization under different RIS sizes and a fixed random phase profile setup involves evaluating the performance of these two methods in terms of accuracy, robustness, and computational complexity.

### Key Points:

1. **RIS-Enabled Cooperative Localization**:
   - **Concept**: RIS (Reconfigurable Intelligent Surface) is used to enhance the signal propagation by reflecting or steering the signals towards the target.
   - **Performance**: With a sufficient number of RIS elements, the system can achieve similar performance as when using active anchors. This means that the presence of RIS elements can replace active anchors without significant performance degradation.
   - **Advantages**: 
     - Reduced energy consumption since RIS elements do not require active power.
     - Potential for scalability due to the flexible nature of RIS.
   - **Challenges**: 
     - Designing the optimal phase profile for RIS elements.
     - Ensuring robustness against channel variations and multipath effects.

2. **Beacon-Aided Cooperative Localization**:
   - **Concept**: Beacons are used to provide additional reference points for localization. These beacons can be either active or passive.
   - **Performance**: The performance depends on the number and distribution of beacons. With enough beacons, the system can achieve high accuracy.
   - **Advantages**: 
     - Robust against multipath effects.
     - Can be easily deployed with minimal infrastructure.
   - **Challenges**: 
     - Increased energy consumption if beacons are active.
     - Complexity in beacon deployment and maintenance.

### Comparison Across Different RIS Sizes:

- **Small RIS Size**: 
  - Performance may degrade significantly compared to larger RIS sizes because fewer RIS elements limit the flexibility in signal manipulation.
  - Active anchors might still be necessary to achieve acceptable accuracy.

- **Medium RIS Size**: 
  - Performance improves but may still not match that of large RIS sizes.
  - Active anchors can often be partially replaced, but full replacement might not be feasible without significant performance loss.

- **Large RIS Size**: 
  - Performance approaches that of systems with active anchors.
  - Active anchors can be fully replaced with RIS elements without noticeable performance degradation.

### Fixed Random Phase Profile Setup:

- **Impact**: A fixed random phase profile setup implies that the phase shifts applied to the signals are randomly distributed but remain constant across all RIS elements.
- **Effectiveness**: The effectiveness of RIS in replacing active anchors depends on how well the random phase profile can be optimized to steer and reflect signals effectively.
- **Optimization**: Techniques such as machine learning or genetic algorithms can be employed to optimize the phase profile for better performance.

### Conclusion:

- For small RIS sizes, beacon-aided localization might be more effective due to its robustness and ease of deployment.
- As the RIS size increases, RIS-enabled cooperative localization becomes more competitive and can potentially outperform beacon-aided localization, especially when the number of RIS elements is sufficiently large.
- The optimal choice between RIS-enabled and beacon-aided localization will depend on the specific application requirements, including the need for energy efficiency, scalability, and robustness against channel variations.