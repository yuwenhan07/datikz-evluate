The description you've provided seems to be a mix of visual and mathematical elements, but it lacks specific details about the comparison between the function \( f \) (the continuous line) and the distance from \( S \) (the dashed line). Additionally, there's no clear indication of what \( X = \mathbb{R} \) signifies in this context.

However, I can provide a general explanation based on your description:

1. **Function \( f \) vs Distance from \( S \)**:
   - If we consider \( X = \mathbb{R} \), we might be looking at a real-valued function \( f: \mathbb{R} \to \mathbb{R} \).
   - The dashed line could represent another function \( S(x) \) that measures some kind of distance or separation from a reference point or set.
   - The comparison would typically involve analyzing how these two functions behave with respect to each other over the domain \( \mathbb{R} \).

2. **White Arrow on Black Background**:
   - The white arrow pointing upwards suggests an increasing trend or behavior of one of the functions as \( x \) increases. This could indicate that either \( f(x) \) or \( S(x) \) is increasing.

3. **Numbers Alongside the Arrow**:
   - These numbers likely represent values of the functions at certain points or parameters. They could be used to illustrate specific points of interest, such as where the functions intersect, their maximum or minimum values, or critical points.

Without more specific information, it's challenging to provide a detailed analysis. However, if you have a specific function \( f(x) \) and \( S(x) \) in mind, or if you can provide more details about the context, I can help you analyze them further.