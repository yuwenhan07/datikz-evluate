To address your question regarding the proof that \( V_0 = 0 \) and the embedding property of the path, let's break down the steps and reasoning involved:

### Proving \( V_0 = 0 \)

The statement "To prove \( V_0 = 0 \), we follow the path from -1 to (5), then the dotted line to (1), dotted line to (6), then complete the path to +1" suggests a specific sequence of points or steps in a graph or diagram.

1. **Path from -1 to (5):**
   - Start at point \(-1\).
   - Follow the path to point \((5)\).

2. **Dotted line to (1):**
   - From point \((5)\), follow the dotted line to point \((1)\).

3. **Dotted line to (6):**
   - From point \((1)\), follow the dotted line to point \((6)\).

4. **Complete the path to +1:**
   - From point \((6)\), complete the path to point \(+1\).

The notation \( V_0 = 0 \) might refer to some value associated with the starting point \(-1\) or the endpoint \(+1\). If \( V_0 \) is defined as the value at the starting point \(-1\), and if this value is zero, then \( V_0 = 0 \) is trivially true based on the initial condition.

### Proving that \( M \) is Embedded

The statement "To prove that \( M \) is embedded, we follow the path from -1 to +1 given by the solid lines" indicates that the path from \(-1\) to \(+1\) using only the solid lines should be continuous and without any self-intersections or gaps.

1. **Path from -1 to +1 using solid lines:**
   - Start at point \(-1\).
   - Follow the path using only the solid lines to reach point \(+1\).

The embedding property of a path means that the path must be continuous and form a closed loop or a simple curve without any intersections or breaks. If the path from \(-1\) to \(+1\) using only the solid lines forms such a continuous and non-self-intersecting curve, then \( M \) is embedded.

### Local Pictures

The reference to "Figure fig: Local pictures about non-embedded point in the Path" implies that there may be specific local configurations or points where the path fails to be embedded. These local pictures would show the non-embedded points \( z_0 \) at various stages along the path.

For example:
- At stage 1, the path might start correctly but deviate.
- At stage 2, the path might intersect itself or have a gap.
- At stage 3, the path might fail to close properly.

These local pictures would help identify the exact points where the path fails to be embedded and provide a visual confirmation of the non-embedding property.

### Conclusion

To summarize:
- \( V_0 = 0 \) is true if the value at the starting point \(-1\) is zero.
- \( M \) is embedded if the path from \(-1\) to \(+1\) using only the solid lines is continuous and non-self-intersecting.
- The local pictures will confirm the non-embedding property at specific points along the path.

If you need further clarification or have additional details about the specific points or the nature of the path, please provide more context.