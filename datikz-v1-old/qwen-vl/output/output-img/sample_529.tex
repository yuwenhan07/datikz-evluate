\documentclass{article}
\usepackage{amsmath}
\usepackage{tikz}

\begin{document}

\begin{equation}
\Psi[Q_R, Q_L] = 0
\end{equation}

\begin{equation}
\partial_t Q + S_1 \, \partial_x Q = \frac{1}{\varepsilon} \begin{pmatrix} 0 \\ F_1(U) - V \end{pmatrix}
\end{equation}

\begin{equation}
\partial_t Q + S_2 \, \partial_x Q = \frac{1}{\varepsilon} \begin{pmatrix} 0 \\ F_2(U) - V \end{pmatrix}
\end{equation}

\begin{tikzpicture}[scale=1]
    % Draw the x-axis
    \draw[->] (-6,0) -- (6,0) node[right] {$x$};
    
    % Mark points on the x-axis
    \foreach \x/\label in {-5/2/-5/2, -3/2/-3/2, -1/-1/-1, 0/0/0, 1/1/1, 3/2/3/2}
        \draw (\x,0pt) -- ++(0,-3pt) node[below] {$x_{\label}$};
    
    % Draw vertical lines
    \draw[dashed] (0,0) -- (0,2);
    
    % Label regions
    \node at (-4,1) {$Q_{-2}$};
    \node at (-2,1) {$Q_{-1}$};
    \node at (0,1) {$Q_L$};
    \node at (2,1) {$Q_1$};
    
    % Label intervals
    \node at (-4,-1) {$I_{-2}$};
    \node at (-2,-1) {$I_{-1}$};
    \node at (0,-1) {$I_0$};
    \node at (2,-1) {$I_1$};
    
    % Label points of interest
    \node at (-4.5,0) {$x_{-5/2}$};
    \node at (-2.5,0) {$x_{-3/2}$};
    \node at (-0.5,0) {$x_{-1/2}=0$};
    \node at (0.5,0) {$x_{1/2}$};
    \node at (2.5,0) {$x_{3/2}$};
\end{tikzpicture}

\end{document}