\documentclass{article}
\usepackage[showframe]{geometry}
\usepackage{lipsum}
\usepackage{tikz}
\usetikzlibrary{calc, positioning}

\begin{document}

\lipsum[1]
\bigskip

\tikzset{%
  T/.style={blue!50!black, fill=blue!50!black, text=white,
    minimum width=0.48\textwidth, text width=0.48\textwidth-1ex
  },
  BW/.style={green!50!black, fill=green!70!black, text=black,
    minimum width=0.17\textwidth, text width=0.17\textwidth-1ex
  },
  BE/.style={red!75!black, fill=red!95!black, text=black,
    minimum width=0.30\textwidth, text width=0.30\textwidth-1ex,
  }
}
\noindent\begin{tikzpicture}[%
  every node/.style={draw, inner sep=1ex, outer sep=0pt, align=center,
  minimum height=7ex}
  ]
  \node[T, anchor=north west] at (0, 0)
  (nTW) {\bfseries \lipsum[2]};
  \node[BW, below=0pt of nTW.south west, anchor=north west]
  (n2) {scope 1, node 2};
  \node[BE, below left=0pt and 0pt of n2.north east, anchor=north west]
  (n3) {scope 1, node 3 that includes longer text to be broken into lines};

  \path ($(nTW.south) -(nTW.north)$);
  \pgfgetlastxy{\newW}{\newH}

  \node[T, right=.02\textwidth of nTW.north east,
  anchor=north west, minimum height={-\newH}]
  (nTE) {\bfseries scope 2, node 1};
  \node[BW, below=0pt of nTE.south west, anchor=north west]
  (n2E) {scope 2, node 2};
  \node[BE, below left=0pt and 0pt of n2E.north east, anchor=north west]
  (n3E) {scope 2, node 3 that includes longer text to be broken
    into lines; can be longer than the West corresponding node};

  \fill[yellow!90!red] (nTW.north east) rectangle (nTE.south west);
\end{tikzpicture}
\end{document}