<think>Got it, let's try to figure out this problem. First, let's make sure I understand the terms. The question is about an in-split directed weighted graph E₂ of E₁. Wait, maybe "in-split" is a specific type of graph transformation? But the problem says that in E₁, there was one edge with weight 1, and in E₂, this edge has been duplicated, so both edges have weight 1. 

First, let's recall what a directed graph (digraph) is. An edge in a digraph is an ordered pair (u, v), meaning from u to v. If an edge is duplicated, that would mean adding another edge from the same u to v, but maybe with the same weight. So if E₁ has an edge (u, v) with weight 1, then E₂ would have two edges from u to v, each with weight 1. 

But wait, the problem says "in-split directed weighted graph, E₂, of E₁". Maybe "in-split" refers to a specific operation, like splitting an edge into multiple edges with the same weight? But the problem states that the edge was duplicated, so maybe the operation is adding a parallel edge (a duplicate edge) with the same weight. 

So, if E₁ has one edge (u, v) with weight 1, then E₂ would have two edges (u, v) each with weight 1. The question might be asking for the properties of E₂, or maybe the total weight of that edge in E₂, or something else. But the problem statement is a bit brief. Let me read again: "The in-split directed weighted graph, E₂, of E₁. In E₁, there was one edge with weight 1. In E₂, this edge has been duplicated, and both edges have the weight 1." 

So the key points are: E₁ has an edge with weight 1; E₂ is the in-split version, which duplicates that edge, resulting in two edges with weight 1 each. Maybe the question is asking for the total weight of that edge in E₂, which would be 1 + 1 = 2? Or maybe the number of edges? But the problem might be expecting a description of E₂ or a specific property. Wait, the original question might have been cut off, but given the information, the main point is that E₂ has two edges where E₁ had one, each with weight 1. So the edge (u, v) in E₁ is replaced by two edges (u, v) in E₂, each with weight 1. Therefore, the edge's weight in E₂ for that pair is 1, but there are two such edges. Alternatively, if we consider the total weight for the edge (u, v) in E₂, it would be 1 + 1 = 2. But maybe the question is asking for the number of edges or the total weight. Wait, the problem says "this edge has been duplicated, and both edges have the weight 1". So each duplicated edge has weight 1, so two edges with weight 1 each. Therefore, the edge (u, v) in E₂ has two edges, each with weight 1. If the question is asking for the total weight of that edge in E₂, it's 1 + 1 = 2. Alternatively, if it's asking for the number of edges, then the number of edges in E₂ for that particular edge is 2. But without more context, it's hard to say, but given the problem statement, the key is that the edge was duplicated, so E₂ has two edges with weight 1 where E₁ had one. Therefore, the answer might be that E₂ has two edges with weight 1 for the original edge, or the total weight is 2. But let me think again. The problem says "in E₂, this edge has been duplicated, and both edges have the weight 1". So each of the two edges has weight 1, so the edge (u, v) in E₂ has two edges, each with weight 1. Therefore, if we're talking about the edge's weight in E₂, it's still 1 per edge, but there are two edges. Alternatively, if the question is asking for the total weight of all edges in E₂ related to that original edge, it would be 1 + 1 = 2. But maybe the question is simpler: since the edge was duplicated, E₂ has two edges with weight 1 for the original edge. So the answer could be that E₂ has two edges with weight 1 for the original edge, or the total weight is 2. Given the problem statement, the most direct answer is that E₂ has two edges with weight 1 where E₁ had one, so the edge's weight in E₂ is duplicated, each with 1, so total for that edge is 2.</think><answer>To solve the problem, we analyze the transformation from \( E_1 \) to \( E_2 \). In \( E_1 \), there is one edge with weight 1. When this edge is duplicated in \( E_2 \), two identical edges (from the same vertices) are created, each with weight 1. This means \( E_2 \) contains two edges where \( E_1 \) had one, and both new edges retain the weight 1.  

The key observation is that the duplication results in two edges with weight 1 for the original edge, so the total weight associated with that edge in \( E_2 \) is \( 1 + 1 = 2 \).  

The answer is \(<|begin_of_box|>2<|end_of_box|>\).</answer>