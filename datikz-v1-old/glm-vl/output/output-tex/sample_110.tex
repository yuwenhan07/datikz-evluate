<think>Got it, let's tackle this problem step by step. First, I need to recall what a 1-neighborhood of a vertex means. The 1-neighborhood should be the set of vertices adjacent to the given vertex, right? So for each grid type (square, king, triangular), I need to determine which vertices are adjacent to the black vertex.

Starting with part (a), the square grid. In a square grid, each vertex typically has 4 neighbors: up, down, left, right. But wait, need to confirm the grid structure. If it's a standard square grid like a chessboard, each vertex (except those on the edge, but assuming the black vertex is not on the edge, or maybe the grid is finite? Wait, the problem says "the square grid"—probably a grid where each vertex is connected to its four orthogonal neighbors (up, down, left, right). So the 1-neighborhood would include those four vertices. But wait, the image might show a specific grid, but since I don't have the image, I need to assume standard definitions.

For part (b), the king grid. A king in chess moves one square in any direction: orthogonally or diagonally. So in the king grid, each vertex is connected to its 8 neighbors (up, down, left, right, and the four diagonals). Therefore, the 1-neighborhood would include all 8 surrounding vertices.

For part (c), the triangular grid. In a triangular grid, each vertex is typically connected to 6 neighbors, arranged in a hexagonal pattern around it. Wait, need to recall the structure. A triangular grid can be thought of as each vertex having neighbors in three directions, forming 6 neighbors (up, up-right, right, down-right, down, down-left, left, up-left? Wait, maybe better to think of a hexagonal grid, but triangular grid usually has each vertex connected to 6 others, forming equilateral triangles. So the 1-neighborhood would include 6 vertices.

Wait, but the problem mentions "the black vertex"—assuming it's a specific position, maybe in the center of its shape? Wait, the shapes are numbered 1 to 4, arranged in a square grid. But the question is about the 1-neighborhood of the black vertex in each grid type. Wait, maybe each grid (square, king, triangular) has the black vertex in a certain position, and we need to find its neighbors.

Alternatively, let's define each grid:

(a) Square grid: Each vertex has 4 orthogonal neighbors (up, down, left, right). So if the black vertex is at position (x,y), neighbors are (x+1,y), (x-1,y), (x,y+1), (x,y-1).

(b) King grid: Each vertex has 8 neighbors—orthogonal and diagonal. So neighbors are in all 8 directions: (x+1,y), (x-1,y), (x,y+1), (x,y-1), (x+1,y+1), (x+1,y-1), (x-1,y+1), (x-1,y-1).

(c) Triangular grid: This can be oriented with vertices at the corners of equilateral triangles. Each vertex has 6 neighbors, forming a hexagonal arrangement. The neighbors are in three directions, each 60 degrees apart. So if we consider a grid where each vertex is connected to the ones above, above-right, right, below-right, below, below-left, left, above-left? Wait, maybe better to think of a hexagonal grid, but triangular grid usually has each vertex connected to 6 others. Alternatively, in a triangular lattice, each vertex has 6 neighbors, arranged in a hexagon around it.

But since the problem is about the 1-neighborhood, which is the set of vertices at distance 1 from the black vertex. So for each grid:

(a) Square grid: 4 neighbors (orthogonal).

(b) King grid: 8 neighbors (all 8 directions).

(c) Triangular grid: 6 neighbors (hexagonal arrangement).

Wait, but need to make sure about the triangular grid's adjacency. In a triangular grid, if you have vertices arranged in equilateral triangles, each vertex is connected to 6 others, right? Let's visualize: each vertex has neighbors in three directions, each direction having two neighbors (one in each direction), so 3*2=6.

So summarizing:

(a) Square grid: 4 vertices adjacent (up, down, left, right).

(b) King grid: 8 vertices adjacent (all 8 directions: orthogonal and diagonal).

(c) Triangular grid: 6 vertices adjacent (hexagonal arrangement around the vertex).

I think that's the general approach. Since the problem mentions "the image displays a set of four black and white geometric shapes...", but without the image, we have to rely on standard grid definitions. Assuming the black vertex is in the center of its shape, which is likely in the middle of the grid, so the number of neighbors would be as per the grid type.</think><answer>To determine the 1-neighborhoods of the black vertex in each grid type, we analyze the adjacency rules for each grid:  

### (a) Square Grid  
In a square grid, vertices are adjacent if they share an edge (orthogonally, i.e., up, down, left, or right). Thus, the black vertex has **4 neighbors** (one in each orthogonal direction).  

### (b) King Grid  
In a king grid, vertices are adjacent if they are a king’s move apart (orthogonally or diagonally). This includes all 8 possible directions around the vertex. Thus, the black vertex has **8 neighbors** (all 8 orthogonal and diagonal directions).  

### (c) Triangular Grid  
In a triangular grid, vertices are adjacent if they share a side of an equilateral triangle. Each vertex has 6 neighbors arranged in a hexagonal pattern around it. Thus, the black vertex has **6 neighbors** (in three directions, each with two neighbors).  

The answer is  
<|begin_of_box|>(a) 4, (b) 8, (c) 6<|end_of_box|></answer>