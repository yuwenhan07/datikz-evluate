<think>Got it, let's try to figure out how to illustrate equivalence classes with swaps and edges. First, I need to recall what an equivalence class is—sets of elements related by an equivalence relation, here swaps. So each word is in a class, and edges connect words that differ by a swap. Directed edges show the direction of the swap in the reduction algorithm.

First, let's pick a simple example, maybe permutations or words with letters, but maybe start with a small set, like three elements, or maybe binary words. Wait, let's think of a specific case. Suppose we have words of length 2, say "ab" and "ba". Swapping the two letters would turn "ab" into "ba" and vice versa. So the equivalence class has two words, connected by a directed edge from "ab" to "ba" (if the reduction goes that way) or maybe both ways, but directed edges show the direction of the swap used in the algorithm.

Wait, the problem says "directed or undirected" edges—so if the swap can go both ways, maybe undirected, but if the reduction algorithm uses a specific direction, then directed. Let's take a concrete example. Let's say we have the equivalence class of words under swapping adjacent letters. For example, consider the word "abc". Swapping the first two letters gives "bac", swapping the last two gives "acb". So the equivalence class has "abc", "bac", "acb". The edges would be from "abc" to "bac" (swap first two), "abc" to "acb" (swap last two), "bac" to "acb" (swap first two of "bac" which is "ba" swapped gives "ab", so "bac" becomes "abc"? Wait, no, swapping first two of "bac" is "ab" and "c", so "bac" becomes "abc". Wait, maybe I need to think of a specific reduction algorithm, like bubble sort, where swaps are adjacent elements.

Let's take a small example with words of length 3. Let's say the words are permutations of {1,2,3}. The equivalence class under swapping adjacent elements. For example, the word "123" can be swapped first two to get "213", then swapped first two of "213" to get "132", then swapped last two to get "213" again, etc. So the equivalence class has "123", "213", "132", "231", "312", "321"—all permutations, connected by adjacent swaps. The directed edges would show the direction of the swap used in the reduction, say from "123" to "213" (swap 1 and 2), then "213" to "132" (swap 2 and 3), "132" to "312" (swap 1 and 3), etc., forming a directed graph where each node has edges to the words obtained by swapping adjacent elements.

Alternatively, if we have a specific word, say "ab", its equivalence class has "ab" and "ba", connected by a directed edge from "ab" to "ba" (if the algorithm swaps them) or vice versa. If the algorithm reduces by swapping, then the directed edge would show the direction of the swap used in the reduction process.

To illustrate, let's take a simple case with two words: equivalence class {w, w'} where w and w' are related by a swap. The edge between them is directed if the reduction goes from w to w' (or w' to w), indicating the swap direction. If the swap can go both ways, maybe an undirected edge, but if the algorithm uses a specific direction, directed.

Another example: consider the equivalence class of the word "aaab" under swapping adjacent letters. The words in the class would be all permutations of the letters with three a's and one b, considering adjacent swaps. The equivalence class has words like "aaab", "aaba", "abaa", "baaa", connected by edges for each adjacent swap. Directed edges would show the direction of the swap used in the reduction, say from "aaab" to "aaba" (swap first two a's), "aaba" to "abaa" (swap second and third a's), "abaa" to "baaa" (swap third and fourth), etc.

In general, the illustration would show each equivalence class as a connected component (if the relation is connected), with edges between words that differ by a swap. Directed edges indicate the direction of the swap in the reduction algorithm, so if the algorithm reduces a word by swapping to another, the edge goes from the original to the swapped one, showing the swap direction.

So, to summarize, the steps would be:
1. Identify the equivalence classes (sets of words related by swaps).
2. For each class, draw the words as nodes.
3. Draw an edge between each pair of words in the class if they are related by a swap.
4. If the edges are directed, indicate the direction as per the reduction algorithm (e.g., from the word before swap to after swap, or vice versa, depending on the algorithm's direction).

For a concrete illustration, let's take the equivalence class with words "ab" and "ba". Draw two nodes labeled "ab" and "ba". Draw a directed edge from "ab" to "ba" (if the reduction algorithm swaps them to get "ba") or from "ba" to "ab" (if it swaps to get "ab"). If the swap can be done both ways, maybe an undirected edge, but if the algorithm uses a specific direction, directed.

Another example with three words: "abc", "bac", "acb". Draw three nodes. Draw directed edges from "abc" to "bac" (swap first two), "abc" to "acb" (swap last two), "bac" to "acb" (swap last two of "bac", which is "ac" swapped to "ca", making "bca"? Wait, no, "bac" swapping last two is "ba" and "c" swapped to "bc" and "a", making "bca"? Wait, maybe I messed up. Let's do it correctly: swapping adjacent elements in "abc" (positions 1-2) gives "bac", swapping 2-3 gives "acb". Swapping adjacent elements in "bac" (1-2) gives "abc", swapping 2-3 gives "bca". Swapping adjacent elements in "acb" (1-2) gives "cab", swapping 2-3 gives "cba". So the equivalence class has "abc", "bac", "acb", "cab", "cba", "bca"—all permutations, connected by adjacent swaps. The directed edges would show the direction of each swap used in the reduction, say from "abc" to "bac" (swap 1-2), "abc" to "acb" (swap 2-3), "bac" to "abc" (swap 1-2 reversed), etc.

In the illustration, each equivalence class is a connected directed graph (or undirected, if swaps are reversible) where each edge represents a swap, and directed edges show the direction used in the reduction algorithm.</think><answer>To illustrate how words in an equivalence class are related by swaps with edges, follow these steps:  

1. **Identify Equivalence Classes**: Group words that differ by a swap into sets (equivalence classes). For example, with words of length 2, the class contains "ab" and "ba" (swapped once).  

2. **Draw Nodes for Words**: Represent each word in a class as a node. For the class { "ab", "ba" }, draw two nodes labeled "ab" and "ba".  

3. **Add Edges for Swaps**: Draw an edge between two words if they differ by a swap. If the reduction algorithm uses a specific direction (e.g., reducing "ab" to "ba"), draw a **directed edge** from "ab" to "ba" (or vice versa, depending on the algorithm’s direction). If swaps are reversible, use an **undirected edge**.  

4. **Show Reduction Direction (Directed Edges)**: If the algorithm reduces a word by swapping, the directed edge indicates the direction of the swap (e.g., from the original word to the swapped word). For example, in the class of 3-letter permutations ("abc", "bac", "acb"), draw directed edges like "abc" → "bac" (swap first two letters) and "abc" → "acb" (swap last two letters), showing the reduction steps.  

**Example Illustration (2-letter words)**:  
- Equivalence class: { "ab", "ba" }  
- Nodes: "ab" and "ba"  
- Edge: A directed edge from "ab" to "ba" (or "ba" to "ab", depending on the algorithm’s swap direction).  

**Example Illustration (3-letter permutations)**:  
- Equivalence class: { "abc", "bac", "acb", "cab", "cba", "bca" }  
- Nodes: Each permutation as a node.  
- Edges: Directed edges between adjacent permutations (e.g., "abc" → "bac", "abc" → "acb", "bac" → "bca", etc.), showing swaps between adjacent letters.  

In summary, each equivalence class forms a connected graph where edges represent swaps, with directed edges indicating the direction of the swap used in the reduction algorithm.  

The answer is <|begin_of_box|>\text{Illustration shows equivalence classes as connected graphs with edges representing swaps; directed edges show reduction algorithm’s swap direction.}<|end_of_box|></answer>