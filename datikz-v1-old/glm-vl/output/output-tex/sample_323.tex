<think>Got it, let's try to figure out how to draw the larger angle outside two smaller angles based on the description. First, let's parse the details given. There's a white triangle on a black background with a black dot at the right angle. The triangle is on a coordinate axis, with its center at (-1, 1), and it's large, covering most of the area. 

First, let's recall that a right angle is 90 degrees. If we have two smaller angles outside the triangle, maybe they are adjacent to the right angle, forming a larger angle. Wait, the problem says "drawing a larger angle outside two smaller angles"—so maybe the two smaller angles are inside the triangle, and the larger angle is formed by extending the sides of the triangle? Wait, the triangle has a right angle at the dot, which is probably a vertex. Let's think about the coordinate system. The center of the triangle is at (-1, 1), but maybe the vertices are positioned around that center. Wait, the dot is at the right angle, so maybe the right angle vertex is at a specific point. If the triangle is large, covering most of the area, its vertices might be at positions like (-1 + a, 1 + b), (-1 - a, 1 + b), (-1, 1 - b), but this is vague. 

Alternatively, since it's a right triangle, the two smaller angles would be at the other two vertices, each acute angles. If we extend the sides of the triangle outside the triangle, the angle formed by those extended sides would be the larger angle. For example, if the two smaller angles inside the triangle are α and β, then the angle outside would be 180° - α or 180° - β, but since it's a right triangle, α + β = 90°, so the larger angle would be 180° - α = 90° + β, or 180° - β = 90° + α. Wait, but the problem says "a larger angle outside two smaller angles", so maybe the two smaller angles are adjacent to the larger angle, forming a straight line or a full angle? 

Wait, let's think about the coordinate axis. The triangle is on a coordinate axis, so maybe one side is along the x-axis and another along the y-axis, making the right angle at the origin? But the center is at (-1, 1), so maybe the triangle is centered at (-1, 1), with vertices at (-1 + h, 1 + k), (-1 - h, 1 + k), (-1, 1 - k), making it a right triangle with right angle at (-1, 1 - k)? Wait, the dot is at the right angle, so the right angle vertex is the dot. If the center of the triangle is at (-1, 1), maybe the right angle vertex is at the center? Wait, the problem says "the x-coordinate of the triangle's center is -1 and the y-coordinate is 1", so center is (-1, 1), and the dot is at the right angle, which might be one of the vertices. If it's a right triangle, then the right angle vertex is a vertex, and the center is the centroid? Wait, centroid is the average of the vertices' coordinates. If the right angle is at the dot, say point D, then the other two vertices are A and B, with angle at D being 90°. The center of the triangle is the average of A, B, D's coordinates. If D is at, say, (x, y), then center is ((A_x + B_x + D_x)/3, (A_y + B_y + D_y)/3) = (-1, 1). Suppose D is at the right angle, maybe at (d_x, d_y), and the other two vertices are along the axes from D. For example, if D is at (0, 0), then the triangle could have vertices at (0,0), (a, 0), (0, b), making the center at ((a + 0 + 0)/3, (0 + b + 0)/3) = (a/3, b/3) = (-1, 1), so a = -3, b = -3? But that doesn't make sense because coordinates can't be negative if the triangle is on a positive axis. Wait, maybe the center is at (-1, 1), so the average of the vertices is (-1, 1). If the right angle is at the dot, say the dot is at (x, y), then the other two vertices are along the lines perpendicular to each other, maybe along the x and y axes. Suppose the right angle vertex is at (h, k), then the other two vertices are (h + a, k) and (h, k + b), making the right angle at (h, k). The center would be ((h + a + h + h + k + k + b)/3, (k + k + b + h + h + k)/3)? Wait, maybe this is getting too complicated. 

Alternatively, since the problem mentions a white triangle on a black background with a black dot at the right angle, and the triangle is large, covering most of the area. The larger angle outside the two smaller angles—maybe the two smaller angles are inside the triangle, each at the acute vertices, and the larger angle is formed by extending the two sides of the triangle that form the right angle, outside the triangle. So if the right angle is at the dot, the two sides forming the right angle are, say, horizontal and vertical. Extending those sides beyond the triangle would form an angle outside the triangle, which would be a straight angle (180°) minus the right angle, but wait, if you extend both sides, the angle between the extensions would be 180° - 90° = 90°? No, wait, if you have a right angle, and you extend both sides beyond the vertex, the angle outside would be 180° - 90° = 90°, but that's the same as the right angle. Wait, maybe the two smaller angles are the acute angles inside the triangle, each less than 90°, and the larger angle outside is their supplement or something. Wait, let's think of the triangle as having angles α, β, 90°, with α + β = 90°. If we extend one side of the triangle, say the side opposite angle α, outside the triangle, the angle formed at the extension point would be 180° - α, which would be greater than 90°, and similarly for 180° - β. If we form an angle outside both smaller angles, maybe the angle between the two extended sides is 180° - (α + β) = 180° - 90° = 90°, but that's the same as the right angle. Hmm, I'm getting confused. 

Wait, the key points are: the triangle has a right angle at the dot, which is a vertex. The triangle is large, centered at (-1, 1) on the coordinate axis. To draw a larger angle outside the two smaller angles, we need to extend the sides of the triangle that form the right angle, outside the triangle, and measure the angle between those extended sides. Since the original right angle is 90°, if we extend both sides beyond the vertex, the angle between the extensions would still be 90°, but maybe the larger angle is formed by extending one side of each of the smaller angles? Wait, the two smaller angles are inside the triangle, each at the acute vertices. If we extend one side of each acute angle outside the triangle, the angle between those two extensions would be the larger angle. Let's say the triangle has vertices at A, B, D, with D being the right angle. At vertex A, there's an angle α, at vertex B, angle β, with α + β = 90°. If we extend side AD beyond D to a point E, and extend side BD beyond D to a point F, then angle EDF would be 180° - 90° = 90°, same as before. Alternatively, if we extend side AB beyond A to E and beyond B to F, then angle EDF might be different. Wait, I need to visualize the diagram. The problem says "drawing a larger angle outside two smaller angles", so the two smaller angles are inside the triangle, and the larger angle is outside, probably adjacent to them, forming a sort of larger angle formed by the external sides. Given that the triangle is right-angled, the two smaller angles are acute, each less than 90°, so the larger angle outside would be greater than 90°, maybe 180° minus the smaller angle, but if there are two smaller angles, maybe the larger angle is 180° minus the sum of the two smaller angles, which is 180° - 90° = 90°, which doesn't make sense. Wait, I think I need to consider that when you have two angles outside the triangle, their sum might form the larger angle. Alternatively, since the problem mentions the triangle is on a coordinate axis, maybe the right angle is at the origin, but the center is at (-1, 1), so the triangle is shifted. If the right angle is at the dot, which is a vertex, maybe the dot is at (0, 0), center at (-1, 1), so the centroid is at (-1, 1), which is the average of the vertices. If the vertices are at (x1, y1), (x2, y2), (x3, y3), then (x1+x2+x3)/3 = -1, (y1+y2+y3)/3 = 1. If the right angle is at (0, 0), then maybe the other vertices are (a, 0) and (0, b), so the centroid is ((a + 0 + 0)/3, (0 + b + 0)/3) = (a/3, b/3) = (-1, 1), so a = -3, b = 3. So the triangle has vertices at (-3, 0), (0, 3), and (0, 0), with the right angle at (0, 0). The center is at (-1, 1), which matches. Now, to draw a larger angle outside the two smaller angles, which are at (-3, 0) and (0, 3), the smaller angles inside the triangle are at those vertices. If we extend the sides from (-3, 0) and (0, 3) beyond the triangle, the angle formed outside would be the angle between the extensions of those sides. The side from (0, 0) to (-3, 0) is along the x-axis, extended beyond (-3, 0) to, say, (-4, 0), and the side from (0, 0) to (0, 3) extended beyond (0, 3) to (0, 4). The angle between the extended lines at, say, (-4, 0) and (0, 4) would be the angle between the negative x-axis and positive y-axis, which is 90°, same as the right angle. Wait, maybe the larger angle is formed by extending the other sides. If we extend the side from (-3, 0) to (0, 3), which has a slope of 3/(3) = 1, extended beyond (-3, 0) would go to (-6, -3), and beyond (0, 3) to (3, 6). The angle between these two extended lines might be 90°, but I'm not sure. 

Given the complexity, maybe the key is that the larger angle outside the two smaller angles (which are each 45° if it's an isosceles right triangle, but it doesn't have to be) would be 180° minus the smaller angle. But since the problem states the triangle is large with a right angle, the larger angle outside would likely be 90°, but I need to make sure. Wait, another approach: if you have two angles, say θ and φ, with θ + φ = 90°, then the angle outside would be 180° - θ or 180° - φ, which would be greater than 90°, and if you combine both, maybe the total is 180° - (θ + φ) = 90°, but I'm getting confused. Given the problem mentions the dot is at the right angle, and the triangle is on a coordinate axis with center at (-1, 1), perhaps the larger angle is formed by the intersection of the two axes extended beyond the triangle, creating a 90° angle, but outside the triangle. Given all this, I think the larger angle is 90 degrees, same as the right angle, but outside the triangle. Alternatively, if the two smaller angles are each 45°, the larger angle would be 90°, but I need to confirm. Wait, the problem says "drawing a larger angle outside two smaller angles", so the larger angle should be greater than each of the smaller angles. If the smaller angles are, say, 30° and 60°, the larger angle would be 180° - 30° = 150° or 180° - 60° = 120°, but without specific angles, it's hard. However, given it's a right triangle, the two smaller angles add up to 90°, so the angle outside would be 180° - (angle of the smaller angle), but if we take both smaller angles, maybe the larger angle is 180° - (α + β) = 180° - 90° = 90°, which is the same as the right angle. I think I'm overcomplicating. The key points are: the triangle has a right angle, the larger angle outside the two smaller angles is likely a straight angle or a right angle, but given the description, probably 90 degrees, drawn by extending the sides of the right angle outside the triangle, forming an angle equal to the right angle but outside. So the final answer would be a 90-degree angle, but I need to make sure. Wait, the problem says "the exact placement and size", so if the triangle has a right angle, the larger angle outside would be a 90-degree angle, same as the right angle, but located outside the triangle. Given the center is at (-1, 1), the angle would be drawn with its vertex at the same point as the right angle, but outside the triangle, forming a 90-degree angle. So the size is 90 degrees, placement outside the triangle.</think><answer>To draw a larger angle outside two smaller angles in the described diagram, start by identifying the right triangle with a right angle at the black dot (vertex). The triangle is centered at (-1, 1) on the coordinate axis. The two smaller angles inside the triangle are acute angles at the other two vertices, summing to 90°. To form the larger angle outside, extend the sides forming the right angle beyond the triangle. This creates an angle equal to the right angle (90°) but located outside the triangle, adjacent to the smaller angles.  

The larger angle is **90 degrees** and is positioned outside the triangle, formed by extending the sides of the right angle vertex outward.  

The answer is \(<|begin_of_box|>90^\circ<|end_of_box|>\).</answer>