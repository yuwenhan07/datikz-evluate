<think>Got it, let's try to tackle this problem step by step. First, let's recall some basic concepts. A 4-regular graph on six vertices means each vertex has degree 4. The complement graph L^c would then have each vertex with degree 6 - 1 - 4 = 1, since each vertex is connected to 6 - 1 (itself) minus the 4 neighbors in L, so 1 neighbor in L^c. Wait, but a graph where each vertex has degree 1 is a matching, right? But the problem says the complement is the disjoint union of three paths of length 1. Wait, a path of length 1 is just an edge, so three paths of length 1 would be three edges, no, wait, a path of length 1 has two vertices connected by an edge, so three such paths would be three edges, making a matching of three edges. But if each vertex has degree 1, then the complement is a 1-regular graph, which is a matching, and for six vertices, a 1-regular graph would have three edges (since each edge covers two vertices, 3 edges × 2 = 6 vertices), so that's a matching of three edges, which is three disjoint edges, each being a path of length 1. So the complement graph L^c is three disjoint edges, which are paths of length 1. 

Now, the original graph L is 4-regular on six vertices. Let's think about what L looks like. Each vertex has four edges. The complement has each vertex with one edge, so each vertex in L^c is connected to exactly one other vertex, forming three disjoint edges. So L^c is three disjoint edges, and L must be the graph where each vertex is connected to four others, which would mean each vertex is connected to all but one vertex (since there are five other vertices, degree 4 means connected to four, so not connected to one vertex). Wait, if L^c has three edges, say between {1,2}, {3,4}, {5,6}, then in L, each vertex is connected to four others. Let's check vertex 1: in L, it's connected to 3,4,5,6 (since it's not connected to 2). Wait, but let's make sure. If L^c has edges between 1-2, 3-4, 5-6, then each vertex's non-neighbor is its partner in L^c. So vertex 1 is not connected to 2 (since they are connected in L^c), so in L, vertex 1 is connected to 3,4,5,6. Similarly, vertex 2 is connected to 1,3,4,5? Wait, no, wait. Wait, if L^c has edges 1-2, 3-4, 5-6, then each vertex's neighbors in L^c are their partner, so each vertex has only one neighbor in L^c, which is their partner. Therefore, in L, each vertex must be connected to the other four vertices except their partner. Wait, let's list the vertices as 1,2,3,4,5,6. L^c has edges {1,2}, {3,4}, {5,6}. So in L, each vertex is connected to four vertices. For vertex 1: connected to 3,4,5,6 (since not connected to 2). Wait, but 1 is connected to 3,4,5,6? That's four vertices. Similarly, vertex 2 is connected to 3,4,5,6? Wait, but vertex 2's non-neighbor in L^c is 1, so in L, vertex 2 must be connected to 3,4,5,6? Wait, but then vertex 2 and vertex 1 are not connected in L, which matches L^c having {1,2}. But then vertex 3 is connected to 1,2,5,6? Wait, vertex 3's non-neighbor in L^c is 4, so in L, vertex 3 is connected to 1,2,5,6. Similarly, vertex 4 is connected to 1,2,5,6? Wait, but then vertex 3 and 4 are not connected in L, which is correct because they are connected in L^c. Wait, but let's check the degrees. Each vertex in L has degree 4. For vertex 1: connected to 3,4,5,6 – that's four edges. Vertex 2: connected to 3,4,5,6 – four edges. Vertex 3: connected to 1,2,5,6 – four edges. Vertex 4: connected to 1,2,5,6 – four edges. Vertex 5: connected to 1,3,4,6 – four edges. Vertex 6: connected to 1,3,4,5 – four edges. Wait, does this form a 4-regular graph? Let's check the edges. The edges in L would be between 1-3,1-4,1-5,1-6, 2-3,2-4,2-5,2-6, 3-5,3-6,4-5,4-6. Wait, let's count the edges. Each pair like 1-3,1-4,1-5,1-6 is four edges from vertex 1. Similarly, vertex 2 has four edges: 2-3,2-4,2-5,2-6. Vertex 3 has edges 3-1,3-2,3-5,3-6 (four edges). Vertex 4 has edges 4-1,4-2,4-5,4-6 (four edges). Vertex 5 has edges 5-1,5-2,5-3,5-4 (four edges). Vertex 6 has edges 6-1,6-2,6-3,6-4 (four edges). Now, how many edges are there in total? Each edge is counted twice in the sum of degrees. The sum of degrees is 6×4=24, so total edges are 12. Let's count the edges listed: from vertex 1: 1-3,1-4,1-5,1-6 (4), vertex 2: 2-3,2-4,2-5,2-6 (4), vertex 3: 3-5,3-6 (wait, wait, earlier I thought vertex 3 has 3-1,3-2,3-5,3-6, which is four edges: 3-1,3-2,3-5,3-6. Similarly, vertex 4 has 4-1,4-2,4-5,4-6. Vertex 5 has 5-1,5-2,5-3,5-4. Vertex 6 has 6-1,6-2,6-3,6-4. So listing all edges: 1-3,1-4,1-5,1-6, 2-3,2-4,2-5,2-6, 3-5,3-6,4-5,4-6. That's 12 edges, which matches 6×4/2. Now, the complement graph L^c should have edges that are not in L. The edges in L^c would be the ones not present in L. The edges in L are the ones we listed, so the complement should have the edges between 1-2, 3-4, 5-6, which are three edges, each of which is a path of length 1 (i.e., an edge), and they are disjoint, so the complement is three disjoint edges, which is three paths of length 1. That makes sense. So the translation is that the complement of the 4-regular graph L on six vertices is a graph consisting of three disjoint edges, which are paths of length 1, and thus the complement is the disjoint union of three such paths.</think><answer>To solve the problem, we start by noting that a 4-regular graph \( L \) on six vertices has each vertex with degree 4. The complement graph \( L^c \) has each vertex with degree \( 6 - 1 - 4 = 1 \), meaning \( L^c \) is a 1-regular graph, which is a matching. For six vertices, a 1-regular graph consists of three disjoint edges (since each edge connects two vertices, and \( 3 \times 2 = 6 \) vertices). Each edge in \( L^c \) is a path of length 1, and since the edges are disjoint, \( L^c \) is the disjoint union of three such paths.

To verify, consider \( L^c \) with edges \( \{1,2\}, \{3,4\}, \{5,6\} \). In the original graph \( L \), each vertex is connected to four others, excluding its complement partner. For example, vertex 1 is connected to 3, 4, 5, 6 (excluding 2), and vertex 2 is connected to 3, 4, 5, 6 (excluding 1), ensuring \( L \) is 4-regular. The complement \( L^c \) thus has three disjoint edges, confirming it is the disjoint union of three paths of length 1.

The answer is \(<|begin_of_box|>3<|end_of_box|>\).</answer>