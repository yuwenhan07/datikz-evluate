<think>Got it, let's try to figure out how to create a 1D illustration for symmetric padding modes. First, let's recall the problem: we have a 1D discrete signal with N=4, represented by black dots. Then, empty circles are reflections, and gray dots are extra added samples. The signal is periodic, so left padding wraps to the right.

First, let's define the original signal. Let's say the original signal is x = [x0, x1, x2, x3], N=4. Now, different symmetric padding modes would have different ways of padding. Common ones are reflection padding, symmetric padding (which might be reflection with wrap), etc. But let's think step by step.

First, the original signal is x = [x0, x1, x2, x3]. Let's index them as positions 0 to 3.

For example, if we're padding on both sides, let's say we need to pad with some mode. Let's consider the padding length, say L (left padding length, right padding length, but maybe symmetric, so same on both sides). But the problem says "symmetric padding modes" and the illustration shows left padding wrapped to the right (periodic), so maybe circular padding.

Let's take a specific example. Let's say the original signal is x = [a, b, c, d] (N=4). Let's say we need to pad with, for example, reflection padding. For reflection padding, the left padding would be the reflection of the original signal's end to the left, and right padding is reflection of start to the right. But since it's periodic, the left padding wraps around.

Wait, the problem says "the signal is periodical and the left padding is wrapped to the right side." So the signal is periodic, meaning x[N] = x[0], x[-1] = x[N-1], etc. So when padding on the left, the reflection might wrap around the end.

Let's try to construct the illustration step by step.

First, original signal (black dots): positions 0,1,2,3 with values, say, x0, x1, x2, x3. Let's assign specific values for clarity, maybe x0=1, x1=2, x2=3, x3=4 (just an example).

Now, let's consider a padding mode, say "reflection padding" (first mode). For left padding, we reflect the original signal's end to the left. Since the signal is periodic, reflecting x3 (the last element) to the left would be x3, x2, x1, x0, etc. Wait, reflection padding usually reflects the signal about the center of the original signal. Wait, maybe I need to clarify the padding modes.

Wait, the problem mentions "symmetric padding modes" which might include:

1. Zero padding (but not symmetric, maybe not)
2. Reflection padding (reflecting the signal about the edge)
3. Symmetric padding (reflecting the signal about the center)
4. Periodic padding (which is what the problem says, left padding wrapped to the right)

But let's think of the illustration. Let's say we need to pad the signal with, for example, 2 samples on each side (left and right). So total padded signal length is N + 2*P, where P is padding length.

For reflection padding (mode 1): left padding is the reflection of the original signal's right end, and right padding is the reflection of the original signal's left end. Since the signal is periodic, reflecting the right end (x3) to the left would be x3, x2, x1, x0, x3, x2,... but if padding length is 2, left padding would be x3, x2 (reflected from the right end), and right padding would be x0, x1 (reflected from the left end). Wait, maybe better to visualize.

Original signal (black dots): positions 0 (x0), 1 (x1), 2 (x2), 3 (x3). Let's say padding length is 2 (left and right). For reflection padding:

Left padding (positions -2, -1) would be the reflection of the original signal's end. Since the signal is periodic, x[-1] = x[N-1] = x3, x[-2] = x[N-2] = x2. So left padding is x3, x2 (as empty circles). Right padding (positions 4,5) would be the reflection of the original signal's start. x[4] = x[0] = x0, x[5] = x[1] = x1. So right padding is x0, x1 (as empty circles). Then, if there's extra added data (gray dots), maybe for some modes, you add zeros or something, but the problem says "extra added data samples to achieve some required properties", so maybe for some padding modes, you need to add extra samples, like extending the reflection further.

Wait, the problem says "Empty circles denotes the reflection of the original signal in the padding area. Gray dots denots extra added data samples... c.f. Section~sec:symmetric-padding." So maybe some padding modes have just reflections (empty circles), and others have reflections plus extra added samples (gray dots).

Let's try to make a specific example. Let's take the original signal as [x0, x1, x2, x3] (black dots). Let's say we're padding with a mode that has reflection and then extra added samples. For example, if the padding is symmetric with reflection and then adding a sample at the center (for even N, maybe). But let's think of a simple case where padding length is 1 on each side (left and right). So total padded signal length is 6.

Original positions: 0,1,2,3 (black dots). Left padding positions -1, -2? Wait, no, if padding length is 1 on each side, then left padding is positions -1, -2? Wait, maybe better to use 0-based indexing for the padded signal. Let's say the original signal is in positions 1,2,3,4 (if we consider the padded signal as starting from 0). Wait, maybe I need to clarify the indexing.

Alternatively, let's consider the original signal as a sequence: [x0, x1, x2, x3] (length N=4). When padding, say, with a symmetric mode that reflects the signal about the center. For a signal of length 4, the center is between positions 1.5 and 2.5, so reflecting about the center would reflect the first element to the right and the last to the left. So left padding would be x3, x2 (reflected from the right end), and right padding would be x1, x0 (reflected from the left end). Then, if there's extra added data, maybe adding a sample at the center (gray dot), but since N=4 is even, the center is between positions 2 and 3, so maybe adding a sample there.

But let's try to make a concrete illustration. Let's say the original signal is:

Black dots (x0 to x3): 0, 1, 2, 3 (positions 0 to 3)

For a padding mode, say "reflection padding" (left and right reflection):

Left padding (positions -1, -2) would be x3, x2 (since reflecting the right end: x3 is the last element, reflecting it to the left gives x3, x2, x1, x0... but if padding length is 2, then positions -2 and -1 are x3 and x2, as empty circles).

Right padding (positions 4, 5) would be x0, x1 (reflecting the left end: x0 is the first element, reflecting it to the right gives x0, x1, x2, x3... so positions 4 and 5 are x0 and x1, as empty circles).

If there's extra added data, say for a mode that requires extending the reflection further, maybe adding a sample at position -3 (x3 again, but since it's periodic, x-3 = x1? Wait, no, periodicity is x[N] = x[0], so x-1 = x3, x-2 = x2, x-3 = x1, x-4 = x0. So if we need to add extra samples, maybe adding x1 at position -3 (gray dot), x2 at position -4 (gray dot), etc., but this is getting complicated.

Alternatively, let's consider the problem says "1D Illustration of several symmetric padding modes". So maybe show two or three modes: reflection padding, symmetric padding, periodic padding, etc.

Let's try to outline the steps for each mode:

1. Original signal (black dots): positions 0,1,2,3 with values, say, [1, 2, 3, 4].

2. Reflection padding mode:
   - Left padding: reflect the original signal's end (right end) to the left. Since periodic, x3 is the last, so reflection is x3, x2, x1, x0... So if padding length is 2, left padding is x3, x2 (empty circles) at positions -1, -2.
   - Right padding: reflect the original signal's start (left end) to the right. x0 is first, reflection is x0, x1, x2, x3... So right padding is x0, x1 (empty circles) at positions 4,5.
   - If extra added data, maybe adding a sample at the center (position 2, which is original x2, but maybe not, need to check).

3. Symmetric padding mode (reflecting about the center):
   - The center of the original signal (length 4) is between positions 1.5 and 2.5. So reflect x0 to the right and x3 to the left. So left padding is x3, x2 (since reflecting x3 (last) to the left gives x3, x2, x1, x0... but for padding length 2, positions -1, -2 would be x3, x2; right padding would be x1, x0 (reflecting x0 (first) to the right gives x0, x1, x2, x3... so positions 4,5 are x1, x0). Then, if extra added data, maybe adding a sample at the center position (which is between 2 and 3, so position 2.5, but in discrete terms, maybe position 2 and 3, but need to adjust).

4. Periodic padding mode (which the problem mentions, left padding wrapped to the right):
   - Left padding is wrapped to the right, so reflecting the signal as if it's circular. So left padding would be the same as right padding's reflection. For example, padding length 2: left padding is x3, x2 (same as reflection padding), right padding is x0, x1 (same as reflection padding), but since it's periodic, the left padding wraps to the right, so maybe the padding is the same on both sides, creating a circular buffer.

Now, let's try to make a simple illustration with one mode, but the problem says "several", so maybe show two modes: reflection and symmetric.

Let's take the original signal as [x0, x1, x2, x3] = [1, 2, 3, 4] (black dots).

For reflection padding (first mode):
- Left padding (positions -1, -2): empty circles with x3=4, x2=3 → positions -1 (4), -2 (3)
- Right padding (positions 4,5): empty circles with x0=1, x1=2 → positions 4 (1), 5 (2)
- If extra added data, maybe none, or maybe adding a sample at the end of left padding (position -3) which is x1=2 (gray dot), but this is getting too vague.

Alternatively, let's use a simple case where padding length is 1 on each side (left and right), so padded signal length is 6.

Original positions (0-3): 1,2,3,4 (wait, no, original N=4, so positions 0,1,2,3: 1,2,3,4? Wait, let's use numerical values: x0=1, x1=2, x2=3, x3=4.

Reflection padding:
- Left padding (left of original signal): since periodic, the left side wraps to the right. So the first element of left padding is the last element of the original signal, then the second element is the second last, etc. So for padding length 1 (left), positions -1: 4 (x3), padding length 1 (right), positions 4: 1 (x0), 5: 2 (x1). Wait, no, padding length is the number of samples added. If padding length is 1 on each side, then left has 1 sample, right has 1 sample.

So padded signal positions: -1 (left padding), 0,1,2,3 (original), 4,5 (right padding).

Left padding position -1: reflection of original's last element (x3=4) → empty circle (4)
Right padding positions 4,5: reflection of original's first element (x0=1) and second (x1=2) → empty circles (1,2)

If there's extra added data, maybe adding a sample at position -2 (which is x2=3, gray dot), but this is not clear.

Alternatively, let's think of the problem's illustration as having the original signal in the center, with left padding as reflections (empty circles) and right padding as reflections, and maybe extra samples on top.

Given that the problem is about a 1D illustration, let's try to describe the positions:

- Original signal (black dots): positions 0,1,2,3 (let's say from left to right: 0,1,2,3)
- Left padding (reflection, empty circles): positions -1, -2 (to the left of 0), which are x3, x2 (since reflecting the right end)
- Right padding (reflection, empty circles): positions 4,5 (to the right of 3), which are x0, x1 (reflecting the left end)
- If extra added data (gray dots), maybe positions -3, -4 (left) with x1, x0 (but periodic, x-3 = x1, x-4 = x0), and positions 6,7 (right) with x3, x2, but this is getting too complex.

Given the problem's description, maybe the key is to show that for each symmetric padding mode, the left padding is a reflection of the original signal (empty circles) and right padding is a reflection, with periodicity wrapping left padding to the right. Also, gray dots are extra samples added for some modes.

In summary, the illustration would have:

- Black dots (original signal) in the center (positions 0-3)
- Empty circles (reflections) to the left and right of the original signal, which are the reflections of the original signal's ends (left reflection is right end, right reflection is left end)
- Gray dots (extra samples) added in some padding modes, which are additional samples added to achieve required properties, maybe extending the reflection further or adding zeros.

Since I need to provide a 1D illustration description, let's outline it:

For a 1D signal with N=4 (original signal length), the illustration would show:

- Original signal (black dots) at positions 0, 1, 2, 3 (let's say values a, b, c, d for clarity, but using numbers for example: 1, 2, 3, 4)
- Left padding area (to the left of position 0) with empty circles (reflection of the original signal's right end). For example, if padding length is 2, positions -1 and -2 would have empty circles with d, c (since reflecting the right end: d is the last element, c is the second last)
- Right padding area (to the right of position 3) with empty circles (reflection of the original signal's left end). For example, positions 4 and 5 would have empty circles with a, b (reflecting the left end: a is the first element, b is the second)
- If extra added data (gray dots) are present, they would be additional samples in the padding area, maybe extending the reflection further (e.g., position -3 with c, position -4 with b on the left; position 6 with d, position 7 with c on the right), or adding zeros, but the problem says "extra added data samples to achieve some required properties", so maybe specific to the padding mode.

Given the problem's mention of "periodical and the left padding is wrapped to the right side", the left padding's reflection is the same as the right padding's reflection's wrap, creating a circular buffer.

In conclusion, the 1D illustration would show the original signal in the center with reflections (empty circles) on both sides (left and right), and possibly extra samples (gray dots) in the padding area for certain modes, with the understanding that the signal is periodic, so left padding wraps to the right.</think><answer>To illustrate 1D symmetric padding modes for a discrete signal of length \( N = 4 \) (with periodicity, where left padding wraps to the right), follow this step-by-step breakdown:  


### 1. Define the Original Signal  
Let the original signal be a sequence of \( N = 4 \) samples, represented by **black dots** (e.g., \( x = [x_0, x_1, x_2, x_3] \), where \( x_0 = 1, x_1 = 2, x_2 = 3, x_3 = 4 \) for clarity). These samples are positioned at indices \( 0, 1, 2, 3 \) (center of the illustration).  


### 2. Understand Symmetric Padding Modes  
Symmetric padding modes reflect the original signal’s edges (left or right) to pad the signal. For periodic signals, left padding “wraps” to the right, creating a circular buffer.  

- **Empty circles**: Represent reflections of the original signal in the padding area.  
- **Gray dots**: Represent extra added samples (for modes requiring additional properties, e.g., extended reflection or zero padding).  


### 3. Illustrate a Specific Symmetric Padding Mode (e.g., Reflection Padding)  
Assume padding