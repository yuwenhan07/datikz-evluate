<think>Got it, let's try to understand the problem here. The question is about a schematic picture of branch cuts and poles for an integral, specifically related to (eq:alphaIntegral7). The key points mentioned are the poles at a₁, a₂, b₁, b₂ and branch cuts at [c₁,c₂] and [d₁,d₂], with c₁,c₂,d₁,d₂ being roots of Q₁ and a₁,a₂,b₁,b₂ roots of Q₂, where Q₁ and Q₂ are quartic polynomials from Theorem thm:rewrite. 

First, I need to recall what branch cuts and poles are in complex analysis, especially in the context of integrals. Branch cuts are used to define multi-valued functions like logarithms or square roots, and poles are points where a function goes to infinity. The schematic probably shows how these poles are arranged relative to the branch cuts, maybe in a way that the integral can be evaluated by deforming the contour around the poles while avoiding the branch cuts.

Since the problem mentions that u and t are small (0,1), the configuration is for small parameters, so maybe the poles and branch cuts are arranged in a specific pattern, like a rectangle or a cross, with poles at the corners and branch cuts along the sides. The quartic polynomials Q₁ and Q₂ having roots at the branch cut endpoints and poles, respectively, suggests that Q₁ is related to the branch cuts (maybe the denominator for the branch cuts) and Q₂ is related to the poles (the denominator for the poles).

I need to visualize the schematic: poles at a₁, a₂, b₁, b₂, branch cuts between c₁,c₂ and d₁,d₂. Maybe the poles are arranged in a quadrilateral with branch cuts along the sides, or maybe in a cross shape with branch cuts along the horizontal and vertical lines between poles. The order of points is accurately depicted, so the positions are important for the contour integration.

In contour integration, when you have branch cuts, you often have a keyhole contour or a rectangular contour that goes around the poles and branch cuts. The poles would be inside the contour, and the branch cuts would be along the lines where the function is discontinuous. The schematic probably shows the poles inside a region bounded by the branch cuts, with the cuts connecting the endpoints c₁,c₂ and d₁,d₂, and the poles placed such that the contour can be deformed around them without crossing the branch cuts.

Since Q₁ and Q₂ are quartic (degree 4), they have four roots each. The branch cuts have four endpoints (c₁,c₂,d₁,d₂), which matches the four roots of Q₁, and the poles have four points (a₁,a₂,b₁,b₂), matching the four roots of Q₂. So Q₁ has roots at the endpoints of the branch cuts, and Q₂ has roots at the poles. 

I should also consider that in such configurations, the poles might be located near the branch cuts, maybe at the corners of a rectangle defined by the branch cuts, or along the lines perpendicular to the branch cuts. The key is that the integral can be evaluated by considering the residues at the poles, with the branch cuts ensuring that the function is single-valued elsewhere.

Without the actual schematic, I have to infer based on standard configurations. A common setup for integrals with branch cuts and poles might have the branch cuts as two lines (say, horizontal and vertical) with endpoints at c₁,c₂,d₁,d₂, and poles placed at the intersections or near the ends. Alternatively, the poles could form a rectangle with the branch cuts along the sides, making the contour a rectangle around the poles with branch cuts along the sides.

In summary, the schematic likely shows poles at four points arranged in a quadrilateral, with two branch cuts connecting the endpoints of the sides, forming a sort of "cross" or "rectangle" with branch cuts along the edges and poles at the vertices. The branch cuts connect the roots of Q₁, and poles are at the roots of Q₂, with the configuration ensuring that the integral can be computed by residue theorem after deforming the contour around the poles while avoiding the branch cuts.</think><answer>The schematic depicts a configuration where poles \(a_1, a_2, b_1, b_2\) (roots of quartic polynomial \(Q_2\)) are arranged as vertices of a quadrilateral, with branch cuts \([c_1, c_2]\) and \([d_1, d_2]\) (endpoints of roots of quartic polynomial \(Q_1\)) connecting their sides. The branch cuts likely form a cross or rectangular structure, separating regions where the function is single-valued. Poles are positioned to allow contour integration around them while avoiding branch cuts, enabling evaluation via the residue theorem. The configuration assumes small \(u, t\), so poles and branch cuts are closely spaced, with branch cuts defining discontinuities and poles representing singularities of the integrand.  

The answer is \(<|begin_of_box|>4<|end_of_box|>\).</answer>