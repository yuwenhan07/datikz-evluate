<think>Got it, let's try to figure out what the problem is asking. First, let's parse the question. We have K_max = 50, so maybe K ranges from some starting point up to 50? Wait, the problem says "= K_max" – maybe that's a typo? Wait, the original says "Let K_max = 50, = K_max and h = 5." Hmm, maybe that's a formatting error. Let me read again: "Let K_max = 50, = K_max and h = 5." Oh, maybe it's supposed to be K = K_max, so K = 50? Wait, no, the next part says "We report the average relative loss (eq: nominal relative loss) with mean absolute deviations as a function of K for 100 random test instances." So, probably, K is varying, maybe from 1 to K_max, or some range, and for each K, we do 100 test instances, compute the average relative loss (mean) and mean absolute deviation (standard deviation maybe, or MAD), then plot as a function of K.

But let's break down the components. First, K_max is 50, so maybe K is an integer from, say, 1 to 50, or maybe starting from a lower value. h = 5, which might be a parameter in the model, like a hyperparameter. The average relative loss is probably the mean of the relative losses over the 100 test instances, and the mean absolute deviation is the average of the absolute deviations from the mean, which is a measure of spread.

So, the process would be: for each K (from, say, 1 to 50), do the following:
1. Generate 100 random test instances (maybe with some distribution, depending on the problem context, but since it's not specified, we can assume they're independent samples).
2. For each instance, compute the relative loss (maybe (predicted - actual)/actual, or similar, depending on the model).
3. Compute the average relative loss for these 100 instances (mean).
4. Compute the mean absolute deviation, which is the average of |relative_loss_i - mean_relative_loss| for each of the 100 instances.
5. Record the mean and MAD for each K, then plot them as a function of K.

But since the problem doesn't provide specific details about the model or the data generation process, we need to make some assumptions or describe the steps. However, the question is probably expecting a description of the process or maybe a summary of what the result would look like, but since it's a computational problem, maybe the answer is about the methodology.

Wait, the user might want a summary of the setup: K_max = 50, so K ranges from, say, 1 to 50, h = 5 (maybe a parameter like a window size, or a step size), average relative loss is the mean of relative losses over 100 instances, MAD is the average absolute deviation. So, the function of K would show how the average relative loss and MAD change as K increases from, say, 1 to 50.

But without specific data, we can't compute exact values, but maybe the question is about explaining the process. Alternatively, if this is a theoretical question, maybe the average relative loss decreases as K increases (if more data helps), and MAD might decrease or increase depending on stability. But since the problem says "report" the average relative loss with MAD as a function of K, the answer would involve describing the methodology and possibly the expected trend.

Wait, let's check the original question again: "Let K_max = 50, = K_max and h = 5. We report the average relative loss (eq: nominal relative loss) with mean absolute deviations as a function of K for 100 random test instances." Oh, maybe the "= K_max" is a typo, and it should be K = K_max, so K is 50? But then "as a function of K" implies K varies. Maybe the correct setup is K ranges from 1 to 50, h = 5 is another parameter, say, a step size for K. So, for each K in 1, 2, ..., 50, perform the experiment with 100 test instances, compute mean relative loss and MAD, then plot.

Since the problem doesn't provide specific data, the answer might need to outline the steps one would take to perform this, or describe the expected outcome. Given that it's a computational experiment, the average relative loss might decrease as K increases (if more data improves performance), and MAD might decrease if the model becomes more consistent. But without specific model details, it's hard to say exactly, but the process is clear: vary K, run 100 tests, compute mean and MAD, plot.

So, summarizing, the process involves:
1. Setting K_max = 50, h = 5.
2. For each K from 1 to K_max (or some range), generate 100 random test instances.
3. For each instance, calculate the relative loss (e.g., (predicted - actual)/actual).
4. Compute the average (mean) relative loss for the 100 instances.
5. Compute the mean absolute deviation (MAD) of the relative losses from their mean.
6. Record the mean relative loss and MAD for each K.
7. Plot these values as a function of K.

The final answer would likely present the plotted data, showing how the average relative loss and MAD change with K, but since we can't generate the actual plot here, we can describe the expected trend or state the methodology. However, given the question is asking for a report, maybe the key points are the setup and the process, and if a numerical answer is expected, but without data, it's impossible. Wait, maybe the question is a standard one, and the answer is about the average relative loss decreasing as K increases, with MAD decreasing, but I need to make sure.

Alternatively, if this is a question about a specific model, like a K-nearest neighbors or something else, but without more info, it's hard. Given the information provided, the best approach is to describe the methodology and the expected function behavior.</think><answer>To address the problem, we analyze the setup and methodology for reporting average relative loss and mean absolute deviation (MAD) as a function of \( K \):  

1. **Parameters Setup**:  
   - \( K_{\text{max}} = 50 \), so \( K \) ranges from 1 to 50 (or a subset, depending on context).  
   - \( h = 5 \) (likely a hyperparameter, e.g., step size or window size).  

2. **Experiment Process**:  
   - For each \( K \) (from 1 to \( K_{\text{max}} \)):  
     - Generate 100 random test instances (independent samples).  
     - For each instance, compute the **relative loss** (e.g., \( \frac{\text{predicted} - \text{actual}}{\text{actual}} \)).  
     - Calculate the **average relative loss** (mean of the 100 relative losses).  
     - Compute the **mean absolute deviation (MAD)**: average of \( | \text{relative loss}_i - \text{mean relative loss} | \) for all 100 instances.  

3. **Function Reporting**:  
   Plot the average relative loss and MAD as functions of \( K \). Typically, as \( K \) increases (with more data or parameters), the average relative loss may decrease (if the model improves) and MAD may decrease (if performance becomes more consistent), though trends depend on the specific model and data.  

Since the problem does not provide specific model/data details, the final result involves reporting these metrics for each \( K \), showing their variation. For a summary, the average relative loss is expected to decrease (or stabilize) as \( K \) increases, with MAD following a similar trend (decreasing or stabilizing).  

The answer is \(<|begin_of_box|>50<|end_of_box|>\) (assuming \( K \) ranges up to \( K_{\text{max}} = 50 \), though the exact plotted values depend on the experiment).</answer>