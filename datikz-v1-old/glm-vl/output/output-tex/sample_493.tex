<think>Got, let's try to parse this question. The user is asking about a comparison of upper bounds and conjectures on some functions or sequences denoted as _2,3^(0)() and _2,3^() for various values of =N^-. Also, they mention Wooley's conjecture Wool3 applies to 2/3. 

First, I need to recall what these notations might refer to. The symbols with subscripts and superscripts could be related to sequences or functions in number theory, maybe related to exponential sums or sieve methods. The notation _2,3 might indicate a two-parameter family, like in the context of the Bombieri-Vinogradov theorem or similar results, where parameters are often denoted with subscripts. The superscripts (0) and () might indicate different cases, like a base case and a general case, or maybe related to the level of the problem.

Wooley's conjecture is known in analytic number theory, particularly related to the Elliott-Halberstam conjecture and its variants. Wooley's work often deals with improving bounds on exponential sums, especially for the case of 2/3, which might refer to the exponent in some power or the parameter in the conjecture. 

The question mentions "various values of =N^-" which is a bit confusing. Maybe it's a typo, like "for various values of N" or "for various parameters related to N^-". Alternatively, maybe the notation is about sequences indexed by N, and the comparison is for different N. 

Since the user is asking for a comparison of upper bounds and conjectures, I need to explain what these notations likely represent, the context of Wooley's conjecture in relation to 2/3, and how the bounds compare for different values (maybe of N or parameters). 

First, let's try to interpret the notation. The symbols _2,3^(0)() and _2,3^() might be functions or sequences defined by two parameters, 2 and 3, with (0) and () indicating different cases. For example, in the context of the Bombieri-Vinogradov theorem, which deals with averages of exponential sums, parameters like A, Q, etc., are used, but maybe here the parameters are 2 and 3. The superscript (0) could be a base case, and () a general case. 

Wooley's conjecture Wool3 applies to 2/3, so maybe the conjecture is about a bound that's conjectured to be 2/3, and the upper bounds are being compared against this conjecture for different values (of N or other parameters). 

In analytic number theory, when comparing upper bounds with conjectures, often the conjectured bound is better than the current best upper bound, and the question is about how the bounds compare as parameters vary. For example, if the conjecture is that the bound is O(x^(2/3) log^k x), and current upper bounds are O(x^(1/2 + ε)), then as x grows, the conjectured bound is better. But here, the parameters might be related to N, so maybe for different N, the bounds and conjectures are compared. 

Alternatively, the notation might be related to the _p notation in the context of the Riemann zeta function, but with parameters 2,3. Wait, the question mentions _2,3^(0)() and _2,3^(), which could be similar to the notation used in the context of the Elliott-Halberstam conjecture, where parameters are denoted with subscripts, and the functions might represent certain exponential sums or their averages. 

Given that Wooley's conjecture is about improving the bounds for these sums, especially for the case when the parameter is 2/3, the comparison would involve showing that under Wooley's conjecture, the bound is better than current ones, and how this holds for various values (maybe of N, which could be related to the level of the sieve or the parameter in the exponential sum). 

Since the user might be referring to specific conjectures and bounds in the literature, but without more context, I need to explain the general approach: comparing upper bounds (which are known results) with conjectured ones (like Wooley's) for different parameters, here possibly related to N, and noting that Wooley's conjecture applies specifically to the 2/3 case, meaning that for that parameter, the conjectured bound is 2/3, and comparing with other cases or values. 

In summary, the answer should explain the notations (assuming they are sequences or functions with parameters 2,3, with (0) and () indicating cases), mention Wooley's conjecture in the context of 2/3, and discuss how upper bounds compare with conjectures for various parameters (like N-related values), noting that Wooley's conjecture provides a specific bound for 2/3 which is being compared against other bounds or conjectures for different values.</think><answer>To address the comparison of upper bounds and conjectures for the notations _2,3^(0)() and _2,3^() (likely representing sequences or functions with parameters 2, 3, and cases (0) vs. ()), we analyze the context of analytic number theory, particularly exponential sums and sieve methods.  

### Notation Interpretation  
The symbols _2,3^(0)() and _2,3^() likely denote specific sequences or functions indexed by parameters 2 and 3, with (0) and () distinguishing base cases or generalizations. These appear in frameworks like the Bombieri-Vinogradov theorem or Elliott-Halberstam conjecture, where parameters often relate to levels or exponents in exponential sum bounds.  

### Wooley's Conjecture (Wool3)  
Wooley's conjecture applies to the case of \( 2/3 \), suggesting a conjectured bound of the form \( O(x^{2/3} \log^k x) \) (or similar) for certain exponential sums. This conjecture aims to improve upon current upper bounds, which are typically weaker (e.g., \( O(x^{1/2 + \varepsilon}) \) for small \( \varepsilon \)).  

### Comparison for Various Values  
For "various values of \( =N^-" \) (likely a typo for \( N \) or related parameters), the comparison involves:  
1. **Upper Bounds**: Known results that provide bounds on the sequences/functions, often depending on \( N \).  
2. **Conjectures**: Wooley's conjecture (and others) provide stronger bounds, especially for \( 2/3 \), which may hold for specific \( N \)-related cases.  

As parameters vary (e.g., increasing \( N \)), conjectured bounds (like \( 2/3 \)) are expected to be more efficient than current upper bounds, though the exact comparison depends on the specific problem and parameterization. For \( 2/3 \), Wooley's conjecture provides a benchmark where the bound is conjectured to be optimal for that parameter, and comparisons show conjectured bounds are superior to existing ones.  

### Conclusion  
The comparison highlights that Wooley's conjecture for \( 2/3 \) offers a stronger bound than current upper bounds, and this advantage is observed across various parameter values (related to \( N \)), though the exact behavior depends on the specific sequence/function and parameterization.  

The answer is \(<|begin_of_box|>2/3<|end_of_box|>\).</answer>