<think>Got it, let's try to figure out the trigonometric relation Archimedes might have used. First, Archimedes was a great mathematician known for his work on geometry, including circles and triangles. The image mentions a right-angled triangle, so maybe it's related to the Pythagorean theorem or some trigonometric identity. Wait, the problem says "trigonometric relation," so maybe something like the sine or cosine of an angle in a right triangle. But Archimedes' specific work—wait, he might have used the relationship involving the area of a circle or the relationship between the circumference and area, but the image has a triangle. Alternatively, maybe the law of cosines? But wait, the question is about a relation exploited by Archimedes. Wait, another thought: Archimedes used the method of exhaustion, which is related to limits, but maybe in the context of a right triangle, the relation could be the Pythagorean theorem, which is a fundamental trigonometric relation (since a² + b² = c² relates to sin²θ + cos²θ = 1 when c is hypotenuse). Alternatively, maybe the relation involving the tangent or cotangent. Wait, but let's think step by step. The image has a right triangle, so let's consider a right-angled triangle with legs a, b, hypotenuse c. The trigonometric ratios are sinθ = a/c, cosθ = b/c, tanθ = a/b, etc. The Pythagorean theorem is a² + b² = c², which can be written as (a/c)² + (b/c)² = 1, leading to sin²θ + cos²θ = 1, which is a key trigonometric identity. Archimedes might have used this in his geometric proofs, especially when dealing with areas or perimeters related to triangles. Alternatively, if the image shows a specific triangle with some markings, maybe the relation is about the area or the relationship between the sides in a specific context. But since the question is about the trigonometric relation exploited by Archimedes, the most likely candidate is the Pythagorean theorem, which is a fundamental relation in right triangles and is a key part of trigonometry. Alternatively, maybe the law of sines or cosines, but those are more general. Wait, another angle: Archimedes' work on the spiral, but that's a different shape. Wait, the problem mentions a white triangle on a white background, maybe a specific triangle like an equilateral or isosceles, but the equations around it—maybe the equations show the Pythagorean theorem or the trigonometric identities. Given that the question is about a trigonometric relation, the identity sin²θ + cos²θ = 1 derived from Pythagoras is a strong candidate. Alternatively, if the triangle is used in a specific formula Archimedes used, like relating the circumference of a circle to the triangle's sides, but without the exact image, we have to go by common knowledge. Archimedes is known for the formula for the area of a circle, but the triangle might be part of a proof using similar triangles or the Pythagorean theorem. Given that the question is about a trigonometric relation, the most probable answer is the Pythagorean theorem, which is a core trigonometric identity, or the relation sin²θ + cos²θ = 1. Alternatively, if the image shows a specific triangle with angles and sides, maybe the tangent relation, but I think the key here is the Pythagorean theorem as the fundamental relation.</think><answer>The trigonometric relation exploited by Archimedes, particularly in the context of right-angled triangles, is the Pythagorean theorem, which forms the basis for key trigonometric identities. For a right triangle with legs \(a\), \(b\) and hypotenuse \(c\), the theorem states \(a^2 + b^2 = c^2\). This can be expressed in trigonometric terms as \(\sin^2\theta + \cos^2\theta = 1\) (where \(\theta\) is one of the acute angles), a fundamental identity in trigonometry derived from the Pythagorean theorem.  

The answer is \(<|begin_of_box|>a^2 + b^2 = c^2 (or \sin^2\theta + \cos^2\theta = 1)<|end_of_box|>\).</answer>