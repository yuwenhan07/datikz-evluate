<think>Got it, let's try to figure out the cycle decomposition here. First, the problem mentions a cycle c_1 in red and c_2 in blue, with the cycle c_2 being = v_0 w_0 v_1 w_6 v_5. Wait, maybe I need to parse the cycle structure correctly. Let's start by understanding what a cycle decomposition is. A cycle decomposition of a permutation is a way of writing the permutation as a product of disjoint cycles, where each cycle is a sequence of elements that are mapped to each other in a cyclic manner.

The given cycle is c_2 = v_0 w_0 v_1 w_6 v_5. Wait, maybe the elements are vertices, and the cycle is a sequence where each element is mapped to the next one, and the last maps back to the first. So let's list the elements in the cycle: starting at v_0, then w_0, then v_1, then w_6, then v_5, and then back to v_0? Wait, but we need to check the mapping. Let's assume that the cycle is a permutation of the vertices, so each element is sent to the next one in the cycle. So the cycle c_2 would be (v_0 w_0 v_1 w_6 v_5), meaning that v_0 is mapped to w_0, w_0 is mapped to v_1, v_1 is mapped to w_6, w_6 is mapped to v_5, and v_5 is mapped back to v_0? Wait, but maybe the cycle is written as a sequence where each consecutive pair is the mapping, and the last maps to the first. So the cycle is (v_0 w_0 v_1 w_6 v_5), which would mean the permutation maps v_0 to w_0, w_0 to v_1, v_1 to w_6, w_6 to v_5, and v_5 to v_0. That forms a 5-cycle, right? A cycle of length 5.

Then, the first cycle c_1 is in red, but the problem doesn't specify what c_1 is, maybe we need to assume another cycle, perhaps a 1-cycle or another cycle. Wait, but the question says "the first cycle is c_1 and is in red, the cycle c_2 is in blue, and = v_0 w_0 v_1 w_6 v_5." Maybe c_1 is a different cycle, perhaps a 2-cycle or something else, but since the example is given for c_2 as that cycle, maybe c_1 is a separate cycle, like (v_2 v_3) or something, but without more info, maybe we need to construct the cycle decomposition with c_1 and c_2 as disjoint cycles.

Wait, let's think step by step. First, identify the elements involved in c_2: v_0, w_0, v_1, w_6, v_5. These are five elements, so c_2 is a 5-cycle. Then, if c_1 is another cycle, maybe involving the remaining elements, but since the problem doesn't specify the rest, maybe c_1 is a 1-cycle (a fixed point), but that's not likely. Alternatively, if the permutation is a product of c_1 and c_2, and they are disjoint, then the cycles don't share any elements. So if c_2 has elements v_0, w_0, v_1, w_6, v_5, then c_1 must consist of the remaining elements, say v_2, v_3, w_1, w_2, w_3, w_4, etc., but without knowing the total number of elements, it's hard. But the key is to write the cycle decomposition as a product of disjoint cycles, with c_1 and c_2 being two of them.

So, summarizing, the cycle c_2 is a 5-cycle: (v_0 w_0 v_1 w_6 v_5), and c_1 is another cycle, maybe a 2-cycle or a 1-cycle, but since the problem says "the first cycle is c_1" and "the cycle c_2 is in blue", perhaps c_1 is a separate cycle, like (v_2 v_3) if there are other elements, but without more info, maybe the example is just showing c_2 as that 5-cycle, and c_1 as another cycle, maybe a 1-cycle, but I need to make sure the cycles are disjoint.

Wait, another approach: in cycle decomposition, each element is in exactly one cycle. So if c_2 includes v_0, w_0, v_1, w_6, v_5, then these elements are not in any other cycle. So c_1 must consist of the remaining elements, say if there are, for example, v_2, v_3, w_1, w_2, w_4, w_7, etc., then c_1 could be a cycle like (v_2 v_3) or another cycle. But since the problem doesn't specify, maybe the example is simply showing c_2 as the given 5-cycle, and c_1 as a 1-cycle (a fixed point), but that's not helpful. Alternatively, maybe c_1 is a 2-cycle involving two elements not in c_2, but without more info, it's hard. However, the key part is writing the cycle c_2 as given, which is (v_0 w_0 v_1 w_6 v_5), and c_1 as another cycle, perhaps a 1-cycle or another cycle, but since the problem says "the first cycle is c_1" and "the cycle c_2 is in blue", maybe c_1 is a different cycle, like (v_4), a fixed point, but I need to make sure the cycles are disjoint.

Wait, let's assume that the permutation is a product of c_1 and c_2, which are disjoint cycles. So if c_2 is (v_0 w_0 v_1 w_6 v_5), then c_1 must be a cycle involving the other elements not in c_2. Suppose there are other vertices like v_2, v_3, w_1, w_2, w_3, w_4, then c_1 could be, for example, (v_2 v_3), but without knowing the exact elements, it's hard. However, the main point is to present the cycle decomposition as a product of disjoint cycles, with c_2 being the given 5-cycle, and c_1 being another cycle, possibly a 1-cycle or another length cycle.

Given the problem statement, maybe the cycle decomposition is c_1 = (v_2) (or a fixed point) and c_2 = (v_0 w_0 v_1 w_6 v_5), but that doesn't make sense. Alternatively, if c_1 is a 2-cycle, say (v_0 v_1), but wait, c_2 already includes v_0, v_1, so they can't be in two cycles. Therefore, c_1 and c_2 must be disjoint, so they share no elements. Therefore, c_1 must consist of elements not in c_2. Suppose the elements are v_0, w_0, v_1, w_6, v_5, and the rest are, say, v_2, v_3, w_1, w_2, w_4, then c_1 could be a cycle like (v_2 v_3) or (w_1 w_2) or (w_4), etc. But since the problem doesn't specify, maybe the example is just showing c_2 as the 5-cycle, and c_1 as a separate cycle, perhaps a 1-cycle, making the decomposition c_1 = (v_4) and c_2 = (v_0 w_0 v_1 w_6 v_5), but this is speculative.

Alternatively, maybe the cycle c_1 is a 1-cycle (a fixed element), and c_2 is the 5-cycle as given. So the cycle decomposition would be c_1 = (v_4) (or any fixed element) and c_2 = (v_0 w_0 v_1 w_6 v_5). But without more information, it's hard to be precise, but the key part is recognizing that c_2 is a 5-cycle involving those five elements, and c_1 is another cycle, possibly a 1-cycle or another length cycle, disjoint from c_2.</think><answer>To determine the cycle decomposition, we first identify the cycle \( c_2 \) given as \( = v_0 w_0 v_1 w_6 v_5 \). This forms a 5-cycle since it includes five distinct elements connected in a cyclic permutation: \( v_0 \to w_0 \to v_1 \to w_6 \to v_5 \to v_0 \). For the cycle decomposition, we need disjoint cycles, so \( c_1 \) must consist of the remaining elements not in \( c_2 \). Assuming \( c_1 \) is a separate cycle (e.g., a 1-cycle or another length cycle involving other vertices not in \( c_2 \)), the cycle decomposition includes both \( c_1 \) and \( c_2 \) as disjoint cycles.  

The cycle \( c_2 \) is explicitly written as a 5-cycle, and \( c_1 \) is a disjoint cycle (e.g., a fixed point or another cycle with remaining elements). The final cycle decomposition is expressed as the product of these disjoint cycles.  

The answer is \(<|begin_of_box|>(c_1)(v_0 w_0 v_1 w_6 v_5)<|end_of_box|>\).</answer>