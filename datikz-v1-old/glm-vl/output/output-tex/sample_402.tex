<think>Got it, let's try to tackle this problem step by step. First, I need to understand the context: we're dealing with a model where players' statuses (FS, FO, CF, UCF) depend on the position of the median voter m, which is between 0 and 1. Also, we have two players, x₁ and x₂, with x₁ < x₂, probably representing their positions or preferences. 

First, let's recall what each acronym might stand for. FS could be "Favorite while election is secured," FO might be "Favorite while election is open," CF is "Contestable favorite," UCF is "Uncontestable favorite." The problem is about illustrating the status of each player (probably two players, since x₁ and x₂) as m varies from 0 to 1, with x₁ < x₂. 

Since the median voter m is the middle point, when m is near 0, the voter is closer to x₁, and when m is near 1, closer to x₂. The status of each player would change based on how their position compares to m. 

Let's consider each player's position relative to m. For player x₁ (with x₁ < x₂), when m is less than x₁, x₁ is probably UCF (uncontestable favorite) because the median voter is closer to x₁, so x₁'s position is above m, making them a strong favorite. As m increases from 0 to x₁, x₁ remains UCF until m reaches x₁? Wait, no, wait. Wait, if m is the median voter, then if x₁ is less than m, the median voter is to the right of x₁, so x₁'s position is to the left of m. Wait, maybe I need to think in terms of the median voter's position relative to the players. 

If m is the median, then for a player x, if x < m, then the median is to the right of x, so x is in the left half, and if x > m, then the median is to the left of x. The status might depend on whether the player is in the majority's position or not. 

For example, when m is less than x₁, then both x₁ and x₂ might be in positions where the election is secured for the player closer to m. Wait, but we have two players. Let's think of the two candidates' positions relative to m. If m is between 0 and x₁, then x₁ is closer to m than x₂ is (since x₁ < x₂), so x₁ would be the favorite, maybe UCF. As m increases from 0 to (x₁ + x₂)/2, the median voter is between x₁ and x₂. Wait, but the problem says "with respect to the position of the median voter m 0,1", so m ranges from 0 to 1, and x₁, x₂ are points in that interval with x₁ < x₂. 

Let's consider different intervals of m:

1. When m < x₁: Both x₁ and x₂ are to the right of m (since x₁ < x₂ < 1, assuming x₂ < 1). Wait, if m < x₁, then x₁ > m, x₂ > m as well. But which one is closer to m? Since x₁ < x₂, the distance from m to x₁ is x₁ - m, and to x₂ is x₂ - m. If m < x₁, then x₁ - m could be small if m is close to x₁, but if m is much less than x₁, then x₁ - m is larger than x₂ - m (since x₂ > x₁). Wait, no, if m is less than x₁, then x₂ is greater than x₁, so x₂ - m = (x₁ + (x₂ - x₁)) - m = (x₁ - m) + (x₂ - x₁). So if m < x₁, then x₂ - m > x₁ - m. Therefore, the player closer to m is x₁, since x₁ - m < x₂ - m. So when m < x₁, x₁ is closer to m, so x₁ is the favorite. Depending on how close m is to x₁, maybe when m is very close to 0, x₁ is UCF (uncontestable favorite), meaning no other player can contest them. As m approaches x₁, maybe x₁ becomes a CF (contestable favorite), meaning another player can contest, but the election is still secured? Wait, the problem mentions FS (secured favorite), FO (open), CF (contestable), UCF (uncontestable). 

Alternatively, let's think of the median voter's position as determining the winner. If the median voter is closer to x₁, then x₁ wins; if closer to x₂, then x₂ wins. The status of each player would change based on whether their position is such that they can win if the election is decided, or if the election is open (FO), meaning neither has a sure win. 

For x₁, when m < x₁, x₁ is the favorite (since m is closer to x₁ than x₂), and if the margin between m and x₁ is large enough, x₁ is UCF (uncontestable favorite), meaning x₂ cannot win. As m increases towards x₁, the margin decreases, so x₁ might become FS (secured favorite), meaning the election is secured for x₁, but maybe when m is close to x₁, it's FS, and when m is between x₁ and x₂, then x₂ becomes the favorite. Wait, when m is between x₁ and x₂, then m is closer to x₂ than to x₁, so x₂ is the favorite. For x₂, when m > x₂, but since m is up to 1, if x₂ < 1, then when m > x₂, x₂ is still closer to m than x₁? Wait, if m > x₂, then both x₁ and x₂ are to the left of m, but x₂ is closer to m than x₁ is. Wait, x₁ < x₂ < m, so the distance from m to x₂ is m - x₂, and to x₁ is m - x₁. Since x₂ > x₁, m - x₂ < m - x₁, so x₂ is closer to m than x₁. Therefore, when m > x₂, x₂ is the favorite. 

Putting this together:

- When m < x₁: x₁ is the favorite. If m is very close to 0, x₁ is UCF (since there's a large margin between m and x₁, and x₂ is further away, so x₁ can't be contested). As m increases towards x₁, the margin decreases, so x₁ becomes FS (secured favorite), meaning the election is secured for x₁, but maybe when m is near x₁, it's FS, and when m is between x₁ and x₂, then x₂ becomes the favorite. 

- When m is between x₁ and x₂: x₂ is the favorite. If m is close to x₁, then x₂ is not too far, but if m is close to x₂, then x₂ is UCF, and as m moves away from x₂ towards x₁, x₂ becomes FS, then maybe FO if the margin is small, but wait, need to check the exact definitions. 

Wait, the problem says "the election is secured (FS) or still open (FO)". So FS means the election is secured for the favorite, meaning the favorite has a sure win regardless of voter preferences (or whatever the model is), while FO means the election is open, so the favorite could lose if the other player contests. 

Also, CF is contestable favorite, meaning the favorite can be contested but still has a chance, while UCF is uncontestable favorite, meaning the favorite cannot be contested. 

So for x₁:

- When m < x₁ - (x₂ - x₁)/2: x₁ is UCF (uncontestable favorite), since the margin is large enough that x₂ can't contest.

- When x₁ - (x₂ - x₁)/2 ≤ m < x₁: x₁ is FS (secured favorite), since the election is secured but maybe the margin is smaller, so it's still a favorite but not uncontestable.

- When m ≥ x₁: x₂ becomes the favorite, so x₁'s status changes.

For x₂:

- When m > x₂ + (x₂ - x₁)/2: x₂ is UCF.

- When x₂ + (x₂ - x₁)/2 ≤ m < x₂: x₂ is FS.

- When m ≤ x₂: x₁ is the favorite.

But since x₁ < x₂, let's define the median voter m. The key is to draw a diagram where for each m between 0 and 1, each player's status is indicated. 

Alternatively, since the problem mentions "with respect to the position of the median voter m 0,1" and "without loss of generality we supposed x₁ < x₂", we can consider the interval [0,1] with x₁ and x₂ in it, x₁ < x₂. The median voter m is the middle point, but if we consider the median between x₁ and x₂, it would be (x₁ + x₂)/2, but the problem says m is between 0 and 1, so maybe m is the overall median, not the median between the two players. 

Let's try to outline the status for each player:

Player x₁:

- If m < x₁: x₁ is the favorite. If m is very close to 0, x₁ is UCF (since x₁ is not too far from 0, but wait, x₁ could be anywhere, but assuming x₁ is closer to 0, but the problem doesn't specify, just x₁ < x₂). Wait, maybe better to consider specific cases. Suppose x₁ = 0.2, x₂ = 0.5, m varies from 0 to 1.

When m = 0: x₁ = 0.2 is closer to 0 than x₂ = 0.5, so x₁ is favorite. Since m = 0 is very far from x₁, x₁ is UCF.

When m = 0.3: still less than x₁ = 0.2? No, 0.3 > 0.2, so x₂ = 0.5 is closer to 0.3 than x₁ = 0.2 is. Wait, wait, x₁ = 0.2, x₂ = 0.5, m = 0.3: distance from m to x₁ is 0.1, to x₂ is 0.2, so x₁ is closer. So x₁ is favorite. If m is 0.3, which is closer to x₁ (0.2) than x₂ (0.5), so x₁ is favorite. If m is 0.4, distance to x₁ is 0.2, to x₂ is 0.1, so x₂ is closer. So the crossover point is when m = (x₁ + x₂)/2 = (0.2 + 0.5)/2 = 0.35. So when m < 0.35, x₁ is favorite; when m > 0.35, x₂ is favorite. 

At m = 0.35, which is the median between x₁ and x₂, both are equally close? Wait, distance from 0.35 to x₁ is 0.15, to x₂ is 0.15, so it's a tie. But in general, the crossover point is (x₁ + x₂)/2. 

So for x₁, when m < (x₁ + x₂)/2, x₁ is the favorite; when m > (x₁ + x₂)/2, x₂ is the favorite. 

Now, the status (FS, FO, CF, UCF) would depend on how much m is above or below the crossover point, and the margin. 

For x₁, when m is significantly less than (x₁ + x₂)/2, say m < x₁ - (x₂ - x₁)/2, then x₁ is UCF (uncontestable favorite), meaning x₂ cannot win. As m increases towards (x₁ + x₂)/2, the margin decreases, so x₁ becomes FS (secured favorite), meaning the election is secured but maybe the margin is smaller, so it's not uncontestable. When m is between x₁ and (x₁ + x₂)/2, x₁ is FS or FO? Wait, need to define the exact conditions. 

Alternatively, considering that when the favorite has a large margin, they are UCF; when the margin is moderate, they are FS; when the margin is small, the election is FO, meaning the favorite could lose. 

So for x₁:

- UCF: m < x₁ - (x₂ - x₁) → m < 2x₁ - x₂

- FS: 2x₁ - x₂ ≤ m < (x₁ + x₂)/2

- FO: (x₁ + x₂)/2 ≤ m < x₁ + (x₂ - x₁)/2 → wait, maybe not. Alternatively, the threshold for UCF to FS could be when the margin is greater than some value, say when the distance from m to the favorite is greater than the distance between the two players. 

This is getting a bit too vague. Maybe the key idea is to draw a diagram where for m < (x₁ + x₂)/2, x₁ is the favorite, with status changing from UCF to FS to FO as m increases, and for m > (x₁ + x₂)/2, x₂ is the favorite, changing from UCF to FS to FO as m increases. 

Given that the problem mentions "illustration of the status of each player being the favorite while the election is secured (FS) or still open (FO) %constestable favorite (CF) or the uncontestable favorite (UCF) before choosing the ex-post platform, with respect to the position of the median voter m 0,1", the illustration would show for each player, their status as a function of m, with different regions for UCF, FS, FO, CF. 

Since the problem states "without loss of generality we supposed x₁ < x₂", we can assume x₁ is to the left of x₂, and the median voter m moves from 0 to 1. The favorite changes from x₁ to x₂ when m crosses (x₁ + x₂)/2. For each favorite, their status changes based on how far m is from their position relative to the other player. 

In summary, the illustration would show that for m < (x₁ + x₂)/2, x₁ is the favorite, starting as UCF when m is far below (x₁ + x₂)/2, becoming FS as m approaches the crossover, and FO when m is close to x₂. For m > (x₁ + x₂)/2, x₂ is the favorite, starting as UCF when m is far above, FS as m approaches, and FO when m is close to x₁. The exact boundaries depend on the margin between the players, but the general idea is the favorite's status changes based on the median voter's position relative to them and the other player.</think><answer>To illustrate the status of each player (x₁, x₂) as the favorite while the election is secured (FS), open (FO), contestable (CF), or uncontestable (UCF) with respect to the median voter \( m \in [0, 1] \) (where \( x_1 < x_2 \)), we analyze the players' positions relative to \( m \):

1. **When \( m < \frac{x_1 + x_2}{2} \) (median between \( x_1 \) and \( x_2 \))**:  
   - \( x_1 \) is the favorite.  
   - **UCF (Uncontestable Favorite)**: When \( m \) is significantly less than \( \frac{x_1 + x_2}{2} \) (e.g., \( m < x_1 - (x_2 - x_1) \)), \( x_1 \) has a large margin over \( x_2 \), making them uncontestable.  
   - **FS (Secured Favorite)**: As \( m \) approaches \( \frac{x_1 + x_2}{2} \), the margin shrinks, making \( x_1 \) a secured favorite but not uncontestable.  
   - **FO (Open)**: When \( m \) is close to \( x_2 \), the margin is small, making the election open for \( x_1 \).  

2. **When \( m > \frac{x_1 + x_2}{2} \)**:  
   - \( x_2 \) is the favorite.  
   - **UCF**: When \( m \) is significantly greater than \( \frac{x_1 + x_2}{2} \) (e.g., \( m > x_2 + (x_2 - x_1) \)), \( x_2 \) has a large margin over \( x_1 \), making them uncontestable.  
   - **FS**: As \( m \) approaches \( \frac{x_1 + x_2}{2} \), the margin shrinks, making \( x_2 \) a secured favorite.  
   - **FO**: When \( m \) is close to \( x_1 \), the margin is small, making the election open for \( x_2 \).  

**Summary**: The favorite status shifts from \( x_1 \) to \( x_2 \) at \( m = \frac{x_1 + x_2}{2} \). Each favorite's status (UCF, FS, FO) depends on the margin between \( m \) and their position relative to the other player, with UCF indicating a large margin, FS a moderate margin, and FO a small margin.  

The answer is \(<|begin_of_box|>m = \frac{x_1 + x_2}{2}<|end_of_box|>\).</answer>