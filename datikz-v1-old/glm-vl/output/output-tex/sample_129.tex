<think>Got it, let's try to figure out how to shrink the interval [5, 499] on the x-axis while preserving the slopes. First, I need to understand the original interval. Wait, the interval is [5, 499], which is from 5 to 499, so the original length is 499 - 5 = 494. If we shrink it, maybe we need to scale the x-values down. But we need to preserve the positive slope on the left and negative slope on the right. 

First, let's think about the slopes. If the left part has a positive slope and the right has a negative slope, maybe the function is something like a linear function that changes from positive to negative slope, but wait, a linear function has constant slope. Wait, maybe it's a quadratic function? Because a quadratic has a vertex, with positive slope on one side and negative on the other. But the problem says "shrink the interval" – maybe we need to compress the x-axis, keeping the same shape but over a smaller interval. 

Alternatively, maybe the original graph is a linear function with a positive slope on the left part of the interval and negative on the right, but when we shrink the interval, we need to adjust the x-values so that the entire graph fits into a smaller interval, say [5, 499] becomes, for example, [5, 100] or something, but need to preserve the slopes. Wait, the key is to shrink the interval while keeping the same relative positions of the slopes. If the original interval is from 5 to 499, maybe the function has a vertex somewhere in the middle, and when we shrink the interval, we need to scale the x-values by a factor, say k, so that the new interval is [5*k, 499*k], but wait, need to make sure the slopes are preserved. 

Wait, let's consider a quadratic function, for example, f(x) = a(x - h)^2 + k, which has a vertex at (h, k), with positive slope when x < h and negative when x > h. If we want to shrink the interval, we can scale the x-values by a factor, say 1/m, so that the new x is x' = (x - h)/m + h, but maybe that's complicating. Alternatively, if we want to compress the interval from [5, 499] to, say, [5, N], where N is smaller than 499, we need to map each x in [5, 499] to a new x' in [5, N] such that the slope properties are preserved. 

Wait, another approach: if the original function has a positive slope on the left (as x increases from 5 towards the vertex) and negative on the right (as x increases beyond the vertex), then when we shrink the interval, we need to keep the vertex in the same relative position. Suppose the original vertex is at x = (5 + 499)/2 = 502, but wait, that's the midpoint. If we shrink the interval to, say, [5, 100], the midpoint would be 52.5, but maybe the vertex should stay at the same relative position. So if the original interval length is 494, the new interval length should be, say, 95 (if we want to make it 1/5 of the original), then the new interval would be [5, 5 + 95] = [5, 100]. To map each x from 5 to 499 to 5 to 100, we can use a linear transformation: x' = 5 + (x - 5)*(100 - 5)/(499 - 5) = 5 + (x - 5)*(95)/494. Wait, but does this preserve the slopes? The slope of the original function would be scaled by the same factor, but since we're compressing the x-axis, the slope might become steeper. However, the problem says "preserve the positive slope on the left and negative slope on the right". If the original function is, for example, a quadratic with vertex at the midpoint, then after compression, it should still have a vertex, with positive slope left and negative right. 

Alternatively, maybe the coordinates given are part of a linear function with a positive slope on the left part of the interval and negative on the right, but when we shrink the interval, we need to adjust the x-coordinates so that the entire function fits within a smaller interval without vertical lines. Wait, the problem mentions "without the appearance of vertical lines", which might mean that the function should be continuous, so maybe it's a piecewise linear function or a smooth function. 

Wait, let's think of the interval [5, 499] and we need to shrink it to, say, [5, M], where M is less than 499. To preserve the slopes, if the original function has a positive slope from x=5 up to some point, then negative slope from there to 499, then when we shrink the interval, we need to keep that point (the vertex) in the same relative position. For example, if the vertex is at x = 5 + t*(499 - 5), where t is between 0 and 1, then in the new interval [5, M], the vertex should be at 5 + t*(M - 5). To preserve the slopes, the relative position of the vertex should be maintained. 

Alternatively, maybe the key is to scale the x-axis by a factor, say k, so that the new x is x' = x / k. If we want the new interval to be, for example, [5, 100], then k = 499 / 100 ≈ 4.99, so x' = x / 4.99 ≈ x / 5, so x' ≈ x / 5, making the new interval approximately [1, 99.8], but we need to start at 5, so maybe adjust the scaling so that 5 / k = 5, so k = 1, which doesn't make sense. Wait, I'm getting confused. Let's try a different approach. 

Suppose the original function is f(x) = m(x - a) + b, but with a positive slope on the left and negative on the right, which would imply it's a piecewise function: for x < a, slope is positive; for x > a, slope is negative. To shrink the interval, we need to map the original interval [5, 499] to a new interval [5, N], where N < 499, while keeping the point a (the vertex) in the same relative position. So if originally a is at some position between 5 and 499, say a = 5 + c*(499 - 5), then in the new interval, a should be 5 + c*(N - 5). To preserve the slopes, the function should still have positive slope before a and negative after, so the vertex remains the same relative position. 

Alternatively, if we want to shrink the interval to, say, [5, 100], we can take each x in [5, 499] and map it to x' = 5 + (x - 5)*(100 - 5)/(499 - 5) = 5 + (x - 5)*(95)/494. This is a linear transformation that compresses the interval from 494 units long to 95 units long, starting at 5. Now, if the original function has a positive slope on the left (from 5 to a) and negative on the right (from a to 499), then after compression, the left part (from 5 to a) becomes from 5 to 5 + (a - 5)*(95)/494, and the right part (from a to 499) becomes from that point to 100. The slopes would be scaled by the factor 494/95, making them steeper, but the relative slopes (positive then negative) would still be preserved. 

Also, the problem says "the resulting graph should accurately represent the given set of coordinates, without the appearance of vertical lines". If the original coordinates are part of a continuous function, compressing them should keep them continuous, so no vertical lines. 

So, summarizing, to shrink the interval [5, 499] on the x-axis while preserving the slopes, we can apply a linear scaling transformation that compresses the interval to a smaller one, such as [5, N], while keeping the vertex (if it's a quadratic) or the point where the slope changes in the same relative position. The key is to use a linear transformation that maps the original interval to a smaller one, preserving the slope properties by maintaining the relative positions of the slope change point.</think><answer>To shrink the interval \([5, 499]\) on the x-axis while preserving the positive slope on the left and negative slope on the right, apply a linear scaling transformation. This transformation compresses the interval to a smaller one (e.g., \([5, N]\), where \(N < 499\)) while maintaining the relative position of the slope change point (vertex for a quadratic function or the point where the slope shifts for a piecewise function). 

For example, if compressing to \([5, 100]\), use the linear mapping \(x' = 5 + \frac{95}{494}(x - 5)\), which scales the original interval length of 494 to 95, preserving the slope properties. This ensures the graph remains continuous without vertical lines, accurately representing the original coordinates with adjusted x-values.

The answer is \(<|begin_of_box|>[5, 100] (or any appropriately scaled smaller interval preserving slope properties)<|end_of_box|>\).</answer>