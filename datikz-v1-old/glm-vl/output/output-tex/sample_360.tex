<think>Got it, let's try to figure out what's being asked here. The question is about H*(Q₄), which is the cohomology algebra of the quaternionic projective space Q₄. The relation {1} M₁ = 0, called Relation relation1, needs to be analyzed. 

First, I remember that quaternionic projective spaces have cohomology rings with generators in even degrees. For Qₙ, the cohomology ring H*(Qₙ; ℚ) is isomorphic to ℚ[x]/(x^{n+1}), where x is in degree 2, but wait, maybe I need to check the specific degrees. Wait, Q₄ is the quaternionic projective space, so its dimension is 4*4 - 1 = 15? Wait, no, the formula for quaternionic projective space Qₙ is (ℍPⁿ), which has dimension 4n - 1. So Q₄ has dimension 15, so its cohomology has generators in degrees 2, 4, 6, 8, 10, 12, 14? Wait, maybe the cohomology ring structure is more like a polynomial ring with degree 2 generators, but with some relations. Wait, another approach: the cohomology of ℍPⁿ is a polynomial ring on a class α in degree 2, with α^{n+1} = 0. Wait, for n=4, that would be α^5 = 0? But maybe the relation here is about a specific class M₁. Wait, the notation {1} M₁ = 0 might refer to the cup product of the generator {1} (which is the unit class, maybe the generator in degree 0, but wait, in cohomology, the unit is in degree 0, but maybe M₁ is a class in degree 2? Wait, the problem says "the relation {1} M₁ = 0", so the cup product of the unit class (which is the generator in degree 0) with M₁ equals zero. But the unit class in cohomology is the generator of degree 0, and cupting with it should be the identity, unless M₁ is a class such that when you cup with the unit, it's zero. Wait, maybe M₁ is a class in degree 2, and {1} is the degree 0 class, so their cup product is M₁, but the relation says {1} M₁ = 0, which would imply M₁ = 0? But that doesn't make sense. Wait, perhaps I need to recall the structure of H*(ℍP⁴). The cohomology ring is generated by a class α in degree 2, with α⁵ = 0, and the ring is ℚ[α]/(α⁵), but maybe with some other relations? Wait, or maybe M₁ is a class in degree 2, and the relation is that the square of α is zero? Wait, no, let's think step by step.

First, the cohomology ring of ℍPⁿ has a generator α in degree 2, and the relation is α^{n+1} = 0. For n=4, that would be α^5 = 0. So H*(ℍP⁴) = ℚ[α]/(α^5), with α in degree 2. Now, if M₁ is α, then α^2 would be a class in degree 4, α^3 in 6, etc. But the relation {1} M₁ = 0: if {1} is the unit class in degree 0, then cup product of degree 0 and degree 2 is the degree 2 class, which is M₁. So {1} M₁ = M₁, so the relation {1} M₁ = 0 would imply M₁ = 0, but that can't be right. Wait, maybe M₁ is a class in degree 1? But cohomology of ℍPⁿ with real coefficients (or rational coefficients) has only even degrees, right? Because it's a space with real orientation, so cohomology is in even degrees. So M₁ must be in an even degree. Wait, maybe the notation {1} refers to the generator in degree 0, and M₁ is a class in degree 2, so their cup product is M₁, and the relation says M₁ = 0? But then why is the relation called relation1? Alternatively, maybe the relation is about the cup product of two classes. Wait, the question says "the relation _{1} M₁ = 0", maybe the subscript is a typo, or maybe it's the cup product symbol. If it's the cup product of {1} and M₁ equals zero, then since {1} is the unit, that would mean M₁ = 0, but that seems trivial. Alternatively, maybe M₁ is a class such that when you take its cup product with something, but the problem states the relation as {1} M₁ = 0. Wait, another angle: in some cohomology rings, there might be a relation where a certain class squared is zero, but the notation here is different. Wait, let's recall that for quaternionic projective space, the cohomology ring is isomorphic to the polynomial ring on one generator in degree 2, modulo the relation that the generator raised to the 5th power is zero (since 4+1=5). So if M₁ is the generator α, then α^5 = 0. But the relation here is {1} M₁ = 0, which would be 1 * α = α = 0, implying α = 0, which contradicts the cohomology being non-trivial. Hmm, maybe I'm misunderstanding the notation. Perhaps {1} is a symbol for the unit class, and M₁ is a class in degree 2, and the relation is that their cup product is zero, which would mean that M₁ is zero, but that doesn't fit. Wait, maybe the problem is referring to the fact that in the cohomology ring, the square of the generator is not zero, but higher powers are. Wait, let's check the cohomology groups: H^*(ℍP⁴; ℚ) has dimensions 1 in degrees 0, 2, 4, 6, 8, 10, 12, 14, 16? Wait, no, the dimension should be 2^(4+1) - 1 = 32 - 1 = 31? Wait, no, the formula for the cohomology dimension of ℍPⁿ is 2^(n+1) - 1. For n=4, that's 2^5 - 1 = 32 - 1 = 31. The degrees would be 0, 2, 4, ..., 14 (since 2*7=14, 2*8=16 which is too big). Wait, 2*(4+1)-2 = 14? Wait, maybe I need to look up the cohomology ring structure, but since I can't do that, let's think about the general case. For ℍPⁿ, the cohomology ring is a polynomial ring on a class α in degree 2, with α^{n+1} = 0. So for n=4, α^5 = 0. Therefore, the ring is ℚ[α]/(α^5), with α in degree 2. Now, if M₁ is α, then the relation would be α^5 = 0, but the question mentions {1} M₁ = 0. If {1} is the unit, then 1 * α = α = 0, which would imply α=0, which is not the case. Alternatively, maybe M₁ is α^2, then {1} M₁ = α^2, and if the relation is α^2 = 0, but that would mean the ring is truncated at degree 2, which is not the case for n=4. Wait, I'm getting confused. Let's try to parse the question again: "In H^*(Q_4), the relation _{1} M_1 = 0 holds (Relation relation1)." Maybe the subscript 1 is a typo, and it's the cup product symbol, so ∪_{1} M_1 = 0, meaning the cup product of the class {1} (degree 0) with M₁ (degree k) is zero. But the cup product of a degree 0 class with any class is just the class itself, so {1} ∪ M₁ = M₁, so the relation would be M₁ = 0. But that can't be the case. Alternatively, maybe M₁ is a class in degree 2, and the relation is that its square is zero, i.e., M₁^2 = 0, but the question says {1} M₁ = 0. Wait, another thought: in some contexts, the notation {1} might refer to the fundamental class, but in cohomology, the fundamental class is in top degree, which for Q₄ is degree 14, so that doesn't fit. Wait, maybe the relation is about the product of two classes, one of which is the generator and the other is something else. Given that I'm stuck, let's recall that for quaternionic projective space Qₙ, the cohomology ring has a generator α in degree 2, and the relation is α^{n+1} = 0. For n=4, α^5 = 0. If M₁ is α, then the relation would be α^5 = 0, but the question states {1} M₁ = 0. If {1} is the unit, then this doesn't fit. Alternatively, maybe the relation is that the square of the generator is zero, but that would be α^2 = 0, which would make the ring have dimension 2, but Q₄ has higher cohomology. I think I need to conclude that the relation is likely α^5 = 0, where α is the generator in degree 2, but the question mentions {1} M₁ = 0. Given the confusion, maybe the answer is that the relation is the trivial one, but I need to make sense of it. Wait, another approach: in the cohomology ring, if you have a class M₁, then the relation {1} M₁ = 0 implies that M₁ is zero, but that's probably not the case. Alternatively, maybe the relation is about the cup product of M₁ with itself being zero, i.e., M₁^2 = 0, but the question says {1} M₁ = 0. Given that I'm not making progress, I'll assume that the relation is the standard one for the cohomology ring of Q₄, which is the generator raised to the 5th power is zero, but expressed as a relation involving M₁. If M₁ is the generator α, then α^5 = 0, which could be written as α^5 = 0, but the question says {1} M₁ = 0. I'm probably overcomplicating. Given the problem states the relation as {1} M₁ = 0, and in cohomology, the unit class times any class is the class itself, so this would imply M₁ = 0, but that's trivial. Therefore, maybe the intended answer is that the relation is the trivial one, or that M₁ is zero, but I need to state the relation as given. Wait, the question says "the relation _{1} M_1 = 0 holds", maybe the underscore is a typo for the cup product symbol, so ∪ M₁ = 0, but with {1} as the generator. Given the confusion, I'll proceed to state that in H*(Q₄), the cohomology ring has a generator in degree 2, say α, with α^5 = 0, and if M₁ is α, then the relation is α^5 = 0, but the question mentions {1} M₁ = 0, which might be a misstatement, and the correct relation is α^5 = 0. However, since the question specifically says {1} M₁ = 0, and given that {1} is the unit, this would imply M₁ = 0, but I need to confirm. Alternatively, maybe M₁ is a class in degree 2, and the relation is that its cup product with the generator is zero, but I'm getting stuck. Given the time I've spent, I'll conclude that the relation is the trivial one or the standard one for the cohomology ring, and state that in H*(Q₄), the relation is α^5 = 0, where α is the generator in degree 2, which might be expressed as {1} M₁ = 0 if M₁ is α^4 or something, but I'm not sure. Wait, another way: if M₁ is a class in degree 2, then {1} M₁ is in degree 2, and setting it to zero implies M₁ = 0, but that's not useful. Given the problem's phrasing, I think the intended answer is that the relation is the trivial one, or the standard cohomology relation, and the answer is that the relation holds because M₁ is zero, but I need to make a final conclusion.</think><answer>In the cohomology ring \( H^*(Q_4) \) (where \( Q_4 \) is the quaternionic projective space), the relation \( \{1\} M_1 = 0 \) implies that the class \( M_1 \) must be trivial, as the cup product of the unit class \( \{1\} \) (in degree 0) with any class \( M_1 \) is \( M_1 \) itself. However, considering the standard cohomology structure of \( Q_4 \), which has a generator \( \alpha \) in degree 2 with the relation \( \alpha^5 = 0 \) (since \( 4 + 1 = 5 \)), if \( M_1 = \alpha \), the relation simplifies to \( \alpha^5 = 0 \). Despite potential notation variations, the core relation in \( H^*(Q_4) \) involves the generator's high-degree truncation, leading to the conclusion that the specified relation holds as a consequence of the cohomology ring's structure.  

The answer is \(<|begin_of_box|>0<|end_of_box|>\).</answer>