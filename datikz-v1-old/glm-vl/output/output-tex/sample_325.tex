<think>Got it, let's try to figure out how to draw a larger angle outside two smaller angles. First, I need to recall that when you have two angles, say angle A and angle B, if they are adjacent and their non-common sides form a straight line, then the angle outside would be supplementary? Wait, no, if they are on the outside, maybe it's the sum of the two angles. Wait, the problem says "drawing a larger angle outside two smaller angles," so probably forming an exterior angle by extending one of the sides of the two smaller angles. 

First, let's visualize the diagram. There's a white triangle on a black background with sides numbered 0 to 2. The numbers are at the intersections, like a coordinate plane. So maybe the triangle has vertices labeled with these numbers. Let's assume the triangle is labeled with vertices at points 0, 1, 2, forming a triangle with sides 0-1, 1-2, 2-0. Then, if we have two smaller angles at a vertex, say at vertex 1, with angles between sides 1-0 and 1-2, and maybe another angle at vertex 0, but need to form a larger angle outside. 

To form a larger angle outside two smaller angles, you might extend one side of each angle. For example, if you have two angles, ∠A and ∠B, meeting at a point, and you extend one side of ∠A and one side of ∠B so that they form a larger angle outside. The measure of the larger angle would be the sum of the two smaller angles if they are adjacent and their non-common sides form a straight line. Wait, if you have two angles, say ∠1 and ∠2, with a common vertex, and you extend one side of ∠1 and one side of ∠2, the angle between the extensions would be 180° - ∠1 - ∠2, but maybe I need to think in terms of the triangle. 

Alternatively, since the triangle has sides numbered 0-2, maybe the angles at each vertex are related to the sides. For example, at vertex 0, the sides are 0-1 and 0-2, forming angle ∠0. At vertex 1, sides 1-0 and 1-2 form angle ∠1, and at vertex 2, sides 2-1 and 2-0 form angle ∠2. If we want to draw a larger angle outside, say at vertex 0, extending side 0-1 beyond 0 or 0-2 beyond 0, then the angle outside would be the supplementary angle to ∠0. But if we have two smaller angles, maybe combining two angles from different vertices. 

Wait, the key idea is that when you have two angles, say ∠A and ∠B, and you place them such that their non-common sides form a straight line, then the angle between their common sides is ∠A + ∠B. But if they are outside, maybe the larger angle is the sum of the two smaller angles. To draw this, you would extend one side of each of the two smaller angles until they meet, forming the larger angle. 

Let's consider the triangle with vertices labeled 0, 1, 2. Suppose we have two smaller angles at vertex 1: one between sides 1-0 and 1-2, and maybe another angle if there's a point inside, but the problem says "two smaller angles" probably meeting at a common vertex. Let's say at vertex 1, there are two angles, ∠1a and ∠1b, such that ∠1a + ∠1b = ∠1 (the angle of the triangle at vertex 1). To form a larger angle outside, extend one side of ∠1a and one side of ∠1b beyond vertex 1, so that the angle between the extensions is 180° - ∠1a - ∠1b = 180° - ∠1, but maybe that's not right. Wait, if you extend one side of each angle outward, the angle between the two extensions would be equal to the sum of the two smaller angles. For example, if you have two angles, each 30°, then the larger angle outside would be 60°. 

In terms of the triangle with sides numbered 0-2, maybe the angles at each vertex are 60° if it's an equilateral triangle, but the problem doesn't specify, so we need to make sure the angles are proportional. The key steps would be: identify the two smaller angles, extend their non-common sides to form a larger angle outside, ensuring the angle is the sum of the two smaller angles, and represent the sides accurately with the numbered coordinate plane. 

So, step by step: 1. Draw the two smaller angles, ensuring they are adjacent or meeting at a common vertex. 2. Extend one side of each angle beyond the vertex, so that the two extended sides form the larger angle outside. 3. Measure the angles to make sure the larger angle is the sum of the two smaller ones, maintaining proportionality. 4. Ensure the triangle's sides are numbered correctly, with the coordinate plane at the intersections, so the numbers 0,1,2 are placed appropriately at the vertices or along the sides. 

Wait, the problem mentions a white triangle on a black background with sides numbered 0 to 2, numbers at intersections forming a coordinate plane. So maybe the triangle is a right triangle or equilateral, but the numbering is like a coordinate system where each side has numbers 0,1,2, with 0 and 2 at the ends of a side, and 1 in the middle. So each side has three points: 0,1,2, with 0 and 2 being the endpoints, 1 the midpoint. Then, the angles at each vertex (0,1,2) would be between the two sides meeting there. To form a larger angle outside, say at vertex 0, extend one of the sides meeting at 0, say the side from 0 to 1, beyond 0 to a new point, and extend the other side from 0 to 2 beyond 0 to another point, then the angle between these two extended sides at the new point would be the larger angle. Wait, this is getting a bit confusing. Maybe a better approach is to recall that the exterior angle theorem states that an exterior angle is equal to the sum of the two opposite interior angles. So if you have a triangle, the exterior angle at a vertex is equal to the sum of the two non-adjacent interior angles. So drawing an exterior angle would involve extending a side of the triangle, and the angle outside is equal to the sum of the two opposite angles. 

Applying that here, if we have two smaller angles inside the triangle, their sum would equal the exterior angle. So to draw the larger angle outside, extend one side of the triangle, say from vertex 2 to 0, beyond 0, and then the angle between the extended side and the side from 0 to 1 would be the exterior angle, equal to the sum of the two opposite angles in the triangle. 

In summary, the process would be: 1. Draw the two smaller angles inside the triangle, ensuring they are at a vertex, say vertex 1, with angles ∠1a and ∠1b. 2. Extend one side of ∠1a and one side of ∠1b beyond the vertex, forming a new angle outside the triangle. 3. The measure of this new angle should be ∠1a + ∠1b, which is the sum of the two smaller angles. 4. Ensure the sides are numbered correctly with 0,1,2 at the intersections, forming a coordinate plane, so the angles are accurately represented with the numbered sides. 

I think the key is using the exterior angle theorem, where the exterior angle is the sum of the two non-adjacent interior angles, and drawing that exterior angle outside the triangle. That way, the larger angle is formed by extending a side, and its measure is the sum of the two smaller angles, maintaining proportionality.</think><answer>To draw a larger angle outside two smaller angles, follow these steps:  

1. **Identify the two smaller angles**: Locate the two smaller angles inside the triangle, typically at a common vertex (e.g., vertex 1 in the triangle with sides labeled 0–2).  
2. **Extend non-common sides**: Extend one side of each smaller angle beyond their shared vertex. The angle formed between these two extended sides outside the triangle will be the larger angle.  
3. **Apply the exterior angle theorem**: By the exterior angle theorem, the measure of the larger angle equals the sum of the two smaller angles. Ensure the angles are proportional and accurately represented.  
4. **Verify the coordinate plane**: Position the numbers 0, 1, 2 at the intersections of the triangle’s sides to form a coordinate plane, ensuring the sides and angles align with the numbered labels.  

The larger angle outside the two smaller angles is the sum of their measures, formed by extending their non-common sides, maintaining proportionality with the triangle’s numbered sides.  

The answer is \(<|begin_of_box|>180^\circ<|end_of_box|>\).</answer>