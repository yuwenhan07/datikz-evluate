<think>Got, let's try to tackle this problem step by step. First, the question is about an edge-color-avoiding connected graph on 8 vertices with the minimum number of edges colored with exactly 4 colors. Wait, maybe I need to parse the question again. Wait, the original question might have some typos or maybe the description is mixed with an image description. Let me read again:

"An edge-color-avoiding connected graph on 8 vertices and with minimum number of edges colored with exactly 4 colors. The image features a white cable with small beads attached to it, creating a colorful pattern. The beads are arranged in a linear sequence, and there are two visible sections of the cable with beads on them. The overall appearance of the image is simple and eye-catching, with the brightly colored beads standing out against the white background."

Hmm, maybe the image description is a distraction, and the actual question is about graph theory: find a connected graph on 8 vertices with edge-coloring (using 4 colors) such that the number of edges colored with exactly 4 colors is minimized. Wait, edge-coloring usually means coloring edges so that no two adjacent edges have the same color. The minimum number of colors needed for a connected graph is related to its maximum degree. For a graph with maximum degree Δ, the edge chromatic number is at most Δ+1, by Vizing's theorem. But the question says "with minimum number of edges colored with exactly 4 colors". Maybe it's asking for a graph that can be edge-colored with 4 colors, and the number of edges that need to be colored (i.e., the number of edges, since each edge is colored) with exactly 4 colors, but maybe the minimum number of edges colored with a specific color? Wait, maybe the question is to find a connected graph on 8 vertices that can be edge-colored with 4 colors, and the number of edges colored with each color is minimized, or the total number of edges colored (but all edges need to be colored). Wait, edge-coloring requires that every edge is colored, and adjacent edges have different colors. So if we use 4 colors, the graph must be such that its edge chromatic number is at most 4. So we need a connected graph on 8 vertices with χ'(G) ≤ 4, and we want the minimum number of edges (but all edges must be colored, so maybe the question is about the number of edges colored with each color, or the total number of edges, but since it's a graph with 8 vertices, the number of edges is at least 7 (for a tree) and at most 28 (complete graph). But the question says "minimum number of edges colored with exactly 4 colors". Maybe it's asking for the minimum number of edges that need to be colored (i.e., the number of edges, since each edge is colored) when using 4 colors, but that doesn't make sense. Wait, maybe the question is to find a connected graph on 8 vertices that can be edge-colored with 4 colors, and the number of edges colored with each color is as balanced as possible, or the minimum number of edges colored with a particular color. Alternatively, maybe the question is misphrased, and it's about vertex coloring, but the question says edge-color-avoiding connected graph. Wait, "edge-color-avoiding" might mean that the graph doesn't have any monochromatic edges or something? Not sure. Alternatively, maybe the problem is to find a connected graph on 8 vertices with edge chromatic number 4, and the minimum number of edges (i.e., the number of edges) such that when you color the edges with 4 colors, each color is used as few times as possible. Wait, the minimum number of edges colored with exactly 4 colors would mean that each color is used on at least one edge, and we want the number of edges colored with 4 colors (i.e., the total number of edges, since each edge is colored with one of 4 colors) but that's just the number of edges. Wait, maybe the question is asking for a connected graph on 8 vertices with the minimum number of edges (i.e., a tree, which has 7 edges) that can be edge-colored with 4 colors. But a tree is bipartite, so its edge chromatic number is equal to its maximum degree. If it's a tree with maximum degree Δ, then χ'(G) = Δ. So if we want χ'(G) ≤ 4, then the maximum degree should be at most 4. So a tree with maximum degree 4 has χ'(G) = 4, and the number of edges is Δ* + 1, where Δ* is the maximum degree. Wait, for a tree, the number of edges is n-1 = 7, so if it's a tree with maximum degree 4, then it has 7 edges, each edge can be colored with 4 colors, but the edge chromatic number is 4, meaning that the maximum degree is 4, so the edge coloring uses 4 colors. But the question says "minimum number of edges colored with exactly 4 colors". Maybe the question is looking for a graph with the minimum number of edges (i.e., a tree, 7 edges) that requires exactly 4 colors for edge-coloring. But a tree with maximum degree 3 can be edge-colored with 3 colors, so to need 4 colors, the maximum degree must be at least 4. So a graph with 8 vertices, maximum degree 4, connected, with 8*4/2 = 16 edges (wait, no, the number of edges in a graph is at most n(n-1)/2 = 28, but the number of edges for maximum degree Δ is at most (nΔ)/2). For n=8, Δ=4, maximum edges is 16. But a connected graph with 8 vertices and maximum degree 4 must have at least n-1=7 edges (if it's a tree), but if it's not a tree, it has more edges. Wait, maybe the question is asking for a connected graph on 8 vertices that can be edge-colored with 4 colors, and the number of edges colored with each color is minimized, so the total number of edges would be divided by 4, so if there are E edges, then each color is used E/4 times, so E should be a multiple of 4. The minimum number of edges for a connected graph on 8 vertices is 7 (a tree), but 7 isn't divisible by 4. The next is 8 edges, then 9, 10, etc. The first multiple of 4 ≥7 is 8, then 12, 16, etc. The minimum number of edges would be 8, but can a connected graph with 8 vertices and 8 edges be edge-colored with 4 colors? Let's see, a connected graph with 8 vertices and 8 edges: the minimum number of edges for a connected graph is 7, so 8 edges would make it a connected graph with one cycle. The edge chromatic number for a graph with an odd cycle: if it has an odd cycle, then χ'(G) = Δ + 1 if Δ is even, or Δ if Δ is odd. Wait, if the graph has an odd cycle, then its edge chromatic number is at least 3 if it's a triangle, but for a graph with maximum degree Δ, if Δ is even, χ'(G) ≤ Δ, if Δ is odd, χ'(G) ≤ Δ + 1. So if we have a connected graph with 8 vertices, 8 edges, maximum degree Δ. Let's say it's a cycle of length 8, which is a 2-regular graph, so Δ=2, then χ'(G)=2, which is less than 4. If it's a graph with maximum degree 3, like a tree plus one edge, making it a connected graph with 8 vertices and 8 edges (since a tree has 7 edges, adding one edge makes 8), then Δ=3 (if the added edge connects two vertices of degree 1 and 2, making their degrees 2 and 3), so χ'(G)=3, which is less than 4. To have χ'(G)=4, the graph must have a higher maximum degree. For example, a complete graph on 4 vertices (which has 6 edges) plus 2 vertices connected in a way that the maximum degree is 4. Wait, maybe a graph with maximum degree 4, like a complete bipartite graph K_{4,4} has 16 edges, maximum degree 4, and its edge chromatic number is 4, since it's bipartite, so χ'(K_{4,4})=4. But K_{4,4} has 8 vertices and 16 edges. The question is asking for the minimum number of edges colored with exactly 4 colors, so maybe the minimum number of edges is 16? But that seems high. Wait, the question says "minimum number of edges colored with exactly 4 colors", which might mean that each color is used on at least one edge, and we want the number of edges (total) to be as small as possible. For a connected graph with χ'(G)=4, the minimum number of edges is when the graph is a complete graph with χ'(G)=Δ+1. For example, if Δ=3, χ'(G)=4, then the minimum number of edges would be a complete graph on 4 vertices (which has 6 edges, χ'=4), but that's only 4 vertices. For 8 vertices, if we take a complete graph on 8 vertices, χ'(G)=8 (since Δ=7), which is more than 4. So we need a graph with Δ ≤ 4, so χ'(G) ≤ 5, but we want χ'(G)=4, so Δ must be 3 or 4. If Δ=4, then χ'(G) ≤ 5, but can be 4 if the graph is bipartite? Wait, a graph is bipartite if and only if it has no odd cycles, and for bipartite graphs, χ'(G)=Δ. So if we have a bipartite graph with maximum degree 4, then χ'(G)=4, and the number of edges is at least 2*4=8 (if it's a complete bipartite graph K_{4,4}, which has 16 edges). Wait, K_{4,4} has 16 edges, which is a connected bipartite graph with 8 vertices, maximum degree 4, edge chromatic number 4. Is there a connected bipartite graph with 8 vertices and fewer edges that still has χ'(G)=4? For example, a bipartite graph with partitions of size 4 and 4, with each vertex connected to 4 vertices on the other side, making it K_{4,4}, which has 16 edges. If we reduce the number of edges, say to 8 edges, then the maximum degree would be lower, maybe 2, so χ'(G)=2. So to have χ'(G)=4, we need a bipartite graph with high enough maximum degree. Alternatively, a non-bipartite graph with χ'(G)=4. For example, a graph with an odd cycle, but χ'(G)=Δ+1 if Δ is even, or Δ if Δ is odd. If Δ=4 (even), then χ'(G)=5, which is too much. If Δ=3 (odd), χ'(G)=4. So a graph with maximum degree 3, which is not bipartite (has odd cycles), can have χ'(G)=4. For example, a graph that's a tree plus some edges, but with maximum degree 3. Wait, a tree has χ'(G)=1, adding edges increases the maximum degree. If we have a connected graph with 8 vertices, maximum degree 3, then χ'(G)=4. The minimum number of edges for a connected graph with 8 vertices and max degree 3 is 7 (a tree), but a tree with max degree 3 has χ'(G)=3, so to have χ'(G)=4, it must have a higher max degree. So maybe a graph with max degree 4, χ'(G)=5, but we need χ'(G)=4, so maybe a graph with max degree 3 and some odd cycles. This is getting a bit confusing. Let's recall that the edge chromatic number is the minimum number of colors needed, so if we want χ'(G)=4, then the graph must have χ'(G) ≤4 and χ'(G) ≥4. So the graph must be such that it's not 3-edge-colorable, but 4-edge-colorable. For example, a graph with an odd cycle has χ'(G) = Δ + 1 if Δ is even, or Δ if Δ is odd. So if we have a graph with an odd cycle and maximum degree Δ, then if Δ is even, χ'(G)=Δ+1, if Δ is odd, χ'(G)=Δ. So to have χ'(G)=4, we need either Δ=3 (odd, so χ'=3) or Δ=4 (even, χ'=5). Wait, that doesn't fit. Wait, Δ=3 (odd), χ'(G)=3, Δ=4 (even), χ'(G)=5. So maybe a graph with maximum degree 3 and some odd cycles would have χ'(G)=4? Wait, no, if Δ=3 and the graph has an odd cycle, then χ'(G)=4? Wait, let's take a triangle (3 vertices, 3 edges), χ'(G)=3. If we add a vertex connected to all three vertices of the triangle, making a graph with 4 vertices, one connected to all three, so maximum degree 3. The edge chromatic number would be 4? Wait, the new vertex connected to the three vertices: the edges from the new vertex to the triangle vertices can be colored with three different colors, and the triangle edges need to be colored with the remaining colors. Wait, maybe this is getting too complicated. Let's think about the minimum number of edges. The question says "minimum number of edges colored with exactly 4 colors". If we need to color the edges with 4 colors, the minimum number of edges would be when each color is used as few times as possible, but since the graph is connected, it must have at least n-1=7 edges. If we need to use 4 colors, then the number of edges must be at least the ceiling of (total edges)/4. But maybe the question is asking for a connected graph on 8 vertices that can be edge-colored with 4 colors, and the number of edges colored with each color is minimized, so the total number of edges is divided by 4, so total edges should be a multiple of 4. The smallest multiple of 4 greater than or equal to 7 (the minimum edges for a connected graph) is 8, then 12, 16, etc. A connected graph with 8 vertices and 8 edges is a connected graph with one cycle (since 8-1=7, plus one more edge makes a cycle). Such a graph has χ'(G)=3 if it's a bipartite graph (if the cycle is even-length), or χ'(G)=4 if the cycle is odd-length. Wait, if the cycle is of odd length, say 5, then the graph has a 5-cycle plus some edges, making it non-bipartite, so χ'(G)=4. If the cycle is even, like 4, then it's bipartite, χ'(G)=2. So a connected graph with 8 vertices and 8 edges, which is a 5-cycle plus 3 edges connecting to the rest, might have χ'(G)=4. But I need to confirm. Alternatively, the minimum number of edges for a connected graph with χ'(G)=4 is when the graph is a complete graph on 4 vertices plus 4 vertices connected in a way that the total graph has χ'(G)=4. But I'm getting stuck here. Maybe the answer is a complete bipartite graph K_{4,4}, which has 8 vertices, 16 edges, and χ'(G)=4, since it's bipartite, so edge chromatic number equals maximum degree, which is 4. But the question says "minimum number of edges", so maybe a smaller graph. Wait, another approach: the minimum number of edges for a connected graph with edge chromatic number 4 is a graph with maximum degree 4, and since the edge chromatic number is at most Δ+1, if Δ=4, then χ'(G) ≤5, but we need χ'(G)=4, so the graph must be such that it's not (Δ+1)-edge-colorable, which would be if it has an odd cycle when Δ is even. Wait, K_{4,4} is bipartite, so χ'(G)=4, and it has 16 edges. If we take a subgraph of K_{4,4} with fewer edges, say 8 edges, then it's not connected or not bipartite, but maybe still has χ'(G)=4. This is getting too vague. Given the confusion, maybe the intended answer is a cycle graph with 8 vertices, which is 2-regular, χ'(G)=2, which is less than 4, so not suitable. A complete graph on 8 vertices has χ'(G)=8, too much. A tree has χ'(G)=1. A graph with maximum degree 3 has χ'(G)≤4, but if it's a tree with max degree 3, χ'(G)=3. To have χ'(G)=4, it must have a higher max degree, say 4, then χ'(G)≤5, but can be 4 if it's bipartite. So a connected bipartite graph with max degree 4, like K_{4,4}, which has 8 vertices and 16 edges, χ'(G)=4. But the question says "minimum number of edges", so maybe the minimum is 8 edges? Wait, I'm stuck. Given the image description is probably a red herring, and the question is about graph theory, the key is to find a connected graph on 8 vertices with edge chromatic number 4, and the minimum number of edges would be when it's a complete bipartite graph K_{4,4} with 16 edges, but maybe a smaller one. Alternatively, the minimum number of edges colored with exactly 4 colors is 8, but I need to conclude. Given the confusion, I think the intended answer is a cycle graph with 8 vertices, but no, its edge chromatic number is 2. Wait, maybe the graph is a complete graph on 4