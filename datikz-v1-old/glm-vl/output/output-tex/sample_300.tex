<think>Got it, let's try to figure out how to compute the packaged surface Tutte polynomial using deletion and contraction. First, I need to recall what the Tutte polynomial is and how deletion and contraction work for graphs, especially trees since the image mentions a tree. Wait, the Tutte polynomial T(G; x, y) for a graph G has the property that T(G) = T(G - e) + T(G / e) for an edge e, where deletion (G - e) removes e, and contraction (G / e) contracts e. For a tree, which is a connected acyclic graph, maybe we can use the fact that trees have a simpler Tutte polynomial.

First, let's recall that for a tree, the Tutte polynomial can be computed by considering the number of spanning trees, but maybe more directly using deletion and contraction. Wait, the Tutte polynomial for a tree T: since a tree with n vertices has n-1 edges. Let's start with a single edge (a tree with 2 vertices). For a single edge e, T(e; x, y) = x + y. Then, if we have a tree with more edges, we can use deletion and contraction. For example, take a tree T, pick an edge e. Then T(T; x, y) = T(T - e; x, y) + T(T / e; x, y). If T - e is a tree with one fewer edge, and T / e is a tree with one fewer vertex (since contracting an edge in a tree reduces the number of vertices by 1). 

Wait, let's consider a simple case. Let's say the tree is a path with 3 vertices (two edges). Let's call the edges e1 and e2, where e1 connects vertex 1-2, e2 connects 2-3. Let's pick edge e1. Then T(T; x, y) = T(T - e1; x, y) + T(T / e1; x, y). T - e1 is a tree with one edge (e2), so T(e2; x, y) = x + y. T / e1 is a tree with two vertices (since contracting e1 merges vertices 1 and 2 into one, and connects to vertex 3), so that's a single edge, so T(single edge; x, y) = x + y. So T(T; x, y) = (x + y) + (x + y) = 2x + 2y. Wait, but maybe the Tutte polynomial for a path with 3 vertices is actually x^2 + xy + y^2? Wait, maybe I made a mistake here. Wait, the Tutte polynomial for a tree with n vertices: for a tree, the Tutte polynomial is equal to the number of spanning trees, but also, for a tree, the Tutte polynomial can be expressed as the sum over k of (number of spanning trees with k edges) times x^(m - k) y^k, where m is the number of edges. For a tree with n vertices, m = n - 1. So the Tutte polynomial would be the sum from k = 0 to n - 1 of (number of spanning trees with k edges) x^(m - k) y^k. But for a tree, the only spanning tree is the tree itself, so the number of spanning trees is 1. Wait, no, wait: the Tutte polynomial counts spanning trees when evaluated at (1,1), but for a general tree, the Tutte polynomial is x^(n-1) + ...? Wait, maybe I need to recall the formula for the Tutte polynomial of a tree. Alternatively, since the Tutte polynomial for a graph is T(G) = Σ_{k} (number of spanning subgraphs with k edges and k - c(G) loops) x^{m - k} y^k, where c(G) is the number of connected components. For a tree, c(G) = 1, and there are no loops, so the number of spanning subgraphs with k edges and 0 loops is 1 if k = n - 1 (the tree itself), and 0 otherwise. Wait, no, that can't be right. Wait, a spanning subgraph of a tree with k edges must have k connected components, since each edge can't create a cycle. So for a tree with n vertices, a spanning subgraph with k edges has n - k connected components. So the Tutte polynomial would be Σ_{k=0}^{n-1} (1 if k = n - 1, else 0) x^{m - k} y^k} = x^{m - (n - 1)} y^{n - 1} }? Wait, m is the number of edges, which is n - 1 for a tree. So m - k = (n - 1) - k. If k = n - 1, then m - k = 0, so x^0 y^{n - 1} = y^{n - 1}. If k = n - 2, then m - k = 1, so x^1 y^{n - 2}, but there are no spanning subgraphs with n - 2 edges and 1 connected component (since a connected spanning subgraph with n - 2 edges would have n - (n - 2) = 2 connected components, which requires 2 - 1 = 1 edge, but n - 2 edges can't form two connected components; wait, maybe I'm getting confused here. Alternatively, let's use deletion and contraction for a tree. Let's take a tree T with m edges. Pick an edge e. Then T(T) = T(T - e) + T(T / e). T - e is a tree with m - 1 edges, so T(T - e) = T(m - 1 edges tree). T / e is a tree with m - 1 edges (since contracting an edge in a tree removes one vertex, so m - 1 edges). Wait, but if T is a tree with m edges, then T - e has m - 1 edges, and T / e has m - 1 edges as well? Wait, no: if you contract an edge in a tree, you remove one vertex, so the number of edges becomes m - 1. So T(T / e) = T(m - 1 edges tree). Therefore, T(T) = T(T - e) + T(T / e). If we apply this recursively, each time we delete or contract an edge, we reduce the number of edges by 1. For a tree with m edges, after m steps, we'll get down to a single edge. The Tutte polynomial of a single edge is x + y (since for an edge e, T(e) = x + y, as it has no loops, so the only terms are when the spanning subgraph has 0 edges (which is not spanning, wait, no: a spanning subgraph of an edge must include the edge, so the spanning subgraphs of an edge are the edge itself (1 edge, 0 loops) and the edge with a loop (but a tree has no loops, so the spanning subgraphs of an edge are just the edge itself, so T(e) = x + y? Wait, let's check the definition again. The Tutte polynomial T(G; x, y) is Σ_{A ⊆ E, A is acyclic} x^{m - |A|} y^{|A|}, where m is the number of edges. For an edge e, the acyclic subsets of E are the empty set and {e} (since a single edge is acyclic). So T(e; x, y) = x^{2 - 0} y^0 + x^{2 - 1} y^1 = x^2 + xy. Wait, that makes sense. So for a single edge, T(e) = x^2 + xy. Wait, earlier I thought it was x + y, but that was wrong. Let's recast: for a graph with m edges, the Tutte polynomial is the sum over all acyclic subgraphs A of x^{m - |A|} y^{|A|}. For an edge e (m=1), the acyclic subgraphs are those with 0 or 1 edge. If we take 0 edges, then x^{1 - 0} y^0 = x. If we take 1 edge, x^{1 - 1} y^1 = y. So total T(e) = x + y. Wait, now I'm confused. Let me check a reference. The Tutte polynomial for a single edge: since it's a connected graph with one edge, the number of spanning trees is 1, and the Tutte polynomial evaluated at (1,1) is the number of spanning trees, which is 1. But the Tutte polynomial is T(G) = Σ (number of acyclic orientations) x^k y^{l}, where k is the number of edges not oriented forward, l is the number of vertices of odd degree. For a single edge, the acyclic orientations are two: one where the edge is oriented one way, and one where it's oriented the other way. Wait, a single edge can be oriented in two ways, each orientation has one vertex of odd degree (the one with the tail), so the Tutte polynomial would be 2x^1 y^1? Wait, no, maybe I need to use the formula for the Tutte polynomial of a graph with n vertices and m edges: T(G) = Σ_{k=0}^m S(m, k) x^k y^{m - k}, where S(m, k) is the number of spanning subgraphs with k edges and no cycles. For a single edge (m=1), the possible k are 0 and 1. S(1, 0) = 0 (since a spanning subgraph must have at least n-1 edges for a tree, but a single edge is a tree with 2 vertices, so n=2, n-1=1, so S(1,1)=1, S(1,0)=0. Wait, that can't be. Wait, for a graph with n vertices, a spanning subgraph must have n-1 edges if it's a tree. So for a single edge (which is a tree with 2 vertices), the only spanning subgraphs are the edge itself (1 edge) and the empty set (0 edges), but the empty set isn't spanning. So S(1,1)=1, S(1,0)=0. Therefore, T(G) = 1*x^{1-1}y^1 + 0*x^{1-0}y^0 = 1*y + 0*x = y. But that doesn't match earlier thoughts. I think I need to step back. The Tutte polynomial for a tree: since a tree has no cycles, the Tutte polynomial should be T(T) = x^{m} + ...? Wait, let's take a path graph with 3 vertices (two edges). The Tutte polynomial should be T(P_3) = x^2 + xy + y^2? Wait, let's compute it via deletion and contraction. Let's take the path with vertices v1-v2-v3, edges e1 (v1-v2), e2 (v2-v3). Pick edge e1. Then T(P_3) = T(P_3 - e1) + T(P_3 / e1). P_3 - e1 is a single edge (v2-v3), so T(v2-v3) = x + y. P_3 / e1 is a graph with vertices v1 (merged with v2) and v3, connected by edge e2, so that's a single edge, T(v1'-v3) = x + y. So T(P_3) = (x + y) + (x + y) = 2x + 2y. But if we use the formula for a tree with 3 vertices (m=2 edges), the Tutte polynomial should be x^2 + xy + y^2? Wait, let's check with another method. For a tree with n vertices, the Tutte polynomial is equal to the sum over k=0 to n-1 of (n - k - 1 choose k - 1) x^k y^{n - 1 - k} }? Not sure. Alternatively, since the Tutte polynomial for a tree T is T(T) = x^{n-1} + ...? Wait, I'm getting confused here. Maybe the key is that for a tree, the Tutte polynomial can be computed as the sum of the Tutte polynomials of its subtrees after deletion and contraction, and since each step reduces the tree to smaller trees, eventually reaching single edges. If we start with a tree of m edges, then T(T) = Σ_{e} [T(T - e) + T(T / e)] = 2 * T(T - e) when considering all edges, but this might not be straightforward. Wait, let's try with a simple tree: a single edge (2 vertices, 1 edge). T(e) = x + y (as per earlier, when m=1, T(G) = x + y). If we have a tree with two edges (3 vertices), like a path of 3 vertices. As above, T(P_3) = 2x + 2y. Wait, but if we use the formula for a tree with n vertices, the Tutte polynomial is (n-1)(x + y) + ...? For n=3, (3-1)(x + y) = 2(x + y) = 2x + 2y, which matches. For n=2, (2-1)(x + y) = x + y, which matches. For a tree with 4 vertices (3 edges), T(T) should be 3(x + y) = 3x + 3y? Let's test with deletion and contraction. Take a tree with 4 vertices, say a path of 4 vertices (3 edges: e1, e2, e3). Pick edge e1. T(T) = T(T - e1) + T(T / e1). T - e1 has 2 edges, T(T - e1) = 2(x + y). T / e1 has 2 edges (since contracting e1 reduces the tree to 3 vertices, 2 edges), so T(T / e1) = 2(x + y). So T(T) = 2(x + y) + 2(x + y) = 4(x + y)? Wait, but 4 vertices would have n-1=3 edges, so if the formula is (n-1)(x + y), then for n=4, 3(x + y) = 3x + 3y, but my calculation gave 4(x + y). Hmm, I must be making a mistake here. Wait, let's try a different approach. The Tutte polynomial for a tree can be expressed as T(T) = Σ_{k=0}^{n-1} (number of spanning trees with k edges) x^{m - k} y^k. For a tree with n vertices, m = n - 1. The number of spanning trees is 1, so when k = m = n - 1, the term is x^0 y^{n - 1} = y^{n - 1}. For other k, the number of spanning subgraphs with k edges and no cycles is 0, since a spanning subgraph must have n - 1 edges to be connected (for a tree), so if k < n - 1, it's not connected, hence no spanning subgraphs. Therefore, T(T) = y^{n - 1}. Wait, for n=2, T(T) = y^1 = y, but earlier we had x + y for a single edge. Contradiction. I think I messed up the definition. The Tutte polynomial is Σ_{A ⊆ E, A is acyclic} x^{m - |A|} y^{|A|}. For a tree with m = n - 1 edges, the acyclic subsets are those with 0, 1, ..., m edges, but since the graph is a tree, any subset of edges with more than m - 1 edges would form a cycle, but a tree has no cycles, so subsets with more than m - 1 edges (which is n - 2 edges) would have to form a cycle, which they can't, so the acyclic subsets are those with 0, 1, ..., m - 1 edges. Wait, m = n - 1, so m - 1 = n - 2. So the acyclic subsets have 0 to n - 2 edges. The number of acyclic subsets with k edges is 1 if k = n - 1 (the whole tree, which is acyclic), wait, no, the whole tree has m = n - 1 edges, which is acyclic, so the subset with all edges is acyclic. Wait, I'm getting confused again. Let's take n=2, m=1. The acyclic subsets are: empty set (0 edges), and the single edge (1 edge). So T(G) = x^{1 - 0} y^0 + x^{1 - 1} y^1 = x + y. For n=3, m=2. Acyclic subsets: empty set (0 edges), one edge (1 edge), two edges (2 edges). But the two edges form a cycle in a tree? No, a tree with two edges (3 vertices) has no cycles, so two edges form a connected acyclic graph (a tree), so it's allowed. So T(G) = x^{2 - 0} y^0 + x^{2 - 1} y^1 + x^{2 - 2} y^2 = x^2 + xy + y^2. Wait, that makes sense. For n=3, m=2, T(G) = x^2 + xy + y^2. Then for n=4, m=3, T(G) = x^3 + x^2 y + x y^2 + y^3? Wait, let's check with deletion and contraction. For a tree with 3 vertices (m=2), T(G) = x^2 + xy + y^2. For a tree with 4 vertices (m=3), pick an edge, delete or contract it. Suppose we delete an edge, leaving a tree with 2 edges (3 vertices), T= x^2 + xy + y^2. Contract an edge, leaving a tree with 2 edges (3 vertices), T= x^2 + xy + y^2.