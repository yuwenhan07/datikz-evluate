<think>Got it, let's try to tackle this problem step by step. First, the question is about deformation contours in the complex plane for different dimensions and signs of the cosmological constant, with a specific rotation direction when the cosmological constant is positive. 

First, I need to recall what deformation contours represent. In complex analysis, especially in the context of black hole solutions or string theory, deformation contours might relate to the moduli space of solutions, where each contour corresponds to a different parameter (like dimension or cosmological constant). The statement says that for a positive cosmological constant (>0), the contour rotates counterclockwise by π/2 (which is 90 degrees) with each increment in dimension. 

Let me think about the general setup. Suppose we have a parameter that changes the dimension, say from d to d+1, and when the cosmological constant is positive, the contour in the complex plane (maybe the complex plane of some modulus, like the dilaton or a scalar field) rotates by 90 degrees counterclockwise. That suggests a rotational symmetry or a specific transformation rule under dimension changes and cosmological constant sign changes.

First, let's parse the statement: "For >0 the contour rotates counterclockwise by π/2 with each increment in dimension." So if we have a contour for dimension d with positive cosmological constant, then for dimension d+1, the contour is rotated by 90 degrees counterclockwise. That implies a rotational symmetry where each dimension increase corresponds to a rotation. 

Now, considering different signs of the cosmological constant. If the cosmological constant is negative, maybe the rotation direction changes, or the angle changes. But the question mentions "different dimensions and signs", so we need to consider both positive and negative cosmological constants and how the contours deform (i.e., rotate) with dimension changes.

Let's consider the case when the cosmological constant is zero first, maybe as a reference. If λ=0, perhaps the contour is a certain shape, and when λ becomes positive, it rotates by 90 degrees per dimension increase. For example, in 2 dimensions with λ>0, the contour might be a circle rotated by 90 degrees compared to λ=0, and in 3 dimensions, another 90 degrees, etc. 

Alternatively, if we think of the complex plane as having a coordinate z, and the contour is a curve in z, then changing dimension and λ sign would transform z in a specific way. The rotation by π/2 counterclockwise can be represented by multiplying z by i (since multiplying by i rotates by 90 degrees counterclockwise). So if the contour for dimension d and λ>0 is z = f(d, λ), then for d+1 and λ>0, it's z = i f(d, λ), and for λ<0, maybe a different transformation, like multiplying by -i or something else.

Wait, the problem says "for >0 the contour rotates counterclockwise by π/2 with each increment in dimension". So if we have a contour for dimension n, then for dimension n+1, it's rotated by 90 degrees counterclockwise. So the transformation would be a rotation operator R(n) = e^(iπ/2 * n) or something, but maybe more simply, each dimension increase adds a rotation of π/2. 

If we consider the first dimension, say d=1, λ>0, the contour is some shape, then d=2, λ>0, it's rotated by 90 degrees, d=3, another 90 degrees, etc. So for d dimensions, the rotation angle would be (d-1)*π/2? Wait, the problem says "with each increment in dimension", so each time you go from d to d+1, you rotate by π/2. So starting from d=1, the rotation angle is 0, d=2 is π/2, d=3 is π, d=4 is 3π/2, etc. But rotation by π is a 180-degree rotation, which is equivalent to multiplying by -1. 

Now, considering the sign of the cosmological constant. If λ is negative, maybe the rotation direction reverses, so instead of counterclockwise, it's clockwise, which would be equivalent to multiplying by -i instead of i. So for λ<0, the rotation would be clockwise by π/2 per dimension increment, which is the same as counterclockwise by -π/2, or multiplying by e^(-iπ/2) = -i. 

Putting this together, the deformation contour in the complex plane for a given dimension d and cosmological constant sign would be related to the previous one by a rotation of π/2 (counterclockwise for λ>0, clockwise for λ<0) per dimension increase. 

To visualize, if you start with a contour in 1 dimension (λ>0), it might be a circle, then in 2 dimensions, it's a circle rotated by 90 degrees, in 3 dimensions, another 90 degrees (total 180), etc. For λ<0, each dimension increase would rotate the contour clockwise by 90 degrees. 

Another angle: in string theory, the moduli space of black holes often involves complex parameters, and the deformation under different dimensions and cosmological constants could relate to the symmetries of the moduli space. The rotation by π/2 suggests a rotational symmetry of order 4, since four increments would bring you back to the original orientation (4*(π/2)=2π, a full rotation). 

Also, considering the cosmological constant's sign: positive cosmological constant might correspond to a "de Sitter" type space, while negative is "anti-de Sitter", and their moduli spaces might have different rotational properties. The counterclockwise rotation for positive λ and clockwise for negative λ could reflect a symmetry between the two cases under a rotation of π. 

In summary, the key points are: 
1. For λ > 0, each dimension increment (d → d+1) rotates the contour in the complex plane counterclockwise by π/2.
2. For λ < 0, the rotation direction reverses (clockwise by π/2 per dimension increment), or equivalently, the rotation angle is -π/2 per increment.
3. This implies a systematic rotational symmetry in the moduli space as dimensions and cosmological constant signs change, with the rotation angle depending on the sign of λ and the dimension increment.

I need to make sure I'm not missing any specific details, but the question is more about describing the relationship rather than calculating a specific contour. The main takeaway is the rotational behavior with dimension and cosmological constant sign.</think><answer>For positive cosmological constant (>0), each increment in dimension rotates the deformation contour in the complex plane counterclockwise by \( \frac{\pi}{2} \) (90 degrees). This rotational symmetry implies that for a given dimension \( d \) with \( \lambda > 0 \), the contour for \( d+1 \) dimensions is obtained by rotating the \( d \)-dimensional contour by \( \frac{\pi}{2} \). For negative cosmological constant (\( \lambda < 0 \)), the rotation direction reverses to clockwise by \( \frac{\pi}{2} \) per dimension increment, maintaining the same angle magnitude but opposite orientation. This systematic rotation reflects a symmetry in the moduli space of solutions, where dimension and cosmological constant sign changes correspond to rotational transformations in the complex plane.  

The answer is \(<|begin_of_box|>\frac{\pi}{2}<|end_of_box|>\).</answer>