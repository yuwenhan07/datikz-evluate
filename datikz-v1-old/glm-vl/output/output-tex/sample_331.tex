<think>Got it, let's try to figure out what the question is asking. The user mentioned a non-bipartite oriented graph with 3-symmetric spectrum, provided by Mohar M. They also described an image of a network with nodes (circles) connected by lines. But the question might be about identifying the graph or its properties. Wait, the user's message is a bit descriptive, maybe the question is to recognize the graph or explain its properties. But since the user provided a description of the graph's image, maybe the task is to describe the graph's structure or confirm its properties. However, without the actual image, we have to rely on the description. The key points are non-bipartite (so not 2-colorable), oriented (directed edges), 3-symmetric (spectrum has 3-fold symmetry, meaning eigenvalues come in triples with the same sum, etc.). Mohar M is known for work on oriented graphs and spectra, so maybe the graph is a specific example. But since the user is asking about the graph, perhaps the answer is to describe its properties or identify it. But given the information, maybe the graph is a specific example like a certain oriented graph with 3-symmetric spectrum. Alternatively, the question might be to explain what a 3-symmetric spectrum is in the context of oriented graphs. But the user's question seems to be a description, so maybe the answer is to state the graph's properties: non-bipartite, oriented, 3-symmetric spectrum, with the image showing nodes connected by directed edges, nodes as circles, etc. But since the user might want the identification of the graph, but without more details, it's hard. Wait, the initial question might be a prompt to describe the graph, but the user's message is a bit of a description. Alternatively, maybe the question is to recognize that the graph is a specific example, but given the information, the best approach is to explain the properties: non-bipartite (so has an odd cycle), oriented (directed edges), 3-symmetric spectrum (eigenvalues have a certain symmetry, like λ, -λ, 0 or similar, with three-fold symmetry). But since the user provided a description of the image, maybe the answer is to confirm that the graph is a non-bipartite oriented graph with 3-symmetric spectrum as described. Alternatively, if the question is to identify the graph, but without more details, it's challenging. However, given the context, the key points are the graph's properties: non-bipartite, oriented, 3-symmetric spectrum. So the answer would summarize those properties.</think><answer>To address the question about the non - bipartite oriented graph with a 3 - symmetric spectrum provided by Mohar M, we analyze its key properties and structure:  

1. **Non - bipartite nature**: The graph cannot be divided into two disjoint sets of nodes such that all edges connect nodes from different sets. This implies the presence of odd - length cycles (a fundamental property of non - bipartite graphs).  
2. **Oriented graph**: Each edge has a direction, meaning connections between nodes are directional rather than undirected.  
3. **3 - symmetric spectrum**: The graph’s adjacency matrix (or Laplacian, depending on the context) has eigenvalues that exhibit 3 - fold symmetry. This means eigenvalues often appear in triples (e.g., \( \lambda, -\lambda, 0 \) or sets with equal sums and symmetric patterns), a property that distinguishes the graph’s spectral structure.  

The image depicts the graph with nodes as circles (of varying sizes, possibly indicating node importance or degree) connected by directed edges (crisp lines) on a white background. The arrangement of nodes and edges illustrates the intricate directional connections, which are essential for analyzing the graph’s topological and spectral properties.  

Since the question focuses on identifying the graph’s characteristics rather than a specific graph name (without additional unique identifiers in the description), the core takeaway is its classification as a non - bipartite oriented graph with a 3 - symmetric spectrum, as described.  

The answer is \(<|begin_of_box|>\text{A non - bipartite oriented graph with a 3 - symmetric spectrum, as described}<|end_of_box|>\).</answer>