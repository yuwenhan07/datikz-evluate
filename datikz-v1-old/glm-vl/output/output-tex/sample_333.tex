<think>Got, let's try to figure out what's being asked here. The user mentioned "Using the pivot" and described an image with a whiteboard, equations, symbols, tubes connected, a square object between them, surrounded by lines and symbols. But wait, maybe this is a physics problem involving a pivot (like a lever) with tubes and a square? Hmm, but without the actual image details, maybe it's a standard problem. Wait, the user might be referring to a problem where a pivot is involved, maybe a seesaw with objects, but the description mentions tubes and a square. Alternatively, maybe it's about torque or equilibrium. But since the image details are vague, perhaps the question is about setting up a pivot system. Wait, the user says "Using the pivot" so maybe calculating something like torque balance. But without specific numbers, it's hard. Wait, maybe the image is a standard one, like a lever with two tubes (maybe cylinders) and a square block as a weight, with distances from the pivot. But since the user provided a description, maybe the key is to identify the pivot point and apply torque equilibrium. Let's assume that the pivot is a point, and the tubes and square create torques. But without specific values, maybe the question is about setting up the equation. Alternatively, if it's a problem where the pivot is at a certain point, and we need to find the force or distance. But since the user's description is a bit vague, maybe the intended answer is about torque balance: sum of torques about the pivot equals zero. So, if the square has mass m, the tubes have masses M1, M2, distances from pivot d1, d2, then M1*g*d1 = M2*g*d2 + m*g*d3, but without numbers, maybe the answer is a general formula. Wait, but the user might have intended a specific problem, but since the description is not clear, maybe I need to make an assumption. Alternatively, if the image is a simple lever with a pivot, two arms with tubes and a square, the key is to use the principle of moments. So, the torque due to the square equals the torque due to the tubes. But without specific data, it's hard. Wait, maybe the question is about identifying the pivot and the forces, but since the user says "using the pivot" to solve, maybe the answer is a step-by-step using torque. But given the lack of specific info, perhaps the intended answer is a general approach, but the user might have had a specific problem in mind. Alternatively, if the image shows a pivot with two tubes (each with mass, say, m1 and m2) and a square (mass m) at a distance, then the equation would be m1*g*r1 = m2*g*r2 + m*g*r3. But since the user's description is vague, maybe the answer is about setting up the torque equation. Alternatively, if the problem is about equilibrium, the sum of forces and torques must be zero. Given that, the final answer might involve setting the torques equal, but without numbers, I can't compute a numerical answer. Wait, but the user might have made a mistake in the description, and maybe the image is a standard one, like a seesaw with a pivot in the center, two tubes (each with mass, say, 10 kg and 15 kg) at distances 2 m and 3 m, and a square of 20 kg at 1.5 m, then torque balance: 10*2 + 20*1.5 = 15*3? Wait, 20*1.5 is 30, 10*2 is 20, total 50; 15*3 is 45, which doesn't balance. But without knowing the exact values, it's impossible. Alternatively, maybe the answer is a conceptual one, stating that the pivot is the fulcrum, and torques must balance. But the user says "put the final result in a box," so maybe a formula. Wait, given the lack of specific info, perhaps the intended answer is the torque balance equation: Στ = 0, so (force1 * distance1) - (force2 * distance2) = 0, but I need to make an educated guess. Alternatively, if the square is the object between the tubes, and the tubes are connected, maybe the system is in equilibrium, so the torques about the pivot cancel. But without more info, I'll have to assume a standard problem. Wait, another angle: the image has a pivot, so maybe it's a lever with a pivot point, and the tubes and square are on either side. If the square is at a distance d from the pivot, and the tubes are at distances d1 and d2, then the torque from the square is m_square * g * d, and the torque from the tubes is (m1 + m2) * g * d1 (if both tubes are on one side), but this is all guesswork. Given the lack of specific data, I might need to state that more information is needed, but since the user expects an answer, maybe the key is to recognize that using the pivot (fulcrum) requires torque balance, so the final answer would be an equation like m1*g*r1 = m2*g*r2 + m3*g*r3, boxed. But without numbers, I can't put a numerical value. Wait, maybe the image is a simple one where the pivot is in the center, tubes are on both sides with equal mass, so the square balances them. But I'm stuck. Given the user's description, maybe the answer is about identifying the pivot and applying the principle of moments, so the final result is the torque balance equation, boxed as \boxed{\tau_1 = \tau_2} or something, but I need to make a choice. Alternatively, if it's a problem where the pivot is at a point, and the forces are balanced, the answer could be a specific value, but without info, I can't. Given the ambiguity, I'll proceed with the torque balance principle and present the equation in a box.</think><answer>The problem involves using the pivot (fulcrum) principle, where the sum of torques about the pivot must equal zero for equilibrium. Assuming the system involves masses \( m_1, m_2, m_3 \) at distances \( r_1, r_2, r_3 \) from the pivot, the torque balance equation is \( m_1 g r_1 = m_2 g r_2 + m_3 g r_3 \). Without specific values, the general principle is applied, leading to the conclusion that torques must balance.  

The answer is \(<|begin_of_box|>\tau_1 = \tau_2 (or \sum \tau = 0)<|end_of_box|>\).</answer>