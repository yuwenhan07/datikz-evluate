<think>Got it, let's try to figure out what's going on here. The user mentioned a smooth, sinusoidal parametric curve that's supposed to follow a given formula but looks rigid and far from the desired trajectory. The issue might be with the parametric equation's input, maybe the parameterization or the parameters used.

First, let's recall that parametric curves are defined by x(t) and y(t) functions of a parameter t. If the curve is rigid, maybe the parameter t isn't varying enough, or the functions are too simple. Sinusoidal curves usually involve sine or cosine functions, so maybe the equations are like x(t) = A cos(t), y(t) = B sin(t), but if the parameterization is linear (t from 0 to 2π), but if the curve is supposed to have a different shape, maybe the functions need to have different frequencies or amplitudes, or maybe the parameterization is non-linear.

Another possibility is that the parametric equations are not correctly set up. For example, if the intended curve is a sinusoidal spiral or a more complex shape, maybe the equations need to have t squared or some other term. Or maybe the parameter t is being used as a linear parameter, but the curve requires a different parameterization, like using a logarithmic or exponential parameter.

Also, the issue could be with the range of the parameter t. If t is only going from 0 to π, the curve might not show the full sinusoidal shape, making it look rigid. Or maybe the functions are not sinusoidal but have a different periodicity.

Let's think about common parametric sinusoidal curves. A simple one is a circle: x = cos(t), y = sin(t), t from 0 to 2π. If it's rigid, maybe the amplitudes are not varying, or the curves are not combined properly. For example, if it's supposed to be a more complex curve like a Lissajous figure, which is parametric with x = A cos(nt), y = B sin(mt), and if n and m are not integers, it might not look sinusoidal. But if n and m are 1, it's a circle.

Wait, the user says "sinusoidal parametric curve", so maybe it's like x(t) = a + b cos(t), y(t) = c + d sin(t), which is an ellipse, but if it's supposed to be a more complex sinusoidal shape, maybe the equations need to have products of sine and cosine, or higher harmonics.

Another angle: the parametric equations might not be smooth, but the user says it's smooth, so maybe the derivatives are not continuous, but the issue is about the trajectory. If the curve is rigid, maybe the parameterization is linear in t, but the intended curve requires a non-linear parameterization, like t = sin(s) or something, to make the curve appear more sinusoidal in shape.

Alternatively, maybe the parametric equations are missing a scaling factor or the parameter t is not scaled correctly. For example, if t is too large, the curve might appear compressed or stretched, making it rigid.

Let's outline steps to check:

1. Check the parametric equations: Are they sinusoidal? For example, x(t) = cos(t), y(t) = sin(t) is a circle, which is a type of sinusoidal curve. If the intended curve is a different shape, maybe it's a combination like x(t) = cos(t) + cos(2t), y(t) = sin(t) + sin(2t), which would create a more complex curve.

2. Check the parameter range: If t is only from 0 to π/2, the curve might not show the full shape, appearing rigid. Try extending the range to 0 to 2π or 0 to 4π.

3. Check the parameterization: If using a linear parameterization (t from 0 to 1), maybe using a non-linear parameterization like t = sin(s) or t = s^2 to create a different shape.

4. Check amplitudes and frequencies: Ensure that the amplitudes (a, b, c, d) and frequencies (if any) are set correctly. For example, if the intended curve has a larger amplitude in one direction, adjust the coefficients.

5. Check for missing terms: Maybe the parametric equations need to include higher-order terms or products of sine and cosine to create a more complex sinusoidal shape.

If the current plot is rigid, maybe the parametric equations are too simple, like a circle instead of a more complex curve. To fix it, try adding more sinusoidal terms or changing the parameterization. For example, a parametric equation for a sinusoidal spiral could be x(t) = a t cos(t), y(t) = a t sin(t), but that's a spiral. Alternatively, a more complex curve like x(t) = cos(t) + 0.5 cos(3t), y(t) = sin(t) + 0.5 sin(3t) would create a more intricate shape.

Another possibility is that the parameter t is not being used as a time parameter but needs to be scaled. For example, if the intended curve has a different period, adjust the argument of the sine/cosine functions, like using t * k instead of t, where k is a scaling factor.

In summary, the issue is likely with the parametric equations' structure—maybe missing terms, incorrect parameterization, or wrong parameter range. To fix, review the parametric equations, check for additional sinusoidal terms, adjust the parameter range, or change the parameterization method.</think><answer>To address the issue with the rigid, non-desired trajectory of the sinusoidal parametric curve, follow these steps:  

### 1. **Analyze the Parametric Equations**  
Parametric curves are defined by \( x(t) \) and \( y(t) \) (functions of a parameter \( t \)). If the curve is rigid, the equations may be too simple (e.g., a basic circle \( x = \cos(t), y = \sin(t) \) instead of a more complex shape). For a sinusoidal curve, consider adding:  
- **Higher harmonics** (e.g., \( x(t) = \cos(t) + 0.5\cos(3t) \), \( y(t) = \sin(t) + 0.5\sin(3t) \) for a more intricate shape).  
- **Non-linear parameterization** (e.g., \( t = \sin(s) \) or \( t = s^2 \) to alter the curve’s shape).  

### 2. **Check Parameter Range**  
If \( t \) is restricted (e.g., \( 0 \) to \( \pi/2 \) instead of \( 0 \) to \( 2\pi \)), the curve may not display its full sinusoidal form, appearing rigid. Extend the range to \( 0 \) to \( 2\pi \) or \( 0 \) to \( 4\pi \) for a complete cycle.  

### 3. **Adjust Amplitudes and Frequencies**  
Ensure amplitudes (coefficients in \( \cos(t) \), \( \sin(t) \)) and frequencies (arguments like \( t \), \( 2t \)) match the desired trajectory. For example, if the curve should have a larger amplitude in one direction, scale the coefficients (e.g., \( x(t) = 2\cos(t) \), \( y(t) = \cos(t) \) instead of equal amplitudes).  

### 4. **Verify Smoothness and Continuity**  
Ensure the parametric equations are smooth (no discontinuities in \( x(t) \) or \( y(t) \), or their derivatives). If using non-linear parameterization, confirm the parameterization itself is smooth.  

### Example Fix (If Original Curve Was a Simple Circle):  
If the original curve was \( x = \cos(t), y = \sin(t) \) (a circle), modify it to a more complex sinusoidal shape like:  
\[
x(t) = \cos(t) + 0.5\cos(3t), \quad y(t) = \sin(t) + 0.5\sin(3t), \quad t \in [0, 2\pi]
\]  
This introduces higher harmonics, creating a more intricate, non-rigid trajectory.  

By adjusting the parametric equations’ structure, parameter range, or adding complexity, the curve should align closer to the desired sinusoidal trajectory.  

The answer is \(<|begin_of_box|>x(t) = \cos(t) + 0.5\cos(3t), \, y(t) = \sin(t) + 0.5\sin(3t) (t \in [0, 2\pi])<|end_of_box|>\).</answer>