<think>Got it, let's try to figure out how to approach this. The user is asking about the publication year distribution of 386 corpus papers, categorized by origin (preliminary manual search or digital library), with multiple counts for papers found multiple times. 

First, I need to understand what kind of data this is. It's a distribution of publication years, so probably a frequency distribution or a histogram showing how many papers were published in each year, broken down by the two origins. But since the user mentions "preliminary manual search" and "digital library" as origins, we need to split the 386 papers into those two groups, then look at their publication years.

But wait, the user might need a summary of the distribution. However, without the actual data (since this is a hypothetical or example scenario), I need to explain how such a distribution would be presented. Maybe a table or a graph showing for each year, how many papers from manual search vs digital library were published. For example, if in 2000, 10 papers from manual search and 20 from digital library, then total 30, but since they're counted multiple times, maybe each instance is a separate count. Wait, the note says "Papers found multiple times are counted multiple times," so if a paper was found twice (once in manual, once in digital), it's counted twice. So the total count is 386, which is the sum of papers from each origin, counting duplicates.

To present the publication year distribution, you'd typically have a list of years, then for each year, the number of papers from manual search, digital library, and total. For example:

Year | Manual Search | Digital Library | Total
--- | --- | --- | ---
1990 | 5 | 10 | 15
1995 | 8 | 12 | 20
2000 | 10 | 20 | 30
... | ... | ... | ...

Then, the distribution would show how the 386 papers are spread across years, with each cell being the count of papers from that origin published in that year. Since the user says "preliminary manual search or digital library," maybe some papers are in both, hence counted twice.

But since the user might want a description or a way to structure the data, the key points are:

1. Categorize the 386 papers into two groups: those from preliminary manual search, those from digital library (with duplicates counted).
2. For each group, count how many papers were published in each year.
3. Present this as a distribution (e.g., table, histogram) showing the number of papers per year for each origin.

Without the actual data, I can't provide the exact numbers, but I can explain the structure. Alternatively, if this is a hypothetical example, maybe a sample distribution:

Suppose the papers are distributed as follows (example):

- Manual search papers: 200 papers, distributed with 50 in 2010, 60 in 2015, 40 in 2020, etc.
- Digital library papers: 186 papers, distributed with 70 in 2010, 80 in 2015, 36 in 2020, etc.
- Total: 200 + 186 = 386 (with overlaps counted multiple times).

Then, the distribution would show that more papers from digital library are in recent years, while manual search has a more even spread, or something like that. But since the user wants the actual distribution, maybe they need to provide the data, but since this is a question about how to present it, the answer would explain the structure.

Wait, the user might be asking for a way to report this, so the final answer would be a description of the distribution, maybe a table, explaining how the 386 papers are split by origin and year, with multiple counts for duplicates.

Alternatively, if this is a statistical question, maybe the distribution is skewed towards recent years, with more papers from digital library in recent years, and manual search having a longer tail or different distribution.

But since the user says "according to origin (preliminary manual search or digital library)", the distribution is split by origin, so for each year, show how many from each origin.

In summary, the answer should outline the structure of the distribution, showing publication years, counts per origin, and total (with duplicates counted), explaining how the 386 papers are distributed across years, split by the two origins.</think><answer>To address the question about the publication year distribution of the 386 corpus papers, broken down by origin (preliminary manual search or digital library, with multiple counts for duplicate papers), here’s a structured approach to presenting the data:  


### 1. Data Structure Overview  
The distribution should categorize papers into two groups (manual search, digital library) and count them by **publication year** (with duplicates counted multiple times). A typical presentation would use a table or histogram showing:  
- **Publication Year**: The year a paper was published.  
- **Manual Search Papers**: Number of papers from manual search published in that year.  
- **Digital Library Papers**: Number of papers from digital library published in that year.  
- **Total Papers (for that year)**: Sum of manual + digital library papers (reflecting duplicates).  


### 2. Example Format (Table)  
A sample table might look like this (hypothetical data, as actual numbers depend on the corpus):  

| Publication Year | Manual Search Papers | Digital Library Papers | Total Papers (with duplicates) |
|------------------|----------------------|------------------------|---------------------------------|
| 1990             | 5                    | 10                     | 15                              |
| 1995             | 8                    | 12                     | 20                              |
| 2000             | 10                   | 20                     | 30                              |
| 2005             | 12                   | 25                     | 37                              |
| 2010             | 15                   | 30                     | 45                              |
| 2015             | 18                   | 35                     | 53                              |
| 2020             | 20                   | 40                     | 60                              |
| 2025             | 22                   | 45                     | 67                              |
| **Total**        | **200**              | **186**                | **386** (with overlaps counted multiple times) |  

*Note*: The total rows sum to 386, accounting for papers found in both origins (hence “counted multiple times”).  


### 3. Key Observations (Hypothetical, for Illustration)  
- **Origin Bias**: If digital library papers are more recent, the distribution might show a skew toward later years (e.g., 2020+ having more digital library papers).  
- **Manual Search Pattern**: Manual search papers might have a more evenly distributed timeline (e.g., older papers from earlier decades).  
- **Overlap Impact**: Papers found in both origins would appear in both rows for their publication year, increasing the total count for that year.  


### 4. Practical Steps to Gather Data  
To create this distribution:  
1. **Categorize Papers**: Label each of the 386 papers as “manual search” or “digital library” (or both, if found in both sources).  
2. **Extract Publication Years**: Record the publication year for each paper.  
3. **Count by Year and Origin**: For each year, count how many papers from each origin were published, then sum for total (with duplicates).  
4. **Visualize/Report**: Use a table, histogram, or bar chart to display the distribution clearly.  


If you have the actual data (paper origins and publication years), you can plug those numbers into the table format above to get the precise distribution. The key is to separate the 386 papers by origin and then analyze their publication year distribution, accounting for multiple counts for duplicates.</answer>