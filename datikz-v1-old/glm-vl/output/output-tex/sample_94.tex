<think>Got it, let's try to figure out the problem. The question is about a PDF f_b defined in the region [v_0 - 2, v_0 + 2] for a bit b_k which can be 0 (left) or 1 (right). I need to recall what kind of PDF this might be. Maybe a uniform distribution or a Gaussian? Wait, the region is an interval of length 4 (from v0-2 to v0+2 is 4 units), so if it's uniform, the PDF would be constant over that interval. Let's think about typical binary signaling where bits are represented by voltages. For b_k=0, maybe the voltage is around v0-2, and for b_k=1, around v0+2. The PDF f_b would be the probability density function for the bit's voltage, so maybe a uniform distribution over the interval for each bit. 

Wait, the problem says "for bit b_k=0 (left) and b_k=1 (right)". So maybe when b_k=0, the PDF is defined on [v0-2, v0+2] but centered left? Wait, no, the region is [v0-2, v0+2] regardless of the bit, but the bit determines which part? Wait, maybe the PDF is uniform over that interval for each bit. So if it's uniform, the PDF f_b(x) would be 1/( (v0+2) - (v0-2) ) = 1/4 for x in [v0-2, v0+2], and 0 otherwise. But wait, the question says "for bit b_k=0 (left) and b_k=1 (right)". Maybe the center is different? For b_k=0, the interval is [v0-2, v0+2], but maybe the PDF is centered at v0-2? No, the interval is given as [v0-2, v0+2], so the length is 4. If it's uniform, the PDF is 1/4 over that interval. Alternatively, if it's a Gaussian, but the problem doesn't mention variance, so maybe uniform is more likely. 

Wait, let's think about binary signaling with equiprobable bits. If the bits are 0 and 1, each with probability 0.5, then the PDF for the voltage when b_k=0 might be uniform over [v0-2, v0+2] and for b_k=1, maybe uniform over [v0+2, v0+4]? But the problem states the region is [v0-2, v0+2] for both, but left and right. Hmm, maybe the PDF is defined as a uniform distribution over the interval [v0-2, v0+2] when the bit is 0, and another uniform distribution over the same interval but shifted? Wait, the question says "for bit b_k=0 (left) and b_k=1 (right)". So maybe when b_k=0, the PDF is uniform on [v0-2, v0+2], and when b_k=1, it's uniform on [v0+2, v0+4]? But the problem states the region is [v0-2, v0+2] for both. Wait, I need to parse the question again: "The PDF f_b in region [v0-2, v0+2] for bit b_k=0 (left) and b_k=1 (right)." So the region is the same, but the bit determines which part? Maybe the PDF is defined as a uniform distribution over [v0-2, v0+2] when b_k=0, and another uniform distribution over the same interval but centered differently? Wait, no, the interval is fixed. Alternatively, maybe the PDF is a rectangle function (uniform) over [v0-2, v0+2] for b_k=0, and another rectangle over [v0+2, v0+4] for b_k=1, but the question says the region is [v0-2, v0+2] for both. Hmm, I might be overcomplicating. Let's assume that for each bit, the PDF is uniform over the interval [v0-2, v0+2], so the PDF f_b(x) = 1/(4) when x is in [v0-2, v0+2], and 0 otherwise. But the question says "for bit b_k=0 (left) and b_k=1 (right)", maybe the center is different. For b_k=0, the interval is [v0-2, v0+2], so the center is v0, and for b_k=1, maybe the interval is [v0+2, v0+4], but the problem states the region is [v0-2, v0+2] for both. Wait, the problem says "in region [v0-2, v0+2] for bit b_k=0 (left) and b_k=1 (right)". Maybe the PDF is defined as a uniform distribution over [v0-2, v0+2] when b_k=0, and a uniform distribution over [v0+2, v0+4] when b_k=1, but the question specifies the region as [v0-2, v0+2] for both. Alternatively, maybe the PDF is a Gaussian centered at v0-2 for b_k=0 and v0+2 for b_k=1, but without more info, it's hard. Wait, the key is that the region is [v0-2, v0+2], which is an interval of length 4, centered at v0. If the PDF is uniform, then f_b(x) = 1/4 for x in that interval. But the question says "for bit b_k=0 (left) and b_k=1 (right)", maybe the PDF is defined as a rectangle function with different centers? Wait, if b_k=0 is left, maybe the PDF is uniform from v0-2 to v0, and for b_k=1, from v0 to v0+2? But the problem says the region is [v0-2, v0+2] for both. I think I need to make an assumption here. Given that it's a PDF over an interval of length 4, uniform distribution is a common choice, so f_b(x) = 1/4 when x is in [v0-2, v0+2], else 0. But the mention of left and right might imply that for b_k=0, the PDF is on the left part of the interval, and for b_k=1 on the right, but the interval is combined. Alternatively, maybe the PDF is a triangular distribution or something else, but without more info, uniform is a safe bet. Wait, another approach: in digital communication, binary signals often use non-return-to-zero (NRZ) signaling where a bit 0 is represented by a voltage level and 1 by another. If the interval is [v0-2, v0+2], maybe the 0 is at v0-2 and 1 at v0+2, so the PDF for each bit is uniform over their respective intervals. But the problem states the region is [v0-2, v0+2] for both, so maybe the PDF is defined as a uniform distribution over the entire interval, but the bit determines the center? Wait, I'm getting confused. Let's try to write the PDF as a uniform distribution over [v0-2, v0+2], so f_b(x) = 1/( (v0+2) - (v0-2) ) = 1/4 for x in that interval, else 0. That makes sense. So the final answer would be the PDF is uniform over that interval with f_b(x) = 1/4 when x is in [v0-2, v0+2], else 0.</think><answer>To determine the PDF \( f_b \) for the given region, we assume a uniform distribution over the interval \([v_0 - 2, v_0 + 2]\), which has a length of 4. For a uniform PDF, the probability density is constant over the interval and zero elsewhere.  

For bit \( b_k = 0 \) (left) and \( b_k = 1 \) (right), the PDF is defined as:  
\[
f_b(x) = \begin{cases} 
\frac{1}{4} & \text{if } v_0 - 2 \leq x \leq v_0 + 2, \\
0 & \text{otherwise}.
\end{cases}
\]  

This uniform distribution is centered at \( v_0 \) with a total interval length of 4, resulting in a constant density of \( \frac{1}{4} \) over the specified region.  

The answer is \(<|begin_of_box|>f_b(x) = \frac{1}{4} \text{ for } v_0 - 2 \leq x \leq v_0 + 2, \text{ else } 0<|end_of_box|>\).</answer>