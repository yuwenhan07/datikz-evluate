python
import matplotlib.pyplot as plt
import numpy as np

def deg_to_rad(deg):
    """Convert degrees to radians."""
    return deg * np.pi / 180

def arc_length(r, theta_deg):
    """Calculate arc length given radius (r) and angle (theta_deg, in degrees)."""
    theta_rad = deg_to_rad(theta_deg)
    return r * theta_rad

def visualize_arc(r, theta_deg):
    """Visualize the arc length for a circle with radius r and angle theta_deg (in degrees)."""
    s = arc_length(r, theta_deg)
    theta_rad = deg_to_rad(theta_deg)
    
    # Create figure and axis with equal aspect ratio (so circle looks circular)
    fig, ax = plt.subplots(figsize=(8, 8))
    ax.set_aspect('equal')
    
    # Draw the circle
    circle = plt.Circle((0, 0), r, fill=False, color='black')
    ax.add_patch(circle)
    
    # Draw the arc (from 0° to theta_deg, starting from the x-axis)
    arc = plt.Arc((0, 0), 2*r, 2*r, angle=0, theta1=0, theta2=theta_deg, color='red')
    ax.add_patch(arc)
    
    # Draw radius lines (from center to start/end of arc)
    ax.plot([0, r * np.cos(0)], [0, r * np.sin(0)], 'k--')  # Start angle (0°)
    ax.plot([0, r * np.cos(theta_rad)], [0, r * np.sin(theta_rad)], 'k--')  # End angle (theta_deg)
    
    # Label the arc length
    ax.text(r/2, -r/2, f's = {s:.2f}', ha='center')
    
    # Set plot limits to ensure the circle is fully visible
    ax.set_xlim(-r*1.2, r*1.2)
    ax.set_ylim(-r*1.2, r*1.2)
    
    # Add title and grid
    ax.set_title(f'Arc Length for r={r}, θ={theta_deg}° = {s:.2f}')
    ax.grid(True)
    
    plt.show()

# Example usage (works for any angle in degrees)
visualize_arc(5, 120)  # Original example (120°)
visualize_arc(5, 90)   # 90° (quarter-circle)
visualize_arc(5, 60)   # 60° (example of a smaller angle)