<think>Got it, let's try to understand the request. The user is asking about the visual impact of blocking a mixed JS script, comparing a normal website (left) vs. a broken one (right). The image description is a simple white square on white background with a red frame. 

First, I need to explain what happens when a script is blocked. If a script is blocked, elements dependent on that script might not load, leading to visual issues like missing content, broken layouts, or placeholders. The left side (normal) would have the script working, so the square with red frame is properly displayed. The right side (blocked) would show the square but maybe the red frame is broken, or the square is not there, or the layout is messed up. 

So, the visual impact is that the right side (blocked) has a broken or incomplete display compared to the left (normal). The image shows a simple element (square with red frame) that works normally when the script is allowed, but fails when blocked. 

I need to describe the visual difference: left is a complete, properly rendered white square with red frame; right is either the square missing, the frame broken, or the entire element not displaying, showing the breakage. 

Let me structure the explanation: start by explaining what blocking a script does, then describe the visual elements of each side.</think><answer>To understand the visual impact of blocking a mixed JavaScript (JS) script, we analyze the comparison between the “normal” (left) and “broken” (right) states:  

### 1. Normal Website (Left Side)  
When the JS script is **not blocked**, the webpage renders as intended. In the described image, this means:  
- A simple visual element (a white square with a red frame) is fully displayed.  
- The layout, styling, and functionality of the element (e.g., the red frame around the square) work correctly, with no visual glitches or missing components.    

### 2. Broken Website (Right Side)  
When the JS script is **blocked**, the webpage’s functionality (including the script’s role in rendering elements) is disrupted. In the described image, this means:  
- The visual element (white square with red frame) is partially or fully broken. For example, the red frame might be distorted, the square might not appear, or the entire element might be replaced by a placeholder (like the simple white square on a white background without the red frame, or with a broken frame).  
- The layout or styling of the element is compromised, showing the direct impact of the script being blocked (e.g., the red frame might be missing, the square might be misaligned, or the element is not rendered at all).    

### Key Visual Impact Summary  
The blocking of the JS script causes a **disruption in rendering**—the right side shows a “broken” version of the element (e.g., missing frame, misrendered square) compared to the left side’s fully functional version. This illustrates how critical scripts can affect visual consistency and user experience when blocked.</answer>