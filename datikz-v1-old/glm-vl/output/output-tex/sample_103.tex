<think>Got, let's try to tackle this problem. First, the user is asking about the solution of discrete problems: p-median, p-envy, and p-intra-envy problems, with an image description of dots and dashes. Wait, maybe the image is a diagram representing a network or a graph, which is common in location problems. Let's start by recalling each problem:

1. **p-median problem**: This is a location problem where we need to open p facilities (like stores, hospitals) to minimize the total weighted distance from all demand points to the nearest facility. The goal is to find the optimal locations for p facilities such that the sum of distances (or costs) from each demand point to its closest facility is minimized.

2. **p-envy problem**: This might refer to a problem where we have p individuals (or facilities) and we need to distribute something (like resources) such that envy is minimized. Envy here could mean that no individual (or facility) envies another, i.e., each has at least as much as the others. But maybe in the context of location, it's about facility locations where the envy (difference in costs) is minimized. Alternatively, it could be a problem where we have p facilities and we need to assign demand points such that the maximum envy (difference in costs for any two demand points) is minimized.

3. **p-intra-envy problem**: This might be a more specific version, like within a group of p facilities, the envy among them is minimized, or maybe the envy within each facility's assignment. Need to clarify, but likely a variant where we consider the envy among the facilities themselves or their assignments.

Now, the image is a black and white picture with dots and dashes, maybe representing a graph with nodes (dots) and edges (dashes). So first, we need to model the problem as a graph. Let's assume the dots are demand points (customers) and the dashes are connections (roads, distances) between them. Then, for each problem:

**Step 1: Model the Problem as a Graph**
- Nodes: Demand points (customers), let's say there are n nodes.
- Edges: Distances between demand points, represented by the dashes (maybe the length of the dash corresponds to distance, but need to assume uniform or given weights).

**p-median Solution Approach:**
- The p-median problem is NP-hard, so we might use approximation algorithms or heuristics. Common approaches include:
  - Greedy algorithm: Start by selecting the demand point with the highest total demand (if weighted) or the one with the largest degree (if unweighted), then iteratively add the demand point that minimizes the increase in total cost, ensuring no more than p facilities.
  - K-means-like approach: Cluster demand points into p clusters, then place a facility at the centroid of each cluster.
  - Linear programming relaxation followed by rounding, but for discrete solutions, maybe use facility location heuristics.

**p-envy Problem:**
- If envy is defined as the maximum difference in costs between any two demand points (i.e., max_{i≠j} (cost_i - cost_j)), then we need to assign demand points to p facilities such that this maximum difference is minimized. This is a multi-objective or constrained optimization problem. Heuristic approaches could include:
  - Iteratively assign demand points to facilities to balance the costs, ensuring that no facility has a cost too much higher than others.
  - Use a genetic algorithm or simulated annealing to explore different assignments and minimize the envy measure.

**p-intra-envy Problem:**
- This might refer to the envy among the facilities themselves, i.e., the maximum difference in costs between any two facilities. So, we need to place p facilities such that the maximum cost of any facility is minimized, or the envy among facilities is minimized. This could be similar to the p-median but with a different objective, or a problem where we have to balance the load among p facilities so that no facility is too much more costly than others. Heuristic approaches might include:
  - Load balancing algorithms, where we distribute demand points as evenly as possible among the p facilities, then place facilities at optimal positions (like median points of clusters).

**Assuming the Image is a Graph with Nodes and Edges:**
Let's say the image has, for example, 10 demand points (dots) connected by dashes (edges with weights). Let's outline a possible solution process:

1. **p-median Solution**:
   - Calculate the total distance from each demand point to all others (if unweighted, just count connections, but need weights).
   - Use a greedy algorithm: Start with the demand point that has the highest total distance to others (or highest demand), add it as a facility. Then, for each remaining demand point, calculate the cost if it's assigned to the existing facilities, and add the one that minimizes the total cost increase. Repeat until p facilities are selected.
   - For example, if there are 5 demand points and p=2, select the two points that minimize the sum of the two largest distances (or average).

2. **p-envy Solution**:
   - Assign each demand point to a facility, calculate the cost for each (sum of distances to its assigned facility).
   - Compute the envy as the maximum difference between any two demand points' costs.
   - Use a heuristic like iterative improvement: randomly assign assignments, then swap demand points between facilities to reduce the envy measure, repeating until no improvement.
   - Alternatively, use a clustering approach where each cluster is assigned to a facility, then assign points to minimize envy.

3. **p-intra-envy Solution**:
   - Similar to p-envy but focus on the facilities' costs. Calculate the cost for each facility (sum of distances of its assigned demand points), then compute the envy as the maximum difference between any two facilities' costs.
   - Use a heuristic to balance the facilities' costs: distribute demand points so that no facility's cost is more than a certain percentage of the others, then place facilities at optimal positions.

**Example with a Simple Graph (Hypothetical):**
Suppose the image has 4 demand points (A, B, C, D) connected as follows (distances between them):
- A-B: 1, A-C: 2, A-D: 3
- B-C: 1, B-D: 2
- C-D: 1
- Assume p=2.

**p-median (p=2):**
- Calculate total distances: A's total is 1+2+3=6, B's is 1+1+2=4, C's is 2+1+1=4, D's is 3+2+1=6.
- Greedy approach: Start with B (lowest total), then add C (total for C is 4, adding to B's assignment would make B's total 4+1=5, C's total 4, total cost 5+4=9? Wait, no, p-median is total cost from all demand points to their closest facility. So for each demand point, assign to the closest of the p facilities. So if we have two facilities, assign each demand point to the closest of the two. Let's say facility 1 is at B, facility 2 is at C. Then A is closest to B (distance 1), B is at B (0), C is closest to C (0), D is closest to C (1). Total cost: 1+0+0+1=2. If facility 1 is at A, facility 2 at C: A is 0, B is 1, C is 0, D is 3. Total cost 0+1+0+3=4. So better to have B and C as facilities, total cost 2.

**p-envy (p=2):**
- Assign each demand point to a facility, calculate their costs. If assign A to B (cost 1), B to B (0), C to C (0), D to C (1). Costs: 1, 0, 0, 1. Envy is max(1-0, 0-1, 1-0, 0-1) = 1. If assign A to C (cost 2), B to B (0), C to C (0), D to C (1). Costs: 2, 0, 0, 1. Envy is max(2-0, 0-0, 0-1, 1-0) = 2. So previous assignment has envy 1, which is better. If assign A to B, B to C, C to C, D to B: A's cost 1, B's cost 1, C's cost 0, D's cost 2. Envy is max(1-0, 0-1, 2-0, 0-1) = 2. So the first assignment (B and C as facilities, each demand point assigned to closest) has envy 1, which is better.

**p-intra-envy (p=2):**
- Focus on the facilities' costs. If facilities are at B and C, their costs are 0 (for B, since B is a facility, its cost is 0) and 0 (C is a facility, cost 0)? Wait, no, the cost for a facility is the sum of distances from its assigned demand points. Wait, maybe I got that wrong. Wait, in p-median, the cost for a facility is the sum of distances from demand points assigned to it. So if facility 1 is at B, demand points assigned to it are A (distance 1) and B (0), so cost is 1+0=1. Facility 2 at C, assigned demand points C (0) and D (1), cost is 0+1=1. So each facility has cost 1, envy between them is 0 (1-1=0). That's good. If we had facility 1 at A, cost is 0 (A is a facility) + distance from A to B (1), A to C (2), A to D (3), so if demand points assigned to A are A, cost is 0, but if we assign B, C, D to A, cost is 1+2+3=6. If assign A to A, B to A (cost 1), C to A (cost 2), D to A (cost 3), total cost 0+1+2+3=6. Facility 2 at, say, C: assign C to C (0), D to C (1), B to C (1), A to C (2), total cost 0+1+1+2=4. Envy between facilities is max(6-4, 4-6)=2. So the first solution with facilities at B and C has each facility cost 1, envy 0, which is better.

**Generalizing to the Image:**
Since the image is a black and white dots and dashes, we need to first identify the graph structure: number of nodes (dots), edges (dashes) with weights (maybe the length of dashes, or given weights). Then, apply the respective algorithms for each problem, using heuristics like greedy, clustering, or iterative improvement, depending on the problem's objective (minimizing total cost, minimizing envy, etc.).

**Summary of Solutions:**
- **p-median**: Use a greedy or clustering heuristic to select p facilities that minimize the total weighted distance from demand points to their closest facility.
- **p-envy**: Use a heuristic to assign demand points to p facilities such that the maximum envy (cost difference) between any two demand points is minimized, possibly using iterative improvement or balanced clustering.
- **p-intra-envy**: Use a heuristic to balance the costs among the p facilities, ensuring that no facility's cost is excessively higher than others, possibly using load balancing or equalized cost allocation.

Since the image is a specific graph, the exact solution would depend on the graph's structure, but the approach involves modeling the problem as a graph, applying appropriate heuristics for each discrete problem, and optimizing the objective functions (total cost, envy, etc.) accordingly.</think><answer>To solve the discrete problems (p-median, p-envy, p-intra-envy) depicted in the image (a graph with dots as demand points and dashes as connections), follow these steps:  


### 1. **Model the Problem as a Graph**  
Assume the image shows a graph where:  
- **Dots** represent demand points (customers, locations, etc.).  
- **Dashes** represent connections (roads, distances, or costs) between demand points.  
- Weights (distances/costs) are associated with edges (dashes), which may be inferred from the image (e.g., longer dashes = higher cost).  


### 2. **Solve the p-median Problem**  
The *p-median problem* aims to open `p` facilities to minimize the total weighted distance from all demand points to their nearest facility.  

- **Approach**: Use a **greedy or clustering heuristic** (common for NP-hard problems).  
  - **Greedy**: Iteratively select the demand point (or location) that minimizes the increase in total cost when added as a facility.  
  - **Clustering**: Group demand points into `p` clusters, then place a facility at the centroid (or median) of each cluster.  
  - **Example**: If the graph has 4 demand points (A, B, C, D) with distances as described earlier, placing facilities at B and C minimizes total cost (e.g., total cost = 1 + 1 = 2 for assignments to B and C).  


### 3. **Solve the p-envy Problem**  
The *p-envy problem* minimizes the maximum envy (cost difference) between any two demand points. Envy is defined as `max_{i≠j} (cost_i - cost_j)`, where `cost_i` is the cost for demand point `i` (distance to its assigned facility).  

- **Approach**: Use **iterative improvement or balanced clustering**.  
  - Randomly assign demand points to `p` facilities, then swap assignments between facilities to reduce envy.  
  - Example: For the 4-demand point graph, assigning A→B, B→B, C→C, D→C gives envy = 1 (max difference between costs is 1), which is optimal for `p=2`.  


### 4. **Solve the p-intra-envy Problem**  
The *p-intra-envy problem* minimizes envy among the `p` facilities (i.e., `max_{k≠l} (cost_k - cost_l)`, where `cost_k` is the total cost for facility `k`).  

- **Approach**: Use **load balancing or equalized cost allocation**.  
  - Distribute demand points evenly among `p` facilities to ensure no facility’s cost is excessively higher than others.  
  - Example: For the 4-demand point graph, placing facilities at B and C results in each facility having cost = 1, so envy = 0 (no facility envies another).  


### General Solution Process (For Any Graph)  
1. **Graph Analysis**: Count nodes (demand points) and edges (connections), then assign weights (distances/costs) to edges.  
2. **Problem-Specific Heuristic**:  
   - **p-median**: Greedy/Clustering (minimize total cost).  
   - **p-envy**: Iterative Improvement/Clustering (minimize max envy between demand points).  
   - **p-intra-envy**: Load Balancing/Clustering (minimize max envy between facilities).  
3. **Optimization**: Run the heuristic until no further improvement in the objective function (total cost, envy, etc.).  


For the specific image (with its unique dot/dash structure), the exact solution depends on the graph’s weights and structure, but the above framework provides a systematic approach to solving each discrete problem.  

The answer is <|begin_of_box|>[Solution depends on the image’s graph structure; general approach involves graph modeling and heuristic optimization for each problem]<|end_of_box|>. (Note: Since the image is illustrative, a specific numerical answer isn’t possible without the exact graph details. The framework above outlines the method.)</answer>