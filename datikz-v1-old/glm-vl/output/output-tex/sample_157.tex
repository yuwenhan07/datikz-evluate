<think>Got it, let's try to figure out what the user is asking. The question mentions "Left: the dual graph for singularity III_n; right: the 3d mirror for the low energy theory associated with the singular fiber III_n." First, I need to recall what a dual graph is in the context of singularities, especially for type III_n singularities. 

Type III_n singularities are a class of singularities in algebraic geometry, often related to elliptic fibrations. The dual graph (or maybe the minimal resolution graph) would represent the exceptional divisors and their intersections after resolving the singularity. For a type III_n singularity, the dual graph is typically a graph with n+1 vertices, maybe a chain or a more complex structure depending on n. 

On the right, the 3d mirror is likely referring to the mirror symmetry in the context of 3d mirror symmetry, which relates the geometry of a Calabi-Yau threefold to a 3d conformal field theory (CFT). The low energy theory associated with the singular fiber III_n would be a 3d CFT, and its mirror would be the geometry. For a singular fiber, the mirror geometry often has a singularity corresponding to the original singularity, and the dual graph might relate to the structure of the mirror's singularities or the brane configuration.

Wait, let's think about the dual graph for a type III_n singularity. The minimal resolution of a type III_n singularity (which is a cyclic quotient singularity, maybe?) would have a dual graph that's a cycle with n+1 nodes, or maybe a more specific structure. Alternatively, in the context of elliptic fibrations, the dual graph might represent the fibers and their intersections. 

The 3d mirror for the low energy theory—low energy theory here would be the CFT living on the singular fiber, which is a 3d theory. The mirror would be the geometry, so the 3d mirror would be a Calabi-Yau threefold with singularities corresponding to the original singularity. The dual graph on the left might be the graph of the exceptional divisors in the resolution of the mirror singularity, or the graph of the singularities in the mirror geometry.

I need to make sure about the specific properties. Type III_n singularities are part of the list of elliptic singularities, and their dual graphs (in the sense of the minimal resolution) are known. For example, type III_0 is a node, III_1 is a chain of two nodes, etc., so the dual graph would be a chain with n+1 nodes. The 3d mirror's geometry would have a singularity whose dual graph is related, maybe the same as the original, or a dual graph in the mirror.

Alternatively, in the context of 3d mirror symmetry, the mirror of a Calabi-Yau threefold with a singularity is another Calabi-Yau threefold with a singularity, and the dual graph might represent the configuration of the singularities or the resolution graph. The low energy theory on the singular fiber would be a theory with a gauge group or a quiver, and its mirror would be the geometry.

Since the user is asking for an explanation of the left and right, maybe the left is the dual graph (resolution graph) of the III_n singularity, which is a graph with n+1 vertices connected in a chain (or a cycle, depending on the type), and the right is the mirror geometry, which is a 3d Calabi-Yau with a singularity corresponding to the dual graph, or the mirror's dual graph is the original's.

Wait, another angle: dual graph for a singularity is the graph whose vertices correspond to the exceptional divisors in the minimal resolution, and edges represent their intersections. For a type III_n singularity, the minimal resolution has n+1 exceptional curves, forming a chain (so the dual graph is a path graph with n+1 vertices). The 3d mirror's geometry would be a Calabi-Yau threefold, and the singularities in the mirror would correspond to the original singularity, with the dual graph of the mirror's singularity being the same as the original's dual graph, or its dual.

So, summarizing: The left image is the dual graph (resolution graph) of the type III_n singularity, which is a graph with n+1 vertices connected in a linear chain (or a cycle, but typically a chain for III_n). The right image is the 3d mirror geometry, which is a Calabi-Yau threefold with a singularity whose dual graph is related to the original, often the same or its dual, representing the low energy 3d CFT's mirror.</think><answer>To address the question about the dual graph and 3d mirror for singularity \( III_n \):  

### Left: Dual Graph for Singularity \( III_n \)  
The dual graph (or minimal resolution graph) of a type \( III_n \) singularity represents the configuration of exceptional divisors after resolving the singularity. For \( III_n \), the minimal resolution typically results in a chain of \( n+1 \) exceptional curves (or divisors), forming a **linear graph** (a path graph with \( n+1 \) vertices). Each vertex corresponds to an exceptional divisor, and edges represent their intersections. This graph encodes the topological structure of the singularity’s resolution.  

### Right: 3d Mirror for the Low-Energy Theory  
The 3d mirror is a Calabi-Yau threefold whose geometry mirrors the low-energy 3d conformal field theory (CFT) associated with the \( III_n \) singular fiber. The low-energy theory on the singular fiber is a 3d CFT, and its mirror geometry has a singularity corresponding to the original \( III_n \) singularity. The dual graph of this mirror’s singularity often relates to the original dual graph (e.g., the same or its dual), reflecting the duality between the geometry and the CFT. The mirror geometry’s structure encodes the quantum field theory’s symmetries and interactions, with its singularities corresponding to the original singularity’s resolution.  

In summary, the left graph is the resolution graph of \( III_n \), while the right is the mirror Calabi-Yau threefold with a singularity related to the original, both connected by mirror symmetry principles.  

The answer is \(<|begin_of_box|>III_n<|end_of_box|>\).</answer>