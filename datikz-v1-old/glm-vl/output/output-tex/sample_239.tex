<think>Got it, let's try to parse this question step by step. First, the user is describing a Directed Acyclic Graph (DAG) related to confounding, specifically with a common parent, and then a deleted edge between U_z and z. They mention a counterfactual quantity independent of its parents, and then there's an image description with a white sphere, blue arrow labeled "U" with "X=Y" equation.

First, let's recall DAG concepts: confounding often involves a common parent (like U) that affects both the treatment (X) and outcome (Y), creating a backdoor path. Deleting an edge (like between U_z and z) might be about removing a direct effect, making the counterfactual quantity (like do(X=x)) independent of its parents. But the image part: white sphere (maybe representing a node), blue arrow labeled U, equation X=Y. Wait, the arrow is from the sphere? So maybe the node is U, and the arrow is from U to X, with X=Y? Wait, the description says "blue arrow pointing to the right" from a white sphere, labeled U, with equation X=Y. Hmm, maybe the graph has a node U, an arrow from U to X, and X=Y is the equation for that arrow. But the question is about a counterfactual quantity being independent of its parents, which in causal DAGs, a counterfactual like do(X=x) would have parents that are not the treatment, so if you remove the direct effect of U on X, then the counterfactual might not depend on U's parents? Wait, maybe the key is that after deleting the edge between U_z and z (maybe U_z is a specific node), the counterfactual quantity (like do(z=z') or something) is independent of its parents. But the image has a white sphere (maybe a node), blue arrow labeled U, equation X=Y. Wait, the user says "the arrow is labeled with a variable symbol, 'U,' and has an associated equation, 'X=Y'." So the arrow is from the white sphere (maybe the node is, say, U, and the arrow is from U to X, with the equation X=Y? Wait, maybe the node is X, and the arrow is from U to X, labeled U, with equation X = Y? No, need to connect the parts.

Wait, let's break down the question components:

1. DAG with common parent confounding: usually, confounding has a common parent U affecting both X (treatment) and Y (outcome), so the path X→Y and U→X→Y, with U as the common parent.

2. Now with deleted edge between U_z and z: maybe U_z is a specific node (like a specific instance of U), and z is another node, so deleting the edge between them. If z is the outcome, then deleting U→z would remove the direct effect of U on z, making the confounding path U→X→z, but if the edge U→z is deleted, then the only path is U→X→z, so the confounding is still there, but maybe the counterfactual quantity (like do(X=x) then the outcome) is independent of U's parents? Wait, counterfactuals often require that the treatment is not affected by the unobserved confounders, so if you have a backdoor adjustment, but if the edge is deleted, maybe the counterfactual is independent of its parents.

3. The image features a white sphere with a blue arrow pointing to the right. The arrow is labeled with a variable symbol, "U," and has an associated equation, "X=Y." So the white sphere is probably a node (maybe U), the blue arrow is from U to another node (maybe X), labeled U, and the equation is X=Y? Wait, maybe the node is X, and the arrow is from U to X, with the label U, and the equation is X = Y? Or maybe the arrow is from X to Y, labeled U, with X=Y? Hmm, the description says "blue arrow pointing to the right" from the white sphere, so the sphere is the start, arrow goes right, labeled U, equation X=Y. So maybe the node is, say, U, and the arrow is from U to X, with the label U (maybe indicating the variable), and the equation is X = Y (so X equals Y). But how does this relate to the counterfactual being independent of its parents?

Wait, a counterfactual quantity, like the potential outcome under a different treatment, is often defined as a function of the treatment and the unobserved confounders. If the treatment's parents (after some adjustments) don't include the confounders, then the counterfactual is independent of its parents. Deleting an edge (like between U and X) would mean that X is no longer directly affected by U, so the parents of X would not include U, making the counterfactual (like do(X=x)) not depend on U's parents. But the image has a white sphere (maybe a node), blue arrow labeled U, equation X=Y. Maybe the sphere is the node X, the arrow is from U to X, labeled U, with equation X = Y (so X is equal to Y, meaning they are the same variable? That might not make sense). Alternatively, maybe the equation is X = U, but the question says "X=Y". Hmm, maybe the image is a simplified graph where U is a parent of X, and X=Y, so Y is equal to X, which is equal to U. But the key point is that after deleting the edge between U_z and z (if z is Y), then the only path is U→X→Y, and the counterfactual for Y given X=x would be independent of U's parents if the adjustment set is appropriate.

But maybe the question is more about the structure: after deleting the edge, the counterfactual quantity (like the potential outcome) is independent of its parents, meaning that the causal effect is identifiable without adjusting for the confounder, because the confounder's effect is no longer direct. The image with the white sphere (maybe the confounder U) and the arrow labeled U pointing to X, with X=Y, might be illustrating that X and Y are the same, so the confounder's effect is direct, but after deleting the edge, the effect is indirect, making the counterfactual independent.

Alternatively, since the question mentions "counterfactual quantity independent of its' parents", in causal DAGs, a counterfactual like do(X=x) has parents that are the unobserved variables not affected by the intervention. If the intervention is on X, then the parents of the counterfactual would be the variables not connected to X, or the ones that are not in the backdoor path. If the edge between U and X is deleted, then U is no longer a parent of X, so the counterfactual do(X=x) would have parents that don't include U, making it independent of U's parents.

Putting it all together, the DAG with common parent confounding (U affecting X and Y) has a deleted edge between U and X (or U and Y), making the confounding indirect, and the counterfactual quantity (like the potential outcome) is now independent of its parents because the direct effect of the confounder is removed. The image with the white sphere (U), blue arrow labeled U pointing to X (or Y), with equation X=Y (or X=U) illustrates this, showing the relationship between the variables after the edge deletion.</think><answer>To address the question, we analyze the causal DAG structure and the image description step-by-step:  

### 1. DAG with Common Parent Confounding  
In a standard confounding scenario, a common parent \( U \) affects both the treatment \( X \) and outcome \( Y \), creating a backdoor path \( U \rightarrow X \rightarrow Y \). This makes \( U \) a confounder, as it influences both \( X \) and \( Y \), potentially biasing causal estimates.  

### 2. Deleted Edge Between \( U_z \) and \( z \)  
Assuming \( U_z \) is a specific instance of the confounder \( U \) and \( z \) is the outcome, deleting the edge \( U_z \rightarrow z \) removes the direct effect of \( U \) on \( z \). This leaves only the indirect path \( U \rightarrow X \rightarrow z \), reducing direct confounding.  

### 3. Counterfactual Quantity Independence  
A counterfactual quantity (e.g., \( \text{do}(X = x) \), the potential outcome under treatment \( x \)) is independent of its parents if the treatment’s direct effect on the outcome is not confounded by unobserved variables. After deleting the edge \( U_z \rightarrow z \), the confounder’s direct influence on the outcome is removed, making the counterfactual quantity (e.g., \( \text{do}(X = x) \)) independent of \( U \)’s parents.  

### 4. Image Description Interpretation  
The image features a white sphere (representing a node, e.g., \( U \)), a blue arrow pointing right (labeled \( U \), indicating the variable), and the equation \( X = Y \) (suggesting \( X \) and \( Y \) are equivalent or directly related). This simplifies the graph to show \( U \) influencing \( X \), which equals \( Y \). After deleting the edge between \( U \) and \( Y \) (or \( U \) and \( X \)), the relationship becomes indirect, aligning with the counterfactual independence.  

### Final Summary  
The DAG with a deleted edge between the confounder \( U \) and the outcome \( z \) removes direct confounding, making the counterfactual quantity independent of its parents. The image’s white sphere (confounder), blue arrow (labeled \( U \)), and equation \( X = Y \) illustrate this simplified causal structure.  

The answer is <|begin_of_box|>\text{Counterfactual quantity independent of its parents after edge deletion}<|end_of_box|>.</answer>