<think>Got it, let's try to understand this step by step. First, the question is about boundary global symmetries and entanglement wedges. Let me recall what entanglement wedges are—they're regions in a quantum field theory (or gravity) related to the entanglement entropy of a region, and the entanglement wedge is the causal development of that region. 

The statement says the action of the boundary global symmetry is restricted to the union of the entanglement wedges (the grey area). So, if an operator (x) is not in any of these entanglement wedges, then the global symmetry can't act on it. That means the operator is not charged under that global symmetry. 

Let me think about the logic here. Global symmetries act on operators, and if an operator is outside the entanglement wedge where the symmetry acts, then the symmetry doesn't affect it. So the key idea is that the entanglement wedge defines the region where the symmetry's action is non-trivial, and outside that, the symmetry doesn't act. Therefore, the operator (x) being outside the union of entanglement wedges means it's not in the symmetry's action region, so the symmetry doesn't act on it. 

I need to make sure I connect the concepts correctly. Entanglement wedge is related to the bulk region dual to a boundary region, and global symmetries in the bulk correspond to symmetries on the boundary. If the symmetry acts only on the union of entanglement wedges, then any operator outside those wedges is not charged. So the conclusion is that (x) isn't acted upon by the global symmetry because it's not in the symmetry's action region (the union of entanglement wedges).</think><answer>To understand the statement, we analyze the relationship between boundary global symmetries, entanglement wedges, and operator charges:  

1. **Entanglement Wedges**: In a quantum field theory (QFT) or gravitational context, the entanglement wedge of a boundary region is the bulk region whose boundary is the original region, and it encodes the entanglement structure. The union of entanglement wedges for various boundary regions defines the "action region" for boundary symmetries.  

2. **Global Symmetry Action**: Boundary global symmetries act non-trivially only on operators within their action region (the union of entanglement wedges). If an operator \((x)\) lies outside this union, it is not in the symmetry’s action region.  

3. **Conclusion**: Since the action of the boundary global symmetry is restricted to the union of entanglement wedges (the grey area), an operator \((x)\) outside this region cannot be acted upon by the symmetry. Thus, \((x)\) is not charged under the global symmetry.  

The answer is \(<|begin_of_box|>(x) \text{ is not acted upon by the global symmetry}<|end_of_box|>\).</answer>