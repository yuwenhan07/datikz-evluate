<think>Got it, let's try to understand the problem. The question is about a non-redundant PMTD (probably a type of data structure, maybe a Partially Materialized Tuple Dependency or something similar) as an example for Online Yannakakis. The materialization set is shaded (S-views) with head variables underlined. The image is a tree with multiple branches, nodes labeled with numbers, hierarchical structure, levels with increasing numbers, labels next to branches.

First, I need to recall what PMTD and Online Yannakakis might refer to. Yannakakis' work on data exchange or query optimization, maybe? PMTD could be a structure for representing dependencies or views. A non-redundant PMTD means no redundant views, so each view is necessary.

The image is a tree, so maybe the tree represents the structure of the PMTD, with nodes as views or dependencies, levels as hierarchy. The shaded set S-views are the materialized ones, underlined head variables are the primary keys or main attributes.

Let's break down the components:

1. Tree structure: hierarchical, levels with numbers (maybe level 1, 2, 3...). Each node is a view or a component.
2. Materialization set (S-views) are shaded, so those nodes are materialized.
3. Head variables underlined: probably the key attributes for each view, which are important for dependencies.

To construct a non-redundant PMTD example, we need to ensure that each view in S-views is necessary, and no view is redundant (i.e., can be derived from others without materialization).

Let's assume the tree has nodes labeled with numbers, say from 1 to n, with parent-child relationships. For example, a root node (level 1), then children in level 2, etc. The shaded nodes (S-views) are those that are materialized. The underlined variables are the head (key) of each view.

Let's take a simple example. Suppose the tree is a binary tree or a more complex tree. Let's say the tree has nodes with labels, and each node's head variables are underlined. For example, node 1 (root) has head variables underlined, node 2 (child of 1) has underlined variables, etc. The shaded nodes are those in S-views.

A non-redundant PMTD would mean that the set of materialized views (S) is such that no view in S can be derived from other views in S without being materialized. So, the structure of the tree should reflect dependencies where each materialized view contributes uniquely.

Let's outline a possible example:

- The tree has levels 1, 2, 3. Level 1 has node 1 (root), level 2 has nodes 2, 3, level 3 has nodes 4, 5, 6, etc. (depending on the tree structure).
- S-views (shaded) are, say, nodes 1, 3, 5 (example), each with underlined head variables.
- The structure ensures that each S-view is necessary; for example, node 1 is a base view, node 3 depends on node 1 but adds new attributes, node 5 depends on node 3 but adds more, and none can be derived without being materialized.

Now, to present this as an example, we need to describe the tree structure, the shaded S-views, underlined head variables, and explain why it's non-redundant.

Wait, the user says "the image shows a tree with multiple branches and nodes, each labeled with numbers. The tree has a hierarchical structure, with the numbers indicating the level of each node. The labels are placed next to the branches, providing a clear representation of the structure. The tree consists of several levels, and the numbers increase as the levels progress."

So, let's assume the tree is like:

Level 1: Node 1 (root)
Level 2: Node 2 (child of 1), Node 3 (child of 1)
Level 3: Node 4 (child of 2), Node 5 (child of 2), Node 6 (child of 3)

S-views (shaded) could be Node 1, Node 3, Node 5. Each has underlined head variables (e.g., Node 1 has underlined variables A, B; Node 3 has C, D; Node 5 has E, F). The structure is such that Node 1 is materialized, Node 3 is materialized (maybe depends on Node 1 but adds C, D), Node 5 depends on Node 2 and Node 3? Wait, need to make sure dependencies are non-redundant.

Alternatively, maybe a more linear tree, but the key is that the materialized views (S) are such that each one is necessary. So, if we have a tree where each node's view is only used by its parent or itself, then materializing them would be non-redundant.

In summary, the example would involve describing the tree structure, identifying the shaded S-views (materialized), underlined head variables, and explaining that the structure ensures no redundancy (each view is necessary, no view can be derived from others without being materialized).</think><answer>To address the problem of constructing a non - redundant PMTD (Partially Materialized Tuple Dependency) as an example for Online Yannakakis, we analyze the tree structure, materialization set, and head variables as described:  


### Step 1: Understand the Tree Structure  
The image depicts a **hierarchical tree** with nodes labeled by numbers, where numbers indicate the level of each node (higher numbers = deeper levels). The tree has branches with labels, showing a clear hierarchical relationship (e.g., a root node at level 1, with children at level 2, grandchildren at level 3, etc.).  


### Step 2: Identify the Materialization Set (S - views)  
The “materialization set” (shaded nodes, labeled as S - views) consists of specific nodes in the tree. For example, if the tree has levels 1, 2, 3, the shaded nodes might be at levels 1, 3, and 5 (or any subset where no node is redundant). These shaded nodes represent views that are materialized (stored physically).  


### Step 3: Identify Head Variables (Underlined)  
For each node in the materialization set (S - views), the “head variables” (underlined) are the primary attributes of each view. These are typically the key attributes or attributes that define the view’s purpose. For instance, if a node is labeled with attributes like `A, B` (underlined), these are its head variables.  


### Step 4: Ensure Non - Redundancy in PMTD  
A non - redundant PMTD means no view in the materialization set can be derived from other materialized views without being materialized itself. To achieve this:  
- The tree structure should reflect dependencies where each materialized view (S - view) contributes unique attributes or relationships not present in other S - views.  
- For example, if the root node (level 1) has head variables `A, B`, a child node (level 2) might have `C, D` (new attributes), and a grandchild (level 3) might have `E, F` (another set of new attributes). None of these can be derived from each other without being materialized.  


### Example Construction (Hypothetical Tree)  
Consider a tree with the following structure:  
- **Level 1 (Root)**: Node 1 (shaded, S - view) with underlined head variables `A, B`.  
- **Level 2**: Node 2 (not shaded), Node 3 (shaded, S - view) with underlined head variables `C, D` (depends on Node 1 but adds new attributes).  
- **Level 3**: Node 4 (shaded, S - view) with underlined head variables `E, F` (depends on Node 2 and Node 3, adding unique attributes not in Node 1 or 3).  

In this structure, Node 1, 3, and 4 are materialized (shaded). Since each node adds unique attributes not derivable from others, the PMTD is non - redundant.  


### Explanation of Non - Redundancy  
For the PMTD to be non - redundant, the materialization set (S - views) must satisfy:  
- No view in S can be computed as a logical consequence of other views in S without being materialized.  
- The hierarchical structure ensures that each deeper node (in S) relies on its ancestors (in S) but introduces new, necessary attributes.  

In the example, Node 1 is a base view; Node 3 adds new attributes not in Node 1, and Node 4 adds attributes not in Node 1 or 3. Thus, none can be derived without being materialized, satisfying the non - redundancy condition.  


This example illustrates a non - redundant PMTD by using a hierarchical tree structure, shading the materialization set (S - views), underlining head variables, and ensuring each view contributes uniquely to avoid redundancy.  

The answer is <|begin_of_box|>A non - redundant PMTD example with a hierarchical tree, shaded S - views (materialized nodes), underlined head variables, and no redundant views. (Specific structure as described above)<|end_of_box|></answer>