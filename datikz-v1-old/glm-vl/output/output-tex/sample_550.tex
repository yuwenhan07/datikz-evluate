<think>Got it, let's try to understand the problem. The question is about topologically equivalent but geometrically distinct partitions of an annulus, labeled as (a) X is ABC and (b) X is A'B'C'. First, I need to recall what an annulus is—like a ring shape, between two concentric circles. A partition of the annulus would divide it into parts, maybe like sectors or arcs. 

Topologically equivalent means that the way the parts are connected or arranged is the same, but geometrically distinct means their shapes or positions differ, even though the overall structure is the same. So, for example, if you have two different ways of dividing the annulus into the same number of parts, but the parts are arranged differently, like rotated or with different angles, they might be topologically equivalent but geometrically distinct.

Looking at the image description: a circular cable with four parts labeled ABCD, but the problem mentions ABC and A'B'C' for the partitions. Maybe the annulus is divided into three parts, like sectors, and in (a) it's ABC, in (b) A'B'C', which might be a rotated version or a different arrangement of the same three parts. 

Topological equivalence here would mean that the partition has the same number of components, same connectivity (like each component is a closed loop or a sector connected appropriately), but geometrically distinct would mean the specific positions or angles of the sectors differ. For example, if you have three sectors each with 120 degrees, but one partition has them arranged with a certain orientation, and the other has them rotated by 60 degrees, they are topologically equivalent (same number of parts, same type of partition) but geometrically distinct (different positions/orientations).

So, the key idea is that the partitions have the same topological structure (same number of parts, same way of connecting them in the annulus) but differ in their geometric arrangement, like rotation or position of the parts. Therefore, the answer would involve explaining that the partitions have the same topological properties (like being a partition into three regions with the same connectivity) but different geometric shapes or positions, making them distinct geometrically even though topologically equivalent.</think><answer>To determine topologically equivalent but geometrically distinct partitions of an annulus, we analyze the structure of the partitions. An annulus is a ring-shaped region, and a partition divides it into subregions. **Topological equivalence** means the partitions have the same number of components and connectivity (e.g., how subregions are connected or arranged), while **geometric distinctness** refers to differences in shape, position, or orientation of the subregions.  

For the given partitions (a) \(X = ABC\) and (b) \(X = A'B'C'\), if both divide the annulus into three subregions (e.g., sectors), they are topologically equivalent because they both partition the annulus into three connected components. However, geometrically distinct means the subregions differ in their arrangement—such as rotation, angle, or position—while maintaining the same topological structure. For example, one partition might have sectors with specific angular measures arranged in one orientation, while the other has the same angular sectors rotated by a certain angle, resulting in different geometric shapes despite identical topological properties.  

The answer is \(<|begin_of_box|>(a) \text{ and } (b) \text{ are topologically equivalent but geometrically distinct partitions of the annulus}<|end_of_box|>\).</answer>