<think>Got it, let's try to figure out this RV move for rotating a vertex around an incident edge. The problem says the vertex is rotated π/3 (which is 60 degrees) around the left edge, and this changes the vertex state to the opposite one, creating a crossing between two edges. The top row shows how rotations change a state into another, and the bottom row is the reverse. 

First, let's recall what vertex states might look like. Maybe they're like "over" or "under" crossings, or some kind of orientation. When you rotate a vertex around an edge, the vertex's position changes, and the edges connected to it might cross. The key points are: rotation by π/3 around the left edge changes the vertex state to the opposite, and creates a crossing between two edges incident to the vertex. 

Let's think about the vertex states. Suppose the vertex has two edges, say edge A and edge B, with the left edge being one of them. When you rotate the vertex around edge A (the left edge) by π/3, the vertex moves, and the other edges (connected to the vertex) might cross. The opposite vertex state would mean if the original state is, say, "A" (no crossing), the rotated state is "B" (with a crossing), or vice versa. 

The top row shows the rotation changing a state to another, and the bottom row is the reverse. So maybe the top row is the forward rotation (changing to the opposite state with a crossing), and the bottom row is the reverse (undoing the rotation, changing back). 

Let's consider the vertex with three edges? Wait, the problem says "incident edges"—a vertex has edges coming into it, maybe two edges meeting at the vertex (like in a polygon, each vertex has two edges, but maybe in a more general graph, a vertex can have multiple edges). Wait, the problem says "rotating a vertex around one of its incident edges"—so the vertex is connected to at least one edge, and we rotate around that edge. 

Let's model the vertex as a point with edges connected to it. When you rotate the vertex around the left edge (which is one of the edges connected to the vertex) by π/3, the vertex moves to a new position, and the other edges connected to the vertex (the ones not being the axis of rotation) will cross each other, creating a crossing. The vertex state changes to the opposite state, meaning if it was a "non-crossing" state, it becomes a "crossing" state, or vice versa. 

The top row shows the rotation from one state to another, and the bottom row is the reverse. So if the top row is the forward move (rotation around left edge), then the bottom row is the reverse move (undoing the rotation, maybe rotating the other way or around the other edge). 

To visualize, imagine a vertex with two edges: left edge (axis of rotation) and another edge. When you rotate the vertex around the left edge by π/3, the other edge moves, and the two edges cross, creating a crossing. The vertex state (maybe indicating if the crossing is over or under) changes to the opposite. 

So the RV move would involve rotating the vertex around the left edge by π/3, changing the vertex's state to the opposite, and creating a crossing between the two edges incident to the vertex (the ones not the axis of rotation). The top row shows the result of the rotation (state change and crossing), and the bottom row shows the reverse (undoing the rotation, changing back to the original state without the crossing, or with the opposite crossing). 

Since the problem mentions "the top row shows how rotations change a state into a state, while the bottom row shows the reverse," we need to describe the move as a rotation around the left edge by π/3, changing the vertex state to the opposite, and creating a crossing between the two incident edges (other than the left edge). 

So the final answer would describe the RV move as rotating the vertex around its left incident edge by π/3, which changes the vertex's state to the opposite one and creates a crossing between the two other incident edges.</think><answer>The RV move involves rotating a vertex around one of its incident edges (the "left edge" in this case) by \( \frac{\pi}{3} \) (60 degrees). This rotation changes the vertex's state to its opposite and creates a crossing between the two edges incident to the vertex that are not the axis of rotation. The top row illustrates the forward rotation (state change and crossing creation), while the bottom row shows the reverse (undoing the rotation and reversing the state and crossing).  

The answer is \(<|begin_of_box|>\frac{\pi}{3}<|end_of_box|>\).</answer>