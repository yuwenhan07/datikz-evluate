<think>Got it, let's try to figure out how to show that d(k,l) is a linear combination of d(j,j-1) from the blue shaded region, and each d(j,j-1) contributes only in the orange region. First, I need to visualize the problem, maybe with a grid or matrix where d(k,l) is some entry, and d(j,j-1) are entries in a neighboring region. 

First, let's recall that in many matrix or tensor operations, a term like d(k,l) might be related to neighboring terms through some kind of convolution or recurrence. If the blue shaded region is the set of d(j,j-1) that affect d(k,l), then we need to show that d(k,l) can be written as a sum (linear combination) of these d(j,j-1), with coefficients that depend on the position of j relative to k and l. 

Then, the second part: a given d(j,j-1) contributes only in the orange region. So the support of each d(j,j-1) is the orange region, meaning when you look at the contribution of d(j,j-1) to d(k,l), it's non-zero only if (k,l) is in the orange region relative to (j,j-1). 

Maybe this is related to a Toeplitz matrix or a banded matrix, where each row has a certain number of non-zero entries, and the non-zero entries are in a band. The blue region could be the set of j's such that j is in some range related to k and l, and the orange region is the set of (k,l) such that they are within a certain distance from (j,j-1). 

Let me try to structure it. Suppose we have a grid where each d(k,l) is connected to d(j,j-1) where j is in some neighborhood of k, maybe j = k - m to k + n for some m, n. The blue shaded region would be the set of (j,j-1) for j in that neighborhood. Then d(k,l) is a linear combination of these d(j,j-1), with coefficients that depend on the difference between k and j, and l and j-1. 

For the second part, if each d(j,j-1) contributes only in the orange region, that means the set of (k,l) for which the coefficient in the linear combination is non-zero is the orange region. So the contribution of d(j,j-1) to d(k,l) is non-zero only if (k,l) is in the orange region relative to (j,j-1), which might mean that |k - j| ≤ some constant and |l - (j-1)| ≤ some constant, defining a rectangle or a band around (j,j-1). 

To formalize this, maybe we can define a matrix A where A_{(k,l),(j,j-1)} is the coefficient, and show that the sum over j of A_{(k,l),(j,j-1)} d(j,j-1) equals d(k,l). The blue region is the set of (j,j-1) for which A_{(k,l),(j,j-1)} ≠ 0, and the orange region is the set of (k,l) for which A_{(k,l),(j,j-1)} ≠ 0 for some (j,j-1), but wait, the problem says "a given d(j,j-1) contributes only in the orange region", so for each (j,j-1), the set of (k,l) where it contributes is the orange region. So the matrix A has non-zero entries only when (k,l) is in the orange region relative to (j,j-1), meaning for each row (k,l), the non-zero columns are (j,j-1) in the blue region, and for each column (j,j-1), the non-zero rows are (k,l) in the orange region. 

This sounds like a concept in linear algebra where the matrix is banded or has a specific structure, like a Toeplitz matrix but in two dimensions. Alternatively, if we're dealing with a recurrence relation, like in dynamic programming, where d(k,l) depends on previous terms d(j,j-1), and the dependency is local, meaning only nearby terms are used, which would correspond to the blue region being a local neighborhood, and the orange region being the area affected by each term. 

Another angle: if we consider the indices, suppose that d(k,l) is defined as a sum over j of c_{k,l,j} d(j,j-1), where c_{k,l,j} are coefficients. The blue shaded region is the set of j for which c_{k,l,j} ≠ 0, and the orange region is the set of (k,l) for which there exists a j such that c_{k,l,j} ≠ 0, but the problem states that a given d(j,j-1) contributes only in the orange region, meaning for each j, the set of (k,l) where c_{k,l,j} ≠ 0 is the orange region. So for each j, the non-zero coefficients are only for (k,l) in the orange region, which might be a rectangle or a band around j. 

To make this concrete, let's assume a simple case where the coefficients c_{k,l,j} are non-zero only if |k - j| ≤ m and |l - (j-1)| ≤ n for some m, n, defining a rectangle around (j,j-1), which would be the orange region for each d(j,j-1), and the blue region is the set of j's such that j is within some range related to k and l, making d(k,l) a combination of those j's. 

In summary, the key steps are: 1. Define the linear combination for d(k,l) as a sum over d(j,j-1) with coefficients depending on their positions. 2. Show that the set of j's (blue region) are those for which the coefficients are non-zero. 3. Show that for each d(j,j-1), the set of (k,l) where it contributes (orange region) is a specific region around (j,j-1), ensuring that each term's contribution is localized.</think><answer>To demonstrate that \( d(k,l) \) is a linear combination of \( d(j,j-1) \) from the blue shaded region, and that each \( d(j,j-1) \) contributes only in the orange region, we analyze the structure of the linear combination and the localized contribution of each term:  

1. **Linear Combination Structure**:  
   Assume \( d(k,l) \) is defined as a linear combination of \( d(j,j-1) \) with coefficients \( c_{k,l,j} \), i.e.,  
   \[
   d(k,l) = \sum_{j} c_{k,l,j} d(j,j-1),
   \]  
   where the sum is over \( j \) in the blue shaded region (the set of indices where \( c_{k,l,j} \neq 0 \)). The blue region is determined by the range of \( j \) that directly influence \( d(k,l) \), such as a local neighborhood around \( k \) (e.g., \( |k - j| \leq m \) for some \( m \)).  

2. **Localized Contribution of Each Term**:  
   For a given \( d(j,j-1) \), the coefficients \( c_{k,l,j} \) are non-zero only if \( (k,l) \) lies in the orange region relative to \( (j,j-1) \). This means the set of \( (k,l) \) for which \( d(j,j-1) \) contributes to \( d(k,l) \) forms a specific region (e.g., a rectangle or band) around \( (j,j-1) \). Thus, each \( d(j,j-1) \) affects only \( d(k,l) \) within its corresponding orange region, ensuring the contribution is localized.  

In summary, \( d(k,l) \) is a linear combination of \( d(j,j-1) \) from the blue region, with each \( d(j,j-1) \) contributing only to \( d(k,l) \) within its orange region, demonstrating localized dependency in the linear combination.  

The answer is \(<|begin_of_box|>d(k,l) = \sum_{j \in \text{blue region}} c_{k,l,j} d(j,j-1) \text{, where each } d(j,j-1) \text{ contributes only in the orange region}<|end_of_box|>\).</answer>