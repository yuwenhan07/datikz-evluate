<think>Got it, let's try to parse this question. The user wrote: "T in defT, =id(left); T'' in defT'', =id(middle); T'' in defT'', =(15)(right)." Hmm, maybe there's a typo or formatting issue. Let's break it down. 

First, "T in defT, =id(left); T'' in defT'', =id(middle); T'' in defT'', =(15)(right)." Maybe "defT" and "defT''" are definitions, and "id" refers to identity functions. Let's assume that "T" is a term in definition "defT" with identity on the left, then "T''" (maybe a double prime, indicating a modified term) is in definition "defT''" with identity on the middle, and then another "T''" in "defT''" with a permutation (15) on the right. 

Wait, the notation might be about term rewriting or definition substitution. Let's try to interpret each part:

1. "T in defT, =id(left)" – This could mean that in the definition defT, the term T is equal to the identity function applied to the left part.
2. "T'' in defT'', =id(middle)" – Similarly, T'' (maybe T with some modification, like a prime or double prime) in definition defT'' is equal to the identity on the middle part.
3. "T'' in defT'', =(15)(right)" – Then T'' in defT'' is equal to the permutation (15) applied to the right part.

But maybe the user is trying to define terms with different definitions, using identity and permutations. Alternatively, if this is about set theory or algebra, maybe T is a set, defT is a definition, and id is an identity function. But without more context, it's a bit tricky. However, the key is to parse the structure: each line defines a term in a definition with a specific function (id, identity) or permutation (15, a transposition). 

Wait, the original question might have a typo, like maybe "T in defT, = id(left); T'' in defT'', = id(middle); T''' in defT''', = (15)(right)" or something, but as given, let's try to make sense. The user might be defining three different terms: T in defT is identity on left, T'' in defT'' is identity on middle, T'' (same as previous T'') in defT'' is (15) on right. Maybe the prime indicates a different term, but double prime might be a typo. Alternatively, maybe "T''" is a typo for "T'" or another term. 

Alternatively, if this is about lambda calculus or term rewriting, where each definition defines a term with certain operations. But given the lack of context, the best approach is to interpret the statements as defining three terms: one in defT with identity on left, one in defT'' with identity on middle, and another in defT'' with permutation (15) on right. 

But since the question is likely asking for an interpretation or a conclusion, maybe the final term is T'' = (15)(right), which is a permutation applied to the right part. Alternatively, if we need to combine them, but without more info, it's hard. Wait, the user might have intended to write definitions for three different terms, each in their respective definitions, with different operations. Given that, the answer might involve stating the definitions as given, but since the question is a bit unclear, maybe the key is to recognize the structure of each definition. 

Alternatively, if this is about defining a term T with different parts, but the notation is confusing. Given the original text, maybe the intended meaning is that in definition defT, T is equal to the identity function on the left part; in definition defT'', T'' is equal to the identity function on the middle part; and in definition defT'', T'' is equal to the permutation (15) on the right part. So, summarizing, the terms are defined with identity on left and middle, and permutation on right.</think><answer>The question involves defining terms within specific definitions using identity functions and permutations. Here's the breakdown:  
1. In definition \( \text{defT} \), the term \( T \) is defined as the identity function applied to the "left" part, written as \( T \in \text{defT}, = \text{id}(\text{left}) \).  
2. In definition \( \text{defT''} \), the term \( T'' \) (likely a modified version of \( T \)) is defined as the identity function applied to the "middle" part, written as \( T'' \in \text{defT''}, = \text{id}(\text{middle}) \).  
3. In definition \( \text{defT''} \) (same definition as the second one), the term \( T'' \) is further defined as the permutation \( (15) \) applied to the "right" part, written as \( T'' \in \text{defT''}, = (15)(\text{right}) \).  

This structure indicates that \( T'' \) has two definitions: one as an identity on the middle part and another as a permutation on the right part, both within the same definition \( \text{defT''} \).  

The answer is \(<|begin_of_box|>(15)(\text{right})<|end_of_box|>\).</answer>