<think>Got it, let's try to figure out what the question is asking. The user mentions a "schematic diagram of and _" with parts (a) and (b). Hmm, maybe this is about data structures related to permutations, like a permutation chain or a data structure that handles permutations with some operations. 

First, let's parse the question again. It says: "Schematic diagram of and _. (a) The permutation chain _1,_2,_q+1 specified by . (b) _, where the permutations are removed after queries." Wait, maybe there's a typo or missing information, but let's think about common data structures. A permutation chain could be a data structure like a permutation tree or a Fenwick tree for permutations, or maybe a segment tree that supports permutation operations. 

For part (a), a permutation chain _1, _2, _q+1—maybe it's a sequence of permutations, and the schematic diagram would show how these permutations are chained, perhaps with operations like applying permutations sequentially. For part (b), where permutations are removed after queries, maybe it's a data structure that processes queries and then discards the permutations used, so the schematic would show the structure before and after queries, or how the permutations are managed.

But since the question is a bit vague without more context, maybe it's about a data structure like a permutation array or a structure that supports range queries on permutations. Alternatively, if this is related to a specific problem, like a problem where you need to maintain permutations and apply them, then the schematic would show the permutation chain as a linked list or a tree, and part (b) shows the structure after removing permutations.

Wait, another angle: "permutation chain" might refer to a sequence of permutations, and the schematic diagram would show the chain structure, maybe as a linked list where each node is a permutation, and part (a) shows the chain from 1 to q+1, while part (b) shows the structure after removing permutations (so maybe the chain is empty or the permutations are deleted).

But since the user might be referring to a standard problem, maybe in competitive programming, a permutation chain could be a data structure that allows applying permutations in sequence, and the schematic shows the chain structure. Alternatively, if it's about a problem where you have to process queries on permutations, then part (a) is the initial chain, part (b) is after processing.

Given the lack of specific details, but trying to provide a possible answer, maybe the schematic diagram is of a permutation array or a linked list of permutations, with part (a) showing the chain from 1 to q+1, and part (b) showing the array after removing permutations (so maybe the array is empty or the permutations are deleted). But I need to make sure the answer is clear.

Alternatively, if the question is about a data structure like a Fenwick tree for permutations, the schematic would show the tree structure, and part (a) shows the tree with the permutation chain, part (b) shows the tree after removing permutations (maybe the tree is cleared).

But since the question is a bit unclear, but given the structure, maybe the answer is about a permutation chain being a linked list of permutations, and the schematic shows that. For part (a), the chain is specified by some conditions, and part (b) shows the structure after removing permutations. So the schematic diagram would illustrate the permutation chain as a sequence, maybe with arrows indicating the order, and part (b) shows the same structure without the permutations (maybe empty).

But to sum up, since the question is asking for a schematic diagram of something (maybe a data structure) and then parts (a) and (b) describing the permutation chain and its state after queries, the answer might involve describing the structure as a linked list or tree, with part (a) showing the chain and part (b) showing the structure after removal. However, without more context, it's hard to be precise, but I'll try to provide a possible answer based on common permutation data structures.</think><answer>The question appears to describe a schematic diagram related to permutation chains and their management after queries. For part (a), the permutation chain \( _1, _2, _{q+1} \) is likely structured as a linked list or sequence of permutations, with the schematic showing their ordered connection (e.g., a chain where each permutation links to the next). For part (b), the schematic would illustrate the same structure after removing permutations, possibly showing an empty chain or the structure without the stored permutations.  

The answer is \(<|begin_of_box|>permutation\ chain<|end_of_box|>\).</answer>