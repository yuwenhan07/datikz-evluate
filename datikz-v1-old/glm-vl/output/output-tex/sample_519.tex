<think>Got it, let's try to understand this problem step by step. First, the question is about the Wadge hierarchy W_X on a zero-dimensional Polish space X, where W_X isn't isomorphic to W_N (which I think is the Wadge hierarchy on the natural numbers, maybe the Baire space or something similar) or W_C (maybe the Cantor space?). The key point is that in this case, the ordinal _X (which probably refers to the level of the hierarchy, like the supremum of the ordinals corresponding to the levels) satisfies < _X < _1, where _1 is the first uncountable ordinal. 

First, let's recall some basics about the Wadge hierarchy. For a Polish space X, the Wadge hierarchy is a classification of closed sets based on their Wadge rank, which is an ordinal. The hierarchy is defined by transfinite recursion: W_0 = X, and for each ordinal α, W_{α+1} is the set of closed sets that are Wadge reducible to some set in W_α, and for limit ordinals λ, W_λ is the union of W_α for α < λ. The Wadge rank of a closed set is the smallest ordinal α such that the set is in W_{α+1}. 

Now, the problem states that when W_X is not isomorphic to W_N or W_C, then the ordinal _X (maybe the supremum of the Wadge ranks of all closed sets in X, or the level of the hierarchy for X) satisfies < _X < _1. Wait, but _1 is the first uncountable ordinal, so if _X is less than _1, that means it's countable. But the hierarchy on N (natural numbers, which is a zero-dimensional Polish space) has W_N with rank ω_1, right? Wait, maybe W_N is the Wadge hierarchy on the Baire space, which is isomorphic to the natural numbers with the product topology, and its Wadge hierarchy has uncountable levels. Similarly, W_C might be the Cantor space, whose Wadge hierarchy also has uncountable levels. So if X's Wadge hierarchy is not isomorphic to either, then its hierarchy is countable, hence the ordinal _X is countable, so less than _1. 

But let's think about the structure. For a zero-dimensional Polish space, the Wadge hierarchy's level is related to the complexity of the space. If the space is such that its Wadge hierarchy is not like N or C, then it must have a hierarchy that's strictly between the countable and uncountable ones. The statement says < _X < _1, meaning _X is a countable ordinal, but maybe even less? Wait, the problem says "ordinal _X satisfies < _X < _1", so _X is an ordinal less than _1, which is countable. But maybe _X is the supremum of the Wadge ranks, which for such a space would be a countable ordinal, hence less than ω_1. 

Another angle: the Wadge hierarchy on a space X is isomorphic to the Wadge hierarchy on another space Y if there's a Wadge reduction between them. If X is not isomorphic to N or C in terms of their Wadge hierarchies, then its hierarchy must be of a different type. For example, if X is a space with a countable Wadge hierarchy, then its ordinal would be countable. But why is it strictly less than _1? Wait, maybe _X is the first ordinal that cannot be realized as the Wadge rank of a set in X, but I need to recall specific theorems. 

Wait, there's a theorem that says that the Wadge hierarchy on a Polish space is isomorphic to either the Wadge hierarchy on the natural numbers (which has uncountable levels) or the Cantor space (also uncountable), or some other space with a countable hierarchy. But if the hierarchy is not isomorphic to either, then it must have a countable hierarchy, hence the ordinal is countable, so less than ω_1. Therefore, the conclusion is that _X is a countable ordinal, so < ω_1, which is _1. 

So putting it all together, when the Wadge hierarchy W_X on X (zero-dimensional Polish space) is not isomorphic to W_N or W_C, then the ordinal associated with its hierarchy (like the supremum of the ranks) is a countable ordinal, hence less than the first uncountable ordinal _1. Therefore, the statement holds.</think><answer>To analyze the problem, we start by recalling the Wadge hierarchy for a zero-dimensional Polish space \( X \). The hierarchy is defined by transfinite recursion, with each level \( W_{\alpha+1} \) containing closed sets Wadge reducible to \( W_\alpha \), and limit levels as unions. The Wadge rank of a closed set is the smallest ordinal \( \alpha \) such that the set belongs to \( W_{\alpha+1} \).  

For Polish spaces like \( \mathbb{N} \) (natural numbers) or the Cantor space \( \mathbb{C} \), their Wadge hierarchies are isomorphic to those with uncountable levels (corresponding to \( \omega_1 \), the first uncountable ordinal). If \( W_X \) is not isomorphic to \( W_\mathbb{N} \) or \( W_\mathbb{C} \), its hierarchy must be countable, meaning the supremum of Wadge ranks (denoted \( \underline{X} \)) is a countable ordinal. Since \( \omega_1 \) is the first uncountable ordinal, a countable ordinal satisfies \( \underline{X} < \omega_1 \).  

Thus, when \( W_X \) is not isomorphic to \( W_\mathbb{N} \) or \( W_\mathbb{C} \), the associated ordinal \( \underline{X} \) is strictly less than \( \omega_1 \).  

The answer is \(<|begin_of_box|>< \omega_1<|end_of_box|>\).</answer>