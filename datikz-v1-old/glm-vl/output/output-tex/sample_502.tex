<think>Got it, let's try to figure out how to approach this. The question is about Petri nets for processes (C B), (C B) h, and (B B) h. First, I need to recall what a Petri net is. A Petri net has places, transitions, and arcs connecting them. Places hold tokens, transitions fire when enabled (i.e., all input places have at least one token), and firing removes tokens from input places and adds to output places.

First, let's parse each process:

1. Process (C B): This is a sequence of two actions, C followed by B. So the Petri net should have transitions for C and B, connected in series. Let's think about the structure. For a sequence, we might have a place before C, then after C, a place before B, then after B. Wait, or maybe two transitions in series with intermediate places. Let's consider the basic structure for a sequence: if you have a process A then B, the Petri net would have a place p1, then a transition A with input p1 and output p2, then a transition B with input p2 and output p3. So tokens start at p1, fire A to p2, then fire B to p3. So for (C B), the Petri net would have places, say, p1, p2, p3, transitions C (from p1 to p2), B (from p2 to p3). Initial marking has a token in p1, final marking has a token in p3.

2. Process (C B) h: The "h" probably indicates a history, so maybe the process is a history of C followed by B, meaning that the sequence is recorded, so the Petri net might have additional places to remember the history. Wait, in Petri net theory, a history might involve adding places to track the history of transitions. Alternatively, maybe it's a parallel composition? No, (C B) h is probably a history, so the Petri net would have the same transitions as (C B) but with additional places to mark the history. Wait, maybe the history is represented by a place that is marked after each transition. So for (C B) h, after firing C, there's a place for the history of C, then after firing B, a place for the history of B. So the Petri net would have places: start, C, B, history_C, history_B, end. Transitions: start → C (with arc from start to C), C → history_C (arc from C to history_C), history_C → B (arc from history_C to B), B → history_B (arc from B to history_B), history_B → end (arc from history_B to end). Initial marking: start has a token, final marking: end has a token. But maybe the exact structure depends on the specific notation, but generally, history Petri nets add places to record the history of transitions.

3. Process (B B) h: Similarly, this is a history of two B actions in sequence. So the Petri net would have transitions for B twice, with history places. So places: start, B1, B2, history_B1, history_B2, end. Transitions: start → B1, B1 → history_B1, history_B1 → B2, B2 → history_B2, history_B2 → end. Initial marking: start has a token, final marking: end has a token. Each B transition would have arcs as before.

Wait, but maybe the question is about the structure of the Petri nets for each process, so we need to describe the places, transitions, and arcs for each case.

Let's start with (C B):

- Places: p1 (start), p2 (after C), p3 (after B), p4 (end)
- Transitions: t1 (C), t2 (B)
- Arcs: p1 → t1 (input), t1 → p2 (output); p2 → t2 (input), t2 → p3 (output); p3 → p4 (input), p4 is final.
Initial marking: p1 has 1 token, others 0. Final marking: p4 has 1 token.

For (C B) h: This is a history, so we need to add places to record the history of each transition. So:

- Places: p1 (start), t1 (C), h1 (history of C), t2 (B), h2 (history of B), p4 (end)
- Transitions: t1, t2
- Arcs: p1 → t1 (input), t1 → h1 (output); h1 → t2 (input), t2 → h2 (output); h2 → p4 (input)
Initial marking: p1 has 1 token. When t1 fires, p1 loses a token, h1 gains a token. Then h1 loses a token, t2 gains a token, t2 fires, h2 gains a token, h2 loses a token, p4 gains a token. So the history places h1 and h2 record that C and B were fired, respectively.

For (B B) h: Similar structure, but with two B transitions and their history.

- Places: p1 (start), t1 (B), h1 (history of B), t2 (B), h2 (history of B), p4 (end)
- Transitions: t1, t2
- Arcs: p1 → t1 (input), t1 → h1 (output); h1 → t2 (input), t2 → h2 (output); h2 → p4 (input)
Initial marking: p1 has 1 token. Firing t1: p1 → h1, then h1 → t2, t2 → h2, h2 → p4. So history places h1 and h2 record the two B transitions.

Wait, but maybe the "h" is indicating a history, so the Petri net for (C B) h would have the same transitions as (C B) but with additional places to mark the history of each transition. So the key is that for a history, you add a place after each transition to record its occurrence.

So summarizing each:

(a) (C B) Petri net:
- Places: start, C, B, end
- Transitions: C, B
- Arcs: start → C, C → B, B → end
Initial marking: start has 1 token, others 0. Final marking: end has 1 token.

(b) (C B) h Petri net:
- Places: start, C, history_C, B, history_B, end
- Transitions: C, B
- Arcs: start → C, C → history_C, history_C → B, B → history_B, history_B → end
Initial marking: start has 1 token. When C fires, start loses a token, history_C gains a token. Then history_C loses a token, B gains a token, B fires, history_B gains a token, history_B loses a token, end gains a token.

(c) (B B) h Petri net:
- Places: start, B, history_B, B, history_B, end
- Transitions: B, B
- Arcs: start → B, B → history_B, history_B → B, B → history_B, history_B → end
Initial marking: start has 1 token. First B fires: start → history_B, then history_B → B, B → history_B, then history_B → end. Wait, wait, need to check the sequence. For (B B) h, it's B followed by B, so the Petri net would have transitions B, B in series, with history places. So places: start, B1, history_B1, B2, history_B2, end. Transitions: B1, B2. Arcs: start → B1, B1 → history_B1, history_B1 → B2, B2 → history_B2, history_B2 → end. Initial marking: start has 1 token. Firing B1: start → history_B1, then history_B1 → B2, B2 → history_B2, history_B2 → end. So two B transitions with history places.

I need to make sure that each Petri net correctly represents the process. For a sequence, the Petri net should have transitions in series with intermediate places. For a history, adding places to record the history of each transition. So the key elements are places, transitions, and the connections (arcs) between them, with initial and final markings.</think><answer>To construct the Petri nets for the processes (C B), (C B) h, and (B B) h, we analyze their structure using Petri net components (places, transitions, arcs) and initial/final markings:  


### (a) Petri Net for Process (C B)  
A sequence (C B) requires a series of transitions with intermediate places to represent the order.  
- **Places**: `start`, `C`, `B`, `end`  
- **Transitions**: `C`, `B`  
- **Arcs**:  
  - `start → C` (input to `C`), `C → B` (input to `B`), `B → end` (input to `end`)  
- **Initial Marking**: `start` has 1 token; others have 0.  
- **Final Marking**: `end` has 1 token after `B` fires.  

This net shows `C` followed by `B` in series, with tokens moving from `start` → `C` → `B` → `end`.  


### (b) Petri Net for Process (C B) h (History)  
The "h" denotes a history, so we add places to record the occurrence of each transition.  
- **Places**: `start`, `C`, `history_C`, `B`, `history_B`, `end`  
- **Transitions**: `C`, `B`  
- **Arcs**:  
  - `start → C` (input to `C`), `C → history_C` (records `C`), `history_C → B` (input to `B`), `B → history_B` (records `B`), `history_B → end` (input to `end`)  
- **Initial Marking**: `start` has 1 token; others have 0.  
- **Final Marking**: `end` has 1 token after `B` fires, with `history_C` and `history_B` marking the history of `C` and `B`.  

This net tracks the history of transitions, adding intermediate places (`history_C`, `history_B`) to record when `C` and `B` occur.  


### (c) Petri Net for Process (B B) h (History)  
Similar to (C B) h, but with two `B` transitions in sequence.  
- **Places**: `start`, `B1`, `history_B1`, `B2`, `history_B2`, `end`  
- **Transitions**: `B1`, `B2`  
- **Arcs**:  
  - `start → B1` (input to `B1`), `B1 → history_B1` (records `B1`), `history_B1 → B2` (input to `B2`), `B2 → history_B2` (records `B2`), `history_B2 → end` (input to `end`)  
- **Initial Marking**: `start` has 1 token; others have 0.  
- **Final Marking**: `end` has 1 token after `B2` fires, with `history_B1` and `history_B2` marking the history of the two `B` transitions.  

This net shows two `B` transitions in series, with history places to track each `B` occurrence.  


In summary, each Petri net is structured to represent the process sequence, with history processes adding intermediate places to record transition history.  

The answer is <|begin_of_box|>(a) Petri net for (C B): Places = start, C, B, end; Transitions = C, B; Arcs = start→C, C→B, B→end.  
(b) Petri net for (C B) h: Places = start, C, history_C, B, history_B, end; Transitions = C, B; Arcs = start→C, C→history_C, history_C→B, B→history_B, history_B→end.  
(c) Petri net for (B B) h: Places = start, B1, history_B1, B2, history_B2, end; Transitions = B1, B2; Arcs = start→B1, B1→history_B1, history_B1→B2, B2→history_B2, history_B2→end. (Structure as described above)<|end_of_box|></answer>