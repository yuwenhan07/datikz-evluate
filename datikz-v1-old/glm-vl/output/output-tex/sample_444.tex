<think>Got it, let's try to parse this information and structure it. First, the user is talking about using KALE and asymmetric training on the lead to when measuring QPS vs. Recall at 100 on the NQ dataset. Then, they mention that with Asymmetry and KALE, you can get 3x QPS with no loss in accuracy and 4.5x with 1% loss. QPS is defined as mean queries per second with batch size 1, max sequence length 32 on T4 GPU.

First, let's break down the components:

1. Techniques: KALE, asymmetric training, Asymmetry (maybe a typo, but probably another term related to asymmetric training), lead to (maybe "lead to" is a typo, like "lead to" vs. "lead to" but maybe "lead to" is part of the setup, or maybe "lead to" is a misphrase, but let's assume the context is about training methods).

2. Metrics: QPS (Queries Per Second) vs. Recall at 100 (Recall@100) on NQ dataset.

3. Results: 3x QPS with no loss in accuracy, 4.5x QPS with 1% loss in performance.

4. QPS calculation: mean number of queries per second, batch size 1, max sequence length 32, T4 GPU.

So, the question might be asking for an explanation or summary of these results, or maybe a specific part. But since the user provided a statement, maybe they want a summary or elaboration.

Let me structure the thoughts:

- Identify the methods: KALE (maybe a technique for knowledge-aware learning, asymmetric training (maybe training with asymmetric data or architecture), Asymmetry (another asymmetric approach), lead to (maybe "lead to" is a misphrase, like "lead to" as in the result leads to...).

- Metrics: QPS (throughput) vs. Recall@100 (accuracy metric) on NQ (Natural Questions) dataset.

- Results: Using these methods, QPS can be increased by 3x with no accuracy loss, 4.5x with 1% loss.

- QPS calculation specifics: batch size 1, max seq length 32, T4 GPU.

So, the key points are the performance improvement in QPS with controlled accuracy loss, using specific training techniques and hardware.

Now, if the user is asking for a summary or explanation, we can outline:

- The study uses KALE and asymmetric training (with Asymmetry) on the NQ dataset, measuring QPS vs. Recall@100.

- Results show that with these methods, QPS can be increased by 3x without accuracy loss, and 4.5x with a 1% loss in performance.

- QPS is calculated as mean queries per second with batch size 1, max sequence length 32 on a T4 GPU.

So, putting it all together, the answer would summarize these points, explaining the techniques, metrics, results, and QPS calculation.</think><answer>To address the information provided, here’s a structured summary of the key points:  


### 1. Context & Techniques  
The study uses **KALE** (likely a knowledge-aware learning technique) combined with **asymmetric training** (and a related approach called “Asymmetry”) on the *Natural Questions (NQ)* dataset. These methods are evaluated by measuring **Queries Per Second (QPS)** versus **Recall at 100 (Recall@100)** (a metric for accuracy).  


### 2. Results  
Using KALE and asymmetric training:  
- **3× QPS increase** with *no loss in accuracy* (i.e., Recall@100 remains unchanged).  
- **4.5× QPS increase** with *1% loss in performance* (i.e., Recall@100 drops by 1% compared to baseline).  


### 3. QPS Calculation Details  
QPS is defined as the **mean number of queries per second** with the following parameters:  
- Batch size = 1  
- Maximum sequence length = 32  
- Hardware = T4 GPU  


### Summary  
The study demonstrates that applying KALE and asymmetric training to the NQ dataset can significantly boost QPS (3–4.5×) while maintaining or nearly maintaining accuracy (no loss or 1% loss). This improvement is achieved by optimizing the model’s throughput (QPS) without sacrificing too much recall performance, using specific hardware and input configurations.  


If you need further clarification on any part (e.g., the role of “KALE,” asymmetric training specifics, or QPS calculation), feel free to ask!</answer>