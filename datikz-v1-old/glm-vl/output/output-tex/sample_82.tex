<think>Got it, let's try to tackle this problem. First, the question mentions a local picture of _i on the surface and a polygon containing _i, with three equations labeled as coordinates of a curve. But wait, the user's description is a bit vague. Maybe we need to consider that each equation defines a curve, and we need to find something related to the curve, like a local picture (maybe a tangent line or a point on the curve) and a polygon (maybe a polygonal approximation or a region containing the point).

But since the problem is a bit abstract without the actual equations, maybe we need to assume standard forms. Wait, the user says "three different equations on a white background" labeled with letters, maybe like x = f(t), y = g(t), z = h(t) for parametric curves, or Cartesian equations. But without the specific equations, it's hard. Wait, maybe the problem is about a curve defined by one of the equations, and we need to describe the local picture (like the tangent line at a point) and the polygon containing the point (maybe a triangle or quadrilateral enclosing the point on the curve).

Alternatively, if the equations are, say, a parabola, a circle, and a more complex curve like a cubic, then for each, the local picture at a point would be the tangent line, and the polygon could be a triangle or quadrilateral around the point. But since the problem is presented as a question with parts (a) and (b), maybe for each curve (each equation), describe the local picture (tangent line, maybe) and the polygon. But without the specific equations, I need to make an educated guess.

Wait, the original question might be from a textbook where the equations are, for example, the first equation is y = x², the second is x² + y² = 1, and the third is a more complex one like y = x³ - 3x. For part (a), the local picture of a point _i on the surface (maybe the curve) would be the tangent line at that point. For part (b), the polygon containing _i could be a polygon like a square or triangle around the point on the curve.

But since the user hasn't provided the specific equations, maybe the problem is expecting a general approach. Alternatively, if the equations are parametric, say for a curve in 2D, then the local picture (tangent line) can be found using derivatives, and the polygon could be a simple polygon enclosing the point.

Wait, another angle: if the equations are defining a curve, then the local picture at a point _i would be the osculating circle or the tangent line, and the polygon containing _i would be a polygon (like a triangle) with vertices near the curve at that point. But without more info, it's tricky. However, since the user mentions "three different equations" and "coordinates of a curve", maybe each equation is a curve, and for each, we need to describe the local picture (tangent line) and the polygon.

Given that, let's assume a simple case. Suppose the first equation is y = x, the second is y = x², the third is y = x³. For a point _i on the first curve (y=x), say at (a, a), the local picture (tangent line) is y = x, since the derivative is 1. The polygon containing it could be a square with vertices around (a, a). For the second curve y = x² at (a, a²), the tangent line is y = 2a(x - a) + a² = 2a x - a², so the local picture is that tangent line. The polygon could be a rectangle enclosing the point.

But since the problem is likely expecting a specific answer based on standard equations, and given the initial description, maybe the key is to recognize that for each curve, the local picture is the tangent line at the point, and the polygon is a simple polygon (like a triangle or quadrilateral) around the point. However, without the exact equations, it's hard to be precise. But given the user's mention of "three different equations" and "coordinates of a curve", perhaps the answer involves describing the tangent line as the local picture and a polygon (like a triangle) as the containing shape.

Alternatively, if the equations are in 3D, the local picture could be a tangent plane, but the problem mentions "surface" and "polygon", so maybe 2D curves.

Given the lack of specific equations, I'll have to make a general assumption. For part (a), the local picture of a point on the curve is the tangent line at that point, which can be found by differentiating the equation and using the point's coordinates. For part (b), the polygon containing the point could be a simple polygon, like a triangle, with vertices near the curve at that point, ensuring it encloses the point.

But since the user might expect a more concrete answer, and considering common problems, if the equations are, for example, y = x², the local picture (tangent line) at (1,1) is y = 2x - 1, and the polygon could be a square with vertices (0.5, 0.5), (1.5, 0.5), (1.5, 1.5), (0.5, 1.5), enclosing the point (1,1). But without specific info, this is a guess.

Wait, the original question's image description mentions "three different equations on a white background" labeled with letters, indicating coordinates of a curve. Maybe the equations are like x = t, y = t², z = t³ for a 3D curve, but the problem mentions "surface" and "polygon", so maybe 2D curves.

Given the ambiguity, but to provide a structured answer, I'll outline the process:

For part (a), the local picture of a point on the curve is typically the tangent line at that point. To find it, differentiate the curve's equation with respect to the parameter (or x/y) and use the point's coordinates to get the slope, then write the equation of the tangent line.

For part (b), the polygon containing the point could be a simple polygon, such as a triangle or quadrilateral, whose vertices are chosen to enclose the point on the curve, possibly using nearby points on the curve to form the polygon.

Since the problem is likely expecting a description based on standard curve equations, and given the initial equations are simple and complex, the local picture for a simple curve (like a parabola) would be a tangent line, and the polygon could be a triangle enclosing the point.</think><answer>To address the problem, we analyze the curve defined by the equations and describe the local picture and containing polygon:  

### (a) Local Picture of _i on the Surface  
The local picture of a point _i on the curve is typically the **tangent line** at that point. For a curve defined by an equation (e.g., \( y = f(x) \)), the tangent line at a point \((x_0, y_0)\) is found using the derivative \( f'(x_0) \), giving the slope. The equation of the tangent line is \( y - y_0 = f'(x_0)(x - x_0) \). For a parametric curve \( x = x(t), y = y(t) \), the tangent line uses the derivatives \( \frac{dx}{dt} \) and \( \frac{dy}{dt} \), with slope \( \frac{dy/dt}{dx/dt} \) at \( t = t_0 \).  

### (b) Polygon Containing _i  
A polygon containing the point _i on the curve can be a simple polygon (e.g., a triangle or quadrilateral) whose vertices are chosen to enclose the point. For a 2D curve, this might involve selecting nearby points on the curve to form the polygon’s vertices, ensuring the polygon fully contains the point. For example, if the curve is a parabola \( y = x^2 \) at point \((a, a^2)\), a triangle with vertices at \((a-1, a^2-1)\), \((a+1, a^2-1)\), and \((a, a^2+1)\) would enclose the point.  

Without specific equations, the general approach is to use calculus for the tangent line and geometric construction for the polygon.  

The answer is \(<|begin_of_box|>\text{Described as above (tangent line for local picture, enclosing polygon)}<|end_of_box|>\).</answer>