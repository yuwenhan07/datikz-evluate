<think>Got it, let's try to tackle this problem. First, the user is asking about a comparison of execution times between non-restarted and thick-restarted methods for solving the invinterp problem with sparse QR for least squares using 1 MPI process. The operations involved are initial QR factorization, solving the projected problem (eq-J-CS-decomp), and restart updates (eq-restart-a-eq-restart-b). 

First, I need to recall what these methods are. Non-restarted usually means the algorithm doesn't restart, so it might continue iteratively without restarting, while thick-restarted probably restarts with a thicker basis, which might affect the number of iterations or the time per iteration. 

The problem mentions "sparse QR for the least squares" so the QR factorization is sparse, which affects the initial computation time. Then, each iteration involves solving the projected problem and updating bases if restarting. 

To compare execution times, we need to consider the total time for each method: initial QR, then iterations (for non-restarted, maybe more iterations, for thick-restarted, maybe fewer iterations but with restarts). 

Let's break down the components:

1. Initial QR factorization: This is a one-time cost. For non-restarted, you do this once, then maybe more iterations. For thick-restarted, you might do the initial QR, then restarts involve re-doing part of the QR? Or maybe the restarts update the basis, which might require re-factorizing or updating the QR. 

2. Projected problem solution (eq-J-CS-decomp): Each iteration requires solving this, so the time per iteration here depends on the size of the projected system. If thick-restarted uses a larger basis, maybe the projected system is bigger, but with better conditioning, leading to faster solves? Or maybe the number of iterations is reduced, so total time could be lower. 

3. Restart updates (eq-restart-a-eq-restart-b): These are done at restarts. So non-restarted doesn't do this (or does it? Wait, the problem says "required at restart (and at the end of the algorithm)". So if non-restarted doesn't restart, then it doesn't do these updates. Wait, the question says "non-restarted and thick-restarted methods when solving the invinterp problem". So non-restarted might be a method that doesn't restart, so it continues until convergence, while thick-restarted restarts periodically, updating the bases. 

So for non-restarted: total time = initial QR + sum over iterations of (projected solve + any updates, but since no restart, no updates except maybe at end? Wait, the problem says "required at restart (and at the end of the algorithm)". So if non-restarted doesn't restart, then restart updates aren't done, so total time is initial QR + iterations * (projected solve time). 

For thick-restarted: total time = initial QR + sum over restarts of (restart updates time + iterations between restarts * (projected solve time)). So if thick-restarted has fewer iterations between restarts but with restart updates, the total time could be lower if the restart updates are cheaper than the extra iterations. 

But the problem is about comparing execution times. So we need to consider that thick-restarted might have a higher initial cost for restart updates but fewer iterations, leading to lower total time. Or maybe the opposite. 

Also, since it's 1 MPI process, the parallelism isn't a factor here, so it's a sequential comparison. 

Another angle: the invinterp problem's structure—maybe the non-restarted method has more iterations, each with a projected solve, while thick-restarted has fewer iterations but with restarts that update the basis, which might make each iteration's projected solve faster (since the basis is updated, the system is better conditioned), leading to lower total time. 

But without specific data, we need to reason based on method characteristics. Thick restarting typically aims to reduce the number of iterations by updating the basis, which can lead to faster convergence, hence lower total execution time compared to non-restarted (which might take more iterations). However, the restart updates add some overhead, so the total time depends on the balance between the number of iterations and the cost of restarts. 

If the restarts are infrequent, the overhead might be small, and the reduced iterations lead to lower total time. If restarts are frequent, the overhead might make thick-restarted slower. But generally, thick restarting is designed to be more efficient. 

So, in summary, the thick-restarted method likely has a lower total execution time than the non-restarted method because it reduces the number of iterations through restarts, which can offset the cost of the restart updates, especially if the problem benefits from the updated basis leading to faster projected solves.</think><answer>To compare the execution times of the non-restarted and thick-restarted methods for solving the invinterp problem with sparse QR for least squares using 1 MPI process, we analyze the key components of each method:  


### 1. **Method Components**  
- **Non-restarted method**: Continues iteratively without restarting. Operations include:  
  - One-time initial sparse QR factorization.  
  - Iterative solution of the projected problem (`eq-J-CS-decomp`) for each iteration.  
  - No restart updates (`eq-restart-a-eq-restart-b`) since no restarts occur.  

- **Thick-restarted method**: Periodically restarts to update the basis. Operations include:  
  - One-time initial sparse QR factorization.  
  - Iterative solution of the projected problem for each iteration *between* restarts.  
  - Restart updates (`eq-restart-a-eq-restart-b`) at each restart (and at the end of the algorithm).  


### 2. **Execution Time Analysis**  
- **Initial QR Factorization**: Both methods share the same initial cost for the sparse QR factorization, as this is a one-time operation.  
- **Projected Problem Solves**: The non-restarted method may require more iterations to converge, leading to a higher total time for solving the projected problem repeatedly. The thick-restarted method, by updating the basis at restarts, can improve the conditioning of the projected system, potentially reducing the time per iteration or the total number of iterations.  
- **Restart Updates**: While restart updates add overhead, thick restarting is designed to offset this by reducing the total number of iterations. If the problem benefits from the updated basis (e.g., better conditioning, faster convergence), the reduced iteration count often outweighs the cost of restart updates.  


### 3. **Conclusion**  
The **thick-restarted method** typically has a lower total execution time than the **non-restarted method** because restarts reduce the number of iterations, which can offset the overhead of basis updates. The non-restarted method may require more iterations without the benefits of basis updates, leading to longer total execution time.  

The answer is <|begin_of_box|>thick-restarted method has lower execution time<|end_of_box|>.</answer>