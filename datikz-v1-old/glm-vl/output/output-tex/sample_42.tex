<think>Got it, let's try to parse this problem step by step. First, the question is about an element A defined as M₁ + M₄ - X in some figure, probably a diagram related to cohomology or some algebraic structure. Then, it mentions that A(3) = A(6) = -x₂ by Figure fig_0-cochain_presentation. Also, A(1), A(2), A(4), A(5) might not be zero in H²(BT³), but the cohomology classes with support in {3,6} for these indices are zero. 

First, let's recall some concepts. H²(BT³) is the second cohomology group of the space BT³, which is a classifying space for the group T³ (the direct product of three tori, maybe? Or maybe the 3-torus, but BT³ is the classifying space for T³). The element A is a cohomology class, maybe a 2-cochain, and we're looking at its values on different cells or simplices labeled 1 to 6. 

The notation A(3) = -x₂ means that when we evaluate the cohomology class A on the 3-cell (or a 0-cochain corresponding to the 3-cell), it gives -x₂, which is an element of H²(BT³), probably a generator or a specific class. Similarly for A(6). Then, for the indices 1,2,4,5, the classes A(1), etc., might not be zero, but when we restrict to the subset {3,6}, the cohomology classes _{3,6}(1), _{3,6}(2), etc., are zero. The subscript {3,6} probably denotes the restriction to the subcomplex or subspace consisting of cells 3 and 6, and the cohomology there has those restrictions.

So, putting it all together: A is a 2-cohomology class defined by M₁ + M₄ - X, and when evaluated on cells 3 and 6, it gives -x₂ each. The other cells (1,2,4,5) might have non-zero cohomology classes, but when we restrict those classes to the subcomplex involving only cells 3 and 6, they vanish. This might be related to some exact sequence or a presentation of the cohomology group, where the relations imposed by the restriction to {3,6} force those classes to be zero in that subcomplex.

I need to make sure I understand the notation: _{3,6}(k) probably means the cohomology class of the k-th element (maybe a 0-cochain) when restricted to the subcomplex spanned by cells 3 and 6. If those restrictions are zero, then the classes 1,2,4,5 are in the kernel of the restriction map to {3,6}, which might imply they are in the image of some map, hence the cohomology might have relations.

Also, the mention of Figure fig_rel5 and fig_0-cochain_presentation suggests that this is part of a larger diagram or presentation of the cohomology ring, where certain relations hold. The key points are the evaluation of A on cells 3 and 6, the possible non-vanishing on other cells, and the vanishing of their restrictions to the {3,6} subcomplex.

So, summarizing the information: Element A is defined as M₁ + M₄ - X. Its values on cells 3 and 6 are both -x₂. The classes A(1), A(2), A(4), A(5) might not be zero in H²(BT³), but when restricted to the subcomplex containing cells 3 and 6, those restrictions are zero. This likely means that in the cohomology ring, the classes 1,2,4,5 are related to A and the relations imposed by the subcomplex {3,6} make their restrictions vanish, hence they might be in the image of some map or satisfy certain relations with A.</think><answer>The element \( A = M_1 + M_4 - X \) is a cohomology class defined in the context of \( H^2(BT^3) \), where \( BT^3 \) is the classifying space for the 3-torus. By Figure fig_0-cochain_presentation, evaluating \( A \) on cells 3 and 6 yields \( A(3) = A(6) = -x_2 \), indicating these cells are mapped to the generator \( -x_2 \) under \( A \). For cells 1, 2, 4, 5, \( A(1), A(2), A(4), A(5) \) may not vanish in \( H^2(BT^3) \), but their restrictions to the subcomplex spanned by cells 3 and 6 (denoted \( _{3,6}(k) \)) are zero. This means these classes are in the kernel of the restriction map to the subcomplex {3,6}, implying they satisfy relations that vanish when restricted to this subcomplex.  

The answer is \(<|begin_of_box|>-x_2<|end_of_box|>\).</answer>