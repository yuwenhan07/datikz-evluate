<think>Got it, let's try to understand the question first. The user is asking about a schematic of possible NP contributions to S=1 and C=1 FCNC semileptonic processes in the two-generation limit of the Standard Model (SM). They mention CP-conserving magnitudes (|z_C=1|, |z_S=1|) depending on the alignment angle _d, and CPV contributions (c_I) being independent of _d. 

First, I need to recall what FCNC ( Flavor Changing Neutral Currents) processes are. FCNC processes are interactions that change the flavor of quarks without changing the charge, which are usually highly suppressed in the SM due to GIM mechanism, but NP can enhance them. Semileptonic processes involve a quark changing flavor, emitting a lepton pair (like a quark turning into another quark and a lepton-antilepton pair). 

S=1 and C=1 refer to the strangeness and charm quantum numbers, maybe? Or perhaps spin and charge? Wait, S=1 would be strangeness changing, C=1 charm changing. FCNC processes with S=1 or C=1 would be those that change strangeness or charm. 

In the two-generation limit, we're probably considering up and down generations, ignoring the top and bottom (or maybe up and down, but need to clarify). The alignment angle _d is likely related to the mixing in the quark sector, maybe the CKM matrix, but in the two-generation limit, the CKM matrix is simplified, and the angle _d could be a parameter related to the mixing between the first two generations. 

CP-conserving NP contributions would have magnitudes |z_C=1|, |z_S=1| depending on _d, meaning their size changes with the alignment angle. CP-violating (CPV) contributions c_I are independent of _d, so their size doesn't depend on the mixing angle. 

To explain this, I should outline the structure: NP contributions can come from new particles (like heavy vector bosons, scalars, etc.) that mediate FCNC. For S=1 and C=1 processes, the NP couplings would have different dependences. The CP-conserving parts are proportional to some functions of _d, while CPV parts are constants (independent of _d). 

Maybe the schematic would show different diagrams: SM diagrams (suppressed) and NP diagrams with new particles, with coefficients z_C=1, z_S=1 for CP-conserving, and c_I for CPV, with the dependence on _d for the first two. 

I need to make sure to connect the terms: FCNC semileptonic processes are processes like b → sℓν, t → cℓν, etc., but in two-generation limit, maybe focusing on first two generations, so b → sℓν would be S=0 (strangeness changes by 1, since b has S=-1/2, s has S=-1/2, so maybe not; wait, S is the sum of quark strangeness. b quark has S=-1/3, s has S=-1/3, so b → s would be S=0. Maybe S=1 refers to processes like c → uℓν (charm changes by 1, c has C=+1/3, u has C=+2/3, so C changes by -1/3, maybe not. Wait, C is charge conjugation eigenvalue. Maybe S=1 is strangeness changing by 1, like s → uℓν (strangeness changes by 1, since s has S=-1, u has S=0), and C=1 would be charm changing by 1, like c → sℓν (charm changes by -2/3, maybe not. Hmm, maybe need to check standard notation. Alternatively, S=1 could mean the process has a change in strangeness of 1, and C=1 change in charm. FCNC processes with S=1 and C=1 would be those where both strangeness and charm change by 1, or one of them. But maybe the key is that the NP contributions have different dependences on the alignment angle. 

In summary, the schematic would show that for CP-conserving NP contributions to S=1 and C=1 FCNC semileptonic processes, their magnitudes |z_C=1| and |z_S=1| depend on the alignment angle _d (which is a parameter in the quark mixing, like the angle in the 2x2 CKM matrix), while the CP-violating NP contributions c_I are independent of _d. This means that the CP-conserving parts are sensitive to the quark mixing angle, while CPV parts are not, which could be important for experimental searches or theoretical predictions. 

I should structure the answer by explaining the context of FCNC, semileptonic processes, S=1/C=1, NP contributions, alignment angle, and the dependence of CP-conserving vs CPV contributions. Maybe mention that in the two-generation limit, the CKM matrix is simplified, and _d is a key parameter for the mixing between the first two generations, so NP effects depend on how the new physics aligns with this mixing.</think><answer>To address the question about the schematic of NP contributions to \( S=1 \) and \( C=1 \) FCNC semileptonic processes in the two-generation Standard Model (SM) limit, here’s a structured explanation:  


### 1. **Context of FCNC and Semileptonic Processes**  
FCNC (Flavor Changing Neutral Currents) processes involve quark flavor changes without charge change, which are highly suppressed in the SM due to the GIM mechanism. Semileptonic FCNC processes (e.g., \( b \to s\ell\nu \), \( c \to u\ell\nu \)) are key for probing NP because they are sensitive to new physics (NP) contributions. \( S=1 \) and \( C=1 \) refer to **strangeness** (\( S \)) and **charm** (\( C \)) quantum number changes:  
- \( S=1 \): Processes with strangeness changing by 1 (e.g., \( s \to u\ell\nu \), where \( s \) has \( S=-1 \), \( u \) has \( S=0 \)).  
- \( C=1 \): Processes with charm changing by 1 (e.g., \( c \to s\ell\nu \), where \( c \) has \( C=+1/3 \), \( s \) has \( C=-1/3 \), so charm changes by \( -2/3 \); alternatively, \( b \to c\ell\nu \) for charm change, but notation may vary).  


### 2. **NP Contributions in the Two-Generation Limit**  
In the two-generation limit (ignoring heavier generations like top/bottom), the SM is simplified, and quark mixing is described by a 2×2 matrix. The **alignment angle \( \phi_d \)** (or \( \theta_d \)) parameterizes the mixing between the first two generations (up/down or down/strange sectors), analogous to the CKM matrix’s Cabibbo angle.  

NP contributions to FCNC processes can come from new particles (e.g., heavy vector bosons, scalars, or fermions) that mediate flavor-changing interactions. These contributions are split into:  
- **CP-conserving (CP-even) magnitudes**: \( |z_{C=1}|, |z_{S=1}| \), which depend on \( \phi_d \).  
- **CP-violating (CP-odd) contributions**: \( c_I \), which are independent of \( \phi_d \).  


### 3. **Dependence on \( \phi_d \) and CP Properties**  
- **CP-conserving contributions** (\( |z_{C=1}|, |z_{S=1}| \)): Their magnitudes scale with the alignment angle \( \phi_d \), meaning the size of NP effects depends on how the new physics “aligns” with the quark mixing. For example, if NP couples to the first two generations, the effective coupling strength for FCNC processes varies as \( \cos(\phi_d) \) or \( \sin(\phi_d) \), depending on the NP model.  
- **CP-violating contributions** (\( c_I \)): These are independent of \( \phi_d \), so their size is fixed (or varies with other CP-violating phases, not the mixing angle). This makes CPV effects more robust to changes in quark mixing, while CP-conserving effects are sensitive to \( \phi_d \).  


### 4. **Schematic Structure (Conceptual)**  
A schematic would typically show:  
- **SM diagrams** (suppressed by GIM, shown as small contributions).  
- **NP diagrams** with new particles (e.g., a heavy vector boson \( X \) mediating FCNC), with coefficients \( z_{C=1}, z_{S=1} \) (for CP-conserving) and \( c_I \) (for CPV), labeled with their dependence on \( \phi_d \).  
- Labels indicating which processes (e.g., \( b \to s\ell\nu \) for \( S=1 \), \( c \to s\ell\nu \) for \( C=1 \)) are affected, with arrows showing quark flavor changes and lepton emission.  


### Summary  
In the two-generation SM limit, NP contributions to \( S=1 \) and \( C=1 \) FCNC semileptonic processes have CP-conserving magnitudes (\( |z_{C=1}|, |z_{S=1}| \)) that depend on the alignment angle \( \phi_d \), while CP-violating contributions (\( c_I \)) are independent of \( \phi_d \). This distinction is crucial for interpreting experimental data, as it allows separating NP effects based on their sensitivity to quark mixing.  

The answer is \(<|begin_of_box|>|z_{C=1}|, |z_{S=1}| \text{ depend on } \phi_d; c_I \text{ are independent of } \phi_d (text for details as needed)<|end_of_box|>\).</answer>