<think>Got it, let's try to figure out the E₂-page of the Atiyah-Hirzebruch spectral sequence (AHSS) as described. First, I need to recall what the AHSS is. The Atiyah-Hirzebruch spectral sequence is a tool in algebraic topology that relates the homology of a space with coefficients in a spectrum to the homology of the space with coefficients in a ring. The E₂-page is the first page of the spectral sequence, which is a bigraded module.

The general form of the E₂-page for AHSS is that the differentials go from degree (p, q) to (p, q + r) where r is the dimension of the spectrum. Wait, the AHSS is for a spectrum E, and the coefficients are in E_0 = Z. The E₂ term is given by the homology of the space with coefficients in the spectrum, but maybe more precisely, the E₂ page is the homology of the product space with the spectrum, or something like that. Wait, let's recall the setup: for a spectrum E, the AHSS for the space X is the spectral sequence converging to the homology of X with coefficients in E, and the E₂ page is E₂^{p,q} = H_q(X; E_p), where E_p is the p-th term of the spectrum. Wait, maybe I need to check the grading. The bidegree is (p, q), where p is the filtration degree (related to the spectrum) and q is the degree in the homology. The differentials are d_r: E_r^{p,q} → E_r^{p, q + r}, where r is the dimension of the spectrum. For the AHSS, the differentials are of the form d_r: E_r^{p,q} → E_r^{p, q + r}, with r being the dimension of the spectrum. Wait, the AHSS is a special case of the Adams spectral sequence, but for a spectrum. The E₂ page is the homology of the space with coefficients in the spectrum, but maybe more precisely, the E₂ term is the homology of the product space with the spectrum. Wait, let's think of the AHSS as starting with the E₂ page given by E₂^{p,q} = H_q(X; E_p), where E_p is the p-th term of the spectrum. The differentials then are induced by the structure of the spectrum. For example, if the spectrum is the sphere spectrum, then E_p = S^p, and the AHSS for a space X would have E₂^{p,q} = H_q(X; Z/p), but wait, no, the coefficients are in the spectrum, so maybe E_p is the p-th space in the spectrum. Wait, perhaps I need to consider the general case. The AHSS is defined for a spectrum E, and the E₂ page is the homology of the space with coefficients in the spectrum, but the exact formula is that E₂^{p,q} = H_q(X; E_p), where E_p is the p-th term of the spectrum. The differentials d_r: E_r^{p,q} → E_r^{p, q + r} are induced by the multiplication maps in the spectrum. For example, if the spectrum is the Eilenberg-MacLane spectrum, then E_p = K(Z, p), and the AHSS would relate the homology of X with coefficients in K(Z, p) to the homology of X with coefficients in Z. But the question mentions a black and white picture with a pattern of squares, which probably represents the bidegree structure of the E₂ page. The E₂ page is a bigraded module, so it's a double complex, and the picture might show the non-zero terms in the bidegree (p, q), with p being the filtration degree (from the spectrum) and q being the homological degree. The differentials go from (p, q) to (p, q + r), where r is the dimension of the spectrum. So, for example, if the spectrum is of finite type, say E = Σ^k S^0, then the differentials are of length k. The E₂ page would have non-zero terms in bidegree (p, q) where p is the filtration degree and q is the homological degree, and the differentials connect these terms. The image with squares might show the positions of non-zero E₂ terms, with each square representing a non-zero entry, and the differentials as arrows between them. To describe the E₂-page, one would need to specify the bidegree of the non-zero terms and the differentials. For example, if the space X is a sphere S^n, then the AHSS would have E₂^{p,q} = 0 unless p + q = n, by the Hurewicz theorem, but maybe for a more general space, the E₂ page has non-zero terms in various bidegrees. However, since the question refers to "the image is a black and white picture featuring a pattern of squares," it's likely that the E₂ page has a grid of squares, with some squares colored (representing non-zero terms) and arrows between them (representing differentials). The key points are that the E₂ page is a bigraded module with bidegree (p, q), differentials of the form d_r: (p, q) → (p, q + r), and the picture shows the structure of these non-zero terms and differentials. Without the actual image, but based on standard AHSS, the E₂ page has terms E₂^{p,q} = H_q(X; E_p), and the differentials are determined by the multiplication maps in the spectrum. So, if the image shows a grid where each square (p, q) has a non-zero entry, and there are arrows from (p, q) to (p, q + r) for each r, then that's the structure. The specific pattern would depend on the space X and the spectrum E, but the general form is a double complex with differentials of a certain length.</think><answer>The E₂-page of the Atiyah-Hirzebruch spectral sequence (AHSS) is a bigraded module that encodes the homology of a space \( X \) with coefficients in a spectrum \( E \). For a spectrum \( E \), the E₂-page is given by \( E_2^{p,q} = H_q(X; E_p) \), where \( p \) is the filtration degree (related to the spectrum's structure) and \( q \) is the homological degree. Differentials \( d_r: E_r^{p,q} \to E_r^{p, q+r} \) (for \( r \) equal to the spectrum's dimension) connect these terms, typically of the form \( d_r: (p, q) \to (p, q+r) \).  

In a black-and-white image with a square pattern, the E₂-page is represented as a grid of squares, where each square corresponds to a non-zero term \( E_2^{p,q} \), and arrows (differentials) connect squares in a pattern reflecting the differential maps. The structure shows the bidegree of non-zero entries and the direction of differentials, which are crucial for computing the spectral sequence's convergence to the final homology groups.  

The answer is \(<|begin_of_box|>E_2^{p,q} = H_q(X; E_p)<|end_of_box|>\).</answer>