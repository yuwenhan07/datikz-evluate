\begin{tikzpicture}
    % Define coordinates for the x-axis (strain values) and y-axis (strain values)
    % Let's say x from 0 to 10, y from 0 to 1 (normalized)
    \draw[->] (0,0) -- (10,0) node[right] {Strain};
    \draw[->] (0,0) -- (0,1) node[above] {Strain Value};
    % Draw colored blocks (maybe for different materials)
    % For example, material A: fill from (0,0) to (5,0.2) to (5,0.5) to (0,0.5) -- fill=red;
    % Then draw the lines for each strain type
    % Unloaded: a flat line, tension: increasing, bending: parabolic, shear: linear
    % Let's define coordinates for each line
    % Unloaded: (0,0.3) -- (10,0.3)
    % Tension: (0,0.1) -- (10,0.6)
    % Bending: (0,0.2) -- (5,0.4) -- (10,0.2)
    % Shear: (0,0.25) -- (10,0.35)
    % Then fill the area under each line with different colors
    % Maybe use a loop to draw each material's blocks
    % But for simplicity, let's do one material first
    % Then repeat for six figures, each with different materials or different strain curves
\end{tikzpicture}