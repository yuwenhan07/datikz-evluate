<think>Got it, let's try to figure out what the user is asking. The question mentions "STG superposed over rectangular domains dividing phase space for the toggle switch with the DSGRN parameter in eq:toggle-switch-param" and the corresponding MG (which I think is Markov Generator or maybe a specific model, but probably a model related to the toggle switch). 

First, I need to recall what STG (Stochastic Timed Graph) and MG (maybe Markov Generator or another model) are in the context of toggle switches. A toggle switch is a system with two stable states, often modeled by a bistable system, maybe with two attractors. The DSGRN (Dynamic Systems Graphical Representation Network) parameter would be parameters defining the system, like rates of transitions between states.

STG is a stochastic timed graph, which models the dynamics with transitions between states with certain rates, and the rectangular domains dividing phase space—so phase space here is probably the state space of the toggle switch, divided into regions (rectangles) that correspond to different regions of the state space, maybe regions where the system is in one of the stable states or in a transition region.

The left image is the STG superposed over these rectangular domains. So the STG would show the transitions between states, and the rectangles are the regions of phase space. The right image is the corresponding MG, which might be a simplified model, maybe a Markov model (since MG could stand for Markov Generator) with states corresponding to the regions or the states of the toggle switch.

To explain, I need to describe what each part represents. The toggle switch has two stable states, say "on" and "off", with transitions between them. The rectangular domains divide the phase space into regions: maybe the "on" region, "off" region, and transition regions. The STG would have nodes representing states (maybe the regions or the specific states) with edges representing transitions, with rates. The MG might be a simplified version, maybe a two-state Markov model, but since it's "corresponding", it should reflect the same dynamics but maybe with fewer states or a different representation.

So, step by step: 1. Understand the toggle switch model—bistable, two stable states, transitions between them. 2. Phase space division into rectangular domains—these are regions where the system is in a certain state or between states. 3. STG (Stochastic Timed Graph) superposed over these domains—so the graph shows the transitions between the regions (or states) with their rates, and the rectangles are the regions. 4. MG (Markov Generator, maybe) is the corresponding model, which might be a simplified Markov model with states corresponding to the regions or the toggle switch's states, showing the transition probabilities.

I need to make sure the explanation connects the STG and MG to the toggle switch's phase space and the DSGRN parameters. The DSGRN parameter in eq:toggle-switch-param would define the rates or parameters of the toggle switch, which in turn define the STG's transitions and the MG's structure.

So, summarizing: The left image shows the STG (stochastic timed graph) with its transitions between states, overlaid on rectangular regions that divide the phase space of the toggle switch. These rectangles represent different regions of the state space (e.g., stable "on" and "off" regions, and transition regions). The right image is the corresponding MG (Markov Generator, or another model) that captures the same dynamics, likely with a simplified structure reflecting the toggle switch's bistable behavior and transitions between the defined regions. The DSGRN parameters in the toggle switch model determine the rates of transitions in the STG and the transition probabilities in the MG, defining how the system moves between the states/regions.</think><answer>To address the question, let’s break down the components and their relationships:  


### 1. Context of the Toggle Switch Model  
A “toggle switch” is a bistable dynamical system with two stable states (e.g., “on” and “off”) and transitions between them. It is often modeled with parameters (e.g., rates of state transitions) that define its behavior. The “DSGRN parameter in ~eq:toggle-switch-param” refers to specific parameters (like transition rates, potential barriers, etc.) that define the toggle switch’s dynamics.  


### 2. STG (Stochastic Timed Graph) on the Left  
- **STG** is a stochastic timed graph, which models the system’s dynamics with nodes (states) and edges (transitions) between them, where transitions have associated rates.  
- **Superposed over rectangular domains dividing phase space**: The “rectangular domains” divide the system’s *phase space* (the space of all possible states) into regions. For a toggle switch, these regions might correspond to:  
  - Stable “on” and “off” states (large regions).  
  - Transition regions (smaller regions between the stable states).  
- The STG’s nodes and edges represent transitions between these regions, with rates determined by the DSGRN parameters.  


### 3. MG (Markov Generator, or Corresponding Model) on the Right  
- **MG** (likely a Markov Generator or a simplified Markov model) is a model that captures the same dynamics as the STG but with a simplified structure.  
- It typically represents the toggle switch’s behavior with a reduced number of states (matching the rectangular domains or stable states) and transition probabilities. The “corresponding” nature means the MG’s structure reflects the STG’s transitions and the toggle switch’s bistable behavior.  


### 4. Relationship Between STG, MG, and the Toggle Switch  
- The DSGRN parameters define the toggle switch’s transition rates, which in turn define the STG’s edge weights (transition probabilities) and the MG’s transition probabilities.  
- The rectangular domains in the STG’s phase space are used to structure the MG’s states (e.g., grouping the phase space into “on,” “off,” or “transition” states), simplifying the model while preserving the toggle switch’s core dynamics.  


In summary, the left image shows the STG (with transitions between phase-space regions) for the toggle switch, and the right image shows a simplified MG that captures the same dynamics, both defined by the DSGRN parameters.  

The answer is <|begin_of_box|>The left image shows the STG (stochastic timed graph) superposed over rectangular phase-space domains for the toggle switch, with transitions defined by DSGRN parameters. The right image shows the corresponding MG (Markov Generator) that simplifies the toggle switch’s dynamics while preserving its bistable behavior.<|end_of_box|></answer>