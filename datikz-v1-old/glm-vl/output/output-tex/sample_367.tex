<think>Got it, let's try to understand the statement about the moduli space of type IIB theory being H/SL(2,Z) with infinite distance limit and cusp singularities. First, I need to recall what the moduli space of type IIB string theory is. 

Type IIB string theory is a supersymmetric string theory, and its moduli space describes the possible vacuum configurations, which are related to the complex structure of the Calabi-Yau three-fold (if we're considering compactification) or the complex structure of the worldsheet, but maybe more generally, the moduli space of the theory includes parameters like the complex structure of the manifold, dilaton, etc. 

The group H here is probably the modular group, which is SL(2,Z), but wait, the statement says H/SL(2,Z), so maybe H is a larger group, and SL(2,Z) is a subgroup acting on it. Wait, the moduli space of type IIB is often related to the quotient of the upper half-plane by SL(2,Z), but maybe with some modifications. Alternatively, the moduli space of the type IIB theory (in the context of compactifications) might be related to the space of complex structures on the Calabi-Yau, which can be a quotient of the upper half-plane by a discrete group, and SL(2,Z) is the modular group acting there. 

Now, the moduli space having an infinite distance limit and cusp singularities. The upper half-plane (the usual moduli space for some theories) has cusps when the group action has fixed points, and the infinite distance limit would be the point at infinity in the upper half-plane, which is a cusp. So the quotient space H/SL(2,Z) would have cusps where the stabilizers of points in H are non-trivial, and the infinite limit is the point corresponding to the cusp. 

Wait, let's think step by step. The moduli space of type IIB string theory, when considering the complex structure moduli, is often a complex manifold, and its topology can have cusps when the group action has fixed points, leading to singularities at the cusps. The infinite distance limit would be a point in the moduli space that can be approached by sequences of points with increasing distance, which in the case of a quotient by a discrete group acting on a space with a metric, the cusp would be the point at infinity, and the infinite distance limit is the limit of points going off to infinity in the quotient space. 

So putting it all together: the moduli space is the quotient of some group H by SL(2,Z), which is a discrete group action, leading to a space with cusps (singularities) and an infinite distance limit (the point at infinity in the quotient). This makes sense because SL(2,Z) acts on the upper half-plane, and the quotient is the modular curve, which has cusps, and the point at infinity is a cusp, and sequences can go to infinity, hence infinite distance limit. 

I need to make sure about the group H. Maybe H is the group of isometries of the hyperbolic plane, but SL(2,Z) is a subgroup. Alternatively, in the context of type IIB, the moduli space might be related to the space of complex structures on a Calabi-Yau, which can be a quotient of the upper half-plane by a subgroup of SL(2,Z), and the cusps correspond to the fixed points of the group action, which are the cusps in the quotient space. The infinite distance limit would be the limit of points as they go to infinity in the hyperbolic metric, which in the quotient space is a cusp. 

So the conclusion is that the moduli space is a quotient of H by SL(2,Z), has cusps (singularities) and an infinite distance limit, which are typical features of such quotient spaces with discrete group actions, especially when the group has non-trivial stabilizers at certain points (the cusps) and the space is hyperbolic.</think><answer>The moduli space of type IIB string theory is described as \( H/\text{SL}(2, \mathbb{Z}) \), where \( H \) is a group (often the upper half-plane or a related space) and \( \text{SL}(2, \mathbb{Z}) \) is the modular group acting on \( H \). This quotient space exhibits two key features: **infinite distance limits** and **cusp singularities**.  

- **Infinite distance limit**: In the hyperbolic metric on \( H \), sequences of points can approach infinity, and their images in the quotient \( H/\text{SL}(2, \mathbb{Z}) \) converge to a single point at infinity, representing an "infinite distance limit" in the moduli space.  
- **Cusp singularities**: The discrete group \( \text{SL}(2, \mathbb{Z}) \) has non-trivial stabilizers at certain points in \( H \), leading to fixed points in the quotient space. These fixed points manifest as cusps, which are singularities in the moduli space.  

This structure is typical of quotient spaces of hyperbolic manifolds by discrete groups, where cusps and infinite limits arise from the group action's fixed points and the hyperbolic geometry of the base space.  

The answer is \(<|begin_of_box|>H/\text{SL}(2,\mathbb{Z})<|end_of_box|>\).</answer>