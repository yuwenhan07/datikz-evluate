<think>Got it, let's try to figure out what the question is asking. The user mentioned "relative 2-neighborhoods" of square grid, triangular grid, and king grid. First, I need to recall what a 2-neighborhood is. In grid graphs, the k-neighborhood usually includes all vertices at distance ≤k from a given vertex. So a 2-neighborhood would be vertices at distance 1 or 2. 

First, let's think about each grid:

1. Square grid: Each vertex has neighbors in the four cardinal directions (up, down, left, right). The 2-neighborhood would include those vertices at distance 1 (direct neighbors) and distance 2. For a square grid, the 2-neighborhood around a vertex would include the four vertices at distance 1 (adjacent) and the four vertices at distance 2 (like diagonally adjacent ones, but wait, in square grid, distance 2 would be two steps in one direction, or one step in two perpendicular directions). Wait, maybe better to visualize: a vertex in square grid has neighbors at (±1,0), (0,±1). The 2-neighborhood would include vertices at (±2,0), (0,±2), (±1,±1) – those are the vertices at distance 1 or 2. So the 2-neighborhood shape would be a cross with arms extending two steps, maybe a plus sign with longer arms.

2. Triangular grid: This is a grid where each vertex is connected to its neighbors in the triangular lattice, which has six neighbors (in a hexagonal arrangement). The 2-neighborhood here would include vertices at distance 1 (the immediate neighbors) and distance 2. In a triangular grid, distance 2 would be vertices two steps away, which form a hexagonal shape around the original vertex. So the 2-neighborhood might look like a hexagon with smaller hexagons inside, or a larger hexagon.

3. King grid: This is like a chess king's move, so each vertex is connected to those adjacent in the four cardinal directions and the four diagonal directions. So each vertex has 8 neighbors. The 2-neighborhood here would include vertices at distance 1 (8 neighbors) and distance 2. For distance 2, it would include vertices two steps away in any direction, which would form a 3x3 square minus the center, but with the center's neighbors. Wait, the king's 2-neighborhood would include all vertices within two moves, so the shape would be a larger square, maybe a 3x3 grid of vertices, excluding the center's immediate neighbors? Wait, let's think: if you start at a vertex, distance 1 is the 8 neighbors, distance 2 is the vertices two steps away, which would form a 3x3 grid of vertices (9 vertices) minus the center one, so 8 vertices? Wait, maybe the shape is a square with side length 3, but the center is the original vertex, so the 2-neighborhood would include all vertices within two steps, forming a larger square.

Now, the image description mentions three shapes, each being the 2-neighborhood of their respective grids. So the square grid's 2-neighborhood would be a shape like a plus sign with longer arms (the cross has length 3, maybe), the triangular grid's would be a hexagon, and the king grid's would be a larger square (like a 3x3 grid of vertices, forming a square shape). 

But the user also says the image is a black and white picture with three different shapes, possibly tiles or designs, arranged in a row. Each shape is unique, geometric pattern, simple drawing, various sizes. So maybe each shape corresponds to one of the grids' 2-neighborhoods. 

To confirm, let's recall that the 2-neighborhood of a square grid is a set of points forming a square with its diagonals extended, maybe a plus sign with longer arms (the cross has arms of length 2, so total size 3x3). The triangular grid's 2-neighborhood is a hexagon, which is a six-sided shape. The king grid's 2-neighborhood is a square (like a chessboard's 2-square neighborhood), which is a larger square shape. 

So the three shapes would be: for square grid, a cross (plus sign) with longer arms; for triangular grid, a hexagon; for king grid, a larger square. Each is a geometric pattern, simple drawing, arranged in a row, different sizes. 

But the question is probably asking for the description of each 2-neighborhood shape. Since the user provided an image description, maybe the answer is identifying each grid's 2-neighborhood shape. 

So, summarizing:

- Square grid 2-neighborhood: forms a cross (plus sign) with arms extending two units in each cardinal direction, creating a shape with a central point and four arms of length 2, total size 3x3.

- Triangular grid 2-neighborhood: forms a hexagonal shape, as the triangular lattice's 2-neighborhood around a vertex includes vertices forming a larger hexagon.

- King grid 2-neighborhood: forms a larger square, as the king's move includes all adjacent and diagonal neighbors, so the 2-neighborhood forms a 3x3 square of vertices (excluding the center if considering distance 2, but actually, king's 2-neighborhood would include all vertices within two moves, forming a square of side length 3).

But since the image has three shapes, each corresponding to one grid, the answer would describe each shape as the 2-neighborhood of their respective grid.</think><answer>To determine the relative 2-neighborhoods of the square grid, triangular grid, and king grid, we analyze each grid’s structure and neighborhood definition (vertices at distance ≤ 2 from a central vertex):  

1. **Square Grid 2-Neighborhood**:  
   In a square grid, each vertex has 4 cardinal neighbors (up, down, left, right). The 2-neighborhood includes these 4 neighbors plus vertices at distance 2 (two steps in any cardinal direction or one step in two perpendicular directions). This forms a **cross (plus sign) with longer arms**—a central vertex with arms extending 2 units in each cardinal direction, creating a 3×3 grid of vertices (excluding the center’s immediate neighbors).  

2. **Triangular Grid 2-Neighborhood**:  
   A triangular grid has 6 neighbors in a hexagonal arrangement. The 2-neighborhood includes these 6 neighbors plus vertices at distance 2, which form a **hexagonal shape**. This is because the triangular lattice’s distance-2 neighborhood around a vertex creates a larger hexagon, with vertices arranged in concentric hexagonal layers.  

3. **King Grid 2-Neighborhood**:  
   The king grid allows moves like a chess king (4 cardinal + 4 diagonal neighbors). The 2-neighborhood includes these 8 neighbors plus vertices at distance 2, forming a **larger square**. This is a 3×3 grid of vertices (excluding the center’s immediate neighbors), as the king’s 2-step moves cover all vertices within a 2×2 square around the central vertex.    

Each shape in the image corresponds to one of these 2-neighborhoods: a cross (square grid), a hexagon (triangular grid), and a larger square (king grid), arranged in a row as described.  

The answer is <|begin_of_box|>The square grid’s 2-neighborhood is a cross (plus sign) with extended arms, the triangular grid’s is a hexagon, and the king grid’s is a larger square.<|end_of_box|></answer>