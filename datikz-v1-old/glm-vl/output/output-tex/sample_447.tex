<think>Got it, let's try to figure out what a "compact capillary horosphere" is, especially with the image description. First, let's break down the terms. A horosphere is a concept in hyperbolic geometry, like a sphere in hyperbolic space, but maybe here it's related to capillary surfaces or minimal surfaces. A capillary horosphere might be a surface that's related to capillary phenomena, which often involve minimal surfaces or surfaces of constant mean curvature. 

A compact surface is one that's closed and bounded, like a sphere, but a horosphere in hyperbolic space is a bit different. Wait, the image has a white sphere with green lines (maybe geodesics or intersections with a horosphere). In hyperbolic geometry, a horosphere is a surface that is "parallel" to a hyperplane, and its intersection with a sphere might create a circle or other curves. But the term "compact capillary horosphere"—maybe it's a compact surface that's a horosphere in some capillary context, like a surface with zero mean curvature (minimal surface) that's a horosphere. 

Wait, the image shows a sphere with green lines intersecting it. If the horosphere is a surface, its intersection with the sphere would be a circle (if the horosphere is tangent to the sphere at a point, the intersection is a great circle, but if it's not tangent, maybe a smaller circle). But the question is about the compact capillary horosphere. Maybe in the context of capillary surfaces, a horosphere could be a surface that's a sphere in some capillary system, but I need to recall that capillary surfaces often have constant mean curvature, and a horosphere might be a special case. 

Alternatively, maybe the compact capillary horosphere is a compact surface (like a sphere) that's a horosphere in a capillary configuration, with the green lines representing the intersections or the capillary lines. But without more context, let's think about the properties: compact means it's a closed surface without boundary, capillary might relate to minimal surfaces or surfaces with specific curvature, horosphere is a surface in hyperbolic space with certain properties. 

Putting it all together, the compact capillary horosphere is likely a compact surface (like a sphere) that serves as a horosphere in a capillary system, with the green lines showing intersections or the structure of the horosphere. The key elements are the compactness (closed, bounded), the capillary aspect (related to surface tension, minimal surfaces), and the horosphere (a specific surface in geometry). The image shows a sphere with lines, which could be the intersections of the horosphere with the sphere, forming a circle or other curves.</think><answer>To understand a "compact capillary horosphere" with the described image, we analyze the key terms and context:  

1. **Horosphere**: In hyperbolic geometry, a horosphere is a surface that is "parallel" to a hyperplane, acting as a limiting case of a sphere in hyperbolic space. Its intersections with spheres can form circles or other curves.  
2. **Capillary**: Relates to capillary phenomena (e.g., surface tension, minimal surfaces with constant mean curvature). A "capillary horosphere" likely refers to a horosphere adapted to capillary systems, possibly a minimal or constant-mean-curvature surface in this context.  
3. **Compact**: A compact surface is closed and bounded (e.g., a sphere), with no boundary.  

In the image, the white sphere with green lines likely shows the intersection of a compact capillary horosphere with the sphere. The green lines represent curves where the horosphere (a surface) intersects the sphere, forming a compact (closed, bounded) structure. This aligns with the horosphere’s role as a limiting surface in hyperbolic geometry, intersecting spheres to create circular or similar curves, while the "capillary" aspect connects it to surface-tension-related properties.  

The answer is <|begin_of_box|>compact capillary horosphere<|end_of_box|>.</answer>