<think>Got it, let's try to figure out what the example1(b) is. The user mentioned a white screen with mathematical equations lined up in a column, different heights, various symbols and numbers. But since there's no actual image content provided here (just the description), maybe I need to think about common types of such examples. Wait, maybe the equations are part of a problem set, like algebraic expressions, maybe involving fractions, exponents, or systems of equations. But without the actual image, it's hard to be precise. Wait, the user might have intended to include the equations, but maybe there was a formatting issue. Alternatively, maybe the example is about displaying equations in a column with different heights, which could relate to typesetting in LaTeX or another system. But since the question is to "determine" something, maybe the example is about recognizing the structure. Wait, the user says "The image is a white screen displaying a series of mathematical equations, lined up in a column. The equations are of different heights and appear to be written with various symbols and numbers." So maybe the task is to describe the equations or identify their properties. But without the actual equations, it's tricky. Wait, maybe the example is a standard one, like a list of equations for a problem, such as solving for variables, or maybe polynomial equations. Alternatively, since the user mentions "example1(b)", maybe it's part of a textbook example, but without more context, it's hard. Wait, perhaps the key is that the equations are in a column, different heights, so maybe they are of varying complexity, like some with more terms. Alternatively, maybe the equations are part of a proof or a set of identities. But since I need to provide a thought process, let's assume that the equations are, for example, linear equations, quadratic equations, or something else, and the task is to analyze them. But since the image is a white screen with equations, maybe the equations are written in a way that they are stacked vertically, each equation on a new line, with different heights (maybe due to fractions, exponents, etc.). So, for example, an equation like "2x + 3 = 7" might be shorter than "x² + 2x - 3 = 0" if typeset with the quadratic term. But without the actual equations, I can't give the exact content. Wait, maybe the user made a mistake in not providing the image content, but since I need to proceed, perhaps the example is a standard one where the equations are, say, a system of equations, or a sequence of equalities. Alternatively, maybe the equations are part of a problem where each equation is a step in solving a problem. But given the lack of specific information, I might need to state that without the actual equations, it's hard to determine, but perhaps the example is about recognizing the structure of mathematical equations in a column format. Alternatively, if this is a question about LaTeX typesetting, maybe the equations are typeset with different heights due to the use of different commands, like \frac for fractions, which can make lines taller. But since the user is asking for an example, maybe the equations are something like:

1. 2x + 3 = 5
2. x² - 4x + 4 = 0
3. 3y/2 = 6
4. a + b = c - d

Each equation is on a separate line, with varying heights because of the quadratic term, fractions, etc. So the example1(b) might be a set of such equations. But without the exact image, this is speculative. However, the user might want a general approach: describe the equations as a column of mathematical expressions, each with different heights (due to symbols like exponents, fractions, parentheses), and note the presence of various symbols (plus, minus, equals, variables, numbers) arranged in equations. Alternatively, if the task is to solve the equations, but without knowing the specific equations, that's not possible. Given the information provided, the best approach is to describe the structure as per the user's description and note the elements.</think><answer>Since the image is a white screen with mathematical equations lined up in a column, each equation having different heights (likely due to symbols like exponents, fractions, or parentheses) and containing various mathematical symbols (e.g., +, -, =, variables, numbers), the example1(b) likely consists of a sequence of distinct mathematical equations, each formatted vertically with varying visual heights. Without the exact content of the equations, the key observation is their columnar arrangement and structural diversity (e.g., linear, quadratic, fractional equations).  

If the equations were, for example, a set like:  
1. \( 2x + 3 = 5 \)  
2. \( x^2 - 4x + 4 = 0 \)  
3. \( \frac{3y}{2} = 6 \)  
4. \( a + b = c - d \)  

Each equation would occupy a separate line, with heights differing due to terms like \( x^2 \) (making the second equation taller) or \( \frac{3y}{2} \) (adding a fraction bar, increasing height). The presence of varied symbols and numbers indicates a mix of algebraic forms.  

Since the image description focuses on the visual structure (column, different heights, symbols), the example1(b) is likely a collection of mathematical equations arranged vertically with structural diversity in their formatting.  

The answer is <|begin_of_box|>[Content would depend on the specific equations in the image; if a numerical result is expected, more context is needed, but based on the description, the key is the columnar equation structure]<|end_of_box|></answer>