<think>Got, let's try to figure out what the user is asking. The question is about comparing Riemann solvers away from corner points with boundary, and there's a description of a whiteboard with triangles, equations, symbols, letters, etc. First, I need to recall what Riemann solvers are in the context of computational fluid dynamics or hyperbolic conservation laws. Riemann solvers are used to solve the Riemann problem, which is a initial value problem with piecewise constant initial data, and they're crucial for methods like Godunov's method.

Now, the user mentions "away from corner points with boundary"—so maybe considering a computational domain with boundaries, and comparing Riemann solvers in regions not near the corners (interior or edge but not corner) versus near corners. The whiteboard has a large triangle and smaller ones, maybe representing a grid or a domain with triangular cells, and the mathematical elements are part of the equations for the Riemann solver.

First, let's break down the components: Riemann solvers typically involve solving for the wave structure, fluxes, etc., across a single interface. When there's a boundary, the Riemann problem changes because one side is the boundary condition. Away from corner points (which are vertices where multiple boundaries meet), the domain might be a simple region (like a triangle without a vertex at the boundary, or a quadrilateral), so the Riemann solver would be applied across internal interfaces. At corner points, the geometry is more complex, so the Riemann solver might need adjustment, maybe using special boundary Riemann solvers or combining multiple solvers.

The image description has a large triangle (maybe the main domain) and smaller triangles (subdomains or cells), letters like a, b, c for vertices, d, e, f for edges, etc. So maybe the domain is a polygon (like a triangle or quadrilateral) with vertices labeled, and the Riemann solvers are applied across the edges (interfaces between cells). When away from the corners (so along edges that are not at a vertex with multiple boundaries), the Riemann solver is a standard one for the interface between two cells. At corners, if the corner is a vertex of the domain (like a polygon vertex), then the Riemann solver might involve the boundary condition, so the solution might be different.

To compare them, one would look at the wave structure, flux calculations, stability, accuracy. Away from corners, the Riemann solver is applied to two cells with constant states, so the solution is a sequence of waves (shock, rarefaction, contact) with fluxes calculated accordingly. At corners, if the domain has a vertex, the interface might connect to a boundary, so the Riemann problem becomes a boundary Riemann problem, where one side is the boundary condition (like Dirichlet or Neumann), so the solution might have different wave patterns, maybe with a single wave or modified waves.

Also, the equations on the whiteboard might include the Riemann invariants, flux functions, or the solution process (like using the method of characteristics, or Roe solver, or exact solver). The symbols and letters could label the states, waves, or cell interfaces. For example, in a triangular domain with cells, each edge (interface) between two cells would have a Riemann solver, and at the vertices (corners), the combination of these solvers might need to satisfy the boundary conditions.

So, summarizing the comparison: Away from corner points (interior interfaces or edge interfaces not at a polygon vertex), the Riemann solver is a standard two-state Riemann problem with two constant states (from the two cells), leading to a wave structure with fluxes calculated via the solver's method. At corner points (where multiple boundaries meet, like a polygon vertex), the Riemann problem involves a boundary condition, so the solver might need to use a boundary Riemann solver, which could have a different wave structure (e.g., a single shock or rarefaction instead of multiple waves), and the flux calculation would account for the boundary condition.

The mathematical elements on the whiteboard (equations, symbols) would likely show the difference in the Riemann problem setup: for interior interfaces, two states (u_L, u_R) with flux F(u_L), F(u_R), and solving for the waves. For boundary interfaces, one state is the boundary condition (u_b), so the Riemann problem is u_L vs u_b, and the solution would depend on the boundary condition type (e.g., if u_b is a fixed value, the solver finds the waves connecting u_L to u_b).

In terms of the image's elements (large triangle, smaller triangles), the large triangle could be the computational domain, with smaller triangles as cells. The Riemann solvers are applied across the edges of these cells. Away from the vertices of the large triangle (if the large triangle's vertices are the domain's corners), the edges between cells are interior edges, so standard Riemann solvers. At the vertices (corners), the edges meet, so the Riemann solver for the edge at the corner might involve the boundary condition if the domain has a boundary there.

So, the key points for comparison are the Riemann problem setup (number of states, boundary conditions), wave structure, flux calculation, and resulting solution behavior. Away from corners, more complex wave structures (multiple waves) might occur, while at corners, simpler structures (fewer waves) due to boundary conditions.</think><answer>To address the comparison of Riemann solvers away from corner points with boundary, we analyze the context of hyperbolic conservation laws and computational fluid dynamics (CFD), where Riemann solvers resolve wave structures across interfaces. Here’s a structured breakdown:  


### 1. **Riemann Solvers Basics**  
Riemann solvers solve the *Riemann problem*: an initial value problem with piecewise constant initial data (two constant states, \( u_L \) and \( u_R \), separated by an interface). They determine the wave structure (shocks, rarefactions, contacts) and fluxes across the interface. For a domain with boundaries, the setup changes when interfaces align with boundary conditions.  


### 2. **Away from Corner Points (Interior/Edge Interfaces)**  
When the interface is not at a *corner point* (i.e., not at a vertex where multiple boundaries meet), the Riemann problem is a standard two-state problem:  
- **Setup**: Two constant states \( u_L \) (left of the interface) and \( u_R \) (right) are given.  
- **Wave Structure**: The solution typically includes multiple waves (shocks, rarefactions, contacts) depending on the equations (e.g., Euler equations).  
- **Flux Calculation**: Fluxes \( F(u_L) \) and \( F(u_R) \) are computed via the Riemann solver’s method (e.g., Roe solver, exact solver, or approximate Riemann solver).  
- **Example**: In a triangular domain with cells (as described, with a large triangle and smaller sub-triangles), interfaces between sub-triangles (not at the large triangle’s vertices) use standard Riemann solvers.  


### 3. **At Corner Points (With Boundary)**  
When the interface is at a *corner point* (a vertex where multiple boundaries meet), the Riemann problem becomes a *boundary Riemann problem*:  
- **Setup**: One state is replaced by a **boundary condition** (e.g., Dirichlet: fixed value \( u_b \), or Neumann: fixed flux).  
- **Wave Structure**: The solution is often simpler (e.g., a single shock or rarefaction) because the boundary condition constrains the solution.  
- **Flux Calculation**: Fluxes account for the boundary condition, so the solver adjusts to match the boundary’s influence (e.g., solving for waves connecting \( u_L \) to \( u_b \) instead of \( u_L \) to \( u_R \)).  
- **Example**: At the vertices of the large triangle (if the domain has boundaries at those vertices), the Riemann solver for an edge at the vertex must incorporate the boundary condition, leading to a modified wave structure.  


### 4. **Comparison Summary**  
| Feature               | Away from Corners (Interior Interfaces)       | At Corners (With Boundary)                     |  
|-----------------------|------------------------------------------------|-------------------------------------------------|  
| **Riemann Problem**   | Two constant states (\( u_L, u_R \))          | One state + boundary condition (\( u_L, u_b \) or \( u_R, u_b \)) |  
| **Wave Structure**    | Multiple waves (shocks, rarefactions, contacts) | Simpler structure (single wave or modified rarefaction) |  
| **Flux Calculation**  | Uses standard Riemann solver for two states    | Adjusts solver to account for boundary condition |  
| **Solution Behavior** | Complex wave interactions across interfaces    | Simplified behavior due to boundary constraints |  


The whiteboard’s elements (large triangle, smaller triangles, letters/numbers) likely represent a computational domain (large triangle) with sub-domains (smaller triangles), where Riemann solvers are applied across interfaces. Away from the large triangle’s vertices (corners), standard solvers apply; at vertices, boundary-adjusted solvers are needed. The equations and symbols on the board probably detail these differences in problem setup, wave structure, and flux calculation.  

In summary, Riemann solvers away from corners use standard two-state problems with complex wave structures, while at corners (with boundaries), they use boundary-adjusted problems with simpler wave structures, reflecting the influence of boundary conditions.</answer>