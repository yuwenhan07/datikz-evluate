<think>Got it, let's try to understand this problem step by step. First, the question is about the "est-tower" of a simplex S, where the layers are circles with exponentially growing radii. The key point seems to be about how the estimation (est) works when a new vertex is added to a simplex. 

First, let's parse the description: "est estimates 1, whenever a simplex's new vertex the new vertex of a simplex lies in the next-level layer of the tower (the dotted circle), the simplex must be included in the layer itself (the continuous circle)." 

So, let's break down the components. A "tower" of circles with exponentially growing radii—probably each layer (circle) has a radius that's, say, multiplied by some factor (like 2) each time, so the radii grow exponentially. The "next-level layer" would be the one with the next larger radius, the "dotted circle" is the next level, and the "continuous circle" is the current layer. 

When a new vertex is added to a simplex, if that new vertex is in the next-level (dotted) circle, then the simplex must be included in the current layer (continuous circle). Also, "est estimates 1"—maybe the estimation function est gives a value of 1 in this scenario. 

We need to understand the relationship between the layers, the position of the new vertex, and the inclusion of the simplex in its layer. Since the layers are circles with exponentially growing radii, the distance from the center increases exponentially. If a new vertex is in the next-level (dotted) circle, its distance from the center is larger than the current layer's radius. But the current layer's radius is smaller than the next one (since radii grow exponentially). Wait, if radii grow exponentially, say radius r_n = r_0 * k^n, then the next level (n+1) has radius k*r_n, which is larger than r_n. So the dotted circle (next level) has a larger radius than the continuous circle (current layer). 

If the new vertex is in the dotted circle (next level), then its position is within distance k*r_n from the center, but the current layer is only up to r_n. So the vertex is outside the current layer but inside the next one. Then, the simplex (which includes the new vertex) must be included in the current layer (continuous circle). That suggests that even though the new vertex is in the next layer, the entire simplex is contained within the current layer. 

Maybe the idea is that if a new vertex is in a higher-level layer (dotted), the simplex (which has the new vertex and some existing vertices) must be entirely within the current layer (continuous), so the existing vertices must be in the current layer, and the new vertex, while being in the next layer, is connected in such a way that the whole simplex is within the current layer. 

Also, "est estimates 1"—maybe the estimation function est, when the new vertex is in the next-level layer, gives an estimate of 1, which might be used to decide whether to include the simplex in its layer. 

Putting it all together: The est-tower has layers as circles with exponentially growing radii. When adding a new vertex to a simplex, if the new vertex is in the next-level (dotted) circle, the entire simplex must be contained within the current layer (continuous circle). This is because the layers are defined such that a simplex containing a vertex in the next layer must lie within the current layer, and the estimation function est, which might be used to verify this, gives a value of 1 in this case, confirming the inclusion. 

So the key takeaway is the relationship between the layers (circles with exponential radii), the position of the new vertex (in the next-level dotted circle), and the requirement that the simplex is included in the current continuous circle layer. The estimation function est helps confirm this inclusion when the new vertex is in the next layer, resulting in an estimate of 1.</think><answer>To understand the "est-tower" of a simplex \( S \), we analyze the structure and conditions described:  

1. **Tower Structure**: The est-tower consists of layers as circles with radii growing exponentially. Let the \( n \)-th layer (continuous circle) have radius \( r_n \), and the next-level (dotted circle) have radius \( r_{n+1} = k \cdot r_n \) (where \( k > 1 \)), ensuring exponential growth.  

2. **New Vertex Condition**: When a new vertex is added to a simplex, if it lies in the next-level (dotted circle, radius \( r_{n+1} \)), the entire simplex must be included in the current layer (continuous circle, radius \( r_n \)). This means all vertices of the simplex, including the new one, must lie within \( r_n \).  

3. **Estimation Function**: The "est" function estimates 1 in this scenario, confirming that the simplex is appropriately contained within its layer despite the new vertex being in the next-level circle. This likely reflects a verification step ensuring the simplex’s inclusion in the current layer, leveraging the exponential layer structure.  

In summary, the est-tower’s exponentially growing circular layers enforce that a simplex with a new vertex in the next-level circle must be entirely contained in the current layer, with the estimation function validating this inclusion.  

The answer is \(<|begin_of_box|>1<|end_of_box|>\).</answer>