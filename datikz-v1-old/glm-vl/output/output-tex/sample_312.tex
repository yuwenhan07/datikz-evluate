<think>Got, let's try to figure out this problem. First, the question mentions a geometric progression formula with triangles and squares, starting with a triangle with sides 1,1,2, then squares of 2,3,4,5. Wait, maybe the triangle is a right triangle? Let's check the sides: 1,1,2. Wait, 1² + 1² = 2²? 1+1=2, which is 2, but 2² is 4, so that doesn't fit. Wait, maybe it's a different triangle, like an equilateral triangle? But sides 1,1,2 can't be equilateral. Hmm, maybe the triangle is a right triangle with legs 1 and 1, then hypotenuse would be √2, but the problem says sides 1,1,2. Wait, maybe it's a degenerate triangle? No, that doesn't make sense. Wait, maybe the triangle is part of a geometric figure where the sides correspond to some progression.

Then the squares are 2,3,4,5. Wait, squares with side lengths 2,3,4,5? Wait, the area of a square is side squared. So area would be 4,9,16,25. But how does that relate to the triangle? The triangle with sides 1,1,2—if it's a right triangle, area would be (1*1)/2 = 0.5. Maybe the progression is about areas or perimeters?

Wait, the problem says "the corners that meet at the same point hold the same value." So maybe each corner (vertex) has a value, and triangles and squares connected at those vertices share the same value. The formula is a geometric progression, so maybe the formula is for the nth term of a GP, like a*r^(n-1), where a is the first term, r is the common ratio.

But let's think about the triangle with sides 1,1,2. If it's a right triangle, then the sides are 1,1,√2, but the problem says sides 1,1,2. Maybe it's a different type of triangle, like a 1-1-√2 triangle, but maybe the problem is referring to a specific type of triangle where the sides follow a pattern. Then the squares are 2,3,4,5—maybe their side lengths are increasing by 1 each time, starting at 2. Wait, square with side 2 has area 4, side 3 area 9, etc. But how does the triangle connect?

Alternatively, maybe the problem is about the Pythagorean theorem, which is a geometric progression formula? The Pythagorean theorem is a² + b² = c². If we have a triangle with sides 1,1, then c would be √2, but the problem says sides 1,1,2. Hmm, maybe the triangle is a 45-45-90 triangle scaled up. If the legs are 1, then hypotenuse is √2, but if we scale it by √2, hypotenuse becomes 2. So maybe the triangle is a scaled right triangle with legs 1, hypotenuse 2. Then the squares might be related to the areas or perimeters of these figures.

Wait, the problem mentions a geometric progression formula. A geometric progression has a first term a, common ratio r, so terms are a, ar, ar², ar³, etc. If the progression starts with the triangle, maybe the first term is related to the triangle's area or side, then the squares follow. Let's calculate the area of the triangle with sides 1,1,2. Wait, using Heron's formula: semi-perimeter s = (1+1+2)/2 = 2. Area = √[s(s-a)(s-b)(s-c)] = √[2(2-1)(2-1)(2-2)] = √[2*1*1*0] = 0. That can't be right, so maybe it's not a triangle with sides 1,1,2. Wait, maybe the triangle is a different one, like a 3-4-5 triangle, but scaled down? Wait, the problem says "sides of 1,1,2"—maybe it's a right triangle with legs 1 and 1, hypotenuse √2, but the problem says 2. I'm getting confused here.

Wait, let's try another approach. The problem says "the corners that meet at the same point hold the same value." So each vertex has a value, and figures (triangles, squares) connected at those vertices share the same value. The formula is a geometric progression, so maybe the formula is for the sum of a geometric series, or the nth term. If the progression starts with a triangle, maybe the first term is the area of the triangle, then the next terms are areas of squares, forming a GP.

Wait, let's assume the first term is the area of the triangle. If the triangle is a right triangle with legs 1 and 1, area is 0.5. Then the squares: if the first square has area 2, then the next would be 3, 4, 5? Wait, but 2,3,4,5 aren't a GP. Wait, a GP has a common ratio, so each term is multiplied by a constant. For example, 2, 6, 18, 54 has ratio 3, but the problem says squares of 2,3,4,5. Maybe the side lengths of the squares are 2,3,4,5, so their areas are 4,9,16,25. Is there a GP here? 4, 9, 16, 25 are squares of 2,3,4,5, but they form a quadratic sequence, not geometric. Hmm.

Wait, the problem says "geometric progression formula written on a whiteboard, with triangles and squares involved in the calculations." Maybe the formula is the Pythagorean theorem, which is a² + b² = c², and it's presented in a step-by-step manner with geometric figures. The triangle with sides 1,1,2 might be an example where 1² + 1² = 2²? Wait, 1+1=2, but 2² is 4, 1+1=2≠4. That doesn't work. Wait, maybe the triangle is a 3-4-5 triangle scaled down by a factor. 3-4-5 triangle has area 6, if we scale it down by 1/2, we get 1.5-2-2.5 triangle, area 1.5. Not sure.

Wait, let's think about the problem statement again: "The corners that meet at the same point hold the same value. The image shows a geometric progression formula written on a whiteboard, with triangles and squares involved in the calculations. The progression starts with a triangle that has sides of 1, 1, and 2, and it continues with squares of 2, 3, 4, and 5. The formula is displayed in a step-by-step manner, making it easy to understand."

Maybe the key is that each figure (triangle, squares) has a value equal to the value at their vertices. The triangle with sides 1,1,2 has a certain value, and the squares with sides 2,3,4,5 have values that form a geometric progression. Wait, if the first term of the GP is the value of the triangle, say V, then the next terms are V*r, V*r², etc. If the squares have values 2,3,4,5, but that's an arithmetic progression, not geometric. Alternatively, maybe the side lengths of the squares are terms of a GP. If the first square has side length 2, then next should be 2*r, 2*r², 2*r³. But the problem says squares of 2,3,4,5, which are increasing by 1, so not a GP.

Wait, maybe the formula is the sum of a geometric progression, and the figures represent the terms. For example, the first term is the triangle's area, then each square's area is the next term. If the first term is area of triangle = 0.5 (if it's a right triangle with legs 1,1), then the next terms would be 2, 6, 18, 54 if r=4, but the squares have areas 4,9,16,25. Not matching. Alternatively, if the first term is 1 (area of triangle), then next terms 2,4,8,16 (GP with r=2), but squares have areas 4,9,16,25. Hmm, 4 is 2², 9 is 3², 16 is 4², 25 is 5². Maybe the formula is about the sum of squares, but the problem says geometric progression.

Wait, I'm overcomplicating. The problem mentions a geometric progression formula, likely the standard one: a, ar, ar², ar³,... The image has a triangle with sides 1,1,2 (maybe representing the first term a=1, r=2?), then squares with sides 2,3,4,5 (maybe the next terms). Wait, if a=1, r=2, then the terms are 1, 2, 4, 8, 16,... but the squares are 2,3,4,5. Not matching. Alternatively, if the first term is the area of the triangle, which we calculated as 0 if sides 1,1,2, but that's not helpful.

Wait, another angle: the problem says "corners that meet at the same point hold the same value." So each vertex has a value, and any figure connected to that vertex has the same value as the vertex. The formula is a geometric progression, so maybe the formula is for the nth term, and the figures are used to visualize the progression. For example, the first figure (triangle) is the first term, the next figures (squares) are the subsequent terms, each multiplied by the common ratio.

If the first term is the area of the triangle, say A, then the next term is A*r, then A*r², etc. If the squares have areas that are these terms, then maybe A=1, r=2, so areas 1, 2, 4, 8, but the squares have areas 4,9,16,25. Not matching. Wait, the squares are 2,3,4,5 as side lengths, so areas 4,9,16,25. If these are terms of a GP, then 4,9,16,25 – let's check the ratios: 9/4 = 2.25, 16/9 ≈ 1.777, 25/16 = 1.5625. Not a constant ratio, so not a GP.

Wait, maybe the formula is the Pythagorean theorem, which is a geometric formula, and the triangle is a right triangle with sides 1,1,√2, but the problem says sides 1,1,2. Maybe it's a typo, and the sides are 1,1,√2, and the squares are built on each side, with areas 1,1,2, forming a GP with ratio √2. But the problem says squares of 2,3,4,5. I'm stuck here.

Wait, let's try to think of the problem as a standard geometric progression problem with a visual representation. The first term is the triangle, maybe with area 1, then the next terms are squares with areas 2, 4, 8, 16 (GP with ratio 2), but the problem mentions squares of 2,3,4,5. Alternatively, if the first term is 1, common ratio 3, then terms are 1,3,9,27, but not matching.

Wait, the problem says "the progression starts with a triangle that has sides of 1, 1, and 2, and it continues with squares of 2, 3, 4, and 5." Maybe the sides of the squares correspond to the sides of the triangle? The triangle has sides 1,1,2, so the squares have sides 1,1,2,3,4,5? Not sure. Alternatively, the first square has side length equal to the hypotenuse of the triangle, which would be √2, but the problem says 2. Maybe the triangle is a 45-45-90 triangle with legs 1, hypotenuse √2, but scaled up by √2 to make hypotenuse 2, then the squares built on the sides would have areas 1,1,2, which is a GP with ratio 2. Then the next squares with sides 3,4,5 would be next terms, but I need to connect this to the problem's mention of squares of 2,3,4,5.

I think I'm overcomplicating. The key might be that the geometric progression formula is a, ar, ar², ar³,... and the first term a is the value of the triangle, which is, say, 1, and the common ratio r is 2, so the terms are 1, 2, 4, 8, 16,... but the squares are 2,3,4,5. Wait, maybe the formula is the sum of the first n terms of a GP, which is a(r^n - 1)/(r - 1). If a=1, r=2, n=4, sum is 1*(16-1)/1 = 15, but not sure.

Wait, the problem says "the formula is displayed in a step-by-step manner, making it easy to understand." Maybe the formula is the general term of a GP, which is a_n = a*r^(n-1). If the first term a=1, r=2, then a_1=1, a_2=2, a_3=4, a_4=8, but the squares are 2,3,4,5. Hmm. Alternatively, if the formula is for the sum, S_n = a(r^n - 1)/(r - 1), and with a=1, r=2, n=4, S_4 = 15. But I need to connect this to the triangle and squares.

Given that I'm stuck, maybe the answer is related to the formula for the nth term of a GP, which is a*r^(n-1), and the first term is the area of the triangle (1), common ratio is 2, so the formula is 1*2^(n-1). But the squares have areas 4,9,16,25, which are 2², 3², 4², 5², maybe the formula is about the sum of squares, but the problem says geometric progression.

Wait, another thought: the triangle with sides 1,1,2 is a right triangle (if we consider 1² + 1² = 2², but 1+1=2≠4), so maybe it's a different triangle. Wait, if it's a 30-60-90 triangle, sides 1, √3, 2, then the sides are 1, √3, 2. Then the squares built on these sides would have areas 1, 3, 4. That's a GP with first term 1, ratio 3, then next term 9, but the problem says squares of 2,3,4,5. Not matching.

I think I need to make an educated guess here. The problem mentions a geometric progression formula, likely the standard one, and the figures are just visual aids. The first term is the triangle's value, say 1, common ratio 2, so the formula is a_n = 1*2^(n-1). The squares continue the progression, so their values are 2,4,8,16, but the problem says squares of 2,3,4,5. Maybe the side lengths of the squares are the terms of the GP, so 2, 4, 8, 16, but the problem says 2,3,4,5. I'm confused.

Wait, the problem says "the corners that meet at the same point hold the same value." So each vertex has a value, and any figure connected to that vertex has the same value. If the triangle and squares share a vertex, their values are equal. The progression starts with the triangle (value V), then the squares have values V, V*r, V*r², V*r³. If the squares have values 2,3,4,5, then V, V*r=2, V*r²=3, V*r³=4, V*r⁴=5. Let's solve for V and r. From V*r=2 and V*r²=3, divide the second by the first: r=3/2. Then V = 2 / (3/2) = 4/3. Then V*r²= (4/3)*(9/4)=3, which matches. V*r³= (4/3)*(27/8)= (4*27)/(3*8)= (108)/24=4.5, which is not 4. Close but not exact. V*r⁴= (4/3)*(81/16)= (324)/48=6.75, not 5. So this doesn't fit.

Given that I'm going in circles, maybe the answer is the formula for the nth term of a geometric progression, which is a * r^(n-1), and the first term a is 1, common ratio r is 2, so the formula is 2^(n-1). But I need to connect it to the triangle and squares. Alternatively, the formula could be the sum, S_n = a(r^n - 1)/(r - 1), but without more info, it's hard.

Wait, the problem says "the progression starts with a triangle that has sides of 1, 1, and 2, and it continues with squares of 2, 3, 4, and 5." Maybe the sides of the squares correspond to the sides of the triangle plus 1 each time. Triangle sides 1,1,2; squares sides 2,3,4,5 (each side is previous side +1). But how does that form a geometric progression? It's an arithmetic progression. I'm stuck. Given the time I've spent, I'll assume the formula is the standard geometric progression formula, and the answer is the formula for the