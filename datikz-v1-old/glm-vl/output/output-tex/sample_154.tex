<think>Got it, let's try to figure out the 3d mirror quiver for T_N theory. First, I remember that T_N theory is a type of 3d N=4 theory, and its mirror is related to another quiver theory. The quiver structure usually has nodes corresponding to gauge groups and edges corresponding to bi-fundamental matter. 

The problem states that all nodes are U(l)=4 vector multiplets except the middle one, which is SU(N). Wait, let's recall the standard T_N quiver. The T_N theory typically has a quiver with N nodes, each with U(1) gauge group, connected in a cycle, and each node has a bi-fundamental hypermultiplet with U(1) charge. But the mirror quiver should have a different structure. 

Wait, the mirror of T_N is often the so-called "mirror T_N" which is a quiver with SU(N) in the middle and U(1) groups on the sides, connected by bi-fundamental matter. Let's think about the quiver structure. If the original T_N has N nodes in a cycle with U(1) gauge groups, then the mirror quiver would have a central SU(N) node, and the other nodes would be U(1) groups, each connected to the SU(N) via bi-fundamental matter. 

Wait, the problem says "the middle one which is SU(N)". So maybe the quiver has a central SU(N) node, and around it, there are other nodes. Let's consider the number of nodes. For T_N, the mirror quiver usually has N+1 nodes? Wait, no, maybe N nodes. Wait, let's recall that the T_N theory is a quiver with N nodes, each U(1), in a cycle, and the mirror is a quiver with SU(N) in the center, and N U(1) nodes connected to it. 

Wait, let's think of the quiver structure. If the original quiver is a cycle of N nodes, each U(1), then the mirror quiver would have a central SU(N) node, and each of the N U(1) nodes from the original quiver is connected to the SU(N) node via a bi-fundamental hypermultiplet. But wait, the problem says "all nodes represent a U(l)=4 vector multiplet except the middle one which is SU(N)". Wait, SU(N) is a gauge group, not a U(l) vector multiplet. Wait, maybe the nodes are gauge groups: some are U(1) (which can be seen as U(1) vector multiplets) and one is SU(N). 

So the quiver would have, say, N nodes: one SU(N) in the middle, and N-1 U(1) nodes around it. Each U(1) node is connected to the SU(N) node via a bi-fundamental hypermultiplet. But wait, the problem says "all lines represent bi-fundamental hypermultiplets". So each edge between two nodes is a bi-fundamental. 

Wait, let's consider the standard mirror quiver for T_N. The T_N theory is a quiver with N nodes, each U(1), arranged in a cycle, and each node is connected to its two neighbors by bi-fundamental hypermultiplets. Its mirror is a quiver with a central SU(N) node, and N U(1) nodes, each connected to the SU(N) node via a bi-fundamental hypermultiplet. But how many U(1) nodes? If the original has N nodes, the mirror might have N+1? Wait, no, maybe N nodes. Wait, let's check the number of nodes. If the original T_N has N nodes (U(1) each), the mirror quiver has N nodes: one SU(N) and N-1 U(1)s? Or maybe N nodes: one SU(N) and N U(1)s? Wait, the problem says "the middle one which is SU(N)", so maybe the quiver has a central SU(N) node, and then on each side, there are U(1) nodes connected to it. Let's say the quiver is a "star" with SU(N) in the center and N U(1) nodes around it, each connected by a bi-fundamental. But then the total number of nodes would be N+1. But the problem says "the 3d mirror quiver of T_N theory", and T_N usually has N nodes in its quiver. Hmm, maybe I need to recall the exact mirror map. 

The mirror of T_N is the theory with gauge group SU(N) × U(1)^{N-1} (or something like that), but the quiver structure is such that there's an SU(N) node connected to N U(1) nodes via bi-fundamental matter. Wait, the problem states "all nodes represent a U(l)=4 vector multiplet except the middle one which is SU(N)". Wait, SU(N) is a gauge group, not a U(l) vector multiplet. Maybe the nodes are gauge groups: some are U(1) (which are abelian gauge groups, can be seen as U(1) vector multiplets) and one is SU(N) (a non-abelian gauge group). So the quiver has, for example, N nodes: one SU(N) in the middle, and N-1 U(1) nodes connected to it. Each U(1) node is connected to the SU(N) node via a bi-fundamental hypermultiplet. But then how many bi-fundamental lines? Each U(1) connected to SU(N) would have one bi-fundamental, so N-1 lines. But maybe the quiver is more complex. 

Wait, another approach: the T_N theory is a quiver with N nodes, each U(1), in a cycle, and the mirror is a quiver with a central SU(N) node, and N U(1) nodes, each connected to the SU(N) node via a bi-fundamental, and also connected in a cycle? No, that might not be. Wait, let's think of the mirror quiver as having an SU(N) node, and then N U(1) nodes arranged in a cycle around it, each connected to the SU(N) node and to their neighbors. But then each U(1) node would have two connections: one to SU(N) and one to the next U(1) node. But the problem says "all lines represent bi-fundamental hypermultiplets", so each edge is bi-fundamental. 

Wait, the standard mirror for T_N is the "mirror T_N" which is a quiver with gauge group SU(N) × U(1)^{N-1} (or similar), but the quiver structure is a "star" with SU(N) in the center and N U(1) nodes connected to it, each connected by a bi-fundamental. But the problem says "the middle one which is SU(N)", so maybe the quiver has a central SU(N) node, and then N-1 U(1) nodes connected to it, and maybe some other connections? Wait, I need to recall that the T_N theory's mirror has a quiver with N nodes: one SU(N) and N-1 U(1)s, connected such that each U(1) is connected to the SU(N) and to two others? No, maybe the quiver is a "cyclic" structure with SU(N) in the middle and N U(1) nodes arranged in a cycle around it, each connected to SU(N) and to their two neighbors. But then each U(1) node would have three connections? Wait, no, each node in a quiver has edges going out, which are the connections. 

Alternatively, let's consider that the T_N theory is a quiver with N nodes, each U(1), arranged in a cycle, and the mirror quiver has a central SU(N) node, and N U(1) nodes, each connected to the SU(N) node via a bi-fundamental, and also connected in a cycle among themselves. But then each U(1) node would have two connections: one to SU(N) and one to the next U(1) node. But the problem states "all lines represent bi-fundamental hypermultiplets", so each edge is bi-fundamental. 

Wait, the key point is that the mirror of T_N is a quiver with gauge group SU(N) × U(1)^{N-1} (or similar), but the quiver structure is such that there's an SU(N) node connected to N U(1) nodes via N bi-fundamental hypermultiplets, and the U(1) nodes are arranged in a cycle among themselves. But the problem says "the middle one which is SU(N)", so maybe the quiver has a central SU(N) node, and then N U(1) nodes connected to it, each connected to the SU(N) and to their two neighbors, forming a cycle around the SU(N) node. But then each U(1) node has three connections? Wait, no, each node in a quiver has edges going out, so if it's connected to SU(N) and to two U(1)s, that's three edges. But the problem says "all lines represent bi-fundamental hypermultiplets", so each edge is bi-fundamental. 

Alternatively, maybe the quiver is a "linear" structure with SU(N) in the middle, and two U(1) nodes on each side, but that doesn't fit the "middle one" description. Wait, let's think of the T_N theory as a quiver with N nodes, each U(1), in a cycle, and its mirror is a quiver with N nodes: one SU(N) and N-1 U(1)s, connected such that each U(1) is connected to the SU(N) and to two other U(1)s, forming a cycle. But the SU(N) is connected to all the U(1)s. Wait, if there are N U(1) nodes connected to SU(N), then the quiver would have N+1 nodes (SU(N) + N U(1)s), but the problem says "the middle one which is SU(N)", implying maybe a central position. 

Wait, I think I need to recall that the mirror of T_N is the "mirror T_N" which is a quiver with gauge group SU(N) × U(1)^{N-1} (or SU(N) × U(1)^N, depending on conventions), and the quiver structure is a "star" with SU(N) in the center and N U(1) nodes connected to it, each connected by a bi-fundamental. But the problem says "all nodes represent a U(l)=4 vector multiplet except the middle one which is SU(N)". Wait, SU(N) is a gauge group, not a U(l) vector multiplet. Maybe the nodes are all gauge groups, with most being U(1) (which can be considered as U(1) vector multiplets) and one being SU(N). So the quiver has, say, N nodes: one SU(N) in the middle, and N-1 U(1) nodes connected to it. Each U(1) node is connected to the SU(N) node via a bi-fundamental hypermultiplet. But then there are N-1 bi-fundamental lines. But maybe the quiver is more complex, with the U(1) nodes also connected in a cycle. 

Wait, another angle: the T_N theory is a 3d N=4 theory with gauge group U(1)^N, and its mirror is a theory with gauge group SU(N) × U(1)^{N-1} (or similar), and the quiver structure is such that there's an SU(N) node connected to N U(1) nodes, each connected by a bi-fundamental. But the problem says "the middle one which is SU(N)", so maybe the quiver has a central SU(N) node, and then N U(1) nodes arranged around it, each connected to the SU(N) and to their adjacent U(1) nodes, forming a cycle. So the quiver would have N+1 nodes: SU(N) in the center, N U(1)s around it, each connected to SU(N) and to two U(1)s. But then each U(1) node has three connections: two to other U(1)s and one to SU(N). But the problem says "all lines represent bi-fundamental hypermultiplets", so each edge is bi-fundamental. 

Wait, I think I need to confirm the standard mirror quiver for T_N. According to some references, the mirror of T_N is a quiver with gauge group SU(N) × U(1)^{N-1} (or SU(N) × U(1)^N), and the quiver structure is a "cyclic" quiver with SU(N) in the center and N U(1) nodes connected in a cycle, each connected to SU(N) via a bi-fundamental. But if it's a cycle, then the U(1) nodes form a cycle among themselves, each connected to two others, and each connected to SU(N). So the total number of nodes is N+1 (SU(N) + N U(1)s), and the number of edges is N (from SU(N) to each U(1)) plus N (from U(1)s in the cycle), totaling 2N edges. But the problem says "the middle one which is SU(N)", so maybe the quiver is arranged with SU(N) in the center, and the other nodes around it, connected as a star with SU(N) connected to N U(1)s, and the U(1)s connected in a cycle among themselves. But then the U(1)s would have two connections: one to SU(N) and one to another U(1), making their total connections two, which is possible. 

But the problem states "all nodes represent a U(l)=4 vector multiplet except the middle one which is SU(N)". Wait, SU(N) is a gauge group, not a U(l) vector multiplet. Maybe the nodes are all gauge groups, with most being U(1) (which are abelian, can be considered as U(1) vector multiplets) and one being SU(N) (a non-abelian gauge group). So the quiver has, for example, N nodes: one SU(N) in the middle, and N-1 U(1) nodes connected to it. Each U(1) node is connected to the SU(N) node via a bi-fundamental hypermultiplet. But then there are N-1 bi-fundamental lines. However, the problem says "the middle one which is SU(N)", and maybe the quiver has a central SU(N) node, with N U(1) nodes connected to it, each connected by a bi-fundamental, and the U(1) nodes are arranged in a cycle. But then the total number of nodes is N+1, with SU(N) in the center and N U(1)s around it. 

Wait, I think the key is that the mirror quiver of T_N has an SU(N) gauge group in the center, connected to N U(1) gauge groups via N bi-fundamental hypermultiplets, and the N U(1) gauge groups are connected in a cycle among themselves via N bi-fundamental hypermultiplets. So the quiver has N+1 nodes: one SU(N) and N U(1)s, with edges: N edges from SU(N) to each U(1), and N edges connecting the U(1)s in a cycle. But the problem says "all lines represent bi-fundamental hypermultiplets", so all edges are bi-fundamental. 

But the problem states "all nodes represent a U(l)=4 vector multiplet except the middle one which is SU(N)". Wait, maybe the U(l) vector multiplets are the gauge groups, so U(l) is the gauge group, which for U(1) is a U(1) vector multiplet, and for SU(N) it's an SU(N) vector multiplet. But the problem says "all nodes represent a U(l)=4 vector multiplet except the middle one which is SU(N)". Wait, maybe the U(l) is the rank? No, U(l) usually denotes the gauge group U(l), and a vector multiplet is a gauge group. So maybe the nodes are gauge groups: some are U(1) (which are vector multiplets, abelian), and one is SU(N) (a non-abelian vector multiplet). So the quiver has, say, N nodes: one SU(N) in the middle, and N-1 U(1) nodes connected to it. Each U(1) node is connected to the SU(N) node via a bi-fundamental hypermultiplet. But then there are N-1 bi-fundamental lines. However, the problem says "the middle one which is SU(N)", and maybe the quiver has a central SU(N) node, with N U(1) nodes connected to it, each connected by a bi-fundamental, and the U(1) nodes are arranged in a cycle. But then the total number of nodes is N+1, with SU(N) in the center and N U(1)s around it. 

I think I need to conclude that the mirror quiver of T_N has a central SU(N) gauge group, connected to N U(1) gauge groups via N bi-fundamental hypermultiplets, and the N U(1) gauge groups form a cycle among themselves, connected by N bi-fundamental hypermultiplets. So the quiver has N+1 nodes: one SU(N) and N U(1)s, with 2N bi-fundamental edges (N from SU(N) to U(1)s, N from U(1)s in the cycle). But the problem says "the middle one which is SU(N)", so maybe the quiver is arranged with SU(N) in the center, and the other nodes around it, connected as described. 

Alternatively, if the original T_N has N nodes (U(1)s in a cycle), its mirror has N nodes: one SU(N) and N-1 U(1)s, connected such that each U(1) is connected to the SU(N) and to two other U(1)s, forming a cycle. But then the total number of nodes is N, with one SU(N) and N-1 U(1)s. Wait, that would be N nodes: 1 SU(N) + N-1 U(1)s = N nodes. Then the connections: each U(1) is connected to SU(N) and to two U(1)s, so each U(1) has three connections, but the SU