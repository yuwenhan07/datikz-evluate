<think>Got it, let's try to figure out this problem. The question is about a torus embedding of the Paley graph P_13, and the image shows a dodecahedron with a black and white pattern, consisting of 12 equilateral triangles, each side length 1. The dodecahedron is shown filled and unfilled to demonstrate structure and symmetry. 

First, I need to recall what a Paley graph is. A Paley graph of order q is a graph with q vertices where each vertex is connected to the vertices at a distance q/2 - 1, but wait, actually, Paley graphs are defined over finite fields. For a prime power q where q ≡ 1 mod 4, the Paley graph is an undirected graph with vertices as elements of the field, and two vertices are adjacent if their difference is a quadratic residue. For P_13, since 13 is a prime ≡ 1 mod 4 (13 ≡ 1 mod 4), the Paley graph P_13 should have 13 vertices, each connected to the vertices at distance 3 (since (q-1)/2 = 6, so the distance is 6, but wait, maybe I need to check the definition again). Wait, the Paley graph for order q has each vertex connected to the vertices at distance (q-1)/2, but maybe for P_13, the graph is 3-regular? Wait, 13 vertices, each vertex has degree (13-1)/2 = 6? Wait, no, wait, the Paley graph's degree is (q-1)/2 when q ≡ 1 mod 4. For q=13, (13-1)/2 = 6, so each vertex has degree 6. 

Now, a torus embedding of a graph is an embedding of the graph on a torus such that the embedding is a topological embedding, meaning the graph doesn't have any crossings, and the embedding respects the graph's structure. The image shows a dodecahedron, which is a polyhedron with 12 faces, 20 vertices, and 30 edges. Wait, but the question mentions 12 equilateral triangles, each side length 1. A regular dodecahedron has 12 regular pentagonal faces, not triangles. Hmm, maybe it's a different polyhedron, or perhaps a dual? Wait, the dual of a dodecahedron is an icosahedron, which has 20 triangular faces. Wait, the image description says 12 equilateral triangles, each side length 1. Maybe it's a regular icosahedron, which has 20 triangular faces, but 12 doesn't match. Wait, the user says "a dodecahedron, which is a 12-sided geometric shape. The pattern consists of 12 equilateral triangles..." Wait, a dodecahedron has 12 faces, but if each face is an equilateral triangle, then it's a regular icosahedron, which has 20 triangular faces. There's a contradiction here. Maybe the image is a dodecahedron with triangular faces, but perhaps it's a different structure. Alternatively, maybe the embedding of the Paley graph P_13 on a torus is related to the dodecahedron's structure. 

Wait, another approach: the Paley graph P_13 is a 6-regular graph with 13 vertices. Embedding it on a torus would require the graph to be planar when embedded on the torus, avoiding crossings. The dodecahedron is a 3-regular polyhedron, but maybe the embedding uses the dodecahedron's structure. Wait, the dodecahedron has 20 vertices, 30 edges, 12 faces. If we consider the dual of the dodecahedron, which is the icosahedron, with 12 vertices, 30 edges, 20 faces. But the Paley graph has 13 vertices, so maybe not directly. Wait, the key might be that the torus embedding of P_13 can be represented as a dodecahedron-like structure, possibly with the dodecahedron's symmetry. Alternatively, since the image shows a black and white pattern, maybe it's a checkerboard coloring of the dodecahedron, which is common in polyhedron embeddings. 

Wait, let's recall that the Paley graph P_13 is known to be a self-complementary graph, and its embedding on a torus might relate to the dodecahedron's properties. Also, the dodecahedron has icosahedral symmetry, which is a high degree of symmetry, and the Paley graph might have an embedding that preserves this symmetry. Alternatively, since the problem mentions the image shows a dodecahedron with 12 equilateral triangles (maybe each face is a triangle, but a dodecahedron should have 12 pentagonal faces), there might be a misstatement, and it's actually an icosahedron with 20 triangular faces, but the user says 12. Hmm. Alternatively, maybe the embedding is on a torus that is a quotient of the dodecahedron's graph, but I need to connect the Paley graph's properties to the dodecahedron. 

Another angle: the number of vertices in the Paley graph is 13, which is the same as the number of faces if it's a dodecahedron, but no, a dodecahedron has 12 faces. Wait, maybe the embedding has 13 vertices, each corresponding to a face or a vertex of the dodecahedron. Alternatively, since the image shows a dodecahedron with a black and white pattern, perhaps the embedding uses the dodecahedron's faces to represent the graph's vertices, with edges connecting vertices in a certain way. But I'm getting a bit stuck here. Let me try to recall that the Paley graph P_13 is a well-known graph, and its torus embedding might be related to the dodecahedron because the dodecahedron's graph is 3-regular, but the Paley graph is 6-regular. Wait, maybe the embedding combines multiple dodecahedrons or uses their symmetry. Alternatively, the key point is that the image displays the dodecahedron's structure, which is used to embed the Paley graph on a torus, possibly by identifying edges or vertices in a way that the graph's properties are preserved. 

Given that the problem states the image shows a dodecahedron with 12 equilateral triangles (even if there's a possible confusion in the number of faces), the main idea is that the torus embedding of P_13 is represented by this dodecahedron structure, demonstrating its symmetry and structure through the black and white pattern. The dodecahedron's properties, like its regularity and symmetry, help in visualizing the graph's embedding on the torus, which is a non-planar surface that can accommodate the graph without crossings by using the dodecahedron's structure as a template. 

So, summarizing, the torus embedding of the Paley graph P_13 is depicted using a dodecahedron (with its 12 triangular faces, though possibly a misdescription, but following the problem's statement) to show the graph's structure and symmetry, with the black and white pattern indicating the embedding's properties. The key takeaway is that the embedding utilizes the dodecahedron's geometric structure to represent the graph on a torus, leveraging its symmetry and regularity.</think><answer>The problem involves understanding the torus embedding of the Paley graph \( P_{13} \), depicted as a dodecahedron with a black-and-white pattern. The Paley graph \( P_{13} \) is a 6-regular graph with 13 vertices, defined over a finite field where vertices are connected if their difference is a quadratic residue. A torus embedding requires the graph to be embedded without crossings, utilizing the dodecahedron's symmetry and structure. The dodecahedron, with 12 faces (described as equilateral triangles here, though typically pentagonal), serves as a geometric template for the embedding, demonstrating the graph's properties through its regularity and symmetry. The black-and-white pattern likely represents a checkerboard coloring or vertex coloring consistent with the graph's structure, emphasizing the embedding's topological and geometric characteristics.  

The answer is \(<|begin_of_box|>12<|end_of_box|>\).</answer>