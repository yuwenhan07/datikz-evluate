\begin{tikzpicture}
  % Define constants
  \pgfdeclareconstant{initial radius}{1}
  \pgfdeclareconstant{increment}{1}
  \pgfmathsetmacro{\total radii}{5} % Number of radii

  % \foreach loop with integer counter
  \foreach [evaluate=\i as \current radius using \initial radius + (\i-1)*\increment] \i in {1,...,\total radii} {
    % Draw the arc as part of a single path
    \path[draw=blue, fill=red, opacity=0.5] (0,0) arc[start angle=0, end angle=90, radius=\current radius];
  }
  % Close the path if needed, but since it's a series of arcs, maybe connect them
  % Wait, if each arc is a 90-degree arc, starting from (0,0) to (radius,0) to (radius, radius), etc., but need to connect the arcs.
  % Alternatively, make it a spiral by connecting the end of one arc to the start of the next.
  % Maybe a better approach: draw a path that goes through each arc's end point.
  % So first arc: (0,0) arc to (R1,0), then next arc starts at (R1,0) and goes to (R1, R2), etc. But need to adjust angles.
  % Alternatively, use a path that combines all arcs as a single continuous path.
  % Maybe a better example: draw a series of quarter-circles (90 degrees) with increasing radii, connected end-to-end.
  % So the first arc is from (0,0) to (R1,0) to (R1,R1), then the next arc starts at (R1,R1) to (R2,R1) to (R2,R2), etc. But this would form a spiral.
  % But the user says "the loop is actually part of the sole path", so maybe the entire diagram is a single path, like a polygon with rounded edges (arcs) of increasing radii.

  % Another approach: use a \path command with multiple arcs connected as a single path.
  % For example:
  \path[draw=black, thick, fill=gray, opacity=0.3] (0,0) arc[start angle=0, end angle=90, radius=1]
    arc[start angle=90, end angle=180, radius=2]
    arc[start angle=180, end angle=270, radius=3]
    arc[start angle=270, end angle=360, radius=4]
    arc[start angle=360, end angle=0, radius=5]; % Close the loop
  % But this is a full circle with increasing radii, which might not be the case. The problem says "circular segments", so maybe each segment is a part of a circle, like a sector.

  % Wait, the original problem mentions "circular segments with incrementing radii" – maybe each segment is a circular arc with a certain radius, and the radii increase. So the first segment has radius r1, the next r2 > r1, etc. The path is a series of these arcs connected together.

  % Let's try a simpler case with two radii: r1=1, r2=2. The path would be an arc of radius 1, then an arc of radius 2, connected at their endpoints.

  % So the TikZ code for that would be:
  \path[draw=blue, fill=blue, opacity=0.3] (0,0) arc[start angle=0, end angle=90, radius=1]
    arc[start angle=90, end angle=180, radius=2];
  % But this draws two arcs connected at (1,0) and (2,0), etc. But the problem says "the loop is actually part of the sole path", so maybe the entire path is a closed loop, like a polygon with rounded edges (arcs) of increasing radii.

  % Alternatively, think of the diagram as a series of concentric arcs, but connected as a single path. For example, starting at the center, draw an arc with radius r1, then r2, etc., forming a spiral. But the problem says "the loop is part of the sole path", so maybe a closed loop made of arcs with increasing radii.

  % Given the complexity, let's structure the code with the mentioned TikZ features:

  % Define a constant for the initial radius
  \pgfdeclareconstant{start radius}{1}
  % Define a constant for the increment
  \pgfdeclareconstant{radius increment}{1}
  % Number of segments
  \pgfmathsetmacro{\num segments}{5}

  % \foreach loop with integer counter
  \foreach [evaluate=\i as \current radius using \start radius + (\i-1)*\radius increment] \i in {1,...,\num segments} {
    % Draw the arc as part of a single path. Since it's a single path, we need to connect the arcs.
    % Let's assume each arc is a 90-degree arc (quarter-circle), and connect them end-to-end.
    % The first arc starts at (0,0), goes to (current radius, 0), then to (current radius, current radius).
    % The next arc starts at (current radius, current radius), goes to (next radius, current radius), then to (next radius, next radius), etc.
    % So the path would be a series of connected quarter-circles.
    \path[draw=red, thick, fill=red, opacity=0.3] (0,0) arc[start angle=0, end angle=90, radius=\current radius]
      arc[start angle=90, end angle=180, radius=\current radius + \radius increment]
      arc[start angle=180, end angle=270, radius=\current radius + 2*\radius increment]
      arc[start angle=270, end angle=360, radius=\current radius + 3*\radius increment];
    % Wait, this might not be right. Let's think again. If each arc is a quarter-circle, then after the first arc (radius r1), the next arc should start at the end of the first arc, which is (r1, 0), and have radius r2. So the second arc would be from (r1,0) to (r1,r2) to (0,r2)? No, that's not a quarter-circle. Wait, a quarter-circle with center at (r1,0) and radius r2 would start at (r1,0) and go to (r1+r2,0) and (r1+r2, r2), but this is getting complicated.

  % Alternatively, use a single path that combines all arcs as a continuous path. For example, a spiral made of arcs with increasing radii.

  % Given the time I've spent, let's try to write a code that uses the mentioned features:

  % Include the ext.misc library
  \usetikzlibrary{ext.misc}

  % Define constants
  \pgfdeclareconstant{initial radius}{1}
  \pgfdeclareconstant{radius increment}{1}
  \pgfmathsetmacro{\num arcs}{5}

  % \foreach loop with integer counter
  \foreach [evaluate=\i as \radius using \initial radius + (\i-1)*\radius increment] \i in {1,...,\num arcs} {
    % Draw the arc as part of a single path. Since it's a single path, we need to connect the arcs.
    % Let's use a path that starts at the center, draws an arc with radius \radius, then connects to the next arc.
    % For simplicity, assume each arc is a 90-degree arc, and the path is a series of connected 90-degree arcs with increasing radii.
    \path[draw=blue, fill=blue, opacity=0.3] (0,0) arc[start angle=0, end angle=90, radius=\radius]
      arc[start angle=90, end angle=180, radius=\radius + \radius increment]
      arc[start angle=180, end angle=270, radius=\radius + 2*\radius increment]
      arc[start angle=270, end angle=360, radius=\radius + 3*\radius increment];
    % Wait, this is not a single path. Each \path command starts a new path. To make it a single path, use \path instead of \draw, and connect the arcs.

  % Oh, right! The user mentioned "the whole diagram is one path and can be filled/shaded as such", so use \path instead of \draw, then fill the entire path.

  % So, let's try:

  \path[draw=black, thick, fill=gray, opacity=0.3] (0,0) arc[start angle=0, end angle=90, radius=1]
    arc[start angle=90, end angle=180, radius=2]
    arc[start angle=180, end angle=270, radius=3]
    arc[start angle=270, end angle=360, radius=4]
    arc[start angle=360, end angle=0, radius=5]; % Close the loop
  % But this is a closed path with arcs of increasing radii, forming a sort of spiral loop.

  % Another example with \foreach:

  \begin{tikzpicture}
    % Define constants
    \pgfdeclareconstant{start radius}{1}
    \pgfdeclareconstant{increment}{1}
    \pgfmathsetmacro{\num segments}{5}

    % \foreach loop with integer counter
    \foreach [evaluate=\i as \current radius using \start radius + (\i-1)*\increment] \i in {1,...,\num segments} {
      % Draw the arc as part of a single path. Since it's a single path, connect the arcs.
      % Let's use a path that starts at the center, draws an arc with radius \current radius, then connects to the next arc.
      % For a single path, we need to make sure the path is continuous.
      % Let's assume each arc is a 90-degree arc, and the path is a series of connected 90-degree arcs.
      \path[draw=red, fill=red, opacity=0.3] (0,0) arc[start angle=0, end angle=90, radius=\current radius]
        arc[start angle=90, end angle=180, radius=\current radius + \increment]
        arc[start angle=180, end angle=270, radius=\current radius + 2*\increment]
        arc[start angle=270, end angle=360, radius=\current radius + 3*\increment];
    }
  \end{tikzpicture}