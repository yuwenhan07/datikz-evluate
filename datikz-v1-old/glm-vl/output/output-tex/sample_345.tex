<think>Got it, let's try to figure out what's being asked here. The user mentioned a plot with F in blue and F⁻¹ (inverse function) in red, on a grid with a checkered pattern. The description talks about red lines (length 1, at x=1,2,3,4) and blue lines (length 2, at x=0,1,2,3), intersecting to make a checkered design. 

First, I need to recall that if you have a function F and its inverse F⁻¹, their graphs are symmetric over the line y=x. But the checkered pattern might be related to the grid lines where the function and its inverse intersect, creating a checkerboard pattern. The red and blue lines could be the grid lines for the inverse function and the function itself, with their intersections forming the checkered squares.

Wait, the red lines are one unit long at x=1,2,3,4—maybe horizontal lines? Or vertical? The blue lines are two units long at x=0,1,2,3—maybe vertical or horizontal. If the grid has both red (length 1, maybe vertical lines at x=1,2,3,4) and blue (length 2, maybe horizontal lines at y=0,1,2,3), then their intersections would form squares of size 1x1, creating a checkered pattern. But the description says red and blue lines intersecting, with red lines at x=1,2,3,4 (so maybe vertical lines at those x-values, each 1 unit long, maybe from y=0 to y=1?) and blue lines at x=0,1,2,3 (maybe horizontal lines at those y-values, each 2 units long, from x=0 to x=2?). Then the intersection points would form a grid where each square is a combination, creating a checkered pattern of red and blue lines.

Alternatively, since F and F⁻¹ are plotted, maybe the checkered pattern is the result of their graphs intersecting with the grid lines, creating a visual pattern. But without the actual image, I need to infer from the description. The key points are the grid with red and blue lines, lengths, positions, and the checkered pattern from intersections.

So, summarizing: The plot has F (blue) and its inverse F⁻¹ (red) plotted on a grid with a checkered pattern. The red lines (F⁻¹) are 1 unit long at x=1,2,3,4 (maybe vertical lines), blue lines (F) are 2 units long at x=0,1,2,3 (maybe horizontal lines), and their intersections form a checkered design. The checkered pattern is created by the intersection of these lines, with each square being a combination of red and blue lines, resulting in a visually appealing grid.</think><answer>To analyze the plot of \( F \) (blue) and its inverse \( F^{-1} \) (red) with a checkered grid pattern:  

1. **Grid Structure**: The grid features a checkered pattern from intersecting red (1-unit) and blue (2-unit) lines.  
   - **Red lines** (length 1) are positioned at \( x = 1, 2, 3, 4 \) (likely vertical lines, each spanning 1 unit in the vertical direction).  
   - **Blue lines** (length 2) are positioned at \( x = 0, 1, 2, 3 \) (likely horizontal lines, each spanning 2 units in the horizontal direction).  

2. **Intersection Pattern**: The intersection of these lines creates a checkered design, where each square in the grid alternates between red and blue lines, forming a visually uniform checkerboard pattern.  

3. **Function Relationship**: Since \( F \) and \( F^{-1} \) are inverse functions, their graphs are symmetric over the line \( y = x \). The checkered pattern likely arises from the alignment of grid lines with both functions, emphasizing their inverse relationship through the grid’s structure.  

The answer is \(<|begin_of_box|>\text{A checkered pattern formed by intersecting red (1-unit) and blue (2-unit) grid lines, symmetric to the inverse function relationship}<|end_of_box|>\).</answer>