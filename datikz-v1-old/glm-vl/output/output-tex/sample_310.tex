<think>Got it, let's try to parse this problem step by step. First, the problem mentions exchanging edges of T₁ by copies of U to create A₁. Then, a map B is obtained by gluing the first u and the last v, along the exterior face. Hmm, I need to recall some concepts from map theory or graph theory, maybe related to planar maps or Eulerian maps. 

First, let's clarify the terms. T₁ might be a specific type of map, like a tree or a triangulation. Exchanging edges with copies of U probably means replacing certain edges with copies of another map U, which could be a submap or a specific structure. Then creating A₁ by doing this edge exchange. Then map B is formed by gluing the first u and last v along the exterior face. Wait, "first u" and "last v"—maybe u and v are vertices or edges? The exterior face is usually the unbounded face in a planar map.

Let me think about the process. If we start with map T₁, and exchange some edges (maybe the edges of a certain subgraph) with copies of map U, then the resulting map is A₁. Then, to get map B, we glue the first vertex u and the last vertex v (or maybe the first and last edges) along the exterior face. Gluing along the exterior face would mean identifying the boundary of the exterior face of one map with the exterior face of another, or maybe attaching them along their outer boundaries.

Wait, maybe T₁ is a map with a certain structure, say a tree with some edges, and U is a submap, like a cycle or a tree. Exchanging edges could mean replacing edges with copies of U, which might create a new map A₁. Then, map B is constructed by taking A₁ and gluing the first u (maybe a vertex) and last v (another vertex) along the exterior face. The exterior face is the one that's not bounded by any edges, so gluing along that face would connect the two maps along their outer boundaries.

Alternatively, maybe this is related to the concept of "edge contraction" or "edge replacement" in maps, where edges are replaced by copies of another map, and then gluing operations are performed. The key steps are: start with T₁, perform edge exchanges with U to get A₁, then form B by gluing the first u and last v along the exterior face. 

I need to make sure I understand the terminology. "Exchanging edges of T₁ by copies of U"—so for each edge in T₁, replace it with a copy of U? Or replace some edges (maybe a set of edges) with copies of U? If it's a specific set of edges, say the edges forming a certain subgraph, then replacing them with copies of U would create a new map A₁. Then, map B is formed by gluing the first vertex u and the last vertex v (maybe the first and last vertices in some ordering) along the exterior face. 

Another angle: in map theory, when you glue maps along a face, you identify the faces. If you glue along the exterior face, you're connecting the two maps such that their exterior faces are identified. So if A₁ has an exterior face, and we take another map (maybe a single vertex or edge) and glue its exterior face to A₁'s exterior face, connecting the first u and last v. 

Wait, the problem says "gluing the first u and the last v, along the exterior face." Maybe u and v are vertices, and the first u is the first vertex in some order, and last v is the last vertex, and gluing them along the exterior face means connecting them via the exterior face's boundary. 

Alternatively, maybe T₁ is a map with a certain number of edges, and U is a map with, say, u edges and v edges, but I need more specific information. Wait, the problem mentions "the first u and the last v"—maybe u and v are the number of edges or vertices? Hmm, this is a bit vague without more context, but perhaps this is a standard construction in map theory where you modify a map by replacing edges with submaps and then gluing along the exterior face.

Given that, the process would be: start with T₁, replace some edges (maybe the edges of a path or a cycle) with copies of U to get A₁. Then, take A₁ and another map (maybe a single edge or vertex) and glue the first vertex u (of the new map) and the last vertex v (of the new map) along the exterior face of A₁. The exterior face is the one that's not adjacent to any other faces, so gluing along it would connect the two maps at their outer boundaries.

In summary, the steps are: 1. Modify T₁ by replacing edges with copies of U to form A₁. 2. Create map B by gluing the first vertex u and the last vertex v along the exterior face of A₁. The exterior face is the unbounded face, so gluing along it would connect the two maps along their outer edges, possibly merging the exterior faces or connecting them through the vertices u and v.

I think the key here is understanding that exchanging edges with copies of U changes the structure of T₁ into A₁, and then gluing along the exterior face connects the maps in a specific way, likely altering the number of faces or vertices. Without more specific definitions of T₁, U, u, v, it's hard to give a precise formula, but the process involves edge replacement and face gluing.</think><answer>To address the problem, we start by modifying map \( T_1 \) by exchanging its edges with copies of map \( U \) to form map \( A_1 \). This edge replacement alters the structure of \( T_1 \) by incorporating submaps \( U \) in place of selected edges. Next, map \( B \) is constructed by gluing the first vertex \( u \) and the last vertex \( v \) along the exterior face of \( A_1 \). Gluing along the exterior face identifies the boundaries of these vertices with the unbounded face of \( A_1 \), effectively connecting the two maps at their outer boundaries.  

The answer is \(<|begin_of_box|>B<|end_of_box|>\).</answer>