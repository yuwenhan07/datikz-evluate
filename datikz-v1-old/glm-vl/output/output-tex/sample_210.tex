<think>Got it, let's try to parse this description step by step. The user is talking about "Extra critically non-line graphs" with images of squares. Let's break down the elements:

First, the images are black and white, with larger squares each having a smaller square in the middle. The pattern involves symbols: starting with an L at top left, then L at top right, 1 at bottom left, 2 at bottom right. They're arranged in a square formation, larger squares surrounding smaller ones.

Wait, need to make sure I understand the structure. The images are arranged in a square formation—so maybe a grid of images, each being a square with a smaller square inside. The symbols (L, L, 1, 2) are probably in specific positions: top left has L, top right has L, bottom left has 1, bottom right has 2. The squares are increasingly larger? Wait, the description says "increasingly larger squares, each with a smaller square in the middle"—so maybe each subsequent image has a larger outer square, with a smaller inner square, and the symbols change position or content.

But the user's description is a bit vague, but let's try to reconstruct. The key elements are: black and white images, squares (outer and inner), symbols (L, L, 1, 2) in specific corners, arranged in a square formation (so maybe a 2x2 grid of images, each being a square with inner square and symbols). The progression is "increasingly larger squares"—so maybe the size of the outer square increases as you go through the images.

Wait, the user says "the pattern starts with an L at the top left and progresses to an L on the top right, followed by a 1 on the bottom left and a 2 on the bottom right." So maybe the first image has L top left, then next has L top right, then 1 bottom left, 2 bottom right. Each image is a square with inner square, arranged in a square formation (like a 2x2 grid of these images). The "increasingly larger squares" might mean that each image's outer square is larger than the previous, so the first image has a smaller outer square, then next larger, etc.

But since the user is asking about "Extra critically non-line graphs" related to these images, maybe the graphs are related to the structure of the images—like the arrangement of squares, the symbols, or the progression. But without seeing the image, we have to infer based on the description.

Let me try to outline the elements:

1. Images are black and white, each is a square with a smaller square inside (so like a frame).
2. Symbols: top left has L, top right has L, bottom left has 1, bottom right has 2. So each image has four symbols in the corners: top left L, top right L, bottom left 1, bottom right 2.
3. Arranged in a square formation—so maybe a 2x2 grid of these images, each being a square with inner square and symbols.
4. The squares are "increasingly larger"—so the outer square of each image is larger than the previous, meaning the first image (maybe top left of the grid) has a smaller outer square, then next (top right) larger, then bottom left larger, bottom right largest? Or maybe the images are arranged in a sequence where each subsequent image has a larger outer square.

"Extra critically non-line graphs"—maybe the graphs are related to the structure of the images, like the number of squares, the arrangement, or the symbols. Non-line graphs could refer to graphs that aren't linear, maybe with nodes representing images or squares, connected by edges based on some relationship (like size, symbol, position).

But since the user's description is a bit abstract, maybe the key is to describe the structure of the images and how they relate to the graphs. Let's try to structure the thoughts:

First, identify the components: black and white images, square shapes (outer and inner), symbols in corners, arrangement in a square formation, increasing size of squares. Then, connect this to "non-line graphs"—maybe the graphs represent the relationships between the images (like size progression, symbol changes) in a non-linear way, with nodes as images and edges as transitions.

But without more context, it's hard, but let's try to form a coherent answer based on the given description.

So, the images are a series of black and white square diagrams, each consisting of a larger square enclosing a smaller square. The arrangement is in a square formation (likely a grid, like 2x2), with each image having symbols in their corners: top left is 'L', top right is 'L', bottom left is '1', bottom right is '2'. The progression involves increasingly larger outer squares (so each subsequent image has a larger outer square than the previous), and the symbols follow a pattern starting with 'L' in top left, then 'L' in top right, then '1' in bottom left, '2' in bottom right. The "extra critically non-line graphs" might refer to graphs that model the non-linear relationships between these images—such as the size progression, symbol transitions, or spatial arrangement—using nodes and edges that don't follow a straight line, representing complex connections between the images' features (size, symbol, position).

Wait, but the user might be referring to a specific type of graph related to the image structure. Alternatively, maybe "non-line graphs" is a typo or specific term, but given the info, we have to work with what's provided.

Putting it all together, the description is about a series of square images with inner squares, symbols in corners, arranged in a square formation, with increasing square sizes, and the graphs related to these images are non-linear, possibly representing the complex relationships between the images' elements (size, symbols, position) in a non-linear graph structure.</think><answer>To address the description of “Extra critically non - line graphs” related to the image series, let’s break down the key elements and their connections:  

### 1. Image Structure  
- **Black and white images**: Each image is a square with a smaller square inside (creating a “frame” effect).  
- **Symbols and arrangement**: The images have symbols in their corners: top - left is an “L”, top - right is an “L”, bottom - left is a “1”, and bottom - right is a “2”. These images are arranged in a square formation (likely a 2×2 grid, as there are four distinct symbols in four corners).  
- **Size progression**: The images feature “increasingly larger squares” (the outer square of each image is larger than the previous one), suggesting a sequence where square size grows as the series progresses.    

### 2. Connection to “Non - line Graphs”  
“Non - line graphs” likely refer to graphs that model complex, non - linear relationships between the images’ features. Here’s how the image elements connect to such graphs:  
- **Nodes and edges**: Each image (or square/symbol) could be a node in a graph. Edges might represent relationships like size progression (larger squares connect to smaller ones in a non - linear way), symbol transitions (from “L” to “1”/“2” in corner positions), or spatial arrangement (how images are grouped in the square formation).  
- **Non - linear structure**: Since the square sizes increase non - uniformly (implied by “increasingly larger” without strict progression), the graph’s edges or connections would reflect this non - linear growth, rather than a straight - line (linear) relationship.    

### Summary  
The image series consists of black - and - white square diagrams with inner squares, symbols in corner positions, arranged in a square grid, and with progressively larger outer squares. The “extra critically non - line graphs” likely model the complex, non - linear relationships between these images’ structural (size, arrangement) and symbolic (L, 1, 2) features using a graph structure with non - linear connections.  

The answer is <|begin_of_box|>The image series consists of black and white square diagrams with inner squares, symbols (L, L, 1, 2) in corner positions, arranged in a square formation with increasingly larger outer squares, and the non - line graphs model complex relationships between these elements.<|end_of_box|> (Note: The final answer is a summary of the key elements and their connection to non - line graphs, as the question is descriptive and doesn’t have a single numerical answer, so a concise summary is provided.)</answer>