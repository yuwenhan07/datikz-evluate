<think>Got it, let's try to parse this question and the given information. First, the question is about the influence of the standard deviation of Gaussian utilities on the average convergence time, specifically for dynamics converging within 100,000 steps. Then, there's a note about resentful agents: NS dynamics didn't converge in 7/23/48/63/75 games for standard deviations 60/70/80/90/100, respectively. For appreciative and resentful-appreciative agents, the dynamics always converged.

First, let's break down the components. The key elements are:

1. **Standard deviation (σ)** of Gaussian utilities: the values given are 60, 70, 80, 90, 100.
2. **Number of games where NS dynamics didn't converge** for resentful agents: 7, 23, 48, 63, 75 for σ=60,70,80,90,100 respectively.
3. **Convergence behavior for other agent types**: appreciative and resentful-appreciative always converged, so their dynamics didn't have non-converging games in the same setup.

The question is about the influence of σ on average convergence time, but the data provided is about the number of non-converging games (for resentful agents) and convergence for others. So, we need to connect σ to convergence time or non-convergence.

First, let's note that as σ increases, the number of non-converging games for resentful agents increases (7 → 23 → 48 → 63 → 75). That suggests that higher σ (more spread in utilities) leads to more cases where the dynamics don't converge within 100,000 steps. So, the influence of σ is that higher standard deviation (more variability in utilities) makes convergence less likely (more non-converging games), hence average convergence time might increase (or become undefined for non-converging cases) or decrease if converging, but here for resentful agents, as σ increases, convergence fails more often.

For the other agent types (appreciative, resentful-appreciative), since they always converged, their average convergence time would be based on the 100,000 steps or less, but the question is about resentful agents. So, the influence of σ on average convergence time for resentful agents: as σ increases, the number of non-converging games increases, so the average convergence time might not be calculable (since some don't converge), or if we consider only converging ones, maybe the time is longer, but the data shows more non-converging as σ increases.

So, summarizing the influence: higher standard deviation of Gaussian utilities (more variability) leads to a higher proportion of games where the dynamics (NS) do not converge within the 100,000 step limit for resentful agents. This implies that the average convergence time for resentful agents is affected by σ such that as σ increases, the likelihood of non-convergence increases, making the average convergence time either increase (if converging) or become non-existent (if many don't converge). For the other agent types, since they always converge, their average convergence time is not affected by σ in the same way (or at least not as much, as they always converge within the limit).

Now, to structure the answer: first, identify the variables (σ, convergence time, agent type), then relate σ to non-convergence (for resentful), and then discuss the influence on average convergence time.</think><answer>To analyze the influence of the standard deviation of Gaussian utilities on the average convergence time, we examine the provided data and relationships:  


### Step 1: Understand the Key Variables  
- **Standard deviation (σ)** of Gaussian utilities: The values considered are 60, 70, 80, 90, 100 (increasing variability in utilities).  
- **Convergence behavior**: Dynamics are analyzed for “dynamics converging within the time limit of 100,000 steps.” For resentful agents, the number of non-converging games increases with σ; for appreciative and resentful-appreciative agents, dynamics always converge.  


### Step 2: Analyze Resentful Agents’ Non - Convergence  
For resentful agents, the number of non - converging games (within 100,000 steps) is:  
- σ = 60: 7 games  
- σ = 70: 23 games  
- σ = 80: 48 games  
- σ = 90: 63 games  
- σ = 100: 75 games  

As σ increases, the number of non - converging games **increases**. This indicates that higher σ (more variability in utilities) makes the dynamics less likely to converge within the 100,000 - step limit for resentful agents.  


### Step 3: Influence on Average Convergence Time  
For resentful agents:  
- With higher σ, more games fail to converge, so the “average convergence time” becomes less meaningful (since many cases do not converge). If we only consider converging games, the time to convergence might increase (as variability hinders convergence), but the overall average is skewed by non - convergence.  
- For appreciative and resentful - appreciative agents, since dynamics always converge, their average convergence time is not affected by σ in the same way (they consistently converge within the limit, so σ’s influence is minimal or absent for these agent types).  


### Conclusion  
The standard deviation of Gaussian utilities has a **negative influence** on convergence for resentful agents: higher σ increases the likelihood of non - convergence within the 100,000 - step limit, thereby reducing the reliability of an average convergence time. For other agent types (appreciative, resentful - appreciative), the dynamics always converge, so σ has little to no impact on their average convergence time.  

The answer is <|begin_of_box|>Higher standard deviation of Gaussian utilities increases the number of non - converging games for resentful agents, reducing average convergence time reliability (or increasing non - convergence likelihood)<|end_of_box|>. (Note: The exact phrasing depends on interpreting the question’s intent, but the core conclusion is that higher σ correlates with more non - convergence for resentful agents, affecting average convergence time.)</answer>