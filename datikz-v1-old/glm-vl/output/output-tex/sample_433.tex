<think>Got it, let's try to figure out the Markov boundary for this collider structure. First, I need to recall what a Markov boundary is. The Markov boundary of a set of variables in a graph is the set of variables that are not conditionally independent of the set given the rest. For a collider structure, like X1 → X2 ← X3 (a V structure), the Markov boundary of the collider (X2) would include its parents and children, but wait, need to check the specific structure.

Wait, the question mentions a "general Markov boundary collider structure" with variables X1, X2, X3. Let's assume the graph is a collider, maybe like X1 → X2 ← X3, forming a V shape with X2 as the collider. In that case, the Markov boundary of the set {X2} would be its parents (X1, X3) and its descendants. But wait, the Markov boundary of a single variable in a collider is the set of variables that are not conditionally independent of it given its Markov blanket. Wait, the Markov boundary of a variable is the set of variables that are not conditionally independent of it given its Markov blanket. For a collider X2 with parents X1 and X3, the Markov blanket of X2 is its parents and children. If there are no children (assuming the graph is just X1→X2←X3), then the Markov blanket of X2 is {X1, X3}. Then the Markov boundary of X2 would be the variables not in the Markov blanket, but wait, maybe the question is about the Markov boundary of the entire set {X1, X2, X3} or a specific subset.

Wait, let's think step by step. In a Markov network, the Markov boundary of a set S is the set of variables not in S that are not conditionally independent of S given the rest of the variables. For a collider structure like X1 → X2 ← X3, the graph is a V structure with X2 as the collider. The Markov boundary of the set {X2} would include X1, X3, and any variables that are not conditionally independent of X2 given X1 and X3. But if we consider the Markov boundary of the entire graph, maybe the variables involved in the collider. Alternatively, if the graph is a more complex collider structure, but since it's a general case, let's assume the structure is a V structure with X1 and X3 as parents of X2. Then, the Markov boundary of X2 is the set of variables that are not conditionally independent of X2 given its parents and children. If there are no children (assuming the graph is just X1→X2←X3), then the Markov boundary of X2 is {X1, X3}. But if we're talking about the Markov boundary of the entire set {X1, X2, X3}, then we need to find which variables are not conditionally independent of the set given the rest. Wait, maybe the question is about the Markov boundary of the collider node, which is X2. In that case, the Markov boundary of X2 would include its parents (X1, X3) and any variables that are not conditionally independent of X2 given its Markov blanket. But if the Markov blanket of X2 is {X1, X3} (assuming no children), then the Markov boundary of X2 is the set of variables not in the Markov blanket, but wait, the Markov boundary is the set of variables that are not conditionally independent of the variable given its Markov blanket. So for X2, the Markov boundary would be the variables that are not conditionally independent of X2 given {X1, X3}. But in the collider structure, X1 and X3 are the parents, and if we condition on both, then X1 and X3 are independent of each other given X2? Wait, no, in the V structure, if you condition on X2, then X1 and X3 become independent. So the Markov boundary of X2 would be the variables that are not conditionally independent of X2 given its Markov blanket. If the Markov blanket is {X1, X3}, then the Markov boundary of X2 is the set of variables not in {X1, X3} that are not conditionally independent of X2 given {X1, X3}. But if there are no other variables, then maybe the Markov boundary is empty? Wait, I might be getting confused. Let's recall that in a collider structure, the Markov boundary of the collider node (X2) includes its parents (X1, X3) and its descendants. If there are no descendants, then the Markov boundary is {X1, X3}. Alternatively, if the question is about the Markov boundary of the entire graph, which is the set of all variables not in the graph's Markov equivalence class, but that's more complex. Wait, the question says "General Markov boundary collider structure" with variables X1, X2, X3. A common collider structure is a V structure where X1 → X2 ← X3, so the Markov boundary of the set {X2} would be the set of variables that are not conditionally independent of X2 given X1 and X3. But since X1 and X3 are the parents of X2, and in the collider, conditioning on X2 makes X1 and X3 independent, so the Markov boundary of X2 would include X1 and X3, because they are not conditionally independent of X2 given each other? Wait, no. Let's use the definition: the Markov boundary of a set S is the set of variables V not in S such that there exists a path from V to S that is not blocked by S. In the V structure X1 → X2 ← X3, the paths from X1 to X2 are direct (X1→X2), from X3 to X2 are direct (X3→X2), and from X1 to X3 there's no direct path. The Markov boundary of {X2} would include X1 and X3 because there are paths from them to X2 that aren't blocked by X2. A path from X1 to X2 is X1→X2, which is not blocked by X2 (since X2 is the collider, the path is direct). Similarly for X3→X2. So the Markov boundary of {X2} is {X1, X3}. Alternatively, if we consider the Markov boundary of the entire graph (all three variables), then we need to find which variables are not conditionally independent of the others given the rest. But maybe the question is asking for the Markov boundary of the collider node, which is X2, so its Markov boundary is {X1, X3}. Alternatively, if the structure is a more complex collider with additional connections, but without the image, we have to assume the standard V structure. So, putting it all together, the Markov boundary for the collider structure involving X1, X2, X3, with X2 as the collider, would include its parent variables, so X1 and X3, making the Markov boundary {X1, X3} for X2, or if considering the entire set, maybe the boundary includes the variables connected to the collider. But I think the key here is recognizing that in a collider structure, the Markov boundary of the collider node includes its parents, so the answer would be the variables that are the parents of the collider, which are X1 and X3, so the Markov boundary is {X1, X3} for the collider X2.</think><answer>To determine the Markov boundary for the general Markov boundary collider structure with variables \( X_1 \), \( X_2 \), and \( X_3 \), we first identify the structure as a **V-collider** (e.g., \( X_1 \to X_2 \leftarrow X_3 \)), where \( X_2 \) is the collider node.  

In a Markov network, the **Markov boundary** of a variable (or set of variables) consists of variables not conditionally independent of the set given the rest. For a collider \( X_2 \), its Markov boundary includes its direct parents ( \( X_1 \) and \( X_3 \) ) because conditioning on \( X_2 \) does not block the dependency between \( X_1 \) and \( X_3 \), and they are not conditionally independent of \( X_2 \) given each other.  

Thus, the Markov boundary for the collider \( X_2 \) (or the set containing \( X_2 \)) is the set of its parent variables.  

The answer is \(<|begin_of_box|>\{X_1, X_3\}<|end_of_box|>\).</answer>