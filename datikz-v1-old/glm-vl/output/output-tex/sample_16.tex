<think>Got it, let's try to parse this problem step by step. First, the question is about illustrating Case 1, involving vertices y_ and x_1, and their roles in collecting vertices from V(Q^)-V_1 and V(Q^)-V_+1 respectively. 

First, let's recall some graph theory terms. V(Q^) probably refers to the vertex set of some graph Q^, maybe a quotient graph or a derived graph. V_1 and V_+1 might be subsets of V(Q^), maybe related to some partition or a set of vertices in a previous step. The paths P_1, P_u, x_1 and P_2, P_y_, v are likely paths in the graph, and the out-branching of S(V_1) might be a subgraph related to S(V_1), maybe a star or some structure.

The first part says: "The vertex y_ is used to collect those vertices of V(Q^)-V_1 which are not covered by the paths P_1, P_u, x_1 and out-branching of S(V_1)". So, V(Q^)-V_1 is the set of vertices in Q^ not in V_1. We need to collect (maybe connect or identify) those vertices using y_, which are not covered by those paths. Similarly, x_1 collects vertices from V(Q^)-V_+1 not covered by P_2, P_y_, v.

Wait, maybe there's a typo in the original text. The second part says "Similarly x_1 is used to collect those vertices of V(Q^)-V_+1 which are not covered by the paths P_2, P_y_, v." Wait, the original says "V_+1" – maybe that's a typo, perhaps V_2 or another index? But assuming the notation is as given.

Let me try to structure this. Let's define the sets:

- V(Q^) is the vertex set of graph Q^.
- V_1 and V_+1 (maybe V_2?) are subsets of V(Q^), so V(Q^)-V_1 is the vertices not in V_1, and V(Q^)-V_+1 similarly.
- The paths P_1, P_u, x_1 and P_2, P_y_, v are paths in the graph, maybe starting or ending at certain vertices.
- The out-branching of S(V_1) might be a subgraph where S(V_1) is a star centered at some vertex, and its out-branching is the edges going out from that star.

So, for y_, we take all vertices in V(Q^)-V_1 that are not in the union of the paths P_1, P_u, x_1, and the out-branching of S(V_1). Then, x_1 does the same for V(Q^)-V_+1 with respect to paths P_2, P_y_, v.

To illustrate Case 1, we might need to draw a graph where Q^ has vertices, V_1 and V_+1 are subsets, the paths are drawn, and y_ and x_1 are vertices connected to the remaining vertices not covered by those paths.

But since this is a text-based explanation, let's outline the steps:

1. Start with graph Q^, identify V(Q^), then V_1 and V_+1 as subsets.
2. Draw paths P_1, P_u, x_1 (maybe starting from V_1 or other vertices), and the out-branching of S(V_1) (which would be a star with center in V_1, and edges going out to other vertices).
3. Identify the vertices in V(Q^)-V_1 not on those paths, and connect them to y_.
4. Similarly, for V(Q^)-V_+1, draw paths P_2, P_y_, v, identify the remaining vertices, connect them to x_1.

The key idea is that y_ and x_1 serve as collection points (maybe hubs) for vertices not covered by the specified paths and subgraphs, ensuring those vertices are connected or grouped through these new vertices.

I need to make sure the notation is clear. Maybe V_+1 is a typo for V_2, but assuming the original uses that, we have to work with it. Also, "out-branching" might refer to the edges going out from a vertex, so S(V_1) could be a star with center in V_1, and its out-branching is the edges from the center to other vertices.

In summary, the process involves identifying the remaining vertices after accounting for the paths and a specific subgraph (S(V_1)'s out-branching), then using y_ and x_1 to collect those remaining vertices, ensuring they're connected appropriately.</think><answer>To illustrate Case 1, we analyze the roles of vertices \( y_ \) and \( x_1 \) in collecting uncovered vertices from specific subsets of \( V(Q^) \). Here is a step-by-step breakdown:  

1. **Define Vertex Sets**:  
   - Let \( V(Q^) \) be the vertex set of graph \( Q^ \).  
   - Let \( V_1 \) and \( V_+1 \) (or a similar index, e.g., \( V_2 \)) be subsets of \( V(Q^) \).  
   - The sets \( V(Q^) - V_1 \) and \( V(Q^) - V_+1 \) contain vertices not in \( V_1 \) or \( V_+1 \), respectively.  

2. **Identify Covered Regions**:  
   - For \( y_ \), vertices in \( V(Q^) - V_1 \) are "not covered" by paths \( P_1, P_u, x_1 \) and the *out-branching* of \( S(V_1) \). Here, \( S(V_1) \) likely denotes a star subgraph centered in \( V_1 \), with its "out-branching" being edges extending from this star.  
   - For \( x_1 \), vertices in \( V(Q^) - V_+1 \) are "not covered" by paths \( P_2, P_y_, v \).  

3. **Collect Uncovered Vertices**:  
   - Vertex \( y_ \) serves as a collection point for vertices in \( V(Q^) - V_1 \) not in the union of paths \( P_1, P_u, x_1 \) or the out-branching of \( S(V_1) \). These vertices are connected to \( y_ \) (e.g., via new edges or paths).  
   - Vertex \( x_1 \) similarly collects vertices in \( V(Q^) - V_+1 \) not in paths \( P_2, P_y_, v \), connecting them to \( x_1 \).  

4. **Graph Illustration (Conceptual)**:  
   - Draw \( Q^ \) with subsets \( V_1 \) and \( V_+1 \) highlighted.  
   - Draw paths \( P_1, P_u, x_1 \) (connecting to \( V_1 \)) and \( P_2, P_y_, v \) (connecting to \( V_+1 \)), along with the out-branching of \( S(V_1) \) (edges from a central vertex in \( V_1 \)).  
   - Mark vertices in \( V(Q^) - V_1 \) and \( V(Q^) - V_+1 \) not on these paths, then connect them to \( y_ \) and \( x_1 \), respectively.  

This process ensures \( y_ \) and \( x_1 \) aggregate all vertices not covered by the specified paths and subgraphs, structuring the graph for further analysis (e.g., connectivity, partitioning).  

The answer is \(<|begin_of_box|>y_ \text{ and } x_1 \text{ collect uncovered vertices as described}<|end_of_box|>\).</answer>