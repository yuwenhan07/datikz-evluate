<think>Got it, I need to understand the graph G_100 described here. Let's start by breaking down the notation. The problem says G_100 = K_1,3 P_2 + \{\{4,5\},\{4,6\}\} on six vertices. First, let's recall what each part means.

K_1,3 is a complete tripartite graph with one vertex in the first partition and three in the second, right? So K_1,3 has 1 + 3 = 4 vertices. Then P_2 is a path with two vertices, which is just an edge between two vertices. So K_1,3 P_2 would be connecting K_1,3 to P_2 in some way. Wait, maybe it's the union of K_1,3 and P_2, but need to check how they are connected. Wait, the notation might be that we take K_1,3, then add a path P_2 (which has two vertices, let's say vertices a and b), and then add the edges {4,5} and {4,6}. Wait, but the graph is on six vertices, so let's list the vertices first. The vertices should be labeled, probably 1,2,3,4,5,6 since it's on six vertices.

First, K_1,3 has one vertex connected to three others. Let's say vertex 1 is connected to vertices 2,3,4. Then P_2 is a path of two vertices, maybe vertices 5 and 6 connected by an edge? Wait, but then we need to add the edges {4,5} and {4,6}. Wait, let's parse the expression again: G_100 = K_1,3 P_2 + \{\{4,5\},\{4,6\}\}. The "+" here probably means the union of the graph K_1,3 P_2 with the edges {4,5} and {4,6}, but we need to make sure the graph is on six vertices.

Let's start by constructing K_1,3. K_1,3 has four vertices: one vertex (let's say vertex 1) connected to three vertices (vertices 2, 3, 4). Then P_2 is a path with two vertices, say vertices 5 and 6 connected by an edge (so P_2 is the edge between 5 and 6). Now, K_1,3 P_2 would be connecting K_1,3 to P_2. How are they connected? Usually, when you have a graph G * H, it's the lexicographical product, but maybe here it's the union of K_1,3 and P_2 with an edge between the last vertex of K_1,3 (which is 4) and the first vertex of P_2 (which is 5)? Wait, if K_1,3 is connected to P_2 via vertex 4 connected to 5, then K_1,3 P_2 would have vertices 1,2,3,4,5,6, with edges: 1-2, 1-3, 1-4, 2-3, 2-4, 3-4 (from K_1,3), and 5-6 (from P_2), plus 4-5 (connecting K_1,3's vertex 4 to P_2's vertex 5). Wait, but then the problem says to add \{\{4,5\},\{4,6\}\}, which are the edges between 4-5 and 4-6. Wait, if we already have 4-5 from connecting K_1,3 to P_2, then adding 4-6 would add another edge from 4 to 6. So let's list all vertices as 1,2,3,4,5,6.

First, K_1,3: vertices 1 connected to 2,3,4. So edges: (1,2), (1,3), (1,4), (2,3), (2,4), (3,4) – wait, K_1,3 is a complete tripartite graph, so the edges are only between the first partition (vertex 1) and the second partition (2,3,4), and no edges within the partitions. So edges are (1,2), (1,3), (1,4), (2,3), (2,4), (3,4)? Wait, no, wait: K_1,3 has partitions of size 1 and 3, so edges are between the single vertex and each of the three, and between the three vertices among themselves? Wait, no, complete tripartite graph K_{1,3} has edges only between the first partition and the second partition, not within the partitions. So edges are (1,2), (1,3), (1,4), (2,3), (2,4), (3,4)? Wait, no, if it's K_{1,3}, then the edges are between the first vertex and each of the three, and between the three vertices two by two? Wait, no, K_{1,3} is a complete bipartite graph between a single vertex and three vertices, so edges are (1,2), (1,3), (1,4), (2,3), (2,4), (3,4) – yes, because in a complete tripartite graph with partitions of size 1,3,0 (if there's a third partition), but here it's K_1,3, so partitions are 1 and 3, so edges between 1 and each of 2,3,4, and between 2,3,4 among themselves? Wait, no, complete tripartite graph K_{a,b,c} has edges between each partition, so K_1,3 would have edges between the first vertex and each of the next three, and between the three vertices two by two, but not within the first partition (which has only one vertex) or the third partition (if there was one, but here it's K_1,3, so third partition size 0? Wait, maybe I'm confusing. Let's check: K_1,3 is a star graph with one central vertex connected to three leaves, and the leaves form a complete graph among themselves. So the leaves are connected to each other, so edges are (2,3), (2,4), (3,4), plus the central vertex connected to each leaf: (1,2), (1,3), (1,4). So total edges: 3 (from central to leaves) + 3 (among leaves) = 6 edges.

Then P_2 is a path of two vertices, say vertices 5 and 6 connected by an edge (5-6). Now, K_1,3 P_2 – what does this mean? It's the lexicographical product of K_1,3 and P_2. The lexicographical product of two graphs G and H is the graph such that the vertex set is V(G) × V(H), and two vertices (u, v) and (u', v') are adjacent if either u = u' and v is adjacent to v' in H, or v = v' and u is adjacent to u' in G. Wait, but maybe in this case, since P_2 has two vertices, say 5 and 6, the lexicographical product would connect each vertex in K_1,3 to each vertex in P_2, and also connect the vertices within P_2 if they are connected in H. Wait, P_2 is a path with two vertices connected by an edge, so in the lexicographical product, for each vertex in K_1,3, we connect it to both vertices in P_2, and also connect the two vertices in P_2 if they are connected in P_2 (which they are, since P_2 is a path, so 5-6 is an edge). So the vertex set would be {1,2,3,4} × {5,6}, so 4×2=8 vertices? Wait, but the problem says it's on six vertices. Hmm, maybe I'm misunderstanding the notation. Alternatively, maybe G_100 = K_1,3 + P_2 + \{\{4,5\},\{4,6\}\}, but need to check the original problem again.

Wait, the problem says "Graph G_100 = K_1,3 P_2 + \{\{4,5\},\{4,6\}\} on six vertices." The notation might mean that we start with K_1,3, then add P_2, then add the edges {4,5} and {4,6}. But K_1,3 has four vertices, P_2 has two vertices, so together they have 4 + 2 = 6 vertices, which matches the "on six vertices" part. So vertices are 1,2,3,4,5,6. K_1,3 is on vertices 1,2,3,4 (as before), P_2 is on vertices 5,6. Then adding the edges {4,5} and {4,6} connects vertex 4 (from K_1,3) to vertices 5 and 6 (from P_2). So let's list all edges:

First, K_1,3 edges: (1,2), (1,3), (1,4), (2,3), (2,4), (3,4) – that's the complete tripartite graph as before.

Then P_2 is a path of two vertices, which would be (5,6), but wait, if we're adding P_2 to K_1,3, do we need to connect them? Maybe the notation K_1,3 P_2 means the union of K_1,3 and P_2, but then add the edges {4,5} and {4,6}. Wait, let's try constructing the graph step by step.

1. Start with K_1,3: vertices 1,2,3,4 with edges as above.
2. Add P_2: vertices 5,6 connected by (5,6). So now the graph has edges from K_1,3 and P_2, but no connection between K_1,3 and P_2 yet.
3. Then add the edges {4,5} and {4,6}, which connect vertex 4 (from K_1,3) to vertices 5 and 6 (from P_2). So now, the edges are:
   - K_1,3 edges: (1,2), (1,3), (1,4), (2,3), (2,4), (3,4)
   - P_2 edge: (5,6)
   - Added edges: (4,5), (4,6)

Wait, but does P_2 need to be connected to K_1,3 in a specific way? If K_1,3 P_2 is the lexicographical product, then each vertex in K_1,3 is connected to both vertices in P_2, and the vertices in P_2 are connected if they are connected in P_2. Since P_2 is a path of two vertices connected by an edge, in the lexicographical product, the two vertices in P_2 would be connected, and each vertex in K_1,3 would be connected to both vertices in P_2. So the vertex set would be 4×2=8 vertices, but the problem says six vertices, so maybe it's a different product. Alternatively, maybe K_1,3 P_2 is a graph formed by taking K_1,3 and connecting it to P_2 via one vertex, say vertex 4 connected to both 5 and 6, making it a star with center 4, connected to 5 and 6, plus the edges of K_1,3 and P_2. Wait, let's check the number of vertices: K_1,3 has 4 vertices, P_2 has 2 vertices, total 6 vertices, which matches. So the graph is formed by taking K_1,3 (vertices 1,2,3,4) and P_2 (vertices 5,6), and connecting vertex 4 (from K_1,3) to both 5 and 6, and also including the edges of P_2 (which is 5-6) and the edges of K_1,3. Wait, but if P_2 is just 5-6, then adding the connections from 4 to 5 and 4 to 6 would make vertex 4 connected to 5 and 6, and 5 connected to 6. So the edges would be:

From K_1,3: (1,2), (1,3), (1,4), (2,3), (2,4), (3,4) – that's 6 edges.

From P_2: (5,6) – 1 edge.

Added edges: (4,5), (4,6) – 2 edges.

Total edges: 6 + 1 + 2 = 9 edges.

Let me verify the vertex connections:

- Vertex 1: connected to 2, 3, 4 (from K_1,3)
- Vertex 2: connected to 1, 3, 4 (from K_1,3)
- Vertex 3: connected to 1, 2, 4 (from K_1,3)
- Vertex 4: connected to 1, 2, 3, 5, 6 (from K_1,3 and added edges)
- Vertex 5: connected to 4, 6 (from added edges and P_2 edge)
- Vertex 6: connected to 4, 5 (from added edges and P_2 edge)

Is that correct? Let's check each vertex's degree:

- Vertex 1: degree 3 (connected to 2,3,4)
- Vertex 2: degree 3 (connected to 1,3,4)
- Vertex 3: degree 3 (connected to 1,2,4)
- Vertex 4: degree 5 (connected to 1,2,3,5,6) – wait, K_1,3 has vertex 4 connected to 1,2,3, so that's 3 connections, plus 5 and 6, total 5
- Vertex 5: degree 2 (connected to 4,6)
- Vertex 6: degree 2 (connected to 4,5)

Wait, but K_1,3 is a complete tripartite graph, so vertex 4 should be connected to 1,2,3, which are the other vertices in its partition, and not connected to 5 and 6 unless we added those edges. The problem says "G_100 = K_1,3 P_2 + \{\{4,5\},\{4,6\}\}", so the "+ \{\{4,5\},\{4,6\}\}" means adding those two edges to the graph K_1,3 P_2. So first, what is K_1,3 P_2? If it's the lexicographical product, then as a graph, it would have vertices as pairs (u, v) where u is from K_1,3 and v is from P_2. So K_1,3 has vertices 1,2,3,4, P_2 has 5,6. So the vertex set is (1,5), (1,6), (2,5), (2,6), (3,5), (3,6), (4,5), (4,6), total 8 vertices. But the problem says it's on six vertices, so maybe it's not the full lexicographical product but a different construction. Alternatively, maybe K_1,3 P_2 is a graph formed by connecting K_1,3 to P_2 at vertex 4, making it a graph with vertices 1,2,3,4,5,6, where K_1,3 is on 1-4 and P_2 is on 5-6, connected by 4-5 and 4-6, and P_2 has its own edge 5-6. That would make sense with six vertices. So the edges are:

From K_1,3: (1,2), (1,3), (1,4), (2,3), (2,4), (3,4)

From P_2: (5,6)

Added edges: (4,5), (4,6)

Total edges: 6 + 1 + 2 = 9, as before.

Now, let's confirm if this graph is correct. Each vertex's connections:

1: connected to 2,3,4
2: connected to 1,3,4
3: connected to 1,2,4
4: connected to 1,2,3,5,6
5: connected to 4,6
6: connected to 4,5

Yes, that seems right. I think that's the graph described. So the final graph has six vertices with the specified edges.</think><answer>To construct the graph \( G_{100} = K_{1,3} P_2 + \{\{4,5\},\{4,6\}\} \) on six vertices, start with \( K_{1,3} \), a complete tripartite graph with vertices \( \{1,2,3,4\} \), where vertex 1 connects to 2, 3, 4, and 2, 3, 4 form a complete subgraph among themselves. Add \( P_2 \), a path with vertices \( \{5,6\} \) connected by \( \{5,6\} \), then add edges \( \{4,5\} \) and \( \{4,6\} \) connecting vertex 4 to 5 and 6. The resulting graph has vertices \( \{1,2,3,4,5,6\} \) with edges:  
- From \( K_{1,3} \): \( (1,2), (1,3), (1,4), (2,3), (2,4), (3,4) \),  
- From \( P_2 \): \( (5,6) \),  
- Added edges: \( (4,5),