<think>Got it, let's try to parse this question. The user is asking about the relative price development of A and B shares between S and T₁ compared to the underlying redeemable asset C, with a graph on a whiteboard having red/blue lines labeled with "interest," "P," "invest," "divest." First, I need to recall what these terms might mean in finance, maybe related to convertible securities or options, where shares (A, B) are related to an underlying asset (C), and the graph shows price movements.

First, identify the components: shares A and B, underlying asset C, time points S and T₁. The graph has lines (maybe representing price paths) with labels like interest, P (maybe price), invest, divest. So, the question is about how the prices of A and B change relative to C between S and T₁.

Assuming the graph shows that during the period, the shares' prices might move relative to C. For example, if the red line is A and blue is B, their price development compared to C's price. The labels "interest" might indicate interest rate effects, "P" could be the price of C or the shares, "invest" and "divest" are actions (buying/selling). So, the relative price development would show how A and B's prices change compared to C's, maybe A is more sensitive, B less, or vice versa, depending on the graph's slope.

But since the image is described as a whiteboard with a red/blue line labeled with those terms, maybe the graph is a price vs time graph, with the red line (A) and blue line (B) showing their prices relative to C's price (which might be a horizontal line or another curve). The points along the line could be key moments: S (start) to T₁ (end), with "invest" and "divest" indicating when investors buy or sell, "interest" might be the interest rate impact, "P" is the price level.

So, the relative price development would compare the percentage change of A and B's prices to C's price between S and T₁. If, for example, A's price increased more than C, then A is outperforming C, and B less so, or vice versa. The labels might indicate that at certain points (like "invest" time), the price of A or B changes relative to C, and "divest" is when they change back.

Since the question is about "relative price development," the answer would likely describe how A and B's prices moved compared to C's, maybe stating that A's price relative to C increased more than B's, or both increased but at different rates, or one decreased while the other increased relative to C.

But without the actual image, I need to make logical assumptions based on typical financial graphs. Convertible securities (like A, B shares related to C, a redeemable asset) often have prices that are linked to C's price, with features like conversion rights. The graph might show that during the period, the price of A and B relative to C fluctuates, with "invest" and "divest" indicating when investors' actions affect the prices, and "interest" representing interest rate effects (which can impact convertible securities' prices).

So, summarizing, the relative price development would show that between S and T₁, the price of A (and/or B) relative to C changed in a certain way—maybe A's price increased more than C, B's price increased less, or both increased but at different rates. The labels on the graph (interest, P, invest, divest) help identify the factors affecting the price changes: interest rates might have an upward or downward effect, "invest" and "divest" times might correspond to buying/selling pressure altering the relative prices.

Given the lack of the actual image, but based on standard financial concepts, the relative price development of A and B shares compared to C would likely show that their prices moved in relation to C's, possibly with A being more sensitive to changes (like interest rates) than B, or both following C's price but with different magnitudes. The key is the comparison between A/B and C's price changes over S to T₁.</think><answer>To analyze the relative price development of shares A and B compared to the underlying asset C between time points S and T₁, we interpret the graph’s components (labeled “interest,” “P,” “invest,” “divest”) as follows:  

1. **Underlying Asset C**: Acts as the reference for comparing A and B’s prices.  
2. **Shares A and B**: Represent securities linked to C (e.g., convertible shares), with their prices relative to C determined by factors like interest rates, investor actions, and market conditions.  
3. **Graph Labels**:  
   - “Interest” likely reflects interest rate effects (which influence convertible securities’ prices).  
   - “P” probably denotes a price metric (e.g., C’s or A/B’s price).  
   - “Invest” and “Divest” indicate key moments when investors buy/sell, altering relative prices.  

Between S and T₁, the relative price development would show how A and B’s prices changed relative to C. For example, if the red/blue lines (representing A/B) slope differently than C’s price line, A might outperform C (higher relative price increase) while B underperforms, or vice versa. The “invest” and “divest” points mark when buying/selling pressure shifts relative prices, while “interest” influences the overall trend.  

Without the exact image, a standard interpretation is that **A and B’s prices relative to C changed in response to interest rates and investor actions, with A (or B) showing a greater/lesser relative price movement** during S–T₁.  

The answer is <|begin_of_box|>The relative price development shows A and B shares’ prices changed relative to C, with A (or B) outperforming (or underperforming) C between S and T₁, influenced by interest rates and investor actions.<|end_of_box|></answer>