<think>Got it, let's try to tackle this problem step by step. First, I need to recall what the question is asking: a composition of transformations to map v_3^(B) to v_3^(A) through two intermediate bases, each step involving an integral over the principal series with a kernel proportional to a 6j symbol. 

First, let's parse the components. We have two bases, A and B, for the vector space v_3. The transformations between them are through two intermediate bases, let's call them maybe C and D, so the path is B → C → D → A, or something like that. Each step (B→C and C→D, or C→D and D→A) involves an integral over the principal series, with the kernel being a 6j symbol. 

I remember that 6j symbols are used in angular momentum coupling, related to Clebsch-Gordan coefficients, and they appear in integrals over group representations, especially in the principal series which are unitary representations. The principal series usually involves integrals over the group with a kernel like the character of the representation divided by the volume element, and the 6j symbols might come into play when dealing with tensor products of representations, which are common in such transformations.

Now, the composition of transformations would be the product of the transformations from each step. If each step is a linear transformation represented by an integral with a 6j symbol, then the composition would be the product of these integrals. But I need to think about how the 6j symbols are used in the transformation between bases. 

Each transformation between bases can be seen as a change of basis matrix, which in the context of group representations might be related to the matrix elements of the representation. The integral over the principal series would involve integrating the kernel (proportional to the 6j symbol) over the group, which is a common technique in representation theory to construct intertwining operators or change of basis matrices.

Let's consider the general form. Suppose the transformation from basis B to C is given by an integral: T_{B→C} = ∫ K_{B→C}(g) dg, where K is proportional to a 6j symbol. Similarly, T_{C→A} = ∫ K_{C→A}(g) dg. Then the composition T_{B→A} = T_{C→A} ∘ T_{B→C} = ∫ ∫ K_{C→A}(g2) K_{B→C}(g1) dg2 dg1, but maybe there's a way to combine them using the properties of 6j symbols, which are symmetric under permutation of their indices and satisfy certain recoupling relations.

Wait, the problem mentions "the appropriate 6j symbol in the figure", but since we don't have the figure, we need to assume a general form. Let's recall that 6j symbols are used in the recoupling of three angular momenta, and in the context of group theory, they might appear in the transformation between different bases when the bases are labeled by different irreducible representations. 

If v_3^(B) and v_3^(A) are both 3-dimensional representations (since v_3 usually denotes a 3-dimensional space), then the transformation between them would involve a 3x3 matrix. Each step through an intermediate basis would involve a 3x3 matrix, and the composition is the product of these matrices. 

The integral over the principal series might be related to the Plancherel measure or some other invariant measure on the group, and the kernel with the 6j symbol could be a specific intertwining operator. The composition of two such transformations would then be the product of the two intertwining operators, which might be expressed as another integral involving a combination of 6j symbols, possibly using the 9j symbol or other recoupling coefficients, but since we have two steps, maybe the composition is a single integral with a 6j symbol that's a product of the two original ones, or a combination.

Alternatively, since each transformation is an integral with a 6j symbol, the composition would be the product of the two integrals, which in the context of group representations might be another integral with a 6j symbol that's the contraction or combination of the two original ones. 

Another angle: in the theory of spherical tensors or in quantum mechanics, changing bases between different irreps often involves Clebsch-Gordan coefficients, which are related to 3j symbols, and 6j symbols are used for more complex recouplings. If each step involves a 6j symbol, then the composition would involve a sequence of recouplings, which can be expressed as a product of 6j symbols or a single one via the 9j symbol, but since we have two steps, maybe the composition is a 6j symbol that's a combination of the two intermediate ones.

Wait, the problem says "a composition of transformations to map v_3^(B) to v_3^(A) going through two intermediate bases". So the path is B → C → D → A, with C and D as intermediate. Each step (B→C and C→D, or C→D and D→A) involves an integral with a 6j symbol. So the first transformation is B→C: T1 = ∫ K1(g) dg, where K1 ~ 6j symbol, then C→D: T2 = ∫ K2(g) dg, then D→A: T3 = ∫ K3(g) dg. The composition is T3 ∘ T2 ∘ T1, which would be ∫ ∫ ∫ K3(g3) K2(g2) K1(g1) dg3 dg2 dg1. But if the 6j symbols are related, maybe the composition can be written as a single integral with a combined 6j symbol, perhaps using the fact that 6j symbols can be multiplied or combined via the 9j symbol when multiple recouplings are involved.

Alternatively, since each kernel is proportional to a 6j symbol, the composition of two such transformations would be proportional to the product of the two 6j symbols, which might be another 6j symbol or a 9j symbol. But without the specific 6j symbols from the figure, it's hard to say exactly, but the general idea is that the composition of two transformations each involving a 6j symbol integral would result in a transformation involving a combination of 6j symbols, possibly a product or a single one through recoupling.

Another thought: in the principal series, the integral might be over the variable related to the parameter of the principal series representation, and the kernel involving the 6j symbol could be a specific intertwining operator that maps one representation to another, and composing two such operators gives a map between the original representations, which would be the composition of the two transformations.

Given that the problem states "the appropriate 6j symbol in the figure", which we can't see, but assuming that each step uses a 6j symbol, the composition would involve the product of the two 6j symbols, or a single 6j symbol that's a combination of them. In many cases, when you compose two recouplings, you get a higher recoupling symbol, like a 9j symbol, but since we have two steps, maybe the final transformation is a 6j symbol that's a combination of the two intermediate ones.

Alternatively, if each transformation is a change of basis matrix, then the composition is the product of the matrices, which in terms of group representations would be the matrix product of the two change-of-basis matrices, each of which is an integral with a 6j symbol. The product of two such matrices might be another integral with a 6j symbol, possibly using the fact that 6j symbols are unitary or satisfy certain multiplication rules.

Given the lack of specific details from the figure, but based on standard representation theory, the composition of two transformations each involving a 6j symbol integral would result in a transformation involving a combined 6j symbol, likely through the use of recoupling coefficients, and the final transformation from v_3^(B) to v_3^(A) would be an integral over the principal series with a kernel proportional to the combined 6j symbol, which is the product or combination of the two intermediate 6j symbols.</think><answer>To determine the composition of transformations mapping \( v_3^{(B)} \) to \( v_3^{(A)} \) through two intermediate bases, we analyze the structure of each transformation step. Each step involves an integral over the principal series with a kernel proportional to a 6j symbol, which is used for recoupling angular momentum representations.  

1. **Intermediate Bases and Transformations**: Let the intermediate bases be \( C \) and \( D \), with transformations \( T_{B \to C} \) and \( T_{C \to D} \), followed by \( T_{D \to A} \). Each transformation \( T \) is an integral:  
   \[
   T = \int K(g) \, dg,
   \]  
   where \( K \propto 6j \)-symbol. The 6j symbol encodes the recoupling of three angular momenta, essential for changing bases between different irreducible representations.  

2. **Composition of Transformations**: The composition \( T_{B \to A} = T_{D \to A} \circ T_{C \to D} \circ T_{B \to C} \) combines the integrals. Using the properties of 6j symbols (recoupling relations, symmetry), the product of two 6j symbols can be expressed as a single combined 6j symbol or a higher recoupling symbol (e.g., 9j). For two intermediate steps, the final transformation involves a kernel proportional to the combined 6j symbol, which accounts for the sequential recoupling through \( C \) and \( D \).  

3. **Principal Series Integral**: The integral over the principal series ensures the transformation respects the unitary and irreducible properties of the representations. The kernel’s proportionality to the 6j symbol guarantees the transformation maps between the specified bases while preserving the representation structure.  

The final composition is a single integral over the principal series with a kernel proportional to the combined 6j symbol, which encodes the sequential recoupling through the intermediate bases.  

The answer is \(<|begin_of_box|>\text{A composition involving a combined 6j symbol integral over the principal series}<|end_of_box|>\).</answer>