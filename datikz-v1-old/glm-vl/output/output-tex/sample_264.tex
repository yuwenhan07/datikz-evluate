<think>Got, let's try to parse the question first. The user is asking about the existence of a convex point on a capillary hypersurface supported on P, with a description of an image involving spheres and circles. Hmm, first, I need to recall what a capillary surface is—those are surfaces that minimize surface energy with a contact line, often in the context of soap films or liquid interfaces. A convex point is a point where the surface curves outward, like the top of a dome. The question is about whether such a point exists on a capillary surface supported on a set P, which in the image description might be the boundary (the circles and spheres' openings). 

First, let's think about the general theory of capillary surfaces. For a capillary surface supported on a closed curve (like the boundary of the outer circle or the inner sphere's opening), the existence of convex points depends on the geometry of the support. If the support has regions where the curvature can be positive, then convex points might exist. The image has an inner sphere with an opening at the bottom, outer sphere with an opening at the top, so the support P might be the union of the boundaries of the inner and outer spheres, or the circles. 

In the case of a capillary surface between two spheres (like a soap film between two concentric spheres), the surface can have various shapes. If the surface is a sphere itself, it's convex everywhere. But if it's a more complex surface, like a catenoid-like shape, but between spheres, the minimal surface might have regions of positive and negative curvature. However, the existence of a convex point would require that there's a point where the mean curvature is positive (since convexity relates to positive curvature). 

Wait, the question is about a capillary hypersurface (so in higher dimensions, but maybe the image is 3D, so 2D surface). The key is that if the support P has a region where the surface can bulge outward, then a convex point exists. If the support is such that the surface is forced to curve inward everywhere, then maybe not. But in the image description, the inner sphere has an opening at the bottom, outer at the top—so the capillary surface might connect the bottom of the inner sphere to the top of the outer sphere, passing through the space between them. If the surface can form a "hill" in some region, then a convex point exists. 

Alternatively, using the theory of minimal surfaces, a convex point on a minimal surface (which capillary surfaces are, in the absence of surface tension differences) requires certain curvature conditions. If the support allows for a point where the surface curves outward, then such a point exists. Given that the image has a more complex structure with openings, it's likely that the capillary surface can have a convex point, especially if the support has regions where the surface isn't constrained to be concave everywhere. 

So, putting it together: the existence of a convex point on a capillary hypersurface supported on P depends on the geometry of P. If P allows for the surface to have a region of positive Gaussian or mean curvature (depending on the dimension), then such a point exists. Given the image's description with spheres and openings, it's plausible that a convex point exists, as the surface can bulge outward in some area between the spheres.</think><answer>To determine the existence of a convex point on a capillary hypersurface supported on \( P \), we analyze the geometric and theoretical context:  

1. **Capillary Surfaces and Convexity**: Capillary surfaces minimize surface energy with a contact line, analogous to soap films. A "convex point" is a point where the surface curves outward (positive mean curvature in 2D, or positive principal curvatures in higher dimensions).  
2. **Support Geometry**: The image describes a capillary surface between two spheres (inner with a bottom opening, outer with a top opening), implying \( P \) is the union of their boundaries. For such a support, the capillary surface can bulge outward in regions not strictly constrained by concave boundaries.  
3. **Theoretical Framework**: Minimal surfaces (a subset of capillary surfaces) can have convex points if the support allows for regions of positive curvature. Given the non-uniform openings (bottom and top), the surface can form a "hill" in the space between the spheres, creating a convex point.  

Thus, under the described geometry, a convex point on the capillary hypersurface is **likely to exist** due to the flexibility in curvature provided by the non-uniform support.  

The answer is \(<|begin_of_box|>Yes<|end_of_box|>\).</answer>