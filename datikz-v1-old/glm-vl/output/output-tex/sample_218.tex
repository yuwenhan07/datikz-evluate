<think>Got it, let's try to parse this example step by step. First, the question is about an embeddable word with an embedding, specifically related to a tree structure T_(5,9,10), paths, binomial numbers, and labels. Let's break down the components mentioned:

1. **Embeddable Word and Embedding**: The example is about a word (probably a sequence of symbols) that can be embedded in some structure, maybe a tree or graph. The embedding would show how the word is placed, with edges labeled (in red here) and other details.

2. **Tree T_(5,9,10)**: This is likely a tree labeled with these numbers, maybe as parameters for the tree's structure (like number of vertices, edges, etc.). Trees are often used in combinatorial structures, so T might denote a tree with certain properties related to 5, 9, 10.

3. **Path root-to-leaf (7,2),(7,1,3),(7,1,4)**: This path is a sequence of nodes from the root to a leaf. Each tuple like (7,2) might indicate a node with label 7, then a child with label 2, etc. The path represents a sequence of choices or extensions.

4. **Binomial numbers and k**: The binomial numbers correspond to the number of extensions via their root-to-leaf path. The "k" indicates the size of the extended edges. So, for each step in the path, the binomial coefficient might count the number of ways to extend, and k is the number of edges being extended.

5. **Leaf vertices' labels**: The labels on leaf vertices are the cardinalities of partition sets. Each set in B_i (maybe a partition) has two vertices related, one being a leaf, and in general, there might be more.

Let's try to reconstruct the example. Suppose we have a tree T with root labeled, then branches. The path (7,2) means from root, go to node 7, then to node 2 (a leaf maybe). Then (7,1,3) would be from root to 7, then 1, then 3 (leaf), and (7,1,4) is root to 7, 1, then 4 (leaf). The extensions of (5,9,10) containing also 2 and 3, but not containing 7,1,4. Wait, the note says "extensions of (5,9,10) containing also the 2 and the 3, but not containing 7, 1 or 4." So the original word is (5,9,10), and extensions add elements 2,3 but exclude 7,1,4.

Also, edges labeled red, others unlabeled (label .). Binomial numbers on edges, k is size of extended edges. Each set in B_i has two vertices, one leaf, general case may have more.

Maybe the example is a small tree where each node has children, and the path shows how the extensions are counted via binomial coefficients along the path. For example, if the path has nodes with binomial coefficients like C(n,k), then the product gives the total number of extensions.

But since the user is asking for an example of an embeddable word with an embedding, maybe the word is a sequence like "a b c" embedded in the tree, with edges labeled red, and the path shows how the embedding follows the tree structure, with binomial numbers indicating the number of ways to extend each edge, and leaf labels showing partition set sizes.

However, without a visual, we have to describe based on the text. The key points are:

- Tree T with parameters (5,9,10), maybe 5 is number of leaves, 9 is something else, 10 is another parameter.
- Path from root to leaf: first step (7,2), then (7,1,3), then (7,1,4). So the root has a child 7, which has children 2,1, etc. The path goes through 7→2 (first edge), then 7→1→3 (second edge), then 7→1→4 (third edge). The extensions correspond to choosing 2 and 3 as additional elements, not choosing 1,4,7.
- Binomial numbers on edges: maybe each edge in the path has a binomial coefficient, like C(5,2) or similar, indicating the number of extensions. The k is the size of the edge set, maybe 2 for some edges.
- Leaf labels are partition set cardinals: each leaf has a number, say 2,3,4, which are the sizes of the sets in the partition.

Putting it all together, the example would show a tree with edges labeled (some red, some unlabeled), a path from root to leaf with binomial numbers on edges, and leaf labels as partition sizes, illustrating how extensions are counted via the path and binomial coefficients.</think><answer>To understand the example of an embeddable word with an embedding in the context of the provided text, let’s break down the key components and reconstruct the example:  


### 1. **Tree Structure \( T_{(5,9,10)} \)**  
This is a tree labeled with parameters \( (5,9,10) \), likely representing a combinatorial structure (e.g., number of leaves, edges, or other properties). Trees are used here to model the “extensions” of a word, with edges and nodes defining the structure of possible extensions.  


### 2. **Path Root-to-Leaf: \( (7,2),(7,1,3),(7,1,4) \)**  
This path describes a sequence of nodes from the root to a leaf in the tree. Each tuple like \( (7,2) \) indicates moving from a parent node (here, 7) to a child node (here, 2). The path represents a “root-to-leaf” traversal, where each step corresponds to a choice in extending the word.  
- \( (7,2) \): From the root, go to node 7, then to leaf node 2.  
- \( (7,1,3) \): From the root, go to node 7 → node 1 → leaf node 3.  
- \( (7,1,4) \): From the root, go to node 7 → node 1 → leaf node 4.  

The note clarifies: *“represents the extensions of \( (5,9,10) \) containing also the 2 and the 3, but not containing 7, 1 or 4”*. This means the original word \( (5,9,10) \) is extended by adding elements 2 and 3 (from the leaf nodes) while excluding 7, 1, and 4.  


### 3. **Binomial Numbers and \( k \) (Size of Extended Edges)**  
- **Binomial numbers**: These correspond to the number of extensions along the root-to-leaf path. For example, if an edge in the path has a binomial coefficient \( \binom{n}{k} \), it indicates \( n \) choices for \( k \) elements.  
- **\( k \)**: Indicates the size of the “extended edges” (i.e., the number of edges being extended at each step). For instance, if \( k = 2 \), two edges are extended in a step.  


### 4. **Edge Labels and Leaf Vertex Cardinals**  
- **Edges labeled red**: Some edges in the tree are highlighted (labeled red), while others have no label (label “.”).  
- **Leaf vertex labels**: The labels on leaf vertices are the cardinalities (sizes) of partition sets. Each set in the partition \( B_i \) has two vertices related to it (one being a leaf), though in general, there may be more vertices per set.  


### Example Reconstruction (Conceptual)  
Imagine a small tree with root labeled, branches to nodes 7, 1, etc. The path \( (7,2),(7,1,3),(7,1,4) \) would show a route from the root to leaves 2, 3, 4. Edges along this path might have binomial numbers (e.g., \( \binom{5}{2} = 10 \) for an edge, indicating 10 extensions), with \( k = 2 \) for some edges. Leaf vertices 2, 3, 4 would have labels (e.g., 2, 3, 4) representing the sizes of partition sets. Edges in the path (like the one to leaf 2) might be red, while others are unlabeled.  


In summary, the example illustrates how a word \( (5,9,10) \) is embedded in a tree \( T_{(5,9,10)} \), with extensions counted via binomial coefficients along a root-to-leaf path, and leaf labels showing partition set sizes. The edges’ labels (red or “.”) and \( k \) values detail the structure of extensions.  

The answer is \(<|begin_of_box|>T_{(5,9,10)} \text{ with path } (7,2),(7,1,3),(7,1,4) \text{ showing extensions of } (5,9,10) \text{ containing 2,3 but not 7,1,4}<|end_of_box|>\).</answer>