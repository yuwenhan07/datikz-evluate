<think>Got it, let's try to break down this problem. The user is describing a proposal for an invertible mapping f between a D-dimensional space and a convex polytope, using a differentiable Voronoi tessellation. They mention adding discrete structure to normalizing flows, and that both the inverse and log determinant Jacobian can be efficiently computed. 

First, I need to make sure I understand each component. An invertible mapping (like a normalizing flow) is used in machine learning to transform a simple distribution (like Gaussian) into a complex one. Adding discrete structure would mean incorporating some kind of discrete partition, which Voronoi tessellation provides—each cell is a convex polytope, right? Voronoi tessellation divides a space into regions based on proximity to points (generators), and if those generators are chosen appropriately, the cells can be convex polytopes. 

So the mapping f would take a point in ^D and map it to a polytope, or maybe to a representative point in the polytope? Wait, the problem says "mapping between ^D and a convex polytope"—wait, maybe it's a mapping from ^D to the polytope, but invertible, so maybe each point in the polytope corresponds to a point in ^D? Wait, the wording is "an invertible mapping f between ^D and a convex polytope". Hmm, maybe the polytope is a subset of ^D, so f: ^D → polytope, and f⁻¹: polytope → ^D. But need to check the details. 

Differentiable Voronoi tessellation: Voronoi diagrams are usually not differentiable everywhere, but if the generators are chosen in a differentiable way, maybe the tessellation can be made differentiable. So the mapping f is parameterized by this differentiable Voronoi tessellation. The key points are invertibility, adding discrete structure (from the Voronoi cells, which are discrete partitions), and efficient computation of inverse and log determinant Jacobian. 

Normalizing flows require that the mapping and its inverse are easy to compute, and the log determinant of the Jacobian is computable for probability density estimation. So this proposal is adding a new type of invertible mapping (using Voronoi tessellation) to normalizing flows, which introduces discrete structure (since Voronoi cells are discrete partitions) while maintaining efficiency. 

I need to make sure the explanation covers the main elements: invertible mapping, convex polytope, differentiable Voronoi tessellation, adding discrete structure to normalizing flows, efficient inverse and log determinant. Maybe also mention why Voronoi tessellation is useful here—its convex cells, differentiability if generators are chosen properly, and how the mapping works. 

So, putting it all together: The proposal introduces an invertible mapping f that maps from a D-dimensional space to a convex polytope (or between ^D and the polytope, with inverse going back), parameterized by a differentiable Voronoi tessellation. Voronoi tessellation provides a discrete partition of the space into convex cells (polytopes), and by using a differentiable version of this tessellation, the mapping f can incorporate this discrete structure. In normalizing flows, which typically use continuous mappings, adding such discrete structure can help model complex distributions more effectively. The invertibility of f (and its inverse f⁻¹) allows for efficient computation of the inverse mapping, and the log determinant of the Jacobian can be efficiently computed due to the properties of the Voronoi tessellation and the differentiable structure. 

I should check if there are standard terms or concepts here. Voronoi tessellation in machine learning is sometimes used for partitioning, and differentiable Voronoi diagrams have been studied, especially in the context of differentiable programming. Normalizing flows with discrete structure might refer to flows that include discrete variables or partitions, which can be useful for modeling distributions with complex geometry. 

So, summarizing the key points in the explanation: the mapping is invertible, uses a differentiable Voronoi tessellation to parameterize it, maps to a convex polytope (or between ^D and the polytope), adds discrete structure to normalizing flows, and both inverse and log determinant Jacobian are efficiently computable.</think><answer>To address the proposal of an invertible mapping \( f \) between \( \mathbb{R}^D \) and a convex polytope, parameterized by a differentiable Voronoi tessellation, here’s a structured explanation:  


### 1. Core Components of the Proposal  
- **Invertible Mapping**: The mapping \( f: \mathbb{R}^D \to \text{convex polytope} \) (or between \( \mathbb{R}^D \) and the polytope, with inverse \( f^{-1} \)) is designed to be bijective (invertible), a key requirement for normalizing flows (which use invertible mappings to transform simple distributions into complex ones).  
- **Differentiable Voronoi Tessellation**: The mapping is parameterized by a *differentiable* Voronoi tessellation of \( \mathbb{R}^D \). Voronoi tessellation divides space into convex cells (polytopes) based on proximity to generators (e.g., points or hyperplanes). When made differentiable (via careful choice of generators or smooth approximations), this tessellation provides a discrete, structured partition of \( \mathbb{R}^D \).  
- **Convex Polytope**: The target space is a convex polytope (a bounded intersection of half-spaces), which is a natural choice for the mapping’s codomain. Each cell in the Voronoi tessellation is a convex polytope, so the mapping can be seen as assigning points in \( \mathbb{R}^D \) to their corresponding Voronoi cell.  


### 2. Adding Discrete Structure to Normalizing Flows  
Normalizing flows typically use continuous, differentiable mappings (e.g., neural networks) to transform simple distributions (like Gaussian) into complex ones. By incorporating a **discrete structure** via the Voronoi tessellation, the proposal aims to:  
- Model distributions with complex geometry (e.g., multi-modal or with sharp boundaries) more effectively, as the Voronoi cells naturally capture local structure.  
- Introduce a form of “discrete partitioning” into the flow, which can regularize the model or improve expressiveness.  


### 3. Efficient Computation of Inverse and Log Determinant Jacobian  
For normalizing flows, two critical requirements are:  
- **Efficient inverse \( f^{-1} \)**: The inverse mapping must be computable quickly to sample from the transformed distribution.  
- **Efficient log determinant of the Jacobian \( \log | \det J_f | \)**: This is needed for probability density estimation (since the density transforms as \( p(x) = p(f(x)) | \det J_f | \)).  

Using a differentiable Voronoi tessellation, both can be computed efficiently:  
- The inverse \( f^{-1} \) maps a point in the polytope back to its corresponding generator (or representative point in \( \mathbb{R}^D \)), which is a straightforward lookup in the tessellation.  
- The log determinant Jacobian leverages the properties of the Voronoi cells (e.g., linear boundaries, which simplify Jacobian computation) and the differentiable structure of the tessellation, allowing efficient calculation.  


### Summary  
The proposal introduces an invertible mapping \( f \) parameterized by a differentiable Voronoi tessellation, mapping \( \mathbb{R}^D \) to a convex polytope. This adds discrete structure to normalizing flows, enabling efficient computation of the inverse mapping and log determinant Jacobian, while leveraging the geometric properties of Voronoi tessellations for improved distribution modeling.  

The answer is \(<|begin_of_box|>f<|end_of_box|>\).</answer>