\documentclass[10pt]{article}
\usepackage{tikz}
\usetikzlibrary{decorations.markings}

\begin{document}

\begin{tikzpicture}[scale=1.5]
    % Draw the axes
    \draw[->] (0,0) -- (3,0) node[right] {$Z_{U(1)}$};
    \draw[->] (0,0) -- (0,3) node[above left] {$Z_{KK}$};

    % Draw the quarter-circle
    \draw[fill=gray!70] (0,0) -- (90:2cm) arc (90:0:2cm) -- cycle;

    % Draw the vertical line
    \draw (0,0) -- (0,2.5);

    % Draw the dots along the vertical line
    \foreach \y in {1,...,8}
    {
        \pgfmathsetmacro{\angle}{180*\y/8}
        \draw[fill=black!90] ({atan(\angle)/180*pi},{1/\y}) circle (0.05);
    }

    % Label the end of the vertical line
    \draw[fill=white] (0,2.5) circle (0.05);
    \node at (0,2.6) {$e^{-\phi_4}$};

    % Mark the points on the quarter-circle
    \foreach \r in {0.5, 0.75, 1, 1.25, 1.5}
    {
        \pgfmathsetmacro{\theta}{acos(1-\r*\r)}
        \draw[fill=gray!50] ({\theta},{\r}) circle (0.05);
    }
    
    % Label the points on the quarter-circle
    \node at (1,0) [below] {$1$};
    \node at (0,1) [left] {$1$};
\end{tikzpicture}

\textbf{The convex hull (see \cite{Heidenreich:2015nta} for notation) of gonions and a KK tower in circle compactification, here just the vertical dot line, always contains the extremal circle, if we remain in a perturbative region $e^{-\phi_4}=g^{-1} \geq 1$.}

\end{document}