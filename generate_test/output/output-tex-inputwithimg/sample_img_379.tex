\documentclass{article}
\usepackage{amsmath}
\usepackage{amssymb}
\usepackage{tikz}
\usetikzlibrary{matrix}

\begin{document}

\begin{equation}
\text{The induced } P_3 \text{ in } \mathcal{C}_3(P_3) \text{ generated by } \left( \{v_1, v_3\}, C \right), \text{ for } C = \{c_1, c_2, c_3\} \text{ as in Eqs.~\eqref{eq:c_example}}
\end{equation}

\begin{tikzpicture}[baseline=(current bounding box.center)]
    \matrix (m) [matrix of math nodes, row sep=2em, column sep=2em] {
        \node[circle,draw,inner sep=2pt] (v1) {$v_1$}; & \node[circle,draw,inner sep=2pt] (v2) {$v_2$}; & \node[circle,draw,inner sep=2pt] (v3) {$v_3$}; \\
        \node[circle,draw,inner sep=2pt] (c1) {$c_1$}; & \node[circle,draw,inner sep=2pt] (c2) {$c_2$}; & \node[circle,draw,inner sep=2pt] (c3) {$c_3$}; \\
        \node[circle,draw,inner sep=2pt] (v4) {$v_4$}; & \node[circle,draw,inner sep=2pt] (v5) {$v_5$}; & \node[circle,draw,inner sep=2pt] (v6) {$v_6$}; \\
    };
    
    \draw[->] (m-1-1) -- (m-2-1);
    \draw[->] (m-1-2) -- (m-2-2);
    \draw[->] (m-1-3) -- (m-2-3);
    
    \draw[->,thick] (m-2-1) -- (m-1-2);
    \draw[->,thick] (m-2-2) -- (m-1-3);
    
    \draw[->,thick] (m-2-2) -- (m-1-2);
    \draw[->,thick] (m-2-3) -- (m-1-2);
\end{tikzpicture}

\end{document}