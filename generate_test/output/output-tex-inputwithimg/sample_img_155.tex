\documentclass{article}
\usepackage{amsmath}
\usepackage{graphicx}

\begin{document}

\textbf{Left Figure:} \small On the left, we depict the lower half of the Euclidean geometry combined with the subsequent Lorentzian evolution. The former part of the Euclidean geometry is used to generate the deformed thermofield initial state. The right figure illustrates the full Euclidean evolution which may be used to compute the normalization factor $Z$ or the thermal expectation value (vev) of operators with an appropriate insertion of operator $O(t)$.

\begin{figure}[h]
    \centering
    \includegraphics[width=0.45\textwidth]{euclidean_lorentzian.png}
    \includegraphics[width=0.45\textwidth]{full_euclidean_evolution.png}
    \caption{(Left) Lower half of the Euclidean geometry combined with the subsequent Lorentzian evolution, where the left part generates the deformed thermofield initial state. (Right) Full Euclidean evolution used for computing the normalization factor $Z$ or the thermal expectation value (vev) of operators with an appropriate insertion of operator $O(t)$.}
    \label{fig:euclidean_evolution}
\end{figure}

Here, we have two figures:
1. The left figure shows the lower half of the Euclidean geometry combined with the subsequent Lorentzian evolution. The left part of the Euclidean geometry is used to generate the deformed thermofield initial state.
2. The right figure illustrates the full Euclidean evolution, which can be used to compute the normalization factor $Z$ or the thermal expectation value (vev) of operators with an appropriate insertion of operator $O(t)$.

\end{document}