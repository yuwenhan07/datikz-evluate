Certainly! Below is the LaTeX code that incorporates the provided information and uses the `nicematrix` package for creating the tables:

```latex
\documentclass{article}
\usepackage[utf8]{inputenc}
\usepackage{nicematrix}
\usepackage{amsmath}

\begin{document}

\begin{table}[htbp]
    \caption{Generators of \( U(2N_F) \). On the left we have explicit, properly normalized generators for \( SU(4) \). To the right their general structure is given for arbitrary values of \( N_F \). The matrices \( H_{1,2} \) denote hermitian matrices. The matrix \( S = S_R + i S_I \) denotes a complex, traceful, symmetric matrix. The matrix \( A = A_R + i A_I \) denotes a complex, antisymmetric matrix. All matrices are defined with respect to the Nambu-Gorkov basis \eqref{eq:dark_strong_quark_lagrangian_nambu_gorkov}.}
    \centering
    \begin{NiceArray}{c|c}
        \hline
        & \( U(4) \) \\
        \hline
        & \\
        \hline
        \( T^T_0 \overset{T}{\to} \left( \begin{array}{cccc} 1 & 0 & 0 & 0 \\ 0 & 1 & 0 & 0 \\ 0 & 0 & 1 & 0 \\ 0 & 0 & 0 & 1 \end{array} \right) \) &
        \( \left( \begin{array}{cc} \nearrow & 0 \\ 0 & \nwarrow \end{array} \right) \) \\
        \hline
        \( T^T_1 \overset{T}{\to} \frac{1}{\sqrt{2}} \left( \begin{array}{cccc} 1 & 0 & 0 & 0 \\ 0 & -1 & 0 & 0 \\ 0 & 0 & 1 & 0 \\ 0 & 0 & 0 & -1 \end{array} \right) \) &
        \( \left( \begin{array}{ccc} 1 & 0 & 0 \\ 0 & \frac{1}{\sqrt{2}} & 0 \\ 0 & 0 & -\frac{1}{\sqrt{2}} \end{array} \right) \) \\
        \hline
        \( T^T_2 \overset{T}{\to} \frac{1}{\sqrt{2}} \left( \begin{array}{cccc} 1 & 0 & 0 & 0 \\ 0 & 1 & 0 & 0 \\ 0 & 0 & 0 & 1 \\ 0 & 0 & 1 & 0 \end{array} \right) \) &
        \( \left( \begin{array}{ccc} 0 & 1 & 0 \\ 1 & 0 & 0 \\ 0 & 0 & 0 \end{array} \right) \) \\
        \hline
        \( T^T_3 \overset{T}{\to} \frac{1}{2} \left( \begin{array}{cccc} 0 & 0 & 1 & 0 \\ 0 & 0 & 0 & 1 \\ 1 & 0 & 0 & 0 \\ 0 & 1 & 0 & 0 \end{array} \right) \) &
        \( \left( \begin{array}{ccc} 0 & 1 & 0 \\ 1 & 0 & 0 \\ 0 & 0 & 0 \end{array} \right) \) \\
        \hline
        \( T^T_4 \overset{T}{\to} \frac{1}{2} \left( \begin{array}{cccc} 0 & 0 & i & 0 \\ 0 & 0 & 0 & 1 \\ 0 & 0 & 0 & -i \\ -1 & 0 & 0 & 0 \end{array} \right) \) &
        \( \left( \begin{array}{ccc} 0 & 0 & 0 \\ 0 & 0 & 0 \\ 0 & 0 & 0 \end{array} \right) \) \\
        \hline
        \( T^T_5 \overset{T}{\to} \frac{1}{2} \left( \begin{array}{cccc} 0