\documentclass{article}
\usepackage{amsmath}
\usepackage{graphicx}

\begin{document}

\section*{Example for the Definition of the Potential Function}

In this example, we take \( k = 3 \) and \( t = 8 \). Let \( \mathcal{R}_t = \{R_1, R_2, R_3, R_4, R_5\} \) and the values of \( R_1, R_2, R_3, R_4, R_5 \) are shown in the figure below.

\begin{figure}[h]
    \centering
    \includegraphics[width=0.8\textwidth]{path_to_your_image.png}
    \caption{The example for the definition of the potential function.}
    \label{fig:potential_function_example}
\end{figure}

We calculate \( \phi_t(2,0) \) and \( \phi_t(2,3) \) as examples:
\begin{itemize}
    \item For \( \phi_t(2,0) \), the max operator in \Cref{eq.k-counter-phi} is taken over \( R \in \{R_1, R_3\} \), and the maximum is 4, which is achieved when \( R = R_3 \).
    \item For \( \phi_t(2,3) \), the max operator in \Cref{eq.k-counter-phi} is taken over \( R \in \{R_1, R_4\} \), and the maximum is 3, which is achieved when \( R = R_4 \).
\end{itemize}

\end{document}