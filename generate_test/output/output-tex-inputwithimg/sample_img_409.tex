\documentclass{standalone}
\usepackage{amsmath}
\usepackage{tikz}
\usetikzlibrary{arrows.meta, decorations.markings}

\begin{document}

% Define the main figure
\begin{figure}[h]
    \centering
    \begin{tikzpicture}[
        >=Latex,
        dot/.style = {circle, fill, inner sep=0.5mm},
        label/.style = {font=\small, text height=1.5ex, text depth=0.25ex},
        perturbation/.style={->, very thick, dashed, shorten <=2pt},
        label distance=3mm
    ]
        % Draw the data points
        \path 
            (-1,-1) node[dot, label=above left:$C$, color=red] {}
            (1,-1) node[dot, label=below right:$B$, color=red] {}
            (2,0)  node[dot, label=left:$H$, color=red] {}
            (4,1)  node[dot, label=right:$G$, color=red] {}
            (3,2)  node[dot, label=above left:$F$, color=red] {}
            (6,3)  node[dot, label=above right:$D$, color=blue] {}
            (8,2)  node[dot, label=above left:$A$, color=blue] {}
            (7,1)  node[dot, label=right:$E$, color=blue] {}
            (5,0)  node[dot, label=below left:$\theta_0$, color=red] {}
            (6,1)  node[dot, label=below right:$\hat{\theta}$] {};

        % Draw the perturbations
        \draw[perturbation] (5,0) -- (F);
        \draw[perturbation] (5,0) -- (B);
        \draw[perturbation] (5,0) -- (H);
        \draw[perturbation] (5,0) -- (G);

        % Draw the dashed lines and labels
        \draw[dashed] (-2,-1) -- (9,3);
        \draw[dashed] (-1,-1) -- (9,3);

        % Draw the arrows indicating the adversarial attack
        \draw[<->,thick] (-2,-1) circle (0.8) node[anchor=north] {\footnotesize Adversarial trade-off};
        \draw[<->,thick] (6,1) -- (5.5,2.5) node[above right] {\footnotesize Generalisation error};
        \draw[<->,thick] (6,1) -- (5.5,0.5) node[below right] {\footnotesize Boundary error};

        % Add the annotations for the points
        \node[label,below right=of F] {\footnotesize Points F and E are correctly classified after the perturbation.};
        \node[label,below right=of C] {\footnotesize The perturbations on B and A cancel each other on average.};
        \node[label,below right=of H] {\footnotesize Points H and G contribute to the boundary error.};
        \node[label,below right=of D] {\footnotesize Only points H and G contribute to the boundary error.};
    \end{tikzpicture}
    \caption{Adversarial trade-off in our model occurs due to improved robustness of a worse estimate $\hat{\boldsymbol{\theta}}$. The colour marks the true label of the data points. The arrows indicate the adversarial attack perturbation. There are four cases to be distinguished. Points F and E are correctly classified after the perturbation. The points C and D contribute to generalisation error. The perturbations on B and A cancel each other on average. Only points H and G contribute to the boundary error. A worse angle decreases the number of points contributing to boundary error at the cost of these points contributing to generalisation error.}
    \label{fig:adversarial_tradeoff}
\end{figure}

\end{document}