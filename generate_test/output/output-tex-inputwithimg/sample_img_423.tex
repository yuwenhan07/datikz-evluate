\documentclass[11pt]{article}
\usepackage{amsmath}
\usepackage{tikz}
\usetikzlibrary{positioning}

\begin{document}

\begin{figure}
    \centering
    \begin{tikzpicture}[
            scale=0.7,
            every node/.style={draw, circle, minimum size=1cm},
            level 1/.style={sibling distance=3.5em},
            level 2/.style={sibling distance=4em},
            level 3/.style={sibling distance=6em}
        ]
        
        %% Root Node
        \node {a,b,c,d,e}
        child {
            node {a,b,c,d}
            child {
                node {a,b}
                child {
                    node {a}
                }
                child {
                    node {b}
                }
            }
            child {
                node {c,d}
                child {
                    node {c}
                }
                child {
                    node {d}
                }
            }
        } 
        child {
            node {e}
        };
        
        %% Pink arrows
        \draw[->, ultra thick, pink] (4,0) to [out=90,in=-90] (5,1);
        \draw[->, ultra thick, pink] (9.5,0.8) to [out=-20,in=-160] (12,-1);
        
        %% Additional nodes for explanation
        \node[pink, draw=red, rounded corners=10pt, fit=(4,0) (5,1)] {};
        \node[pink, draw=red, rounded corners=10pt, fit=(9.5,0.8) (12,-1)] {};
        
        %% Labels for the arrows
        \node[pink, below right=of 4, inner sep=0pt, font=\tiny] {2};
        \node[pink, below right=of 9.5, inner sep=0pt, font=\tiny] {1};
        \node[pink, below left=of 4, inner sep=0pt, font=\tiny] {1};
        
        %% Text annotations
        \node at (10,-3) {$\{a,b,d,e\}$};
        \node at (14.8,-2.5) {$\{a,b,c,d,e\}$};
        
        %% Circle around the entire structure
        \draw[blue, very thick, rounded corners=10pt] (-5,-5) rectangle (19.5,5);
    \end{tikzpicture}
    \caption{Visualization of the $\sqrt{n}$-decomposition of the blue group from Figure~\ref{fig:group-partition}. The processes $a,b,c,d,e$ in the group are logically decomposed into a binary tree. The pink arrows visualize the three-round process of relaying operative counts of the two children of the root to the root itself. First, the counts are relayed to all processes in the group (arrow \#1), then the processes send a confirmation if they received the counts (arrow \#2), finally, all in the group transmit the received counts to the higher layer -- the root in this case (arrow \#3). Some processes can be faulty (process $c$ does not communicate, only $\{a,b,d,e\}$) and their values are not guaranteed to be accumulated accurately.}
    \label{fig:binary_tree_decomposition}
\end{figure}

\end{document}