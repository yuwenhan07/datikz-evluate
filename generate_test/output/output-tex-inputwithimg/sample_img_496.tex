\documentclass{standalone}
\usepackage{tikz}
\usetikzlibrary{matrix}

\begin{document}

\begin{tikzpicture}[node distance = 1cm]
    \matrix (grid) [matrix of nodes,
                    row sep=-\pgflinewidth,
                    column sep=-\pgflinewidth,
                    nodes={circle,draw,fill=gray!80, minimum size=5mm}]
    {
        & & & & & & & & & \\
        & & & & & & & & & \\
        & & & & & & & & & \\
        & & & & & & & & & \\
        & & & & & & & & & \\
        & & & & & & & & & \\
        & & & & & & & & & \\
        & & & & & & & & & \\
        & & & & & & & & & \\
        & & & & & & & & & \\
    };
    \foreach \x in {1,...,10} {
        \draw (grid-\x-2.north west) -| (grid-\x-9.south east);
    }
    \foreach \y in {1,...,10} {
        \draw (grid-1-\y.west) -- (grid-10-\y.east);
    }

    % Drawing the blue diamond shape and labels
    \draw[dashed,blue] (grid-6-5.north west) -- (grid-7-6.north east) -- (grid-8-5.south east) -- (grid-7-4.south west) -- cycle;
    \node at (grid-6-5) {$w_1$};
    \node at (grid-7-5) {$w_2$};
    \node at (grid-7-6) {$w_3$};
    \node at (grid-6-6) {$w_4$};
    \node[circle,inner sep=1pt,fill=blue] at (grid-6-5) {};
    \node[circle,inner sep=1pt,fill=blue] at (grid-7-6) {};
    \node[circle,inner sep=1pt,fill=blue] at (grid-6-6) {};
    \node at (grid-7-5) {$b$};

    % Highlighting the qubits
    \node[draw,circle,fill=gray!80] at (grid-5-4) {};
    \node[draw,circle,fill=gray!80] at (grid-5-6) {};
    \node[draw,circle,fill=gray!80] at (grid-6-4) {};
    \node[draw,circle,fill=gray!80] at (grid-6-6) {};
    
    % Description annotations
    \node at (grid-6-3) {\small This lattice diagram represents a translation-invariant structure for a vertex-type hidden neuron, using simplified notation without the subscript $v$ and $(b_v,w_{v,j})=i*(b,w_j)$.};
    \node at (grid-6-3) {\small If we flip the four qubits surrounding the central vertex, qubits contributing to the phase difference are circled for clarity.};
\end{tikzpicture}

\end{document}