\documentclass[12pt]{article}
\usepackage{tikz}
\usetikzlibrary{intersections}

\begin{document}

\begin{tikzpicture}[scale=2]
    % Define coordinates
    \coordinate (A) at (0,0);
    \coordinate (B) at (1,1);
    \coordinate (C) at (2,0);
    \coordinate (D) at (1,-1);
    
    % Draw the square
    \draw (A) -- (B) -- (C) -- (D) -- cycle;
    
    % Define the Cauchy slice (black curve)
    \path[name path=curve] plot [smooth,tension=1] coordinates {(0.5,-0.5) (0.7,0.3) (1.3,0.8) (1.9,0.6) (2.1,0.4)};
    
    % Draw the region spanned by equivalent slices
    \fill[blue!20,name path global=region1] plot [smooth,tension=1] coordinates {(0.5,-0.5) (0.7,0.3) (1.3,0.8)} -- (1.9,0.6) -- (1.9,0.4) -- (2.1,0.4) -- (2,0) -- (0.9,-0.9) -- cycle;
    \fill[red!20, name path global=region2] plot [smooth,tension=1] coordinates {(1.9,0.6) (1.9,0.4) (2,0) (2.1,0.4)} -- (2.3,0.2) -- (2.3,0.3) -- (2.5,0.3) -- (2.5,0.1) -- (2.7,0.1) -- (2.7,0.2) -- cycle;
    
    % Intersect with the diagonals
    \path [name intersections={of=region1 and {A--C},by={I1}}];
    \path [name intersections={of=region1 and {B--D},by={I2}}];
    \path [name intersections={of=region2 and {A--C},by={I3}}];
    \path [name intersections={of=region2 and {B--D},by={I4}}];
    
    % Draw the points where the black curve intersects the diagonals
    \fill (I1) circle (1pt) node[above right] {$\chi_{\rm L}$};
    \fill (I2) circle (1pt) node[above left] {$\chi_{\rm R}$};
    \fill (I3) circle (1pt) node[above right] {$\chi_{\rm E}$};
    \fill (I4) circle (1pt);
    
    % Draw the red dot as a subregion
    \fill[red] (1.5,0) circle (1pt) node[below] {$A$};
    
    % Draw the dashed line for the Cauchy slice
    \draw[dashed] (1,-1) -- (1,1);
    
    % Draw the red segments representing quantum extremal surfaces
    \draw[red] (1.5,0) -- (2.1,0.2);
    \draw[red] (1.5,0) -- (2.1,0.3);
    \draw[red] (1.5,0) -- (2.3,0.1);
    
    % Draw the entanglement wedge
    \fill[cyan!20, name path global=entanglement_wedge] plot [smooth,tension=1] coordinates {(0.5,-0.5) (0.7,0.3) (1.3,0.8)} -- (1.9,0.6) -- (2.3,0.2) -- (2.3,0.3) -- (2.5,0.3) -- (2.5,0.1) -- (2.7,0.1) -- (2.7,0.2) -- cycle;
    
    % Draw the black dots for quantum extremal surfaces
    \fill (1.5,0) circle (1pt);
    \fill (2,0) circle (1pt);
    \fill (1.9,0.4) circle (1pt);
\end{tikzpicture}

\end{document}