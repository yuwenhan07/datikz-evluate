\documentclass{article}
\usepackage{amsmath}
\usepackage{tikz}

\begin{document}

\begin{figure}[h]
    \centering
    \begin{tikzpicture}[level distance=1.5cm,
        level 1/.style={sibling distance=3.5cm},
        level 2/.style={sibling distance=2cm},
        level 3/.style={sibling distance=1.5cm},
        level 4/.style={sibling distance=1cm}]
        \node {$q_0$}
            child { node {$c$} 
                child { node {$c$} 
                    child { node {$d$} 
                        child { node {$c/d$}
                            child { node {$c$} }
                            child [missing] {} }
                        child { node {$c$} } }
                    child [missing] {}
                    child { node {$c$}} }
            }
            child { node {$q_1$} 
                child { node {$c$} }
                child [missing] {}
                child { node {$a$} 
                    child { node {$q_k = q$} }
                    child [missing] {}
                    child { node {$a$} 
                        child { node {$q_{k+p} = q$} }
                        child [missing] {}
                        child { node {$a$} 
                            child { node {$q_{k+2p} = q$} }
                            child [missing] {}
                            child { node {$a$} 
                                child { node {$q_{k+3p}$} }
                                child [missing] {}
                                child { node {$c$} } }
                        } }
                } };
    \end{tikzpicture}
    \caption{Illustration for the proof of Proposition~\ref{twoexamples}. All inner nodes of the tree are labeled with $a$. All leaves not shown in the illustration are labeled with $c$.}
    \label{fig:tree-diagram}
\end{figure}

\end{document}