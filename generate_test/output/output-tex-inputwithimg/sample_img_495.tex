\documentclass[12pt]{article}
\usepackage{amsmath, amssymb}
\usepackage{tikz}
\usetikzlibrary{matrix}

\begin{document}

\begin{center}
\begin{tikzpicture}[node distance=1.5cm]
\node[state] (1) {$1$};
\node[state] (lambda-2) [right of=1] {$\lambda^{-2}$};
\node[state] (lambda-3) [right of=lambda-2] {$\lambda^{-3}$};
\node[state] (lambda-4) [right of=lambda-3] {$\lambda^{-4}$};
\node[state] (lambda-5) [below right of=lambda-4] {$\lambda^{-5}$};
\node[state] (lambda-6) [below of=lambda-5] {$\lambda^{-6}$};
\node[state] (lambda-1) [above left of=lambda-2] {$\lambda^{-1}$};
\node[state] (lambda-2-2) [above of=lambda-4] {$\lambda^{-2}$};
\node[state] (lambda-3-3) [above of=lambda-5] {$\lambda^{-3}$};
\node[state] (lambda-4-4) [below right of=lambda-5] {$\lambda^{-4}$};
\node[state] (lambda-5-5) [below right of=lambda-3-3] {$\lambda^{-5}$};

\path[->] (1) edge node [left] {$b$} (lambda-2)
          (1) edge node [above] {$a$} (lambda-1)
          (lambda-2) edge node [left] {$a$} (lambda-3)
          (lambda-2) edge node [below] {$b$} (lambda-1)
          (lambda-3) edge node [above] {$a$} (lambda-4)
          (lambda-3) edge node [below] {$b$} (lambda-2)
          (lambda-4) edge node [above] {$a$} (lambda-5)
          (lambda-4) edge node [below] {$b$} (lambda-3)
          (lambda-5) edge node [above] {$a$} (lambda-6)
          (lambda-5) edge node [below] {$b$} (lambda-4)
          (lambda-6) edge node [above] {$a$} (lambda-5-5)
          (lambda-6) edge node [below] {$b$} (lambda-5);
\end{tikzpicture}
\end{center}

The invariant probability measure on the Fibonacci shift ($\lambda = \text{the golden ratio}$). Circled nodes represent elements from the language $L = \{w \in \{a, b\}^* \mid |w|_a \equiv 0 \mod 2\}$.
\end{document}