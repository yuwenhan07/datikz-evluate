\documentclass{article}
\usepackage{amsmath}
\usepackage{graphicx}
\usepackage{tikz}

\begin{document}

\begin{figure}[h]
    \centering
    \includegraphics[width=0.8\textwidth]{path/to/your/image.png}
    \caption{Decomposition $\boldsymbol{J} = \boldsymbol{U}_J \boldsymbol{\Sigma}_J \boldsymbol{V}_J^\top$ in the case where $r_J = 2$. Each of the joint components has an $n$-dimensional score vector (column of $\boldsymbol{U}_J$) and a $p$-dimensional loading vector (row of $\boldsymbol{V}_J^\top$) associated with it; in this example, $n = 3$ and $p = 4$. Each subward has an $r_J$-dimensional score vector (row of $\boldsymbol{U}_J$) associated with it, and each feature has a $r_J$-dimensional loading vector (column of $\boldsymbol{V}_J^\top$). Given that the singular values in $\boldsymbol{\Sigma}_J$ are distinct and ordered from largest to smallest, the decomposition is identifiable up to multiplication of components by -1: we can multiply any column of $\boldsymbol{U}_J$ and the corresponding column of $\boldsymbol{V}_J$ (row of $\boldsymbol{V}_J^\top$) by -1 without changing the value of $\boldsymbol{U}_J \boldsymbol{\Sigma}_J \boldsymbol{V}_J^\top$.}
    \label{fig:decomposition_example}
\end{figure}

\end{document}