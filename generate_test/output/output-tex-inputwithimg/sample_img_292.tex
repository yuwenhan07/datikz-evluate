\documentclass[10pt, conference]{IEEEtran}
\usepackage{amsmath}
\usepackage{bm}
\usepackage{tikz}
\usetikzlibrary{positioning, arrows.meta}

\begin{document}

\begin{figure}[h]
    \centering
    \begin{tikzpicture}[
        node distance = 20mm,
        >=Stealth',
        every pin edge/.style = {->},
        block/.style={rectangle, draw=black, fill=orange!60, text width=3cm, align=center, minimum height=1cm, font=\small},
        circ/.style={circle, draw=black, fill=white, text width=8mm, align=center, inner sep=0pt, minimum height=8mm},
        dot/.style={circle, fill=black, inner sep=0pt, minimum size=2mm}
    ]

        \node[circ] (x) {$\mathbf{x}$};
        \foreach \i/\j in {1/-0.75,2/-1.5,...,9/-7.5}{
            \node[dot,pin={[pin distance=6mm]\j:$\mathbf{x}_i^b$}] at ($(x)+(-1*\i*1.5,-1*\j)$) {};
        }
        \draw[->] (x) |- ++(1,0.5) node[pos=.35,below right,scale=0.8]{$-\mathbf{.}$};
        \draw[->] (x) |- ++(1,-0.5) node[pos=.35,below left,scale=0.8]{$-\mathbf{.}$};

        \foreach \i [count=\ni from 1] in {1,...,9}{
            \draw[->] ($ (x)!0.7!(\ni) $) -- ++(1,0) node[midway,above]{$r_{\ni}$};
        }

        \node[block, right=of x] (KNNG_1) {$-b_1\text{KNN}_G(r_1;\Theta)$};
        \node[circ, below=of KNNG_1] (prod1) {$\times$};
        \node[block, below=of prod1] (mul1) {$-b_n\text{KNN}_G(r_n;\Theta)$};
        \node[circ, below=of mul1] (prod2) {$\times$};
        \node[circ, right=of prod2] (sum) {$\sum$};
        \node[right=of sum] (output) {$u(\mathbf{x})$};

        \node[block, right=of x |- KNNG_1] (MLP_1) {$\text{MLP}_{g}(\mathbf{x}_{1}^{b};\theta_{g})$};
        \node[block, below=of MLP_1] (MLP_2) {$\text{MLP}_{g}(\mathbf{x}_{n}^{b};\theta_{g})$};

        \foreach \i in {1,...,9}{
            \draw[->] (x) -- (KNNG_1);
            \draw[->] (x |- KNNG_1) -- (MLP_1);
            \draw[->] (x |- MLP_1) -- (MLP_2);
            \draw[->] (KNNG_1 |- \i) -- (prod1);
            \draw[->] (mul1 |- \i) -- (prod2);
            \draw[->] (MLP_1 |- \i) -- (prod2);
            \draw[->] (MLP_2 |- \i) -- (prod2);
        }

        \draw[->] (prod1) -- (prod2);
        \draw[->] (prod2) -- (sum);
        \draw[->] (sum) -- (output);

    \end{tikzpicture}
    \caption{Network architecture of the BIN-G. Using the distance $r_i$ between $\mathbf{x}$ and $\mathbf{x}_i^b$ enforcing the symmetry of the Green's function. The block highlighted in orange evaluates a domain-independent Green's function learned using a KNN.}
    \label{fig:BING}
\end{figure}

\end{document}