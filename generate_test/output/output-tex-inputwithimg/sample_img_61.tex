\documentclass[border=10pt]{standalone}
\usepackage{tikz}
\usetikzlibrary{shapes.geometric}

\tikzset{
    value/.style={circle, fill=green!40},
    variable/.style={circle, fill=red!60},
    constraint/.style={circle, fill=blue!40},
    operation/.style={circle, fill=orange!60},
    model/.style={circle, draw=black, fill=yellow!50},
    lhs/.style={rectangle, draw=gray!80, rounded corners=5pt, fill=gray!20, anchor=south},
    rhs/.style={rectangle, draw=gray!80, rounded corners=5pt, fill=gray!20, anchor=west},
    table/.style={rectangle, draw=gray!80, rounded corners=5pt, fill=gray!20, anchor=west},
    arrow/.style={->, shorten <=2pt, shorten >=2pt},
    label/.style={align=center, text width=1cm, inner sep=2pt, font=\small}
}

\begin{document}

\begin{tikzpicture}[node distance=1cm, auto]

% Value-vertices
\node[value] (v1) at (0, 0) {$1$};
\node[value] (v2) at (4, 0) {$2$};
\node[value] (v3) at (8, 0) {$3$};

% Variable-vertices
\node[variable, label] (x1) at (2, -3) {$x_1$};
\node[variable, label] (x2) at (6, -3) {$x_2$};

% Operation-vertices
\node[operation, label] (op1) at (3, -3) {$\times 3$};
\node[operation, label] (op2) at (7, -3) {$\times 4$};

% Model-vertex
\node[model, label] (m1) at (4, -4) {$M$};

% Constraint-vertices
\node[constraint, label] (c1) at (5, -5) {$\leq$};
\node[constraint, label] (c2) at (5, -6) {$\textsf{ext}$};

% lhs and rhs rectangles
\node[lhs, label] (lhsh) at (1.5, -2) {\textit{lhs}};
\node[rhs, label] (rshh) at (6.5, -2) {\textit{rhs}};

% Table vertices
\node[table, label] (tab1) at (2.5, -6) {$(1,2)$};
\node[table, label] (tab2) at (5.5, -6) {$(2,3)$};

% Connections
\draw[arrow] (v1) -- (op1);
\draw[arrow] (v2) -- (op1);
\draw[arrow] (op1) -- (x1);
\draw[arrow] (op2) -- (x2);

\draw[arrow] (v1) -- (op2);
\draw[arrow] (v3) -- (op2);
\draw[arrow] (op2) -- (x2);

\draw[arrow] (x1) -- (op1);
\draw[arrow] (x2) -- (op2);

\draw[arrow] (op1) -- (m1);
\draw[arrow] (op2) -- (m1);

\draw[arrow] (m1) -- (c1);
\draw[arrow] (m1) -- (c2);

\draw[arrow] (c1) -- (op1);
\draw[arrow] (c1) -- (op2);

\draw[arrow] (c2) -- (tab1);
\draw[arrow] (c2) -- (tab2);

\end{tikzpicture}

\end{document}