\documentclass{article}
\usepackage{tikz}
\usetikzlibrary{decorations.pathreplacing}

\begin{document}

\begin{figure}[h]
    \centering
    \begin{tikzpicture}[scale=1.5]

        % Draw the O5^+ and O5^- branes
        \draw[thick] (-2,0) -- (-1,0);
        \draw[thick] (1,0) -- (2,0);
        \draw[thick] (-1,0) -- (-1,1);
        \draw[thick] (-1,1) -- (1,1);
        \draw[thick] (1,1) -- (1,0);
        \draw[dashed, thick] (-2,-1) -- (-1,-1);
        \draw[dashed, thick] (1,-1) -- (2,-1);
        \draw[dashed, thick] (-1,-1) -- (-1,0);
        \draw[dashed, thick] (-1,0) -- (-1,1);
        \draw[dashed, thick] (-1,1) -- (1,1);
        \draw[dashed, thick] (1,1) -- (1,0);
        \draw[dashed, thick] (1,0) -- (2,0);

        % Label the branes
        \node at (-2.5,0) {$\mathrm{O5}^{+}$};
        \node at (2.5,0) {$\mathrm{O5}^{+}$};
        \node at (0,0) {$\mathrm{O5}^{-}$};

        % Draw the toric nodes
        \foreach \i in {-1,0,...,4} {
            \filldraw[black] (90-72*\i+180:1) circle (2pt);
            \filldraw[white] (90-72*\i+180:1) circle (2.5pt);
        }

        % Connect the nodes
        \draw (90:1) -- ++(-60:0.5) -- ++(-120:0.5) -- ++(-60:0.5) -- cycle;
        \draw (90:1) -- ++(60:0.5) -- ++(120:0.5) -- ++(60:0.5) -- cycle;

    \end{tikzpicture}
    \caption{The brane construction of $SO(8)$ gauge group with toric nodes indicated by the brane construction. Notice that O$5^+$ branes are replaced by 3 nodes instead of 1. See \cite{Hayashi:2023boy} for details.}
    \label{fig:so8_brane}
\end{figure}

\end{document}