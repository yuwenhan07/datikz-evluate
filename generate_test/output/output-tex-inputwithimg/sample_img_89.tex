\documentclass[11pt]{article}
\usepackage{amsmath, amssymb, amsthm, graphicx, enumitem}

\newcommand{\R}{\mathbb{R}}

\begin{document}

\begin{figure}[h]
    \centering
    \includegraphics[width=\textwidth]{image.png} % Replace 'image.png' with your actual image file name
    \caption{Saddle moves can be used to turn connect summands that are 2-component links into knots, as in the Link Lemma \ref{lem:link}. Twist moves are indicated with a purple segment and crossing changes by highlighting the crossing.}
    \label{fig:saddle_moves}
\end{figure}

\begin{proof}[Proof of Theorem~\ref{thm:main}]
Here we detail the proof of Theorem~\ref{thm:main}. We consider the saddle move and its corresponding effect on the link diagram shown in Figure~\ref{fig:saddle_moves}.
\end{proof}

\begin{lemma}[Link Lemma] \label{lem:link}
The Link Lemma states that the two diagrams in Figure~\ref{fig:saddle_moves} are equivalent under a series of Reidemeister moves. Specifically, the diagram on the left can be transformed into the diagram on the right using only R I, R II, and R III Reidemeister moves.
\end{lemma}

\end{document}