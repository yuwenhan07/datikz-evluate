\documentclass[12pt]{article}
\usepackage{graphicx}
\usepackage{amsmath}
\usepackage{amsfonts}
\usepackage{tikz}

\begin{document}

A visualization of trajectories of the RDS from Example~\ref{ex:forward_not_pullback}. The $x$-axis denotes time; the $y$-axis is the state space $\mathds{X}=\mathds{N}_0$. An arrow from $(\theta_{-n} \omega, x)$ to $(\theta_{-n+1} \omega, y)$ indicates that $\varphi_{\theta_{-n}\omega}^1 x = y$. The thick line shows the value of $m_n(\omega)$. The depicted graph on the set of nodes $\mathds{Z} \times \mathds{X}$ has been called \emph{Doeblin graph} in the literature \cite{Baccellietal2019}.

\begin{figure}[h]
    \centering
    \includegraphics[width=0.8\textwidth]{example-image.png} % Replace with your actual image file name
    \caption{Visualization of trajectories of the RDS from Example~\ref{ex:forward_not_pullback}. The $x$-axis denotes time; the $y$-axis is the state space $\mathds{X}=\mathds{N}_0$. An arrow from $(\theta_{-n} \omega, x)$ to $(\theta_{-n+1} \omega, y)$ indicates that $\varphi_{\theta_{-n}\omega}^1 x = y$. The thick line shows the value of $m_n(\omega)$. The depicted graph on the set of nodes $\mathds{Z} \times \mathds{X}$ has been called \emph{Doeblin graph} in the literature \cite{Baccellietal2019}.}
    \label{fig:example_image}
\end{figure}

\end{document}