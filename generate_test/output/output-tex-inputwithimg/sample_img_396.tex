\documentclass[12pt]{article}
\usepackage{amsmath,empheq}
\usepackage{pgf,tikz}
\usetikzlibrary{arrows}
\begin{document}

\begin{flushright}
\begin{tikzpicture}[line cap=round,line join=round,>=triangle 45,x=0.7cm,y=0.7cm]
\clip(-0.5,-0.5) rectangle (18.6785534971334,.7785534971334);
\draw [->] (9.,2.)-- (10.5,2.);
\draw [->] (3.5,3)-- (6.,3.);
\draw [->] (13.5,4)-- (15.,4.);
\draw (7.4,3.8) node[anchor=north west] {$x_{2}=0$};
\draw (10.1,3.8) node[anchor=north west] {$x_{3}=2$};
\draw (13.4,3.8) node[anchor=north west] {$x_{4}=0$};
\draw (17.4,4.1) node[anchor=north west] {$x_{5}=3$};
% Knoten der ersten Liste
\draw[fill=green] (1.0,2.) circle (0.1);
\draw[fill=red]   (2.0,3.) circle (0.1);
% Knoten der zweiten Liste
\draw[fill=green] (6.0,2.) circle (0.1);
\draw[fill=red]   (4.0,3.) circle (0.1);
% Knoten der dritten Liste
\draw[fill=green] (12.0,2.) circle (0.1);
\draw[fill=red]   (10.0,3.) circle (0.1);
% Knoten der vierten Liste
\draw[fill=green] (18.0,2.) circle (0.1);
\draw[fill=red]   (16.0,3.) circle (0.1);
% Knoten der zehnten Liste
\draw[fill=green] (2.0,1.) circle (0.1);
\draw[fill=red]   (1.0,2.) circle (0.1);
% Knoten der elften Liste
\draw[fill=green] (4.0,1.) circle (0.1);
\draw[fill=red]   (3.0,2.) circle (0.1);
% Knoten der zwölften Liste
\draw[fill=green] (10.0,1.) circle (0.1);
\draw[fill=red]   (9.0,2.) circle (0.1);
% Knoten der dreizehnten Liste
\draw[fill=green] (16.0,1.) circle (0.1);
\draw[fill=red]   (15.0,2.) circle (0.1);
% Knoten der vierzehnten Liste
\draw[fill=green] (1.0,0.) circle (0.1);
\draw[fill=red]   (0.0,1.) circle (0.1);
% Knoten der fünfzehnten Liste
\draw[fill=green] (3.0,0.) circle (0.1);
\draw[fill=red]   (2.0,1.) circle (0.1);
% Knoten der sechzehnten Liste
\draw[fill=green] (9.0,0.) circle (0.1);
\draw[fill=red]   (8.0,1.) circle (0.1);
% Knoten der siebzehnten Liste
\draw[fill=green] (15.0,0.) circle (0.1);
\draw[fill=red]   (14.0,1.) circle (0.1);
\draw[fill=green] (18.0,1.) circle (0.1);
\draw[fill=red]   (17.0,2.) circle (0.1);
\end{tikzpicture}
\end{flushright}

An example of the bijection $\psi$ between $\mathcal{P}^d_n$ and $\mathcal{A}^d_n$.
\end{document}