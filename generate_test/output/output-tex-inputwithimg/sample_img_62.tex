\documentclass{article}
\usepackage[utf8]{inputenc}
\usepackage{amsmath}
\usepackage{amsfonts}
\usepackage{amssymb}
\usepackage{graphicx}
\usepackage{pgfplots}
\usepackage{subcaption}

\begin{document}

\begin{figure}[h]
    \centering
    \begin{subfigure}{0.45\textwidth}
        \centering
        \begin{tikzpicture}
            \begin{loglogaxis}[
                xlabel=Time (t),
                ylabel=Changes,
                xmin=0, xmax=1,
                ymin=1e-8, ymax=1e-7,
                xtick={0,0.2,0.4,0.6,0.8,1},
                ytick={1e-8,1e-7},
                legend pos=south west,
                grid=major,
                log basis x=10,
                log basis y=10,
                ]
                \addplot[color=red, thick] table [x index=0, y index=1] {data_example1.txt};
                \addplot[color=blue, thick] table [x index=0, y index=1] {data_example2.txt};
                \legend{L2 energy (N=100), L2 energy (N=1000)}
            \end{loglogaxis}
        \end{tikzpicture}
    \end{subfigure}
    \hfill
    \begin{subfigure}{0.45\textwidth}
        \centering
        \begin{tikzpicture}
            \begin{loglogaxis}[
                xlabel=Time (t),
                ylabel=$\ell_2$ error,
                xmin=0.001, xmax=10,
                ymin=1e-10, ymax=1e0,
                xtick={0.001,0.01,0.1,1,10},
                ytick={1e-10,1e-8,1e-6,1e-4,1e-2,1e0},
                legend pos=south west,
                grid=major,
                log basis x=10,
                log basis y=10,
                ]
                \addplot[color=black, dashed, thick] table [x index=0, y index=1] {data_example3.txt};
                \addplot[color=green, thick] table [x index=0, y index=1] {data_example4.txt};
                \addplot[color=cyan, thick] table [x index=0, y index=1] {data_example5.txt};
                \legend{Hamiltonian energy (N=1000), MHK Example ??, MHK Example ?}
            \end{loglogaxis}
        \end{tikzpicture}
    \end{subfigure}
    \caption{
        Left: change of observables over time for Example \ref{ex: matrix, nonlinear, time-independent}.
        Right: convergence of the $\ell_2$ error of $u$ at $t = 1$ computed by different methods for Examples \ref{ex: matrix, nonlinear, time-dependent}, \ref{ex: matrix, nonlinear, time-independent}.
    }
    \label{fig:example_convergence}
\end{figure}

\end{document}