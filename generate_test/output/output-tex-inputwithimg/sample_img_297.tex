\documentclass{article}
\usepackage{amsmath}
\usepackage{tikz}
\usetikzlibrary{matrix}

\begin{document}

\[
M \left\{
\begin{array}{c}
z \rightarrow \\
z \rightarrow \\
z \rightarrow \\
z \rightarrow \\
\multicolumn{1}{c}{{{\makebox[0pt]{\hskip5pt$\vdots$\hskip5pt}}}\cdots }\\
\sigma(3) \\ 
\end{array}
\right\}
N
\left\{
\begin{array}{c}
1 \rightarrow \\
1 \rightarrow \\
1 \rightarrow \\
1 \rightarrow \\
\end{array}
\right\}
\]

\begin{tikzpicture}[scale=0.7]
    % Draw the grid lines
    \draw[help lines] (0,0) grid (8,5);
    
    % Draw vertical dashed line
    \draw[thick,dashed] (6,0) -- (6,5);
    
    % Draw arrows
    \draw[-stealth,red,thick] (6,4) -- (7,4);
    \draw[-stealth,orange,thick] (6,3) -- (7,3);
    \draw[-stealth,blue,thick] (6,2) -- (7,2);
    
    % Draw horizontal lines for N
    \draw[thick,orange] (0,3) -- (6,3);
    \draw[thick,red] (0,4) -- (6,4);
    \draw[thick,blue] (0,2) -- (6,2);
    
    % Label points
    \node at (6,-0.5) {$(1,N+2)$};
    \node at (-0.5,4) {$z\rightarrow$};
    \node at (-0.5,3) {$z\rightarrow$};
    \node at (-0.5,2) {$z\rightarrow$};
    \node at (-0.5,1) {$z\rightarrow$};
    \node at (-0.5,0) {$z\rightarrow$};
    \node at (-0.5,5) {$M$};
    \node at (-0.5,6) {$\left\{$};
    \node at (-0.5,7) {};
    \node at (-0.5,8) {{\makebox[0pt]{\hskip5pt$\vdots$\hskip5pt}\cdots}};
    \node at (-0.5,9) {$\sigma(3)$};
    \node at (-0.5,10) {$\sigma(2)$};
    \node at (-0.5,11) {$\sigma(1)$};
    \node at (-0.5,12) {$\left\{$};
    \node at (-0.5,13) {$N$};
    \node at (-0.5,14) {$1\rightarrow$};
    \node at (-0.5,15) {$1\rightarrow$};
    \node at (-0.5,16) {$1\rightarrow$};
    \node at (-0.5,17) {$1\rightarrow$};
\end{tikzpicture}

\captionof{figure}{The colored $q$-Boson model. The $z\to$ and $1\to$ at the left indicate the spectral parameters of the $\bigL$ weights associated to vertices in the corresponding rows, the expressions above the entry point of the arrows indicate the corresponding arrow's color, and the $\bm\cdots$ indicate that there are infinitely many columns, though, by definition of the boundary conditions, the arrows travel purely horizontally except until finitely many columns from the right.}
\end{document}