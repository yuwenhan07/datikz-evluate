\documentclass{article}
\usepackage{tikz}
\usetikzlibrary{graphs,graphdrawing}
\usegdlibrary{lcf}
\begin{document}

\begin{figure}[h]
    \centering
    \begin{tikzpicture}[every node/.style={circle, draw=black!50, fill=blue!20, minimum size=1cm}]
        \node (a) at (0, 2) {$a$};
        \node (v1) at (-1, 2) {$v_1$};
        \node (v2) at (+1, 2) {$v_2$};
        
        \node (b) at (0, 0) {$b$};
        \node (d1) at (-1.5, -1) {$d_1$};
        \node (d2) at (-0.5, -1) {$d_2$};
        \node (d3) at (-1.5, -2) {$d_3$};
        \node (c1) at (1.5, -1) {$c_1$};
        \node (c2) at (0.5, -2) {$c_2$};

        \draw[thick] (a) -- (v1);
        \draw[thick] (a) -- (v2);

        \draw[thick] (d1) -- (b);
        \draw[thick] (d2) -- (b);
        \draw[thick] (d3) -- (b);
        \draw[thick] (b) -- (c1);
        \draw[thick] (b) -- (c2);

        % Highlight the edges that form the graph G^{(2)}_f
        \draw[ultra thick, pink] (a) -- (v1);
        \draw[ultra thick, pink] (a) -- (v2);
        \draw[ultra thick, pink] (d1) -- (b);
        \draw[ultra thick, pink] (d2) -- (b);
        \draw[ultra thick, pink] (d3) -- (b);
        \draw[ultra thick, pink] (b) -- (c1);
        \draw[ultra thick, pink] (b) -- (c2);
    \end{tikzpicture}
    
    \caption{The neighborhood of \(a \times b\) in \(\mathcal{G}\). The graph \(G^{(2)}_f\) is represented by the pink lines.}
    \label{fig:neighborhood}
\end{figure}

\end{document}