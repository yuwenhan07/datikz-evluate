\documentclass{article}
\usepackage{tikz}
\usetikzlibrary{shapes.geometric, arrows}

\begin{document}

\begin{center}
    \begin{tikzpicture}[node distance=2cm, auto]
        \node[draw, rounded rectangle, fill=black!10] (available) {Available Data};
        \node[below of=available, node distance=3cm, text width=5cm, align=center] (unlabeled) {Unlabeled Data \\ $\mathcal{X}_U$};
        \node[left of=unlabeled, node distance=3cm, text width=4cm, align=center] (labeled) {Labeled Data \\ \textcolor{blue}{$\mathcal{X}_L$}};
        
        \draw[->] (labeled.west) -- ++(-1, 0) -| (available);
        \draw[->] (unlabeled.east) -- ++(1, 0) -| (available);
    \end{tikzpicture}
\end{center}

\textbf{Weakly Supervised Learning:} From all of the available data, only a small fraction has been labeled, as represented by the blue area. The remaining data --- gray area --- must be used in an unsupervised manner.

\end{document}