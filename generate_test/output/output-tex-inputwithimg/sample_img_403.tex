\documentclass{standalone}
\usepackage{tikz}
\usetikzlibrary{arrows.meta, matrix}

\begin{document}

\begin{tikzpicture}[
    >={Stealth[round, length=2mm, width=3mm]},
    block/.style={
        draw,
        thick,
        text width=#1,
        minimum height=0.75cm,
        align=center
    },
    block/.default=4cm
]

% Define the dimensions of the blocks
\def\environmentwidth{6cm}
\def\agentwidth{4cm}

% Draw the environment block
\node[block=\environmentwidth] (env) at (-2,0) {Environment};

% Define the position of the agent block relative to the environment
\def\agentpositionx{-2}
\def\agentpositiony{2}

% Draw the agent block
\node[block=\agentwidth, right=of env] (agent) at (\agentpositionx,\agentpositiony) {Agent};

% Connect the agent to the environment with a line
\draw[->] (agent.east) -- ++(1,0) |- (env);

% Draw the reward and state labels inside the environment block
\node[above] at (env.north west) {$S_t$};
\node[below] at (env.south east) {$S_{t+1}$};
\node[left] at (env.west) {\small Reward\\$R_t$\\$R_{t+1}$};
\node[right] at (env.east) {\small Action\\$A_t$};

% Draw the dotted line representing the transition between steps t and t+1
\draw[dotted, thick] (env.south) -- (env.north);
\draw[dotted, thick] (env.south west) -- (env.south east);

\end{tikzpicture}

\caption*{Basic agent-environment relationship in a Markov decision process. The agent chooses an action \( A_t \) and the environment returns a new state \( S_{t+1} \) and a reward \( R_{t+1} \). The dotted line represents the transition from step \( t \) to step \( t+1 \) \cite{suttonbarto}.}

\end{document}