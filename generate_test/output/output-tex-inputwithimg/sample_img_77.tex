\documentclass{article}
\usepackage[utf8]{inputenc}
\usepackage{amsmath}
\usepackage{tikz}
\usetikzlibrary{shapes,arrows,positioning}

\begin{document}

\begin{figure}[h]
    \centering
    \begin{tikzpicture}[node distance=4cm, auto]

        \node [circle, fill=black!20] (v1) {$v_1$};
        \node [circle, fill=black!20, right=of v1] (v2) {$v_2$};
        \node [circle, fill=black!20, below right=0.5cm and 2cm of v1] (v3) {$v_3$};

        \path [->]
            (v1) edge[bend left] node [above] {$+$} (v3)
            (v2) edge[bend left] node [above] {$-$} (v3)
            (v1) edge[dashed, bend right] node [left] {,}'{+}'$\otimes${}'{+}' = '{+}'
            (v2) edge[dashed, bend right] node [right] {,}'{+}'$\otimes${}'{-}' = '{-}'
            (v3) edge[in=-90,out=90,looseness=10,distance=2cm] node [below] {,}'{+}'$\oplus${}'{-}' = '?'
            ;

    \end{tikzpicture}
    \caption{An example for propagating the sign of $v_1$ to $v_3$. During the propagation, all other parents of $v_3$ (here only $v_2$) are also taken into account. Both $v_1$ and $v_2$ are assigned the sign '$+$' and therefore $v_3$ receives the two messages $\text{'$+$'} \otimes \text{'$+$'} = \text{'$+$'}$ and $\text{'$+$'} \otimes \text{'$-$'} = \text{'$-$'}$, which are then combined using the sign addition operator to obtain $\text{'$+$'} \oplus \text{'$-$'} = \text{'$?$'}$ as a new sign for $v_3$.}
    \label{fig:propagation_example}
\end{figure}

\end{document}