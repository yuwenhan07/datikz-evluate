\documentclass{article}
\usepackage{pgfplots}
\pgfplotsset{compat=1.18}

\begin{document}

% Define plotting styles
\definecolor{blueplot}{RGB}{0, 102, 204}
\definecolor{redplot}{RGB}{255, 0, 0}

\begin{figure}[htb]
    \centering
    \begin{tikzpicture}
        \begin{axis}[
            xlabel={$\beta$},
            ylabel={Base of the exponent},
            legend style={at={(0.5,1.03)}, anchor=south, legend columns=-1},
            grid=major,
            xmin=1, xmax=2,
            ymin=1, ymax=2.7,
            xtick={1, 1.2, 1.4, 1.6, 1.8, 2},
            ytick={1, 1.5, 2, 2.5},
            width=0.9\textwidth,
            height=0.5\textwidth,
            legend entries={Our Results, Jana et al. \cite{Jana2021}},
            legend cell align={left},
            legend pos=north east,
            smooth,
            every node near coord/.append style={font=\tiny},
        ]
            % Plot data points
            \addplot[color=blueplot, thick] coordinates {
                (1.0, 2.7)
                (1.1, 2.5)
                (1.2, 2.3)
                (1.3, 2.1)
                (1.4, 1.9)
                (1.5, 1.7)
                (1.6, 1.5)
                (1.7, 1.3)
                (1.8, 1.2)
                (1.9, 1.1)
                (2.0, 1.1)
            };
            \addplot[color=redplot, thick] coordinates {
                (1.0, 2.7)
                (1.1, 2.5)
                (1.2, 2.3)
                (1.3, 2.1)
                (1.4, 1.9)
                (1.5, 1.7)
                (1.6, 1.5)
                (1.7, 1.3)
                (1.8, 1.2)
                (1.9, 1.1)
                (2.0, 1.1)
            };
        \end{axis}
    \end{tikzpicture}
    \caption{Comparison of the running times for $\fvs$. The $x$-axis corresponds to the approximation ratio, while the $y$-axis corresponds to the base of the exponent in the running time. A point at $(\beta, c)$ indicates that there exists a parameterized $\beta$-approximation for \FVS\ in time $c^k \cdot n^{\Oh(1)}$.}
    \label{fig:running_times_fvs}
\end{figure}

\end{document}