\documentclass{article}
\usepackage{tikz}

\begin{document}

\begin{figure}[h]
    \centering
    \begin{tikzpicture}[scale=0.75, transform shape]
        \draw [dashed][ultra thick] (-1,-4) -- (-1,8);
        \draw [dashed][ultra thick] (9,-4) -- (9,8);

        \foreach \y/\state in {0/in-use, 1/in-use, 2/free, 3/in-use, 100/in-use, 101/free, 103/in-use, 1000/in-use, 1001/free, 1002/in-use, 1002/in-use}
        {
            \draw (-0.5,\y) -- ++(10,0);
            \filldraw[black] (0,\y) circle (2pt);
            \node at (0.5,\y+0.25) {\small Chunk \y: \state};
            \node at (5,\y+0.25) {\small header has P = 1};
        }

        \node at (5,5+0.25) {\small ...in-use...};
        \node at (5,3+0.25) {\small ...in-use...};
        \node at (5,9+0.25) {\small ...in-use...};

        \foreach \y/\state in {6/in-use, 7/in-use, 8/free, 9/in-use}
        {
            \draw (-0.5,\y) -- ++(10,0);
            \filldraw[black] (0,\y) circle (2pt);
            \node at (0.5,\y+0.25) {\small Chunk \y: \state};
            \node at (5,\y+0.25) {\small header has P = 1};
        }
    \end{tikzpicture}
    \caption{Heap dump showing a mix of free and in-use chunks. Note: each chunk immediately after a free chunk has a P flag set to 0. Each rectangle represents a chunk.}
    \label{fig:heap_dump}
\end{figure}

\end{document}