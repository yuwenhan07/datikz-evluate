\documentclass{standalone}
\usepackage{pgfplots}
\pgfplotsset{compat=newest}

\begin{document}

\begin{tikzpicture}
    \begin{axis}[
        width=12cm,
        height=8cm,
        xlabel={Frequency(GHz)},
        ylabel={$S_{11}$ (dB)},
        xmin=325, xmax=500,
        ymin=-50, ymax=0,
        legend style={at={(0.97,0.97)}, anchor=north east},
        ytick={-50,-25,0},
        yticklabels={-50,$-25$,0},
        grid=both,
        minor tick num=4,
        legend entries={Measurement, Simulation},
        legend cell align={left},
        label style={font=\small},
        tick label style={font=\small},
        title style={font=\scriptsize},
        title={Return loss. Comparison of measured and simulated $S_{11}$ versus frequency of an aligned diagonal horn antenna. Measurement results of the WM570 open-ended waveguide probe are shown in the black line.},
        ]
        % Plot measurement data
        \addplot[red, thick, mark=x] coordinates {
            % Add your data points here
            (325, -26)    % Example data point
            (330, -26.5)
            ...
            (495, -24.5)
        };
        
        % Plot simulation data
        \addplot[black, thick, mark=x] coordinates {
            % Add your data points here
            (325, -27)    % Example data point
            (330, -27.5)
            ...
            (495, -25.5)
        };
        
        % Annotations
        \node[above=2ex, anchor=west, font=\footnotesize] at (axis cs:360,-27) {Open-ended wg. probe};
        \draw[-latex, gray] (axis cs:360,-27.5) -- (axis cs:355,-26);
        
        \node[above=2ex, anchor=west, font=\footnotesize] at (axis cs:390,-35) {Diagonal horn};
        \draw[-latex, gray] (axis cs:390,-35.5) -- (axis cs:385,-34);
    \end{axis}
\end{tikzpicture}

\end{document}