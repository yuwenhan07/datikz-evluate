\documentclass[11pt]{article}
\usepackage{amsmath, amssymb, amsthm}
\usepackage{color}
\usepackage{tikz}

\begin{document}

\begin{figure}[ht]
    \centering
    \begin{tikzpicture}[scale=0.8]
        
        % PART (a)
        \draw[blue] (-7,-0.5) -- (-4,-0.5);
        \draw[red] (-6,-0.5) -- (-5,-0.5);
        \draw[thin, black] (-3,-0.5) -- (-2,-0.5);
        
        \filldraw[black, fill=red, radius=1mm] (-5,-0.5) circle;
        
        \draw (-6,-0.5) node[above] {$X$};
        \draw (-4,-0.5) node[below] { (a) };
        
        % PART (b)
        \draw[blue] (-1,-0.5) -- (-0.5,-0.5);
        \draw[red] (-0.5,-0.5) -- (-0.45,-0.5);
        \draw[thin, black] (-0.5,-0.5) -- (0,-0.5);
        \filldraw[black, fill=red, radius=1mm] (-0.5,-0.5) circle;
        
        \draw (-0.5,-0.5) node[above] {$X$};
        \draw (0,-0.5) node[below] {(b)};
        
        % PART (c)
        \draw[blue] (1,-0.5) -- (4,-0.5);
        \draw[red] (1,-0.5) -- (1.05,-0.5);
        \draw[thin, black] (4,-0.5) -- (4.5,-0.5);
        \filldraw[black, fill=red, radius=1mm] (1,-0.5) circle;
        
        \draw (1,-0.5) node[above] {$X$};
        \draw (4.5,-0.5) node[below] {(c)};
    \end{tikzpicture}
    
    \caption{Cartoon plot for various boundary subsystems. CFT$_{\text{I,II}}$ are two halves of the straight line and the red dot is the interface. $X$ is the subsystem. (a) the interface is at the interior of the subsystem and the subsystem is finite; (b) the interface is at one end of the subsystem and the subsystem is finite; (c) the interface is at one end of the subsystem and the subsystem is infinite.}
    \label{fig:boundary_subsystems}
\end{figure}

\end{document}