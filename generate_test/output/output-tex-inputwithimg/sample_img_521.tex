\documentclass[12pt]{article}
\usepackage{amsmath}
\usepackage{amsfonts}
\usepackage{graphicx}

\begin{document}

Consider the set of distributions $\{P_{Y|X}(\bullet|x): x \in \mathbb{X}\}$ lying in a transformation model $\mathcal{P}_{\mathbb{G}} = \{g_*P_0 : g \in \mathbb{G}\} \subseteq \mathcal{P}(\mathbb{Y})$. For each $x \in \mathbb{X}$, there exists an element $g_x$ of the group $\mathbb{G}$ such that $P_{Y|X}(\bullet|\phi x) = g_x*P_0$. When transforming $x$ with $\phi$, there is also an element $g_{\phi x}$ such that $P_{Y|X}(\bullet|x) = g_{\phi x}*P_0$. This enables direct transition from $P_{Y|X}(\bullet|x)$ to $P_{Y|X}(\bullet|\phi x)$ via the coboundary $c(\phi,x): (\phi,x) \mapsto g_{\phi x}g_x^{-1} \in \mathbb{G}$. The coboundary $c(\phi,x)$ does not depend on $P_0$ and thus applies to (infinitely) many other conditional distributions. Modelling the coboundary instead of $P_{Y|X}$ helps improve robustness against model mis-specification.

\begin{figure}[h]
    \centering
    \includegraphics[width=\textwidth]{image.png}
\end{figure}

\end{document}