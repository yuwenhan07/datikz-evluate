\documentclass[12pt]{article}
\usepackage{amsmath}
\usepackage{tikz}

\begin{document}

\begin{figure}[h]
    \centering
    \begin{tikzpicture}[scale=1.5]

        % Draw the horizontal line
        \draw[thick] (-3,0) -- (4.5,0);

        % Labels for points
        \node at (-3,-0.2) {$c(k)$};
        \node at (2.5,-0.2) {$c((\gamma_k+\beta)s)$};
        \node at (3.6,-0.2) {$c((\gamma_k+\beta+t\alpha)s)$};

        % Draw circles
        \draw[red, thick, fill=white] (-2,0) circle (0.5cm);
        \draw[red, thick, fill=white] (2.5,0) circle (0.5cm);
        \draw[red, thick, fill=white] (3.5,0) circle (0.5cm);

        % Points within circles
        \filldraw[black] (-2+0.5*cos(30),0+0.5*sin(30)) circle (0.05cm) node[anchor=south] {$\hat{q}_k$};
        \filldraw[black] (2.5+0.5*cos(90),0+0.5*sin(90)) circle (0.05cm) node[anchor=east] {$\hat{q}_{l_0}$};
        \filldraw[black] (3.5+0.5*cos(60),0+0.5*sin(60)) circle (0.05cm) node[anchor=north west] {$\hat{q}_2$};

        \filldraw[black] (2.5,0) circle (0.05cm) node[anchor=south] {$\hat{q}_1$};

        % Draw the lines connecting the points
        \draw[thick] (-2+0.5*cos(30),0+0.5*sin(30)) -- (2.5+0.5*cos(90),0+0.5*sin(90));
        \draw[thick] (2.5+0.5*cos(90),0+0.5*sin(90)) -- (3.5+0.5*cos(60),0+0.5*sin(60));

    \end{tikzpicture}
    \caption{Picture in Case 1. The disks $B(\hat{q}_k)$'s cover the segment connecting $c((\gamma_k+\beta)s)$ and $c((\gamma_k+\beta+t\alpha)s)$. There is a smallest $l_0$ such that $d(q_k,\hat q_{l_0})\geq (\beta-t\alpha)s.$}
    \label{fig:case1}
\end{figure}

\end{document}