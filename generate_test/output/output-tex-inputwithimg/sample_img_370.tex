\documentclass{article}
\usepackage{amsmath}
\usepackage{tikz}
\usetikzlibrary{shapes.geometric, shapes.multipart}

\tikzset{
    circ/.style={circle, draw, minimum size=10mm},
    box/.style={rectangle split, rectangle split parts=2, draw, text width=9mm, minimum height=6mm, align=center},
    plus/.style={draw, minimum size=8mm},
    tree/.style={level distance=15mm, sibling distance=30mm},
    root/.style={circ},
    step/.style={plus},
    q/.style={box},
    R/.style={root},
    r/.style={R,text depth=+1ex},
    q'.style={box,fill=white,minimum width=14mm,minimum height=12mm},
    R'.style={R'},
    step'.style={q'$},
    line/.style={draw,-latex}
}

\begin{document}

\begin{center}
\begin{tikzpicture}[tree]
    \node[r] {$\mathcal{R}_0$}
        child{node[r] {$\mathcal{R}_1$}
            child{node[q] {$q_{1}$}
                child{node[R'] {$\mathbf{R}_{0}$}
                    child{node[step]{}
                        child{node[step']{$+$} }
                        child{node[step']{$+$} }
                        child{node[step']{}}
                        child{node[step']{}}
                        child{node[q'] {
                            edge from parent[draw=none]
                            node[left,near start] {$q^{(0)}_{4i}$}
                            node[right,near end] {$q^{(1)}_{4i}$}
                        } }
                        child{node[q'] {
                            edge from parent[draw=none]
                            node[left,near start] {$q^{(0)}_{4i+1}$}
                            node[right,near end] {$q^{(1)}_{4i+1}$}
                        } }
                        child{node[q'] {
                            edge from parent[draw=none]
                            node[left,near start] {$q^{(0)}_{4i+2}$}
                            node[right,near end] {$q^{(1)}_{4i+2}$}
                        } }
                        child{node[q'] {
                            edge from parent[draw=none]
                            node[left,near start] {$q^{(0)}_{4i+3}$}
                            node[right,near end] {$q^{(1)}_{4i+3}$}
                        } }
                    }
                    child{node[R'] {$\mathbf{R}_{1}'$}
                        child{node[R'] {$\mathbf{R}_{0}'$}}
                    }
                    child{node[R'] {$\mathbf{R}_{2}'$}}
                }
            }
            child{node[plus] {}
                child{node[plus] {}}
                child{node[plus] {}}
            }
        };
\end{tikzpicture}
\captionof{figure}{A high-level scheme for sequential readout of word size $b=2$, an extension to the scheme in \cref{fig: unconditional_optimal_layout}.}
\end{center}

\end{document}