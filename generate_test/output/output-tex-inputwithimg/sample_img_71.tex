\documentclass{article}
\usepackage{tikz}
\usetikzlibrary{arrows.meta}

\begin{document}

\begin{figure}[h]
    \centering
    \begin{tikzpicture}[scale=0.8, transform shape, font=\scriptsize]

        % Draw the timeline
        \draw[thick,-Triangle] (0,0) -- (14,0);

        % Labels for time steps
        \foreach \x/\label in {1/1, 4/1+S_j, 11/1+S_j+S_{j'}}
        {
            \draw (\x,0) -- ++(0,-2mm) node[below] {$\label$};
        }

        % Labels for available jobs
        \node at (0.5, -1) {\textbf{Available Jobs}};
        \node at (0.5, -2) {$[n]$};

        % Labels for jobs run
        \node at (0.5, -3) {\textbf{Job Run}};
        \node at (0.5, -4) {$j$};
        \node at (0.5, -6) {$j'$};

        % Labels for rewards up to time t
        \node at (0.5, -5) {\textbf{Reward up to time $t$}};
        \node at (0.5, -7) {$v_j$};
        \node at (0.5, -9) {$v_j + v_{j'}$};

        % Additional values
        \node at (8.5, -1) {\dots};
        \node at (12.5, -2) {$R_{[t+2,S_{j''}]}$};
        \node at (12.5, -3) {$j''$};
        \node at (12.5, -5) {$v_j + v_{j'} + v_{j''}$};

    \end{tikzpicture}
    \caption{Sample execution according to our model. We select job $j$ at time $t=1$, which gives a value $v_j$ and causes the system to be busy for $S_j$ time units. Once the system is free, we select job $j'$ from the set of available jobs and obtain an additional value $v_{j'}$, and so on.}
    \label{fig:sample_execution}
\end{figure}

\end{document}