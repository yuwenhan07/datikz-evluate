\documentclass{article}
\usepackage{amsmath} % For mathematical environments
\usepackage{xcolor} % For color customization

\begin{document}

\begin{center}
    \begin{tabular}{|c|c|c|c|c|c|c|}
    \hline
        3 & 1 & 1 & 1 & 0 & 2 & 0 \\
        \hline
        2 & 0 & 1 & 1 & 1 & 0 & 0 \\
        \hline
        1 & 2 & 0 & 1 & 1 & 1 & 1 \\
        \hline
        4 & 3 & 0 & 1 & 1 & 0 & 0 \\
        \hline
        3 & 0 & 1 & 1 & 0 & 2 & 4 \\
        \hline
        0 & 1 & 1 & 0 & 4 & 0 & 5 \\
        \hline
        1 & 1 & 0 & 0 & 5 & 0 & 0 \\
        \hline
    \end{tabular}
    \quad
    \begin{tabular}{|c|c|c|c|c|c|c|}
    \hline
        0 & 1 & 0 & 1 & 0 & 0 & 0 \\
        \hline
        0 & 0 & 0 & 0 & 0 & 0 & 0 \\
        \hline
        0 & 0 & 0 & 0 & 0 & 1 & 1 \\
        \hline
        0 & 0 & 0 & 0 & 1 & 0 & 0 \\
        \hline
        0 & 0 & 0 & 1 & 0 & 0 & 0 \\
        \hline
        0 & 0 & 0 & 1 & 0 & 0 & 0 \\
        \hline
        1 & 1 & 0 & 0 & 0 & 0 & 0 \\
        \hline
    \end{tabular}

    \vspace{5mm}

    \begin{tabular}{c}
        2D dense matrix \\
        \hline
        2D sparse matrix \\
    \end{tabular}

\end{center}

Illustrative comparison of a 2D dense matrix and a 2D sparse matrix. The dense matrix predominantly features non-zero elements, whereas the sparse matrix consists mainly of zero values with a few non-zero entries scattered throughout. Such distinctions highlight the storage and computational differences between the two matrix types.

\end{document}