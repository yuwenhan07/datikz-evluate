\documentclass{article}
\usepackage{tikz}
\usetikzlibrary{decorations.pathmorphing}
\usepackage{amsmath}

\begin{document}

\begin{tikzpicture}[scale=1.2]
    \draw[->] (-7,0) -- (7,0) node[right] {$x \ln x$};
    \draw[->] (-7,-1) -- (7,-1) node[right] {$(1+a)x \ln x$};
    \draw[->] (-7,-2) -- (7,-2) node[right] {$(1+2a)x \ln x$};
    
    % Red line with random fluctuations
    \draw[red, thick, decorate, decoration={random steps, segment length=1mm, amplitude=3pt}] (-7,-1.5) -- (0,-1.5);
    \filldraw[red] (0,-1.5) circle (2pt);
    
    % Black curve representing the upper bound
    \draw[thick, black] plot[domain=0:6, samples=100] (\x, {-(\x+1)*ln(\x)});
    
    % Markers for the different layers
    \node[above=2pt] at (7,-1) {$(1+a)x \ln x$};
    \node[above=2pt] at (7,-2) {$(1+2a)x \ln x$};
    
    % Description of the process
    \node at (0,-2.5) {(The red line tries to cross the layer $(1+a)n \ln(n)$)};
    \node at (0,-3) {but is blocked by the black bound}};
    
    % Arrows indicating the direction of the red line
    \draw[->, red, ultra thick] (0,-1.5) -- (3,-1.5);
    \draw[->, red, ultra thick] (0,-1.5) -- (0.5, -1.8);
    
\end{tikzpicture}

\end{document}