\documentclass{article}
\usepackage{amsmath}
\usepackage{tikz}
\usetikzlibrary{matrix}

\begin{document}

\begin{figure}[h]
\centering
\begin{tikzpicture}
    \matrix (m2) [matrix of math nodes,
                  nodes={draw, minimum width=1cm, minimum height=1cm},
                  column sep=-\pgflinewidth,
                  row sep=-\pgflinewidth] {
        b_1 & b_2 \\
        b_3 & b_4 \\
    };
    
    \node[below left] at (m2.south west) {$B_2 = \text{Hex}(b_1b_2b_3b_4)$};
    
    \matrix (m4) [matrix of math nodes,
                  nodes={draw, minimum width=1cm, minimum height=1cm},
                  column sep=-\pgflinewidth,
                  row sep=-\pgflinewidth,
                  right=of m2] {
        B_1^1 & B_2^2 \\
        B_2^3 & B_2^4 \\
    };
    
    \node[below] at (m4.south) {$B_4 = B_1^1B_2^2B_2^3B_2^4$};
    
    \matrix (m8) [matrix of math nodes,
                  nodes={draw, minimum width=1cm, minimum height=1cm},
                  column sep=-\pgflinewidth,
                  row sep=-\pgflinewidth,
                  right=of m4] {
        B_1^1 & B_2^4 \\
        B_4^3 & B_4^4 \\
    };
    
    \node[below] at (m8.south) {$B_8 = B_1^1B_1^2B_4^3B_4^4$};
\end{tikzpicture}
\caption{Raster scan order of bits or blocks $B_n^i$.}
\label{fig:raster_scan_order}
\end{figure}

The bits (\(b_i\)) or blocks \(B_n^i\) are read in a raster scan order, where \(n\) indicates the patch size and \(i\) indicates the bit or the block.
\end{document}