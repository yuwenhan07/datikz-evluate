\documentclass{article}
\usepackage{tikz}

\begin{document}

\begin{center}
    \begin{tikzpicture}[scale=0.8]
        % Draw the horizontal axis
        \draw[thick] (-5,0) -- (10,0);

        % Draw dashed lines
        \draw[dashed] (-4,-0.5) -- (9,-0.5);

        % Labels for points a
        \node at (-4,0) {$a_1$};
        \node at (-2,0) {$a_2$};
        \node at (0,0) {$a_3 \& b_1$};
        \node at (2,0) {$a_4 \& b_2$};

        % Labels for points b
        \node at (6,0) {$a_i \& b_{i-2}$};
        \node at (8,0) {$b_{i-1}$};
        \node at (10,0) {$b_{i}$};

        % Draw arcs for the connections
        \foreach \x in {1,...,4}
            \draw (\x*2-4,0) arc (180:90:\x*2-4) arc (90:-90:\x*2-4);
        \foreach \x in {6,...,9}
            \draw (\x*2-8,0) arc (-180:-90:\x*2-8) arc (90:180:\x*2-8);

        % Dotted line between the two groups
        \draw[dotted] (4,-0.5) -- (7,-0.5);
        \node at (5.5,-0.5) {$\cdots$};
    \end{tikzpicture}
\end{center}

The output of \ref{mainint}. $a_3, b_1$ have been drawn in the same place because we do not know which is larger.

\end{document}