\documentclass[12pt]{article}
\usepackage{amsmath}
\usepackage{graphicx}

\begin{document}

\noindent \textbf{Penrose Diagram of a FRW Solution with Negative Spatial Curvature}

The Penrose diagram in Figure~\ref{fig:penrose_diagram} illustrates the evolution of a Friedmann-Robertson-Walker (FRW) model with negative spatial curvature, which transitions from a phase where the Cosmic Censorship Conjecture (CCC) is violated to a nonaccelerating phase.

\begin{figure}[h]
    \centering
    \includegraphics[width=0.8\textwidth]{penrose_diagram.png}
    \caption{Penrose diagram of a FRW solution with negative spatial curvature. Phase I shows epochs where the expansion rate $\ddot{a}$ is positive, violating the Cosmic Censorship Conjecture (TCC). Phase II comprises two subphases:
        - In Phase II-A, $\dot{\ddot{a}} < 0$, the spatial curvature $\Omega_k$ is negligible, and the expansion is driven solely by the scalar field.
        - In Phase II-B, $\dot{\ddot{a}} \ll 0$, the spatial curvature becomes significant, and $\Omega_k$ approaches a non-zero value.}
    \label{fig:penrose_diagram}
\end{figure}

\textbf{Phase I:} This phase features epochs characterized by $\ddot{a} > 0$, leading to collective violations of the Cosmic Censorship Conjecture (TCC).

\textbf{Phase II:} This phase includes two subphases:
- \textbf{Phase II-A:} $\dot{\ddot{a}} < 0$, $\Omega_k \ll 1$.
    - The spatial curvature is negligible, and the expansion is solely driven by the scalar field.
- \textbf{Phase II-B:} $\dot{\ddot{a}} \ll 0$, $aH < \infty$.
    - The spatial curvature becomes important, and $\Omega_k$ converges to a non-zero value.

The figure demonstrates how the cosmic expansion dynamics evolve from a phase where the CCC is violated to one where it is preserved, illustrating the transition through distinct phases of the FRW model with negative spatial curvature.

\end{document}