\documentclass{article}
\usepackage[margin=1in]{geometry}
\usepackage{tikz}
\usetikzlibrary{matrix}

\begin{document}

\begin{center}
\begin{tikzpicture}[node distance=2cm]
    \matrix (m) [
        matrix of nodes,
        row sep={3mm,scale=2},
        column sep={3mm,scale=2},
        nodes={
            rectangle,
            draw,
            minimum width=5cm,
            minimum height=4cm,
            text width=5cm,
            align=center,
            inner sep=0pt,
            outer sep=0pt
        }
    ]
    {
        State = (x,y) \\
        Reward = cx \\
        & \\
        & \\
    };
    
    \node at (m-1-1.south east) [above left, anchor=south west] {\includegraphics[width=1cm]{example-image-circle}};
    \draw (m-1-1.south east) -- ++(-90:2mm) -- ++(-90:4mm);
    \draw (m-1-1.south east) -- ++(-90:6mm) -- ++(-90:4mm);
    
    \matrix (n) [
        matrix of nodes,
        row sep={3mm,scale=2},
        column sep={3mm,scale=2},
        nodes={
            rectangle,
            draw,
            minimum width=5cm,
            minimum height=4cm,
            text width=5cm,
            align=center,
            inner sep=0pt,
            outer sep=0pt
        }
    ]
    at ([shift=(m-1-1.east)] m-1-1.east)
    {
        +10 \\
        & \\
        & \\
        +1 \\
    };
    
    \node at (n-1-1.south west) [above right, anchor=south east] {\includegraphics[width=1cm]{example-image-circle}};
    \draw (n-1-1.south west) -- ++(90:2mm) -- ++(90:4mm);
    \draw (n-1-1.south west) -- ++(90:6mm) -- ++(90:4mm);
    
    \draw[dotted, thick] (m-1-1.south east) -- (n-1-1.north east);
    \draw[->,thick] (m-1-1.south east) -- (n-1-1.south east);
    \draw[->,thick] (n-1-1.south west) -- (n-1-1.south east);
    
    \draw (m-1-1.south east) -- ++(90:2mm) -- ++(90:4mm);
    \draw (m-1-1.south east) -- ++(90:6mm) -- ++(90:4mm);
    
    \draw (n-1-1.south east) -- ++(90:2mm) -- ++(90:4mm);
    \draw (n-1-1.south east) -- ++(90:6mm) -- ++(90:4mm);
    
    \draw (n-1-1.south east) -- ++(-90:2mm) -- ++(-90:4mm);
    \draw (n-1-1.south east) -- ++(-90:6mm) -- ++(-90:4mm);
    
    \node (arrow) at (n-1-1.north west) {$+$};
\end{tikzpicture}
\captionof{figure}{A diagram depicting example MDPs where the reward or transition function varies. This diagram is used to illustrate how policies may vary with respect to changes in their reward (left) or transition functions (right). In the figure on the right, the box with arrows in it indicates a treadmill. If the agent is faster than the treadmill, it can pass over the treadmill. Otherwise, the treadmill would push the agent backwards.}
\end{center}

\end{document}