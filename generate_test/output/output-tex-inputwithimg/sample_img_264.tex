\documentclass{article}
\usepackage{pgfplots}
\pgfplotsset{compat=newest}

\begin{document}

% Define the data for the plot
\def\data{
    -10/0.5, -9/0.8, -8/0.6, -7/0.4, -6/0.3, -5/0.2, -4/0.1, -3/0, -2/-0.1,
    -1/-0.2, 0/-0.3, 1/-0.4, 2/-0.5, 3/-0.6, 4/-0.7, 5/-0.8, 6/-0.9,
    7/-1, 8/-0.9, 9/-0.8, 10/-0.7
}

% Set up the plot
\begin{tikzpicture}
    \begin{axis}[
        width=0.8\textwidth,
        height=0.6\textwidth,
        xmin=-15,
        xmax=15,
        ymin=-1,
        ymax=1,
        xlabel={Base position relative to alignment},
        ylabel={Conditional mean},
        grid=major,
        minor tick num=4,
        xtick={-15,-12,...,15},
        ytick={-1,-0.8,...,1},
        legend style={at={(0.95,0.95)}, anchor=north east, draw=black, fill=white, font=\footnotesize}
    ]
        % Plot the data points with markers
        \addplot[only marks, mark size=1pt, mark options={solid}, mark repeat=1] table [x index=0, y index=1] {\data};
        
        % Fill the area under the line
        \draw[fill=gray!20] (-10,0.5) -- plot coordinates {(-10,0.5)} -- plot coordinates {(9,-0.8)} -- (9,-0.8);
        
        % Add the legend
        \legend{A, C, G, T};
    \end{axis}
\end{tikzpicture}

% Caption for the plot
\captionof{figure}{Mean (normalized) R10 ONT raw signal value conditioned on individual nucleotide base at a given (relative) position. Over a large set of sequences aligned to their respective raw signals, the value on the y-axis is calculated by computing the average signal value over all signal values where a base at a position relative to the aligned base attains the value indicated by the legend. The lighter area indicates the sequence of bases that forms the \(10\)-mers used in this work. Our \(10\)-mers extend the \(9\)-mers used in ONTs Remora toolkit by the leftmost base at relative position \(-7\), which allows us to more effectively handle the step-back mechanism described later.}

\end{document}