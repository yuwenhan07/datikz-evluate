\documentclass{article}
\usepackage{graphicx}
\usepackage{amsmath}
\usepackage{amsfonts}
\usepackage{amssymb}
\usepackage[utf8]{inputenc}
\usepackage{color}
\usepackage{hyperref}

\begin{document}

\begin{figure}[h]
    \centering
    \includegraphics[width=\linewidth]{path/to/your/image.png} % Replace with actual path to your image
    \caption{\textbf{Evaluation on Hi4D~\cite{hi4d} with tight thresholds}. We report 3DPCKs with a tight threshold from 0 to 100mm. The results show that our method outperforms others by a large margin even in tight thresholds. The notations of methods follow those in Tab.~\ref{tab:hi4d_eval}.}
    \label{fig:hi4d_tight_thresholds}
\end{figure}

The evaluation is carried out using the Hi4D dataset~\cite{hi4d} under tight thresholds, as shown in Fig.~\ref{fig:hi4d_tight_thresholds}. We present the 3DPCK scores across various thresholds ranging from 0 to 100 millimeters. Our proposed method demonstrates superior performance compared to other techniques across all thresholds, particularly in the tighter settings. The notation for each method corresponds to the entries in Table~\ref{tab:hi4d_eval}, which details their respective configurations.

\begin{table}[h]
    \centering
    \begin{tabular}{c c}
        \hline
        Method & Notation \\
        \hline
        Ours & Ours \\
        \hline
        Feature VP & Feature VP \\
        \hline
        VehiclePose & VehiclePose \\
        \hline
        Feature VP & Feature VP \\
        \hline
        Ours & Ours \\
        \hline
        4DA & 4DA \\
        \hline
    \end{tabular}
    \caption{Notations of Methods Used in Hi4D Evaluation.}
    \label{tab:hi4d_eval}
\end{table}

\end{document}