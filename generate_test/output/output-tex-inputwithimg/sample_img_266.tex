\documentclass{article}
\usepackage{tikz}
\usetikzlibrary{matrix}

\begin{document}

\section*{Diagonal Transformation and Encoding}

The image illustrates a diagonal transformation that is part of Pinsky's combinatorial theorem. The choice of the next step being among \(\{(0,1),(1,0),(-1,0),(0,-1)\}\) is equivalent to a diagonal version of the encoding being in the set \(\{(+1,+1),(+1,-1),(-1,-1),(-1,+1)\}\). This equivalence is demonstrated by the pattern of crosses shown.

\begin{figure}[h]
    \centering
    \begin{tikzpicture}
        \matrix (m) [matrix of nodes]{
            \draw (0,0) -- (1,1); & \draw (0,0) -- (1,1); & \draw (0,0) -- (1,1); & \draw (0,0) -- (1,1); \\
            \draw (0,1) -- (1,0); & \draw (0,1) -- (1,0); & \draw (0,1) -- (1,0); & \draw (0,1) -- (1,0); \\
            \draw (0,2) -- (1,3); & \draw (0,2) -- (1,3); & \draw (0,2) -- (1,3); & \draw (0,2) -- (1,3); \\
            \draw (0,3) -- (1,4); & \draw (0,3) -- (1,4); & \draw (0,3) -- (1,4); & \draw (0,3) -- (1,4); \\
        };
        
        \node at (2.5, 0) {\(x_{1,1}\)};
        \node at (2.5, 1) {\(x_{1,2}\)};
        \node at (2.5, 2) {\(x_{1,3}\)};
        \node at (2.5, 3) {\(x_{1,4}\)};
        
        \node at (4.5, 0) {\(x_{2,1}\)};
        \node at (4.5, 1) {\(x_{2,2}\)};
        \node at (4.5, 2) {\(x_{2,3}\)};
        \node at (4.5, 3) {\(x_{2,4}\)};
        
        \node at (6.5, 0) {\(x_{3,1}\)};
        \node at (6.5, 1) {\(x_{3,2}\)};
        \node at (6.5, 2) {\(x_{3,3}\)};
        \node at (6.5, 3) {\(x_{3,4}\)};
        
        \node at (8.5, 0) {\(x_{4,1}\)};
        \node at (8.5, 1) {\(x_{4,2}\)};
        \node at (8.5, 2) {\(x_{4,3}\)};
        \node at (8.5, 3) {\(x_{4,4}\)};
    \end{tikzpicture}
    \caption{A grid with diagonals representing the transitions and encoding in the context of Pinsky's combinatorial theorem.}
    \label{fig:diagonal_transformation}
\end{figure}

\end{document}