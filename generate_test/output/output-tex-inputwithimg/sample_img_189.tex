\documentclass{article}
\usepackage{pgfplots}
\pgfplotsset{compat=1.16}

\begin{document}

% Left: Common rail pressure system (system for high-pressure fuel injection) with controllable inputs $v_k, n_k$ and measured output $\psi_k$, taken from \cite{zimmer2018safe, tietze2014model}
% Right: This diagram shows the decline in RMSE in the rail pressure model from Subsection~\ref{subsection_railpressure}. The entropy method (red line) shows a slow decline over the 1000 steps, whereas our approach T-IMSPE (blue line) declines more consistently, faster, and ends in much smaller RMSE values.

\begin{figure}[h]
    \centering
    \begin{tikzpicture}
        \begin{axis}[
            title={},
            xlabel={steps},
            ylabel={RMSE},
            xmin=0, xmax=1000,
            ymin=0, ymax=8,
            xtick={0,250,500,750,1000},
            ytick={0,2,4,6,8},
            legend pos=north west,
            ymajorgrids=true,
            grid style=dashed,
            ]
            
            \addplot[
                color=red,
                mark=none,
                domain=0:1000,
                samples=100,
            ]
            {random(6,8)};
            \addlegendentry{entropy};
            
            \addplot[
                color=blue,
                mark=none,
                domain=0:1000,
                samples=100,
            ]
            {random(2,4)};
            \addlegendentry{T-IMSPE (ours)};
            
        \end{axis}
    \end{tikzpicture}
    \caption{Decline in RMSE in the rail pressure model from Subsection~\ref{subsection_railpressure}. The entropy method (red line) shows a slow decline over the 1000 steps, whereas our approach T-IMSPE (blue line) declines more consistently, faster, and ends in much smaller RMSE values.}
    \label{fig:rmse_railpressure}
\end{figure}

\end{document}