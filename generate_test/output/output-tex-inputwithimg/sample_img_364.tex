\documentclass{article}
\usepackage{amsmath}
\usepackage{tikz}
\usepackage{tikz-cd}

\begin{document}

\begin{center}
    \begin{tikzcd}[column sep=large]
        \text{attn0[5]@37} \arrow{dr}\\ 
        & \text{embed0@37} \arrow{r}{1.1} \arrow[red]{dr}|{\textcolor{white}{-}} \arrow[leftarrow, bend left=40]{dll}{2.1} & \text{tc0[9188]@37} \arrow{r}{1.1} \arrow[dashed, bend right=40]{ddl}{3.1} & \text{tc2[3900]@37}\\
        \text{attn0[1]@37} \arrow{rrr}{1.1}\arrow{dr}& \\
        & \text{tc0[16632]@37} \arrow{r}{2.4} & \text{tc1[22184]@37} & \text{tc8[355]@37} \arrow[red]{dl}|{\textcolor{white}{-}}\\
        \text{attn0[3]@37} \arrow[bend right=40]{urr}{2.1} & & &\\
        & & \text{tc3[6238]@37} \arrow[red, bend right=40]{uuu}{4.1} \\
    \end{tikzcd}
\end{center}

\begin{description}
    \item[Description:] An example of a computational graph produced using the method in \S\ref{sec:subgraphs} characterizing how our unknown feature is computed on an unseen input. A single path is highlighted in red and annotated with component-by-component attributions.
\end{description}

\end{document}