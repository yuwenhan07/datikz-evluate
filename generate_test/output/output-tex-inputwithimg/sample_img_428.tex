\documentclass{standalone}
\usepackage{tikz}
\usetikzlibrary{shapes,arrows}

\begin{document}

\tikzset{
    block/.style = {draw, fill=white, rectangle, minimum height=3em,
                    minimum width=6em},
    line/.style = {draw, -latex'},
}

\begin{tikzpicture}[node distance = 2cm, auto]
    \node [coordinate] (input) {};
    \node [block, right of=input] (esti) {Local Estimation $\mathcal{T}$};
    \node [block, right of=esti, node distance=4cm] (nnet) {$G_{\theta}(\cdot)$};
    \node [coordinate, right of=nnet] (output) {};
    
    \draw [line] (input) -- node [near start] {$Y(\mathbf{x})$} (esti);
    \draw [line] (esti) -- node [near start] {$\tilde{\varphi}(\mathbf{x})$} (nnet);
    \draw [line] (nnet) -- node [near start] {$\hat{\varphi}(\mathbf{x})$} (output);
\end{tikzpicture}

Data processing pipeline: we first perform an initial and simple estimation on $Y(\mathbf{x})$, generating the initial estimate $\tilde{\varphi}(\mathbf{x})$; then, we use a neural network to refine it, producing the final estimate $\hat{\varphi}(\mathbf{x})$.

\end{document}