\documentclass{article}
\usepackage{amsmath}
\usepackage{tikz}
\usetikzlibrary{arrows.meta}

\begin{document}

\begin{figure}[h]
    \centering
    \begin{tikzpicture}[scale=0.8]
        % Draw the horizontal line for the path
        \draw[thick] (-10, 2) -- (10, 2);
        
        % Draw the dashed lines for the heights
        \draw[dashed, thick] (-7, 3) -- (-5, 1.5);
        \draw[dashed, thick] (-5, 1.5) -- (-3, 0.5);
        \draw[dashed, thick] (-3, 0.5) -- (0, 0.5);
        \draw[dashed, thick] (0, 0.5) -- (2, -0.5);
        \draw[dashed, thick] (2, -0.5) -- (4, -1.5);
        
        % Draw the solid lines for the paths
        \draw[thick] (-7, 3) -- (-5, 1.5);
        \draw[thick] (-5, 1.5) -- (-3, 0.5);
        \draw[thick] (-3, 0.5) -- (0, 0.5);
        \draw[thick] (0, 0.5) -- (2, -0.5);
        \draw[thick] (2, -0.5) -- (4, -1.5);
        
        % Draw the labels for the paths
        \node at (-6.5, 2.5) {$h_0$};
        \node at (-6.5, 0.75) {$h_0'$};
        
        % Draw the particle configuration
        \foreach \x in {-9, -7, -5, -3, -1, 1, 3, 5, 7, 9} {
            \draw[fill=white] (\x, -2) circle (0.2);
        }
        
        % Mark specific particles
        \draw[fill=black] (-4, -2) circle (0.2);
        \draw[fill=black] (0, -2) circle (0.2);
        \draw[fill=black] (4, -2) circle (0.2);
        
        % Draw the labels for the particles
        \node at (-3.5, -1.5) {$k$};
        \node at (0.5, -1.5) {$d$};
        
        % Draw the arrows for the particles
        \draw[->, thick] (-3.5, -1.5) -- (-3.5, 0);
        \draw[->, thick] (0.5, -1.5) -- (0.5, 0);
        
        % Draw the dotted lines connecting the particles
        \draw[dotted, thick] (-4, -2) -- (-4, -1.5);
        \draw[dotted, thick] (0, -2) -- (0, -1.5);
        \draw[dotted, thick] (4, -2) -- (4, -1.5);
    \end{tikzpicture}
    \caption{A depiction of a Bernoulli path \( h_0 \) and the associated particle configuration \( \eta_{h_0} \). The dashed portions of the path illustrate the definition of the new height function, under the corresponding displayed particle movements.}
    \label{fig:BernoulliPathAndParticles}
\end{figure}

\end{document}