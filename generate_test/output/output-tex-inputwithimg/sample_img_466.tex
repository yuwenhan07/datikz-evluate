\documentclass{article}
\usepackage{amsmath, amssymb, amsthm, graphicx}

\newtheorem{lemma}{Lemma}

\begin{document}

Illustration for Lemma \ref{lem:gdint}. The boundary Harnack principle cannot be used to estimate increments between \( y \) and \( y' \) due to the singularity at \( z \). Instead, we demonstrate regularity within the smaller ball by leveraging harmonicity in the larger ball.

\begin{figure}[h]
    \centering
    \includegraphics[width=0.8\textwidth]{path_to_image} % Replace with actual path to image
    \caption{Illustration for Lemma \ref{lem:gdint}: Regularity within the smaller ball \( D \cap B(Q, 3\delta_D(y)) \) using harmonicity in the larger ball.}
    \label{fig:gdint_illustration}
\end{figure}

The figure above shows the region \( D \cap B(Q, 3\delta_D(y)) \), which is a smaller ball centered at \( Q \) with radius \( 3\delta_D(y) \), where \( \delta_D(y) \) is the distance from \( y \) to the boundary \( \partial D \). The larger ball \( B(Q, r) \) is used to ensure harmonicity properties, while the singularity at \( z \) is handled independently.

\end{document}